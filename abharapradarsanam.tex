% Nityanand Misra: LaTeX code to typeset a book in Sanskrit
% Copyright (C) 2016 Nityanand Misra
%
% This program is free software: you can redistribute it and/or modify it under
% the terms of the GNU General Public License as published by the Free Software
% Foundation, either version 3 of the License, or (at your option) any later
% version.
%
% This program is distributed in the hope that it will be useful, but WITHOUT
% ANY WARRANTY; without even the implied warranty of  MERCHANTABILITY or FITNESS
% FOR A PARTICULAR PURPOSE. See the GNU General Public License for more details.
%
% You should have received a copy of the GNU General Public License along with
% this program.  If not, see <http://www.gnu.org/licenses/>.

\renewcommand\chaptername{}
\chapter[आभारप्रदर्शनम्]{आभारप्रदर्शनम्}
\markboth{आभारप्रदर्शनम्}{}
\fontsize{14}{21}\selectfont
\begin{center}
(१)\nopagebreak\\
उर्वीं गुर्वीं प्रकुर्वन्निजपदकमलान्मेखलां यज्ञसूत्रं\nopagebreak\\
बिभ्राणः शिक्षमाणः सनियमनिगमां बाणविद्यां वसिष्ठात्।\nopagebreak\\
कन्दश्यामोऽभिरामो विहितजनमनस्सद्मवासो वसानः\nopagebreak\\
पातु श्रीरामचन्द्रस्त्वचमथ रुचिरां रौरवीं रौरवान्माम्॥\\
(२)\nopagebreak\\
साकार जगदाधार कृष्ण कृष्णाजिनाम्बर।\nopagebreak\\
तुष्टसान्दीपने चित्तं ज्ञानदीपेन दीपय॥\\
(३)\nopagebreak\\
मन्दरौ भवपाथोधेः काशीकैलासमन्दिरौ।\nopagebreak\\
रातां रामे रतिं रम्यां भवानीभूतभावनौ॥\\
(४)\nopagebreak\\
भागीरथीविमलवीचिविलासरम्या\nopagebreak\\
वृन्दारकाविबुधवन्दिवरप्रणम्या।\nopagebreak\\
श्रीविश्वनाथपदपङ्कजपूतरेणु-\nopagebreak\\
र्वाराणसी विजयते जनकामधेनुः॥\\
(५)\nopagebreak\\
दिव्योल्लसद्गिरिधरान्वितगाङ्गधारा\nopagebreak\\
चण्डांशुजासितजलाश्रितचारुगीता।\nopagebreak\\
बिभ्रत्सुगुप्तसलिला जनरञ्जना सा\nopagebreak\\
धारात्रयी लसति पूतमनःप्रयागे॥\\
(६)\nopagebreak\\
श्यामप्रिया श्याममुखप्रसूता जगत्कृते पार्थमिषेण भूता।\nopagebreak\\
श्यामावदाता धृतशास्त्रजाता मन्मानसे सा विचकास्तु गीता॥\\
(७)\nopagebreak\\
मुखात्पूज्यगोस्वामिनः सम्प्रवृत्ते भवानीपतेः काकवर्यस्य वित्ते।\nopagebreak\\
लसद्रामसीतायशोवारिपूरे मनो राजहंसो भवेन्मानसे मे॥\\
(८)\nopagebreak\\
यत्प्रेरणा समभवन्मम जीवनाय\nopagebreak\\
पीयूषयूषविलसन्नवचन्द्रलेखा।\nopagebreak\\
तां रामचन्द्रचरणाम्बुजचञ्चरीकां\nopagebreak\\
गीतां भजे बहुमतां भगिनीं स्वकीयाम्॥\\
(९)\nopagebreak\\
श्रीराजदेवं पितरं शचीं तथा स्वां मातरं पूज्यपितामहं तथा।\nopagebreak\\
स्वभ्रातरं ज्येष्ठमहं रमापतिं नमामि पुष्टोऽस्मि शिशुर्यदाशिषा॥\\
(१०)\nopagebreak\\
श्रीचन्द्रकान्तः सशशाङ्कशेखरो हिमांशुरम्यो महितो मनीषया।\nopagebreak\\
सदोम्प्रकाशः भवतात्प्रसन्नधीर्यत्सेवयाऽहं विपदो न्यवारयम्॥\\
(११)\nopagebreak\\
पिण्डीजनिं सज्जनशिष्यवश्यं विद्वद्वरं शाब्दिकपूज्यपादम्।\nopagebreak\\
चरित्रमूर्तिं स्वगुरुं सुशीलं त्रिपाठिभूपेन्द्रपतिं प्रपद्ये॥\\
(१२)\nopagebreak\\
अध्यात्मरामायणशब्दजातमपाणिनीयं यदनुग्रहेण।\nopagebreak\\
व्यमर्शयं स्वीयनिदेशकं तं सदैव भूपेन्द्रपतिं स्मरामि॥\\
(१३)\nopagebreak\\
यदाशिषा नष्टविलोचनोऽप्यहं शोधप्रबन्धाब्धिमहोऽवतीर्णवान्।\nopagebreak\\
निर्देशकं स्वं सरलं त्रिपाठिनं नमामि भूपेन्द्रपतिं समादरात्॥\\
(१४)\nopagebreak\\
शब्दाटवीसिंहमदभ्रबुद्धिं शिष्यप्रियं मञ्जुलभाषणञ्च।\nopagebreak\\
त्रिपाठिनं शास्त्रसुपाठिनं तं रामप्रसादं गुरुमानतोऽस्मि॥\\
(१५)\nopagebreak\\
नव्यप्राच्यसुशब्दशासनविधाप्रारब्धसङ्काययो-\nopagebreak\\
राध्यक्ष्यं कलयन्मुदा सुरगवीश्रीविश्वविद्यालये।\nopagebreak\\
श्रीराधाचरणारविन्दमधुपो वैदुष्यभूषान्वितो\nopagebreak\\
नित्यं शुक्लवरो गुरुर्विजयते वात्सल्यरत्नाकरः॥\\
(१६)\nopagebreak\\
शब्दाटवीकेसरिणं महामतिं छात्त्रप्रियं शिष्यदयावशंवदम्।\nopagebreak\\
कविं च शुक्लं प्रणमामि कालिकाप्रसादमार्यं स्वगुरुं मुहुर्मुहुः॥\\
(१७)\nopagebreak\\
दोषाः शोधप्रबन्धे चेन्ममैते नष्टचक्षुषः।\nopagebreak\\
गुणाश्चेत्परिलोक्येरन्मद्गुरूणां च ते स्फुटम्॥\\
(१८)\nopagebreak\\
प्रायः प्रबन्धेऽत्र विमृश्य बुद्ध्या मयैव भावा निहिताः समग्राः।\nopagebreak\\
त्रुटिर्यदि स्यान्मनुजस्वभावान्मत्वा शिशुं तां सुधियः क्षमन्ताम्॥\\
(१९)\nopagebreak\\
हिन्द्युदाहरणं दत्तं मौलिकत्वान्मया क्वचित्।\nopagebreak\\
तद्दृष्ट्वा चापलं क्षान्त्वा भावं गृह्णन्तु साधवः॥\\
(२०)\nopagebreak\\
अध्यायत्रितये चास्मिन् बहिर्भूताश्च पाणिनेः।\nopagebreak\\
शब्दाः स्वीयोपपत्त्या च यथाबुद्धि समाहिताः॥\\
(२१)\nopagebreak\\
गतदृक्स्वल्पबुद्धिश्च शास्त्रसाधनवर्जितः।\nopagebreak\\
प्रयासमाचरं चैनं दृष्ट्वा नन्दन्तु सज्जनाः॥\\
(२२)\nopagebreak\\
निरीक्ष्य भावुका बन्धं परीक्षन्तां परीक्षकाः।\nopagebreak\\
सकृन्मां करुणादृष्ट्या रामभद्रोऽपि वीक्षताम्॥\\
(२३)\nopagebreak\\
मद्भावान् परिधाप्य वर्णवसनं शोधप्रबन्धेऽत्र हि\nopagebreak\\
संस्थाप्य प्रगुणान्स्म संवितनुते रम्या च यल्लेखनी।\nopagebreak\\
सेवाभावपरायणं कृतधियं वात्सल्यभाजं मुदा\nopagebreak\\
स्वाशीर्भिः परिलालयामि सततं शिष्यं दयाशङ्करम्॥\\
(२४)\nopagebreak\\
नीलनीरजसङ्काशकान्तये दिव्यकान्तये।\nopagebreak\\
रामाय पूर्णकामाय जानकीजानये नमः॥\\
\end{center}
\raggedleft{(गिरिधरलालमिश्रः प्रज्ञाचक्षुः)}