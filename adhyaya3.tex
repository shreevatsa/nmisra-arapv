% Nityanand Misra: LaTeX code to typeset a book in Sanskrit
% Copyright (C) 2016 Nityanand Misra
%
% This program is free software: you can redistribute it and/or modify it under
% the terms of the GNU General Public License as published by the Free Software
% Foundation, either version 3 of the License, or (at your option) any later
% version.
%
% This program is distributed in the hope that it will be useful, but WITHOUT
% ANY WARRANTY; without even the implied warranty of  MERCHANTABILITY or FITNESS
% FOR A PARTICULAR PURPOSE. See the GNU General Public License for more details.
%
% You should have received a copy of the GNU General Public License along with
% this program.  If not, see <http://www.gnu.org/licenses/>.

\renewcommand\chaptername{अथ तृतीयोऽध्यायः}
\chapter[\texorpdfstring{धातुप्रकरणम्}{तृतीयोऽध्यायः}]{धातुप्रकरणम्}
\vspace{-5mm}
\fontsize{16}{24}\selectfont\centering\hyphenrules{nohyphenation}\textcolor{blue}{प्रणम्य सीतापतिपादपद्मं गौरीं गिरीशं किल धातुशब्दान्।\nopagebreak\\
अपाणिनीयांश्च विमर्शयेऽत्र\footnote{\textcolor{red}{विमर्शये} इत्यत्र स्वार्थे णिच्। \textcolor{red}{निवृत्त\-प्रेषणाद्धातोः प्राकृतेऽर्थे णिजुच्यते} (वा॰प॰~३.७.६०)। स्वान्तःसुखाय विमृशामीति कर्त्रभिप्रायं ध्वनयितुमात्मने\-पदप्रयोगः। \textcolor{red}{णिचश्च} (पा॰सू॰~१.३.७४) इत्यनेन। वि~\textcolor{red}{मृशँ आमर्शने} (धा॰पा॰~१४२५)~\arrow वि~मृश्~\arrow स्वार्थे णिच्~\arrow वि~मृश्~णिच्~\arrow वि~मृश्~इ~\arrow \textcolor{red}{पुगन्त\-लघूपधस्य च} (पा॰सू॰~७.३.८६)~\arrow \textcolor{red}{उरण् रपरः} (पा॰सू॰~१.१.५१)~\arrow वि~मर्श्~इ~\arrow विमर्शि~\arrow \textcolor{red}{सनाद्यन्ता धातवः} (पा॰सू॰~३.१.३२)~\arrow धातु\-सञ्ज्ञा~\arrow \textcolor{red}{णिचश्च} (पा॰सू॰~१.३.७४)~\arrow \textcolor{red}{वर्तमाने लट्} (पा॰सू॰~३.२.१२३)~\arrow विमर्शि~इट्~\arrow विमर्शि~इ~\arrow \textcolor{red}{कर्तरि शप्‌} (पा॰सू॰~३.१.६८)~\arrow विमर्शि~शप्~इ~\arrow विमर्शि~अ~इ~\arrow \textcolor{red}{सार्वधातुकार्ध\-धातुकयोः} (पा॰सू॰~७.३.८४)~\arrow विमर्शे~अ~इ~\arrow \textcolor{red}{एचोऽयवायावः} (पा॰सू॰~६.१.७८)~\arrow विमर्शय्~अ~इ~\arrow \textcolor{red}{आद्गुणः} (पा॰सू॰~६.१.८७)~\arrow विमर्शय्~ए~\arrow विमर्शये। यद्वा \textcolor{red}{विमर्शं कुर्वे} इति विग्रहे \textcolor{red}{विमर्शये}। अत्रापि कर्त्रभिप्राये \textcolor{red}{णिचश्च} (पा॰सू॰~१.३.७४) इत्यनेनाऽत्मने\-पदम्। विमर्श~\arrow \textcolor{red}{तत्करोति तदाचष्टे} (धा॰पा॰ ग॰सू॰)~\arrow विमर्श~णिच्~\arrow विमर्श~इ~\arrow \textcolor{red}{णाविष्ठवत्प्राति\-पदिकस्य पुंवद्भाव\-रभाव\-टिलोप\-यणादि\-परार्थम्} (वा॰~६.४.४८)~\arrow विमर्श्~इ~\arrow विमर्शि~\arrow \textcolor{red}{सनाद्यन्ता धातवः} (पा॰सू॰~३.१.३२)~\arrow धातु\-सञ्ज्ञा~\arrow \textcolor{red}{णिचश्च} (पा॰सू॰~१.३.७४)~\arrow \textcolor{red}{वर्तमाने लट्} (पा॰सू॰~३.२.१२३)~\arrow विमर्शि~इट्~\arrow शेषं पूर्ववत्।} अध्यात्मरामायणमध्यगान्वै॥}\nopagebreak\\
\vspace{4mm}
\pdfbookmark[1]{प्रथमः परिच्छेदः}{Chap3Part1}
\phantomsection
\addtocontents{toc}{\protect\setcounter{tocdepth}{1}}
\addcontentsline{toc}{section}{प्रथमः परिच्छेदः}
\addtocontents{toc}{\protect\setcounter{tocdepth}{0}}
\centering ॥ अथ तृतीयाध्याये प्रथमः परिच्छेदः ॥\nopagebreak\\
\vspace{4mm}
\pdfbookmark[2]{बालकाण्डम्}{Chap3Part1Kanda1}
\phantomsection
\addtocontents{toc}{\protect\setcounter{tocdepth}{2}}
\addcontentsline{toc}{subsection}{बालकाण्डीयप्रयोगाणां विमर्शः}
\addtocontents{toc}{\protect\setcounter{tocdepth}{0}}
\centering ॥ अथ बालकाण्डीयप्रयोगाणां विमर्शः ॥\nopagebreak\\
\fontsize{14}{21}\selectfont
\section[पठन्ति शृण्वन्ति यान्ति]{पठन्ति शृण्वन्ति यान्ति}
\centering\textcolor{blue}{पठन्ति ये नित्यमनन्यचेतसः शृण्वन्ति चाध्यात्मिकसञ्ज्ञितं शुभम्।\nopagebreak\\
रामायणं सर्वपुराणसम्मतं निर्धूतपापा हरिमेव यान्ति ते॥}\nopagebreak\\
\raggedleft{–~अ॰रा॰~१.१.३}\\
\fontsize{14}{21}\selectfont\begin{sloppypar}\hyphenrules{nohyphenation}\justifying\noindent\hspace{10mm} साम्प्रतमहं तृतीयाध्याये धातु\-सम्बन्धिनोऽपाणिनीय\-प्रयोगान् विमर्शये। प्रथमे सर्गे बाल\-काण्डस्य फल\-श्रुतिं वर्णयन्नाह ग्रन्थ\-कृद्यद् \textcolor{red}{य इमामध्यात्म\-रामायण\-संहितां पठन्ति ते निर्धूत\-पापा भगवन्तमेव प्राप्नुवन्ति}। आरभ्यमाण\-ग्रन्थस्य फल\-श्रुतिरियम्। पूर्णतामपि न गतेऽस्मिन् वर्तमान\-कालीन\-पाठः कथं सङ्गंस्यत इति चेत्। \textcolor{red}{पठन्ति} इति वर्तमान\-कालिक\-प्रयोगो भविष्यत्तात्पर्य\-वाचकः।\footnote{\textcolor{red}{पठँ व्यक्तायां वाचि} (धा॰पा॰~३३०)~\arrow पठ्~\arrow \textcolor{red}{शेषात्कर्तरि परस्मैपदम्} (पा॰सू॰~१.३.७८)~\arrow \textcolor{red}{वर्तमान\-सामीप्ये वर्तमानवद्वा} (पा॰सू॰~३.३.१३१)~\arrow \textcolor{red}{वर्तमाने लट्} (पा॰सू॰~३.२.१२३)~\arrow पठ्~लट्~\arrow पठ्~झि~\arrow \textcolor{red}{झोऽन्तः} (पा॰सू॰~७.१.३)~\arrow पठ्~अन्ति~\arrow \textcolor{red}{कर्तरि शप्} (पा॰सू॰~३.१.६८)~\arrow पठ्~शप्~अन्ति~\arrow पठ्~अ~अन्ति~\arrow \textcolor{red}{अतो गुणे} (पा॰सू॰~६.१.९७)~\arrow पठ्~अन्ति~\arrow पठन्ति।} एवं च \textcolor{red}{वर्तमान\-सामीप्ये वर्तमानवद्वा} (पा॰सू॰~३.३.१३१) इत्यनेन वर्तमान\-समीप\-भविष्यत्काले विकल्पेन वर्तमान\-काल\-निमित्त\-कार्याण्यर्थाद्वर्तमान\-वद्भावः। यथा च \textcolor{red}{पठिष्यन्ति}\footnote{\textcolor{red}{पठँ व्यक्तायां वाचि} (धा॰पा॰~३३०)~\arrow पठ्~\arrow \textcolor{red}{शेषात्कर्तरि परस्मैपदम्} (पा॰सू॰~१.३.७८)~\arrow \textcolor{red}{लृट् शेषे च} (पा॰सू॰~३.३.१३)~\arrow पठ्~लृट्~\arrow पठ्~झि~\arrow \textcolor{red}{झोऽन्तः} (पा॰सू॰~७.१.३)~\arrow पठ्~अन्ति~\arrow \textcolor{red}{स्यतासी लृलुटोः} (पा॰सू॰~३.१.३३)~\arrow पठ्~स्य~अन्ति~\arrow \textcolor{red}{आर्धधातुकस्येड्वलादेः} (पा॰सू॰~७.२.३५)~\arrow पठ्~इट्~स्य~अन्ति~\arrow पठ्~इ~स्य~अन्ति~\arrow \textcolor{red}{आदेश\-प्रत्यययोः} (पा॰सू॰~८.३.५९)~\arrow पठ्~इ~ष्य~अन्ति~\arrow \textcolor{red}{अतो गुणे} (पा॰सू॰~६.१.९७)~\arrow पठ्~इ~ष्यन्ति~\arrow पठिष्यन्ति।} इत्यर्थे पठन्ति। एवं \textcolor{red}{नाना\-कर्तृकाध्यात्म\-रामायण\-कर्मक\-भविष्यत्कालावच्छिन्न\-वर्तमान\-कालाभासिक उच्चारणानुकूलो व्यापारः} इति शाब्द\-बोधः। धातोः खलु व्यापार एव शक्तिरेवं \textcolor{red}{तिङ्} प्रत्ययस्याऽश्रये। व्यापार\-वाचकत्वे प्रधानतया व्यापारः प्रधानमेवं तिङर्थो विशेषणम्। अर्थाद्वैयाकरणा व्यापार\-मुख्य\-विशेष्यकं शाब्द\-बोधमङ्गीकुर्वन्ति। यद्यपि मीमांसका अपि भावना\-मुख्य\-विशेष्यकं शाब्द\-बोधं मन्यन्ते किन्तु कुत्रचिदप\-सिद्धान्तितमपि तैः। नैयायिकाः प्रथमान्त\-मुख्य\-विशेष्यकं शाब्द\-बोधं मन्यन्ते। किन्तु \textcolor{red}{पश्य मृगो धावति} इत्यत्र भाष्य\-सम्मतैक\-वाक्यता भग्ना भवति प्रथमान्त\-मुख्य\-विशेष्यके शाब्द\-बोधे स्वीकृते।\footnote{\textcolor{red}{क्रियाऽपि क्रिययेप्सिततमा भवति। कया क्रियया। सन्दर्शनक्रियया वा प्रार्थयतिक्रियया वाऽध्यवस्यतिक्रियया वा। इह य एष मनुष्यः प्रेक्षापूर्वकारी भवति स बुद्ध्या तावत्कञ्चिदर्थं सम्पश्यति सन्दृष्टे प्रार्थना प्रार्थनायामध्यवसायोऽध्यवसाय आरम्भ आरम्भे निर्वृत्तिर्निर्वृत्तौ फलावाप्तिः। एवं क्रियाऽपि कृत्रिमं कर्म} (भा॰पा॰सू॰~१.४.३२)।} \textcolor{red}{धावनानुकूल\-कृतिमान् मृगः}। \textcolor{red}{मृग\-कर्तृक\-धावन\-कर्मक\-दर्शनानुकूल\-कृतिमांस्त्वम्}। अस्मन्मते तु \textcolor{red}{मृग\-कर्तृक\-धावनानुकूल\-व्यापार\-कर्मक\-त्वत्कर्तृक\-दर्शनानुकूल\-व्यापारः}। इत्थं तिङर्थाश्चत्वारः कर्तृ\-कर्म\-सङ्ख्या\-कालाः। कर्तुर्व्यापारेऽन्वयः कर्मणश्च फले सङ्ख्यायाश्च कर्म\-प्रत्यये कर्मणि कर्तृ\-प्रत्यये च कर्तरि। कालस्य च व्यापारेऽन्वयः। \textcolor{red}{वर्तमाने लट्} (पा॰सू॰~३.२.१२३) इत्यादि\-सूत्र\-निर्देशात्। इत्थं तत्र कारिका~–\end{sloppypar}
\centering\textcolor{red}{फलव्यापारयोर्धातुराश्रये तु तिङः स्मृताः।\nopagebreak\\
फले प्रधानं व्यापारस्तिङर्थस्तु विशेषणम्॥}\nopagebreak\\
\raggedleft{–~वै॰सि॰का॰~१.२}\\
\fontsize{14}{21}\selectfont\begin{sloppypar}\hyphenrules{nohyphenation}\justifying\noindent फलस्य व्यापारे स्वानुकूलत्व\-सम्बन्धाश्रयः। एवं \textcolor{red}{भक्ताभिन्नैक\-कर्तृक\-भविष्यत्कालावच्छिन्न\-वर्तमान\-कालाभासिक\-स्पष्टोच्चारणानुकूल\-व्यापारः}। प्राचीन\-नवीनयोरिदमन्तरम्। प्राचीनास्तूभयत्र समान\-व्यवस्थां मन्यन्ते। नवीनास्तु कर्तृ\-भाव\-वाच्ययोः फलानुकूल\-व्यापार एवं कर्म\-वाच्ये व्यापार\-विशिष्ट\-फलम्।\footnote{\textcolor{red}{इति कथयन्ति} इति शेषः।} यथा \textcolor{red}{रामेणायोध्या गम्यते}~– \textcolor{red}{रामाभिन्नैक\-कर्तृक\-वर्तमानकालावच्छिन्न\-व्यापार\-जन्यं गृह\-निष्ठोत्तर\-देश\-संयोग\-रूपं फलम्} अयमत्र विवेकः।\footnote{नवीन\-वैयाकरण\-शाब्दबोधोऽयम्।} किं फले व्यापारे च धातोः पृथक्शक्तिरुताहो समुदिता चेत्। अनुकूलत्व\-सम्बन्धेन फलस्य व्यापारेऽन्वयः। तदा \textcolor{red}{पदार्थः पदार्थेनान्वेति न तु तदेकदेशेन} इति नियमेन कथमत्रान्वयः। अतो विशिष्टे शक्तिः।\footnote{फल\-विशिष्ट\-व्यापारे व्यापार\-विशिष्ट\-फले च शक्तिरिति भावः। \textcolor{red}{तस्मात्फलावच्छिन्ने व्यापारे व्यापारावच्छिन्ने फले च धातूनां शक्तिः कर्तृ\-कर्मार्थक\-तत्तत्प्रत्यय\-समभि\-व्याहारश्च ततद्बोधे नियामक इत्याहुः} (प॰ल॰म॰~४७)।} गुरु\-चरणास्त्वेक\-वृन्तावलम्बि\-फलद्वयवदेकस्मिन् धातावेव फल\-व्यापारयोः शक्तिः।\footnote{\textcolor{red}{मम त्वेक\-वृन्ताव\-लम्बि\-फल\-द्वय\-वदुभयांश एका खण्डशश्शक्तिरिति न शक्ति\-द्वय\-कल्पनं न वा बोध\-जनकत्व\-सम्बन्ध\-द्वय\-कल्पनम्} (प॰ल॰म॰ ज्यो॰टी॰~४७) इति प्रणेतॄणां गुरुचरणाः कालिका\-प्रसाद\-शुक्ल\-वर्याः परम\-लघु\-मञ्जूषाया ज्योत्स्ना\-टीकायाम्।} विस्तार\-भयाद्विरम्यते। एवं \textcolor{red}{भक्ताभिन्नैक\-कर्तृक\-भविष्यत्कालिक\-वर्तमान\-कालाभासिक\-स्पष्टोच्चारणानुकूल\-व्यापारः} इति शाब्द\-बोधः।\footnote{एवमेव \textcolor{red}{शृण्वन्ति} इति प्रयोगः \textcolor{red}{श्रोष्यन्ति} इत्यर्थे \textcolor{red}{यान्ति} इति च \textcolor{red}{यास्यन्ति} इत्यर्थे। उभयत्र वर्तमानवद्भावः। \textcolor{red}{श्रु श्रवणे} (धा॰पा॰~९४२)~\arrow \textcolor{red}{शेषात्कर्तरि परस्मैपदम्} (पा॰सू॰~१.३.७८)~\arrow \textcolor{red}{वर्तमान\-सामीप्ये वर्तमानवद्वा} (पा॰सू॰~३.३.१३१)~\arrow \textcolor{red}{वर्तमाने लट्} (पा॰सू॰~३.२.१२३)~\arrow श्रु~लट्~\arrow श्रु~झि~\arrow \textcolor{red}{झोऽन्तः} (पा॰सू॰~७.१.३)~\arrow श्रु~अन्ति~\arrow \textcolor{red}{श्रुवः शृ च} (पा॰सू॰~३.१.७४)~\arrow शृ~श्नु~अन्ति~\arrow शृ~नु~अन्ति~\arrow \textcolor{red}{सार्वधातुकमपित्} (पा॰सू॰~१.२.४)~\arrow ङित्त्वम्~\arrow \textcolor{red}{ग्क्ङिति च} (पा॰सू॰~१.१.५)~\arrow सार्वधातुक\-गुण\-निषेधः~\arrow \textcolor{red}{रषाभ्यां णत्व ऋकारग्रहणम्} (वा॰~८.४.१)~\arrow शृ~णु~अन्ति~\arrow \textcolor{red}{हुश्नुवोः सार्वधातुके} (पा॰सू॰~६.४.८७)~\arrow शृ~ण्व्~अन्ति~\arrow शृण्वन्ति। \textcolor{red}{या प्रापणे} (धा॰पा॰~१०४९)~\arrow या~\arrow \textcolor{red}{शेषात्कर्तरि परस्मैपदम्} (पा॰सू॰~१.३.७८)~\arrow \textcolor{red}{वर्तमान\-सामीप्ये वर्तमानवद्वा} (पा॰सू॰~३.३.१३१)~\arrow \textcolor{red}{वर्तमाने लट्} (पा॰सू॰~३.२.१२३)~\arrow या~लट्~\arrow या~झि~\arrow \textcolor{red}{झोऽन्तः} (पा॰सू॰~७.१.३)~\arrow या~अन्ति~\arrow \textcolor{red}{कर्तरि शप्‌} (पा॰सू॰~३.१.६८)~\arrow या~शप्~अन्ति~\arrow \textcolor{red}{अदिप्रभृतिभ्यः शपः} (पा॰सू॰~२.४.७२)~\arrow या~अन्ति~\arrow \textcolor{red}{अकः सवर्णे दीर्घः} (पा॰सू॰~६.१.१०१)~\arrow यान्ति। \textcolor{red}{श्रु श्रवणे} (धा॰पा॰~९४२)~\arrow \textcolor{red}{शेषात्कर्तरि परस्मैपदम्} (पा॰सू॰~१.३.७८)~\arrow \textcolor{red}{लृट् शेषे च} (पा॰सू॰~३.३.१३)~\arrow श्रु~लृट्~\arrow श्रु~झि~\arrow \textcolor{red}{झोऽन्तः} (पा॰सू॰~७.१.३)~\arrow श्रु~अन्ति~\arrow \textcolor{red}{स्यतासी लृलुटोः} (पा॰सू॰~३.१.३३)~\arrow श्रु~स्य~अन्ति~\arrow \textcolor{red}{एकाच उपदेशेऽनुदात्तात्‌} (पा॰सू॰~७.२.१०)~\arrow इडागम\-निषेधः~\arrow \textcolor{red}{सार्वधातुकार्धधातुकयोः} (पा॰सू॰~७.३.८४)~\arrow श्रो~स्य~अन्ति~\arrow \textcolor{red}{आदेश\-प्रत्यययोः} (पा॰सू॰~८.३.५९)~\arrow श्रो~ष्य~अन्ति~\arrow \textcolor{red}{अतो गुणे} (पा॰सू॰~६.१.९७)~\arrow श्रो~ष्यन्ति~\arrow श्रोष्यन्ति। \textcolor{red}{या प्रापणे} (धा॰पा॰~१०४९)~\arrow \textcolor{red}{शेषात्कर्तरि परस्मैपदम्} (पा॰सू॰~१.३.७८)~\arrow \textcolor{red}{लृट् शेषे च} (पा॰सू॰~३.३.१३)~\arrow या~लृट्~\arrow या~झि~\arrow \textcolor{red}{झोऽन्तः} (पा॰सू॰~७.१.३)~\arrow या~अन्ति~\arrow \textcolor{red}{स्यतासी लृलुटोः} (पा॰सू॰~३.१.३३)~\arrow या~स्य~अन्ति~\arrow \textcolor{red}{अतो गुणे} (पा॰सू॰~६.१.९७)~\arrow या~स्यन्ति~\arrow यास्यन्ति।}\end{sloppypar}
\section[लभेत्]{लभेत्}
\centering\textcolor{blue}{अध्यात्मरामायणमेव नित्यं पठेद्यदीच्छेद्भवबन्धमुक्तिम्।\nopagebreak\\
गवां सहस्रायुतकोटिदानात्फलं लभेद्यः शृणुयात्स नित्यम्॥}\nopagebreak\\
\raggedleft{–~अ॰रा॰~१.१.४}\\
\fontsize{14}{21}\selectfont\begin{sloppypar}\hyphenrules{nohyphenation}\justifying\noindent\hspace{10mm} अत्र फल\-श्रुतावेव \textcolor{red}{लभेत्} इति हलन्त\-प्रयोगः। \textcolor{red}{लभ्} धातुः (\textcolor{red}{डुलभँष् प्राप्तौ} धा॰पा॰~९७५) आत्मनेपदी। \textcolor{red}{विधि\-निमन्त्रणामन्त्रणाधीष्ट\-सम्प्रश्न\-प्रार्थनेषु लिङ्} (पा॰सू॰~३.३.१६१) इत्यनेन लिङ्लकारे \textcolor{red}{लिङः सीयुट्} (पा॰सू॰~३.४.१०२) इत्यनेन सीयुट्यनु\-बन्धलोपे सकारस्य च लोपे गुणे यकार\-लोपे \textcolor{red}{लभेत} इत्येव पाणिनीयम्।\footnote{\textcolor{red}{डुलभँष् प्राप्तौ} (धा॰पा॰~९७५)~\arrow लभ्~\arrow \textcolor{red}{अनुदात्तङित आत्मने\-पदम्} (पा॰सू॰~१.३.१२)~\arrow \textcolor{red}{विधि\-निमन्‍त्रणामन्‍त्रणाधीष्‍ट\-सम्प्रश्‍न\-प्रार्थनेषु लिङ्} (पा॰सू॰~३.३.१६१)~\arrow लभ्~लिङ्~\arrow लभ्~त~\arrow \textcolor{red}{कर्तरि शप्} (पा॰सू॰~३.१.६८)~\arrow लभ्~शप्~त~\arrow लभ्~अ~त~\arrow \textcolor{red}{लिङः सीयुट्} (पा॰सू॰~३.४.१०२)~\arrow लभ्~अ~सीयुँट्~त~\arrow लभ्~अ~सीय्~त~\arrow \textcolor{red}{सुट् तिथोः} (पा॰सू॰~३.४.१०७)~\arrow \textcolor{red}{आद्यन्तौ टकितौ} (पा॰सू॰~१.१.४६)~\arrow लभ्~अ~सीय्~सुँट्~त~\arrow लभ्~अ~सीय्~स्~त~\arrow \textcolor{red}{लिङः सलोपोऽनन्त्यस्य} (पा॰सू॰~७.२.७९)~\arrow लभ्~अ~ईय्~त~\arrow \textcolor{red}{लोपो व्योर्वलि} (पा॰सू॰~६.१.६६)~\arrow लभ्~अ~ई~त~\arrow \textcolor{red}{आद्गुणः} (पा॰सू॰~६.१.८७)~\arrow लभ्~ए~त~\arrow लभेत।} \textcolor{red}{लभेत्} इति कथमिति चेत्। \textcolor{red}{अनुदात्तेत्त्व\-लक्षणमात्मने\-पदमनित्यम्} (प॰शे॰~९३.४) इति वचनेनात्राऽत्मने\-पदाभावे परस्मैपद उक्तं रूपम्।\footnote{\textcolor{red}{डुलभँष् प्राप्तौ} (धा॰पा॰~९७५)~\arrow लभ्~\arrow \textcolor{red}{अनुदात्तेत्त्व\-लक्षणमात्मने\-पदमनित्यम्} (प॰शे॰~९३.४)~\arrow \textcolor{red}{शेषात्कर्तरि परस्मैपदम्} (पा॰सू॰~१.३.७८)~\arrow \textcolor{red}{विधि\-निमन्‍त्रणामन्‍त्रणाधीष्‍ट\-सम्प्रश्‍न\-प्रार्थनेषु लिङ्} (पा॰सू॰~३.३.१६१)~\arrow लभ्~लिङ्~\arrow लभ्~तिप्~\arrow लभ्~ति~\arrow \textcolor{red}{कर्तरि शप्} (पा॰सू॰~३.१.६८)~\arrow लभ्~शप्~ति~\arrow लभ्~अ~ति~\arrow \textcolor{red}{यासुट् परस्मैपदेषूदात्तो ङिच्च} (पा॰सू॰~३.४.१०३)~\arrow \textcolor{red}{आद्यन्तौ टकितौ} (पा॰सू॰~१.१.४६)~\arrow लभ्~अ~यासुँट्~ति~\arrow लभ्~अ~यास्~ति~\arrow \textcolor{red}{सुट् तिथोः} (पा॰सू॰~३.४.१०७)~\arrow \textcolor{red}{आद्यन्तौ टकितौ} (पा॰सू॰~१.१.४६)~\arrow लभ्~अ~यास्~सुँट्~ति~\arrow  लभ्~अ~यास्~स्~ति~\arrow \textcolor{red}{लिङः सलोपोऽनन्त्यस्य} (पा॰सू॰~७.२.७९)~\arrow लभ्~अ~या~ति~\arrow \textcolor{red}{अतो येयः} (पा॰सू॰~७.२.८०)~\arrow लभ्~अ~इय्~ति~\arrow \textcolor{red}{लोपो व्योर्वलि} (पा॰सू॰~६.१.६६)~\arrow लभ्~अ~इ~ति~\arrow \textcolor{red}{आद्गुणः} (पा॰सू॰~६.१.८७)~\arrow लभ्~ए~ति~\arrow \textcolor{red}{इतश्च} (पा॰सू॰~३.४.१००)~\arrow लभ्~ए~त्~\arrow लभेत्।} यद्वा \textcolor{red}{लभत इति लभः}।\footnote{\textcolor{red}{नन्दि\-ग्रहि\-पचादिभ्यो ल्युणिन्यचः} (पा॰सू॰~३.१.१३४) इत्यनेन कर्तरि पचाद्यच्।} \textcolor{red}{लभ इवाऽचरतीति लभति}।\footnote{लभ~\arrow \textcolor{red}{सर्वप्राति\-पदिकेभ्य आचारे क्विब्वा वक्तव्यः} (वा॰~३.१.११)~\arrow लभ~क्विँप्~\arrow लभ~व्~\arrow \textcolor{red}{वेरपृक्तस्य} (पा॰सू॰~६.१.६७)~\arrow लभ~\arrow \textcolor{red}{सनाद्यन्ता धातवः} (पा॰सू॰~३.१.३२)~\arrow धातुसञ्ज्ञा~\arrow \textcolor{red}{शेषात्कर्तरि परस्मैपदम्} (पा॰सू॰~१.३.७८)~\arrow \textcolor{red}{वर्तमाने लट्} (पा॰सू॰~३.२.१२३)~\arrow लभ~लट्~\arrow लभ~तिप्~\arrow लभ~ति~\arrow \textcolor{red}{कर्तरि शप्‌} (पा॰सू॰~३.१.६८)~\arrow लभ~शप्~ति~\arrow लभ~अ~ति~\arrow \textcolor{red}{अतो गुणे} (पा॰सू॰~६.१.९७)~\arrow लभ~ति~\arrow लभति।} तत आचारक्विबन्ताल्लिङि तिपि शपि पररूपे \textcolor{red}{यासुट् परस्मैपदेषूदात्तो ङिच्च} (पा॰सू॰~३.४.१०३) इत्यनेन यासुटि कृते सुडागमे सकारद्वयलोपे \textcolor{red}{अतो येयः} (पा॰सू॰~७.२.८०) इत्यनेनेयादेशे गुणे यलोपे \textcolor{red}{इतश्च} (पा॰सू॰~३.४.१००) इत्यनेनेकार\-लोपे \textcolor{red}{लभेत्}।\footnote{लभ~\arrow \textcolor{red}{सनाद्यन्ता धातवः} (पा॰सू॰~३.१.३२)~\arrow धातुसञ्ज्ञा (पूर्ववत्)~\arrow \textcolor{red}{शेषात्कर्तरि परस्मैपदम्} (पा॰सू॰~१.३.७८)~\arrow \textcolor{red}{विधि\-निमन्‍त्रणामन्‍त्रणाधीष्‍ट\-सम्प्रश्‍न\-प्रार्थनेषु लिङ्} (पा॰सू॰~३.३.१६१)~\arrow लभ~लिङ~\arrow लभ~तिप्~\arrow लभ~ति~\arrow \textcolor{red}{कर्तरि शप्‌} (पा॰सू॰~३.१.६८)~\arrow लभ~शप्~ति~\arrow लभ~अ~ति~\arrow \textcolor{red}{अतो गुणे} (पा॰सू॰~६.१.९७)~\arrow लभ~ति~\arrow \textcolor{red}{यासुट् परस्मैपदेषूदात्तो ङिच्च} (पा॰सू॰~३.४.१०३)~\arrow \textcolor{red}{आद्यन्तौ टकितौ} (पा॰सू॰~१.१.४६)~\arrow लभ~यासुँट्~ति~\arrow लभ~यास्~ति~\arrow \textcolor{red}{सुट् तिथोः} (पा॰सू॰~३.४.१०७)~\arrow \textcolor{red}{आद्यन्तौ टकितौ} (पा॰सू॰~१.१.४६)~\arrow लभ~यास्~सुँट्~ति~\arrow  लभ~यास्~स्~ति~\arrow \textcolor{red}{लिङः सलोपोऽनन्त्यस्य} (पा॰सू॰~७.२.७९)~\arrow लभ~या~ति~\arrow\textcolor{red}{अतो येयः} (पा॰सू॰~७.२.८०)~\arrow लभ~इय्~ति~\arrow \textcolor{red}{लोपो व्योर्वलि} (पा॰सू॰~६.१.६६)~\arrow लभ~इ~ति~\arrow \textcolor{red}{आद्गुणः} (पा॰सू॰~६.१.८७)~\arrow लभे~ति~\arrow \textcolor{red}{इतश्च} (पा॰सू॰~३.४.१००)~\arrow लभे~त्~\arrow लभेत्।}\end{sloppypar}
\section[जानामि]{जानामि}
\centering\textcolor{blue}{ज्ञानं सविज्ञानमथानुभक्तिवैराग्ययुक्तं च मितं विभास्वत्।\nopagebreak\\
जानाम्यहं योषिदपि त्वदुक्तं यथा तथा ब्रूहि तरन्ति येन॥}\nopagebreak\\
\raggedleft{–~अ॰रा॰~१.१.९}\\
\fontsize{14}{21}\selectfont\begin{sloppypar}\hyphenrules{nohyphenation}\justifying\noindent\hspace{10mm} अत्र भगवती पार्वती भगवन्तं शिवं प्रार्थयते यत् \textcolor{red}{भक्ति\-वैराग्य\-सहितं ज्ञानं यथा जानीयां तथा कुर्वन्तु देवाः}। अत्र \textcolor{red}{जानीयाम्}\footnote{\textcolor{red}{ज्ञा अवबोधने} (धा॰पा॰~१५०७)~\arrow ज्ञा~\arrow \textcolor{red}{शेषात्कर्तरि परस्मैपदम्} (पा॰सू॰~१.३.७८)~\arrow \textcolor{red}{विधि\-निमन्‍त्रणामन्‍त्रणाधीष्‍ट\-सम्प्रश्‍न\-प्रार्थनेषु लिङ्} (पा॰सू॰~३.३.१६१)~\arrow ज्ञा~लिङ्~\arrow ज्ञा~मिप्~\arrow ज्ञा~मि~\arrow \textcolor{red}{क्र्यादिभ्यः श्ना} (पा॰सू॰~३.१.८१)~\arrow ज्ञा~श्ना~मि~\arrow ज्ञा~ना~मि~\arrow \textcolor{red}{ज्ञाजनोर्जा} (पा॰सू॰~७.३.७९)~\arrow जा~ना~मि~\arrow \textcolor{red}{इतश्च} (पा॰सू॰~३.४.१००)~\arrow जा~ना~म्~\arrow \textcolor{red}{यासुट् परस्मैपदेषूदात्तो ङिच्च} (पा॰सू॰~३.४.१०३)~\arrow \textcolor{red}{आद्यन्तौ टकितौ} (पा॰सू॰~१.१.४६)~\arrow जा~ना~यासुँट्~म्~\arrow जा~ना~यास्~म्~\arrow \textcolor{red}{लिङः सलोपोऽनन्त्यस्य} (पा॰सू॰~७.२.७९)~\arrow जा~ना~या~म्~\arrow \textcolor{red}{ई हल्यघोः} (पा॰सू॰~६.४.११३)~\arrow जा~नी~या~म्~\arrow जानीयाम्।} इति प्रयोक्तव्ये \textcolor{red}{जानामि} इति प्रयुक्तम्। \textcolor{red}{वर्तमान\-सामीप्ये वर्तमानवद्वा} (पा॰सू॰~३.३.१३१) इत्यनेन वर्तमानवद्भावः।\footnote{\textcolor{red}{ज्ञा अवबोधने} (धा॰पा॰~१५०७)~\arrow ज्ञा~\arrow \textcolor{red}{शेषात्कर्तरि परस्मैपदम्} (पा॰सू॰~१.३.७८)~\arrow \textcolor{red}{वर्तमान\-सामीप्ये वर्तमानवद्वा} (पा॰सू॰~३.३.१३१)~\arrow \textcolor{red}{वर्तमाने लट्} (पा॰सू॰~३.२.१२३)~\arrow ज्ञा~लट्~\arrow ज्ञा~मिप्~\arrow ज्ञा~मि~\arrow \textcolor{red}{क्र्यादिभ्यः श्ना} (पा॰सू॰~३.१.८१)~\arrow ज्ञा~श्ना~मि~\arrow ज्ञा~ना~मि~\arrow \textcolor{red}{ज्ञाजनोर्जा} (पा॰सू॰~७.३.७९)~\arrow जा~ना~मि~\arrow जानामि।}\end{sloppypar}
\section[जानाति]{जानाति}
\centering\textcolor{blue}{वदन्ति केचित्परमोऽपि रामः स्वाविद्यया संवृतमात्मसञ्ज्ञम्।\nopagebreak\\
जानाति नात्मानमतः परेण सम्बोधितो वेद परात्मतत्त्वम्॥}\nopagebreak\\
\raggedleft{–~अ॰रा॰~१.१.१३}\\
\fontsize{14}{21}\selectfont\begin{sloppypar}\hyphenrules{nohyphenation}\justifying\noindent\hspace{10mm} भगवती पार्वती श्रीराम\-लीला\-विषयकं संशयं करोति यत् \textcolor{red}{श्रीरामः स्वाविद्यया संवृतमात्म\-सञ्ज्ञं नाजानात्}। अत्र भूत\-कालिक\-क्रियायां प्रयोक्तव्यायां \textcolor{red}{जानाति} इति वर्तमान\-कालिक\-क्रिया प्रयुक्ताऽपाणिनीयेव। किन्तु \textcolor{red}{पुरा} इत्यस्याध्याहारे \textcolor{red}{पुरि लुङ् चास्मे} (पा॰सू॰~३.२.१२२) इत्यनेन पुरा\-योगे लड्लकारः। अर्थात् \textcolor{red}{पुरा न जानाति}।\footnote{\textcolor{red}{ज्ञा अवबोधने} (धा॰पा॰~१५०७)~\arrow ज्ञा~\arrow \textcolor{red}{शेषात्कर्तरि परस्मैपदम्} (पा॰सू॰~१.३.७८)~\arrow \textcolor{red}{पुरि लुङ् चास्मे} (पा॰सू॰~३.२.१२२)~\arrow ज्ञा~लट्~\arrow ज्ञा~तिप्~\arrow ज्ञा~ति~\arrow \textcolor{red}{क्र्यादिभ्यः श्ना} (पा॰सू॰~३.१.८१)~\arrow ज्ञा~श्ना~ति~\arrow ज्ञा~ना~ति~\arrow \textcolor{red}{ज्ञाजनोर्जा} (पा॰सू॰~७.३.७९)~\arrow जा~ना~ति~\arrow जानाति।} \textcolor{red}{पूर्वं नाजानात्} इत्यर्थः।\footnote{\textcolor{red}{ज्ञा अवबोधने} (धा॰पा॰~१५०७)~\arrow ज्ञा~\arrow \textcolor{red}{शेषात्कर्तरि परस्मैपदम्} (पा॰सू॰~१.३.७८)~\arrow \textcolor{red}{अनद्यतने लङ्} (पा॰सू॰~३.२.१११)~\arrow ज्ञा~लङ्~\arrow ज्ञा~तिप्~\arrow ज्ञा~ति~\arrow \textcolor{red}{लुङ्लङ्लृङ्क्ष्वडुदात्तः} (पा॰सू॰~६.४.७१)~\arrow \textcolor{red}{आद्यन्तौ टकितौ} (पा॰सू॰~१.१.४६)~\arrow अट्~ज्ञा~ति~\arrow अ~ज्ञा~ति~\arrow \textcolor{red}{क्र्यादिभ्यः श्ना} (पा॰सू॰~३.१.८१)~\arrow अ~ज्ञा~श्ना~ति~\arrow अ~ज्ञा~ना~ति~\arrow \textcolor{red}{ज्ञाजनोर्जा} (पा॰सू॰~७.३.७९)~\arrow अ~जा~ना~ति~\arrow \textcolor{red}{इतश्च} (पा॰सू॰~३.४.१००)~\arrow अ~जा~ना~त्~\arrow अजानात्।} ब्रह्मणा बोधितः सन् पश्चाद्व्यजानादिति पार्वत्यास्तात्पर्यम्। न च पुरा शब्दो दृश्यत इति वाच्यम्। गम्यमानः सः।\end{sloppypar}
\section[ब्रूत]{ब्रूत}
\centering\textcolor{blue}{अत्रोत्तरं किं विदितं भवद्भिस्तद्ब्रूत मे संशयभेदि वाक्यम्॥}\nopagebreak\\
\raggedleft{–~अ॰रा॰~१.१.१५}\\
\fontsize{14}{21}\selectfont\begin{sloppypar}\hyphenrules{nohyphenation}\justifying\noindent\hspace{10mm} भगवती पार्वती भगवन्तं शशाङ्क\-शेखरं पृच्छति \textcolor{red}{तत्संशय\-भेदि\-वाक्यं ब्रूत}। \textcolor{red}{यूयं ब्रूत} इत्यध्याहार्यम्। प्रथमपुरुषे प्रयोक्तव्ये\footnote{\textcolor{red}{विदितं भवद्भभिः} इति समभि\-व्याहारेण प्रथमपुरुषः प्रयोक्तव्यः।} मध्यम\-पुरुष\-प्रयोग अपाणिनीयः। एवं च भगवती भवानी पतिव्रता\-शिरोमणिः। सा च स्वकीय\-प्राण\-वल्लभाय मध्यम\-पुरुषस्य विशेषणं ददाति तत्रापि बहुवचनान्तमिति न परं प्रतिभाति। तथा च नियमः~– \textcolor{red}{एकत्वं न प्रयुञ्जीत गुरावात्मनि चेश्वरे}। पार्वत्यास्तु भगवाञ्छिव एव गुरुः। ऐश्वर्ये त्रिभुवन\-गुरुत्वान्माधुर्ये प्राण\-वल्लभत्वाच्च \textcolor{red}{पतिरेव गुरुः स्त्रीणाम्} (म॰भा॰~१४.१०८.६, ब्र॰पु॰~८०.४८, कू॰पु॰~१२.४९, चा॰नी॰~५.१) इति वचनाच्च। अतो बहुवचन\-प्रयोगः। तथाऽपि कथं \textcolor{red}{यूयं ब्रूत} इति चेत्। \textcolor{red}{अत्रोत्तरं किं विदितं भवद्भिः} इत्यत्र \textcolor{red}{भवद्भिः} इति विशेषणं शिवस्य कृते दत्तम्। अतः \textcolor{red}{यूयम्} इति तत्सम्बन्धि\-कर्तृ\-विशेषणं नैव विचारसहम्। \textcolor{red}{भवद्भिः} इत्यस्य हि \textcolor{red}{भवन्तः} इत्यनेन सम्बन्धः स्यात्। यथा \textcolor{red}{यदुत्तरं भवद्भिर्विदितं तद्भवन्तो वदन्तु}। किन्तु प्रश्ने कृते शिवः समाधौ न्यलीयत। तं समाधिस्थं विलोक्य तस्य पञ्च वक्त्राण्युद्दिश्य कथयति यत् \textcolor{red}{भोः पञ्च\-मुखानि यूयमेव संशय\-भेदकं वाक्यं ब्रूत}।\footnote{\textcolor{red}{ब्रूञ् व्यक्तायां वाचि} (धा॰पा॰~१०४४)~\arrow ब्रू~\arrow \textcolor{red}{शेषात्कर्तरि परस्मैपदम्} (पा॰सू॰~१.३.७८)~\arrow \textcolor{red}{लोट् च} (पा॰सू॰~३.३.१६२)~\arrow ब्रू~लोट्~\arrow ब्रू~थस्~\arrow \textcolor{red}{कर्तरि शप्} (पा॰सू॰~३.१.६८)~\arrow ब्रू~शप्~थस्~\arrow \textcolor{red}{अदिप्रभृतिभ्यः शपः} (पा॰सू॰~२.४.७२)~\arrow ब्रू~थस्~\arrow \textcolor{red}{तस्थस्थमिपां तान्तन्तामः} (पा॰सू॰~३.४.१०१)~\arrow ब्रू~ त~\arrow ब्रूत।}\end{sloppypar}
\section[वक्ष्ये]{वक्ष्ये}
\centering\textcolor{blue}{त्वयाऽद्य भक्त्या परिनोदितोऽहं वक्ष्ये नमस्कृत्य रघूत्तमं ते।\nopagebreak\\
रामः परात्मा प्रकृतेरनादिरानन्द एकः पुरुषोत्तमो हि॥}\nopagebreak\\
\raggedleft{–~अ॰रा॰~१.१.१७}\\
\fontsize{14}{21}\selectfont\begin{sloppypar}\hyphenrules{nohyphenation}\justifying\noindent\hspace{10mm}
भगवती\-पार्वती\-प्रश्नं श्रुत्वा शिवः कथयति \textcolor{red}{अहं भगवन्तं नमस्कृत्य वक्ष्ये}। \textcolor{red}{ब्रू}\-धातोः (\textcolor{red}{ब्रूञ् व्यक्तायां वाचि} धा॰पा॰~१०४४) \textcolor{red}{भूवादयो धातवः} (पा॰सू॰~१.३.१) इत्यनेन धातु\-सञ्ज्ञायां \textcolor{red}{लृट् शेषे च} (पा॰सू॰~३.३.१३) इत्यनेन लृड्लकार उत्तम\-पुरुषे \textcolor{red}{तिप्तस्झि\-सिप्थस्थ\-मिब्वस्मस्ताताञ्झ\-थासाथान्ध्वमिड्वहिमहिङ्} (पा॰सू॰~३.४.७८) इत्यनेन \textcolor{red}{तङानावात्मने\-पदम्} (पा॰सू॰~१.४.१००) इति सूत्रानुसारमिट्प्रत्यये \textcolor{red}{स्यतासी लृलुटोः} (पा॰सू॰~३.१.३३) इत्यनेन \textcolor{red}{स्य}\-प्रत्यये \textcolor{red}{ब्रुवो वचिः} (पा॰सू॰~२.४.५३) इत्यनेन ब्रू\-धातोर्वचादेशे \textcolor{red}{टित आत्मने\-पदानां टेरे} (पा॰सू॰~३.४.७९) इत्यनेनैकारादेशे \textcolor{red}{अतो गुणे} (पा॰सू॰~६.१.९७) इत्यनेन पररूपे पश्चाच्चकारस्य कुत्वे षत्वे \textcolor{red}{वक्ष्ये} इति। अत्र \textcolor{red}{वक्ष्यामि} इति कथं नेति चेत्।\footnote{\textcolor{red}{ब्रूञ् व्यक्तायां वाचि} (धा॰पा॰~१०४४)~\arrow ब्रू~\arrow \textcolor{red}{शेषात्कर्तरि परस्मैपदम्} (पा॰सू॰~१.३.७८)~\arrow \textcolor{red}{लृट् शेषे च} (पा॰सू॰~३.३.१३)~\arrow ब्रू~लृट्~\arrow ब्रू~मिप्~\arrow ब्रू~मि~\arrow \textcolor{red}{स्यतासी लृलुटोः} (पा॰सू॰~३.१.३३)~\arrow ब्रू~स्य~मि~\arrow \textcolor{red}{ब्रुवो वचिः} (पा॰सू॰~२.४.५३)~\arrow वच्~स्य~मि~\arrow \textcolor{red}{अतो दीर्घो यञि} (पा॰सू॰~७.३.१०१)~\arrow वच्~स्या~मि~\arrow \textcolor{red}{चोः कुः} (पा॰सू॰~८.२.३०)~\arrow वक्~स्या~मि~\arrow \textcolor{red}{आदेश\-प्रत्यययोः} (पा॰सू॰~८.३.५९)~\arrow वक्~ष्या~मि~\arrow वक्ष्यामि।} क्रिया\-फलस्य कर्तृ\-गामित्वात्।\footnote{\textcolor{red}{स्वरितञितः कर्त्रभिप्राये क्रियाफले} (पा॰सू॰~१.३.७२) इत्यनेन। \textcolor{red}{ब्रूञ् व्यक्तायां वाचि} (धा॰पा॰~१०४४)~\arrow ब्रू~\arrow \textcolor{red}{स्वरितञितः कर्त्रभिप्राये क्रियाफले} (पा॰सू॰~१.३.७२)~\arrow \textcolor{red}{लृट् शेषे च} (पा॰सू॰~३.३.१३)~\arrow ब्रू~लृट्~\arrow ब्रू~इट्~\arrow ब्रू~इ~\arrow \textcolor{red}{स्यतासी लृलुटोः} (पा॰सू॰~३.१.३३)~\arrow ब्रू~स्य~इ~\arrow \textcolor{red}{ब्रुवो वचिः} (पा॰सू॰~२.४.५३)~\arrow वच्~स्य~इ~\arrow \textcolor{red}{टित आत्मने\-पदानां टेरे} (पा॰सू॰~३.४.७९)~\arrow वच्~स्य~ए~\arrow \textcolor{red}{अतो गुणे} (पा॰सू॰~६.१.९७)~\arrow वच्~स्ये~\arrow \textcolor{red}{चोः कुः} (पा॰सू॰~८.२.३०)~\arrow वक्~स्ये~\arrow \textcolor{red}{आदेश\-प्रत्यययोः} (पा॰सू॰~८.३.५९)~\arrow वक्~ष्ये~\arrow वक्ष्ये।} रामायण\-कथा\-प्रश्नेन पार्वत्यास्तु सन्देहो नष्टो भविष्यत्येव किन्तु तत्कथनेन वक्तुः शशाङ्क\-मौलेरपि स्वान्तः\-सुखमुत्पत्स्यते। भगवत्कथा\-प्रश्नो वक्तारं प्रश्न\-कर्तारं श्रोतारमिति त्रीनपि पुनाति। तथा चोक्तं भागवते~–\end{sloppypar}
\centering\textcolor{red}{वासुदेवकथाप्रश्नः पुरुषांस्त्रीन् पुनाति हि।\nopagebreak\\
वक्तारं पृच्छकं श्रोतृंस्तत्पादसलिलं यथा॥}\nopagebreak\\
\raggedleft{–~भा॰पु॰~१०.१.१६}\\
\fontsize{14}{21}\selectfont\begin{sloppypar}\hyphenrules{nohyphenation}\justifying\noindent इयं हि गङ्गा। तथा चोक्तमत्रैव ग्रन्थे~–\end{sloppypar}
\centering\textcolor{blue}{पुरारिगिरिसम्भूता श्रीरामार्णवसङ्गता।\nopagebreak\\
अध्यात्मरामगङ्गेयं पुनाति भुवनत्रयम्॥}\nopagebreak\\
\raggedleft{–~अ॰रा॰~१.१.५}\\
\fontsize{14}{21}\selectfont\begin{sloppypar}\hyphenrules{nohyphenation}\justifying\noindent अतो \textcolor{red}{वक्ष्ये} इति समूलमेव।\end{sloppypar}
\section[भ्रमतीव दृश्यते]{भ्रमतीव दृश्यते}
\centering\textcolor{blue}{यथा हि चाक्ष्णा भ्रमता गृहादिकं विनष्टदृष्टेर्भ्रमतीव दृश्यते।\nopagebreak\\
तथैव देहेन्द्रियकर्तुरात्मनः कृते परेऽध्यस्य जनो विमुह्यति॥}\nopagebreak\\
\raggedleft{–~अ॰रा॰~१.१.२२}\\
\fontsize{14}{21}\selectfont\begin{sloppypar}\hyphenrules{nohyphenation}\justifying\noindent\hspace{10mm} अत्र \textcolor{red}{भ्रम्‌}\-धातुः (\textcolor{red}{भ्रमुँ अनवस्थाने} धा॰पा॰~१२०४) दिवादिः। एवं \textcolor{red}{दिवादिभ्यः श्यन्} (पा॰सू॰~३.१.६९) इत्यनेन लटि तिपि श्यन्प्रत्यये कृते \textcolor{red}{शमामष्टानां दीर्घः श्यनि} (पा॰सू॰~७.३.७४) इत्यनेन दीर्घे \textcolor{red}{भ्राम्यति} इति पाणिनीयम्।\footnote{\textcolor{red}{भ्रमुँ अनवस्थाने} (धा॰पा॰~१२०४)~\arrow भ्रम्~\arrow \textcolor{red}{शेषात्कर्तरि परस्मैपदम्} (पा॰सू॰~१.३.७८)~\arrow \textcolor{red}{वर्तमाने लट्} (पा॰सू॰~३.२.१२३)~\arrow भ्रम्~लट्~\arrow भ्रम्~तिप्~\arrow भ्रम्~ति~\arrow \textcolor{red}{दिवादिभ्यः श्यन्} (पा॰सू॰~३.१.६९)~\arrow भ्रम्~श्यन्~ति~\arrow भ्रम्~य~ति~\arrow \textcolor{red}{शमामष्टानां दीर्घः श्यनि} (पा॰सू॰~७.३.६४)~\arrow भ्राम्~य~ति~\arrow भ्राम्यति।} \textcolor{red}{भ्रमति} इति कथमिति चेत्। \textcolor{red}{वा भ्राश\-भ्लाश\-भ्रमु\-क्रमु\-क्लमु\-त्रसि\-त्रुटि\-लषः} (पा॰सू॰~३.१.७०) इत्यनेन श्यनो विकल्पात् \textcolor{red}{भ्रमति}।\footnote{\textcolor{red}{भ्रमुँ अनवस्थाने} (धा॰पा॰~१२०४)~\arrow भ्रम्~\arrow \textcolor{red}{शेषात्कर्तरि परस्मैपदम्} (पा॰सू॰~१.३.७८)~\arrow \textcolor{red}{वर्तमाने लट्} (पा॰सू॰~३.२.१२३)~\arrow भ्रम्~लट्~\arrow भ्रम्~तिप्~\arrow भ्रम्~ति~\arrow \textcolor{red}{वा भ्राश\-भ्लाश\-भ्रमु\-क्रमु\-क्लमु\-त्रसि\-त्रुटि\-लषः} (पा॰सू॰~३.१.७०)~\arrow भ्रम्~शप्~ति~\arrow भ्रम्~अ~ति~\arrow भ्रमति।} यद्वा \textcolor{red}{भ्रममाचरतीति भ्रमति} अनेन प्रकारेण सिद्धम्।\footnote{भ्रमणं भ्रमः। \textcolor{red}{भावे} (पा॰सू॰~३.३.१८) इत्यनेन भावे घञ्। भ्रम~\arrow \textcolor{red}{सर्वप्राति\-पदिकेभ्य आचारे क्विब्वा वक्तव्यः} (वा॰~३.१.११)~\arrow भ्रम~क्विँप्~\arrow भ्रम~व्~\arrow \textcolor{red}{वेरपृक्तस्य} (पा॰सू॰~६.१.६७)~\arrow भ्रम~\arrow \textcolor{red}{सनाद्यन्ता धातवः} (पा॰सू॰~३.१.३२)~\arrow धातुसञ्ज्ञा~\arrow \textcolor{red}{शेषात्कर्तरि परस्मैपदम्} (पा॰सू॰~१.३.७८)~\arrow वर्तमाने लट्~\arrow भ्रम~लट्~\arrow भ्रम~तिप्~\arrow भ्रम~ति~\arrow \textcolor{red}{कर्तरि शप्‌} (पा॰सू॰~३.१.६८)~\arrow भ्रम~शप्~ति~\arrow भ्रम~अ~ति~\arrow \textcolor{red}{अतो गुणे} (पा॰सू॰~६.१.९७)~\arrow भ्रम~ति~\arrow भ्रमति।} यद्वा \textcolor{red}{गण\-कार्यमनित्यम्} (प॰शे॰~९३.३) इत्यनेन श्यनभावे भ्वादित्वाच्छपि।\footnote{\textcolor{red}{भ्रमुँ अनवस्थाने} (धा॰पा॰~१२०४)~\arrow भ्रम्~\arrow \textcolor{red}{शेषात्कर्तरि परस्मैपदम्} (पा॰सू॰~१.३.७८)~\arrow \textcolor{red}{वर्तमाने लट्} (पा॰सू॰~३.२.१२३)~\arrow भ्रम्~लट्~\arrow भ्रम्~तिप्~\arrow भ्रम्~ति~\arrow \textcolor{red}{गण\-कार्यमनित्यम्} (प॰शे॰~९३.३)~\arrow \textcolor{red}{कर्तरि शप्} (पा॰सू॰~३.१.६८)~\arrow भ्रम्~शप्~ति~\arrow भ्रम्~अ~ति~\arrow भ्रमति।} यद्वा \textcolor{red}{भ्रमति} इति नास्ति क्रियाऽपि तु 
\textcolor{red}{भ्रमती इव} इति \textcolor{red}{भ्रम्‌}\-धातोरौणादिक\-तृच्प्रत्ययान्तो \textcolor{red}{भ्रमती}।\footnote{अत्र \textcolor{red}{तृँच्} प्रत्यय इति भावः। नायं \textcolor{red}{बहुलमन्यत्रापि} (प॰उ॰~२.९५) इति तृच्। स नोगित्। \textcolor{red}{कार्याद्विद्यादनूबन्धम्} (भा॰पा॰सू॰~३.३.१) \textcolor{red}{केचिदविहिता अप्यूह्याः} (वै॰सि॰कौ॰~३१६९) इत्यनुसारमूह्योऽ\-यमविहित उगित्प्रत्ययः। \textcolor{red}{तृँच्} प्रत्यये चात्र शबागमोऽप्यूह्यः। \textcolor{red}{नयतेः षुगागमः} (प॰उ॰ श्वे॰वृ॰~२.९६) इतिवत्। \textcolor{red}{भ्रमुँ अनवस्थाने} (धा॰पा॰~१२०४)~\arrow भ्रम्~\arrow \textcolor{red}{उणादयो बहुलम्} (पा॰सू॰~३.३.१)~\arrow भ्रम्~तृँच्~\arrow भ्रम्~शप्~अत्~\arrow भ्रम्~अ~त्~\arrow भ्रमत्~\arrow \textcolor{red}{उगितश्च} (पा॰सू॰~४.१.६)~\arrow भ्रमत्~ङीप्~\arrow भ्रमत्~ई~\arrow भ्रमती~\arrow विभक्तिकार्यम्~\arrow भ्रमती~सुँ~\arrow भ्रमती~स्~\arrow \textcolor{red}{हल्ङ्याब्भ्यो दीर्घात्सुतिस्यपृक्तं हल्} (पा॰सू॰~६.१.६८)~\arrow भ्रमती।}\end{sloppypar}
\section[आकाङ्क्षते]{आकाङ्क्षते}
\centering\textcolor{blue}{रामो न गच्छति न तिष्ठति नानुशोचत्याकाङ्क्षते त्यजति नो न करोति किञ्चित्।\nopagebreak\\
आनन्दमूर्तिरचलः परिणामहीनो मायागुणाननुगतो हि तथा विभाति॥}\nopagebreak\\
\raggedleft{–~अ॰रा॰~१.१.४३}\\
\fontsize{14}{21}\selectfont\begin{sloppypar}\hyphenrules{nohyphenation}\justifying\noindent\hspace{10mm} अत्र सीता श्रीराम\-तत्त्वं वर्णयन्ती \textcolor{red}{आकाङ्क्षते} इति प्रयुङ्क्ते। यत् \textcolor{red}{रामो नाऽकाङ्क्षते}। \textcolor{red}{काङ्क्ष्‌}\-धातुः (\textcolor{red}{काक्षिँ काङ्क्षायाम्} धा॰पा॰~६६७) परस्मैपदी। एवं ततो लटि तिपि शपि \textcolor{red}{आकाङ्क्षति}।\footnote{आङ्~\textcolor{red}{काक्षिँ काङ्क्षायाम्} (धा॰पा॰~६६७)~\arrow आ~काक्ष्~\arrow \textcolor{red}{इदितो नुम् धातोः} (पा॰सू॰~७.१.५८)~\arrow \textcolor{red}{मिदचोऽन्त्यात्परः} (पा॰सू॰~१.१.४७)~\arrow आ~का~नुँम्~क्ष्~\arrow आ~कान्~क्ष्~\arrow \textcolor{red}{नश्चापदान्तस्य झलि} (पा॰सू॰~८.३.२४)~\arrow आ~कांक्ष्~\arrow \textcolor{red}{अनुस्वारस्य ययि परसवर्णः} (पा॰सू॰~८.४.५८)~\arrow आ~काङ्क्ष्~\arrow \textcolor{red}{शेषात्कर्तरि परस्मैपदम्} (पा॰सू॰~१.३.७८)~\arrow \textcolor{red}{वर्तमाने लट्} (पा॰सू॰~३.२.१२३)~\arrow आ~काङ्क्ष्~लट्~\arrow आ~काङ्क्ष्~तिप्~\arrow आ~काङ्क्ष्~ति~\arrow \textcolor{red}{कर्तरि शप्‌} (पा॰सू॰~३.१.६८)~\arrow आ~काङ्क्ष्~शप्~ति~\arrow आ~काङ्क्ष्~अ~ति~\arrow आकाङ्क्षति।} तथा च \textcolor{red}{नहि प्रफुल्लं सहकारमेत्य वृक्षान्तरं काङ्क्षति षट्पदाली} (र॰वं॰~६.६९) इति कालिदासोऽपि प्रायुङ्क्तेति चेत्कथमत्रात्मनेपदम्। अत्र विमृश्यते। \textcolor{red}{कर्तरि कर्म\-व्यतिहारे} (पा॰सू॰~१.३.१४) इत्यनेनाऽत्मनेपदम्।\footnote{आ~काङ्क्ष् (पूर्ववत्)~\arrow \textcolor{red}{कर्तरि कर्म\-व्यतिहारे} (पा॰सू॰~१.३.१४)~\arrow \textcolor{red}{वर्तमाने लट्} (पा॰सू॰~३.२.१२३)~\arrow आ~काङ्क्ष्~लट्~\arrow आ~काङ्क्ष्~त~\arrow \textcolor{red}{कर्तरि शप्‌} (पा॰सू॰~३.१.६८)~\arrow आ~काङ्क्ष्~शप्~त~\arrow आ~काङ्क्ष्~अ~त~\arrow \textcolor{red}{टित आत्मनेपदानां टेरे} (पा॰सू॰~३.४.७९)~\arrow आ~काङ्क्ष्~अ~ते~\arrow आकाङ्क्षते।} कर्म\-व्यतिहारो हि क्रिया\-विनिमयः। जीवस्याऽकाङ्क्षा भगवत्यारोपिता। तामेव निरस्यति \textcolor{red}{नाकाङ्क्षते} इति।\end{sloppypar}
\section[पठति]{पठति}
\centering\textcolor{blue}{योऽतिभ्रष्टोऽतिपापी परधनपरदारेषु नित्योद्यतो वा\nopagebreak\\
स्तेयी ब्रह्मघ्नमातापितृवधनिरतो योगिवृन्दापकारी।\nopagebreak\\
यः सम्पूज्याभिरामं पठति च हृदयं रामचन्द्रस्य भक्त्या\nopagebreak\\
योगीन्द्रैरप्यलभ्यं पदमिह लभते सर्वदेवैः स पूज्यम्॥}\nopagebreak\\
\raggedleft{–~अ॰रा॰~१.१.५६}\\
\fontsize{14}{21}\selectfont\begin{sloppypar}\hyphenrules{nohyphenation}\justifying\noindent\hspace{10mm} \textcolor{red}{पठति} इति भविष्यत्कालार्थे वर्तमान\-कालं प्रयुङ्क्त इति चेत्। \textcolor{red}{वर्तमान\-सामीप्ये वर्तमानवद्वा} (पा॰सू॰~३.३.१३१) इत्यनेन वर्तमान\-कालः।\footnote{\textcolor{red}{पठँ व्यक्तायां वाचि} (धा॰पा॰~३३०)~\arrow पठ्~\arrow \textcolor{red}{शेषात्कर्तरि परस्मैपदम्} (पा॰सू॰~१.३.७८)~\arrow \textcolor{red}{वर्तमान\-सामीप्ये वर्तमानवद्वा} (पा॰सू॰~३.३.१३१)~\arrow \textcolor{red}{वर्तमाने लट्} (पा॰सू॰~३.२.१२३)~\arrow पठ्~लट्~\arrow पठ्~तिप्~\arrow पठ्~ति~\arrow \textcolor{red}{कर्तरि शप्} (पा॰सू॰~३.१.६८)~\arrow पठ्~शप्~ति~\arrow पठ्~अ~ति~\arrow पठति। लृटि च \textcolor{red}{पठिष्यति} इति रूपम्। \textcolor{red}{पठँ व्यक्तायां वाचि} (धा॰पा॰~३३०)~\arrow पठ्~\arrow \textcolor{red}{शेषात्कर्तरि परस्मैपदम्} (पा॰सू॰~१.३.७८)~\arrow \textcolor{red}{लृट् शेषे च} (पा॰सू॰~३.३.१३)~\arrow पठ्~लृट्~\arrow पठ्~तिप्~\arrow पठ्~ति~\arrow \textcolor{red}{स्यतासी लृलुटोः} (पा॰सू॰~३.१.३३)~\arrow पठ्~स्य~ति~\arrow \textcolor{red}{आर्धधातुकस्येड्वलादेः} (पा॰सू॰~७.२.३५)~\arrow पठ्~इट्~स्य~ति~\arrow पठ्~इ~स्य~ति~\arrow \textcolor{red}{आदेश\-प्रत्यययोः} (पा॰सू॰~८.३.५९)~\arrow पठ्~इ~ष्य~ति~\arrow पठिष्यति।} एवमेव \textcolor{red}{लभते} इत्यत्रापि। अर्थात्त्वरितमेव फलसिद्धिर्भविष्यति।\footnote{\textcolor{red}{डुलभँष् प्राप्तौ} (धा॰पा॰~९७५)~\arrow \textcolor{red}{अनुदात्तङित आत्मने\-पदम्} (पा॰सू॰~१.३.१२)~\arrow \textcolor{red}{वर्तमान\-सामीप्ये वर्तमानवद्वा} (पा॰सू॰~३.३.१३१)~\arrow \textcolor{red}{वर्तमाने लट्} (पा॰सू॰~३.२.१२३)~\arrow लभ्~लट्~\arrow लभ्~त~\arrow \textcolor{red}{कर्तरि शप्} (पा॰सू॰~३.१.६८)~\arrow लभ्~शप्~त~\arrow लभ्~अ~त~\arrow \textcolor{red}{टित आत्मनेपदानां टेरे} (पा॰सू॰~३.४.७९)~\arrow लभ्~अ~ते~\arrow लभते। लृटि च \textcolor{red}{लप्स्यते} इति रूपम्। \textcolor{red}{डुलभँष् प्राप्तौ} (धा॰पा॰~९७५)~\arrow \textcolor{red}{अनुदात्तङित आत्मने\-पदम्} (पा॰सू॰~१.३.१२)~\arrow \textcolor{red}{लृट् शेषे च} (पा॰सू॰~३.३.१३)~\arrow लभ्~लृट्~\arrow लभ्~त~\arrow \textcolor{red}{स्यतासी लृलुटोः} (पा॰सू॰~३.१.३३)~\arrow लभ्~स्य~त~\arrow \textcolor{red}{एकाच उपदेशेऽनुदात्तात्‌} (पा॰सू॰~७.२.१०)~\arrow इडागम\-निषेधः~\arrow \textcolor{red}{खरि च} (पा॰सू॰~८.४.५५)~\arrow लप्~स्य~त~\arrow \textcolor{red}{टित आत्मनेपदानां टेरे} (पा॰सू॰~३.४.७९)~\arrow लप्~स्य~ते~\arrow लप्स्यते।}\end{sloppypar}
\section[मुच्यते]{मुच्यते}
\centering\textcolor{blue}{तदद्य कथयिष्यामि शृणु तापत्रयापहम्।\nopagebreak\\
यच्छ्रुत्वा मुच्यते जन्तुरज्ञानोत्थमहाभयात्।\nopagebreak\\
प्राप्नोति परमामृद्धिं दीर्घायुः पुत्रसन्ततिम्॥}\nopagebreak\\
\raggedleft{–~अ॰रा॰~१.२.५}\\
\fontsize{14}{21}\selectfont\begin{sloppypar}\hyphenrules{nohyphenation}\justifying\noindent\hspace{10mm} अत्र रामायणस्य शाश्वतत्वाद्वर्तमान\-कालः।\footnote{रामायणस्य शाश्वतत्वं रामायण एवोक्तम्~– \textcolor{red}{यावत्स्थास्यन्ति गिरयः सरितः च महीतले॥ तावद्रामायणकथा लोकेषु प्रचरिष्यति।} (वा॰रा॰~१.२.३६-३७)। \textcolor{red}{मुचॢँ मोक्षणे} (धा॰पा॰~१४३०)~\arrow मुच्~\arrow \textcolor{red}{भावकर्मणोः} (पा॰सू॰~१.३.१३)~\arrow \textcolor{red}{वर्तमाने लट्} (पा॰सू॰~३.२.१२३)~\arrow मुच्~लट्~\arrow मुच्~त~\arrow \textcolor{red}{सार्वधातुके यक्} (पा॰सू॰~३.१.६७)~\arrow मुच्~यक्~त~\arrow मुच्~य~त~\arrow \textcolor{red}{ग्क्ङिति च} (पा॰सू॰~१.१.५)~\arrow लघूपध\-गुण\-निषेधः~\arrow मुच्~य~त~\arrow \textcolor{red}{टित आत्मनेपदानां टेरे} (पा॰सू॰~३.४.७९)~\arrow मुच्~य~ते~\arrow मुच्यते।} वैयाकरणानां मते शब्दानां नित्यत्वात्। एवं भक्तानां मते भगवतस्तत्कथायाश्च नित्यत्वात्। अतो जीवानां दृष्टौ वर्तमान\-भूत\-भविष्यत्कालाः। त्रिकालाबाध्यत्वात्कालातीतत्वाच्चेश्वरस्य समक्षं निरन्तरं वर्तमान\-कालः।\end{sloppypar}
\section[सृजामि]{सृजामि}
\centering\textcolor{blue}{तस्याहं पुत्रतामेत्य कौसल्यायां शुभे दिने।\nopagebreak\\
चतुर्धाऽऽत्मानमेवाहं सृजामीतरयोः पृथक्॥}\nopagebreak\\
\raggedleft{–~अ॰रा॰~१.२.२७}\\
\fontsize{14}{21}\selectfont\begin{sloppypar}\hyphenrules{nohyphenation}\justifying\noindent\hspace{10mm} अत्र भगवाञ्छ्रीरामभद्रोऽवतार\-कथां वर्णयन्नाह यत् \textcolor{red}{कौशल्यायामितरयोश्चाऽत्मानं चतुर्धा सृजामि}। अत्र \textcolor{red}{स्रक्ष्यामि} इति हि पाणिनीयम्।\footnote{\textcolor{red}{सृजँ विसर्गे} (धा॰पा॰~१४१४)~\arrow सृज्~\arrow \textcolor{red}{शेषात्कर्तरि परस्मैपदम्} (पा॰सू॰~१.३.७८)~\arrow \textcolor{red}{लृट् शेषे च} (पा॰सू॰~३.३.१३)~\arrow सृज्~लृट्~\arrow सृज्~मिप्~\arrow सृज्~मि~\arrow \textcolor{red}{स्यतासी लृलुटोः} (पा॰सू॰~३.१.३३)~\arrow सृज्~स्य~मि~\arrow \textcolor{red}{सृजि\-दृशोर्झल्यमकिति} (पा॰सू॰~६.१.५८)~\arrow \textcolor{red}{मिदचोऽन्त्यात्परः} (पा॰सू॰~१.१.४७)~\arrow सृ~अम्~ज्~स्य~मि~\arrow सृ~अ~ज्~स्य~मि~\arrow \textcolor{red}{इको यणचि} (पा॰सू॰~६.१.७७)~\arrow स्र्~अ~ज्~स्य~मि~\arrow \textcolor{red}{व्रश्चभ्रस्ज\-सृजमृज\-यजराज\-भ्राजच्छशां षः} (पा॰सू॰~८.२.३६)~\arrow स्र्~अ~ष्~स्य~मि~\arrow \textcolor{red}{षढोः कः सि} (पा॰सू॰~८.२.४१)~\arrow स्र्~अ~क्~स्य~मि~\arrow \textcolor{red}{आदेश\-प्रत्यययोः} (पा॰सू॰~८.३.५९)~\arrow स्र्~अ~क्~ष्य~मि~\arrow \textcolor{red}{अतो दीर्घो यञि} (पा॰सू॰~७.३.१०१)~\arrow स्र्~अ~क्~ष्या~मि~\arrow स्रक्ष्यामि।} किन्तु वर्तमान\-सामीप्यात् \textcolor{red}{सृजामि} इति न दोषः।\footnote{\textcolor{red}{वर्तमान\-सामीप्ये वर्तमानवद्वा} (पा॰सू॰~३.३.१३१) इत्यनेन। \textcolor{red}{सृजँ विसर्गे} (धा॰पा॰~१४१४)~\arrow सृज्~\arrow \textcolor{red}{शेषात्कर्तरि परस्मैपदम्} (पा॰सू॰~१.३.७८)~\arrow \textcolor{red}{वर्तमान\-सामीप्ये वर्तमानवद्वा} (पा॰सू॰~३.३.१३१)~\arrow \textcolor{red}{वर्तमाने लट्} (पा॰सू॰~३.२.१२३)~\arrow सृज्~लट्~\arrow सृज्~मिप्~\arrow सृज्~मि~\arrow \textcolor{red}{तुदादिभ्यः शः} (पा॰सू॰~३.१.७७)~\arrow सृज्~श~मि~\arrow सृज्~अ~मि~\arrow \textcolor{red}{सार्वधातुकमपित्} (पा॰सू॰~१.२.४)~\arrow ङित्त्वम्~\arrow \textcolor{red}{ग्क्ङिति च} (पा॰सू॰~१.१.५)~\arrow लघूपध\-गुण\-निषेधः~\arrow सृज्~अ~मि~\arrow \textcolor{red}{अतो दीर्घो यञि} (पा॰सू॰~७.३.१०१)~\arrow सृज्~आ~मि~\arrow सृजामि।}\end{sloppypar}
\section[सृजध्वम्]{सृजध्वम्}
\centering\textcolor{blue}{यूयं सृजध्वं सर्वेऽपि वानरेष्वंशसम्भवान्।\nopagebreak\\
विष्णोः सहायं कुरुत यावत्स्थास्यति भूतले॥}\nopagebreak\\
\raggedleft{–~अ॰रा॰~१.२.३०}\\
\fontsize{14}{21}\selectfont\begin{sloppypar}\hyphenrules{nohyphenation}\justifying\noindent\hspace{10mm} ब्रह्मा देवानादिशति यत् \textcolor{red}{यूयं वानरेष्वात्मानं सृजध्वम्}। रचयध्वमिति भावः। \textcolor{red}{सृज्‌}\-धातुः (\textcolor{red}{सृजँ विसर्गे} धा॰पा॰~१४१४) आत्मने\-पदी न।\footnote{तस्माल्लोड्लकारे मध्यमपुरुषे बहुवचने \textcolor{red}{सृजथ} इति रूपम्। यथा~– \textcolor{red}{यूयमप्यहह पूजनमस्या यन्निजैः सृजथ पादपयोजैः} (नै॰च॰~५.९६) इति श्रीहर्षप्रयोगे। \textcolor{red}{सृजँ विसर्गे} (धा॰पा॰~१४१४)~\arrow सृज्~\arrow \textcolor{red}{शेषात्कर्तरि परस्मैपदम्} (पा॰सू॰~१.३.७८)~\arrow \textcolor{red}{लोट् च} (पा॰सू॰~३.३.१६२)~\arrow सृज्~लोट्~\arrow सृज्~थ~\arrow \textcolor{red}{तुदादिभ्यः शः} (पा॰सू॰~३.१.७७)~\arrow सृज्~श~थ~\arrow सृज्~अ~ध्वम्~\arrow \textcolor{red}{सार्वधातुकमपित्} (पा॰सू॰~१.२.४)~\arrow ङित्त्वम्~\arrow \textcolor{red}{ग्क्ङिति च} (पा॰सू॰~१.१.५)~\arrow लघूपध\-गुण\-निषेधः~\arrow सृज्~अ~थ~\arrow सृजथ।} कथमत्र \textcolor{red}{सृजध्वम्} इति चेत्। अत्र कर्म\-व्यतिहारादात्मने\-पदम्।\footnote{\textcolor{red}{कर्तरि कर्म\-व्यतिहारे} पा॰सू॰~१.३.१४) इत्यनेन।} सर्जनं ब्रह्मणः कर्म तदेव देवेभ्यो दीयत इति क्रिया\-विनिमयः। आत्मनेपदे लोड्लकारे ध्वमि शे \textcolor{red}{सृजध्वम्}।\footnote{\textcolor{red}{सृजँ विसर्गे} (धा॰पा॰~१४१४)~\arrow सृज्~\arrow \textcolor{red}{कर्तरि कर्म\-व्यतिहारे} (पा॰सू॰~१.३.१४)~\arrow \textcolor{red}{लोट् च} (पा॰सू॰~३.३.१६२)~\arrow सृज्~लोट्~\arrow सृज्~ध्वम्~\arrow \textcolor{red}{तुदादिभ्यः शः} (पा॰सू॰~३.१.७७)~\arrow सृज्~श~ध्वम्~\arrow सृज्~अ~ध्वम्~\arrow \textcolor{red}{सार्वधातुकमपित्} (पा॰सू॰~१.२.४)~\arrow ङित्त्वम्~\arrow \textcolor{red}{ग्क्ङिति च} (पा॰सू॰~१.१.५)~\arrow लघूपध\-गुण\-निषेधः~\arrow सृज्~अ~ध्वम्~\arrow सृजध्वम्।}\end{sloppypar}
\section[दर्शयस्व]{दर्शयस्व}
\centering\textcolor{blue}{उपसंहर विश्वात्मन्नदो रूपमलौकिकम्।\nopagebreak\\
दर्शयस्व महानन्दबालभावं सुकोमलम्।\nopagebreak\\
ललितालिङ्गनालापैस्तरिष्याम्युत्कटं तमः॥}\nopagebreak\\
\raggedleft{–~अ॰रा॰~१.३.२९}\\
\fontsize{14}{21}\selectfont\begin{sloppypar}\hyphenrules{nohyphenation}\justifying\noindent\hspace{10mm} अत्र भगवती कौशल्या स्व\-समक्षं प्रकटं श्रीराम\-भद्रं प्रार्थयते यत् \textcolor{red}{बाल\-भावं दर्शयस्व}। अत्र \textcolor{red}{दृश्‌}\-धातोः (\textcolor{red}{दृशिँर् प्रेक्षणे} धा॰पा॰~९८८) 
णिचि लघूपध\-गुणे रपरत्वे धातु\-सञ्ज्ञायां पुनर्लोड्लकार आत्मने\-पदे \textcolor{red}{थास्‌}\-प्रत्यये शपि गुणेऽयादेशे स्वादेशे \textcolor{red}{दर्शयस्व}।\footnote{\textcolor{red}{दृशिँर् प्रेक्षणे} (धा॰पा॰~९८८)~\arrow दृश्~\arrow \textcolor{red}{हेतुमति च} (पा॰सू॰~३.१.२६)~\arrow दृश्~णिच्~\arrow दृश्~इ~\arrow \textcolor{red}{पुगन्त\-लघूपधस्य च} (पा॰सू॰~७.३.८६)~\arrow दश्~इ~\arrow \textcolor{red}{उरण् रपरः} (पा॰सू॰~१.१.५१)~\arrow दर्श्~इ~\arrow दर्शि~\arrow \textcolor{red}{सनाद्यन्ता धातवः} (पा॰सू॰~३.१.३२)~\arrow धातुसञ्ज्ञा~\arrow \textcolor{red}{णिचश्च} (पा॰सू॰~१.३.७४)~\arrow \textcolor{red}{लोट् च} (पा॰सू॰~३.३.१६२)~\arrow दर्शि~लोट्~\arrow दर्शि~थास्~\arrow \textcolor{red}{कर्तरि शप्} (पा॰सू॰~३.१.६८)~\arrow दर्शि~शप्~थास्~\arrow दर्शि~अ~थास्~\arrow \textcolor{red}{सार्वधातुकार्धधातुकयोः} (पा॰सू॰~७.३.८४)~\arrow दर्शे~अ~थास्~\arrow \textcolor{red}{एचोऽयवायावः} (पा॰सू॰~६.१.७८)~\arrow दर्शय्~अ~थास्~\arrow \textcolor{red}{थासस्से} (पा॰सू॰~३.४.८०)~\arrow दर्शय्~अ~से~\arrow \textcolor{red}{सवाभ्यां वामौ} (पा॰सू॰~३.४.९१)~\arrow दर्शय्~अ~स्व~\arrow दर्शयस्व।} आत्मनेपदं यदा कर्तरि फलं दृश्येत। अत्र राम\-रूपे कर्तरि किं फलं श्रीरामस्य फलानपेक्षत्वादिति चेत्। भक्तानन्द एव तस्यापूर्वं फलम्। भक्त\-हितार्थमेव गृहीत\-जन्मत्वात्। यद्वा \textcolor{red}{स्व} इति सम्बोधनम्। \textcolor{red}{हे स्व दर्शय बाल\-भावम्}।\footnote{दर्शि~\arrow धातुसञ्ज्ञा (पूर्ववत्)~\arrow \textcolor{red}{शेषात्कर्तरि परस्मैपदम्} (पा॰सू॰~१.३.७८)~\arrow \textcolor{red}{लोट् च} (पा॰सू॰~३.३.१६२)~\arrow दर्शि~लोट्~\arrow दर्शि~सिप्~\arrow दर्शि~सि~\arrow \textcolor{red}{कर्तरि शप्} (पा॰सू॰~३.१.६८)~\arrow दर्शि~शप्~सि~\arrow दर्शि~अ~सि~\arrow \textcolor{red}{सार्वधातुकार्धधातुकयोः} (पा॰सू॰~७.३.८४)~\arrow दर्शे~अ~सि~\arrow \textcolor{red}{एचोऽयवायावः} (पा॰सू॰~६.१.७८)~\arrow दर्शय्~अ~सि~\arrow \textcolor{red}{सेर्ह्यपिच्च} (पा॰सू॰~३.४.८७)~\arrow दर्शय्~अ~हि~\arrow \textcolor{red}{अतो हेः} (पा॰सू॰~६.४.१०५)~\arrow दर्शय्~अ~\arrow दर्शय।} स्व\-शब्दस्य चत्वारोऽर्था आत्माऽऽत्मीयो ज्ञातिर्धनञ्च।\footnote{\textcolor{red}{स्वे स्वाः। आत्मीया इत्यर्थः। आत्मान इति वा। ज्ञाति\-धन\-वाचिनस्तु स्वाः। ज्ञातयोऽर्था वा} (वै॰सि॰कौ॰~२२०)।} अतः \textcolor{red}{हे स्व हे मम आत्मन्मम धन} इति निजं दर्शय। यद्वा \textcolor{red}{स्व} इति \textcolor{red}{महानन्दम्}
इत्यस्य विशेषणम्। एवं च \textcolor{red}{स्वेभ्य आत्मीयेभ्यो महानानन्दो यस्मात्स स्वमहानन्दः}। \textcolor{red}{स्वमहानन्दश्चासौ बालभावश्चेति स्वमहानन्द\-बालभावस्तं स्वमहानन्द\-बालभावम्}।\end{sloppypar}
\section[याति]{याति}
\centering\textcolor{blue}{संवादमावयोर्यस्तु पठेद्वा शृणुयादपि।\nopagebreak\\
स याति मम सारूप्यं मरणे मत्स्मृतिं लभेत्॥}\nopagebreak\\
\raggedleft{–~अ॰रा॰~१.३.३४}\\
\fontsize{14}{21}\selectfont\begin{sloppypar}\hyphenrules{nohyphenation}\justifying\noindent\hspace{10mm} अत्र भगवाञ्छ्रीरामो मात्रा सह स्वकीय\-संवादस्य फल\-श्रुतिं वर्णयति \textcolor{red}{य आवयोः संवादं पठेच्छृणुयाद्वा स मम सारूप्यं यायात्}। \textcolor{red}{यायात्}\footnote{आशीर्वादार्थे लिङि \textcolor{red}{यायात्} इति रूपम्। \textcolor{red}{या प्रापणे} (धा॰पा॰~१०४९)~\arrow या~\arrow \textcolor{red}{शेषात्कर्तरि परस्मैपदम्} (पा॰सू॰~१.३.७८)~\arrow \textcolor{red}{आशिषि लिङ्लोटौ} (पा॰सू॰~३.३.१७३)~\arrow या~लिङ्~\arrow या~तिप्~\arrow या~ति~\arrow \textcolor{red}{यासुट् परस्मैपदेषूदात्तो ङिच्च} (पा॰सू॰~३.४.१०३)~\arrow \textcolor{red}{आद्यन्तौ टकितौ} (पा॰सू॰~१.१.४६)~\arrow या~यासुँट्~ति~\arrow या~यास्~ति~\arrow \textcolor{red}{सुट् तिथोः} (पा॰सू॰~३.४.१०७)~\arrow \textcolor{red}{आद्यन्तौ टकितौ} (पा॰सू॰~१.१.४६)~\arrow या~यास्~सुँट्~ति~\arrow या~यास्~स्~ति~\arrow \textcolor{red}{स्कोः संयोगाद्योरन्ते च} (पा॰सू॰~८.२.२९)~\arrow या~या~ति~\arrow \textcolor{red}{इतश्च} (पा॰सू॰~३.४.१००)~\arrow या~या~त्~\arrow यायात्। विध्यर्थे लिङ्यपि \textcolor{red}{यायात्} इत्येव रूपं प्रक्रिया तु भिन्ना। \textcolor{red}{या प्रापणे} (धा॰पा॰~१०४९)~\arrow या~\arrow \textcolor{red}{शेषात्कर्तरि परस्मैपदम्} (पा॰सू॰~१.३.७८)~\arrow \textcolor{red}{विधि\-निमन्‍त्रणामन्‍त्रणाधीष्‍ट\-सम्प्रश्‍न\-प्रार्थनेषु लिङ्} (पा॰सू॰~३.३.१६१)~\arrow या~लिङ्~\arrow या~तिप्~\arrow या~ति~\arrow \textcolor{red}{कर्तरि शप्} (पा॰सू॰~३.१.६८)~\arrow या~शप्~ति~\arrow \textcolor{red}{अदिप्रभृतिभ्यः शपः} (पा॰सू॰~२.४.७२)~\arrow या~ति~\arrow \textcolor{red}{यासुट् परस्मैपदेषूदात्तो ङिच्च} (पा॰सू॰~३.४.१०३)~\arrow \textcolor{red}{आद्यन्तौ टकितौ} (पा॰सू॰~१.१.४६)~\arrow या~यासुँट्~ति~\arrow या~यास्~ति~\arrow \textcolor{red}{सुट् तिथोः} (पा॰सू॰~३.४.१०७)~\arrow \textcolor{red}{आद्यन्तौ टकितौ} (पा॰सू॰~१.१.४६)~\arrow या~यास्~सुँट्~ति~\arrow या~यास्~स्~ति~\arrow \textcolor{red}{लिङः सलोपोऽनन्त्यस्य} (पा॰सू॰~७.२.७९)~\arrow या~या~ति~\arrow \textcolor{red}{इतश्च} (पा॰सू॰~३.४.१००)~\arrow या~या~त्~\arrow यायात्।} इति प्रयोक्तव्ये \textcolor{red}{याति} इति प्रयुक्तम्। \textcolor{red}{वर्तमान\-सामीप्ये वर्तमानवद्वा} (पा॰सू॰~३.३.१३१) इत्यनेन वर्तमान\-कालः।\footnote{\textcolor{red}{या प्रापणे} (धा॰पा॰~१०४९)~\arrow या~\arrow \textcolor{red}{शेषात्कर्तरि परस्मैपदम्} (पा॰सू॰~१.३.७८)~\arrow \textcolor{red}{वर्तमान\-सामीप्ये वर्तमानवद्वा} (पा॰सू॰~३.३.१३१)~\arrow \textcolor{red}{वर्तमाने लट्} (पा॰सू॰~३.२.१२३)~\arrow या~लट्~\arrow या~तिप्~\arrow या~ति~\arrow \textcolor{red}{कर्तरि शप्‌} (पा॰सू॰~३.१.६८)~\arrow या~शप्~ति~\arrow \textcolor{red}{अदिप्रभृतिभ्यः शपः} (पा॰सू॰~२.४.७२)~\arrow या~ति~\arrow या~ति~\arrow याति।} अर्थात् \textcolor{red}{संवादमिममनुशील्य सद्य एव मम सारूप्यं प्राप्स्यति} इति भगवतस्तात्पर्यम्।\end{sloppypar}
\section[अहनत्]{अहनत्}
\centering\textcolor{blue}{भोजनं देहि मे मातर्न श्रुतं कार्यसक्तया।\nopagebreak\\
ततः क्रोधेन भाण्डानि लगुडेनाहनत्तदा॥}\nopagebreak\\
\raggedleft{–~अ॰रा॰~१.३.५३}\\
\fontsize{14}{21}\selectfont\begin{sloppypar}\hyphenrules{nohyphenation}\justifying\noindent\hspace{10mm} भगवाञ्छ्रीरामो बाल\-लीलामाचरन्मातरं भोजनं याचमानोऽप्राप्य बाल\-क्रोधं विडम्बयल्लँगुडेन भाण्डान्यचूर्णयत्। अत्र \textcolor{red}{अहनत्} इति प्रयोगः। वस्तुतस्तु \textcolor{red}{हन्‌}\-धातोः (\textcolor{red}{हनँ हिंसागत्योः} धा॰पा॰~१०१२) धातु\-सञ्ज्ञायाम् \textcolor{red}{अनद्यतने लङ्} (पा॰सू॰~३.२.१११) इत्यनेन लङ्लकारे \textcolor{red}{तिप्तस्झि\-सिप्थस्थ\-मिब्वस्मस्ताताञ्झ\-थासाथान्ध्वमिड्वहिमहिङ्} (पा॰सू॰~३.४.७८) इत्यनेन तिप्प्रत्यये \textcolor{red}{कर्तरि शप्} (पा॰सू॰~३.१.६८) इत्यनेन शब्विकरणे \textcolor{red}{अदि\-प्रभृतिभ्यः शपः} (पा॰सू॰~२.४.७२) इत्यनेन शपो लुकि \textcolor{red}{लुङ्लङ्लृङ्क्ष्वडुदात्तः} (पा॰सू॰~६.४.७१) इत्यनेनाडागमे \textcolor{red}{इतश्च} (पा॰सू॰~३.४.१००) इत्यनेनेकार\-लोपे \textcolor{red}{हल्ङ्याब्भ्यो दीर्घात्सुतिस्यपृक्तं हल्} (पा॰सू॰~६.१.६८) इत्यनेन तकार\-लोपे \textcolor{red}{अहन्} इति पाणिनीयम्।\footnote{\textcolor{red}{हनँ हिंसागत्योः} (धा॰पा॰~१०१२)~\arrow हन्~\arrow \textcolor{red}{शेषात्कर्तरि परस्मैपदम्} (पा॰सू॰~१.३.७८)~\arrow \textcolor{red}{अनद्यतने लङ्} (पा॰सू॰~३.२.१११)~\arrow हन्~लङ्~\arrow हन्~तिप्~\arrow हन्~ति~\arrow \textcolor{red}{लुङ्लङ्लृङ्क्ष्वडुदात्तः} (पा॰सू॰~६.४.७१)~\arrow \textcolor{red}{आद्यन्तौ टकितौ} (पा॰सू॰~१.१.४६)~\arrow अट्~हन्~ति~\arrow अ~हन्~ति~\arrow \textcolor{red}{कर्तरि शप्‌} (पा॰सू॰~३.१.६८)~\arrow अ~हन्~शप्~ति~\arrow \textcolor{red}{अदि\-प्रभृतिभ्यः शपः} (पा॰सू॰~२.४.७२)~\arrow अ~हन्~ति~\arrow \textcolor{red}{इतश्च} (पा॰सू॰~३.४.१००)~\arrow अ~हन्~त्~\arrow \textcolor{red}{हल्ङ्याब्भ्यो दीर्घात्सुतिस्यपृक्तं हल्} (पा॰सू॰~६.१.६८)~\arrow अ~हन्~\arrow अहन्।} \textcolor{red}{अहनत्} इत्यत्र \textcolor{red}{गण\-कार्यमनित्यम्} (प॰शे॰~९३.३) इत्यनेन शब्लुगभावे हलन्तत्वाभावान्न तकारलोपः।\footnote{\textcolor{red}{हनँ हिंसागत्योः} (धा॰पा॰~१०१२)~\arrow हन्~\arrow \textcolor{red}{शेषात्कर्तरि परस्मैपदम्} (पा॰सू॰~१.३.७८)~\arrow \textcolor{red}{अनद्यतने लङ्} (पा॰सू॰~३.२.१११)~\arrow हन्~लङ्~\arrow हन्~तिप्~\arrow हन्~ति~\arrow \textcolor{red}{लुङ्लङ्लृङ्क्ष्वडुदात्तः} (पा॰सू॰~६.४.७१)~\arrow \textcolor{red}{आद्यन्तौ टकितौ} (पा॰सू॰~१.१.४६)~\arrow अट्~हन्~ति~\arrow अ~हन्~ति~\arrow \textcolor{red}{कर्तरि शप्‌} (पा॰सू॰~३.१.६८)~\arrow अ~हन्~शप्~ति~\arrow \textcolor{red}{गण\-कार्यमनित्यम्} (प॰शे॰~९३.३)~\arrow शब्लुगभावः~\arrow अ~हन्~अ~ति~\arrow \textcolor{red}{इतश्च} (पा॰सू॰~३.४.१००)~\arrow अ~हन्~अ~त्~\arrow अहनत्।} यद्वा \textcolor{red}{हन्तीति हनः} इति विग्रहे पचादित्वादच्।\footnote{\textcolor{red}{नन्दि\-ग्रहि\-पचादिभ्यो ल्युणिन्यचः} (पा॰सू॰~३.१.१३४) इत्यनेन कर्तरि।} \textcolor{red}{हन इवाऽचरतीति हनति}\footnote{हन~\arrow \textcolor{red}{सर्वप्राति\-पदिकेभ्य आचारे क्विब्वा वक्तव्यः} (वा॰~३.१.११)~\arrow हन~क्विँप्~\arrow हन~व्~\arrow \textcolor{red}{वेरपृक्तस्य} (पा॰सू॰~६.१.६७)~\arrow हन~\arrow \textcolor{red}{सनाद्यन्ता धातवः} (पा॰सू॰~३.१.३२)~\arrow धातुसञ्ज्ञा~\arrow \textcolor{red}{शेषात्कर्तरि परस्मैपदम्} (पा॰सू॰~१.३.७८)~\arrow वर्तमाने लट्~\arrow हन~लट्~\arrow हन~तिप्~\arrow हन~ति~\arrow \textcolor{red}{कर्तरि शप्‌} (पा॰सू॰~३.१.६८)~\arrow हन~शप्~ति~\arrow हन~अ~ति~\arrow \textcolor{red}{अतो गुणे} (पा॰सू॰~६.१.९७)~\arrow हन~ति~\arrow हनति।} इति विग्रह आचार\-क्विबन्ताल्लङ्लकारे प्रथम\-पुरुष एकवचन एवम् \textcolor{red}{अहनत्}।\footnote{हन~\arrow धातुसञ्ज्ञा (पूर्ववत्)~\arrow \textcolor{red}{शेषात्कर्तरि परस्मैपदम्} (पा॰सू॰~१.३.७८)~\arrow \textcolor{red}{अनद्यतने लङ्} (पा॰सू॰~३.२.१११)~\arrow हन्~लङ्~\arrow हन्~तिप्~\arrow हन्~ति~\arrow \textcolor{red}{लुङ्लङ्लृङ्क्ष्वडुदात्तः} (पा॰सू॰~६.४.७१)~\arrow \textcolor{red}{आद्यन्तौ टकितौ} (पा॰सू॰~१.१.४६)~\arrow अट्~हन्~ति~\arrow अ~हन्~ति~\arrow \textcolor{red}{कर्तरि शप्‌} (पा॰सू॰~३.१.६८)~\arrow अ~हन~शप्~ति~\arrow अ~हन~अ~ति~\arrow \textcolor{red}{अतो गुणे} (पा॰सू॰~६.१.९७)~\arrow अ~हन~ति~\arrow \textcolor{red}{इतश्च} (पा॰सू॰~३.४.१००)~\arrow अ~हन~त्~\arrow अहनत्।} श्रीरामो निर्लेपत्वादकर्तृत्वाच्च किमपि न करोति। \textcolor{red}{रामो न गच्छति न तिष्ठति नानुशोचति} इत्यादि राम\-हृदये सीतया स्पष्टं निगदितत्वात्। यथा~–\end{sloppypar}
\centering\textcolor{blue}{रामो न गच्छति न तिष्ठति नानुशोचत्याकाङ्क्षते त्यजति नो न करोति किञ्च।\nopagebreak\\
आनन्दमूर्तिरचलः परिणामहीनो मायागुणाननुगतो हि तथा विभाति॥}\nopagebreak\\
\raggedleft{–~अ॰रा॰~१.१.५३}\\
\fontsize{14}{21}\selectfont\begin{sloppypar}\hyphenrules{nohyphenation}\justifying\noindent एवं \textcolor{red}{रामाभिन्नैक\-कर्तृकानद्यतन\-भूत\-कालावच्छिन्न\-भाण्ड\-कर्मक\-हननानुकूल\-व्यापाराश्रय\-सदृश\-व्यापारः} इति शाब्द\-बोधः।\end{sloppypar}
\section[चक्रे]{चक्रे}
\centering\textcolor{blue}{एवं परात्मा मनुजावतारो मनुष्यलोकाननुसृत्य सर्वम्।\nopagebreak\\
चक्रेऽविकारी परिणामहीनो विचार्यमाणे न करोति किञ्चित्॥}\nopagebreak\\
\raggedleft{–~अ॰रा॰~१.३.६६}\\
\fontsize{14}{21}\selectfont\begin{sloppypar}\hyphenrules{nohyphenation}\justifying\noindent\hspace{10mm} अत्र भगवतः स्वरूपं वर्णयति \textcolor{red}{अकर्ता सन् सर्वं चकार विचार्यमाणे सति किमपि नाकरोत्}। अत्र \textcolor{red}{चकार} इति हि पाणिनीयः।\footnote{\textcolor{red}{डुकृञ् करणे} (धा॰पा॰~१४७२)~\arrow कृ~\arrow \textcolor{red}{शेषात्कर्तरि परस्मैपदम्} (पा॰सू॰~१.३.७८)~\arrow \textcolor{red}{परोक्षे लिट्} (पा॰सू॰~३.२.११५)~\arrow कृ~लिट्~\arrow कृ~तिप्~\arrow कृ~ति~\arrow \textcolor{red}{परस्मैपदानां णलतुसुस्थलथुस\-णल्वमाः} (पा॰सू॰~३.४.८२)~\arrow कृ~णल्~\arrow कृ~अ~\arrow \textcolor{red}{लिटि धातोरनभ्यासस्य} (पा॰सू॰~६.१.८)~\arrow कृ~कृ~अ~\arrow \textcolor{red}{उरत्‌} (पा॰सू॰~७.४.६६)~\arrow \textcolor{red}{उरण् रपरः} (पा॰सू॰~१.१.५१)~\arrow कर्~कृ~अ~\arrow \textcolor{red}{हलादिः शेषः} (पा॰सू॰~७.४.६०)~\arrow क~कृ~अ~\arrow \textcolor{red}{कुहोश्चुः} (पा॰सू॰~७.४.६२)~\arrow च~कृ~अ~\arrow \textcolor{red}{अचो ञ्णिति} (पा॰सू॰~७.२.११५)~\arrow \textcolor{red}{उरण् रपरः} (पा॰सू॰~१.१.५१)~\arrow च~कार्~अ~\arrow चकार।} तच्चरित्रेण भक्तेष्वानन्द\-जननात्क्रिया\-फलस्य पर\-निष्ठत्वात्। किन्तु कर्म\-व्यतिहारेणात्राऽत्मने\-पदम्।\footnote{\textcolor{red}{कर्तरि कर्म\-व्यतिहारे} पा॰सू॰~१.३.१४) इत्यनेन।} साङ्ख्य\-दृष्ट्या भगवति प्रकृति\-गत\-कर्तव्यमारोपितम्। राम\-हृदये श्रीसीतया स्पष्टं प्रतिपादितत्वात्। वेदान्त\-दृष्ट्याऽपि जीव\-गत\-कर्तृत्वं भगवत्यारोपितम्। अत आत्मनेपदम्। तथा च \textcolor{red}{कृ}\-धातोः (\textcolor{red}{डुकृञ् करणे} धा॰पा॰~१४७२) लिड्लकारे \textcolor{red}{परोक्षे लिट्} (पा॰सू॰~३.२.११५) इत्यनेन \textcolor{red}{त}\-प्रत्यये \textcolor{red}{लिटस्तझयोरेशिरेच्} (पा॰सू॰~३.४.८१) इत्यनेन \textcolor{red}{एश्} आदेशे \textcolor{red}{लिटि धातोरनभ्यासस्य} (पा॰सू॰~६.१.८) इत्यनेन द्वित्वे \textcolor{red}{पूर्वोऽभ्यासः} (पा॰सू॰~६.१.४) इत्यनेनाभ्यास\-सञ्ज्ञायां \textcolor{red}{उरत्‌} (पा॰सू॰~७.४.६६) इत्यनेनर्कारस्याकारे \textcolor{red}{उरण् रपरः} (पा॰सू॰~१.१.५१) इत्यनेन रपरत्वे \textcolor{red}{हलादिः शेषः} (पा॰सू॰~७.४.०) इत्यनेन रलोपे \textcolor{red}{कुहोश्चुः} (पा॰सू॰~७.४.६२) इत्यनेन चुत्वे \textcolor{red}{इको यणचि} (पा॰सू॰~६.१.७७) इत्यनेन यणि \textcolor{red}{चक्रे}।\footnote{\textcolor{red}{डुकृञ् करणे} (धा॰पा॰~१४७२)~\arrow कृ~\arrow \textcolor{red}{कर्तरि कर्म\-व्यतिहारे} (पा॰सू॰~१.३.१४)~\arrow \textcolor{red}{परोक्षे लिट्} (पा॰सू॰~३.२.११५)~\arrow कृ~लिट्~\arrow कृ~त~\arrow \textcolor{red}{लिटस्तझयोरेशिरेच्} (पा॰सू॰~३.४.८१)~\arrow कृ~एश्~\arrow कृ~ए~\arrow \textcolor{red}{लिटि धातोरनभ्यासस्य} (पा॰सू॰~६.१.८)~\arrow कृ~कृ~ए~\arrow \textcolor{red}{उरत्‌} (पा॰सू॰~७.४.६६)~\arrow \textcolor{red}{उरण् रपरः} (पा॰सू॰~१.१.५१)~\arrow कर्~कृ~ए~\arrow \textcolor{red}{हलादिः शेषः} (पा॰सू॰~७.४.६०)~\arrow क~कृ~ए~\arrow \textcolor{red}{कुहोश्चुः} (पा॰सू॰~७.४.६२)~\arrow च~कृ~ए~\arrow \textcolor{red}{असंयोगाल्लिट् कित्} (पा॰सू॰~१.२.५)~\arrow कित्त्वम्~\arrow \textcolor{red}{ग्क्ङिति च} (पा॰सू॰~१.१.५)~\arrow गुणनिषेधः~\arrow \textcolor{red}{इको यणचि} (पा॰सू॰~६.१.७७)~\arrow च~क्र्~ए~\arrow चक्रे।} एवं \textcolor{red}{करोति} इत्यपि भूत\-काले प्रयोक्तव्ये वर्तमाने प्रयुक्तं रूपम्। अत्र \textcolor{red}{लट् स्मे} (पा॰सू॰~३.२.११८) इत्यनेन स्म\-योगे लड्लकारो भूतकालेऽपि।\footnote{\textcolor{red}{डुकृञ् करणे} (धा॰पा॰~१४७२)~\arrow कृ~\arrow \textcolor{red}{शेषात्कर्तरि परस्मैपदम्} (पा॰सू॰~१.३.७८)~\arrow \textcolor{red}{लट् स्मे} (पा॰सू॰~३.२.११८)~\arrow कृ~लट्~\arrow कृ~तिप्~\arrow कृ~ति~\arrow \textcolor{red}{तनादि\-कृञ्भ्य उः} (पा॰सू॰~३.१.७९)~\arrow कृ~उ~ति~\arrow \textcolor{red}{सार्वधातुकार्ध\-धातुकयोः} (पा॰सू॰~७.३.८४)~\arrow \textcolor{red}{उरण् रपरः} (पा॰सू॰~१.१.५१)~\arrow कर्~उ~ति~\arrow \textcolor{red}{सार्वधातुकार्ध\-धातुकयोः} (पा॰सू॰~७.३.८४)~\arrow कर्~ओ~ति~\arrow करोति।} न चाऽत्र \textcolor{red}{स्म} इति न दृश्यते। \textcolor{red}{विनाऽपि प्रत्ययं पूर्वोत्तर\-पद\-लोपो वक्तव्यः} (वा॰~५.३.८३) इत्यनेन लोपात्।\end{sloppypar}
\section[करोमि]{करोमि}
\centering\textcolor{blue}{त्वद्विधा यद्गृहं यान्ति तत्रैवाऽयान्ति सम्पदः।\nopagebreak\\
यदर्थमागतोऽसि त्वं ब्रूहि सत्यं करोमि तत्॥}\nopagebreak\\
\raggedleft{–~अ॰रा॰~१.४.४}\\
\fontsize{14}{21}\selectfont\begin{sloppypar}\hyphenrules{nohyphenation}\justifying\noindent\hspace{10mm} अत्र \textcolor{red}{करिष्यामि}\footnote{\textcolor{red}{डुकृञ् करणे} (धा॰पा॰~१४७२)~\arrow कृ~\arrow \textcolor{red}{शेषात्कर्तरि परस्मैपदम्} (पा॰सू॰~१.३.७८)~\arrow \textcolor{red}{लृट् शेषे च} (पा॰सू॰~३.३.१३)~\arrow कृ~लृट्~\arrow कृ~मिप्~\arrow कृ~मि~\arrow \textcolor{red}{स्यतासी लृलुटोः} (पा॰सू॰~३.१.३३)~\arrow कृ~स्य~मि~\arrow \textcolor{red}{ऋद्धनोः स्ये} (पा॰सू॰~७.२.७०)~\arrow कृ~इट्~स्य~मि~\arrow कृ~इ~स्य~मि~\arrow \textcolor{red}{सार्वधातुकार्ध\-धातुकयोः} (पा॰सू॰~७.३.८४)~\arrow \textcolor{red}{उरण् रपरः} (पा॰सू॰~१.१.५१)~\arrow कर्~इ~स्य~मि~\arrow \textcolor{red}{अतो दीर्घो यञि} (पा॰सू॰~७.३.१०१)~\arrow कर्~इ~स्या~मि~\arrow \textcolor{red}{आदेश\-प्रत्यययोः} (पा॰सू॰~८.३.५९)~\arrow कर्~इ~ष्या~मि~\arrow करिष्यामि।} इति प्रयोक्तव्ये वर्तमान\-सामीप्य\-लकारात् \textcolor{red}{करोमि}।\footnote{\textcolor{red}{वर्तमान\-सामीप्ये वर्तमानवद्वा} (पा॰सू॰~३.३.१३१) इत्यनेन। \textcolor{red}{डुकृञ् करणे} (धा॰पा॰~१४७२)~\arrow कृ~\arrow \textcolor{red}{शेषात्कर्तरि परस्मैपदम्} (पा॰सू॰~१.३.७८)~\arrow \textcolor{red}{वर्तमान\-सामीप्ये वर्तमानवद्वा} (पा॰सू॰~३.३.१३१)~\arrow \textcolor{red}{वर्तमाने लट्} (पा॰सू॰~३.२.१२३)~\arrow कृ~लट्~\arrow कृ~मिप्~\arrow कृ~मि~\arrow \textcolor{red}{तनादि\-कृञ्भ्य उः} (पा॰सू॰~३.१.७९)~\arrow कृ~उ~मि~\arrow \textcolor{red}{सार्वधातुकार्ध\-धातुकयोः} (पा॰सू॰~७.३.८४)~\arrow \textcolor{red}{उरण् रपरः} (पा॰सू॰~१.१.५१)~\arrow कर्~उ~मि~\arrow \textcolor{red}{सार्वधातुकार्ध\-धातुकयोः} (पा॰सू॰~७.३.८४)~\arrow कर्~ओ~मि~\arrow करोमि।}\end{sloppypar}
\section[किं करोमि]{किं करोमि}
\centering\textcolor{blue}{किं करोमि गुरो रामं त्यक्तुं नोत्सहते मनः।\nopagebreak\\
बहुवर्षसहस्रान्ते कष्टेनोत्पादिताः सुताः॥}\nopagebreak\\
\raggedleft{–~अ॰रा॰~१.४.९}\\
\fontsize{14}{21}\selectfont\begin{sloppypar}\hyphenrules{nohyphenation}\justifying\noindent\hspace{10mm} अत्र \textcolor{red}{किंवृत्ते लिप्सायाम्} (पा॰सू॰~३.३.६) इत्यनेन लट्।\footnote{\textcolor{red}{डुकृञ् करणे} (धा॰पा॰~१४७२)~\arrow कृ~\arrow \textcolor{red}{शेषात्कर्तरि परस्मैपदम्} (पा॰सू॰~१.३.७८)~\arrow \textcolor{red}{किंवृत्ते लिप्सायाम्} (पा॰सू॰~३.३.६)~\arrow कृ~लट्~\arrow कृ~मिप्~\arrow कृ~मि~\arrow \textcolor{red}{तनादि\-कृञ्भ्य उः} (पा॰सू॰~३.१.७९)~\arrow कृ~उ~मि~\arrow \textcolor{red}{सार्वधातुकार्ध\-धातुकयोः} (पा॰सू॰~७.३.८४)~\arrow \textcolor{red}{उरण् रपरः} (पा॰सू॰~१.१.५१)~\arrow कर्~उ~मि~\arrow \textcolor{red}{सार्वधातुकार्ध\-धातुकयोः} (पा॰सू॰~७.३.८४)~\arrow कर्~ओ~मि~\arrow करोमि।}\end{sloppypar}
\section[न जीवामि]{न जीवामि}
\centering\textcolor{blue}{चत्वारोऽमरतुल्यास्ते तेषां रामोऽतिवल्लभः।\nopagebreak\\
रामस्त्वितो गच्छति चेन्न जीवामि कथञ्चन॥}\nopagebreak\\
\raggedleft{–~अ॰रा॰~१.४.१०}\\
\fontsize{14}{21}\selectfont\begin{sloppypar}\hyphenrules{nohyphenation}\justifying\noindent\hspace{10mm} \textcolor{red}{न जीवामि} इति वर्तमान\-सामीप्याल्लट्।\footnote{\textcolor{red}{वर्तमान\-सामीप्ये वर्तमानवद्वा} (पा॰सू॰~३.३.१३१) इत्यनेन। \textcolor{red}{जीवँ प्राणधारणे} (धा॰पा॰~५६२)~\arrow जीव्~\arrow \textcolor{red}{शेषात्कर्तरि परस्मैपदम्} (पा॰सू॰~१.३.७८)~\arrow \textcolor{red}{वर्तमान\-सामीप्ये वर्तमानवद्वा} (पा॰सू॰~३.३.१३१)~\arrow \textcolor{red}{वर्तमाने लट्} (पा॰सू॰~३.२.१२३)~\arrow जीव्~लट्~\arrow जीव्~मिप्~\arrow जीव्~मि~\arrow \textcolor{red}{कर्तरि शप्‌} (पा॰सू॰~३.१.६८)~\arrow जीव्~शप्~मि~\arrow जीव्~अ~मि~\arrow \textcolor{red}{अतो दीर्घो यञि} (पा॰सू॰~७.३.१०१)~\arrow जीव्~आ~मि~\arrow जीवामि।} \textcolor{red}{सद्यो मरिष्यामि} इति दशरथस्य तात्पर्यम्।\end{sloppypar}
\section[तेपाथे]{तेपाथे}
\centering\textcolor{blue}{त्वं तु प्रजापतिः पूर्वं कश्यपो ब्रह्मणः सुतः।\nopagebreak\\
कौसल्या चादितिर्देवमाता पूर्वं यशस्विनी।\nopagebreak\\
भवन्तौ तप उग्रं वै तेपाथे बहुवत्सरम्॥}\nopagebreak\\
\raggedleft{–~अ॰रा॰~१.४.१४}\\
\fontsize{14}{21}\selectfont\begin{sloppypar}\hyphenrules{nohyphenation}\justifying\noindent\hspace{10mm} अत्र वसिष्ठो दशरथस्य पूर्व\-जन्म स्मारयति। अत्र \textcolor{red}{तप्‌}\-धातुः (\textcolor{red}{तपँ दाहे ऐश्वर्ये वा} धा॰पा॰~११५९) दिवादिरात्मनेपदीयो न तु भ्वादिः परस्मैपदी (\textcolor{red}{तपँ सन्तापे} धा॰पा॰~९८५)।\footnote{भौवादिकाद्धातोस्तु \textcolor{red}{तेपथुः} इति रूपम्। \textcolor{red}{तपँ सन्तापे} (धा॰पा॰~९८५)~\arrow तप्~\arrow \textcolor{red}{शेषात्कर्तरि परस्मैपदम्} (पा॰सू॰~१.३.७८)~\arrow \textcolor{red}{परोक्षे लिट्} (पा॰सू॰~३.२.११५)~\arrow तप्~लिट्~\arrow तप्~थस्~\arrow \textcolor{red}{परस्मैपदानां णलतुसुस्थलथुस\-णल्वमाः} (पा॰सू॰~३.४.८२)~\arrow तप्~अथुस्~\arrow \textcolor{red}{लिटि धातोरनभ्यासस्य} (पा॰सू॰~६.१.८)~\arrow तप्~तप्~अथुस्~\arrow \textcolor{red}{हलादिः शेषः} (पा॰सू॰~७.४.६०)~\arrow त~तप्~अथुस्~\arrow \textcolor{red}{अत एकहल्मध्येऽनादेशादेर्लिटि} (पा॰सू॰~६.४.१२०)~\arrow तेप्~अथुस्~\arrow तेपथुः।} तस्माल्लिटि \textcolor{red}{आथाम्} प्रत्यये द्वित्वेऽभ्यास\-कार्ये \textcolor{red}{अत एकहल्मध्येऽनादेशादेर्लिटि} (पा॰सू॰~६.४.१२०) इत्यनेनाभ्यास\-लोप एकार एत्वे च \textcolor{red}{तेपाथे} इति साधु।\footnote{\textcolor{red}{तपँ दाहे ऐश्वर्ये वा} (धा॰पा॰~११५९)~\arrow तप्~\arrow \textcolor{red}{अनुदात्तङित आत्मने\-पदम्} (पा॰सू॰~१.३.१२)~\arrow \textcolor{red}{परोक्षे लिट्} (पा॰सू॰~३.२.११५)~\arrow तप्~लिट्~\arrow तप्~आथाम्~\arrow \textcolor{red}{लिटि धातोरनभ्यासस्य} (पा॰सू॰~६.१.८)~\arrow तप्~तप्~आथाम्~\arrow \textcolor{red}{हलादिः शेषः} (पा॰सू॰~७.४.६०)~\arrow त~तप्~आथाम्~\arrow \textcolor{red}{अत एकहल्मध्येऽनादेशादेर्लिटि} (पा॰सू॰~६.४.१२०)~\arrow तेप्~आथाम्~\arrow \textcolor{red}{टित आत्मनेपदानां टेरे} (पा॰सू॰~३.४.७९)~\arrow तेप्~आथे~\arrow तेपाथे। न चात्र \textcolor{red}{तप्‌}\-धातुश्चुरादिराधृषीय उभयपदी (\textcolor{red}{तपँ दाहे} धा॰पा॰~१८१९)। तस्माल्लिटि परस्मैपदे णिच्पक्षे \textcolor{red}{तापयाञ्चक्रथुः तापयाम्बभूवथुः तापयामासथुः} इति रूपाणि णिजभाव\-पक्षे \textcolor{red}{तेपथुः} इति रूपमात्मने\-पदे च णिच्पक्षे \textcolor{red}{तापयाञ्चक्राथे} इति रूपं णिजभाव\-पक्षे \textcolor{red}{तेपाथे} इति रूपम्। अत्र क्रियाफलं रामावतरणम्। तन्नादिति\-कश्यप\-कर्तृगाम्यपि तु परगामि। सकल\-संसार\-कल्याणकारित्वात्। रामावतारेण साधूनां परित्राणं दुष्कृतानां विनाशो धर्मस्य संस्थापना च। तेन \textcolor{red}{स्वरितञितः कर्त्रभिप्राये क्रियाफले} (पा॰सू॰~१.३.७२) इत्यस्याप्रवृत्तिरत्र। विस्तार\-भिया प्रक्रिया न दीयन्ते।}\end{sloppypar}
\section[प्रेषयस्व]{प्रेषयस्व}
\centering\textcolor{blue}{अतः प्रीतेन मनसा पूजयित्वाथ कौशिकम्।\nopagebreak\\
प्रेषयस्व रमानाथं राघवं सहलक्ष्मणम्॥}\nopagebreak\\
\raggedleft{–~अ॰रा॰~१.४.२०}\\
\fontsize{14}{21}\selectfont\begin{sloppypar}\hyphenrules{nohyphenation}\justifying\noindent\hspace{10mm} वसिष्ठोऽनुजानाति यत् \textcolor{red}{रमा\-नाथं श्रीराम\-भद्रं विश्वामित्राय प्रयच्छ}। अत्र \textcolor{red}{प्रेषयस्व} इति प्रयुक्तं \textcolor{red}{प्रेषय} इति प्रयोक्तव्यम्।\footnote{\textcolor{red}{प्रेषृँ गतौ} (धा॰पा॰~६१९)~\arrow प्रेष्~\arrow \textcolor{red}{हेतुमति च} (पा॰सू॰~३.१.२६)~\arrow प्रेष्~णिच्~\arrow प्रेष्~इ~\arrow प्रेषि~\arrow \textcolor{red}{सनाद्यन्ता धातवः} (पा॰सू॰~३.१.३२)~\arrow धातुसञ्ज्ञा~\arrow \textcolor{red}{शेषात्कर्तरि परस्मैपदम्} (पा॰सू॰~१.३.७८)~\arrow \textcolor{red}{लोट् च} (पा॰सू॰~३.३.१६२)~\arrow प्रेषि~लोट्~\arrow प्रेषि~सिप्~\arrow प्रेषि~सि~\arrow \textcolor{red}{कर्तरि शप्} (पा॰सू॰~३.१.६८)~\arrow प्रेषि~शप्~सि~\arrow प्रेषि~अ~सि~\arrow \textcolor{red}{सार्वधातुकार्धधातुकयोः} (पा॰सू॰~७.३.८४)~\arrow प्रेषे~अ~सि~\arrow \textcolor{red}{एचोऽयवायावः} (पा॰सू॰~६.१.७८)~\arrow प्रेषय्~अ~सि~\arrow \textcolor{red}{सेर्ह्यपिच्च} (पा॰सू॰~३.४.८७)~\arrow प्रेषय्~अ~हि~\arrow \textcolor{red}{अतो हेः} (पा॰सू॰~६.४.१०५)~\arrow प्रेषय्~अ~\arrow प्रेषय।} यतो हि \textcolor{red}{णिचश्च} (पा॰सू॰~१.३.७४) इत्यात्मनेपदम्।\footnote{\textcolor{red}{प्रेषृँ गतौ} (धा॰पा॰~६१९)~\arrow प्रेष्~\arrow \textcolor{red}{हेतुमति च} (पा॰सू॰~३.१.२६)~\arrow प्रेष्~णिच्~\arrow प्रेष्~इ~\arrow प्रेषि~\arrow धातुसञ्ज्ञा~\arrow \textcolor{red}{णिचश्च} (पा॰सू॰~१.३.७४)~\arrow \textcolor{red}{लोट् च} (पा॰सू॰~३.३.१६२)~\arrow प्रेषि~लोट्~\arrow प्रेषि~थास्~\arrow \textcolor{red}{कर्तरि शप्} (पा॰सू॰~३.१.६८)~\arrow प्रेषि~शप्~थास्~\arrow प्रेषि~अ~थास्~\arrow \textcolor{red}{सार्वधातुकार्ध\-धातुकयोः} (पा॰सू॰~७.३.८४)~\arrow प्रेषे~अ~थास्~\arrow \textcolor{red}{एचोऽयवायावः} (पा॰सू॰~६.१.७८)~\arrow प्रेषय्~अ~थास्~\arrow \textcolor{red}{थासस्से} (पा॰सू॰~३.४.८०)~\arrow प्रेषय्~अ~से~\arrow \textcolor{red}{सवाभ्यां वामौ} (पा॰सू॰~३.४.९१)~\arrow प्रेषय्~अ~स्व~\arrow प्रेषयस्व।} तदपि क्रिया\-फले कर्तृ\-गामिनि सति। \textcolor{red}{राघव\-प्रेषणेन त्वामपि यशो\-रूपं फलं मिलिष्यति} इति तात्पर्यादत्राऽत्मनेपदम्। यद्वा \textcolor{red}{स्व} इति सम्बोधनम्। \textcolor{red}{हे स्व आत्मीय रमा\-नाथं रामं प्रेषय} इति नापाणिनीयता।\end{sloppypar}
\section[दर्शयस्व]{दर्शयस्व}
\centering\textcolor{blue}{दर्शयस्व महाभाग कुतस्तौ राक्षसाधमौ।\nopagebreak\\
तथेत्युक्त्वा मुनिर्यष्टुमारेभे मुनिभिः सह॥}\nopagebreak\\
\raggedleft{–~अ॰रा॰~१.५.४}\\
\fontsize{14}{21}\selectfont\begin{sloppypar}\hyphenrules{nohyphenation}\justifying\noindent\hspace{10mm} श्रीरामो राक्षसौ प्रति पृच्छन् \textcolor{red}{दर्शयस्व} इति प्रयुङ्क्ते। अत्र राक्षस\-निधनेन दर्शन\-कारयितरि विश्वामित्रे फलमिति क्रिया\-फलस्य कर्तृ\-गामित्वादात्मने\-पदम्।\footnote{\textcolor{red}{णिचश्च} (पा॰सू॰~१.३.७४) इत्यनेन। \textcolor{red}{दृशिँर् प्रेक्षणे} (धा॰पा॰~९८८)~\arrow दृश्~\arrow \textcolor{red}{हेतुमति च} (पा॰सू॰~३.१.२६)~\arrow दृश्~णिच्~\arrow दृश्~इ~\arrow \textcolor{red}{पुगन्त\-लघूपधस्य च} (पा॰सू॰~७.३.८६)~\arrow \textcolor{red}{उरण् रपरः} (पा॰सू॰~१.१.५१)~\arrow दर्श्~इ~\arrow दर्शि~\arrow \textcolor{red}{सनाद्यन्ता धातवः} (पा॰सू॰~३.१.३२)~\arrow धातुसञ्ज्ञा~\arrow \textcolor{red}{णिचश्च} (पा॰सू॰~१.३.७४)~\arrow \textcolor{red}{लोट् च} (पा॰सू॰~३.३.१६२)~\arrow दर्शि~लोट्~\arrow दर्शि~थास्~\arrow \textcolor{red}{कर्तरि शप्} (पा॰सू॰~३.१.६८)~\arrow दर्शि~शप्~थास्~\arrow दर्शि~अ~थास्~\arrow \textcolor{red}{सार्वधातुकार्धधातुकयोः} (पा॰सू॰~७.३.८४)~\arrow दर्शे~अ~थास्~\arrow \textcolor{red}{एचोऽयवायावः} (पा॰सू॰~६.१.७८)~\arrow दर्शय्~अ~थास्~\arrow \textcolor{red}{थासस्से} (पा॰सू॰~३.४.८०)~\arrow दर्शय्~अ~से~\arrow \textcolor{red}{सवाभ्यां वामौ} (पा॰सू॰~३.४.९१)~\arrow दर्शय्~अ~स्व~\arrow दर्शयस्व।}\end{sloppypar}
\section[ददृशाते]{ददृशाते}
\centering\textcolor{blue}{मध्याह्ने ददृशाते तौ राक्षसौ कामरूपिणौ।\nopagebreak\\
मारीचश्च सुबाहुश्च वर्षन्तौ रुधिरास्थिनी॥}\nopagebreak\\
\raggedleft{–~अ॰रा॰~१.५.५}\\
\fontsize{14}{21}\selectfont\begin{sloppypar}\hyphenrules{nohyphenation}\justifying\noindent\hspace{10mm} \textcolor{red}{ददृशाते} इति कर्म\-वाच्य\-प्रयोगः।\footnote{अत्र \textcolor{red}{कर्तरि कर्म\-व्यतिहारे} (पा॰सू॰~१.३.१४) इत्यनेन कर्तर्यात्मनेपदं न वेत्याशङ्कां परिहर्तुं विमर्शः प्रारब्धः। \textcolor{red}{दृशिँर् प्रेक्षणे} (धा॰पा॰~९८८)~\arrow दृश्~\arrow \textcolor{red}{भावकर्मणोः} (पा॰सू॰~१.३.१३)~\arrow \textcolor{red}{परोक्षे लिट्} (पा॰सू॰~३.२.११५)~\arrow दृश्~लिट्~\arrow दृश्~आताम्~\arrow \textcolor{red}{लिटि धातोरनभ्यासस्य} (पा॰सू॰~६.१.८)~\arrow दृश्~दृश्~आताम्~\arrow \textcolor{red}{उरत्‌} (पा॰सू॰~७.४.६६)~\arrow \textcolor{red}{उरण् रपरः} (पा॰सू॰~१.१.५१)~\arrow दर्श्~दृश्~आताम्~\arrow \textcolor{red}{हलादिः शेषः} (पा॰सू॰~७.४.६०)~\arrow द~दृश्~आताम्~\arrow \textcolor{red}{टित आत्मनेपदानां टेरे} (पा॰सू॰~३.४.७९)~\arrow द~दृश्~आते~\arrow \textcolor{red}{असंयोगाल्लिट् कित्} (पा॰सू॰~१.२.५)~\arrow कित्त्वम्~\arrow \textcolor{red}{ग्क्ङिति च} (पा॰सू॰~१.१.५)~\arrow गुणनिषेधः~\arrow ददृशाते।} \textcolor{red}{राम\-लक्ष्मणाभ्याम्} इति शेषः। अत्र \textcolor{red}{परोक्षे लिट्} (पा॰सू॰~३.२.११५) इत्यनेन लिड्लकारः। प्रथम\-पुरुष\-द्वि\-वचने रूपम्। परोक्षत्वं नाम \textcolor{red}{साक्षात्करोमीति क्रियाशालि\-ज्ञानाविषयतावच्छेदकत्वम्}।\footnote{\textcolor{red}{परोक्षत्वं च साक्षात्करोमीत्येतादृश\-विषयता\-शालि\-ज्ञानाविषयत्वम्} (वै॰भू॰सा॰~२२)।}\end{sloppypar}
\section[गच्छामहे]{गच्छामहे}
\centering\textcolor{blue}{चतुर्थेऽहनि संप्राप्ते कौशिको राममब्रवीत्।\nopagebreak\\
राम राम महायज्ञं द्रष्टुं गच्छामहे वयम्॥}\nopagebreak\\
\raggedleft{–~अ॰रा॰~१.५.१२}\\
\fontsize{14}{21}\selectfont\begin{sloppypar}\hyphenrules{nohyphenation}\justifying\noindent\hspace{10mm} ताटका\-मारीच\-सुबाहून् व्यापाद्य यज्ञ\-रक्षां विधाय दिन\-त्रयं यावत्तत्र प्रोष्यायोध्यां प्रति गन्तुकामं लक्ष्मणाभिरामं श्रीरामं विश्वामित्रो मिथिलां प्रस्थातुं समामन्त्रयत्। तत्रैव प्रयुक्तम् \textcolor{red}{गच्छामहे}। इदं कथम्। \textcolor{red}{गमॢँ}\-धातुः (\textcolor{red}{गमॢँ गतौ} धा॰पा॰~९८२) परस्मैपदी। ततो लटि मसि शपि \textcolor{red}{अतो दीर्घो यञि} (पा॰सू॰~७.३.१०१) इत्यनेन दीर्घे \textcolor{red}{इषुगमियमां छः} (पा॰सू॰~७.३.७७) इत्यनेन छत्वे \textcolor{red}{छे च} (पा॰सू॰~६.१.७३) इत्यनेन तुकि \textcolor{red}{स्तोः श्चुना श्चुः} (पा॰सू॰~८.४.४०) इत्यनेन श्चुत्वे विसर्गे \textcolor{red}{गच्छामः} इति पाणिनीयम्।\footnote{\textcolor{red}{गमॢँ गतौ} (धा॰पा॰~९८२)~\arrow गम्~\arrow \textcolor{red}{शेषात्कर्तरि परस्मैपदम्} (पा॰सू॰~१.३.७८)~\arrow \textcolor{red}{वर्तमाने लट्} (पा॰सू॰~३.२.१२३)~\arrow गम्~लट्~\arrow गम्~मस्~\arrow \textcolor{red}{कर्तरि शप्‌} (पा॰सू॰~३.१.६८)~\arrow गम्~शप्~मस्~\arrow गम्~अ~मस्~\arrow \textcolor{red}{इषुगमियमां छः} (पा॰सू॰~७.३.७७)~\arrow गछ्~अ~मस्~\arrow \textcolor{red}{छे च} (पा॰सू॰~६.१.७३)~\arrow \textcolor{red}{आद्यन्तौ टकितौ} (पा॰सू॰~१.१.४६)~\arrow गतुँक्~छ्~अ~मस्~\arrow गत्~छ्~अ~मस्~\arrow \textcolor{red}{स्तोः श्चुना श्चुः} (पा॰सू॰~८.४.४०)~\arrow गच्~छ्~अ~मस्~\arrow \textcolor{red}{अतो दीर्घो यञि} (पा॰सू॰~७.३.१०१)~\arrow गच्~छ्~आ~मस्~\arrow \textcolor{red}{ससजुषो रुः} (पा॰सू॰~८.२.६६)~\arrow गच्~छ्~आ~मरुँ~\arrow \textcolor{red}{खरवसानयोर्विसर्जनीयः} (पा॰सू॰~८.३.१५)~\arrow गच्~छ्~आ~मः~\arrow गच्छामः।} \textcolor{red}{गच्छामहे} इति \textcolor{red}{सम्} उपसर्ग\-संयोजने \textcolor{red}{समो गम्यृच्छिभ्याम्} (पा॰सू॰~१.३.२९) इत्यात्मने\-पदे \textcolor{red}{महिङ्} प्रत्यये ङकारानुबन्ध\-कार्ये \textcolor{red}{टित आत्मनेपदानां टेरे} (पा॰सू॰~३.४.७९) इत्यनेन चैत्वे \textcolor{red}{सङ्गच्छामहे}।\footnote{सम्~\textcolor{red}{गमॢँ गतौ} (धा॰पा॰~९८२)~\arrow सम्~गम्~\arrow \textcolor{red}{समो गम्यृच्छिभ्याम्} (पा॰सू॰~१.३.२९)~\arrow \textcolor{red}{वर्तमान\-सामीप्ये वर्तमानवद्वा} (पा॰सू॰~३.३.१३१)~\arrow \textcolor{red}{वर्तमाने लट्} (पा॰सू॰~३.२.१२३)~\arrow सम्~गम्~लट्~\arrow सम्~गम्~महिङ्~\arrow सम्~गम्~महि~\arrow \textcolor{red}{कर्तरि शप्‌} (पा॰सू॰~३.१.६८)~\arrow सम्~गम्~शप्~महि~\arrow सम्~गम्~अ~महि~\arrow \textcolor{red}{इषुगमियमां छः} (पा॰सू॰~७.३.७७)~\arrow सम्~गछ्~अ~महि~\arrow \textcolor{red}{छे च} (पा॰सू॰~६.१.७३)~\arrow \textcolor{red}{आद्यन्तौ टकितौ} (पा॰सू॰~१.१.४६)~\arrow सम्~गतुँक्~छ्~अ~महि~\arrow सम्~गत्~छ्~अ~महि~\arrow \textcolor{red}{स्तोः श्चुना श्चुः} (पा॰सू॰~८.४.४०)~\arrow सम्~गच्~छ्~अ~महि~\arrow \textcolor{red}{अतो दीर्घो यञि} (पा॰सू॰~७.३.१०१)~\arrow सम्~गच्~छ्~आ~महि~\arrow \textcolor{red}{टित आत्मनेपदानां टेरे} (पा॰सू॰~३.४.७९)~\arrow सम्~गच्~छ्~आ~महे~\arrow \textcolor{red}{मोऽनुस्वारः} (पा॰सू॰~८.३.२३)~\arrow सं~गच्~छ्~आ~महे~\arrow \textcolor{red}{वा पदान्तस्य} (पा॰सू॰~८.४.५९)~\arrow सङ्~गच्~छ्~आ~महे~\arrow सङ्गच्छामहे।} \textcolor{red}{यज्ञं द्रष्टुं सङ्गता भवामः}। वर्तमान\-सामीप्याल्लट्।\footnote{\textcolor{red}{वर्तमान\-सामीप्ये वर्तमानवद्वा} (पा॰सू॰~३.३.१३१) इत्यनेन।} उपसर्ग\-लोपस्तु \textcolor{red}{विनाऽपि प्रत्ययं पूर्वोत्तर\-पद\-लोपो वक्तव्यः} (वा॰~५.३.८३) इति वचनात्। यद्वा \textcolor{red}{हे} इति पृथक्पदम्। \textcolor{red}{गच्छाम} पदं लोड्लकारोत्तम\-पुरुषस्य बहुवचनस्य।\footnote{\textcolor{red}{गमॢँ गतौ} (धा॰पा॰~९८२)~\arrow गम्~\arrow \textcolor{red}{शेषात्कर्तरि परस्मैपदम्} (पा॰सू॰~१.३.७८)~\arrow \textcolor{red}{इच्छार्थेषु लिङ्लोटौ} (पा॰सू॰~३.३.१५७)~\arrow गम्~लोट्~\arrow गम्~मस्~\arrow \textcolor{red}{कर्तरि शप्‌} (पा॰सू॰~३.१.६८)~\arrow गम्~शप्~मस्~\arrow गम्~अ~मस्~\arrow \textcolor{red}{इषुगमियमां छः} (पा॰सू॰~७.३.७७)~\arrow गछ्~अ~मस्~\arrow \textcolor{red}{छे च} (पा॰सू॰~६.१.७३)~\arrow \textcolor{red}{आद्यन्तौ टकितौ} (पा॰सू॰~१.१.४६)~\arrow गतुँक्~छ्~अ~मस्~\arrow गत्~छ्~अ~मस्~\arrow \textcolor{red}{स्तोः श्चुना श्चुः} (पा॰सू॰~८.४.४०)~\arrow गच्~छ्~अ~मस्~\arrow \textcolor{red}{आडुत्तमस्य पिच्च} (पा॰सू॰~३.४.९२)~\arrow गच्~छ्~अ~आट्~मस्~\arrow \textcolor{red}{अकः सवर्णे दीर्घः} (पा॰सू॰~६.१.१०१)~\arrow गच्~छ्~आ~मस्~\arrow \textcolor{red}{लोटो लङ्वत्‌} (पा॰सू॰~३.४.८५)~\arrow ङिद्वत्त्वम्~\arrow \textcolor{red}{नित्यं ङितः} (पा॰सू॰~३.४.९९)~\arrow गच्~छ्~आ~म~\arrow गच्छाम।} \textcolor{red}{हे राम हे राम महायज्ञं द्रष्टुं वयं गच्छामेतीच्छामः} इत्थमन्वये \textcolor{red}{इच्छार्थेषु लिङ्लोटौ} (पा॰सू॰~३.३.१५७) इत्यनेन \textcolor{red}{इच्छामः} इत्युपपदे लोड्लकारे कृते मस्प्रत्यये \textcolor{red}{आडुत्तमस्य पिच्च} (पा॰सू॰~३.४.९२) इत्यनेनाऽडागमे \textcolor{red}{लोटो लङ्वत्} (पा॰सू॰~३.४.८५) इत्यनेन लङ्वद्भावे \textcolor{red}{नित्यं ङितः} (पा॰सू॰~३.४.९९) इत्यनेन सकार\-लोपे \textcolor{red}{गच्छाम हे}। \textcolor{red}{हे राम, इच्छामो यन्महायज्ञं द्रष्टुं गच्छाम} इति योजना।\footnote{\textcolor{red}{इच्छामः} इत्यध्याहार्यमिति भावः। ततः \textcolor{red}{इच्छार्थेषु लिङ्लोटौ} (पा॰सू॰~३.३.१५७) इत्यनेन सर्वलकारापवादेन लोट्।}\end{sloppypar}
\section[पूज्यसे]{पूज्यसे}
\centering\textcolor{blue}{द्रक्ष्यसि त्वं महासत्त्वं पूज्यसे जनकेन च।\nopagebreak\\
इत्युक्त्वा मुनिभिस्ताभ्यां ययौ गङ्गासमीपगम्॥}\nopagebreak\\
\raggedleft{–~अ॰रा॰~१.५.१४}\\
\fontsize{14}{21}\selectfont\begin{sloppypar}\hyphenrules{nohyphenation}\justifying\noindent\hspace{10mm} \textcolor{red}{द्रक्ष्यसि} इति समभिव्याहारेण \textcolor{red}{पूजयिष्यसे}\footnote{\textcolor{red}{पूजँ पूजायाम्} (धा॰पा॰~१६४२)~\arrow पूज्~\arrow \textcolor{red}{सत्याप\-पाश\-रूप\-वीणा\-तूल\-श्लोक\-सेना\-लोम\-त्वच\-वर्म\-वर्ण\-चूर्ण\-चुरादिभ्यो णिच्} (पा॰सू॰~३.१.२५)~\arrow पूज्~णिच्~\arrow पूज्~इ~\arrow पूजि~\arrow \textcolor{red}{सनाद्यन्ता धातवः} (पा॰सू॰~३.१.३२)~\arrow धातु\-सञ्ज्ञा~\arrow \textcolor{red}{भावकर्मणोः} (पा॰सू॰~१.३.१३)~\arrow \textcolor{red}{लृट् शेषे च} (पा॰सू॰~३.३.१३)~\arrow पूजि~लृट्~\arrow पूजि~थास्~\arrow \textcolor{red}{स्यतासी लृलुटोः} (पा॰सू॰~३.१.३३)~\arrow पूजि~स्य~थास्~\arrow \textcolor{red}{आर्धधातुकस्येड्वलादेः} (पा॰सू॰~७.२.३५)~\arrow पूजि~इट्~स्य~थास्~\arrow पूजि~इ~स्य~थास्~\arrow \textcolor{red}{सार्वधातुकार्धधातुकयोः} (पा॰सू॰~७.३.८४)~\arrow पूजे~इ~स्य~थास्~\arrow \textcolor{red}{एचोऽयवायावः} (पा॰सू॰~६.१.७८)~\arrow पूजय्~इ~स्य~थास्~\arrow \textcolor{red}{थासस्से} (पा॰सू॰~३.४.८०)~\arrow पूजय्~इ~स्य~से~\arrow \textcolor{red}{आदेश\-प्रत्यययोः} (पा॰सू॰~८.३.५९)~\arrow पूजय्~इ~ष्य~से~\arrow पूजयिष्यसे।} इति हि पाणिनीय\-प्रयोगः स्याच्चेत्सङ्गतिः स्यात्। किन्तु \textcolor{red}{पूज्यसे} इति प्रयोगे त्वसङ्गतिरेकत्र \textcolor{red}{द्रक्ष्यसि} इति भविष्यत्प्रयोगोऽपरत्र \textcolor{red}{पूज्यसे} इति वर्तमान\-कालिक इति चेत्। वर्तमान\-सामीप्याद्भविष्यत्काले लटि नासङ्गतिः।\footnote{\textcolor{red}{वर्तमान\-सामीप्ये वर्तमानवद्वा} (पा॰सू॰~३.३.१३१) इत्यनेन। पूजि~\arrow धातु\-सञ्ज्ञा (पूर्ववत्)~\arrow \textcolor{red}{भावकर्मणोः} (पा॰सू॰~१.३.१३)~\arrow \textcolor{red}{वर्तमान\-सामीप्ये वर्तमानवद्वा} (पा॰सू॰~३.३.१३१)~\arrow \textcolor{red}{वर्तमाने लट्} (पा॰सू॰~३.२.१२३)~\arrow पूजि~लट्~\arrow पूजि~थास्~\arrow \textcolor{red}{सार्वधातुके यक्} (पा॰सू॰~३.१.६७)~\arrow पूजि~यक्~थास्~\arrow पूजि~य~थास्~\arrow \textcolor{red}{णेरनिटि} (पा॰सू॰~६.४.५१)~\arrow पूज्~य~थास्~\arrow \textcolor{red}{थासस्से} (पा॰सू॰~३.४.८०)~\arrow पूज्~य~से~\arrow पूज्यसे।} किमर्थमिदमिति चेत्। \textcolor{red}{त्वं यदा महासत्त्वं द्रक्ष्यसि ततो धनुषि भग्ने शीघ्रमेव जनकेन पूजयिष्यसे} इति शीघ्रतां ध्वनयितुं भविष्यति वर्तमान\-वत्कार्यम्।\end{sloppypar}
\section[आगमिष्यति]{आगमिष्यति}
\centering\textcolor{blue}{एवं वर्षसहस्रेषु ह्यनेकेषु गतेषु च।\nopagebreak\\
रामो दाशरथिः श्रीमानागमिष्यति सानुजः॥}\nopagebreak\\
\raggedleft{–~अ॰रा॰~१.५.३०}\\
\fontsize{14}{21}\selectfont\begin{sloppypar}\hyphenrules{nohyphenation}\justifying\noindent\hspace{10mm} अत्र यद्यप्यनद्य\-तनत्वाल्लुड्लकारः\footnote{लुटि \textcolor{red}{आगन्ता} इति रूपम्। आङ्~\textcolor{red}{गमॢँ गतौ} (धा॰पा॰~९८२)~\arrow आ~गम्~\arrow \textcolor{red}{शेषात्कर्तरि परस्मैपदम्} (पा॰सू॰~१.३.७८)~\arrow \textcolor{red}{अनद्यतने लुट्} (पा॰सू॰~३.३.१५)~\arrow आ~गम्~\arrow आ~गम्~लुट्~\arrow आ~गम्~तिप्~\arrow आ~गम्~ति~\arrow \textcolor{red}{स्यतासी लृलुटोः} (पा॰सू॰~३.१.३३)~\arrow आ~गम्~तास्~ति~\arrow \textcolor{red}{लुटः प्रथमस्य डारौरसः} (पा॰सू॰~२.४.८५)~\arrow आ~गम्~तास्~डा~\arrow आ~गम्~तास्~आ~\arrow \textcolor{red}{डित्यभस्याप्यनु\-बन्धकरण\-सामर्थ्यात्} (वा॰~६.४.१४३)~\arrow आ~गम्~त्~आ~\arrow \textcolor{red}{मोऽनुस्वारः} (पा॰सू॰~८.३.२३)~\arrow आ~गं~त्~आ~\arrow \textcolor{red}{अनुस्वारस्य ययि परसवर्णः} (पा॰सू॰~८.४.५८)~\arrow आ~गन्~त्~आ~\arrow आगन्ता।} \textcolor{red}{अनद्यतने लुट्} (पा॰सू॰~३.३.१५) इति सूत्रानुशासनात्किन्तु
सामान्य\-समयतोऽनिश्चित\-भविष्यद्विवक्षायां
लृडेव।\footnote{आङ्~\textcolor{red}{गमॢँ गतौ} (धा॰पा॰~९८२)~\arrow आ~गम्~\arrow \textcolor{red}{शेषात्कर्तरि परस्मैपदम्} (पा॰सू॰~१.३.७८)~\arrow \textcolor{red}{लृट् शेषे च} (पा॰सू॰~३.३.१३)~\arrow आ~गम्~लृँट्~\arrow आ~गम्~तिप्~\arrow आ~गम्~ति~\arrow \textcolor{red}{स्यतासी लृलुटोः} (पा॰सू॰~३.१.३३)~\arrow आ~गम्~स्य~ति~\arrow \textcolor{red}{गमेरिट् परस्मैपदेषु} (पा॰सू॰~७.२.५८)~\arrow आ~गम्~इट्~स्य~ति~\arrow आ~गम्~इ~स्य~ति~\arrow \textcolor{red}{आदेशप्रत्यययोः} (पा॰सू॰~८.३.५९)~\arrow आ~गम्~इ~ष्य~ति~\arrow आगमिष्यति।} अतो न दोषः।\end{sloppypar}
\section[काङ्क्षते]{काङ्क्षते}
\centering\textcolor{blue}{तव पादरजःस्पर्शं काङ्क्षते पवनाशना।\nopagebreak\\
आस्तेऽद्यापि रघुश्रेष्ठ तपो दुष्करमास्थिता॥}\nopagebreak\\
\raggedleft{–~अ॰रा॰~१.५.३४}\\
\fontsize{14}{21}\selectfont\begin{sloppypar}\hyphenrules{nohyphenation}\justifying\noindent\hspace{10mm} विश्वामित्रः श्रीरामं प्रति वेदयति यत् \textcolor{red}{अहल्या भवच्चरण\-धूलिं वाञ्छति}। तत्र \textcolor{red}{काङ्क्षते} इति प्रयुक्तम्। अत्र कर्म\-व्यतिहार आत्मनेपदम्।\footnote{\textcolor{red}{कर्तरि कर्म\-व्यतिहारे} (पा॰सू॰~१.३.१४) इत्यनेन। \textcolor{red}{काक्षिँ काङ्क्षायाम्} (धा॰पा॰~६६७)~\arrow काक्ष्~\arrow \textcolor{red}{इदितो नुम् धातोः} (पा॰सू॰~७.१.५८)~\arrow \textcolor{red}{मिदचोऽन्त्यात्परः} (पा॰सू॰~१.१.४७)~\arrow का~नुँम्~क्ष्~\arrow कान्~क्ष्~\arrow \textcolor{red}{नश्चापदान्तस्य झलि} (पा॰सू॰~८.३.२४)~\arrow कांक्ष्~\arrow \textcolor{red}{अनुस्वारस्य ययि परसवर्णः} (पा॰सू॰~८.४.५८)~\arrow काङ्क्ष्~\arrow \textcolor{red}{कर्तरि कर्म\-व्यतिहारे} (पा॰सू॰~१.३.१४)~\arrow \textcolor{red}{वर्तमाने लट्} (पा॰सू॰~३.२.१२३)~\arrow काङ्क्ष्~लट्~\arrow काङ्क्ष्~त~\arrow \textcolor{red}{कर्तरि शप्‌} (पा॰सू॰~३.१.६८)~\arrow काङ्क्ष्~शप्~त~\arrow काङ्क्ष्~अ~त~\arrow \textcolor{red}{टित आत्मनेपदानां टेरे} (पा॰सू॰~३.४.७९)~\arrow काङ्क्ष्~अ~ते~\arrow काङ्क्षते।} अन्यस्य योग्यं कार्यमन्यः करोति यदा तदा कर्म\-व्यतिहारः। तव पाद\-रजो योग्यतया श्रीर्वाञ्छति। यथा श्रीमद्भागवते~–\end{sloppypar}
\centering\textcolor{red}{श्रीर्यत्पदाम्बुजरजश्चकमे तुलस्या लब्ध्वाऽपि वक्षसि पदं किल भृत्यजुष्टम्।\nopagebreak\\
यस्याः स्ववीक्षणकृतेऽन्यसुरप्रयासस्तद्वद्वयं च तव पादरजःप्रपन्नाः॥}\nopagebreak\\
\raggedleft{–~भा॰पु॰~१०.२९.३७}\\
\fontsize{14}{21}\selectfont\begin{sloppypar}\hyphenrules{nohyphenation}\justifying\noindent अर्थात्साक्षान्महा\-लक्ष्मीर्भगवती त्वत्पाद\-रजोऽधिकारिणी किन्तु परम\-पातक\-कारिणी व्यभिचारिणी सत्यपीयं रजो\-गुण\-नाशाय तव चरण\-धूलिं काङ्क्षतीत्यन्य\-योग्य\-काङ्क्षण आत्मनेपदम्। अहल्याऽऽत्मनः पराभव\-मूलं कामं तदुद्भवं च रजो\-गुणं मन्यमाना तं च रजो\-गुणं मुकुन्द\-श्रीराम\-भद्रचरणारविन्द\-रजसा परिमार्ष्टुमिच्छतीति कौशिक\-हार्दम्। कामो रजो\-गुणोद्भव इति गीतायां निर्दिष्टं यथा~–\end{sloppypar}
\centering\textcolor{red}{काम एष क्रोध एष रजोगुणसमुद्भवः।\nopagebreak\\
महाशनो महापाप्मा विद्ध्येनमिह वैरिणम्॥}\nopagebreak\\
\raggedleft{–~भ॰गी॰~३.३७}\\
\fontsize{14}{21}\selectfont\begin{sloppypar}\hyphenrules{nohyphenation}\justifying\noindent यद्वा ब्राह्मणी सत्यपि क्षत्त्रियस्य ते चरणारविन्द\-रजः काङ्क्षत्यतो दयस्वेत्यन्य\-योग\-काङ्क्षण\-रूपे कर्म\-व्यतिहार आत्मनेपदे \textcolor{red}{काङ्क्षते}।\end{sloppypar}
\section[पावयस्व]{पावयस्व}
\centering\textcolor{blue}{पावयस्व मुनेर्भार्यामहल्यां ब्रह्मणः सुताम्।\nopagebreak\\
इत्युक्त्वा राघवं हस्ते गृहीत्वा मुनिपुङ्गवः॥}\nopagebreak\\
\raggedleft{–~अ॰रा॰~१.५.३५}\\
\fontsize{14}{21}\selectfont\begin{sloppypar}\hyphenrules{nohyphenation}\justifying\noindent\hspace{10mm} विश्वामित्रः कथयति यत्
\textcolor{red}{भगवन्नहल्यां पुनीहि}।
अत्र \textcolor{red}{पावय} इति प्रयोक्तव्यम्।\footnote{\textcolor{red}{पूञ् पवने} (धा॰पा॰~१४८२)~\arrow पू~\arrow \textcolor{red}{हेतुमति च} (पा॰सू॰~३.१.२६) (यद्वा स्वार्थे णिच्)~\arrow पू~णिच्~\arrow पू~इ~\arrow \textcolor{red}{अचो ञ्णिति} (पा॰सू॰~७.२.११५)~\arrow पौ~इ~\arrow \textcolor{red}{एचोऽयवायावः} (पा॰सू॰~६.१.७८)~\arrow पाव्~इ~\arrow पावि~\arrow \textcolor{red}{सनाद्यन्ता धातवः} (पा॰सू॰~३.१.३२)~\arrow धातु\-सञ्ज्ञा~\arrow \textcolor{red}{शेषात्कर्तरि परस्मैपदम्} (पा॰सू॰~१.३.७८)~\arrow \textcolor{red}{लोट् च} (पा॰सू॰~३.३.१६२)~\arrow पावि~लोट्~\arrow पावि~सिप्~\arrow पावि~सि~\arrow \textcolor{red}{कर्तरि शप्} (पा॰सू॰~३.१.६८)~\arrow पावि~शप्~सि~\arrow पावि~अ~सि~\arrow \textcolor{red}{सार्वधातुकार्धधातुकयोः} (पा॰सू॰~७.३.८४)~\arrow पावे~अ~सि~\arrow \textcolor{red}{एचोऽयवायावः} (पा॰सू॰~६.१.७८)~\arrow पावय्~अ~सि~\arrow \textcolor{red}{सेर्ह्यपिच्च} (पा॰सू॰~३.४.८७)~\arrow पावय्~अ~हि~\arrow \textcolor{red}{अतो हेः} (पा॰सू॰~६.४.१०५)~\arrow पावय्~अ~\arrow पावय।} \textcolor{red}{पावयस्व} इति पावन\-रूपं फलं त्वय्येव स्थास्यति तव भक्त\-वात्सल्यं पतित\-पावनत्वं वा दिगन्ते प्रसरिष्यत्यत आत्मनेपदम्।\footnote{पावि~\arrow धातु\-सञ्ज्ञा (पूर्ववत्)~\arrow \textcolor{red}{णिचश्च} (पा॰सू॰~१.३.७४)~\arrow \textcolor{red}{लोट् च} (पा॰सू॰~३.३.१६२)~\arrow पावि~लोट्~\arrow पावि~थास्~\arrow \textcolor{red}{कर्तरि शप्} (पा॰सू॰~३.१.६८)~\arrow पावि~शप्~थास्~\arrow पावि~अ~थास्~\arrow \textcolor{red}{सार्वधातुकार्धधातुकयोः} (पा॰सू॰~७.३.८४)~\arrow पावे~अ~थास्~\arrow \textcolor{red}{एचोऽयवायावः} (पा॰सू॰~६.१.७८)~\arrow पावय्~अ~थास्~\arrow \textcolor{red}{थासस्से} (पा॰सू॰~३.४.८०)~\arrow पावय्~अ~से~\arrow \textcolor{red}{सवाभ्यां वामौ} (पा॰सू॰~३.४.९१)~\arrow पावय्~अ~स्व~\arrow पावयस्व।} यद्वा \textcolor{red}{हे स्व} अर्थात् \textcolor{red}{हे आत्मीय हे आत्मन् हे धन पावय} इत्यन्वये परिहारः। \textcolor{red}{स्वमज्ञाति\-धनाख्यायाम्} (पा॰सू॰~१.१.३५) इति सूत्रस्य भाष्ये \textcolor{red}{स्व}\-शब्दस्य चत्वारोऽर्थाः सङ्केतेन कथिताः~– आत्माऽऽत्मीयो ज्ञातिर्धनमिति।\footnote{\textcolor{red}{आख्याग्रहणं किमर्थम्। ज्ञाति\-धन\-पर्यायवाची यः स्वशब्दस्तस्य यथा स्यात्। इह मा भूत्~– स्वे पुत्राः स्वाः पुत्राः स्वे गावः स्वाः गावः} (भा॰पा॰सू॰~१.१.३५)। \textcolor{red}{स्वे स्वाः। आत्मीया इत्यर्थः। आत्मान इति वा। ज्ञाति\-धन\-वाचिनस्तु स्वाः। ज्ञातयोऽर्था वा} (वै॰सि॰कौ॰~२२०, १.१.३५)।} अत्र चत्वारोऽप्यर्था अनुसन्धेयाः। विश्वामित्रः कथयति यत्त्वमात्मा त्वमात्मीयस्त्वमेवास्मज्ज्ञातिर्ब्रह्मण्य\-देवत्वात्त्वमेवास्मद्धनमाराध्यत्वात्। अतः \textcolor{red}{हे स्व पावय} इति सम्यक्पाणिनीयता।\end{sloppypar}
\section[किं वर्ण्यते]{किं वर्ण्यते}
\centering\textcolor{blue}{यत्पादपङ्कजपरागपवित्रगात्रा भागीरथी भवविरिञ्चिमुखान्पुनाति।\nopagebreak\\
साक्षात्स एव मम दृग्विषयो यदास्ते किं वर्ण्यते मम पुराकृतभागधेयम्॥}\nopagebreak\\
\raggedleft{–~अ॰रा॰~१.५.४५}\\
\fontsize{14}{21}\selectfont\begin{sloppypar}\hyphenrules{nohyphenation}\justifying\noindent\hspace{10mm} अहल्या स्वकीयं भाग्यं प्रशंसति। \textcolor{red}{किं वर्ण्यते} इति।
अत्र \textcolor{red}{वर्णयितुं शक्यते} इति वर्तमान\-सामीप्याल्लट्।\footnote{\textcolor{red}{वर्तमान\-सामीप्ये वर्तमानवद्वा} (पा॰सू॰~३.३.१३१) इत्यनेन। \textcolor{red}{वर्ण वर्णक्रिया\-विस्तार\-गुण\-वचनेषु} (धा॰पा॰~१९३९)~\arrow वर्ण~\arrow \textcolor{red}{सत्याप\-पाश\-रूप\-वीणा\-तूल\-श्लोक\-सेना\-लोम\-त्वच\-वर्म\-वर्ण\-चूर्ण\-चुरादिभ्यो णिच्} (पा॰सू॰~३.१.२५)~\arrow वर्ण~णिच्~\arrow वर्ण~इ~\arrow \textcolor{red}{णाविष्ठवत्प्राति\-पदिकस्य पुंवद्भाव\-रभाव\-टिलोप\-यणादि\-परार्थम्} (वा॰~६.४.४८)~\arrow वर्ण्~इ~\arrow वर्णि~\arrow धातु\-सञ्ज्ञा~\arrow \textcolor{red}{भावकर्मणोः} (पा॰सू॰~१.३.१३)~\arrow \textcolor{red}{वर्तमान\-सामीप्ये वर्तमानवद्वा} (पा॰सू॰~३.३.१३१)~\arrow \textcolor{red}{वर्तमाने लट्} (पा॰सू॰~३.२.१२३)~\arrow वर्णि~लट्~\arrow वर्णि~त~\arrow वर्णि~त~\arrow \textcolor{red}{सार्वधातुके यक्} (पा॰सू॰~३.१.६७)~\arrow वर्णि~यक्~त~\arrow वर्णि~य~त~\arrow \textcolor{red}{णेरनिटि} (पा॰सू॰~६.४.५१)~\arrow वर्ण्~य~त~\arrow \textcolor{red}{टित आत्मनेपदानां टेरे} (पा॰सू॰~३.४.७९)~\arrow वर्ण्~य~ते~\arrow वर्ण्यते।}\end{sloppypar}
\section[अधिगच्छति]{अधिगच्छति}
\centering\textcolor{blue}{अहल्यया कृतं स्तोत्रं यः पठेद्भक्तिसंयुतः।\nopagebreak\\
स मुच्यतेऽखिलैः पापैः परं ब्रह्माधिगच्छति॥}\nopagebreak\\
\raggedleft{–~अ॰रा॰~१.५.६२}\\
\fontsize{14}{21}\selectfont\begin{sloppypar}\hyphenrules{nohyphenation}\justifying\noindent\hspace{10mm} अत्र वर्तमान\-सामीप्याल्लट्।\footnote{\textcolor{red}{वर्तमान\-सामीप्ये वर्तमानवद्वा} (पा॰सू॰~३.३.१३१) इत्यनेन। अधि~\textcolor{red}{गमॢँ गतौ} (धा॰पा॰~९८२)~\arrow अधि~गम्~\arrow \textcolor{red}{शेषात्कर्तरि परस्मैपदम्} (पा॰सू॰~१.३.७८)~\arrow \textcolor{red}{वर्तमान\-सामीप्ये वर्तमानवद्वा} (पा॰सू॰~३.३.१३१)~\arrow \textcolor{red}{वर्तमाने लट्} (पा॰सू॰~३.२.१२३)~\arrow अधि~गम्~लट्~\arrow अधि~गम्~तिप्~\arrow अधि~गम्~ति~\arrow \textcolor{red}{कर्तरि शप्} (पा॰सू॰~३.१.६८)~\arrow अधि~गम्~शप्~ति~\arrow अधि~गम्~अ~ति~\arrow \textcolor{red}{इषुगमियमां छः} (पा॰सू॰~७.३.७७)~\arrow अधि~गछ्~अ~ति~\arrow \textcolor{red}{छे च} (पा॰सू॰~६.१.७३)~\arrow \textcolor{red}{आद्यन्तौ टकितौ} (पा॰सू॰~१.१.४६)~\arrow अधि~गतुँक्~छ्~अ~ति~\arrow अधि~गत्~छ्~अ~ति~\arrow \textcolor{red}{स्तोः श्चुना श्चुः} (पा॰सू॰~८.४.४०)~\arrow अधि~गच्~छ्~अ~ति~\arrow अधिगच्छति।} \textcolor{red}{अहल्या\-कृत\-स्तोत्रं पठित्वा सद्यः परं ब्रह्माधि\-गमिष्यति} इति ध्वनयितुं वर्तमान\-प्रत्ययः। अत्र कर्म\-कर्तृ\-प्रयोगः। यदा कार्य\-सौकर्यातिशयं बोधयितुं कर्तृ\-व्यापारो न विवक्ष्यते तदा कारकान्तराण्यपि कर्तृ\-सञ्ज्ञां लभन्ते स्व\-व्यापारे स्वतन्त्रत्वात्। यथा \textcolor{red}{सीता कन्द\-मूलं पचति}। सीतया किं कन्द\-मूलं स्वयमेव पच्यते तथैवात्रापि। \textcolor{red}{स ब्रह्माधिगच्छति}। तेन किं पर\-ब्रह्म स्वयमेवाधिगम्यते। \textcolor{red}{अधिगच्छति शास्त्रार्थः} (वै॰सि॰कौ॰~२७६६) इतिवत्।
\end{sloppypar}
\section[क्षालयामि]{क्षालयामि}
\centering\textcolor{blue}{क्षालयामि तव पादपङ्कजं नाथ दारुदृषदोः किमन्तरम्।\nopagebreak\\
मानुषीकरणचूर्णमस्ति ते पादयोरिति कथा प्रथीयसी॥}\nopagebreak\\
\raggedleft{–~अ॰रा॰~१.६.३}\\
\fontsize{14}{21}\selectfont\begin{sloppypar}\hyphenrules{nohyphenation}\justifying\noindent\hspace{10mm} अत्र वर्तमान\-सामीप्याल्लट्।\footnote{\textcolor{red}{वर्तमान\-सामीप्ये वर्तमानवद्वा} (पा॰सू॰~३.३.१३१) इत्यनेन। \textcolor{red}{क्षलँ शौचकर्मणि} (धा॰पा॰~१५९७)~\arrow क्षल्~\arrow \textcolor{red}{सत्याप\-पाश\-रूप\-वीणा\-तूल\-श्लोक\-सेना\-लोम\-त्वच\-वर्म\-वर्ण\-चूर्ण\-चुरादिभ्यो णिच्} (पा॰सू॰~३.१.२५)~\arrow क्षल्~णिच्~\arrow क्षल्~इ~\arrow \textcolor{red}{अत उपधायाः} (पा॰सू॰~७.२.११६)~\arrow क्षाल्~इ~\arrow क्षालि~\arrow \textcolor{red}{सनाद्यन्ता धातवः} (पा॰सू॰~३.१.३२)~\arrow धातु\-सञ्ज्ञा~\arrow \textcolor{red}{शेषात्कर्तरि परस्मैपदम्} (पा॰सू॰~१.३.७८)~\arrow \textcolor{red}{वर्तमान\-सामीप्ये वर्तमानवद्वा} (पा॰सू॰~३.३.१३१)~\arrow \textcolor{red}{वर्तमाने लट्} (पा॰सू॰~३.२.१२३)~\arrow क्षालि~लट्~\arrow क्षालि~मिप्~\arrow क्षालि~मि~\arrow \textcolor{red}{कर्तरि शप्} (पा॰सू॰~३.१.६८)~\arrow क्षालि~शप्~मि~\arrow क्षालि~अ~मि~\arrow \textcolor{red}{सार्वधातुकार्धधातुकयोः} (पा॰सू॰~७.३.८४)~\arrow क्षाले~अ~मि~\arrow \textcolor{red}{एचोऽयवायावः} (पा॰सू॰~६.१.७८)~\arrow क्षालय्~अ~मि~\arrow \textcolor{red}{अतो दीर्घो यञि} (पा॰सू॰~७.३.१०१)~\arrow क्षालय्~आ~मि~\arrow क्षालयामि।}\end{sloppypar}
\section[अनुशुश्रुवे]{अनुशुश्रुवे}
\centering\textcolor{blue}{पूजितं राजभिः सर्वैर्दृष्टमित्यनुशुश्रुवे।\nopagebreak\\
अतो दर्शय राजेन्द्र शैवं चापमनुत्तमम्।\nopagebreak\\
दृष्ट्वाऽयोध्यां जिगमिषुः पितरं द्रष्टुमिच्छति॥}\nopagebreak\\
\raggedleft{–~अ॰रा॰~१.६.१६}\\
\fontsize{14}{21}\selectfont\begin{sloppypar}\hyphenrules{nohyphenation}\justifying\noindent\hspace{10mm} अत्र विश्वामित्रो लिड्लकार\-क्रियां प्रयुङ्क्ते। तस्य कथं पारोक्ष्यं त्रिकाल\-दर्शित्वात्। वस्तुतस्त्वपरोक्षानुभूतेः सरूपं सगुणं ब्रह्म श्रीरामं दर्शं दर्शं विस्मृत\-सकल\-व्यापारतया भावातिरेके प्रेम\-सिन्धौ ज्ञान\-प्लवस्य मग्नतां सूचयितुं परोक्ष\-प्रयोगः।
\textcolor{red}{बहु जगद पुरस्तात्तस्य मत्ता किलाहम्} (शि॰~११.३९) इति प्रयोग इव।\end{sloppypar}
\section[आकर्षयामास]{आकर्षयामास}
\centering\textcolor{blue}{ईषदाकर्षयामास पाणिना दक्षिणेन सः।\nopagebreak\\
बभञ्जाखिलहृत्सारो दिशः शब्देन पूरयन्॥}\nopagebreak\\
\raggedleft{–~अ॰रा॰~१.६.२५}\\
\fontsize{14}{21}\selectfont\begin{sloppypar}\hyphenrules{nohyphenation}\justifying\noindent\hspace{10mm} अत्र स्वार्थे णिजन्तत्वात् \textcolor{red}{आकर्षयामास} इति।\footnote{\textcolor{red}{कृषँ विलेखने} (धा॰पा॰~९९०)~\arrow कृष्~\arrow स्वार्थे णिच्~\arrow कृष्~णिच्~\arrow कृष्~इ~\arrow \textcolor{red}{पुगन्त\-लघूपधस्य च} (पा॰सू॰~७.३.८६)~\arrow \textcolor{red}{उरण् रपरः} (पा॰सू॰~१.१.५१)~\arrow कर्ष्~इ~\arrow कर्षि~\arrow \textcolor{red}{सनाद्यन्ता धातवः} (पा॰सू॰~३.१.३२)~\arrow धातुसञ्ज्ञा। आङ्~कर्षि~\arrow आ~कर्षि~\arrow \textcolor{red}{शेषात्कर्तरि परस्मैपदम्} (पा॰सू॰~१.३.७८)~\arrow \textcolor{red}{परोक्षे लिट्} (पा॰सू॰~३.२.११५)~\arrow आ~कर्षि~लिट्~\arrow आ~कर्षि~तिप्~\arrow \textcolor{red}{परस्मैपदानां णलतुसुस्थलथुस\-णल्वमाः} (पा॰सू॰~३.४.८२)~\arrow आ~कर्षि~ण~\arrow आ~कर्षि~अ~\arrow \textcolor{red}{कास्यनेकाचश्चुलुम्पाद्यर्थम्} (वा॰~३.३.३५)~\arrow आ~कर्षि~आम्~अ~\arrow \textcolor{red}{सार्वधातुकार्धधातुकयोः} (पा॰सू॰~७.३.८४)~\arrow आ~कर्षे~आम्~अ~\arrow \textcolor{red}{एचोऽयवायावः} (पा॰सू॰~६.१.७८)~\arrow आ~कर्षय्~आम्~अ~\arrow \textcolor{red}{आमः} (पा॰सू॰~२.४.८१)~\arrow आ~कर्षय्~आम्~\arrow \textcolor{red}{कृञ्चानुप्रयुज्यते लिटि} (पा॰सू॰~३.१.४०)~\arrow आ~कर्षय्~आम्~अस्~लिट्~\arrow आ~कर्षय्~आम्~अस्~तिप्~\arrow \textcolor{red}{परस्मैपदानां णलतुसुस्थलथुस\-णल्वमाः} (पा॰सू॰~३.४.८२)~\arrow आ~कर्षय्~आम्~अस्~ण~\arrow आ~कर्षय्~आम्~अस्~अ~\arrow \textcolor{red}{लिटि धातोरनभ्यासस्य} (पा॰सू॰~६.१.८)~\arrow आ~कर्षय्~आम्~अस्~अस्~अ~\arrow \textcolor{red}{हलादिः शेषः} (पा॰सू॰~७.४.६०)~\arrow आ~कर्षय्~आम्~अ~अस्~अ~\arrow \textcolor{red}{अत आदेः} (पा॰सू॰~७.४.७०)~\arrow आ~कर्षय्~आम्~आ~अस्~अ~\arrow \textcolor{red}{अत उपधायाः} (पा॰सू॰~७.२.११६)~\arrow आ~कर्षय्~आम्~आ~आस्~अ~\arrow \textcolor{red}{अकः सवर्णे दीर्घः} (पा॰सू॰~६.१.१०१)~\arrow आ~कर्षय्~आम्~आस्~अ~\arrow आकर्षयामास। यद्वा \textcolor{red}{आकर्षणमाकर्षः}। \textcolor{red}{भावे} (पा॰सू॰~३.३.१८) इत्यनेन भावे घञ्। ततः \textcolor{red}{तत्करोति तदाचष्टे} (धा॰पा॰ ग॰सू॰~१८७) इत्यनेन णिचि गुणे रपरत्वे \textcolor{red}{आकर्षि} इति धातुसञ्ज्ञायां लिटि तिप्यामि लुकि \textcolor{red}{अस्‌}\-धात्वनु\-प्रयोगे पूर्वोक्तदिशा \textcolor{red}{आकर्षयामास}।}\end{sloppypar}
\section[आनयामास]{आनयामास}
\centering\textcolor{blue}{ततः शुभे दिने लग्ने सुमुहूर्ते रघूत्तमम्।\nopagebreak\\
आनयामास धर्मज्ञो रामं सभ्रातृकं तदा॥}\nopagebreak\\
\raggedleft{–~अ॰रा॰~१.६.४५}\\
\fontsize{14}{21}\selectfont\begin{sloppypar}\hyphenrules{nohyphenation}\justifying\noindent\hspace{10mm} \textcolor{red}{आङ्} पूर्वकः \textcolor{red}{णीञ् प्रापणे} (धा॰पा॰~९०१) इति धातुः। \textcolor{red}{णो नः} (पा॰सू॰~६.१.६५) इत्यनेन नकारः। अत्र परोक्षे लिड्लकारः। \textcolor{red}{परस्मैपदानां णलतुसुस्थलथुस\-णल्वमाः} (पा॰सू॰~३.४.८२) इत्यनेन णलादेशे वृद्धौ द्वित्वेऽभ्यास\-लोपेऽयादेशे \textcolor{red}{आनिनाय} इत्येव।\footnote{आङ्~\textcolor{red}{णीञ् प्रापणे} (धा॰पा॰~९०१)~\arrow आ~नी~\arrow \textcolor{red}{शेषात्कर्तरि परस्मैपदम्} (पा॰सू॰~१.३.७८)~\arrow \textcolor{red}{परोक्षे लिट्} (पा॰सू॰~३.२.११५)~\arrow आ~नी~लिट्~\arrow आ~नी~तिप्~\arrow आ~नी~ति~\arrow \textcolor{red}{परस्मैपदानां णलतुसुस्थलथुस\-णल्वमाः} (पा॰सू॰~३.४.८२)~\arrow आ~नी~ण~\arrow आ~नी~अ~\arrow \textcolor{red}{लिटि धातोरनभ्यासस्य} (पा॰सू॰~६.१.८)~\arrow आ~नी~नी~अ~\arrow \textcolor{red}{ह्रस्वः} (पा॰सू॰~७.४.५९)~\arrow आ~नि~नी~अ~\arrow \textcolor{red}{अचो ञ्णिति} (पा॰सू॰~७.२.११५)~\arrow आ~नि~नै~अ~\arrow \textcolor{red}{एचोऽयवायावः} (पा॰सू॰~६.१.७८)~\arrow आ~नि~नाय्~अ~\arrow आनिनाय।} \textcolor{red}{आनयामास} इति कथम्। उच्यते।
\textcolor{red}{आनयतीत्यानया} भावे \textcolor{red}{अच्‌}\-प्रत्यये गुणेऽयादेशे टापि।\footnote{\textcolor{red}{नन्दि\-ग्रहि\-पचादिभ्यो ल्युणिन्यचः} (पा॰सू॰~३.१.१३४) इत्यनेन। बाहुलकाद्भावे स्त्रियाम्।}
\textcolor{red}{आनयाम्}। क्रिया\-विशेषणत्वाद्द्वितीया। \textcolor{red}{आस} बभूवेति।\footnote{\textcolor{red}{बभूव} इत्यर्थे \textcolor{red}{आस} इति बहुधा शिष्टप्रयोगेषु पुराणेषु च दृश्यते। यथा लघुत्रय्याम्~– \textcolor{red}{लावण्य उत्पाद्य इवास यत्नः} (कु॰स॰~१.३५) \textcolor{red}{निष्प्रभश्च रिपुरास भूभृताम्} (र॰वं॰~११.८१) \textcolor{red}{तेनास लोकः पितृमान्विनेत्रा} (र॰वं॰~१४.२३)। अत्र मल्लिनाथः~– \textcolor{red}{आसेति बभूवार्थे “तिङन्त\-प्रतिरूपकमव्ययम्” इत्याह शाकटायनः। वल्लभस्तु “न तिङन्त\-प्रतिरूपकमव्ययम् ‘अस्तेर्भूः’ इति भ्वादेश\-नियमात्तादृक्तिङन्तस्यैवाभावात्। किन्तु कवीनामयं प्रामादिकः प्रयोगः” इत्याह। वामनस्तु “अस गतिदीप्त्यादानेष्विति धातोर्लिटि रूपमिदम्” इत्याह। अस इत्यनुदात्तेद्दीप्त्यर्थे। आस दिदीपे। प्रवृत्त इत्यर्थः}। भागवते च~– \textcolor{red}{मैत्रेयेणास सङ्गमः} (भा॰पु॰~३.१.३) \textcolor{red}{भगवानेक आसेदम्} (भा॰पु॰~३.५.२३) \textcolor{red}{निष्क्रामति निर्विशती द्विधाऽऽस सा} (भा॰पु॰~४.४.१) \textcolor{red}{असन्नपि क्लेशद आस देहः} (भा॰पु॰~५.५.४) \textcolor{red}{न यदिदमग्र आस न भविष्यदतो निधनात्} (भा॰पु॰~१०.८७.३७)। यद्वा \textcolor{red}{अस्‌}\-धातोर्लिटि छान्दस\-वैकल्पिक\-प्रयोगोऽयम्। तथा च \textcolor{red}{बहुलं छन्दसि} (पा॰सू॰~२.४.७३) इति सूत्रे मण्डूकप्लुत्या \textcolor{red}{अस्तेर्भूः} (पा॰सू॰~२.४.५२) इति सूत्रं चाप्यनुवर्तनीयम्। तेन क्वचिदादेशाप्रवृत्तिः। यथा~– \textcolor{red}{तस्य ह नचिकेता नाम पुत्र आस} (क॰उ॰~१.१.१) इत्यत्र श्रीराघव\-कृपा\-भाष्ये ग्रन्थ\-प्रणेतारः~– \textcolor{red}{आस बभूव। अत्र ‘बहुलं छन्दसि’ (पा॰सू॰~२.४.७३) इत्यप्रवृत्ति\-लक्षण\-बाहुलकेन ‘अस्तेर्भूः’ (पा॰सू॰~२.४.५२) इत्यस्याप्रवृत्तौ ‘आस’ इति छान्दस\-प्रयोगः} (क॰उ॰ रा॰कृ॰भा॰~१.१.१)। \textcolor{red}{असँ भुवि} (धा॰पा॰~१०६५)~\arrow अस्~\arrow \textcolor{red}{शेषात्कर्तरि परस्मैपदम्} (पा॰सू॰~१.३.७८)~\arrow \textcolor{red}{परोक्षे लिट्} (पा॰सू॰~३.२.११५)~\arrow अस्~लिट्~\arrow अस्~तिप्~\arrow \textcolor{red}{परस्मैपदानां णलतुसुस्थलथुस\-णल्वमाः} (पा॰सू॰~३.४.८२)~\arrow अस्~णल्~\arrow अस्~अ~\arrow \textcolor{red}{बहुलं छन्दसि} (पा॰सू॰~२.४.७३)~\arrow भ्वादेशाप्रवृत्तिः~\arrow \textcolor{red}{लिटि धातोरनभ्यासस्य} (पा॰सू॰~६.१.८)~\arrow अस्~अस्~अ~\arrow \textcolor{red}{हलादिः शेषः} (पा॰सू॰~७.४.६०)~\arrow अ~अस्~अ~\arrow \textcolor{red}{अत आदेः} (पा॰सू॰~७.४.७०)~\arrow आ~अस्~अ~\arrow \textcolor{red}{अत उपधायाः} (पा॰सू॰~७.२.११६)~\arrow आ~आस्~अ~\arrow \textcolor{red}{अकः सवर्णे दीर्घः} (पा॰सू॰~६.१.१०१)~\arrow आस्~अ~\arrow आस। छन्दसि द्वयोरपि रूपयोः प्रयोगः। \textcolor{red}{द॒क्षाय्यो॒ यो दम॒ आस॒ नित्य॑} (ऋ॰वे॰सं॰~७.१.२) इत्यत्र \textcolor{red}{आस}। \textcolor{red}{अ॒रान्न ने॒मिः परि॒ ता ब॑भूव} (ऋ॰वे॰सं॰~१.३२.१५) इत्यत्र \textcolor{red}{बभूव} च।}\end{sloppypar}
\section[मुमोद]{मुमोद}
\centering\textcolor{blue}{मुमोद जनको लक्ष्मीं क्षीराब्धिरिव विष्णवे।\nopagebreak\\
ऊर्मिलां चौरसीं कन्यां लक्ष्मणाय ददौ मुदा॥}\nopagebreak\\
\raggedleft{–~अ॰रा॰~१.६.५५}\\
\fontsize{14}{21}\selectfont\begin{sloppypar}\hyphenrules{nohyphenation}\justifying\noindent\hspace{10mm} अत्र \textcolor{red}{मुद्‌}\-धातुः (\textcolor{red}{मुदँ हर्षे} धा॰पा॰~१६) आत्मनेपदी। एवं तत्र लिटि लकारे \textcolor{red}{मुमुदे} इति पाणिनीयम्।\footnote{\textcolor{red}{मुदँ हर्षे} (धा॰पा॰~१६)~\arrow मुद्~\arrow \textcolor{red}{अनुदात्तङित आत्मने\-पदम्} (पा॰सू॰~१.३.१२)~\arrow \textcolor{red}{परोक्षे लिट्} (पा॰सू॰~३.२.११५)~\arrow मुद्~लिट्~\arrow मुद्~त~\arrow \textcolor{red}{लिटस्तझयोरेशिरेच्} (पा॰सू॰~३.४.८१)~\arrow मुद्~एश्~\arrow मुद्~ए~\arrow \textcolor{red}{लिटि धातोरनभ्यासस्य} (पा॰सू॰~६.१.८)~\arrow मुद्~मुद्~ए~\arrow हलादिः शेषः~\arrow मु~मुद्~ए~\arrow \textcolor{red}{असंयोगाल्लिट् कित्} (पा॰सू॰~१.२.५)~\arrow कित्त्वम्~\arrow \textcolor{red}{ग्क्ङिति च} (पा॰सू॰~१.१.५)~\arrow लघूपध\-गुणनिषेधः~\arrow मु~मुद्~ए~\arrow मुमुदे।} \textcolor{red}{मुमोद} इत्यपि। \textcolor{red}{मोदनं मोदः}। भावे घञ्।\footnote{\textcolor{red}{भावे} (पा॰सू॰~३.३.१८) इत्यनेन।}
\textcolor{red}{मोदं करोति मोदयति}।\footnote{मोद~\arrow \textcolor{red}{तत्करोति तदाचष्टे} (धा॰पा॰ ग॰सू॰~१८७)~\arrow मोद~णिच्~\arrow मोद~इ~\arrow \textcolor{red}{णाविष्ठवत्प्राति\-पदिकस्य पुंवद्भाव\-रभाव\-टिलोप\-यणादि\-परार्थम्} (वा॰~६.४.४८)~\arrow मोद्~इ~\arrow मोदि~\arrow \textcolor{red}{सनाद्यन्ता धातवः} (पा॰सू॰~३.१.३२)~\arrow धातुसञ्ज्ञा~\arrow \textcolor{red}{शेषात्कर्तरि परस्मैपदम्} (पा॰सू॰~१.३.७८)~\arrow \textcolor{red}{वर्तमाने लट्} (पा॰सू॰~३.२.१२३)~\arrow मोदि~लट्~\arrow मोदि~तिप्~\arrow मोदि~ति~\arrow \textcolor{red}{कर्तरि शप्‌} (पा॰सू॰~३.१.६८)~\arrow मोदि~शप्~ति~\arrow मोदि~अ~ति~\arrow \textcolor{red}{सार्वधातुकार्धधातुकयोः} (पा॰सू॰~७.३.८४)~\arrow मोदे~अ~ति~\arrow \textcolor{red}{एचोऽयवायावः} (पा॰सू॰~६.१.७८)~\arrow मोदय्~अ~ति~\arrow मोदयति।} \textcolor{red}{मोदयतीति मोद्}।\footnote{मोदि~\arrow धातुसञ्ज्ञा (पूर्ववत्)~\arrow \textcolor{red}{क्विप् च} (पा॰सू॰~३.२.७६)~\arrow मोदि~क्विँप्~\arrow मोदि~व्~\arrow \textcolor{red}{वेरपृक्तस्य} (पा॰सू॰~६.१.६७)~\arrow मोदि~\arrow \textcolor{red}{णेरनिटि} (पा॰सू॰~६.४.५१)~\arrow मोद्~\arrow विभक्तिकार्यम्~\arrow मोद्~सुँ~\arrow \textcolor{red}{हल्ङ्याब्भ्यो दीर्घात्सुतिस्यपृक्तं हल्} (पा॰सू॰~६.१.६८)~\arrow मोद्~\arrow \textcolor{red}{वाऽवसाने} (पा॰सू॰~८.४.५६)~\arrow मोद्, मोत्।} \textcolor{red}{मोदिवाऽचरति मोदति}।\footnote{मोद्~\arrow \textcolor{red}{सर्वप्रातिपतिकेभ्य आचारे क्विब्वा वक्तव्यः} (वा॰~३.१.११)~\arrow मोद्~क्विँप्~\arrow मोद्~व्~\arrow \textcolor{red}{वेरपृक्तस्य} (पा॰सू॰~६.१.६७)~\arrow मोद्~\arrow \textcolor{red}{सनाद्यन्ता धातवः} (पा॰सू॰~३.१.३२)~\arrow धातु\-सञ्ज्ञा~\arrow \textcolor{red}{शेषात्कर्तरि परस्मैपदम्} (पा॰सू॰~१.३.७८)~\arrow \textcolor{red}{वर्तमाने लट्} (पा॰सू॰~३.२.१२३)~\arrow मोद्~लट्~\arrow मोद्~तिप्~\arrow मोद्~ति~\arrow \textcolor{red}{कर्तरि शप्‌} (पा॰सू॰~३.१.६८)~\arrow मोद्~शप्~ति~\arrow मोद्~अ~ति~\arrow मोदति।} ततो लिटि लकारे णल्प्रत्यये द्वित्वेऽभ्यास\-कार्ये ह्रस्वे \textcolor{red}{मुमोद}।\footnote{मोद्~\arrow \textcolor{red}{धातु\-सञ्ज्ञा} (पूर्ववत्)~\arrow \textcolor{red}{शेषात्कर्तरि परस्मैपदम्} (पा॰सू॰~१.३.७८)~\arrow \textcolor{red}{परोक्षे लिट्} (पा॰सू॰~३.२.११५)~\arrow मोद्~लिट्~\arrow मोद्~तिप्~\arrow मोद्~ति~\arrow \textcolor{red}{परस्मैपदानां णलतुसुस्थलथुस\-णल्वमाः} (पा॰सू॰~३.४.८२)~\arrow मोद्~ण~\arrow मोद्~अ~\arrow \textcolor{red}{लिटि धातोरनभ्यासस्य} (पा॰सू॰~६.१.८)~\arrow मोद्~मोद्~अ~\arrow हलादिः शेषः~\arrow मो~मोद्~अ~\arrow \textcolor{red}{ह्रस्वः} (पा॰सू॰~७.४.५९)~\arrow \textcolor{red}{एच इग्घ्रस्वादेशे} (पा॰सू॰~१.१.४८)~\arrow मु~मोद्~अ~\arrow मुमोद। मोदयितेवाऽचचारेति मुमोद। वस्तुतस्तु जनको योगिराजस्तस्य कुतो मोद इति भावः। यद्वा \textcolor{red}{अनुदात्तेत्त्व\-लक्षणमात्मने\-पदमनित्यम्} (प॰शे॰~९३.४) इत्यपि समाधानम्। \textcolor{red}{मुदँ हर्षे} (धा॰पा॰~१६)~\arrow मुद्~\arrow \textcolor{red}{अनुदात्तेत्त्व\-लक्षणमात्मने\-पदमनित्यम्} (प॰शे॰~९३.४)~\arrow \textcolor{red}{शेषात्कर्तरि परस्मैपदम्} (पा॰सू॰~१.३.७८)~\arrow \textcolor{red}{परोक्षे लिट्} (पा॰सू॰~३.२.११५)~\arrow मुद्~लिट्~\arrow मुद्~तिप्~\arrow मुद्~ति~\arrow \textcolor{red}{परस्मैपदानां णलतुसुस्थलथुस\-णल्वमाः} (पा॰सू॰~३.४.८२)~\arrow मुद्~ण~\arrow मुद्~अ~\arrow \textcolor{red}{लिटि धातोरनभ्यासस्य} (पा॰सू॰~६.१.८)~\arrow मुद्~मुद्~अ~\arrow हलादिः शेषः~\arrow मु~मुद्~अ~\arrow \textcolor{red}{पुगन्त\-लघूपधस्य च} (पा॰सू॰~७.३.८६)~\arrow मु~मोद्~अ~\arrow मुमोद।}\end{sloppypar}
\section[दीयते]{दीयते}
\centering\textcolor{blue}{तदारभ्य मया सीता विष्णोर्लक्ष्मीर्विभाव्यते।\nopagebreak\\
कथं मया राघवाय दीयते जानकी शुभा॥}\nopagebreak\\
\raggedleft{–~अ॰रा॰~१.६.६७}\\
\fontsize{14}{21}\selectfont\begin{sloppypar}\hyphenrules{nohyphenation}\justifying\noindent\hspace{10mm} अत्र कथं \textcolor{red}{मया दीयते} इति। \textcolor{red}{दीयताम्}\footnote{\textcolor{red}{डुदाञ् दाने} (धा॰पा॰~१०९१)~\arrow दा~\arrow \textcolor{red}{भावकर्मणोः} (पा॰सू॰~१.३.१३)~\arrow \textcolor{red}{लोट् च} (पा॰सू॰~३.३.१६२)~\arrow दा~लोट्~\arrow दा~त~\arrow \textcolor{red}{सार्वधातुके यक्} (पा॰सू॰~३.१.६७)~\arrow दा~यक्~त~\arrow दा~य~त~\arrow \textcolor{red}{घुमा\-स्थागापा\-जहातिसां हलि} (पा॰सू॰~६.४.६६)~\arrow दी~य~त~\arrow \textcolor{red}{टित आत्मनेपदानां टेरे} (पा॰सू॰~३.४.७९)~\arrow दी~य~ते~\arrow \textcolor{red}{आमेतः} (पा॰सू॰~३.४.९०)~\arrow दी~य~ताम्~\arrow दीयताम्।} इति \textcolor{red}{लोट् च} (पा॰सू॰~३.३.१६२) इत्यनेन प्रश्नार्थे\footnote{सम्प्रश्नार्थ इति भावः। \textcolor{red}{विधि\-निमन्‍त्रणामन्‍त्रणाधीष्‍ट\-सम्प्रश्न\-प्रार्थनेषु लिङ्} (पा॰सू॰~३.३.१६१) इत्यत्र \textcolor{red}{सम्प्रश्न} इत्युपादानेन।} लोट्। यद्वा \textcolor{red}{दीयेत}\footnote{\textcolor{red}{डुदाञ् दाने} (धा॰पा॰~१०९१)~\arrow दा~\arrow \textcolor{red}{भावकर्मणोः} (पा॰सू॰~१.३.१३)~\arrow \textcolor{red}{आशंसावचने लिङ्} (पा॰सू॰~३.३.१३४)~\arrow दा~लिङ्~\arrow दा~त~\arrow \textcolor{red}{सार्वधातुके यक्} (पा॰सू॰~३.१.६७)~\arrow दा~यक्~त~\arrow दा~य~त~\arrow \textcolor{red}{घुमा\-स्थागापा\-जहातिसां हलि} (पा॰सू॰~६.४.६६)~\arrow दी~य~त~\arrow \textcolor{red}{लिङः सीयुट्} (पा॰सू॰~३.४.१०२)~\arrow दी~य~सीयुँट्~त~\arrow दी~य~सीय्~त~\arrow \textcolor{red}{सुट् तिथोः} (पा॰सू॰~३.४.१०७)~\arrow \textcolor{red}{आद्यन्तौ टकितौ} (पा॰सू॰~१.१.४६)~\arrow दी~य~सीय्~सुँट्~त~\arrow दी~य~सीय्~स्~त~\arrow \textcolor{red}{लिङः सलोपोऽनन्त्यस्य} (पा॰सू॰~७.२.७९)~\arrow दी~य~ईय्~त~\arrow \textcolor{red}{लोपो व्योर्वलि} (पा॰सू॰~६.१.६६)~\arrow दी~य~ई~त~\arrow \textcolor{red}{आद्गुणः} (पा॰सू॰~६.१.८७)~\arrow दी~ये~त~\arrow दीयेत।} इति \textcolor{red}{आशंसावचने लिङ्} (पा॰सू॰~३.३.१३४) इत्यनेन लिङ् प्रयोक्तव्यमासीत्। किन्तु \textcolor{red}{धातु\-सम्बन्धे प्रत्ययाः} (पा॰सू॰~३.४.१) इत्यनेन लड्लकारः।\footnote{\textcolor{red}{डुदाञ् दाने} (धा॰पा॰~१०९१)~\arrow दा~\arrow \textcolor{red}{भावकर्मणोः} (पा॰सू॰~१.३.१३)~\arrow \textcolor{red}{धातुसम्बन्धे प्रत्ययाः} (पा॰सू॰~३.४.१)~\arrow दा~लट्~\arrow दा~त~\arrow \textcolor{red}{सार्वधातुके यक्} (पा॰सू॰~३.१.६७)~\arrow दा~यक्~त~\arrow दा~य~त~\arrow \textcolor{red}{घुमा\-स्थागापा\-जहातिसां हलि} (पा॰सू॰~६.४.६६)~\arrow दी~य~त~\arrow \textcolor{red}{टित आत्मनेपदानां टेरे} (पा॰सू॰~३.४.७९)~\arrow दी~य~ते~\arrow दीयते।}\end{sloppypar}
\section[त्राहि त्राहि]{त्राहि त्राहि}
\centering\textcolor{blue}{तं दृष्ट्वा भयसन्त्रस्तो राजा दशरथस्तदा।\nopagebreak\\
अर्घ्यादिपूजां विस्मृत्य त्राहि त्राहीति चाब्रवीत्॥}\nopagebreak\\
\raggedleft{–~अ॰रा॰~१.७.९}\\
\fontsize{14}{21}\selectfont\begin{sloppypar}\hyphenrules{nohyphenation}\justifying\noindent\hspace{10mm} विवाहं
कृत्वा 
परावर्तमानो दशरथः सपरिकरः परशुरामं दृष्ट्वा \textcolor{red}{त्राहि त्राहि} इति ब्रवीति। \textcolor{red}{त्रै}\-धातोस्तु (\textcolor{red}{त्रैङ् पालने} धा॰पा॰~९६५) लोटि मध्यम\-पुरुष एक\-वचने गुणेऽयादेशे \textcolor{red}{त्रायस्व} इति पाणिनीयम्।\footnote{\textcolor{red}{त्रैङ् पालने} (धा॰पा॰~९६५)~\arrow त्रै~\arrow \textcolor{red}{अनुदात्तङित आत्मने\-पदम्} (पा॰सू॰~१.३.१२)~\arrow \textcolor{red}{लोट् च} (पा॰सू॰~३.३.१६२)~\arrow त्रै~लोट्~\arrow त्रै~थास्~\arrow \textcolor{red}{कर्तरि शप्} (पा॰सू॰~३.१.६८)~\arrow त्रै~शप्~थास्~\arrow त्रै~अ~थास्~\arrow \textcolor{red}{एचोऽयवायावः} (पा॰सू॰~६.१.७८)~\arrow त्राय्~अ~थास्~\arrow \textcolor{red}{थासस्से} (पा॰सू॰~३.४.८०)~\arrow त्राय्~अ~से~\arrow \textcolor{red}{सवाभ्यां वामौ} (पा॰सू॰~३.४.९१)~\arrow त्राय्~अ~स्व~\arrow त्रायस्व।} \textcolor{red}{त्राहि} इति कथम्। आकृति\-गणत्वाददादि\-गणे \textcolor{red}{पा} (\textcolor{red}{पा रक्षणे} धा॰पा॰~१०५६) \textcolor{red}{रा} (\textcolor{red}{रा दाने} धा॰पा॰~१०५७) \textcolor{red}{प्रा} (\textcolor{red}{प्रा पूरणे} धा॰पा॰~१०६१) इत्यादिवद्रक्षणार्थः \textcolor{red}{त्रा} इति पठितव्यः। ततश्च \textcolor{red}{सेर्ह्यपिच्च} (पा॰सू॰~३.४.८७) इत्यनेन \textcolor{red}{हि} आदेशे \textcolor{red}{त्राहि} इति पाणिनीय एव।\footnote{\textcolor{red}{भूवादिष्वेतदन्तेषु दशगणीषु धातूनां पाठो निदर्शनाय तेन स्तम्भुप्रभृतयः सौत्राश्चुलुम्पादयो वाक्यकारीयाः प्रयोगसिद्धा विक्लवत्यादयश्च} (मा॰धा॰वृ॰~१०.३२८) इत्यनुसारमाकृति\-गणत्वाददादि\-गण ऊह्योऽयं त्राधातू रक्षण इति भावः। \textcolor{red}{त्रा रक्षणे} (अदादिगण ऊह्यः)~\arrow \textcolor{red}{शेषात्कर्तरि परस्मैपदम्} (पा॰सू॰~१.३.७८)~\arrow \textcolor{red}{लोट् च} (पा॰सू॰~३.३.१६२)~\arrow त्रा~लोट्~\arrow त्रा~सिप्~\arrow त्रा~सि~\arrow \textcolor{red}{कर्तरि शप्} (पा॰सू॰~३.१.६८)~\arrow त्रा~शप्~सि~\arrow \textcolor{red}{अदिप्रभृतिभ्यः शपः} (पा॰सू॰~२.४.७२)~\arrow त्रा~सि~\arrow \textcolor{red}{सेर्ह्यपिच्च} (पा॰सू॰~३.४.८७)~\arrow त्रा~हि~\arrow त्राहि।}\end{sloppypar}
\section[भूयात्]{भूयात्}
\label{sec:bhuyat1}
\centering\textcolor{blue}{जडस्य चित्समायोगाच्चित्त्वं भूयाच्चितेस्तथा।\nopagebreak\\
जडसङ्गाज्जडत्वं हि जलाग्न्योर्मेलनं यथा॥}\nopagebreak\\
\raggedleft{–~अ॰रा॰~१.७.३७}\\
\centering\textcolor{blue}{देव यद्यत्कृतं पुण्यं मया लोकजिगीषया।\nopagebreak\\
तत्सर्वं तव बाणाय भूयाद्राम नमोऽस्तु ते॥}\nopagebreak\\
\raggedleft{–~अ॰रा॰~१.७.४५}\\
\fontsize{14}{21}\selectfont\begin{sloppypar}\hyphenrules{nohyphenation}\justifying\noindent\hspace{10mm} यद्यपि सत्तार्थक\-\textcolor{red}{भू}\-धातोः (\textcolor{red}{भू सत्तायाम्} धा॰पा॰~१) लिङ्लकारे तु \textcolor{red}{भवेत्}\footnote{\textcolor{red}{भू सत्तायाम्} (धा॰पा॰~१)~\arrow भू~\arrow \textcolor{red}{शेषात्कर्तरि परस्मैपदम्} (पा॰सू॰~१.३.७८)~\arrow \textcolor{red}{विधि\-निमन्‍त्रणामन्‍त्रणाधीष्‍ट\-सम्प्रश्‍न\-प्रार्थनेषु लिङ्} (पा॰सू॰~३.३.१६१)~\arrow भू~लिङ्~\arrow भू~तिप्~\arrow भू~ति~\arrow \textcolor{red}{कर्तरि शप्} (पा॰सू॰~३.१.६८)~\arrow भू~शप्~ति~\arrow भू~अ~ति~\arrow \textcolor{red}{सार्वधातुकार्धधातुकयोः} (पा॰सू॰~७.३.८४)~\arrow भो~अ~ति~\arrow \textcolor{red}{एचोऽयवायावः} (पा॰सू॰~६.१.७८)~\arrow भव्~अ~ति~\arrow \textcolor{red}{यासुट् परस्मैपदेषूदात्तो ङिच्च} (पा॰सू॰~३.४.१०३)~\arrow \textcolor{red}{आद्यन्तौ टकितौ} (पा॰सू॰~१.१.४६)~\arrow भव्~अ~यासुँट्~ति~\arrow भव्~अ~यास्~ति~\arrow \textcolor{red}{सुट् तिथोः} (पा॰सू॰~३.४.१०७)~\arrow \textcolor{red}{आद्यन्तौ टकितौ} (पा॰सू॰~१.१.४६)~\arrow भव्~अ~यास्~सुँट्~ति~\arrow भव्~अ~यास्~स्~ति~\arrow \textcolor{red}{लिङः सलोपोऽनन्त्यस्य} (पा॰सू॰~७.२.७९)~\arrow भव्~अ~या~ति~\arrow \textcolor{red}{अतो येयः} (पा॰सू॰~७.२.८०)~\arrow भव्~अ~इय्~ति~\arrow \textcolor{red}{लोपो व्योर्वलि} (पा॰सू॰~६.१.६६)~\arrow भव्~अ~इ~ति~\arrow \textcolor{red}{आद्गुणः} (पा॰सू॰~६.१.८७)~\arrow भव्~ए~ति~\arrow \textcolor{red}{इतश्च} (पा॰सू॰~३.४.१००)~\arrow भव्~ए~त्~\arrow भवेत्।} इति प्रयोग एवं लड्लकारे च \textcolor{red}{भवति}\footnote{\textcolor{red}{भू सत्तायाम्} (धा॰पा॰~१)~\arrow भू~\arrow \textcolor{red}{शेषात्कर्तरि परस्मैपदम्} (पा॰सू॰~१.३.७८)~\arrow \textcolor{red}{वर्तमाने लट्} (पा॰सू॰~३.२.१२३)~\arrow भू~लट्~\arrow भू~तिप्~\arrow भू~ति~\arrow \textcolor{red}{कर्तरि शप्} (पा॰सू॰~३.१.६८)~\arrow भू~शप्~ति~\arrow भू~अ~ति~\arrow \textcolor{red}{सार्वधातुकार्धधातुकयोः} (पा॰सू॰~७.३.८४)~\arrow भो~अ~ति~\arrow \textcolor{red}{एचोऽयवायावः} (पा॰सू॰~६.१.७८)~\arrow भव्~अ~ति~\arrow भवति।} इति प्रयोगः पाणिनीयः। \textcolor{red}{भूयात्} इति प्रयोग आशिषि।\footnote{\textcolor{red}{भू सत्तायाम्} (धा॰पा॰~१)~\arrow भू~\arrow \textcolor{red}{शेषात्कर्तरि परस्मैपदम्} (पा॰सू॰~१.३.७८)~\arrow \textcolor{red}{आशिषि लिङ्लोटौ} (पा॰सू॰~३.३.१७३)~\arrow भू~लिङ्~\arrow भू~तिप्~\arrow भू~ति~\arrow \textcolor{red}{लिङाशिषि} (पा॰सू॰~३.४.११६)~\arrow शबभावः~\arrow \textcolor{red}{यासुट् परस्मैपदेषूदात्तो ङिच्च} (पा॰सू॰~३.४.१०३)~\arrow \textcolor{red}{आद्यन्तौ टकितौ} (पा॰सू॰~१.१.४६)~\arrow भू~यासुँट्~ति~\arrow भू~यास्~ति~\arrow \textcolor{red}{सुट् तिथोः} (पा॰सू॰~३.४.१०७)~\arrow \textcolor{red}{आद्यन्तौ टकितौ} (पा॰सू॰~१.१.४६)~\arrow भू~यास्~सुँट्~ति~\arrow भू~यास्~स्~ति~\arrow \textcolor{red}{किदाशिषि} (पा॰सू॰~३.४.१०४)~\arrow कित्त्वम्~\arrow \textcolor{red}{ग्क्ङिति च} (पा॰सू॰~१.१.५)~\arrow गुणनिषेधः~\arrow \textcolor{red}{इतश्च} (पा॰सू॰~३.४.१००)~\arrow भू~यास्~स्~त~\arrow \textcolor{red}{स्कोः संयोगाद्योरन्ते च} (पा॰सू॰~८.२.२९)~\arrow भू~या~त्~\arrow भूयात्।} \textcolor{red}{आशिषि लिङ्\-लोटौ} (पा॰सू॰~३.३.१७३) इत्यनेन। किन्तु \textcolor{red}{यु} (\textcolor{red}{यु मिश्रणेऽमिश्रणे च} धा॰पा॰~१०३३) \textcolor{red}{णु} (\textcolor{red}{णु स्तुतौ} धा॰पा॰~१०३५) इत्यादि\-वद्भूधातुरप्यदादिस्तस्यैव लिङ्लकारे \textcolor{red}{भूयात्}।\footnote{\textcolor{red}{भूवादिष्वेतदन्तेषु दशगणीषु धातूनां पाठो निदर्शनाय तेन स्तम्भुप्रभृतयः सौत्राश्चुलुम्पादयो वाक्यकारीयाः प्रयोगसिद्धा विक्लवत्यादयश्च} (मा॰धा॰वृ॰~१०.३२८) इत्यनुसारमाकृति\-गणत्वाददादि\-गण ऊह्योऽयं भूधातुः सत्तायामिति भावः। \textcolor{red}{भू सत्तायाम्} (अदादिगण ऊह्यः)~\arrow \textcolor{red}{शेषात्कर्तरि परस्मैपदम्} (पा॰सू॰~१.३.७८)~\arrow \textcolor{red}{आशंसावचने लिङ्} (पा॰सू॰~३.३.१३४)~\arrow भू~लिङ्~\arrow भू~ति~\arrow \textcolor{red}{कर्तरि शप्} (पा॰सू॰~३.१.६८)~\arrow भू~शप्~ति~\arrow \textcolor{red}{अदिप्रभृतिभ्यः शपः} (पा॰सू॰~२.४.७२)~\arrow भू~ति~\arrow \textcolor{red}{यासुट् परस्मैपदेषूदात्तो ङिच्च} (पा॰सू॰~३.४.१०३)~\arrow \textcolor{red}{आद्यन्तौ टकितौ} (पा॰सू॰~१.१.४६)~\arrow भू~यासुँट्~ति~\arrow भू~यास्~ति~\arrow \textcolor{red}{सुट् तिथोः} (पा॰सू॰~३.४.१०७)~\arrow \textcolor{red}{आद्यन्तौ टकितौ} (पा॰सू॰~१.१.४६)~\arrow भू~यास्~सुँट्~ति~\arrow भू~यास्~स्~ति~\arrow \textcolor{red}{ग्क्ङिति च} (पा॰सू॰~१.१.५)~\arrow गुणनिषेधः~\arrow \textcolor{red}{इतश्च} (पा॰सू॰~३.४.१००)~\arrow भू~यास्~स्~त्~\arrow \textcolor{red}{स्कोः संयोगाद्योरन्ते च} (पा॰सू॰~८.२.२९)~\arrow भू~या~त्~\arrow भूयात्।} \textcolor{red}{आशंसावचने लिङ्} (पा॰सू॰~३.३.१३४) इत्यनेन।\end{sloppypar}
\vspace{2mm}
\centering ॥ इति बालकाण्डीयप्रयोगाणां विमर्शः ॥\nopagebreak\\
\vspace{4mm}
\pdfbookmark[2]{अयोध्याकाण्डम्}{Chap3Part1Kanda2}
\phantomsection
\addtocontents{toc}{\protect\setcounter{tocdepth}{2}}
\addcontentsline{toc}{subsection}{अयोध्याकाण्डीयप्रयोगाणां विमर्शः}
\addtocontents{toc}{\protect\setcounter{tocdepth}{0}}
\centering ॥ अथायोध्याकाण्डीयप्रयोगाणां विमर्शः ॥\nopagebreak\\
\section[मोहयस्व]{मोहयस्व}
\centering\textcolor{blue}{अहं त्वद्भक्तभक्तानां तद्भक्तानां च किङ्करः।\nopagebreak\\
अतो मामनुगृह्णीष्व मोहयस्व न मां प्रभो॥}\nopagebreak\\
\raggedleft{–~अ॰रा॰~२.१.३०}\\
\fontsize{14}{21}\selectfont\begin{sloppypar}\hyphenrules{nohyphenation}\justifying\noindent\hspace{10mm} नारदः कथयति \textcolor{red}{मां मा मोहयस्व}। अत्र \textcolor{red}{मोहय} इति समीचीनं प्रतीयते।\footnote{\textcolor{red}{मुहँ वैचित्ये} (धा॰पा॰~११९८)~\arrow मुह्~\arrow \textcolor{red}{हेतुमति च} (पा॰सू॰~३.१.२६)~\arrow मुह्~णिच्~\arrow मुह्~इ~\arrow \textcolor{red}{पुगन्त\-लघूपधस्य च} (पा॰सू॰~७.३.८६)~\arrow मोह्~इ~\arrow मोहि~\arrow \textcolor{red}{सनाद्यन्ता धातवः} (पा॰सू॰~३.१.३२)~\arrow \textcolor{red}{शेषात्कर्तरि परस्मैपदम्} (पा॰सू॰~१.३.७८)~\arrow \textcolor{red}{लोट् च} (पा॰सू॰~३.३.१६२)~\arrow मोहि~लोट्~\arrow मोहि~सिप्~\arrow मोहि~सि~\arrow \textcolor{red}{कर्तरि शप्} (पा॰सू॰~३.१.६८)~\arrow मोहि~शप्~सि~\arrow मोहि~अ~सि~\arrow \textcolor{red}{सार्वधातुकार्धधातुकयोः} (पा॰सू॰~७.३.८४)~\arrow मोहे~अ~सि~\arrow \textcolor{red}{एचोऽयवायावः} (पा॰सू॰~६.१.७८)~\arrow मोहय्~अ~सि~\arrow \textcolor{red}{सेर्ह्यपिच्च} (पा॰सू॰~३.४.८७)~\arrow मोहय्~अ~हि~\arrow \textcolor{red}{अतो हेः} (पा॰सू॰~६.४.१०५)~\arrow मोहय्~अ~\arrow मोहय।} किन्तु \textcolor{red}{स्व} इति पृथगुपादानेन \textcolor{red}{हे स्व हे आत्मीय मां मा मोहय} इति परिहारः।\end{sloppypar}
\section[नाशयामि]{नाशयामि}
\centering\textcolor{blue}{सीतामिषेण तं दुष्टं सकुलं नाशयाम्यहम्।\nopagebreak\\
एवं रामे प्रतिज्ञाते नारदः प्रमुमोद ह॥}\nopagebreak\\
\raggedleft{–~अ॰रा॰~२.१.३९}\\
\fontsize{14}{21}\selectfont\begin{sloppypar}\hyphenrules{nohyphenation}\justifying\noindent\hspace{10mm} अत्र वर्तमान\-सामीप्ये लट्।\footnote{\textcolor{red}{णशँ अदर्शने} (धा॰पा॰~११९४)~\arrow णश्~\arrow \textcolor{red}{णो नः} (पा॰सू॰~६.१.६५)~\arrow नश्~\arrow \textcolor{red}{हेतुमति च} (पा॰सू॰~३.१.२६)~\arrow नश्~णिच्~\arrow नश्~इ~\arrow \textcolor{red}{अत उपधायाः} (पा॰सू॰~७.२.११६)~\arrow नाश्~इ~\arrow नाशि~\arrow \textcolor{red}{सनाद्यन्ता धातवः} (पा॰सू॰~३.१.३२)~\arrow धातु\-सञ्ज्ञा~\arrow \textcolor{red}{बुध\-युध\-नश\-जनेङ्प्रु\-द्रु\-स्रुभ्यो णेः} (पा॰सू॰~१.३.८६)~\arrow \textcolor{red}{वर्तमान\-सामीप्ये वर्तमानवद्वा} (पा॰सू॰~३.३.१३१)~\arrow \textcolor{red}{वर्तमाने लट्} (पा॰सू॰~३.२.१२३)~\arrow नाशि~लट्~\arrow नाशि~मिप्~\arrow नाशि~मि~\arrow \textcolor{red}{कर्तरि शप्‌} (पा॰सू॰~३.१.६८)~\arrow नाशि~शप्~मि~\arrow नाशि~अ~मि~\arrow \textcolor{red}{सार्वधातुकार्ध\-धातुकयोः} (पा॰सू॰~७.३.८४)~\arrow नाशे~अ~मि~\arrow \textcolor{red}{एचोऽयवायावः} (पा॰सू॰~६.१.७८)~\arrow नाशय्~अ~मि~\arrow \textcolor{red}{अतो दीर्घो यञि} (पा॰सू॰~७.३.१०१)~\arrow नाशय्~आ~मि~\arrow नाशयामि।} यद्वा \textcolor{red}{नाशमाचष्ट इति नाशयति}।\footnote{नाश~\arrow \textcolor{red}{तत्करोति तदाचष्टे} (धा॰पा॰ ग॰सू॰~१८७)~\arrow नाश~णिच्~\arrow नाश~इ~\arrow \textcolor{red}{णाविष्ठवत्प्राति\-पदिकस्य पुंवद्भाव\-रभाव\-टिलोप\-यणादि\-परार्थम्} (वा॰~६.४.४८)~\arrow नाश्~इ~\arrow नाशि~\arrow \textcolor{red}{सनाद्यन्ता धातवः} (पा॰सू॰~३.१.३२)~\arrow धातुसञ्ज्ञा~\arrow \textcolor{red}{शेषात्कर्तरि परस्मैपदम्} (पा॰सू॰~१.३.७८)~\arrow \textcolor{red}{वर्तमाने लट्} (पा॰सू॰~३.२.१२३)~\arrow नाशि~लट्~\arrow नाशि~तिप्~\arrow नाशि~ति~\arrow \textcolor{red}{कर्तरि शप्‌} (पा॰सू॰~३.१.६८)~\arrow नाशि~शप्~ति~\arrow नाशि~अ~ति~\arrow \textcolor{red}{सार्वधातुकार्ध\-धातुकयोः} (पा॰सू॰~७.३.८४)~\arrow नाशे~अ~ति~\arrow \textcolor{red}{एचोऽयवायावः} (पा॰सू॰~६.१.७८)~\arrow नाशय्~अ~ति~\arrow नाशयति।} \textcolor{red}{नाशयतीति नाश्}।\footnote{नाशि~\arrow धातुसञ्ज्ञा (पूर्ववत्)~\arrow \textcolor{red}{क्विप् च} (पा॰सू॰~३.२.७६)~\arrow नाशि~क्विँप्~\arrow नाशि~व्~\arrow \textcolor{red}{वेरपृक्तस्य} (पा॰सू॰~६.१.६७)~\arrow नाशि~\arrow \textcolor{red}{णेरनिटि} (पा॰सू॰~६.४.५१)~\arrow नाश्~\arrow विभक्तिकार्यम्~\arrow नाश्~सुँ~\arrow \textcolor{red}{हल्ङ्याब्भ्यो दीर्घात्सुतिस्यपृक्तं हल्} (पा॰सू॰~६.१.६८)~\arrow नाश्~\arrow \textcolor{red}{अयस्मयादीनि च्छन्दसि} (पा॰सू॰~१.४.२०)~\arrow षत्वाभावः~\arrow जश्त्वाभावः~\arrow नाश्।} क्विप्कर्तरि। \textcolor{red}{अयस्मयादीनि च्छन्दसि} (पा॰सू॰~१.४.२०) इत्यनेन छान्दस\-भत्वात्षत्वाभावो\footnote{\textcolor{red}{व्रश्चभ्रस्ज\-सृजमृज\-यजराज\-भ्राजच्छशां षः} (पा॰सू॰~८.२.३६) इत्यस्याप्रवृत्तेः।} जश्त्वाभावश्च।\footnote{\textcolor{red}{झलां जशोऽन्ते} (पा॰सू॰~८.२.३९) इत्यस्याप्रवृत्तेः।} अहमेव \textcolor{red}{नाश्} रावण\-नाशयिता \textcolor{red}{अयामि}\footnote{\textcolor{red}{कटी गतौ} (धा॰पा॰~३२०) इत्यत्रेकारप्रश्लेषपक्षे गतौ \textcolor{red}{इ}\-धातुरपि। \textcolor{red}{इ इति चतुर्थधातुवादिनाम् – अयति} (मा॰धा॰वृ॰~१.२१५)। \textcolor{red}{इ गतौ} (धा॰पा॰~३२०)~\arrow \textcolor{red}{शेषात्कर्तरि परस्मैपदम्} (पा॰सू॰~१.३.७८)~\arrow \textcolor{red}{वर्तमाने लट्} (पा॰सू॰~३.२.१२३)~\arrow इ~लट्~\arrow इ~मिप्~\arrow इ~मि~\arrow \textcolor{red}{कर्तरि शप्‌} (पा॰सू॰~३.१.६८)~\arrow इ~शप्~मि~\arrow इ~अ~मि~\arrow \textcolor{red}{सार्वधातुकार्ध\-धातुकयोः} (पा॰सू॰~७.३.८४)~\arrow ए~अ~मि~\arrow \textcolor{red}{एचोऽयवायावः} (पा॰सू॰~६.१.७८)~\arrow अय्~अ~मि~\arrow \textcolor{red}{अतो दीर्घो यञि} (पा॰सू॰~७.३.१०१)~\arrow अय्~आ~मि~\arrow अयामि। यद्वा \textcolor{red}{अयत इत्ययः}। \textcolor{red}{अयँ गतौ} (धा॰पा॰~४७४) इति धातोः \textcolor{red}{नन्दि\-ग्रहि\-पचादिभ्यो ल्युणिन्यचः} (पा॰सू॰~३.१.१३४) इत्यनेन कर्तरि पचाद्यचि विभक्तिकार्ये। ततः \textcolor{red}{अय इवाऽचरामीत्ययामि}। अय~\arrow \textcolor{red}{सर्वप्राति\-पदिकेभ्य आचारे क्विब्वा वक्तव्यः} (वा॰~३.१.११)~\arrow अय~क्विँप्~\arrow अय~व्~\arrow \textcolor{red}{वेरपृक्तस्य} (पा॰सू॰~६.१.६७)~\arrow अय~\arrow \textcolor{red}{सनाद्यन्ता धातवः} (पा॰सू॰~३.१.३२)~\arrow धातुसञ्ज्ञा~\arrow \textcolor{red}{शेषात्कर्तरि परस्मैपदम्} (पा॰सू॰~१.३.७८)~\arrow \textcolor{red}{वर्तमाने लट्} (पा॰सू॰~३.२.१२३)~\arrow अय~लट्~\arrow अय~मिप्~\arrow अय~मि~\arrow \textcolor{red}{कर्तरि शप्‌} (पा॰सू॰~३.१.६८)~\arrow अय~शप्~मि~\arrow अय~अ~मि~\arrow \textcolor{red}{अतो गुणे} (पा॰सू॰~६.१.९७)~\arrow अय~मि~\arrow \textcolor{red}{अतो दीर्घो यञि} (पा॰सू॰~७.३.१०१)~\arrow अया~मि~\arrow अयामि।} अरण्यं व्रजामि।\end{sloppypar}
\section[अभिषेक्ष्यामि]{अभिषेक्ष्यामि}
\centering\textcolor{blue}{आज्ञापयति यद्यत्त्वां मुनिस्तत्तत्समानय।\nopagebreak\\
यौवराज्येऽभिषेक्ष्यामि श्वोभूते रघुनन्दनम्॥}\nopagebreak\\
\raggedleft{–~अ॰रा॰~२.२.७}\\
\fontsize{14}{21}\selectfont\begin{sloppypar}\hyphenrules{nohyphenation}\justifying\noindent\hspace{10mm} यद्यपि \textcolor{red}{श्वोभूते} इति पद\-समभिव्याहारेण लुड्लकारः प्राप्नोति\footnote{\textcolor{red}{अनद्यतने लुट्} (पा॰सू॰~३.३.१५) इत्यनेन।} तथा च स्पष्टं लकारार्थ\-निर्णये भूषणे~–\end{sloppypar}
\centering\textcolor{red}{वर्तमाने परोक्षे श्वोभाविन्यर्थे भविष्यति।\nopagebreak\\
विध्यादौ प्रार्थनादौ च क्रमाज्ज्ञेया लडादयः॥}\nopagebreak\\
\raggedleft{–~वै॰सि॰का॰~२२}\\
\fontsize{14}{21}\selectfont\begin{sloppypar}\hyphenrules{nohyphenation}\justifying\noindent तथाऽप्यनद्यतनत्वस्याविवक्षया लृट्।\footnote{अभि \textcolor{red}{षिचँ क्षरणे} (धा॰पा॰~१४३४)~\arrow अभि~षिच्~\arrow \textcolor{red}{धात्वादेः षः सः} (पा॰सू॰~६.१.६४)~\arrow अभि~सिच्~\arrow \textcolor{red}{शेषात्कर्तरि परस्मैपदम्} (पा॰सू॰~१.३.७८)~\arrow \textcolor{red}{लृट् शेषे च} (पा॰सू॰~३.३.१३)~\arrow अभि~सिच्~लृट्~\arrow अभि~सिच्~मिप्~\arrow अभि~सिच्~मि~\arrow \textcolor{red}{स्यतासी लृलुटोः} (पा॰सू॰~३.१.३३)~\arrow अभि~सिच्~स्य~मि~\arrow \textcolor{red}{एकाच उपदेशेऽनुदात्तात्‌} (पा॰सू॰~७.२.१०)~\arrow इडागम\-निषेधः~\arrow \textcolor{red}{पुगन्त\-लघूपधस्य च} (पा॰सू॰~७.३.८६)~\arrow अभि~सेच्~स्य~मि~\arrow \textcolor{red}{चोः कुः} (पा॰सू॰~८.२.३०)~\arrow अभि~सेक्~स्य~मि~\arrow \textcolor{red}{अतो दीर्घो यञि} (पा॰सू॰~७.३.१०१)~\arrow अभि~सेक्~स्या~मि~\arrow \textcolor{red}{आदेश\-प्रत्यययोः} (पा॰सू॰~८.३.५९)~\arrow अभि~षेक्~ष्या~मि~\arrow अभिषेक्ष्यामि।}\end{sloppypar}
\section[प्रविशस्व]{प्रविशस्व}
\centering\textcolor{blue}{रामाभिषेकविघ्नार्थं यतस्व ब्रह्मवाक्यतः।\nopagebreak\\
मन्थरां प्रविशस्वादौ कैकेयीं च ततः परम्॥}\nopagebreak\\
\raggedleft{–~अ॰रा॰~२.२.४५}\\
\fontsize{14}{21}\selectfont\begin{sloppypar}\hyphenrules{nohyphenation}\justifying\noindent\hspace{10mm} अत्र देवाः सरस्वतीं प्रार्थयन्ते यत् \textcolor{red}{त्वं मन्थरां प्रविशस्व}। \textcolor{red}{प्र}\-पूर्वको \textcolor{red}{विश्‌}\-धातुः (\textcolor{red}{विशँ प्रवेशने} धा॰पा॰~१४२४) परस्मैपदी। तत्र लोड्लकारे \textcolor{red}{प्रविश} इति पाणिनीयम्।\footnote{प्र~\textcolor{red}{विशँ प्रवेशने} (धा॰पा॰~१४२४)~\arrow प्र~विश्~\arrow \textcolor{red}{शेषात्कर्तरि परस्मैपदम्} (पा॰सू॰~१.३.७८)~\arrow \textcolor{red}{लोट् च} (पा॰सू॰~३.३.१६२)~\arrow प्र~विश्~लोट्~\arrow प्र~विश्~सिप्~\arrow प्र~विश्~सि~\arrow \textcolor{red}{तुदादिभ्यः शः} (पा॰सू॰~३.१.७७)~\arrow प्र~विश्~श~सि~\arrow प्र~विश्~अ~सि~\arrow \textcolor{red}{सार्वधातुकमपित्} (पा॰सू॰~१.२.४)~\arrow ङित्त्वम्~\arrow \textcolor{red}{ग्क्ङिति च} (पा॰सू॰~१.१.५)~\arrow लघूपध\-गुण\-निषेधः~\arrow \textcolor{red}{सेर्ह्यपिच्च} (पा॰सू॰~३.४.८७)~\arrow प्र~विश्~अ~हि~\arrow \textcolor{red}{अतो हेः} (पा॰सू॰~६.४.१०५)~\arrow प्रविश।} \textcolor{red}{प्रविशस्व} इति तु \textcolor{red}{कर्तरि कर्म\-व्यतिहारे} (पा॰सू॰~१.३.१४) इत्यनेन क्रिया\-विनिमय आत्मनेपदे सति \textcolor{red}{प्रविशस्व}।\footnote{प्र~\textcolor{red}{विशँ प्रवेशने} (धा॰पा॰~१४२४)~\arrow प्र~विश्~\arrow \textcolor{red}{कर्तरि कर्मव्यतिहारे} (पा॰सू॰~१.३.१४)~\arrow \textcolor{red}{लोट् च} (पा॰सू॰~३.३.१६२)~\arrow प्र~विश्~लोट्~\arrow प्र~विश्~थास्~\arrow प्र~विश्~थास्~\arrow \textcolor{red}{तुदादिभ्यः शः} (पा॰सू॰~३.१.७७)~\arrow प्र~विश्~श~थास्~\arrow प्र~विश्~अ~थास्~\arrow \textcolor{red}{सार्वधातुकमपित्} (पा॰सू॰~१.२.४)~\arrow ङित्त्वम्~\arrow \textcolor{red}{ग्क्ङिति च} (पा॰सू॰~१.१.५)~\arrow लघूपध\-गुण\-निषेधः~\arrow \textcolor{red}{थासस्से} (पा॰सू॰~३.४.८०)~\arrow प्र~विश्~अ~से~\arrow \textcolor{red}{सवाभ्यां वामौ} (पा॰सू॰~३.४.९१)~\arrow प्र~विश्~अ~स्व~\arrow प्रविशस्व।} 
यद्वा \textcolor{red}{प्रवेशनं प्रविश्}।\footnote{स्त्रियां भावे क्विप्। प्र~\textcolor{red}{विशँ प्रवेशने} (धा॰पा॰~१४२४)~\arrow प्र~विश्~\arrow \textcolor{red}{सम्पदादिभ्‍यः क्विप्} (वा॰~३.३.१०८)~\arrow प्र~विश्~क्विँप्~\arrow प्र~विश्~व्~\arrow \textcolor{red}{वेरपृक्तस्य} (पा॰सू॰~६.१.६७)~\arrow प्र~विश्~\arrow \textcolor{red}{ग्क्ङिति च} (पा॰सू॰~१.१.५)~\arrow लघूपध\-गुण\-निषेधः~\arrow प्रविश्~\arrow विभक्ति\-कार्यम्~\arrow प्रविश्~सुँ~\arrow \textcolor{red}{हल्ङ्याब्भ्यो दीर्घात्सुतिस्यपृक्तं हल्} (पा॰सू॰~६.१.६८)~\arrow \textcolor{red}{अयस्मयादीनि च्छन्दसि} (पा॰सू॰~१.४.२०)~\arrow षत्वाभावः~\arrow जश्त्वाभावः~\arrow प्रविश्।} \textcolor{red}{अयस्मयादीनि च्छन्दसि} (पा॰सू॰~१.४.२०) इत्यनेन छान्दस\-भत्वात्षत्वाभावो\footnote{\textcolor{red}{व्रश्चभ्रस्ज\-सृजमृज\-यजराज\-भ्राजच्छशां षः} (पा॰सू॰~८.२.३६) इत्यस्याप्रवृत्तेः।} जश्त्वाभावश्च।\footnote{\textcolor{red}{झलां जशोऽन्ते} (पा॰सू॰~८.२.३९) इत्यस्याप्रवृत्तेः।}
\textcolor{red}{प्रविशमस्वेनाऽत्मभिन्नेन कपटेनाऽतिक्रामतीति प्रविशस्वयति}।\footnote{अस्वमात्मभिन्नं कपटम्। अस्वेन कपटेनातिक्रामतीति \textcolor{red}{अस्वयति}। अस्व~\arrow \textcolor{red}{तेनातिक्रामति} (धा॰पा॰ ग॰सू॰~१८८)~\arrow अस्व णिच्~\arrow अस्व इ~\arrow \textcolor{red}{णाविष्ठवत्प्राति\-पदिकस्य पुंवद्भाव\-रभाव\-टिलोप\-यणादि\-परार्थम्} (वा॰~६.४.४८)~\arrow अस्व्~इ~\arrow अस्वि~\arrow \textcolor{red}{सनाद्यन्ता धातवः} (पा॰सू॰~३.१.३२)~\arrow धातु\-सञ्ज्ञा~\arrow \textcolor{red}{शेषात्कर्तरि परस्मैपदम्} (पा॰सू॰~१.३.७८)~\arrow \textcolor{red}{वर्तमाने लट्} (पा॰सू॰~३.२.१२३)~\arrow अस्वि~लट्~\arrow अस्वि~तिप्~\arrow अस्वि~ति~\arrow \textcolor{red}{कर्तरि शप्‌} (पा॰सू॰~३.१.६८)~\arrow अस्वि~शप्~ति~\arrow अस्वि~अ~ति~\arrow \textcolor{red}{सार्वधातुकार्धधातुकयोः} (पा॰सू॰~७.३.८४)~\arrow अस्वे~अ~ति~\arrow \textcolor{red}{एचोऽयवायावः} (पा॰सू॰~६.१.७८)~\arrow अस्वय्~अ~ति~\arrow अस्वयति। \textcolor{red}{प्रविशम् अस्वयति} इति स्थिते \textcolor{red}{सह सुपा} (पा॰सू॰~२.१.४) इत्यत्र \textcolor{red}{सह} इति योगविभागात्सुबन्तस्य तिङन्तेन समासे \textcolor{red}{प्रविशस्वयति}। \textcolor{red}{पर्यभूषयत्} (वै॰सि॰कौ॰~९३९) इतिवत्। तत्रत्या बालमनोरमा च~– \textcolor{red}{‘पर्यभूषयदिति॥’ ‘सह सुपा’ इत्यत्र सहेति योगविभागात् परीति सुबन्तस्य तिङन्तेन समासः} (बा॰म॰~९३९)।} \textcolor{red}{प्रविशस्वयतीति प्रविशस्वः}।\footnote{अस्वयतीत्यस्वः। \textcolor{red}{अस्वि}\-नामधातोः \textcolor{red}{नन्दि\-ग्रहि\-पचादिभ्यो ल्युणिन्यचः} (पा॰सू॰~३.१.१३४) इत्यनेन कर्तरि पचाद्यचि। \textcolor{red}{प्रविशि अस्वः} इति \textcolor{red}{प्रविशस्वः}। \textcolor{red}{सप्तमी शौण्डैः} (पा॰सू॰~२.१.४०) इत्यत्र \textcolor{red}{सप्तमी} इति योगविभागात्समासः। यद्वा \textcolor{red}{सह सुपा} (पा॰सू॰~२.१.४) इत्यनेन सुप्सुपासमासः।} \textcolor{red}{प्रविशस्व इवाऽचरतीति प्रविशस्वति}।\footnote{प्रविशस्व~\arrow \textcolor{red}{सर्वप्रातिपतिकेभ्य आचारे क्विब्वा वक्तव्यः} (वा॰~३.१.११)~\arrow प्रविशस्व~क्विप्~\arrow प्रविशस्व~व्~\arrow \textcolor{red}{वेरपृक्तस्य} (पा॰सू॰~६.१.६७)~\arrow प्रविशस्व~\arrow \textcolor{red}{सनाद्यन्ता धातवः} (पा॰सू॰~३.१.३२)~\arrow धातु\-सञ्ज्ञा~\arrow \textcolor{red}{शेषात्कर्तरि परस्मैपदम्} (पा॰सू॰~१.३.७८)~\arrow \textcolor{red}{वर्तमाने लट्} (पा॰सू॰~३.२.१२३)~\arrow प्रविशस्व~लट्~\arrow प्रविशस्व~तिप्~\arrow प्रविशस्व~ति~\arrow \textcolor{red}{कर्तरि शप्‌} (पा॰सू॰~३.१.६८)~\arrow प्रविशस्व~शप्~ति~\arrow प्रविशस्व~अ~ति~\arrow \textcolor{red}{अतो गुणे} (पा॰सू॰~६.१.९७)~\arrow प्रविशस्व~ति~\arrow प्रविशस्वति।} इत्थमाचक्षाण\-णिजन्तात्पचाद्यच्पुनराचारक्विप्ततो लोण्मध्यमपुरुष एकवचने \textcolor{red}{प्रविशस्व}।\footnote{प्रविशस्व~\arrow धातु\-सञ्ज्ञा (पूर्ववत्)~\arrow \textcolor{red}{शेषात्कर्तरि परस्मैपदम्} (पा॰सू॰~१.३.७८)~\arrow \textcolor{red}{लोट् च} (पा॰सू॰~३.३.१६२)~\arrow प्रविशस्व~लोट्~\arrow प्रविशस्व~सिप्~\arrow प्रविशस्व~सि~\arrow \textcolor{red}{कर्तरि शप्‌} (पा॰सू॰~३.१.६८)~\arrow प्रविशस्व~शप्~सि~\arrow प्रविशस्व~अ~सि~\arrow \textcolor{red}{अतो गुणे} (पा॰सू॰~६.१.९७)~\arrow प्रविशस्व~सि~\arrow \textcolor{red}{सेर्ह्यपिच्च} (पा॰सू॰~३.४.८७)~\arrow प्रविशस्व~हि~\arrow \textcolor{red}{अतो हेः} (पा॰सू॰~६.४.१०५)~\arrow प्रविशस्व।}\end{sloppypar}
\section[चक्रे]{चक्रे}
\centering\textcolor{blue}{ततो विघ्ने समुत्पन्ने पुनरेहि दिविं शुभे।\nopagebreak\\
तथेत्युक्त्वा तथा चक्रे प्रविवेशाथ मन्थराम्॥}\nopagebreak\\
\raggedleft{–~अ॰रा॰~२.२.४६}\\
\fontsize{14}{21}\selectfont\begin{sloppypar}\hyphenrules{nohyphenation}\justifying\noindent\hspace{10mm} क्रियाफलस्य कर्तृ\-भूतायां सरस्वत्यां गामित्वादात्मनेपदम्।\footnote{\textcolor{red}{स्वरितञितः कर्त्रभिप्राये क्रियाफले} (पा॰सू॰~१.३.७२) इत्यनेन। \textcolor{red}{डुकृञ् करणे} (धा॰पा॰~१४७२)~\arrow कृ~\arrow \textcolor{red}{स्वरितञितः कर्त्रभिप्राये क्रियाफले} (पा॰सू॰~१.३.७२)~\arrow \textcolor{red}{परोक्षे लिट्} (पा॰सू॰~३.२.११५)~\arrow कृ~लिट्~\arrow कृ~इट्~\arrow कृ~इ~\arrow \textcolor{red}{लिटि धातोरनभ्यासस्य} (पा॰सू॰~६.१.८)~\arrow कृ~कृ~इ~\arrow \textcolor{red}{उरत्‌} (पा॰सू॰~७.४.६६)~\arrow \textcolor{red}{उरण् रपरः} (पा॰सू॰~१.१.५१)~\arrow कर्~कृ~ए~\arrow \textcolor{red}{हलादिः शेषः} (पा॰सू॰~७.४.६०)~\arrow क~कृ~ए~\arrow \textcolor{red}{कुहोश्चुः} (पा॰सू॰~७.४.६२)~\arrow च~कृ~ए~\arrow \textcolor{red}{असंयोगाल्लिट् कित्} (पा॰सू॰~१.२.५)~\arrow कित्त्वम्~\arrow \textcolor{red}{ग्क्ङिति च} (पा॰सू॰~१.१.५)~\arrow गुणनिषेधः~\arrow \textcolor{red}{इको यणचि} (पा॰सू॰~६.१.७७)~\arrow च~क्र्~ए~\arrow चक्रे।} क्रिया\-फलं हि लीला\-साहाय्येन भगवदानुकूल्य\-रूपम्।\end{sloppypar}
\section[विवास्यते]{विवास्यते}
\centering\textcolor{blue}{भरतो राघवस्याग्रे किङ्करो वा भविष्यति।\nopagebreak\\
विवास्यते वा नगरात्प्राणैर्वा हायतेऽचिरात्॥}\nopagebreak\\
\raggedleft{–~अ॰रा॰~२.२.६२}\\
\fontsize{14}{21}\selectfont\begin{sloppypar}\hyphenrules{nohyphenation}\justifying\noindent\hspace{10mm} अत्रापि \textcolor{red}{विवासयिष्यते}\footnote{\textcolor{red}{वसँ निवासे} (धा॰पा॰~१००५)~\arrow वस्~\arrow \textcolor{red}{हेतुमति च} (पा॰सू॰~३.१.२६)~\arrow वस्~णिच्~\arrow वस्~इ~\arrow \textcolor{red}{अत उपधायाः} (पा॰सू॰~७.२.११६)~\arrow वास्~इ~\arrow वासि~\arrow \textcolor{red}{सनाद्यन्ता धातवः} (पा॰सू॰~३.१.३२)~\arrow धातु\-सञ्ज्ञा। वि~वासि~\arrow \textcolor{red}{भावकर्मणोः} (पा॰सू॰~१.३.१३)~\arrow \textcolor{red}{लृट् शेषे च} (पा॰सू॰~३.३.१३)~\arrow वि~वासि~लृट्~\arrow वि~वासि~त~\arrow \textcolor{red}{स्यतासी लृलुटोः} (पा॰सू॰~३.१.३३)~\arrow वि~वासि~स्य~त~\arrow \textcolor{red}{आर्धधातुकस्येड्वलादेः} (पा॰सू॰~७.२.३५)~\arrow वि~वासि~इट्~स्य~त~\arrow वि~वासि~इ~स्य~त~\arrow \textcolor{red}{सार्वधातुकार्ध\-धातुकयोः} (पा॰सू॰~७.३.८४)~\arrow वि~वासे~इ~स्य~त~\arrow \textcolor{red}{एचोऽयवायावः} (पा॰सू॰~६.१.७८)~\arrow वि~वासय्~इ~स्य~त~\arrow \textcolor{red}{टित आत्मनेपदानां टेरे} (पा॰सू॰~३.४.७९)~\arrow वि~वासय्~इ~स्य~ते~\arrow \textcolor{red}{आदेश\-प्रत्यययोः} (पा॰सू॰~८.३.५९)~\arrow वि~वासय्~इ~ष्य~ते~\arrow विवासयिष्यते।} इति प्रयोक्तव्ये \textcolor{red}{विवास्यते}\footnote{वि~वासि (पूर्ववत्)~\arrow \textcolor{red}{भावकर्मणोः} (पा॰सू॰~१.३.१३)~\arrow \textcolor{red}{वर्तमान\-सामीप्ये वर्तमानवद्वा} (पा॰सू॰~३.३.१३१)~\arrow \textcolor{red}{वर्तमाने लट्} (पा॰सू॰~३.२.१२३)~\arrow वि~वासि~लट्~\arrow वि~वासि~त~\arrow \textcolor{red}{सार्वधातुके यक्} (पा॰सू॰~३.१.६७)~\arrow \textcolor{red}{आद्यन्तौ टकितौ} (पा॰सू॰~१.१.४६)~\arrow वि~वासि~यक्~त~\arrow वि~वासि~य~त~\arrow \textcolor{red}{णेरनिटि} (पा॰सू॰~६.४.५१)~\arrow वि~वास्~य~त~\arrow \textcolor{red}{टित आत्मनेपदानां टेरे} (पा॰सू॰~३.४.७९)~\arrow वि~वास्~य~ते~\arrow विवास्यते।
} इति वर्तमान\-सामीप्ये लट्।\footnote{\textcolor{red}{वर्तमान\-सामीप्ये वर्तमानवद्वा} (पा॰सू॰~३.३.१३१) इत्यनेन।}\end{sloppypar}
\section[भूयात्]{भूयात्}
\centering\textcolor{blue}{त्वय्येव तिष्ठतु चिरं न्यासभूतं ममानघ।\nopagebreak\\
यदा मेऽवसरो भूयात्तदा देहि वरद्वयम्॥}\nopagebreak\\
\raggedleft{–~अ॰रा॰~२.२.७२}\\
\fontsize{14}{21}\selectfont\begin{sloppypar}\hyphenrules{nohyphenation}\justifying\noindent\hspace{10mm} अत्र पूर्वोक्तमेव समाधानम्।\footnote{\pageref{sec:bhuyat1}तमे पृष्ठे \ref{sec:bhuyat1} \nameref{sec:bhuyat1} इति प्रयोगस्य विमर्शं पश्यन्तु।}\end{sloppypar}
\section[न जाने]{न जाने}
\centering\textcolor{blue}{एवं त्वां बुद्धिसम्पन्नां न जाने वक्रसुन्दरि।\nopagebreak\\
भरतो यदि राजा मे भविष्यति सुतः प्रियः॥}\nopagebreak\\
\raggedleft{–~अ॰रा॰~२.२.७७}\\
\fontsize{14}{21}\selectfont\begin{sloppypar}\hyphenrules{nohyphenation}\justifying\noindent\hspace{10mm} मन्थरां कैकेयी प्रशंसति यत्पूर्वं \textcolor{red}{त्वां बुद्धि\-सम्पन्नां न जाने}। न चात्र \textcolor{red}{न अजाने} इत्येव लङ्प्रयोगः।\footnote{पूर्वपक्षोऽयम्।} अट्कथं न श्रूयत इति चेत्पर\-रूपत्वात् \textcolor{red}{विनाऽपि प्रत्ययं पूर्वोत्तर\-पद\-लोपो वक्तव्यः} (वा॰~५.३.८३) इत्यनेन लोपाद्वा।\footnote{अयमपि पूर्वपक्षः।} तथाऽप्याकार\-लोपस्तु दुर्वार एव।\footnote{उत्तरपक्षोऽयम्। \textcolor{red}{श्नाभ्यस्तयोरातः} (पा॰सू॰~६.४.११२) इत्यनेनाकार\-लोपे \textcolor{red}{अजानि} इति रूपम्। \textcolor{red}{ज्ञा अवबोधने} (धा॰पा॰~१५०७)~\arrow ज्ञा~\arrow \textcolor{red}{अनुपसर्गाज्ज्ञः} (पा॰सू॰~१.३.७६)~\arrow \textcolor{red}{अनद्यतने लङ्} (पा॰सू॰~३.२.१११)~\arrow ज्ञा~लङ्~\arrow ज्ञा~इट्~\arrow ज्ञा~इ~\arrow \textcolor{red}{लुङ्लङ्लृङ्क्ष्वडुदात्तः} (पा॰सू॰~६.४.७१)~\arrow \textcolor{red}{आद्यन्तौ टकितौ} (पा॰सू॰~१.१.४६)~\arrow अट्~ज्ञा~इ~\arrow अ~ज्ञा~इ~\arrow \textcolor{red}{क्र्यादिभ्यः श्ना} (पा॰सू॰~३.१.८१)~\arrow अ~ज्ञा~श्ना~इ~\arrow अ~ज्ञा~ना~इ~\arrow \textcolor{red}{ज्ञाजनोर्जा} (पा॰सू॰~७.३.७९)~\arrow अ~जा~ना~इ~\arrow \textcolor{red}{श्नाभ्यस्तयोरातः} (पा॰सू॰~६.४.११२)~\arrow अ~जा~न्~इ~\arrow अजानि।} उच्यते। अत्र स्मयोगे लट्।\footnote{\textcolor{red}{स्म} इत्यध्याहार्यमिति भावः। \textcolor{red}{ज्ञा अवबोधने} (धा॰पा॰~१५०७)~\arrow ज्ञा~\arrow \textcolor{red}{अनुपसर्गाज्ज्ञः} (पा॰सू॰~१.३.७६)~\arrow \textcolor{red}{लट् स्मे} (पा॰सू॰~३.२.११८)~\arrow ज्ञा~लट्~\arrow ज्ञा~इट्~\arrow ज्ञा~इ~\arrow \textcolor{red}{क्र्यादिभ्यः श्ना} (पा॰सू॰~३.१.८१)~\arrow ज्ञा~श्ना~इ~\arrow ज्ञा~ना~इ~\arrow \textcolor{red}{ज्ञाजनोर्जा} (पा॰सू॰~७.३.७९)~\arrow जा~ना~इ~\arrow \textcolor{red}{श्नाभ्यस्तयोरातः} (पा॰सू॰~६.४.११२)~\arrow जा~न्~इ~\arrow \textcolor{red}{टित आत्मनेपदानां टेरे} (पा॰सू॰~३.४.७९)~\arrow जा~न्~ए~\arrow जाने।} यद्वा पुरा\-योगे लट् \textcolor{red}{पुरि लुङ् चास्मे} (पा॰सू॰~३.२.१२२) इत्यनेन।\footnote{\textcolor{red}{पुरा} इत्यध्याहार्यमिति भावः। सिद्धिः पूर्ववत्।} \textcolor{red}{पुरा त्वामीदृशीं बुद्धिमतीं न जाने}।\end{sloppypar}
\section[त्यक्ष्ये]{त्यक्ष्ये}
\centering\textcolor{blue}{प्राणांस्त्यक्ष्येऽथ वा वक्रे शयिष्ये तावदेव हि।\nopagebreak\\
निश्चयं कुरु कल्याणि कल्याणं ते भविष्यति॥}\nopagebreak\\
\raggedleft{–~अ॰रा॰~२.२.८१}\\
\fontsize{14}{21}\selectfont\begin{sloppypar}\hyphenrules{nohyphenation}\justifying\noindent\hspace{10mm} \textcolor{red}{त्यज्}\-धातुः (\textcolor{red}{त्यजँ हानौ} धा॰पा॰~९८६) आत्मनेपदी न। अत्र \textcolor{red}{इट्} प्रत्ययस्य सम्भावना कथम्। अत्रापि कर्मव्यतिहारादात्मनेपदम्।\footnote{\textcolor{red}{कर्तरि कर्मव्यतिहारे} (पा॰सू॰~१.३.१४) इत्यनेन। \textcolor{red}{त्यजँ हानौ} (धा॰पा॰~९८६)~\arrow त्यज्~\arrow \textcolor{red}{कर्तरि कर्मव्यतिहारे} (पा॰सू॰~१.३.१४)~\arrow \textcolor{red}{लृट् शेषे च} (पा॰सू॰~३.३.१३)~\arrow त्यज्~लृट्~\arrow त्यज्~इट्~\arrow त्यज्~इ~\arrow \textcolor{red}{स्यतासी लृलुटोः} (पा॰सू॰~३.१.३३)~\arrow त्यज्~स्य~इ~\arrow \textcolor{red}{चोः कुः} (पा॰सू॰~८.२.३०)~\arrow त्यक्~स्य~इ~\arrow \textcolor{red}{टित आत्मनेपदानां टेरे} (पा॰सू॰~३.४.७९)~\arrow त्यक्~स्य~ए~\arrow \textcolor{red}{अतो गुणे} (पा॰सू॰~६.१.९७)~\arrow त्यक्~स्ये~\arrow \textcolor{red}{आदेश\-प्रत्यययोः} (पा॰सू॰~८.३.५९)~\arrow तयक्~ष्ये~\arrow \textcolor{red}{टित आत्मनेपदानां टेरे} (पा॰सू॰~३.४.७९)~\arrow त्यक्ष्ये।}\end{sloppypar}
\section[शयिष्ये]{शयिष्ये}
\centering\textcolor{blue}{प्राणांस्त्यक्ष्येऽथ वा वक्रे शयिष्ये तावदेव हि।\nopagebreak\\
निश्चयं कुरु कल्याणि कल्याणं ते भविष्यति॥}\nopagebreak\\
\raggedleft{–~अ॰रा॰~२.२.८१}\\
\fontsize{14}{21}\selectfont\begin{sloppypar}\hyphenrules{nohyphenation}\justifying\noindent\hspace{10mm} न च \textcolor{red}{शीङ्‌}\-धातुः (\textcolor{red}{शीङ् स्वप्ने} धा॰पा॰~१०३२) अदादिरनिट्।\footnote{पूर्वपक्षोऽयम्।} यथा \textcolor{red}{शेष्ये} इति वाल्मीकिः।\footnote{\textcolor{red}{शेष्ये पुरस्ताच्छालाया यावन्न प्रतियास्यति} (वा॰रा॰~२.१०३.१४) इति गोविन्द\-राजीय\-टीका\-पाठः।} \textcolor{red}{शयिष्ये} इति कथम्।\footnote{अयमपि पूर्वपक्ष इति बोध्यम्।} अत्र शयनं शयः।\footnote{\textcolor{red}{एरच्} (पा॰सू॰~३.३.५६) इत्यनेन \textcolor{red}{अच्‌}\-प्रत्यये। \textcolor{red}{शीङ् स्वप्ने} (धा॰पा॰~१०३२)~\arrow शी~\arrow \textcolor{red}{एरच्} (पा॰सू॰~३.३.५६)~\arrow शी~अच्~\arrow शी~अ~\arrow \textcolor{red}{सार्वधातुकार्ध\-धातुकयोः} (पा॰सू॰~७.३.८४)~\arrow शे~अ~\arrow \textcolor{red}{एचोऽयवायावः} (पा॰सू॰~६.१.७८)~\arrow शय्~अ~\arrow शय~\arrow विभक्तिकार्यम्~\arrow शयः। शयशब्दः शय्यायामपि। \textcolor{red}{शयः शय्याहिपाणिषु} (मे॰को॰~२६.५७) इति मेदिनी। तत्र \textcolor{red}{घ}\-प्रत्ययः। \textcolor{red}{शीङ् स्वप्ने} (धा॰पा॰~१०३२)~\arrow \textcolor{red}{पुंसि संज्ञायां घः प्रायेण} (पा॰सू॰~३.३.११८)~\arrow शी~घ~\arrow शी~अ~\arrow \textcolor{red}{सार्वधातुकार्ध\-धातुकयोः} (पा॰सू॰~७.३.८४)~\arrow शे~अ~\arrow \textcolor{red}{एचोऽयवायावः} (पा॰सू॰~६.१.७८)~\arrow शय्~अ~\arrow शय~\arrow विभक्तिकार्यम्~\arrow शयः।} \textcolor{red}{शयमाचरतीति शयते}\footnote{शय~\arrow \textcolor{red}{सर्वप्राति\-पदिकेभ्य आचारे क्विब्वा वक्तव्यः} (वा॰~३.१.११)~\arrow शय~क्विँप्~\arrow शय~व्~\arrow \textcolor{red}{वेरपृक्तस्य} (पा॰सू॰~६.१.६७)~\arrow शय~\arrow \textcolor{red}{सनाद्यन्ता धातवः} (पा॰सू॰~३.१.३२)~\arrow धातुसञ्ज्ञा~\arrow \textcolor{red}{कर्तरि कर्मव्यतिहारे} (पा॰सू॰~१.३.१४)~\arrow \textcolor{red}{वर्तमाने लट्} (पा॰सू॰~३.२.१२३)~\arrow शय~लट्~\arrow शय~त~\arrow \textcolor{red}{कर्तरि शप्‌} (पा॰सू॰~३.१.६८)~\arrow शय~शप्~त~\arrow शय~अ~त~\arrow \textcolor{red}{अतो गुणे} (पा॰सू॰~६.१.९७)~\arrow शय~त~\arrow \textcolor{red}{टित आत्मनेपदानां टेरे} (पा॰सू॰~३.४.७९)~\arrow शय~ते~\arrow शयते।} पुनर्लृड्लकार उत्तम\-पुरुष एकवचने \textcolor{red}{शयिष्ये}।\footnote{शय~\arrow धातुसञ्ज्ञा (पूर्ववत्)~\arrow \textcolor{red}{कर्तरि कर्मव्यतिहारे} (पा॰सू॰~१.३.१४)~\arrow \textcolor{red}{लृट् शेषे च} (पा॰सू॰~३.३.१३)~\arrow शय~लृट्~\arrow शय~इट्~\arrow शय~इ~\arrow \textcolor{red}{स्यतासी लृलुटोः} (पा॰सू॰~३.१.३३)~\arrow शय~स्य~इ~\arrow \textcolor{red}{आर्धधातुकस्येड्वलादेः} (पा॰सू॰~७.२.३५)~\arrow शय~इट्~स्य~इ~\arrow शय~इ~स्य~ति~\arrow \textcolor{red}{अतो लोपः} (पा॰सू॰~६.४.४८)~\arrow शय्~इ~स्य~इ~\arrow \textcolor{red}{टित आत्मनेपदानां टेरे} (पा॰सू॰~३.४.७९)~\arrow शय्~इ~स्य~ए~\arrow \textcolor{red}{अतो गुणे} (पा॰सू॰~६.१.९७)~\arrow शय्~इ~स्ये~\arrow \textcolor{red}{आदेश\-प्रत्यययोः} (पा॰सू॰~८.३.५९)~\arrow शय्~इ~ष्ये~\arrow शयिष्ये।} यद्वा \textcolor{red}{शय्} इति हलन्तं क्रिया\-विशेषणम्। शयनं कुर्वतीष्टास्मीति \textcolor{red}{इष्ये}।\footnote{\textcolor{red}{इषुँ इच्छायाम्} धा॰पा॰~१३५१)~\arrow इष्~\arrow \textcolor{red}{भावकर्मणोः} (पा॰सू॰~१.३.१३)~\arrow \textcolor{red}{वर्तमाने लट्} (पा॰सू॰~३.२.१२३)~\arrow इष्~लट्~\arrow इष्~इट्~\arrow इष्~इ~\arrow \textcolor{red}{सार्वधातुके यक्} (पा॰सू॰~३.१.६७)~\arrow \textcolor{red}{आद्यन्तौ टकितौ} (पा॰सू॰~१.१.४६) इष्~यक्~इ~\arrow इष्~य~इ~\arrow \textcolor{red}{ग्क्ङिति च} (पा॰सू॰~१.१.५)~\arrow लघूपध\-गुण\-निषेधः~\arrow इष्~य~इ~\arrow \textcolor{red}{टित आत्मनेपदानां टेरे} (पा॰सू॰~३.४.७९)~\arrow इष्~य~ए~\arrow \textcolor{red}{अतो गुणे} (पा॰सू॰~६.१.९७)~\arrow इष्~ये~\arrow इष्ये।} \textcolor{red}{इषु}\-धातोः (\textcolor{red}{इषुँ इच्छायाम्} धा॰पा॰~१३५१) कर्म\-वाच्ये वर्तमान\-काल उत्तम\-पुरुष एक\-वचने रूपम्।\footnote{पूर्वपक्षं मत्वैवैते द्वे समाधाने उक्ते।} वस्तुतस्तु \textcolor{red}{शीङ्‌}\-धातुः (\textcolor{red}{शीङ् स्वप्ने} धा॰पा॰~१०३२) सेडेव। यथा तत्रैव \textcolor{red}{शयित्वा}।\footnote{\textcolor{red}{शयित्वा पुरुषव्याघ्रः कथं शेते महीतले} (वा॰रा॰~२.८८.४)।} \textcolor{red}{शेष्ये} इत्यत्र त्विडभावोऽनित्यत्वात्।\footnote{\textcolor{red}{आगम\-शास्त्रमनित्यम्} (प॰शे॰~९३.२)। \textcolor{red}{शेष्ये} (वा॰रा॰~२.१०३.१४) इत्यत्र भूषण\-कारा गोविन्द\-राजाश्च~– \textcolor{red}{शेष्ये शयिष्ये। इडभाव आर्षः} (वा॰रा॰ भू॰टी॰~२.१०३.१४)।}\end{sloppypar}
\section[उपायाति]{उपायाति}
\centering\textcolor{blue}{हसन्ती मामुपायाति सा किं नैवाद्य दृश्यते।\nopagebreak\\
इत्यात्मन्येव सञ्चिन्त्य मनसाऽतिविदूयता॥}\nopagebreak\\
\raggedleft{–~अ॰रा॰~२.३.३}\\
\fontsize{14}{21}\selectfont\begin{sloppypar}\hyphenrules{nohyphenation}\justifying\noindent\hspace{10mm} \textcolor{red}{उपायात्}\footnote{उप~आङ्~\textcolor{red}{या प्रापणे} (धा॰पा॰~१०४९)~\arrow उप~आ~या~\arrow \textcolor{red}{शेषात्कर्तरि परस्मैपदम्} (पा॰सू॰~१.३.७८)~\arrow \textcolor{red}{अनद्यतने लङ्} (पा॰सू॰~३.२.१११)~\arrow उप~आ~या~लङ्~\arrow उप~आ~या~तिप्~\arrow उप~आ~या~ति~\arrow \textcolor{red}{लुङ्लङ्लृङ्क्ष्वडुदात्तः} (पा॰सू॰~६.४.७१)~\arrow \textcolor{red}{आद्यन्तौ टकितौ} (पा॰सू॰~१.१.४६)~\arrow उप~आ~अट्~या~ति~\arrow उप~आ~अ~या~ति~\arrow \textcolor{red}{कर्तरि शप्‌} (पा॰सू॰~३.१.६८)~\arrow उप~आ~अ~या~शप्~ति~\arrow \textcolor{red}{अदिप्रभृतिभ्यः शपः} (पा॰सू॰~२.४.७२)~\arrow उप~आ~अ~या~ति~\arrow \textcolor{red}{इतश्च} (पा॰सू॰~३.४.१००)~\arrow उप~आ~अ~या~त्~\arrow \textcolor{red}{अकः सवर्णे दीर्घः} (पा॰सू॰~६.१.१०१)~\arrow उपा~अ~या~त्~\arrow \textcolor{red}{अकः सवर्णे दीर्घः} (पा॰सू॰~६.१.१०१)~\arrow उपा~या~त~\arrow उपायात्।} इति प्रयोक्तव्ये \textcolor{red}{उपायाति} इति हि \textcolor{red}{स्म}\-शब्द\-योगे।\footnote{\textcolor{red}{स्म} इत्यध्याहार्यमिति भावः।} तथा च \textcolor{red}{लट् स्मे} (पा॰सू॰~३.२.११८) इत्यनेन लड्लकारः।\footnote{उप~आङ्~\textcolor{red}{या प्रापणे} (धा॰पा॰~१०४९)~\arrow उप~आ~या~\arrow \textcolor{red}{शेषात्कर्तरि परस्मैपदम्} (पा॰सू॰~१.३.७८)~\arrow \textcolor{red}{लट् स्मे} (पा॰सू॰~३.२.११८)~\arrow उप~आ~या~लट्~\arrow उप~आ~या~तिप्~\arrow उप~आ~या~ति~\arrow \textcolor{red}{कर्तरि शप्‌} (पा॰सू॰~३.१.६८)~\arrow उप~आ~या~शप्~ति~\arrow \textcolor{red}{अदिप्रभृतिभ्यः शपः} (पा॰सू॰~२.४.७२)~\arrow उप~आ~या~ति~\arrow \textcolor{red}{अकः सवर्णे दीर्घः} (पा॰सू॰~६.१.१०१)~\arrow उपा~या~ति~\arrow उपायाति।} लोपत्वात् \textcolor{red}{स्म} इत्यस्य श्रवणं न।\footnote{\textcolor{red}{विनाऽपि प्रत्ययं पूर्वोत्तर\-पद\-लोपो वक्तव्यः} (वा॰~५.३.८३) इत्यनेन।}\end{sloppypar}
\section[विद्महे]{विद्महे}
\centering\textcolor{blue}{ता ऊचुः क्रोधभवनं प्रविष्टा नैव विद्महे।\nopagebreak\\
कारणं तत्र देव त्वं गच्छ निश्चेतुमर्हसि॥}\nopagebreak\\
\raggedleft{–~अ॰रा॰~२.३.५}\\
\fontsize{14}{21}\selectfont\begin{sloppypar}\hyphenrules{nohyphenation}\justifying\noindent\hspace{10mm} \textcolor{red}{विद्महे} इत्यत्र \textcolor{red}{हे} इति पृथक्पदम्। \textcolor{red}{हे देव कारणं नैव विद्म} इत्यन्वयः करणीयः। किमिदं \textcolor{red}{विद्म}। न चात्र \textcolor{red}{वेदितुं वयं न शक्नुमह} इत्यर्थे \textcolor{red}{शकि लिङ् च} (पा॰सू॰~३.३.१७२) इत्यनेन यद्वा \textcolor{red}{कारणं वेदितुं वयं नार्हाः} इत्यर्थे \textcolor{red}{अर्हे कृत्यतृचश्च} (पा॰सू॰~३.३.१६९) इत्यनेन लिङ् लोड्वा।\footnote{\textcolor{red}{विद्म} इत्यत्राडभावान्न लुङ्लङ्लृङो विसर्गाभावान्न लड्द्वित्वाभावान्न लिट्स्यतास्यभावाच्च न लृलुटावित्याशङ्क्य पूर्वपक्षोऽयम्।} तत्र \textcolor{red}{विद्याम}\footnote{\textcolor{red}{विदँ ज्ञाने} (धा॰पा॰~१०६४)~\arrow विद्~\arrow \textcolor{red}{शेषात्कर्तरि परस्मैपदम्} (पा॰सू॰~१.३.७८)~\arrow \textcolor{red}{शकि लिङ् च} (पा॰सू॰~३.३.१७२)~\arrow विद्~लिङ्~\arrow विद्~मस्~\arrow \textcolor{red}{कर्तरि शप्‌} (पा॰सू॰~३.१.६८)~\arrow विद्~शप्~मस्~\arrow \textcolor{red}{अदिप्रभृतिभ्यः शपः} (पा॰सू॰~२.४.७२)~\arrow विद्~मस्~\arrow \textcolor{red}{यासुट् परस्मै\-पदेषूदात्तो ङिच्च} (पा॰सू॰~३.४.१०३)~\arrow विद्~यासुँट्~मस्~\arrow विद्~यास्~मस्~\arrow \textcolor{red}{ग्क्ङिति च} (पा॰सू॰~१.१.५)~\arrow पुगन्त\-लघूपध\-गुण\-निषेधः~\arrow \textcolor{red}{लिङः सलोपोऽनन्त्यस्य} (पा॰सू॰~७.२.७९)~\arrow विद्~या~मस्~\arrow नित्यं ङितः~\arrow विद्~या~म~\arrow विद्याम।} \textcolor{red}{वेदाम}\footnote{\textcolor{red}{विदँ ज्ञाने} (धा॰पा॰~१०६४)~\arrow विद्~\arrow \textcolor{red}{शेषात्कर्तरि परस्मैपदम्} (पा॰सू॰~१.३.७८)~\arrow \textcolor{red}{अर्हे कृत्यतृचश्च} (पा॰सू॰~३.३.१६९)~\arrow विद्~लोट्~\arrow विद्~मस्~\arrow \textcolor{red}{कर्तरि शप्‌} (पा॰सू॰~३.१.६८)~\arrow विद्~शप्~मस्~\arrow \textcolor{red}{अदिप्रभृतिभ्यः शपः} (पा॰सू॰~२.४.७२)~\arrow विद्~मस्~\arrow \textcolor{red}{आडुत्तमस्य पिच्च} (पा॰सू॰~३.४.९२)~\arrow विद्~आट्~मस्~\arrow विद्~आ~मस्~\arrow \textcolor{red}{पुगन्त\-लघूपधस्य च} (पा॰सू॰~७.३.८६)~\arrow वेद्~आ~मस्~\arrow \textcolor{red}{लोटो लङ्वत्‌} (पा॰सू॰~३.४.८५)~\arrow ङिद्वत्त्वम्~\arrow नित्यं ङितः~\arrow वेद्~आ~म~\arrow वेदाम।} इति रूपे। अत्र \textcolor{red}{विदो लटो वा} (पा॰सू॰~३.४.८३) इत्यनेन लटि मसो मादेशे \textcolor{red}{विद्म}।\footnote{\textcolor{red}{हे} इति पृथक्पदमिति भावः। \textcolor{red}{विदँ ज्ञाने} (धा॰पा॰~१०६४)~\arrow विद्~\arrow \textcolor{red}{शेषात्कर्तरि परस्मैपदम्} (पा॰सू॰~१.३.७८)~\arrow \textcolor{red}{वर्तमाने लट्} (पा॰सू॰~३.२.१२३)~\arrow विद्~लट्~\arrow विद्~मस्~\arrow \textcolor{red}{विदो लटो वा} (पा॰सू॰~३.४.८३)~\arrow विद्~म~\arrow \textcolor{red}{कर्तरि शप्‌} (पा॰सू॰~३.१.६८)~\arrow विद्~शप्~म~\arrow \textcolor{red}{अदिप्रभृतिभ्यः शपः} (पा॰सू॰~२.४.७२)~\arrow विद्~म~\arrow विद्म।} यद्वा कर्म\-व्यतिहार आत्मनेपदम्।\footnote{\textcolor{red}{कर्तरि कर्मव्यतिहारे} (पा॰सू॰~१.३.१४) इत्यनेन।} ततो लड्लकार उत्तमपुरुषे बहुवचने \textcolor{red}{महिङ्} प्रत्ययः।\footnote{\textcolor{red}{विदँ ज्ञाने} (धा॰पा॰~१०६४)~\arrow विद्~\arrow \textcolor{red}{कर्तरि कर्मव्यतिहारे} (पा॰सू॰~१.३.१४)~\arrow \textcolor{red}{वर्तमाने लट्} (पा॰सू॰~३.२.१२३)~\arrow विद्~लट्~\arrow विद्~महिङ्~\arrow विद्~महि~\arrow \textcolor{red}{कर्तरि शप्‌} (पा॰सू॰~३.१.६८)~\arrow विद्~शप्~महि~\arrow \textcolor{red}{अदिप्रभृतिभ्यः शपः} (पा॰सू॰~२.४.७२)~\arrow विद्~महि~\arrow \textcolor{red}{टित आत्मनेपदानां टेरे} (पा॰सू॰~३.४.७९)~\arrow विद्~महे~\arrow विद्महे।}
\end{sloppypar}
\section[वधिष्यामि]{वधिष्यामि}
\label{sec:vadhisyami}
\centering\textcolor{blue}{ब्रूहि किं वा वधिष्यामि वधार्हो वा विमोक्ष्यसे।\nopagebreak\\
किमत्र बहुनोक्तेन प्राणान्दास्यामि ते प्रिये॥}\nopagebreak\\
\raggedleft{–~अ॰रा॰~२.३.१३}\\
\fontsize{14}{21}\selectfont\begin{sloppypar}\hyphenrules{nohyphenation}\justifying\noindent\hspace{10mm} अत्र महाराजो दशरथः \textcolor{red}{हनिष्यामि}\footnote{\textcolor{red}{हनँ हिंसागत्योः} (धा॰पा॰~१०१२)~\arrow हन्~\arrow \textcolor{red}{शेषात्कर्तरि परस्मैपदम्} (पा॰सू॰~१.३.७८)~\arrow \textcolor{red}{लृट् शेषे च} (पा॰सू॰~३.३.१३)~\arrow हन्~लृट्~\arrow हन्~मिप्~\arrow हन्~मि~\arrow \textcolor{red}{स्यतासी लृलुटोः} (पा॰सू॰~३.१.३३)~\arrow हन्~स्य~मि~\arrow \textcolor{red}{ऋद्धनोः स्ये} (पा॰सू॰~७.२.७०)~\arrow हन्~इट्~स्य~मि~\arrow हन्~इ~स्य~मि~\arrow \textcolor{red}{अतो दीर्घो यञि} (पा॰सू॰~७.३.१०१)~\arrow हन्~इ~स्या~मि~\arrow \textcolor{red}{आदेश\-प्रत्यययोः} (पा॰सू॰~८.३.५९)~\arrow हन्~इ~ष्या~मि~\arrow हनिष्यामि।} इत्यर्थे \textcolor{red}{वधिष्यामि} इति प्रयोगं करोति। \textcolor{red}{वध्‌}\-धातुश्चुरादिः।\footnote{\textcolor{red}{बहुलमेतन्निदर्शनम्} (धा॰पा॰ ग॰सू॰~१९३८) \textcolor{red}{आकृतिगणोऽयम्} (धा॰पा॰ ग॰सू॰~१९९२) \textcolor{red}{भूवादिष्वेतदन्तेषु दशगणीषु धातूनां पाठो निदर्शनाय तेन स्तम्भुप्रभृतयः सौत्राश्चुलुम्पादयो वाक्यकारीयाः प्रयोगसिद्धा विक्लवत्यादयश्च} (मा॰धा॰वृ॰~१०.३२८) इत्यनुसारमाकृति\-गणत्वाच्चुरादि\-गण ऊह्योऽयमाधृषीयो धातुर्हिंसायाम्।} ततो णिजभावे\footnote{\textcolor{red}{आ धृषाद्वा} (धा॰पा॰ ग॰सू॰~१८०६) इत्यनेन वैकल्पिको णिच्। तेनात्र णिजभावः।} \textcolor{red}{वधिष्यामि} इति पाणिनीयः।\footnote{वध्~\arrow \textcolor{red}{आ धृषाद्वा} (धा॰पा॰ ग॰सू॰~१८०६)~\arrow णिजभावः~\arrow \textcolor{red}{शेषात्कर्तरि परस्मैपदम्} (पा॰सू॰~१.३.७८)~\arrow \textcolor{red}{लृट् शेषे च} (पा॰सू॰~३.३.१३)~\arrow वध्~लृट्~\arrow वध्~मिप्~\arrow वध्~मि~\arrow \textcolor{red}{स्यतासी लृलुटोः} (पा॰सू॰~३.१.३३)~\arrow वध्~स्य~मि~\arrow \textcolor{red}{आर्धधातुकस्येड्वलादेः} (पा॰सू॰~७.२.३५)~\arrow वध्~इट्~स्य~मि~\arrow वध्~इ~स्य~मि~\arrow \textcolor{red}{अतो दीर्घो यञि} (पा॰सू॰~७.३.१०१)~\arrow वध्~इ~स्या~मि~\arrow \textcolor{red}{आदेश\-प्रत्यययोः} (पा॰सू॰~८.३.५९)~\arrow वध्~इ~ष्या~मि~\arrow वधिष्यामि।} यद्वा \textcolor{red}{वधमाचरतीति वधति}\footnote{वध~\arrow \textcolor{red}{सर्वप्राति\-पदिकेभ्य आचारे क्विब्वा वक्तव्यः} (वा॰~३.१.११)~\arrow वध~क्विँप्~\arrow वध~व्~\arrow \textcolor{red}{वेरपृक्तस्य} (पा॰सू॰~६.१.६७)~\arrow वध~\arrow \textcolor{red}{सनाद्यन्ता धातवः} (पा॰सू॰~३.१.३२)~\arrow धातुसञ्ज्ञा~\arrow \textcolor{red}{शेषात्कर्तरि परस्मैपदम्} (पा॰सू॰~१.३.७८)~\arrow \textcolor{red}{वर्तमाने लट्} (पा॰सू॰~३.२.१२३)~\arrow वध~लट्~\arrow वध~तिप्~\arrow वध~ति~\arrow \textcolor{red}{कर्तरि शप्‌} (पा॰सू॰~३.१.६८)~\arrow वध~शप्~ति~\arrow वध~अ~ति~\arrow \textcolor{red}{अतो गुणे} (पा॰सू॰~६.१.९७)~\arrow वध~ति~\arrow वधति।} इति विग्रह आचार\-क्विबन्ताल्लृड्लकारे \textcolor{red}{वधिष्यामि}।\footnote{वध~\arrow धातुसञ्ज्ञा (पूर्ववत्)~\arrow \textcolor{red}{शेषात्कर्तरि परस्मैपदम्} (पा॰सू॰~१.३.७८)~\arrow \textcolor{red}{लृट् शेषे च} (पा॰सू॰~३.३.१३)~\arrow वध~लृट्~\arrow वध~मिप्~\arrow वध~मि~\arrow \textcolor{red}{स्यतासी लृलुटोः} (पा॰सू॰~३.१.३३)~\arrow वध~स्य~मि~\arrow \textcolor{red}{आर्धधातुकस्येड्वलादेः} (पा॰सू॰~७.२.३५)~\arrow वध~इट्~स्य~मि~\arrow वध~इ~स्य~मि~\arrow \textcolor{red}{अतो लोपः} (पा॰सू॰~६.४.४८)~\arrow वध्~इ~स्य~मि~\arrow \textcolor{red}{अतो दीर्घो यञि} (पा॰सू॰~७.३.१०१)~\arrow वध्~इ~स्या~मि~\arrow \textcolor{red}{आदेश\-प्रत्यययोः} (पा॰सू॰~८.३.५९)~\arrow वध्~इ~ष्या~मि~\arrow वधिष्यामि।} इति नापाणिनीयः।\footnote{\pageref{sec:vadhayisyati}तमे पृष्ठे \ref{sec:vadhayisyati} \nameref{sec:vadhayisyati} इति प्रयोगस्य विमर्शमपि पश्यन्तु।}\end{sloppypar}
\section[मरिष्ये]{मरिष्ये}
\centering\textcolor{blue}{वनं न गच्छेद्यदि रामचन्द्रः प्रभातकालेऽजिनचीरयुक्तः।\nopagebreak\\
उद्बन्धनं वा विषभक्षणं वा कृत्वा मरिष्ये पुरतस्तवाहम्॥}\nopagebreak\\
\raggedleft{–~अ॰रा॰~२.३.३१}\\
\fontsize{14}{21}\selectfont\begin{sloppypar}\hyphenrules{nohyphenation}\justifying\noindent\hspace{10mm} कैकेयी कथयति यद् \textcolor{red}{यदि राघवो वनं न गमिष्यति तदाऽहं मरिष्यामि}।\footnote{\textcolor{red}{मृङ् प्राणत्यागे} (धा॰पा॰~१४०३)~\arrow मृ~\arrow \textcolor{red}{शेषात्कर्तरि परस्मैपदम्} (पा॰सू॰~१.३.७८)~\arrow \textcolor{red}{लृट् शेषे च} (पा॰सू॰~३.३.१३)~\arrow मृ~लृट्~\arrow मृ~मिप्~\arrow मृ~मि~\arrow \textcolor{red}{स्यतासी लृलुटोः} (पा॰सू॰~३.१.३३)~\arrow मृ~स्य~मि~\arrow \textcolor{red}{ऋद्धनोः स्ये} (पा॰सू॰~७.२.७०)~\arrow \textcolor{red}{आद्यन्तौ टकितौ} (पा॰सू॰~१.१.४६)~\arrow मृ~इट्~स्य~मि~\arrow मृ~इ~स्य~मि~\arrow \textcolor{red}{सार्वधातुकार्ध\-धातुकयोः} (पा॰सू॰~७.३.८४)~\arrow म~इ~स्य~मि~\arrow \textcolor{red}{उरण् रपरः} (पा॰सू॰~१.१.५१)~\arrow मर्~इ~स्य~मि~\arrow \textcolor{red}{अतो दीर्घो यञि} (पा॰सू॰~७.३.१०१)~\arrow मर्~इ~स्या~मि~\arrow \textcolor{red}{आदेश\-प्रत्यययोः} (पा॰सू॰~८.३.५९)~\arrow मर्~इ~ष्या~मि~\arrow मरिष्यामि।} अत्र म्रियतेर्लुङ्लिङोः शिति चैवाऽत्मनेपद\-विधानात् \textcolor{red}{मरिष्ये} इति कथम्। \textcolor{red}{म्रियतेर्लुङ्लिङोश्च} (पा॰सू॰~१.३.६१) इति हि सूत्रं लृटि नैवात्मनेपदं करोतीति चेत्। \textcolor{red}{मरणं मरः}।\footnote{\textcolor{red}{कृत्यल्युटो बहुलम्} (पा॰सू॰~३.३.११३) इत्यनेन \textcolor{red}{मृ}\-धातोः \textcolor{red}{अप्‌}\-प्रत्ययः। \textcolor{red}{मरणं मरः। ‘कृत्यल्युटो बहुलम्’ (पा॰सू॰~३.३.११३) इत्यप्। तेनायमपि पूर्ववदाद्युदात्तः} (का॰वि॰प॰~६.२.११४) इति जिनेन्द्र\-बुद्धिः। यद्वा \textcolor{red}{नञो जर\-मर\-मित्र\-मृताः} (पा॰सू॰~६.२.११६) इत्यत्र निपातनात्सिद्धम्। \textcolor{red}{जरणं जरः ‘ॠदोरप्’ (पा॰सू॰~३.३.५७)। मरणं मरः। अमरम्। अस्मादेव निपातनादप्} (त॰बो॰~३८५०, ६.२.११६) इति ज्ञानेन्द्र\-सरस्वती च।} \textcolor{red}{मरमाचरतीति मरते} आचार\-क्विबन्तात्कर्म\-व्यतिहार आत्मनेपदम्।\footnote{मर~\arrow \textcolor{red}{सर्वप्राति\-पदिकेभ्य आचारे क्विब्वा वक्तव्यः} (वा॰~३.१.११)~\arrow मर~क्विँप्~\arrow मर~व्~\arrow \textcolor{red}{वेरपृक्तस्य} (पा॰सू॰~६.१.६७)~\arrow मर~\arrow \textcolor{red}{सनाद्यन्ता धातवः} (पा॰सू॰~३.१.३२)~\arrow धातुसञ्ज्ञा~\arrow \textcolor{red}{कर्तरि कर्मव्यतिहारे} (पा॰सू॰~१.३.१४)~\arrow \textcolor{red}{वर्तमाने लट्} (पा॰सू॰~३.२.१२३)~\arrow मर~लट्~\arrow मर~त~\arrow \textcolor{red}{कर्तरि शप्‌} (पा॰सू॰~३.१.६८)~\arrow मर~शप्~त~\arrow मर~अ~त~\arrow \textcolor{red}{अतो गुणे} (पा॰सू॰~६.१.९७)~\arrow मर~त~\arrow \textcolor{red}{टित आत्मनेपदानां टेरे} (पा॰सू॰~३.४.७९)~\arrow मर~ते~\arrow मरते।} लृड्लकार उत्तम\-पुरुष एक\-वचने \textcolor{red}{मरिष्ये}।\footnote{मर~\arrow धातुसञ्ज्ञा (पूर्ववत्)~\arrow \textcolor{red}{कर्तरि कर्मव्यतिहारे} (पा॰सू॰~१.३.१४)~\arrow \textcolor{red}{लृट् शेषे च} (पा॰सू॰~३.३.१३)~\arrow मर~लृट्~\arrow मर~इट्~\arrow मर~इ~\arrow \textcolor{red}{स्यतासी लृलुटोः} (पा॰सू॰~३.१.३३)~\arrow मर~स्य~इ~\arrow \textcolor{red}{आर्धधातुकस्येड्वलादेः} (पा॰सू॰~७.२.३५)~\arrow मर~इट्~स्य~इ~\arrow मर~इ~स्य~इ~\arrow \textcolor{red}{अतो लोपः} (पा॰सू॰~६.४.४८)~\arrow मर्~इ~स्य~इ~\arrow \textcolor{red}{टित आत्मनेपदानां टेरे} (पा॰सू॰~३.४.७९)~\arrow मर्~इ~स्य~ए~\arrow \textcolor{red}{अतो गुणे} (पा॰सू॰~६.१.९७)~\arrow मर्~इ~स्ये~\arrow \textcolor{red}{आदेश\-प्रत्यययोः} (पा॰सू॰~८.३.५९)~\arrow मर्~इ~ष्ये~\arrow मरिष्ये।} यद्वा \textcolor{red}{मरं मरणं गच्छतीति मरिष्यति}\footnote{\textcolor{red}{इषँ गतौ} (धा॰पा॰~११२७)~\arrow इष्~\arrow \textcolor{red}{शेषात्कर्तरि परस्मैपदम्} (पा॰सू॰~१.३.७८)~\arrow \textcolor{red}{वर्तमान\-सामीप्ये वर्तमानवद्वा} (पा॰सू॰~३.३.१३१)~\arrow \textcolor{red}{वर्तमाने लट्} (पा॰सू॰~३.२.१२३)~\arrow इष्~लट्~\arrow इष्~तिप्~\arrow इष्~ति~\arrow \textcolor{red}{दिवादिभ्यः श्यन्} (पा॰सू॰~३.१.६९)~\arrow इष्~शयन्~ति~\arrow इष्~य~ति~\arrow इष्यति। \textcolor{red}{मरम् इष्यति} इति स्थिते \textcolor{red}{शकन्ध्वादिषु पर\-रूपं वाच्यम्} (वा. ६.१.९१) इत्यनेन पररूपे \textcolor{red}{मरिष्यति}।} तस्यैव कर्मव्यतिहार आत्मनेपदत्वात्\footnote{\textcolor{red}{कर्तरि कर्म\-व्यतिहारे} (पा॰सू॰~१.३.१४) इत्यनेन।} \textcolor{red}{मरिष्ये} इति लड्लकार एवोत्तम\-पुरुष एक\-वचने।\footnote{\textcolor{red}{इषँ गतौ} (धा॰पा॰~११२७)~\arrow इष्~\arrow \textcolor{red}{कर्तरि कर्मव्यतिहारे} (पा॰सू॰~१.३.१४)~\arrow \textcolor{red}{वर्तमान\-सामीप्ये वर्तमानवद्वा} (पा॰सू॰~३.३.१३१)~\arrow \textcolor{red}{वर्तमाने लट्} (पा॰सू॰~३.२.१२३)~\arrow इष्~लट्~\arrow इष्~इट्~\arrow इष्~इ~\arrow \textcolor{red}{दिवादिभ्यः श्यन्} (पा॰सू॰~३.१.६९)~\arrow इष्~शयन्~इ~\arrow इष्~य~इ~\arrow \textcolor{red}{टित आत्मनेपदानां टेरे} (पा॰सू॰~३.४.७९)~\arrow इष्~य~ए~\arrow \textcolor{red}{अतो गुणे} (पा॰सू॰~६.१.९७)~\arrow इष्~ये~\arrow इष्ये। \textcolor{red}{मरम् इष्ये} इति स्थिते \textcolor{red}{शकन्ध्वादिषु पर\-रूपं वाच्यम्} (वा. ६.१.९१) इत्यनेन पररूपे \textcolor{red}{मरिष्ये}।}\end{sloppypar}
\section[द्रक्ष्यामहे]{द्रक्ष्यामहे}
\centering\textcolor{blue}{स्त्रियो बालाश्च वृद्धाश्च रात्रौ निद्रां न लेभिरे।\nopagebreak\\
कदा द्रक्ष्यामहे रामं पीतकौशेयवाससम्॥}\nopagebreak\\
\raggedleft{–~अ॰रा॰~२.३.३८}\\
\fontsize{14}{21}\selectfont\begin{sloppypar}\hyphenrules{nohyphenation}\justifying\noindent\hspace{10mm} अत्र कर्म\-व्यतिहार आत्मनेपदम्।\footnote{\textcolor{red}{कर्तरि कर्मव्यतिहारे} (पा॰सू॰~१.३.१४) इत्यनेन।} ततो लृड्लकारे \textcolor{red}{महिङ्} प्रत्यये दीर्घ एत्वे \textcolor{red}{द्रक्ष्यामहे}।\footnote{\textcolor{red}{दृशिँर प्रेक्षणे} (धा॰पा॰~९८८)~\arrow दृश्~\arrow \textcolor{red}{कर्तरि कर्मव्यतिहारे} (पा॰सू॰~१.३.१४)~\arrow \textcolor{red}{लृट् शेषे च} (पा॰सू॰~३.३.१३)~\arrow दृश्~लृट्~\arrow दृश्~महिङ्~\arrow दृश्~महि~\arrow \textcolor{red}{स्यतासी लृलुटोः} (पा॰सू॰~३.१.३३)~\arrow दृश्~स्य~महि~\arrow \textcolor{red}{सृजि\-दृशोर्झल्यमकिति} (पा॰सू॰~६.१.५८)~\arrow \textcolor{red}{मिदचोऽन्त्यात्परः} (पा॰सू॰~१.१.४७)~\arrow दृ~अम्~श्~स्य~महि~\arrow दृ~अ~श्~स्य~महि~\arrow \textcolor{red}{इको यणचि} (पा॰सू॰~६.१.७७)~\arrow द्र्~अ~श्~स्य~महि~\arrow \textcolor{red}{व्रश्चभ्रस्ज\-सृजमृज\-यजराज\-भ्राजच्छशां षः} (पा॰सू॰~८.२.३६)~\arrow द्र्~अ~ष्~स्य~महि~\arrow \textcolor{red}{षढोः कः सि} (पा॰सू॰~८.२.४१)~\arrow द्र्~अ~क्~स्य~महि~\arrow \textcolor{red}{अतो दीर्घो यञि} (पा॰सू॰~७.३.१०१)~\arrow द्र्~अ~क्~स्या~महि~\arrow \textcolor{red}{टित आत्मनेपदानां टेरे} (पा॰सू॰~३.४.७९)~\arrow द्र्~अ~क्~स्या~महे~\arrow \textcolor{red}{आदेश\-प्रत्यययोः} (पा॰सू॰~८.३.५९)~\arrow द्र्~अ~क्~ष्या~महे~\arrow द्रक्ष्यामहे।}\end{sloppypar}
\section[समपृच्छत]{समपृच्छत}
\centering\textcolor{blue}{वर्धयन् जयशब्देन प्रणमञ्छिरसा नृपम्।\nopagebreak\\
अतिखिन्नं नृपं दृष्ट्वा कैकेयीं समपृच्छत॥}\nopagebreak\\
\raggedleft{–~अ॰रा॰~२.३.४३}\\
\fontsize{14}{21}\selectfont\begin{sloppypar}\hyphenrules{nohyphenation}\justifying\noindent\hspace{10mm} अत्र कर्म\-व्यतिहार आत्मनेपदम्।\footnote{\textcolor{red}{कर्तरि कर्मव्यतिहारे} (पा॰सू॰~१.३.१४) इत्यनेन। सम्~\textcolor{red}{प्रच्छँ ज्ञीप्सायाम्} (धा॰पा॰~१४१३)~\arrow सम्~प्रच्छ्~\arrow \textcolor{red}{कर्तरि कर्मव्यतिहारे} (पा॰सू॰~१.३.१४)~\arrow \textcolor{red}{अनद्यतने लङ्} (पा॰सू॰~३.२.१११)~\arrow सम्~प्रच्छ्~लङ्~\arrow सम्~प्रच्छ्~त~\arrow \textcolor{red}{तुदादिभ्यः शः} (पा॰सू॰~३.१.७७)~\arrow सम्~प्रच्छ्~श~त~\arrow प्रच्छ्~अ~थास्~\arrow \textcolor{red}{सार्वधातुकमपित्} (पा॰सू॰~१.२.४)~\arrow ङिद्वत्त्वम्~\arrow \textcolor{red}{ग्रहिज्या\-वयिव्यधि\-वष्टिविचति\-वृश्चति\-पृच्छति\-भृज्जतीनां ङिति च} (पा॰सू॰~६.१.१६)~\arrow सम्~पृ~अ~च्छ्~अ~त~\arrow \textcolor{red}{सम्प्रसारणाच्च} (पा॰सू॰~६.१.१०८)~\arrow सम्~पृ~च्छ्~अ~त~\arrow सम्~पृ~च्छ्~अ~त~\arrow समपृच्छत। \textcolor{red}{विदि\-प्रच्छि\-स्वरतीनामुपसङ्ख्यानम्} इत्युप\-सङ्ख्यानस्याकर्मकाधिकारादत्राप्रवृत्तिः। यद्वा \textcolor{red}{अतिखिन्नं नृपं कैकेयीं च दृष्ट्वा} इत्यन्वयेन कर्मणोऽविवक्षायामकर्मकत्वादात्मने\-पदम्। अनेनैवोप\-सङ्ख्यानेन।}\end{sloppypar}
\section[शासतु]{शासतु}
\centering\textcolor{blue}{आश्वासयामास नृपं शनैः स नयकोविदः।\nopagebreak\\
किमत्र दुःखेन विभो राज्यं शासतु मेऽनुजः॥}\nopagebreak\\
\raggedleft{–~अ॰रा॰~२.३.७३}\\
\fontsize{14}{21}\selectfont\begin{sloppypar}\hyphenrules{nohyphenation}\justifying\noindent\hspace{10mm} अत्र \textcolor{red}{गण\-कार्यमनित्यम्} (प॰शे॰~९३.३) इति नियमाल्लोड्लकारे तिपि शपि \textcolor{red}{एरुः} (पा॰सू॰~३.४.८६) इत्यनेनोकारे \textcolor{red}{शासतु}।\footnote{\textcolor{red}{शासुँ अनुशिष्टौ} (धा॰पा॰~१०७४) इति धातोर्लोड्लकारे प्रथमपुरुष एकवचने \textcolor{red}{शास्तु} इति रूपम्। यथा \textcolor{red}{त्वया परिगृहीतोऽयमङ्गदः शास्तु मेदिनीम्} (वा॰रा॰~४.२१.९) इति वाल्मीकिप्रयोगे। शासुँ~\arrow शास्~\arrow \textcolor{red}{शेषात्कर्तरि परस्मैपदम्} (पा॰सू॰~१.३.७८)~\arrow \textcolor{red}{लोट् च} (पा॰सू॰~३.३.१६२)~\arrow शास्~लोट्~\arrow शास्~तिप्~\arrow शास्~ति~\arrow \textcolor{red}{कर्तरि शप्‌} (पा॰सू॰~३.१.६८)~\arrow शास्~शप्~ति~\arrow \textcolor{red}{अदिप्रभृतिभ्यः शपः} (पा॰सू॰~२.४.७२)~\arrow शब्लुक्~\arrow शास्~ति~\arrow \textcolor{red}{एरुः} (पा॰सू॰~३.४.८६)~\arrow शास्~तु~\arrow शास्तु। \textcolor{red}{गण\-कार्यमनित्यम्} (प॰शे॰~९३.३) इति परिभाषयाऽत्र \textcolor{red}{अदिप्रभृतिभ्यः शपः} (पा॰सू॰~२.४.७२) इत्यस्याप्रवृत्तौ शब्लुगभावे शास्~शप्~ति~\arrow शास्~अ~ति~\arrow \textcolor{red}{एरुः} (पा॰सू॰~३.४.८६)~\arrow शास्~अ~तु~\arrow \textcolor{red}{शासतु} इति सिद्धम्।}\end{sloppypar}
\section[यास्ये]{यास्ये}
\label{sec:yasye}
\centering\textcolor{blue}{मातरं समनुश्वास्य अनुनीय च जानकीम्।\nopagebreak\\
आगत्य पादौ वन्दित्वा तव यास्ये सुखं वनम्॥}\nopagebreak\\
\raggedleft{–~अ॰रा॰~२.३.७७}\\
\fontsize{14}{21}\selectfont\begin{sloppypar}\hyphenrules{nohyphenation}\justifying\noindent\hspace{10mm} श्रीरामः कैकेयीं प्रति कथयति यत् \textcolor{red}{अहं तव पादं वन्दित्वा वनं यास्यामि}। \textcolor{red}{यास्यामि}\footnote{\textcolor{red}{या प्रापणे} (धा॰पा॰~१०४९)~\arrow \textcolor{red}{शेषात्कर्तरि परस्मैपदम्} (पा॰सू॰~१.३.७८)~\arrow \textcolor{red}{लृट् शेषे च} (पा॰सू॰~३.३.१३)~\arrow या~लृट्~\arrow या~मिप्~\arrow या~मि~\arrow \textcolor{red}{स्यतासी लृलुटोः} (पा॰सू॰~३.१.३३)~\arrow या~स्य~मि~\arrow \textcolor{red}{अतो दीर्घो यञि} (पा॰सू॰~७.३.१०१)~\arrow या~स्या~मि~\arrow यास्यामि।} इति प्रयोक्तव्ये \textcolor{red}{यास्ये} इति प्रयुज्यते। अत्रापि कर्म\-व्यतिहारः।\footnote{तस्मात् \textcolor{red}{कर्तरि कर्मव्यतिहारे} (पा॰सू॰~१.३.१४) इत्यनेनाऽत्मनेपदमिति भावः। \textcolor{red}{या प्रापणे} (धा॰पा॰~१०४९)~\arrow \textcolor{red}{कर्तरि कर्मव्यतिहारे} (पा॰सू॰~१.३.१४)~\arrow \textcolor{red}{लृट् शेषे च} (पा॰सू॰~३.३.१३)~\arrow या~लृट्~\arrow या~इट्~\arrow या~इ~\arrow \textcolor{red}{स्यतासी लृलुटोः} (पा॰सू॰~३.१.३३)~\arrow या~स्य~इ~\arrow \textcolor{red}{टित आत्मनेपदानां टेरे} (पा॰सू॰~३.४.७९)~\arrow या~स्य~ए~\arrow \textcolor{red}{अतो गुणे} (पा॰सू॰~६.१.९७)~\arrow या~स्ये~\arrow यास्ये।} यतो हि वन\-गमनं तु वृद्धानां कृते। तथा चोक्तम्~–\end{sloppypar}
\centering\textcolor{red}{शैशवेऽभ्यस्तविद्यानां यौवने विषयैषिणाम्।\nopagebreak\\
वार्धके मुनिवृत्तीनां योगेनान्ते तनुत्यजाम्॥}\nopagebreak\\
\raggedleft{–~र॰वं॰~१.८}\\
\fontsize{14}{21}\selectfont\begin{sloppypar}\hyphenrules{nohyphenation}\justifying\noindent
 अतो वृद्धानां कर्म वन\-गमनं युवको भूत्वा श्रीराम आचरति तस्मादत्र \textcolor{red}{यास्ये} इत्येव प्रयोगः साधु। श्रीरामचरितमानसे कौसल्याऽपि कथयति~–\end{sloppypar}
\centering\textcolor{red}{अंतहुँ उचित नृपहि बनबासू। बय बिलोकि हिय होइ हरासू॥}\footnote{एतद्रूपान्तरम्–\textcolor{red}{अन्ते निवासो विपिने समीचीनो महीपतेः। किन्तु दृष्ट्वा तवावस्थां दुःखं मे मनसि स्थितम्॥} (मा॰भा॰~२.५६.४)।}\nopagebreak\\
\raggedleft{–~रा॰च॰मा॰~२.५६.४}\\
\fontsize{14}{21}\selectfont\begin{sloppypar}\hyphenrules{nohyphenation}\justifying\noindent तस्मात् \textcolor{red}{यास्ये}। वृद्धोचितमपि गमनं त्वत्सन्तोषार्थं करिष्यामीत्येव राजीव\-लोचनस्य विवक्षितोऽर्थ इट्प्रत्ययेन ध्वन्यते। न च \textcolor{red}{न गति\-हिंसार्थेभ्यः} (पा॰सू॰~१.३.१५) इत्यनेन कर्मव्यतिहार आत्मनेपद\-निषेधः। मण्डूक\-प्लुत्या \textcolor{red}{नपुंसकमनपुंसकेनैक\-वच्चास्यान्यतरस्याम्} (पा॰सू॰~१.२.६९) इत्यतः \textcolor{red}{अन्यतरस्याम्} इत्यनुवर्तनीयम्।\end{sloppypar}
\section[कुरुते]{कुरुते}
\centering\textcolor{blue}{इत्युक्त्वा तु परिक्रम्य मातरं द्रष्टुमाययौ।\nopagebreak\\
कौसल्याऽपि हरेः पूजां कुरुते रामकारणात्॥}\nopagebreak\\
\raggedleft{–~अ॰रा॰~२.३.७८}\\
\fontsize{14}{21}\selectfont\begin{sloppypar}\hyphenrules{nohyphenation}\justifying\noindent\hspace{10mm} अत्र \textcolor{red}{अकुरुत}\footnote{\textcolor{red}{डुकृञ् करणे} (धा॰पा॰~१४७२)~\arrow कृ~\arrow \textcolor{red}{स्वरितञितः कर्त्रभिप्राये क्रियाफले} (पा॰सू॰~१.३.७२)~\arrow \textcolor{red}{अनद्यतने लङ्} (पा॰सू॰~३.२.१११)~\arrow कृ~लङ्~\arrow कृ~त~\arrow \textcolor{red}{लुङ्लङ्लृङ्क्ष्वडुदात्तः} (पा॰सू॰~६.४.७१)~\arrow \textcolor{red}{आद्यन्तौ टकितौ} (पा॰सू॰~१.१.४६)~\arrow अट्~कृ~त~\arrow अ~कृ~त~\arrow \textcolor{red}{तनादि\-कृञ्भ्य उः} (पा॰सू॰~३.१.७९)~\arrow अ~कृ~उ~त~\arrow \textcolor{red}{सार्वधातुकार्ध\-धातुकयोः} (पा॰सू॰~७.३.८४)~\arrow \textcolor{red}{उरण् रपरः} (पा॰सू॰~१.१.५१)~\arrow अ~क्~अर्~उ~त~\arrow \textcolor{red}{अत उत्सार्वधातुके} (पा॰सू॰~६.४.११०)~\arrow अ~क्~उर्~उ~त~\arrow अकुरुत।} इति प्रयोक्तव्यम्। \textcolor{red}{कुरुते} इति प्रयुक्तम्। \textcolor{red}{स्म}\-योगे \textcolor{red}{लट् स्मे} (पा॰सू॰~३.२.११८) इत्यनेन लड्लकार\-विधानात्सम्यक्।\footnote{\textcolor{red}{स्म} इत्यध्याहार्यमिति भावः। \textcolor{red}{डुकृञ् करणे} (धा॰पा॰~१४७२)~\arrow कृ~\arrow \textcolor{red}{स्वरितञितः कर्त्रभिप्राये क्रियाफले} (पा॰सू॰~१.३.७२)~\arrow \textcolor{red}{लट् स्मे} (पा॰सू॰~३.२.११८)~\arrow कृ~लट्~\arrow कृ~त~\arrow \textcolor{red}{तनादि\-कृञ्भ्य उः} (पा॰सू॰~३.१.७९)~\arrow कृ~उ~त~\arrow \textcolor{red}{सार्वधातुकार्ध\-धातुकयोः} (पा॰सू॰~७.३.८४)~\arrow \textcolor{red}{उरण् रपरः} (पा॰सू॰~१.१.५१)~\arrow क्~अर्~उ~त~\arrow \textcolor{red}{अत उत्सार्वधातुके} (पा॰सू॰~६.४.११०)~\arrow क्~उर्~उ~त~\arrow \textcolor{red}{टित आत्मनेपदानां टेरे} (पा॰सू॰~३.४.७९)~\arrow क्~उर्~उ~ते~\arrow कुरुते।}\end{sloppypar}
\section[ध्यायते]{ध्यायते}
\centering\textcolor{blue}{होमं च कारयामास ब्राह्मणेभ्यो ददौ धनम्।\nopagebreak\\
ध्यायते विष्णुमेकाग्रमनसा मौनमास्थिता॥}\nopagebreak\\
\raggedleft{–~अ॰रा॰~२.३.७९}\\
\fontsize{14}{21}\selectfont\begin{sloppypar}\hyphenrules{nohyphenation}\justifying\noindent\hspace{10mm} अत्रापि कर्म\-व्यतिहारादेवाऽत्मने\-पदम्।\footnote{\textcolor{red}{कर्तरि कर्मव्यतिहारे} (पा॰सू॰~१.३.१४) इत्यनेन।} यतो हि महाविष्णुस्तु तस्याः पुत्रः। अंशिनि महाविष्णौ श्रीरामभद्रे पुत्र\-रूपेण वर्तमानेऽप्यंशं विष्णुं वामनावतारे तन्माताऽदितिः सत्यपि प्राकृत\-नारीव ध्यायतीत्येव कर्म\-व्यतिहारस्य सङ्गतिः। वर्तमान\-कालोऽपि पूर्वोक्त\-दिशा।\footnote{\textcolor{red}{लट् स्मे} (पा॰सू॰~३.२.११८) इत्यनेन। \textcolor{red}{स्म} इत्यध्याहार्यमिति भावः। \textcolor{red}{ध्यै चिन्तायाम्} (धा॰पा॰~९०८)~\arrow ध्यै~\arrow \textcolor{red}{कर्तरि कर्मव्यतिहारे} (पा॰सू॰~१.३.१४)~\arrow \textcolor{red}{लट् स्मे} (पा॰सू॰~३.२.११८)~\arrow ध्यै~लट्~\arrow ध्यै~त~\arrow \textcolor{red}{कर्तरि शप्‌} (पा॰सू॰~३.१.६८)~\arrow ध्यै~शप्~त~\arrow ध्यै~अ~त~\arrow \textcolor{red}{एचोऽयवायावः} (पा॰सू॰~६.१.७८)~\arrow ध्याय्~अ~त~\arrow \textcolor{red}{टित आत्मनेपदानां टेरे} (पा॰सू॰~३.४.७९)~\arrow ध्याय्~अ~ते~\arrow ध्यायते।}\end{sloppypar}
\section[आगमिष्ये]{आगमिष्ये}
\centering\textcolor{blue}{चतुर्दश समास्तत्र ह्युषित्वा मुनिवेषधृक्।\nopagebreak\\
आगमिष्ये पुनः शीघ्रं न चिन्तां कर्तुमर्हसि॥}\nopagebreak\\
\raggedleft{–~अ॰रा॰~२.४.६}\\
\fontsize{14}{21}\selectfont\begin{sloppypar}\hyphenrules{nohyphenation}\justifying\noindent\hspace{10mm} श्रीरामः कथयति वनवासान्ते \textcolor{red}{आगमिष्ये}। अत्र \textcolor{red}{आगमिष्यामि} इति हि पाणिनीयम्।\footnote{आङ् \textcolor{red}{गमॢँ गतौ} (धा॰पा॰~९८२)~\arrow आ~गम्~\arrow \textcolor{red}{शेषात्कर्तरि परस्मैपदम्} (पा॰सू॰~१.३.७८)~\arrow \textcolor{red}{लृट् शेषे च} (पा॰सू॰~३.३.१३)~\arrow आ~गम्~लृँट्~\arrow आ~गम्~मिप्~\arrow आ~गम्~मि~\arrow \textcolor{red}{स्यतासी लृलुटोः} (पा॰सू॰~३.१.३३)~\arrow आ~गम्~स्य~मि~\arrow \textcolor{red}{गमेरिट् परस्मैपदेषु} (पा॰सू॰~७.२.५८)~\arrow आ~गम्~इट्~स्य~मि~\arrow आ~गम्~इ~स्य~मि~\arrow \textcolor{red}{अतो दीर्घो यञि} (पा॰सू॰~७.३.१०१)~\arrow आ~गम्~इ~स्या~मि~\arrow \textcolor{red}{आदेशप्रत्यययोः} (पा॰सू॰~८.३.५९)~\arrow आ~गम्~इ~ष्या~ति~\arrow आगमिष्यामि।} समुपसर्गाध्याहार आत्मनेपदम् \textcolor{red}{समो गम्यृच्छिभ्याम्} (पा॰सू॰~१.३.२९) इत्यनेन। न च \textcolor{red}{आङ्‌}\-उपसर्गस्यैव \textcolor{red}{गम्‌}\-धातुनाऽन्वय इति चेत्। \textcolor{red}{व्यवहिते} इति वक्तव्यम्।
न च सत्यप्यात्मनेपद इट् कथं \textcolor{red}{गमेरिट् परस्मैपदेषु} (पा॰सू॰~७.२.५८) इति सूत्रेण परस्मैपद एवेड्विधानादिति चेत्सत्यम्। उच्यते। \textcolor{red}{आगच्छतीत्यागम्} आचारे क्विप्ततश्च सर्वापहारि\-लोपे।\footnote{आङ् \textcolor{red}{गमॢँ गतौ} (धा॰पा॰~९८२)~\arrow \textcolor{red}{क्विप् च} (पा॰सू॰~३.२.७६)~\arrow आ~गम्~क्विँप्~\arrow आ~गम्~व्~\arrow \textcolor{red}{वेरपृक्तस्य} (पा॰सू॰~६.१.६७)~\arrow आ~गम्~\arrow आगम्~\arrow \textcolor{red}{कृत्तद्धित\-समासाश्च} (पा॰सू॰~१.२.४६)~\arrow प्रातिपदिक\-सञ्ज्ञा~\arrow विभक्तिकार्यम्~\arrow आगम्~सुँ~\arrow \textcolor{red}{हल्ङ्याब्भ्यो दीर्घात्सुतिस्यपृक्तं हल्} (पा॰सू॰~६.१.६८)~\arrow आगम्।} \textcolor{red}{इष्‌}\-धातोः कर्म\-व्यतिहार आत्मनेपदे \textcolor{red}{इष्ये}।\footnote{\textcolor{red}{इषँ गतौ} (धा॰पा॰~११२७)~\arrow इष्~\arrow \textcolor{red}{कर्तरि कर्मव्यतिहारे} (पा॰सू॰~१.३.१४)~\arrow \textcolor{red}{वर्तमान\-सामीप्ये वर्तमानवद्वा} (पा॰सू॰~३.३.१३१)~\arrow \textcolor{red}{वर्तमाने लट्} (पा॰सू॰~३.२.१२३)~\arrow इष्~लट्~\arrow इष्~इट्~\arrow इष्~इ~\arrow \textcolor{red}{दिवादिभ्यः श्यन्} (पा॰सू॰~३.१.६९)~\arrow इष्~शयन्~इ~\arrow इष्~य~इ~\arrow \textcolor{red}{टित आत्मनेपदानां टेरे} (पा॰सू॰~३.४.७९)~\arrow इष्~य~ए~\arrow \textcolor{red}{अतो गुणे} (पा॰सू॰~६.१.९७)~\arrow इष्~ये~\arrow इष्ये।} \textcolor{red}{आगम् इष्ये} इति \textcolor{red}{आगमिष्ये}।\end{sloppypar}
\section[जायते]{जायते}
\centering\textcolor{blue}{यथा प्रवाहपतितप्लवानां सरितां तथा।\nopagebreak\\
चतुर्दशसमासङ्ख्या क्षणार्द्धमिव जायते॥}\nopagebreak\\
\raggedleft{–~अ॰रा॰~२.४.४६}\\
\fontsize{14}{21}\selectfont\begin{sloppypar}\hyphenrules{nohyphenation}\justifying\noindent\hspace{10mm} सामीप्याभिप्रायेणैव \textcolor{red}{क्षणार्द्धमिव} इति वाक्य\-खण्ड\-प्रयोगेण च शीघ्रतां द्योतयितुं वर्तमान\-समीपे लटं प्रयुङ्क्ते। \textcolor{red}{वर्तमान\-सामीप्ये वर्तमानवद्वा} (पा॰सू॰~३.३.१३१) इत्यनेन।\footnote{\textcolor{red}{जनीँ प्रादुर्भावे} (धा॰पा॰~११४९)~\arrow \textcolor{red}{अनुदात्तङित आत्मने\-पदम्} (पा॰सू॰~१.३.१२)~\arrow \textcolor{red}{वर्तमान\-सामीप्ये वर्तमानवद्वा} (पा॰सू॰~३.३.१३१)~\arrow \textcolor{red}{वर्तमाने लट्} (पा॰सू॰~३.२.१२३)~\arrow जन्~लट्~\arrow जन्~त~\arrow \textcolor{red}{दिवादिभ्यः श्यन्} (पा॰सू॰~३.१.६९)~\arrow जन्~श्यन्~त~\arrow जन्~य~त~\arrow \textcolor{red}{ज्ञाजनोर्जा} (पा॰सू॰~७.३.७९)~\arrow जा~य~त~\arrow \textcolor{red}{टित आत्मनेपदानां टेरे} (पा॰सू॰~३.४.७९)~\arrow जा~य~ते~\arrow जायते।}\end{sloppypar}
\section[नेष्ये]{नेष्ये}
\centering\textcolor{blue}{तामाह राघवः प्रीतः स्वप्रियां प्रियवादिनीम्।\nopagebreak\\
कथं वनं त्वां नेष्येऽहं बहुव्याघ्रमृगाकुलम्॥}\nopagebreak\\
\raggedleft{–~अ॰रा॰~२.४.६४}\\
\fontsize{14}{21}\selectfont\begin{sloppypar}\hyphenrules{nohyphenation}\justifying\noindent\hspace{10mm} \textcolor{red}{णीञ् प्रापणे} (धा॰पा॰~९०१) इति धातुरकर्त्रभिप्राये क्रियाफले परस्मैपदी। तथा च \textcolor{red}{नेष्यामि} इति प्रयोक्तव्यम्।\footnote{\textcolor{red}{णीञ् प्रापणे} (धा॰पा॰~९०१)~\arrow णी~\arrow \textcolor{red}{णो नः} (पा॰सू॰~६.१.६५)~\arrow नी~\arrow \textcolor{red}{शेषात्कर्तरि परस्मैपदम्} (पा॰सू॰~१.३.७८)~\arrow \textcolor{red}{लृट् शेषे च} (पा॰सू॰~३.३.१३)~\arrow नी~लृट्~\arrow नी~मिप्~\arrow नी~मि~\arrow \textcolor{red}{स्यतासी लृलुटोः} (पा॰सू॰~३.१.३३)~\arrow नी~स्य~मि~\arrow \textcolor{red}{सार्वधातुकार्ध\-धातुकयोः} (पा॰सू॰~७.३.८४)~\arrow ने~स्य~मि~\arrow \textcolor{red}{अतो दीर्घो यञि} (पा॰सू॰~७.३.१०१)~\arrow ने~स्या~मि~\arrow \textcolor{red}{आदेश\-प्रत्यययोः} (पा॰सू॰~८.३.५९)~\arrow ने~ष्या~मि~\arrow नेष्यामि।} अत्र \textcolor{red}{उपनेष्ये} इति हि पदम्।\footnote{\textcolor{red}{उप}उपसर्गस्य लोप इति भावः।} \textcolor{red}{कथमुपनेष्ये} इत्यर्थे सम्मानन आत्मनेपदम्।\footnote{उप~\textcolor{red}{णीञ् प्रापणे} (धा॰पा॰~९०१)~\arrow उप~णी~\arrow \textcolor{red}{णो नः} (पा॰सू॰~६.१.६५)~\arrow उप~नी~\arrow \textcolor{red}{सम्माननोत्सञ्जनाचार्य\-करण\-ज्ञान\-भृति\-विगणन\-व्ययेषु नियः} (पा॰सू॰~१.३.३६)~\arrow \textcolor{red}{लृट् शेषे च} (पा॰सू॰~३.३.१३)~\arrow उप~नी~लृट्~\arrow उप~नी~इट्~\arrow उप~नी~इ~\arrow \textcolor{red}{स्यतासी लृलुटोः} (पा॰सू॰~३.१.३३)~\arrow उप~नी~स्य~इ~\arrow \textcolor{red}{सार्वधातुकार्ध\-धातुकयोः} (पा॰सू॰~७.३.८४)~\arrow उप~ने~स्य~इ~\arrow \textcolor{red}{टित आत्मनेपदानां टेरे} (पा॰सू॰~३.४.७९)~\arrow उप~ने~स्य~ए~\arrow \textcolor{red}{अतो गुणे} (पा॰सू॰~६.१.९७)~\arrow उप~ने~स्ये~\arrow \textcolor{red}{आदेश\-प्रत्यययोः} (पा॰सू॰~८.३.५९)~\arrow उप~ने~ष्ये~\arrow \textcolor{red}{विनाऽपि प्रत्ययं पूर्वोत्तर\-पद\-लोपो वक्तव्यः} (वा॰~५.३.८३)~\arrow ने~ष्ये~\arrow नेष्ये।} \textcolor{red}{कथं सम्मानितं करिष्यामि} इति भावः। यद्वा व्यये। \textcolor{red}{कथं दिनानि यापयिष्यामि}। यद्वोत्सञ्जने। अत्र सूत्रं \textcolor{red}{सम्माननोत्सञ्जनाचार्य\-करण\-ज्ञान\-भृति\-विगणन\-व्ययेषु नियः} (पा॰सू॰~१.३.३६)।
\end{sloppypar}
\section[हास्यसे]{हास्यसे}
\centering\textcolor{blue}{पादचारेण गन्तव्यं शीतवातातपादिमत्।\nopagebreak\\
राक्षसादीन्वने दृष्ट्वा जीवितं हास्यसेऽचिरात्॥}\nopagebreak\\
\raggedleft{–~अ॰रा॰~२.४.६९}\\
\fontsize{14}{21}\selectfont\begin{sloppypar}\hyphenrules{nohyphenation}\justifying\noindent\hspace{10mm} अत्र \textcolor{red}{हास्यसि}\footnote{\textcolor{red}{ओँहाक् त्यागे} (धा॰पा॰~१०९०)~\arrow हा~\arrow \textcolor{red}{शेषात्कर्तरि परस्मैपदम्} (पा॰सू॰~१.३.७८)~\arrow \textcolor{red}{लृट् शेषे च} (पा॰सू॰~३.३.१३)~\arrow हा~लृट्~\arrow हा~सिप्~\arrow हा~सि~\arrow \textcolor{red}{स्यतासी लृलुटोः} (पा॰सू॰~३.१.३३)~\arrow हा~स्य~सि~\arrow हास्यसि।} इति प्रयोक्तव्यं किन्तु कर्म\-व्यतिहार आत्मनेपदम्।\footnote{\textcolor{red}{कर्तरि कर्मव्यतिहारे} (पा॰सू॰~१.३.१४) इत्यनेन। \textcolor{red}{ओँहाक् त्यागे} (धा॰पा॰~१०९०)~\arrow हा~\arrow \textcolor{red}{कर्तरि कर्मव्यतिहारे} (पा॰सू॰~१.३.१४)~\arrow \textcolor{red}{लृट् शेषे च} (पा॰सू॰~३.३.१३)~\arrow हा~लृट्~\arrow हा~थास्~\arrow \textcolor{red}{स्यतासी लृलुटोः} (पा॰सू॰~३.१.३३)~\arrow हा~स्य~थास्~\arrow \textcolor{red}{थासस्से} (पा॰सू॰~३.४.८०)~\arrow हा~स्य~से~\arrow हास्यसे।} यद्वा \textcolor{red}{जीवितम्} इति क्रिया\-विशेषणम् \textcolor{red}{उपेक्ष्य} इति वाऽध्याहार्यम्। एवं \textcolor{red}{जीवितमुपेक्ष्य त्वं प्राणैः हास्यसे} इति कर्म\-वाच्य आत्मनेपदम्।\footnote{\textcolor{red}{ओँहाक् त्यागे} (धा॰पा॰~१०९०) इति धातोः कर्मणि लृटि मध्यम\-पुरुष एकवचने \textcolor{red}{हास्यसे हायिष्यसे} इति रूपद्वयम्। अचिण्वद्भाव इडभावे \textcolor{red}{हास्यसे} इति रूपम्। हा~\arrow \textcolor{red}{भावकर्मणोः} (पा॰सू॰~१.३.१३)~\arrow \textcolor{red}{लृट् शेषे च} (पा॰सू॰~३.३.१३)~\arrow हा~लृट्~\arrow हा~थास्~\arrow \textcolor{red}{स्यतासी लृलुटोः} (पा॰सू॰~३.१.३३)~\arrow हा~स्य~थास्~\arrow \textcolor{red}{थासस्से} (पा॰सू॰~३.४.८०)~\arrow हा~स्य~से~\arrow हास्यसे। पक्षे चिण्वद्भाव इडागमे हायिष्यसे इति रूपम्। हा~\arrow \textcolor{red}{भावकर्मणोः} (पा॰सू॰~१.३.१३)~\arrow \textcolor{red}{लृट् शेषे च} (पा॰सू॰~३.३.१३)~\arrow हा~लृट्~\arrow हा~थास्~\arrow \textcolor{red}{स्यतासी लृलुटोः} (पा॰सू॰~३.१.३३)~\arrow हा~स्य~थास्~\arrow \textcolor{red}{स्यसिच्सीयुट्तासिषु भाव\-कर्मणोरुपदेशेऽज्झन\-ग्रहदृशां वा चिण्वदिट् च} (पा॰सू॰~६.४.६२)~\arrow हा~इट्~स्य~थास्~\arrow हा~इ~स्य~थास्~\arrow \textcolor{red}{आतो युक् चिण्कृतोः} (पा॰सू॰~७.३.३३)~\arrow हा~युँक्~इ~स्य~थास्~\arrow हा~य्~इ~स्य~थास्~\arrow \textcolor{red}{आदेश\-प्रत्यययोः} (पा॰सू॰~८.३.५९)~\arrow हा~य्~इ~ष्य~थास्~\arrow \textcolor{red}{थासस्से} (पा॰सू॰~३.४.८०)~\arrow हा~य्~इ~ष्य~से~\arrow हायिष्यसे।}\end{sloppypar}
\section[इच्छसे]{इच्छसे}
\centering\textcolor{blue}{प्रत्युवाच स्फुरद्वक्त्रा किञ्चित्कोपसमन्विता।\nopagebreak\\
कथं मामिच्छसे त्यक्तुं धर्मपत्नीं पतिव्रताम्॥}\nopagebreak\\
\raggedleft{–~अ॰रा॰~२.४.७१}\\
\fontsize{14}{21}\selectfont\begin{sloppypar}\hyphenrules{nohyphenation}\justifying\noindent\hspace{10mm} अत्रापि कर्म\-व्यतिहार आत्मनेपदम्।\footnote{\textcolor{red}{कर्तरि कर्मव्यतिहारे} (पा॰सू॰~१.३.१४) इत्यनेन। \textcolor{red}{इषँ इच्छायाम्} (धा॰पा॰~१३५१)~\arrow \textcolor{red}{कर्तरि कर्मव्यतिहारे} (पा॰सू॰~१.३.१४)~\arrow वर्तमाने लट्~\arrow इष्~लट्~\arrow इष्~थास्~\arrow \textcolor{red}{तुदादिभ्यः शः} (पा॰सू॰~३.१.७७)~\arrow इष्~श~थास्~\arrow इष्~अ~थास्~\arrow \textcolor{red}{इषुगमियमां छः} (पा॰सू॰~७.३.७७)~\arrow इछ्~अ~थास्~\arrow \textcolor{red}{छे च} (पा॰सू॰~६.१.७३)~\arrow \textcolor{red}{आद्यन्तौ टकितौ} (पा॰सू॰~१.१.४६)~\arrow इतुँक्~छ्~अ~थास्~\arrow इत्~छ्~अ~थास्~\arrow \textcolor{red}{स्तोः श्चुना श्चुः} (पा॰सू॰~८.४.४०)~\arrow इच्~छ्~अ~थास्~\arrow \textcolor{red}{थासस्से} (पा॰सू॰~३.४.८०)~\arrow इच्~छ्~अ~से~\arrow इच्छसे।}\end{sloppypar}
\section[रमामि]{रमामि}
\centering\textcolor{blue}{फलमूलादिकं यद्यत्तव भुक्तावशेषितम्।\nopagebreak\\
तदेवामृततुल्यं मे तेन तुष्टा रमाम्यहम्॥}\nopagebreak\\
\raggedleft{–~अ॰रा॰~२.४.७३}\\
\fontsize{14}{21}\selectfont\begin{sloppypar}\hyphenrules{nohyphenation}\justifying\noindent\hspace{10mm} अत्र \textcolor{red}{रमुँ क्रीडायाम्} (धा॰पा॰~८५३) इत्यात्मनेपदी धातुः। एवं भविष्यत्काले \textcolor{red}{रंस्ये} इति हि युक्तं पाणिनीयमपि।\footnote{\textcolor{red}{रमुँ क्रीडायाम्} (धा॰पा॰~८५३)~\arrow रम्~\arrow \textcolor{red}{अनुदात्तङित आत्मने\-पदम्} (पा॰सू॰~१.३.१२)~\arrow \textcolor{red}{लृट् शेषे च} (पा॰सू॰~३.३.१३)~\arrow रम्~लृट्~\arrow रम्~इट्~\arrow रम्~इ~\arrow \textcolor{red}{स्यतासी लृलुटोः} (पा॰सू॰~३.१.३३)~\arrow रम्~स्य~इ~\arrow \textcolor{red}{टित आत्मनेपदानां टेरे} (पा॰सू॰~३.४.७९)~\arrow रम्~स्य~ए~\arrow \textcolor{red}{अतो गुणे} (पा॰सू॰~६.१.९७)~\arrow रम्~स्ये~\arrow \textcolor{red}{मोऽनुस्वारः} (पा॰सू॰~८.३.२३)~\arrow रं~स्ये~\arrow रंस्ये।} अत्र \textcolor{red}{रमामि} इत्यत्र द्वावंशावयुक्ताविव। परस्मैपद\-प्रयोगः काल\-व्यत्ययश्च। कालव्यत्यये तु भविष्यत्कालेऽपि \textcolor{red}{लट् स्मे} (पा॰सू॰~३.२.११८) इत्यनेन \textcolor{red}{स्म}\-शब्द\-योगे वर्तमानवत्कार्यम्।\footnote{\textcolor{red}{स्म} इत्यध्याहार्यमिति भावः।} \textcolor{red}{आरमामि} इत्यत्र \textcolor{red}{व्याङ्परिभ्यो रमः} (पा॰सू॰~१.३.८३) इत्यनेन परस्मैपदम्।\footnote{आङ्~रम्~\arrow आ~रम्~\arrow \textcolor{red}{व्याङ्परिभ्यो रमः} (पा॰सू॰~१.३.८३)~\arrow \textcolor{red}{लट् स्मे} (पा॰सू॰~३.२.११८)~\arrow आ~रम्~लट्~\arrow आ~रम्~मिप्~\arrow आ~रम्~मि~\arrow \textcolor{red}{कर्तरि शप्‌} (पा॰सू॰~३.१.६८)~\arrow आ~रम्~शप्~मि~\arrow आ~रम्~अ~मि~\arrow \textcolor{red}{अतो दीर्घो यञि} (पा॰सू॰~७.३.१०१)~\arrow आ~रम्~आ~मि~\arrow आरमामि।} \textcolor{red}{आङ्} इत्यस्य \textcolor{red}{विनाऽपि प्रत्ययं पूर्वोत्तर\-पद\-लोपो वक्तव्यः} (वा॰~५.३.८३) इत्यनेन लोपः। यद्वा \textcolor{red}{तुष्टा आरमामि} इति स्थिते \textcolor{red}{अकः सवर्णे दीर्घः} (पा॰सू॰~६.१.१०१) इत्यनेन दीर्घैकादेशे \textcolor{red}{तुष्टाऽऽरमामि} इति।\footnote{यद्वा \textcolor{red}{अनुदात्तेत्त्व\-लक्षणमात्मने\-पदमनित्यम्} (प॰शे॰~९३.४) इत्यपि समाधानम्। \textcolor{red}{रमुँ क्रीडायाम्} (धा॰पा॰~८५३)~\arrow रम्~\arrow \textcolor{red}{अनुदात्तेत्त्व\-लक्षणमात्मने\-पदमनित्यम्} (प॰शे॰~९३.४)~\arrow \textcolor{red}{शेषात्कर्तरि परस्मैपदम्} (पा॰सू॰~१.३.७८)~\arrow \textcolor{red}{लट् स्मे} (पा॰सू॰~३.२.११८)~\arrow रम्~लट्~\arrow रम्~मिप्~\arrow रम्~मि~\arrow \textcolor{red}{कर्तरि शप्‌} (पा॰सू॰~३.१.६८)~\arrow रम्~शप्~मि~\arrow रम्~अ~मि~\arrow \textcolor{red}{अतो दीर्घो यञि} (पा॰सू॰~७.३.१०१)~\arrow रम्~आ~मि~\arrow रमामि।}\end{sloppypar}
\section[द्रक्ष्यथ]{द्रक्ष्यथ}
\centering\textcolor{blue}{रामोऽपि पादचारेण गजाश्वादिविवर्जितः।\nopagebreak\\
गच्छति द्रक्ष्यथ विभुं सर्वलोकैकसुन्दरम्॥}\nopagebreak\\
\raggedleft{–~अ॰रा॰~२.५.७}\\
\fontsize{14}{21}\selectfont\begin{sloppypar}\hyphenrules{nohyphenation}\justifying\noindent\hspace{10mm} अत्र \textcolor{red}{पश्यत} इति प्रयोक्तव्ये \textcolor{red}{द्रक्ष्यथ} इति प्रयुक्तम्। \textcolor{red}{द्रक्ष्} इति स्वतन्त्रो धातुः।\footnote{\textcolor{red}{बहुलमेतन्निदर्शनम्} (धा॰पा॰ ग॰सू॰~१९३८) \textcolor{red}{भूवादिष्वेतदन्तेषु दशगणीषु धातूनां पाठो निदर्शनाय तेन स्तम्भुप्रभृतयः सौत्राश्चुलुम्पादयो वाक्यकारीयाः प्रयोगसिद्धा विक्लवत्यादयश्च} (मा॰धा॰वृ॰~१०.३२८) इत्यनुसारं दिवादिगण ऊह्योऽयं \textcolor{red}{द्रक्ष्‌}\-धातुर्दर्शने।} तस्य लटि मध्यम\-पुरुष\-बहु\-वचनम् \textcolor{red}{द्रक्ष्यथ}।\footnote{ द्रक्ष्~\arrow \textcolor{red}{शेषात्कर्तरि परस्मैपदम्} (पा॰सू॰~१.३.७८)~\arrow \textcolor{red}{वर्तमाने लट्} (पा॰सू॰~३.२.१२३)~\arrow द्रक्ष्~लट्~\arrow द्रक्ष्~थ~\arrow \textcolor{red}{दिवादिभ्यः श्यन्} (पा॰सू॰~३.१.६९)~\arrow द्रक्ष्~श्यन्~थ~\arrow द्रक्ष्~य~थ~\arrow द्रक्ष्यथ।} यद्वा \textcolor{red}{अभिज्ञा\-वचने लृट्} (पा॰सू॰~३.२.११२)।\footnote{\textcolor{red}{दृशिँर प्रेक्षणे} (धा॰पा॰~९८८)~\arrow दृश्~\arrow \textcolor{red}{शेषात्कर्तरि परस्मैपदम्} (पा॰सू॰~१.३.७८)~\arrow \textcolor{red}{अभिज्ञा\-वचने लृट्} (पा॰सू॰~३.२.११२)~\arrow दृश्~लृट्~\arrow दृश्~थ~\arrow \textcolor{red}{स्यतासी लृलुटोः} (पा॰सू॰~३.१.३३)~\arrow दृश्~स्य~थ~\arrow \textcolor{red}{सृजि\-दृशोर्झल्यमकिति} (पा॰सू॰~६.१.५८)~\arrow \textcolor{red}{मिदचोऽन्त्यात्परः} (पा॰सू॰~१.१.४७)~\arrow दृ~अम्~श्~स्य~थ~\arrow दृ~अ~श्~स्य~थ~\arrow \textcolor{red}{इको यणचि} (पा॰सू॰~६.१.७७)~\arrow द्र्~अ~श्~स्य~थ~\arrow \textcolor{red}{व्रश्चभ्रस्ज\-सृजमृज\-यजराज\-भ्राजच्छशां षः} (पा॰सू॰~८.२.३६)~\arrow द्र्~अ~ष्~स्य~थ~\arrow \textcolor{red}{षढोः कः सि} (पा॰सू॰~८.२.४१)~\arrow द्र्~अ~क्~स्य~थ~\arrow \textcolor{red}{आदेश\-प्रत्यययोः} (पा॰सू॰~८.३.५९)~\arrow द्र्~अ~क्~ष्य~थ~\arrow द्रक्ष्यथ।} \textcolor{red}{स्मरथ किमीदृशं द्रक्ष्यथ रामं पूर्वम्}। अतो नापाणिनीयम्।\end{sloppypar}
\section[जीवामि]{जीवामि}
\centering\textcolor{blue}{किञ्चित्कालं भवेत्तत्र जीवनं दुःखितस्य मे।\nopagebreak\\
अत ऊर्ध्वं न जीवामि चिरं रामं विना कृतः॥}\nopagebreak\\
\raggedleft{–~अ॰रा॰~२.५.४९}\\
\fontsize{14}{21}\selectfont\begin{sloppypar}\hyphenrules{nohyphenation}\justifying\noindent\hspace{10mm} अत्र वर्तमान\-सामीप्ये लट्।\footnote{\textcolor{red}{वर्तमान\-सामीप्ये वर्तमानवद्वा} (पा॰सू॰~३.३.१३१) इत्यनेन। \textcolor{red}{जीवँ प्राणधारणे} (धा॰पा॰~५६२)~\arrow जीव्~\arrow \textcolor{red}{शेषात्कर्तरि परस्मैपदम्} (पा॰सू॰~१.३.७८)~\arrow \textcolor{red}{वर्तमान\-सामीप्ये वर्तमानवद्वा} (पा॰सू॰~३.३.१३१)~\arrow \textcolor{red}{वर्तमाने लट्} (पा॰सू॰~३.२.१२३)~\arrow जीव्~लट्~\arrow जीव्~मिप्~\arrow जीव्~मि~\arrow \textcolor{red}{कर्तरि शप्‌} (पा॰सू॰~३.१.६८)~\arrow जीव्~शप्~मि~\arrow जीव्~अ~मि~\arrow \textcolor{red}{अतो दीर्घो यञि} (पा॰सू॰~७.३.१०१)~\arrow जीव्~आ~मि~\arrow जीवामि।}\end{sloppypar}
\section[गच्छामहे]{गच्छामहे}
\centering\textcolor{blue}{पौराः सर्वे समागत्य स्थितास्तस्याविदूरतः।\nopagebreak\\
शक्ता रामं पुरं नेतुं नोचेद्गच्छामहे वनम्॥}\nopagebreak\\
\raggedleft{–~अ॰रा॰~२.५.५३}\\
\fontsize{14}{21}\selectfont\begin{sloppypar}\hyphenrules{nohyphenation}\justifying\noindent\hspace{10mm} \textcolor{red}{हे पौरा वनं गच्छाम} इत्यन्वये शक्यार्थे लोट्।\footnote{\textcolor{red}{शकि लिङ् च} (पा॰सू॰~३.३.१७२) इत्यत्र मण्डूक\-प्लुत्या \textcolor{red}{स्मे लोट्} (पा॰सू॰~३.३.१६५) इत्यतः \textcolor{red}{लोट्} इत्यनुवर्तनीयमिति भावः। \textcolor{red}{गमॢँ गतौ} (धा॰पा॰~९८२)~\arrow गम्~\arrow \textcolor{red}{शेषात्कर्तरि परस्मैपदम्} (पा॰सू॰~१.३.७८)~\arrow \textcolor{red}{शकि लिङ् च} (पा॰सू॰~३.३.१७२)~\arrow गम्~लोट्~\arrow गम्~मस्~\arrow \textcolor{red}{कर्तरि शप्‌} (पा॰सू॰~३.१.६८)~\arrow गम्~शप्~मस्~\arrow गम्~अ~मस्~\arrow \textcolor{red}{इषुगमियमां छः} (पा॰सू॰~७.३.७७)~\arrow गछ्~अ~मस्~\arrow \textcolor{red}{छे च} (पा॰सू॰~६.१.७३)~\arrow \textcolor{red}{आद्यन्तौ टकितौ} (पा॰सू॰~१.१.४६)~\arrow गतुँक्~छ्~अ~मस्~\arrow गत्~छ्~अ~मस्~\arrow \textcolor{red}{स्तोः श्चुना श्चुः} (पा॰सू॰~८.४.४०)~\arrow गच्~छ्~अ~मस्~\arrow \textcolor{red}{आडुत्तमस्य पिच्च} (पा॰सू॰~३.४.९२)~\arrow गच्~छ्~अ~आट्~मस्~\arrow \textcolor{red}{अकः सवर्णे दीर्घः} (पा॰सू॰~६.१.१०१)~\arrow गच्~छ्~आ~मस्~\arrow \textcolor{red}{लोटो लङ्वत्‌} (पा॰सू॰~३.४.८५)~\arrow ङिद्वत्त्वम्~\arrow नित्यं ङितः~\arrow गच्~छ्~आ~म~\arrow गच्छाम।} \textcolor{red}{वनं गन्तुं शक्ताः} इति भावः।\end{sloppypar}
\section[बभूव]{बभूव}
\centering\textcolor{blue}{बभूव परमानदः स्पृष्ट्वा तेऽङ्गं रघूत्तम।\nopagebreak\\
नैषादराज्यमेतत्ते किङ्करस्य रघूत्तम॥}\nopagebreak\\
\raggedleft{–~अ॰रा॰~२.५.६५}\\
\fontsize{14}{21}\selectfont\begin{sloppypar}\hyphenrules{nohyphenation}\justifying\noindent\hspace{10mm} इह प्रेमातिशयात्प्रभुं प्रत्यक्षं मत्वा सर्वं परोक्षं मनुते निषादः। अतो \textcolor{red}{बभूव} इति क्रियां प्रयुङ्क्ते।\footnote{\textcolor{red}{भू सत्तायाम्} (धा॰पा॰~१)~\arrow भू~\arrow \textcolor{red}{शेषात्कर्तरि परस्मैपदम्} (पा॰सू॰~१.३.७८)~\arrow \textcolor{red}{परोक्षे लिट्} (पा॰सू॰~३.२.११५)~\arrow भू~लिट्~\arrow भू~तिप्~\arrow \textcolor{red}{परस्मैपदानां णलतुसुस्थलथुस\-णल्वमाः} (पा॰सू॰~३.४.८२)~\arrow भू~ण~\arrow भू~अ~\arrow \textcolor{red}{भुवो वुग्लुङ्लिटोः} (पा॰सू॰~६.४.८८)~\arrow \textcolor{red}{आद्यन्तौ टकितौ} (पा॰सू॰~१.१.४६)~\arrow भूवुँक्~अ~\arrow भूव्~अ~\arrow \textcolor{red}{लिटि धातोरनभ्यासस्य} (पा॰सू॰~६.१.८)~\arrow भूव्~भूव्~अ~\arrow \textcolor{red}{हलादिः शेषः} (पा॰सू॰~७.४.६०)~\arrow भू~भूव्~अ~\arrow \textcolor{red}{ह्रस्वः} (पा॰सू॰~७.४.५९)~\arrow भु~भूव्~अ~\arrow \textcolor{red}{भवतेरः} (पा॰सू॰~७.४.७३)~\arrow भ~भूव्~अ~\arrow \textcolor{red}{अभ्यासे चर्च} (पा॰सू॰~८.४.५४)~\arrow ब~भूव्~अ~\arrow बभूव।}\end{sloppypar}
\section[शेते]{शेते}
\centering\textcolor{blue}{शयानं कुशपत्रौघसंस्तरे सीतया सह।\nopagebreak\\
यः शेते स्वर्णपर्यङ्के स्वास्तीर्णे भवनोत्तमे॥}\nopagebreak\\
\raggedleft{–~अ॰रा॰~२.६.२}\\
\fontsize{14}{21}\selectfont\begin{sloppypar}\hyphenrules{nohyphenation}\justifying\noindent\hspace{10mm} भूतकालविवक्षया \textcolor{red}{अशेत}\footnote{\textcolor{red}{शीङ् स्वप्ने} (धा॰पा॰~१०३२)~\arrow शी~\arrow \textcolor{red}{अनुदात्तङित आत्मने\-पदम्} (पा॰सू॰~१.३.१२)~\arrow \textcolor{red}{अनद्यतने लङ्} (पा॰सू॰~३.२.१११)~\arrow शी~लङ्~\arrow शी~त~\arrow \textcolor{red}{लुङ्लङ्लृङ्क्ष्वडुदात्तः} (पा॰सू॰~६.४.७१)~\arrow \textcolor{red}{आद्यन्तौ टकितौ} (पा॰सू॰~१.१.४६)~\arrow अट्~शी~त~\arrow अ~शी~त~\arrow \textcolor{red}{कर्तरि शप्} (पा॰सू॰~३.१.६८)~\arrow अ~शी~शप्~त~\arrow \textcolor{red}{शीङः सार्वधातुके गुणः} (पा॰सू॰~७.४.२१)~\arrow अ~शे~शप्~त~\arrow \textcolor{red}{अदिप्रभृतिभ्यः शपः} (पा॰सू॰~२.४.७२)~\arrow अ~शे~त~\arrow अशेत।} इति प्रयोक्तव्ये \textcolor{red}{शेते} इति प्रयुक्तम्। अत्र \textcolor{red}{पुरि लुङ् चास्मे} (पा॰सू॰~३.२.१२२) इति सूत्रेण वर्तमानवद्व्यवहारः।\footnote{\textcolor{red}{पुरा} इत्यध्याहार्यमिति भावः। \textcolor{red}{शीङ् स्वप्ने} (धा॰पा॰~१०३२)~\arrow शी~\arrow \textcolor{red}{अनुदात्तङित आत्मने\-पदम्} (पा॰सू॰~१.३.१२)~\arrow \textcolor{red}{पुरि लुङ् चास्मे} (पा॰सू॰~३.२.१२२)~\arrow शी~लट्~\arrow शी~त~\arrow \textcolor{red}{कर्तरि शप्} (पा॰सू॰~३.१.६८)~\arrow शी~शप्~त~\arrow \textcolor{red}{शीङः सार्वधातुके गुणः} (पा॰सू॰~७.४.२१)~\arrow शे~शप्~त~\arrow \textcolor{red}{अदिप्रभृतिभ्यः शपः} (पा॰सू॰~२.४.७२)~\arrow शे~त~\arrow \textcolor{red}{टित आत्मनेपदानां टेरे} (पा॰सू॰~३.४.७९)~\arrow शे~ते~\arrow शेते।}\end{sloppypar}
\section[प्रार्थयामास]{प्रार्थयामास}
\centering\textcolor{blue}{गुहस्तान्वाहयामास ज्ञातिभिः सहितः स्वयम्।\nopagebreak\\
गङ्गामध्ये गतां गङ्गां प्रार्थयामास जानकी॥}\nopagebreak\\
\raggedleft{–~अ॰रा॰~२.६.२१}\\
\fontsize{14}{21}\selectfont\begin{sloppypar}\hyphenrules{nohyphenation}\justifying\noindent\hspace{10mm} अनुदात्तेत्त्व\-लक्षणमात्मनेपदमनित्यमत एव \textcolor{red}{प्रार्थयामास}।\footnote{\pageref{sec:prarthaya}तमे पृष्ठे \ref{sec:prarthaya} \nameref{sec:prarthaya} इति प्रयोगस्य \pageref{sec:prarthayami}तमे पृष्ठे \ref{sec:prarthayami} \nameref{sec:prarthayami} इति प्रयोगस्य च विमर्शमपि पश्यन्तु।}\end{sloppypar}
\section[वदस्व]{वदस्व}
\centering\textcolor{blue}{भवन्तो यदि जानन्ति किं वक्ष्यामोऽत्र कारणम्।\nopagebreak\\
यत्र मे सुखवासाय भवेत्स्थानं वदस्व तत्॥}\nopagebreak\\
\raggedleft{–~अ॰रा॰~२.६.५०}\\
\fontsize{14}{21}\selectfont\begin{sloppypar}\hyphenrules{nohyphenation}\justifying\noindent\hspace{10mm} \textcolor{red}{भासनोपसम्भाषा\-ज्ञान\-यत्न\-विमत्युपमन्त्रणेषु वदः} (पा॰सू॰~१.३.४७) इत्यनेन भासन आत्मनेपदम्। \textcolor{red}{भासमानो वद इति} तात्पर्यम्।\footnote{\textcolor{red}{वदँ व्यक्तायां वाचि} (धा॰पा॰~१००९)~\arrow वद्~\arrow \textcolor{red}{भासनोपसम्भाषा\-ज्ञान\-यत्न\-विमत्युपमन्त्रणेषु वदः} (पा॰सू॰~१.३.४७)~\arrow \textcolor{red}{लोट् च} (पा॰सू॰~३.३.१६२)~\arrow वद्~लोट्~\arrow वद्~थास्~\arrow \textcolor{red}{कर्तरि शप्} (पा॰सू॰~३.१.६८)~\arrow वद्~शप्~थास्~\arrow वद्~अ~थास्~\arrow \textcolor{red}{थासस्से} (पा॰सू॰~३.४.८०)~\arrow वद्~अ~से~\arrow \textcolor{red}{सवाभ्यां वामौ} (पा॰सू॰~३.४.९१)~\arrow वद्~अ~स्~व~\arrow वदस्व।}\end{sloppypar}
\section[स्थास्यामहे]{स्थास्यामहे}
\centering\textcolor{blue}{वयं स्थास्यामहे तावदागमिष्यसि निश्चयः।\nopagebreak\\
तथेत्युक्त्वा गृहं गत्वा मुनिभिर्यदुदीरितम्॥}\nopagebreak\\
\raggedleft{–~अ॰रा॰~२.६.७३}\\
\fontsize{14}{21}\selectfont\begin{sloppypar}\hyphenrules{nohyphenation}\justifying\noindent\hspace{10mm} \textcolor{red}{अकर्मकाच्च} (पा॰सू॰~१.३.२६) इत्यनेन आत्मनेपदम्।\footnote{\textcolor{red}{उप}\-उपसर्गस्य लोप इति भावः। उप~\textcolor{red}{ष्ठा गतिनिवृत्तौ} (धा॰पा॰~९२८)~\arrow उप~ष्ठा~\arrow \textcolor{red}{धात्वादेः षः सः} (पा॰सू॰~६.१.६४)~\arrow निमित्तापाये नैमित्तिकस्याप्यपायः~\arrow उप~स्था~\arrow \textcolor{red}{अकर्मकाच्च} (पा॰सू॰~१.३.२६)~\arrow \textcolor{red}{लृट् शेषे च} (पा॰सू॰~३.३.१३)~\arrow उप~स्था~लृट्~\arrow उप~स्था~महिङ्~\arrow उप~स्था~महि~\arrow \textcolor{red}{स्यतासी लृलुटोः} (पा॰सू॰~३.१.३३)~\arrow उप~स्था~स्य~महि~\arrow \textcolor{red}{अतो दीर्घो यञि} (पा॰सू॰~७.३.१०१)~\arrow उप~स्था~स्या~महि~\arrow \textcolor{red}{टित आत्मनेपदानां टेरे} (पा॰सू॰~३.४.७९)~\arrow उप~स्था~स्या~महे~\arrow उप~स्थास्यामहे~\arrow \textcolor{red}{विनाऽपि प्रत्ययं पूर्वोत्तर\-पद\-लोपो वक्तव्यः} (वा॰~५.३.८३)~\arrow स्थास्यामहे।} यद्वा \textcolor{red}{प्रकाशन\-स्थेयाख्ययोश्च} (पा॰सू॰~१.३.२३) इत्यनेन स्थेयाख्यायामात्मने\-पदम्।\footnote{प्रक्रिया पूर्ववत्। \textcolor{red}{ष्ठा गतिनिवृत्तौ} (धा॰पा॰~९२८)~\arrow \textcolor{red}{स्था} (पूर्ववत्)~\arrow \textcolor{red}{प्रकाशन\-स्थेयाख्ययोश्च} (पा॰सू॰~१.३.२३)~\arrow \textcolor{red}{लृट् शेषे च} (पा॰सू॰~३.३.१३)~\arrow स्था~लृट्~\arrow स्था~महिङ्~\arrow स्था~महि~\arrow \textcolor{red}{स्यतासी लृलुटोः} (पा॰सू॰~३.१.३३)~\arrow स्था~स्य~महि~\arrow \textcolor{red}{अतो दीर्घो यञि} (पा॰सू॰~७.३.१०१)~\arrow स्था~स्या~महि~\arrow \textcolor{red}{टित आत्मनेपदानां टेरे} (पा॰सू॰~३.४.७९)~\arrow स्था~स्या~महे~\arrow स्थास्यामहे।}\end{sloppypar}
\section[तिष्ठन्ति]{तिष्ठन्ति}
\centering\textcolor{blue}{तच्छ्रुत्वा जातनिर्वेदो विचार्य पुनरागमम्।\nopagebreak\\
मुनयो यत्र तिष्ठन्ति करुणापूर्णमानसाः॥}\nopagebreak\\
\raggedleft{–~अ॰रा॰~२.६.७५}\\
\fontsize{14}{21}\selectfont\begin{sloppypar}\hyphenrules{nohyphenation}\justifying\noindent\hspace{10mm} अत्र \textcolor{red}{तिष्ठन्ति स्म} इति प्रयोगः। अतः \textcolor{red}{लट् स्मे} (पा॰सू॰~३.२.११८) इत्यनेन लड्लकारः।\footnote{\textcolor{red}{स्म} इत्यध्याहार्यमिति भावः। \textcolor{red}{ष्ठा गतिनिवृत्तौ} (धा॰पा॰~९२८)~\arrow ष्ठा~\arrow \textcolor{red}{धात्वादेः षः सः} (पा॰सू॰~६.१.६४)~\arrow निमित्तापाये नैमित्तिकस्याप्यपायः~\arrow स्था~\arrow \textcolor{red}{शेषात्कर्तरि परस्मैपदम्} (पा॰सू॰~१.३.७८)~\arrow \textcolor{red}{लट् स्मे} (पा॰सू॰~३.२.११८)~\arrow स्था~लट्~\arrow स्था~झि~\arrow \textcolor{red}{कर्तरि शप्‌} (पा॰सू॰~३.१.६८)~\arrow स्था~शप्~झि~\arrow स्था~अ~झि~\arrow \textcolor{red}{पाघ्रा\-ध्मास्थाम्ना\-दाण्दृश्यर्त्ति\-सर्त्तिशदसदां पिब\-जिघ्र\-धम\-तिष्ठ\-मन\-यच्छ\-पश्यर्च्छ\-धौ\-शीय\-सीदाः} (पा॰सू॰~७.३.७८)~\arrow तिष्ठ्~अ~झि~\arrow \textcolor{red}{झोऽन्तः} (पा॰सू॰~७.१.३)~\arrow तिष्ठ्~अ~न्ति~\arrow तिष्ठन्ति।}\end{sloppypar}
\section[अस्मि]{अस्मि}
\centering\textcolor{blue}{मुनीनां दर्शनादेव शुद्धान्तःकरणोऽभवम्।\nopagebreak\\
धनुरादीन्परित्यज्य दण्डवत्पतितोऽस्म्यहम्॥}\nopagebreak\\
\raggedleft{–~अ॰रा॰~२.६.७६}\\
\fontsize{14}{21}\selectfont\begin{sloppypar}\hyphenrules{nohyphenation}\justifying\noindent\hspace{10mm} अत्रापि \textcolor{red}{स्म}\-योगे वर्तमानो भावः।\footnote{\textcolor{red}{स्म} इत्यध्याहार्यमिति भावः। \textcolor{red}{असँ भुवि} (धा॰पा॰~१०६५)~\arrow अस्~\arrow \textcolor{red}{शेषात्कर्तरि परस्मैपदम्} (पा॰सू॰~१.३.७८)~\arrow \textcolor{red}{लट् स्मे} (पा॰सू॰~३.२.११८)~\arrow अस्~लट्~\arrow अस्~मिप्~\arrow अस्~मि~\arrow \textcolor{red}{कर्तरि शप्‌} (पा॰सू॰~३.१.६८)~\arrow अस्~शप्~मि~\arrow \textcolor{red}{अदिप्रभृतिभ्यः शपः} (पा॰सू॰~२.४.७२)~\arrow अस्~मि~\arrow अस्मि।} अथवा \textcolor{red}{अस्मि} इत्यव्ययम्।\footnote{तथा च वाचस्पत्ये~– \textcolor{red}{अस्मि। अव्य॰~अस्–मिन्। अहमर्थे, ‘त्वामस्मि वच्मि विदुषां समवायोऽत्र तिष्ठति’ सा॰द॰। ‘अस्मिता’। ‘उडुपेनास्मि सागरम्’ रघुः। ‘ब्रह्मैवास्मि न शोकभाक्।’} योगसूत्रे च~– \textcolor{red}{अविद्यास्मिता\-रागद्वेषाभि\-निवेशाः क्लेशाः} (यो॰सू॰~२.३)।}\end{sloppypar}
\section[रक्षध्वम्]{रक्षध्वम्}
\centering\textcolor{blue}{रक्षध्वं मां मुनिश्रेष्ठा गच्छन्तं निरयार्णवम्।\nopagebreak\\
इत्यग्रे पतितं दृष्ट्वा मामूचुर्मुनिसत्तमाः॥}\nopagebreak\\
\raggedleft{–~अ॰रा॰~२.६.७७}\\
\fontsize{14}{21}\selectfont\begin{sloppypar}\hyphenrules{nohyphenation}\justifying\noindent\hspace{10mm} इह परस्मैपदीय\-\textcolor{red}{रक्ष्‌}\-धातोः (\textcolor{red}{रक्षँ पालने} धा॰पा॰~६५७) आत्मनेपद\-प्रयोगस्तु कर्म\-व्यतिहारे।\footnote{\textcolor{red}{कर्तरि कर्मव्यतिहारे} (पा॰सू॰~१.३.१४) इत्यनेन। \textcolor{red}{रक्षँ पालने} (धा॰पा॰~६५७)~\arrow रक्ष्~\arrow \textcolor{red}{कर्तरि कर्मव्यतिहारे} (पा॰सू॰~१.३.१४)~\arrow \textcolor{red}{लोट् च} (पा॰सू॰~३.३.१६२)~\arrow रक्ष्~लोट्~\arrow रक्ष्~ध्वम्~\arrow \textcolor{red}{कर्तरि शप्‌} (पा॰सू॰~३.१.६८)~\arrow रक्ष्~शप्~ध्वम्~\arrow रक्ष्~अ~ध्वम्~\arrow रक्षध्वम्।}
यद्वा \textcolor{red}{रक्षस इव अध्वा यस्य}\footnote{\textcolor{red}{रक्षः} इत्यस्य राक्षस इत्यर्थः। \textcolor{red}{यातुधानः पुण्यजनो नैऋतो यातुरक्षसी} (अ॰को॰~१.१.६०) इत्यमरः।}
इति बहुव्रीहि\-समासे पृषोदरादित्वाट्टचि शकन्ध्वादित्वात्पर\-रूपे विभक्तिकार्ये \textcolor{red}{रक्षध्वः}।\footnote{रक्षस्~ङस्~अध्वन्~सुँ~\arrow \textcolor{red}{अनेकमन्यपदार्थे} (पा॰सू॰~२.२.२४)~\arrow \textcolor{red}{कृत्तद्धित\-समासाश्च} (पा॰सू॰~१.२.४६)~\arrow प्रातिपदिक\-सञ्ज्ञा~\arrow \textcolor{red}{सुपो धातु\-प्रातिपदिकयोः} (पा॰सू॰~२.४.७१)~\arrow रक्षस्~अध्वन्~सुँ~\arrow \textcolor{red}{शकन्ध्वादिषु पररूपं वाच्यम्} (वा॰~६.१.९४)~\arrow रक्षध्वन्~सुँ~\arrow \textcolor{red}{पृषोदरादीनि यथोपदिष्टम्} (पा॰सू॰~६.३.१०९)~\arrow टज्भावः~\arrow रक्षध्वन्~सुँ~\arrow रक्षध्वन्~टच्~सुँ~\arrow रक्षध्वन्~अ~सुँ~\arrow \textcolor{red}{नस्तद्धिते} (पा॰सू॰~६.४.१४४)~\arrow रक्षध्व्~अ~सुँ~\arrow रक्षध्व~सुँ~\arrow रक्षध्वस्~\arrow \textcolor{red}{ससजुषो रुः} (पा॰सू॰~८.२.६६)~\arrow रक्षध्वरुँ~\arrow \textcolor{red}{खरवसानयोर्विसर्जनीयः} (पा॰सू॰~८.३.१५)~\arrow रक्षध्वः।} तं \textcolor{red}{रक्षध्वम्}।\footnote{रक्षध्व~अम्~\arrow \textcolor{red}{अमि पूर्वः} (पा॰सू॰~६.१.१०७)~\arrow रक्षध्वम्।
} \textcolor{red}{रक्षध्वं मां पात} इति तात्पर्यम्।\footnote{\textcolor{red}{पात}~({\englishfont =}रक्षत) इत्यध्याहार्यमिति भावः।}\end{sloppypar}
\section[निष्क्रमस्व]{निष्क्रमस्व}
\centering\textcolor{blue}{ततो युगसहस्रान्ते ऋषयः पुनरागमन्।\nopagebreak\\
मामूचुर्निष्क्रमस्वेति तच्छ्रुत्वा तूर्णमुत्थितः॥}\nopagebreak\\
\raggedleft{–~अ॰रा॰~२.६.८४}\\
\fontsize{14}{21}\selectfont\begin{sloppypar}\hyphenrules{nohyphenation}\justifying\noindent\hspace{10mm} अस्मिन् प्रयोगे \textcolor{red}{वृत्ति\-सर्ग\-तायनेषु क्रमः} (पा॰सू॰~१.३.३८) इत्यनेनाऽत्मनेपदम्। \textcolor{red}{वर्धमानो निष्क्रमस्व} इति तायने \textcolor{red}{थास्}।\footnote{निस्~\textcolor{red}{क्रमुँ पादविक्षेपे} (धा॰पा॰~१९०५)~\arrow निस्~क्रम्~\arrow \textcolor{red}{वृत्ति\-सर्ग\-तायनेषु क्रमः} (पा॰सू॰~१.३.३८)~\arrow \textcolor{red}{लोट् च} (पा॰सू॰~३.३.१६२)~\arrow निस्~क्रम्~लोट्~\arrow निस्~क्रम्~थास्~\arrow \textcolor{red}{कर्तरि शप्‌} (पा॰सू॰~३.१.६८)~\arrow निस्~क्रम्~शप्~थास्~\arrow निस्~क्रम्~अ~थास्~\arrow \textcolor{red}{थासस्से} (पा॰सू॰~३.४.८०)~\arrow निस्~क्रम्~अ~से~\arrow \textcolor{red}{सवाभ्यां वामौ} (पा॰सू॰~३.४.९१)~\arrow निस्~क्रम्~अ~स्व~\arrow \textcolor{red}{नुम्विसर्जनीय\-शर्व्यवायेऽपि} (पा॰सू॰~८.३.५८)~\arrow निष्~क्रम्~अ~स्व~\arrow निष्क्रमस्व।}\end{sloppypar}
\section[प्रार्थय]{प्रार्थय}
\label{sec:prarthaya}
\centering\textcolor{blue}{मा भैषीरिति मां प्राह ब्रह्महत्याभयं न ते।\nopagebreak\\
मत्पित्रोः सलिलं दत्त्वा नत्वा प्रार्थय जीवितम्॥}\nopagebreak\\
\raggedleft{–~अ॰रा॰~२.७.३९}\\
\fontsize{14}{21}\selectfont\begin{sloppypar}\hyphenrules{nohyphenation}\justifying\noindent\hspace{10mm} \textcolor{red}{प्रार्थयस्व}\footnote{प्र~\textcolor{red}{अर्थँ उपयाच्ञायाम्} (धा॰पा॰~१९०५)~\arrow प्र~अर्थ्~\arrow \textcolor{red}{सत्याप\-पाश\-रूप\-वीणा\-तूल\-श्लोक\-सेना\-लोम\-त्वच\-वर्म\-वर्ण\-चूर्ण\-चुरादिभ्यो णिच्} (पा॰सू॰~३.१.२५)~\arrow प्र~अर्थ्~णिच्~\arrow प्र~अर्थ्~इ~\arrow प्र~अर्थि~\arrow \textcolor{red}{सनाद्यन्ता धातवः} (पा॰सू॰~३.१.३२)~\arrow धातु\-सञ्ज्ञा~\arrow \textcolor{red}{आगर्वादात्मने\-पदिनः} (धा॰पा॰ ग॰सू॰)~\arrow \textcolor{red}{लोट् च} (पा॰सू॰~३.३.१६२)~\arrow प्र~अर्थि~लोट्~\arrow प्र~अर्थि~थास्~\arrow \textcolor{red}{कर्तरि शप्‌} (पा॰सू॰~३.१.६८)~\arrow प्र~अर्थि~शप्~थास्~\arrow प्र~अर्थि~अ~थास्~\arrow \textcolor{red}{सार्वधातुकार्ध\-धातुकयोः} (पा॰सू॰~७.३.८४)~\arrow प्र~अर्थे~अ~थास्~\arrow \textcolor{red}{एचोऽयवायावः} (पा॰सू॰~६.१.७८)~\arrow प्र~अर्थय्~अ~थास्~\arrow \textcolor{red}{थासस्से} (पा॰सू॰~३.४.८०)~\arrow प्र~अर्थय्~अ~से~\arrow \textcolor{red}{सवाभ्यां वामौ} (पा॰सू॰~३.४.९१)~\arrow प्र~अर्थय्~अ~स्व~\arrow \textcolor{red}{अकः सवर्णे दीर्घः} (पा॰सू॰~६.१.१०१)~\arrow प्रार्थय्~अ~स्व~\arrow प्रार्थयस्व।} इति प्रयोक्तव्ये \textcolor{red}{प्रार्थय} इति प्रयोग आत्मने\-पदस्यानित्यत्वात्।\footnote{\textcolor{red}{‘अनुदात्तेत्त्व\-प्रयुक्तमात्मने\-पदमनित्यम्’ इति ज्ञापनार्थोऽयं ङकारः। तेन ‘तालैः शिञ्जद्वलयसुभगैः’ ‘तृष्णे जृम्भसि’ ‘प्रार्थयन्ति शयनोत्थितं प्रियाः’ इत्यादयः प्रयोगा उपपद्यन्ते} (मा॰धा॰वृ॰~२.९)। प्र~अर्थि~\arrow धातु\-सञ्ज्ञा (पूर्ववत्)~\arrow \textcolor{red}{अनुदात्तेत्त्व\-लक्षणमात्मने\-पदमनित्यम्} (प॰शे॰~९३.४)~\arrow \textcolor{red}{शेषात्कर्तरि परस्मैपदम्} (पा॰सू॰~१.३.७८)~\arrow \textcolor{red}{लोट् च} (पा॰सू॰~३.३.१६२)~\arrow प्र~अर्थि~लोट्~\arrow प्र~अर्थि~सिप्~\arrow प्र~अर्थि~सि~\arrow \textcolor{red}{कर्तरि शप्‌} (पा॰सू॰~३.१.६८)~\arrow प्र~अर्थि~शप्~सि~\arrow प्र~अर्थि~अ~सि~\arrow \textcolor{red}{सार्वधातुकार्ध\-धातुकयोः} (पा॰सू॰~७.३.८४)~\arrow प्र~अर्थे~अ~सि~\arrow \textcolor{red}{एचोऽयवायावः} (पा॰सू॰~६.१.७८)~\arrow प्र~अर्थय्~अ~से~\arrow \textcolor{red}{सेर्ह्यपिच्च} (पा॰सू॰~३.४.८७)~\arrow प्र~अर्थय्~अ~हि~\arrow \textcolor{red}{अतो हेः} (पा॰सू॰~६.४.१०५)~\arrow प्र~अर्थय्~अ~\arrow \textcolor{red}{अकः सवर्णे दीर्घः} (पा॰सू॰~६.१.१०१)~\arrow प्रार्थय्~अ~\arrow प्रार्थय।} यद्वा \textcolor{red}{प्रार्थयत इति प्रार्थयः}।\footnote{\textcolor{red}{प्रार्थि}\-धातोरौणादिके \textcolor{red}{अयच्} प्रत्यये। \textcolor{red}{कार्याद्विद्यादनूबन्धम्} (भा॰पा॰सू॰~३.३.१) \textcolor{red}{केचिदविहिता अप्यूह्याः} (वै॰सि॰कौ॰~३१६९) इत्यनुसारमूह्योऽ\-त्राविहितो \textcolor{red}{अयच्} प्रत्ययः। \textcolor{red}{अटच्} (प॰उ॰~४.११४, द॰उ॰~१०.१५) \textcolor{red}{अतच्} (प॰उ॰~३.१०३–१०५, द॰उ॰~६.१४–१६) \textcolor{red}{अभच्} (प॰उ॰~३.११६–१२०, द॰उ॰~७.१८–२२) \textcolor{red}{अमच्} (प॰उ॰~५.७३–७४, द॰उ॰~५.७३–७४) \textcolor{red}{अलच्} (प॰उ॰~५.८१, द॰उ॰~५.८१) \textcolor{red}{असच्} (प॰उ॰~३.१११–११४, द॰उ॰~९.४४–४७) इतिवत्। प्र~अर्थि~\arrow धातु\-सञ्ज्ञा (पूर्ववत्)~\arrow \textcolor{red}{उणादयो बहुलम्} (पा॰सू॰~३.३.१)~\arrow प्र~अर्थि~अयच्~\arrow प्र~अर्थि~अय~\arrow \textcolor{red}{णेरनिटि} (पा॰सू॰~६.४.५१)~\arrow प्र~अर्थ्~अय~\arrow \textcolor{red}{अकः सवर्णे दीर्घः} (पा॰सू॰~६.१.१०१)~\arrow प्रार्थ्~अय~\arrow प्रार्थय~\arrow विभक्ति\-कार्यम्~\arrow प्रार्थय~सुँ~\arrow प्रार्थय~स्~\arrow \textcolor{red}{ससजुषो रुः} (पा॰सू॰~८.२.६६)~\arrow प्रार्थयरुँ~\arrow प्रार्थयर्~\arrow \textcolor{red}{खरवसानयोर्विसर्जनीयः} (पा॰सू॰~८.३.१५)~\arrow प्रार्थयः।} \textcolor{red}{प्रार्थय इवाऽचर} इति \textcolor{red}{प्रार्थय}।\footnote{प्रार्थय~\arrow \textcolor{red}{सर्वप्राति\-पदिकेभ्य आचारे क्विब्वा वक्तव्यः} (वा॰~३.१.११)~\arrow प्रार्थय~क्विँप्~\arrow प्रार्थय~व्~\arrow \textcolor{red}{वेरपृक्तस्य} (पा॰सू॰~६.१.६७)~\arrow प्रार्थय~\arrow \textcolor{red}{सनाद्यन्ता धातवः} (पा॰सू॰~३.१.३२)~\arrow धातुसञ्ज्ञा~\arrow \textcolor{red}{शेषात्कर्तरि परस्मैपदम्} (पा॰सू॰~१.३.७८)~\arrow \textcolor{red}{लोट् च} (पा॰सू॰~३.३.१६२)~\arrow प्रार्थय~लोट्~\arrow प्रार्थय~सिप्~\arrow प्रार्थय~सि~\arrow \textcolor{red}{कर्तरि शप्‌} (पा॰सू॰~३.१.६८)~\arrow प्रार्थय~शप्~सि~\arrow प्रार्थय~अ~सि~\arrow \textcolor{red}{अतो गुणे} (पा॰सू॰~६.१.९७)~\arrow प्रार्थय~सि~\arrow \textcolor{red}{सेर्ह्यपिच्च} (पा॰सू॰~३.४.८७)~\arrow प्रार्थय~हि~\arrow \textcolor{red}{अतो हेः} (पा॰सू॰~६.४.१०५)~\arrow प्रार्थय।} आचार\-क्विबन्त\-प्रत्ययः।\footnote{यद्वा \textcolor{red}{निवृत्त\-प्रेषणाद्धातोः प्राकृतेऽर्थे णिजुच्यते} (वा॰प॰~३.७.६०) इत्यनुसारं प्राकृतेऽर्थे णावन्याभिप्राये क्रियाफले परस्मैपदे लोटि सिपि शपि हौ हेर्लुकि \textcolor{red}{प्रार्थय}। यथा \textcolor{red}{निवृत्त\-प्रेषणाद्धातोर्हेतुमण्णौ शुद्धेन तुल्योऽर्थः। तेन ‘प्रार्थयन्ति शयनोत्थितं प्रियाः’ इत्यादि सिद्धम्। एवं सकर्मकेषु सर्वमूह्यम्} (ल॰सि॰कौ॰~२६०७)। यद्वा \textcolor{red}{प्रार्थनं प्रार्थस्तं कुरु} इति विग्रह आचार\-णिजन्ताद्धातोर्लोटि सिपि शपि हौ हेर्लुकि \textcolor{red}{प्रार्थय}। यथा \textcolor{red}{प्रार्थनं प्रार्थस्तत्करोति णौ ‘प्रार्थयति’ इति} (मा॰धा॰वृ॰~२.९) \textcolor{red}{‘प्रार्थयन्ति शयनोत्थितं प्रियाः’ इत्यादि कृदन्तात्तत्करोतीति णिचि नेयम्} (मा॰धा॰वृ॰~१०.२८७)। यत्तु तत्त्व\-बोधिन्यां \textcolor{red}{केचित्तु परस्मैपद\-सिद्ध्यर्थं प्रार्थनं प्रार्थस्तं कुर्वन्ति प्रार्थयन्तीति व्याचक्षते तदसत्। धातुसंज्ञा\-प्रयोजक\-प्रत्यये चिकीर्षित उपसर्गाणां पृथक्करणस्य वक्ष्यमाणतया ‘अर्थवेदे’त्यापुगागमस्य दुर्वारत्वात्}। तच्चिन्त्यम्। \textcolor{red}{प्रार्थयित्वा अप्रार्थयत्} इत्यादीनां शिष्टप्रयुक्तत्वात् यथा \textcolor{red}{अप्रार्थयत्कामधेनुम्} (अ॰पु॰~४.१६) \textcolor{red}{प्रार्थयित्वा द्विजान् भोज्य} (अ॰पु॰~१८४.८) \textcolor{red}{प्रार्थयित्वा विरोधितम्} (अ॰शा॰~८.५.२८) \textcolor{red}{प्रार्थयित्वा द्विजान्नृपान्} (ग॰सं॰~१०.५७.१) \textcolor{red}{प्रार्थयित्वा निजेश्वरम्} (ना॰पु॰~६६.१४) \textcolor{red}{तं प्रार्थयित्वा विधिवत्} (ब्रह्मा॰पु॰~२.५४.२८) इत्यादिषु। \textcolor{red}{प्रार्थाप्य अप्रार्थापयत्} इत्यादि\-प्रयोगाणामनु\-पलब्धेश्च।
\pageref{sec:prarthayami}तमे पृष्ठे \ref{sec:prarthayami} \nameref{sec:prarthayami} इति प्रयोगस्य विमर्शमपि पश्यन्तु।}\end{sloppypar}
\section[पृच्छसे]{पृच्छसे}
\centering\textcolor{blue}{तथाऽपि पृच्छसे किञ्चित्तदनुग्रह एव मे।\nopagebreak\\
कैकेय्या मत्कृतं कर्म रामराज्यविघातनम्॥}\nopagebreak\\
\raggedleft{–~अ॰रा॰~२.८.४६}\\
\fontsize{14}{21}\selectfont\begin{sloppypar}\hyphenrules{nohyphenation}\justifying\noindent\hspace{10mm} अत्रापि \textcolor{red}{कर्तरि कर्म\-व्यतिहारे} (पा॰सू॰~१.३.१४) इत्यनेनैवाऽत्मनेपदम्।\footnote{\textcolor{red}{प्रच्छँ ज्ञीप्सायाम्} (धा॰पा॰~१४१३)~\arrow प्रच्छ्~\arrow \textcolor{red}{कर्तरि कर्मव्यतिहारे} (पा॰सू॰~१.३.१४)~\arrow \textcolor{red}{वर्तमाने लट्} (पा॰सू॰~३.२.१२३)~\arrow प्रच्छ्~लट्~\arrow प्रच्छ्~थास्~\arrow \textcolor{red}{तुदादिभ्यः शः} (पा॰सू॰~३.१.७७)~\arrow प्रच्छ्~श~थास्~\arrow प्रच्छ्~अ~थास्~\arrow \textcolor{red}{सार्वधातुकमपित्} (पा॰सू॰~१.२.४)~\arrow ङिद्वत्त्वम्~\arrow \textcolor{red}{ग्रहिज्या\-वयिव्यधि\-वष्टिविचति\-वृश्चति\-पृच्छति\-भृज्जतीनां ङिति च} (पा॰सू॰~६.१.१६)~\arrow पृ~अ~च्छ्~अ~थास्~\arrow \textcolor{red}{सम्प्रसारणाच्च} (पा॰सू॰~६.१.१०८)~\arrow पृ~च्छ्~अ~थास्~\arrow \textcolor{red}{थासस्से} (पा॰सू॰~३.४.८०)~\arrow पृ~च्छ्~अ~से~\arrow पृच्छसे।} यतो हि \textcolor{red}{ज्ञात्वाऽपि सर्वं त्वं प्राकृत इव प्रश्नं करोषि} इति तात्पर्यम्।\end{sloppypar}
\section[नेष्ये]{नेष्ये}
\centering\textcolor{blue}{अभिषेक्ष्ये वसिष्ठाद्यैः पौरजानपदैः सह।\nopagebreak\\
नेष्येऽयोध्यां रमानाथं दासः सेवेऽतिनीचवत्॥}\nopagebreak\\
\raggedleft{–~अ॰रा॰~२.८.५१}\\
\fontsize{14}{21}\selectfont\begin{sloppypar}\hyphenrules{nohyphenation}\justifying\noindent\hspace{10mm} अस्मिन्प्रयोगे \textcolor{red}{सम्माननोत्सञ्जनाचार्य\-करण\-ज्ञान\-भृति\-विगणन\-व्ययेषु नियः} (पा॰सू॰~१.३.३६) इत्यादिनोत्सञ्जनार्थमात्मनेपदम्।\footnote{ञित्त्वात्कर्त्रभिप्राये क्रियाफले तु सिद्धमेवात्मने\-पदम्। परन्त्वत्र कर्त्रभिप्रायो न। क्रियाफलं त्वत्र राममेवाभिप्रैति न कर्तारं मामिति भक्तशिरोमणेर्भरतस्य भावात्। अकर्त्रभिप्राये कथमात्मने\-पदमित्याशङ्क्याहुः समाधानम्। \textcolor{red}{णीञ् प्रापणे} (धा॰पा॰~९०१)~\arrow णी~\arrow \textcolor{red}{णो नः} (पा॰सू॰~६.१.६५)~\arrow नी~\arrow \textcolor{red}{सम्माननोत्सञ्जनाचार्य\-करण\-ज्ञान\-भृति\-विगणन\-व्ययेषु नियः} (पा॰सू॰~१.३.३६)~\arrow \textcolor{red}{लृट् शेषे च} (पा॰सू॰~३.३.१३)~\arrow नी~लृट्~\arrow नी~इट्~\arrow नी~इ~\arrow \textcolor{red}{स्यतासी लृलुटोः} (पा॰सू॰~३.१.३३)~\arrow नी~स्य~इ~\arrow \textcolor{red}{सार्वधातुकार्ध\-धातुकयोः} (पा॰सू॰~७.३.८४)~\arrow ने~स्य~इ~\arrow \textcolor{red}{टित आत्मनेपदानां टेरे} (पा॰सू॰~३.४.७९)~\arrow ने~स्य~ए~\arrow \textcolor{red}{अतो गुणे} (पा॰सू॰~६.१.९७)~\arrow ने~स्ये~\arrow \textcolor{red}{आदेश\-प्रत्यययोः} (पा॰सू॰~८.३.५९)~\arrow ने~ष्ये~\arrow नेष्ये।} \textcolor{red}{सेवे} इत्यत्र च वर्तमान\-सामीप्ये\footnote{\textcolor{red}{वर्तमान\-सामीप्ये वर्तमानवद्वा} (पा॰सू॰~३.३.१३१) इत्यनेन। \textcolor{red}{षेवृँ सेवने} (धा॰पा॰~५०१)~\arrow षेव्~\arrow \textcolor{red}{धात्वादेः षः सः} (पा॰सू॰~६.१.६४)~\arrow सेव्~\arrow \textcolor{red}{अनुदात्तङित आत्मने\-पदम्} (पा॰सू॰~१.३.१२)~\arrow \textcolor{red}{लट् स्मे} (पा॰सू॰~३.२.११८)~\arrow सेव्~लट्~\arrow सेव्~इट्~\arrow सेव्~इ~\arrow \textcolor{red}{कर्तरि शप्‌} (पा॰सू॰~३.१.६८)~\arrow सेव्~शप्~इ~\arrow सेव्~अ~इ~\arrow \textcolor{red}{टित आत्मनेपदानां टेरे} (पा॰सू॰~३.४.७९)~\arrow सेव्~ए~\arrow \textcolor{red}{अतो गुणे} (पा॰सू॰~६.१.९७)~\arrow सेवे।} यद्वा \textcolor{red}{स्म} इति योगे भविष्यत्काले लट्।\footnote{\textcolor{red}{स्म} इत्यध्याहार्यमिति भावः। प्रक्रिया पूर्ववत्।}\end{sloppypar}
\section[भाषयेत्]{भाषयेत्}
\centering\textcolor{blue}{सर्वं देवकृतं नोचेदेवं सा भाषयेत्कथम्।\nopagebreak\\
तस्मात्त्यजाग्रहं तात रामस्य विनिवर्तने॥}\nopagebreak\\
\raggedleft{–~अ॰रा॰~२.९.४६}\\
\fontsize{14}{21}\selectfont\begin{sloppypar}\hyphenrules{nohyphenation}\justifying\noindent\hspace{10mm} इह स्वार्थे णिचि परस्मैपदम्।\footnote{ \textcolor{red}{भाषँ व्यक्तायां वाचि} (धा॰पा॰~६१२)~\arrow भाष्~\arrow स्वार्थे णिच्~\arrow भाष्~णिच्~\arrow भाष्~इ~\arrow भाषि~\arrow \textcolor{red}{सनाद्यन्ता धातवः} (पा॰सू॰~३.१.३२)~\arrow धातु\-सञ्ज्ञा~\arrow \textcolor{red}{शेषात्कर्तरि परस्मैपदम्} (पा॰सू॰~१.३.७८)~\arrow \textcolor{red}{विधि\-निमन्‍त्रणामन्‍त्रणाधीष्‍ट\-सम्प्रश्‍न\-प्रार्थनेषु लिङ्} (पा॰सू॰~३.३.१६१)~\arrow भाषि~लिङ्~\arrow भाषि~तिप्~\arrow भाषि~ति~\arrow \textcolor{red}{कर्तरि शप्‌} (पा॰सू॰~३.१.६८)~\arrow भाषि~शप्~ति~\arrow भाषि~अ~ति~\arrow \textcolor{red}{सार्वधातुकार्ध\-धातुकयोः} (पा॰सू॰~७.३.८४)~\arrow भाषे~अ~ति~\arrow \textcolor{red}{यासुट् परस्मैपदेषूदात्तो ङिच्च} (पा॰सू॰~३.४.१०३)~\arrow \textcolor{red}{आद्यन्तौ टकितौ} (पा॰सू॰~१.१.४६)~\arrow भाषे~अ~यासुँट्~ति~\arrow भाषे~अ~यास्~ति~\arrow \textcolor{red}{सुट् तिथोः} (पा॰सू॰~३.४.१०७)~\arrow \textcolor{red}{आद्यन्तौ टकितौ} (पा॰सू॰~१.१.४६)~\arrow भाषे~अ~यास्~सुँट्~ति~\arrow भाषे~अ~यास्~स्~ति~\arrow \textcolor{red}{लिङः सलोपोऽनन्त्यस्य} (पा॰सू॰~७.२.७९)~\arrow भाषे~अ~या~ति~\arrow\textcolor{red}{अतो येयः} (पा॰सू॰~७.२.८०)~\arrow भाषे~अ~इय्~ति~\arrow \textcolor{red}{लोपो व्योर्वलि} (पा॰सू॰~६.१.६६)~\arrow भाषे~अ~इ~ति~\arrow \textcolor{red}{एचोऽयवायावः} (पा॰सू॰~६.१.७८)~\arrow भाषय्~अ~इ~ति~\arrow \textcolor{red}{आद्गुणः} (पा॰सू॰~६.१.८७)~\arrow भाषय्~ए~ति~\arrow \textcolor{red}{इतश्च} (पा॰सू॰~३.४.१००)~\arrow भाषय्~ए~त्~\arrow भाषयेत्। न च चुरादिगणे पाठाभावात्स्वार्थे न णिच्। शिष्टप्रयोगेषु दृश्यते। यथा भारते वनपर्वणि भीमं प्रति नव\-व्याकरणार्थ\-वेत्ता हनुमान्~– \textcolor{red}{दशवर्षसहस्राणि दशवर्षशतानि च। राज्यं कारितवान् रामस्ततः स्वभवनं गतः॥} (म॰भा॰~३.१४८.१९) । अत्र नीलकण्ठोऽपि~– \textcolor{red}{कारितवान् कृतवान्। स्वार्थे णिच्} (म॰भा॰ भा॰दी॰~३.१४८.१९)। 
}\end{sloppypar}
\section[त्यक्ष्यते]{त्यक्ष्यते}
\centering\textcolor{blue}{अङ्गरागं च सीतायै ददौ दिव्यं शुभानना।\nopagebreak\\
न त्यक्ष्यतेऽङ्गरागेण शोभा त्वां कमलानने॥}\nopagebreak\\
\raggedleft{–~अ॰रा॰~२.९.८९}\\
\fontsize{14}{21}\selectfont\begin{sloppypar}\hyphenrules{nohyphenation}\justifying\noindent\hspace{10mm} चित्रकूटं परित्यज्य दण्डकारण्यं गन्तुकामः श्रीरामोऽत्रि\-दर्शनं करोति। तत्रानुसूया सीताया अङ्गरागं प्रयच्छति। इह \textcolor{red}{त्यक्ष्यते} इति प्रयोगो विमृश्यते। \textcolor{red}{त्यज्‌}\-धातुः (\textcolor{red}{त्यजँ हानौ} धा॰पा॰~९८६) परस्मैपदीयः। अत्र कर्म\-व्यतिहार आत्मनेपदम्।\footnote{त्यज्~\arrow \textcolor{red}{कर्तरि कर्मव्यतिहारे} (पा॰सू॰~१.३.१४)~\arrow \textcolor{red}{लृट् शेषे च} (पा॰सू॰~३.३.१३)~\arrow त्यज्~लृट्~\arrow त्यज्~त~\arrow \textcolor{red}{स्यतासी लृलुटोः} (पा॰सू॰~३.१.३३)~\arrow त्यज्~स्य~त~\arrow \textcolor{red}{चोः कुः} (पा॰सू॰~८.२.३०)~\arrow त्यक्~स्य~त~\arrow \textcolor{red}{आदेश\-प्रत्यययोः} (पा॰सू॰~८.३.५९)~\arrow त्यक्~ष्य~त~\arrow \textcolor{red}{टित आत्मनेपदानां टेरे} (पा॰सू॰~३.४.७९)~\arrow त्यक्ष्यते। सामान्यतस्तु \textcolor{red}{त्यक्ष्यति} इति रूपम्। यथा \textcolor{red}{सोऽपि शोकसमाविष्टो ननु त्यक्ष्यति जीवितम्} (वा॰रा॰~२.६६.११) इति वाल्मीकि\-प्रयोगे। त्यज्~\arrow \textcolor{red}{शेषात्कर्तरि परस्मैपदम्} (पा॰सू॰~१.३.७८)~\arrow \textcolor{red}{लृट् शेषे च} (पा॰सू॰~३.३.१३)~\arrow त्यज्~लृट्~\arrow त्यज्~तिप्~\arrow त्यज्~ति~\arrow \textcolor{red}{स्यतासी लृलुटोः} (पा॰सू॰~३.१.३३)~\arrow त्यज्~स्य~ति~\arrow \textcolor{red}{चोः कुः} (पा॰सू॰~८.२.३०)~\arrow तयक्~स्य~ति~\arrow \textcolor{red}{आदेश\-प्रत्यययोः} (पा॰सू॰~८.३.५९)~\arrow तयक्~ष्य~ति~\arrow त्यक्ष्यति।} यद्वा कर्म\-कर्तृक\-प्रयोगे \textcolor{red}{त्वामधिश्रित्य शोभा स्वयमेव न त्यक्ष्यते त्वया किम्} इत्यात्मनेपदम्।\footnote{\textcolor{red}{अधिश्रित्य} इत्यध्याहार्यमिति शेषः। त्यज्~\arrow \textcolor{red}{कर्मवत्कर्मणा तुल्यक्रियः} (पा॰सू॰~३.१.८७)~\arrow \textcolor{red}{भावकर्मणोः} (पा॰सू॰~१.३.१३)~\arrow \textcolor{red}{लृट् शेषे च} (पा॰सू॰~३.३.१३)~\arrow त्यज्~लृट्~\arrow त्यज्~त~\arrow \textcolor{red}{स्यतासी लृलुटोः} (पा॰सू॰~३.१.३३)~\arrow त्यज्~स्य~त~\arrow \textcolor{red}{चोः कुः} (पा॰सू॰~८.२.३०)~\arrow तयक्~स्य~त~\arrow \textcolor{red}{आदेश\-प्रत्यययोः} (पा॰सू॰~८.३.५९)~\arrow तयक्~ष्य~त~\arrow \textcolor{red}{टित आत्मनेपदानां टेरे} (पा॰सू॰~३.४.७९)~\arrow त्यक्ष्यते।}\end{sloppypar}
\vspace{2mm}
\centering ॥ इत्ययोध्याकाण्डीयप्रयोगाणां विमर्शः ॥\nopagebreak\\
\vspace{4mm}
\centering इत्यध्यात्म\-रामायणेऽपाणिनीय\-प्रयोगाणां\-विमर्श\-नामके शोध\-प्रबन्धे तृतीयाध्याये प्रथम\-परिच्छेदः।\\
\pagebreak
\pdfbookmark[1]{द्वितीयः परिच्छेदः}{Chap3Part2}
\phantomsection
\addtocontents{toc}{\protect\setcounter{tocdepth}{1}}
\addcontentsline{toc}{section}{द्वितीयः परिच्छेदः}
\addtocontents{toc}{\protect\setcounter{tocdepth}{0}}
\centering ॥ अथ तृतीयाध्याये द्वितीयः परिच्छेदः ॥\nopagebreak\\
\vspace{4mm}
\pdfbookmark[2]{अरण्यकाण्डम्}{Chap3Part2Kanda3}
\phantomsection
\addtocontents{toc}{\protect\setcounter{tocdepth}{2}}
\addcontentsline{toc}{subsection}{अरण्यकाण्डीयप्रयोगाणां विमर्शः}
\addtocontents{toc}{\protect\setcounter{tocdepth}{0}}
\centering ॥ अथारण्यकाण्डीयप्रयोगाणां विमर्शः ॥\nopagebreak\\
\section[गच्छामहे]{गच्छामहे}
\centering\textcolor{blue}{मुने गच्छामहे सर्वे मुनिमण्डलमण्डितम्।\nopagebreak\\
विपिनं दण्डकं यत्र त्वमाज्ञातुमिहार्हसि॥}\nopagebreak\\
\raggedleft{–~अ॰रा॰~३.१.२}\\
\fontsize{14}{21}\selectfont\begin{sloppypar}\hyphenrules{nohyphenation}\justifying\noindent\hspace{10mm} इह श्रीरामभद्रोऽत्रिमाज्ञां याचते। अत्र \textcolor{red}{गच्छामहे} इति प्रयोगोऽपि नाऽपाणिनीयः। यद्यपि परस्मैपदत्वादात्मनेपदं न \textcolor{red}{गच्छामः}\footnote{\textcolor{red}{गमॢँ गतौ} (धा॰पा॰~९८२)~\arrow गम्~\arrow \textcolor{red}{शेषात्कर्तरि परस्मैपदम्} (पा॰सू॰~१.३.७८)~\arrow \textcolor{red}{वर्तमान\-सामीप्ये वर्तमानवद्वा} (पा॰सू॰~३.३.१३१)~\arrow \textcolor{red}{वर्तमाने लट्} (पा॰सू॰~३.२.१२३)~\arrow गम्~लट्~\arrow गम्~मस्~\arrow \textcolor{red}{कर्तरि शप्‌} (पा॰सू॰~३.१.६८)~\arrow गम्~शप्~मस्~\arrow गम्~अ~मस्~\arrow \textcolor{red}{इषुगमियमां छः} (पा॰सू॰~७.३.७७)~\arrow गछ्~अ~मस्~\arrow \textcolor{red}{छे च} (पा॰सू॰~६.१.७३)~\arrow \textcolor{red}{आद्यन्तौ टकितौ} (पा॰सू॰~१.१.४६)~\arrow गतुँक्~छ्~अ~मस्~\arrow गत्~छ्~अ~मस्~\arrow \textcolor{red}{स्तोः श्चुना श्चुः} (पा॰सू॰~८.४.४०)~\arrow गच्~छ्~अ~मस्~\arrow \textcolor{red}{अतो दीर्घो यञि} (पा॰सू॰~७.३.१०१)~\arrow गच्~छ्~आ~मस्~\arrow \textcolor{red}{ससजुषो रुः} (पा॰सू॰~८.२.६६)~\arrow गच्~छ्~आ~मरुँ~\arrow \textcolor{red}{खरवसानयोर्विसर्जनीयः} (पा॰सू॰~८.३.१५)~\arrow गच्~छ्~आ~मः~\arrow गच्छामः।} इति सर्व\-विदितं तथाऽपि \textcolor{red}{सम्‌}\-उपसर्ग\-संयोजने \textcolor{red}{समो गम्यृच्छिभ्याम्} (पा॰सू॰~१.३.२९) इत्यनेनाऽत्मनेपदम्। लुप्तत्वात् समुपसर्गः न श्रूयते।\footnote{सम् \textcolor{red}{गमॢँ गतौ} (धा॰पा॰~९८२)~\arrow सम्~गम्~\arrow \textcolor{red}{समो गम्यृच्छिभ्याम्} (पा॰सू॰~१.३.२९)~\arrow \textcolor{red}{वर्तमान\-सामीप्ये वर्तमानवद्वा} (पा॰सू॰~३.३.१३१)~\arrow \textcolor{red}{वर्तमाने लट्} (पा॰सू॰~३.२.१२३)~\arrow सम्~गम्~लट्~\arrow सम्~गम्~महिङ्~\arrow सम्~गम्~महि~\arrow \textcolor{red}{कर्तरि शप्‌} (पा॰सू॰~३.१.६८)~\arrow सम्~गम्~शप्~महि~\arrow सम्~गम्~अ~महि~\arrow \textcolor{red}{इषुगमियमां छः} (पा॰सू॰~७.३.७७)~\arrow सम्~गछ्~अ~महि~\arrow \textcolor{red}{छे च} (पा॰सू॰~६.१.७३)~\arrow \textcolor{red}{आद्यन्तौ टकितौ} (पा॰सू॰~१.१.४६)~\arrow सम्~गतुँक्~छ्~अ~महि~\arrow सम्~गत्~छ्~अ~महि~\arrow \textcolor{red}{स्तोः श्चुना श्चुः} (पा॰सू॰~८.४.४०)~\arrow सम्~गच्~छ्~अ~महि~\arrow \textcolor{red}{अतो दीर्घो यञि} (पा॰सू॰~७.३.१०१)~\arrow सम्~गच्~छ्~आ~महि~\arrow \textcolor{red}{टित आत्मनेपदानां टेरे} (पा॰सू॰~३.४.७९)~\arrow सम्~गच्~छ्~आ~महे~\arrow \textcolor{red}{विनाऽपि प्रत्ययं पूर्वोत्तर\-पद\-लोपो वक्तव्यः} (वा॰~५.३.८३)~\arrow गच्~छ्~आ~महे~\arrow गच्छामहे।} लडपि \textcolor{red}{वर्तमान\-सामीप्ये वर्तमानवद्वा} (पा॰सू॰~३.३.१३१) इत्यनेन। \textcolor{red}{मुनि\-मण्डलमाश्रित्याति\-शीघ्रं सङ्गता भविष्यामः} इति रामभद्रस्य तात्पर्यम्।\footnote{यद्वा \textcolor{red}{कर्तरि कर्मव्यतिहारे} (पा॰सू॰~१.३.१४) इत्यनेनात्मने\-पदम्। \pageref{sec:yasye}तमे पृष्ठे \ref{sec:yasye} \nameref{sec:yasye} इति प्रयोगस्य विमर्शं पश्यन्तु।}\end{sloppypar}
\section[तारयिष्यामहे]{तारयिष्यामहे}
\centering\textcolor{blue}{तारयिष्यामहे युष्मान्वयमेव क्षणादिह।\nopagebreak\\
ततो नावि समारोप्य सीतां राघवलक्ष्मणौ॥}\nopagebreak\\
\raggedleft{–~अ॰रा॰~३.१.८}\\
\fontsize{14}{21}\selectfont\begin{sloppypar}\hyphenrules{nohyphenation}\justifying\noindent\hspace{10mm} अत्र कर्म\-व्यतिहार आत्मनेपदम्।\footnote{\textcolor{red}{कर्तरि कर्मव्यतिहारे} (पा॰सू॰~१.३.१४) इत्यनेन।} \textcolor{red}{निखिलं जगत्संसार\-सागराद्भवान् तारयति किन्तु भवन्तमपि वयं सरितस्तारयिष्यामहे} इति क्रिया\-व्यत्यय आत्मनेपदम्।\footnote{\textcolor{red}{तॄ प्लवनतरणयोः} (धा॰पा॰~९६९)~\arrow तॄ~\arrow \textcolor{red}{हेतुमति च} (पा॰सू॰~३.१.२६)~\arrow तॄ~णिच्~\arrow तॄ~इ~\arrow \textcolor{red}{अचो ञ्णिति} (पा॰सू॰~७.२.११५)~\arrow ता~इ~\arrow \textcolor{red}{उरण् रपरः} (पा॰सू॰~१.१.५१)~\arrow तार्~इ~\arrow तारि~\arrow \textcolor{red}{सनाद्यन्ता धातवः} (पा॰सू॰~३.१.३२)~\arrow धातु\-सञ्ज्ञा~\arrow \textcolor{red}{कर्तरि कर्मव्यतिहारे} (पा॰सू॰~१.३.१४)~\arrow \textcolor{red}{लृट् शेषे च} (पा॰सू॰~३.३.१३)~\arrow तारि~लृट्~\arrow तारि~महिङ्~\arrow तारि~महि~\arrow \textcolor{red}{स्यतासी लृलुटोः} (पा॰सू॰~३.१.३३)~\arrow तारि~स्य~महि~\arrow \textcolor{red}{आर्धधातुकस्येड्वलादेः} (पा॰सू॰~७.२.३५)~\arrow तारि~इट्~स्य~महि~\arrow तारि~इ~स्य~महि~\arrow \textcolor{red}{सार्वधातुकार्ध\-धातुकयोः} (पा॰सू॰~७.३.८४)~\arrow तारे~इ~स्य~महि~\arrow \textcolor{red}{एचोऽयवायावः} (पा॰सू॰~६.१.७८)~\arrow तारय्~इ~स्य~महि~\arrow \textcolor{red}{अतो दीर्घो यञि} (पा॰सू॰~७.३.१०१)~\arrow तारय्~इ~स्या~महि~\arrow \textcolor{red}{टित आत्मनेपदानां टेरे} (पा॰सू॰~३.४.७९)~\arrow तारय्~इ~स्या~महे~\arrow \textcolor{red}{आदेश\-प्रत्यययोः} (पा॰सू॰~८.३.५९)~\arrow तारय्~इ~ष्या~महे~\arrow तारयिष्यामहे।}\end{sloppypar}
\section[आस्ते]{आस्ते}
\centering\textcolor{blue}{इत्येवं भाषमाणौ तौ जग्मतुः सार्धयोजनम्।\nopagebreak\\
तत्रैका पुष्करिण्यास्ते कह्लारकुमुदोत्पलैः॥}\nopagebreak\\
\raggedleft{–~अ॰रा॰~३.१.१५}\\
\fontsize{14}{21}\selectfont\begin{sloppypar}\hyphenrules{nohyphenation}\justifying\noindent\hspace{10mm} \textcolor{red}{आस्त}\footnote{\textcolor{red}{आसँ उपवेशने} (धा॰पा॰~१०२१)~\arrow आस्~\arrow \textcolor{red}{अनुदात्तङित आत्मने\-पदम्} (पा॰सू॰~१.३.१२)~\arrow \textcolor{red}{अनद्यतने लङ्} (पा॰सू॰~३.२.१११)~\arrow आस्~लङ्~\arrow आस्~त~\arrow \textcolor{red}{आडजादीनाम्} (पा॰सू॰~६.४.७२)~\arrow \textcolor{red}{आद्यन्तौ टकितौ} (पा॰सू॰~१.१.४६)~\arrow आट्~आस्~त~\arrow आ~आस्~त~\arrow \textcolor{red}{कर्तरि शप्‌} (पा॰सू॰~३.१.६८)~\arrow आ~आस्~शप्~त~\arrow \textcolor{red}{अदिप्रभृतिभ्यः शपः} (पा॰सू॰~२.४.७२)~\arrow आ~आस्~त~\arrow \textcolor{red}{अकः सवर्णे दीर्घः} (पा॰सू॰~६.१.१०१)~\arrow आस्~त~\arrow आस्त।
} इति प्रयोक्तव्ये \textcolor{red}{स्म}\-योगे लड्लकारः।\footnote{\textcolor{red}{स्म} इत्यध्याहार्यमिति भावः। \textcolor{red}{आसँ उपवेशने} (धा॰पा॰~१०२१)~\arrow आस्~\arrow \textcolor{red}{अनुदात्तङित आत्मने\-पदम्} (पा॰सू॰~१.३.१२)~\arrow \textcolor{red}{लट् स्मे} (पा॰सू॰~३.२.११८)~\arrow आस्~लट्~\arrow आस्~त~\arrow \textcolor{red}{कर्तरि शप्‌} (पा॰सू॰~३.१.६८)~\arrow आस्~शप्~त~\arrow \textcolor{red}{अदिप्रभृतिभ्यः शपः} (पा॰सू॰~२.४.७२)~\arrow आस्~त~\arrow \textcolor{red}{टित आत्मनेपदानां टेरे} (पा॰सू॰~३.४.७९)~\arrow आस्~ते~\arrow आस्ते।
}\end{sloppypar}
\section[पलायतम्]{पलायतम्}
\centering\textcolor{blue}{यदि जीवितुमिच्छाऽस्ति त्यक्त्वा सीतां निरायुधौ।\nopagebreak\\
पलायतं न चेच्छीघ्रं भक्षयामि युवामहम्॥}\nopagebreak\\
\raggedleft{–~अ॰रा॰~३.१.२९}\\
\fontsize{14}{21}\selectfont\begin{sloppypar}\hyphenrules{nohyphenation}\justifying\noindent\hspace{10mm} अनुदात्तेत्त्व\-लक्षणस्याऽत्मनेपदस्यानित्यत्वात्परस्मैपदे लोड्लकारे मध्यम\-पुरुषे द्वि\-वचने थसस्तमादेशे \textcolor{red}{पलायतम्}।\footnote{\textcolor{red}{अनुदात्तेत्त्व\-लक्षणमात्मने\-पदमनित्यम्} (प॰शे॰~९३.४)। परा~\textcolor{red}{अयँ गतौ} (धा॰पा॰~४७४)~\arrow परा~अय्~\arrow \textcolor{red}{अनुदात्तेत्त्व\-लक्षणमात्मने\-पदमनित्यम्} (प॰शे॰~९३.४)~\arrow \textcolor{red}{शेषात्कर्तरि परस्मैपदम्} (पा॰सू॰~१.३.७८)~\arrow \textcolor{red}{लोट् च} (पा॰सू॰~३.३.१६२)~\arrow परा~अय्~लोट्~\arrow परा~अय्~थस्~\arrow \textcolor{red}{कर्तरि शप्‌} (पा॰सू॰~३.१.६८)~\arrow परा~अय्~शप्~तस्~\arrow परा~अय्~अ~तस्~\arrow \textcolor{red}{तस्थस्थमिपां तान्तन्तामः} (पा॰सू॰~३.४.१०१)~\arrow परा~अय्~अ~तम्~\arrow \textcolor{red}{उपसर्गस्यायतौ} (पा॰सू॰~८.२.१९)~\arrow पला~अय्~अ~तम्~\arrow \textcolor{red}{अकः सवर्णे दीर्घः} (पा॰सू॰~६.१.१०१)~\arrow पलाय्~अ~तम्~\arrow पलायतम्। सामान्यतस्तु \textcolor{red}{पलायेथाम्} इति रूपम्। यथा \textcolor{red}{त्वरमाणौ पलायेथां न वां जीवितमाददे} (वा॰रा॰~३.३.८) इति वाल्मीकि\-प्रयोगे। परा~\textcolor{red}{अयँ गतौ} (धा॰पा॰~४७४)~\arrow परा~अय्~\arrow \textcolor{red}{अनुदात्तङित आत्मने\-पदम्} (पा॰सू॰~१.३.१२)~\arrow \textcolor{red}{लोट् च} (पा॰सू॰~३.३.१६२)~\arrow परा~अय्~लोट्~\arrow परा~अय्~आथाम्~\arrow \textcolor{red}{कर्तरि शप्‌} (पा॰सू॰~३.१.६८)~\arrow परा~अय्~शप्~आथाम्~\arrow परा~अय्~अ~आथाम्~\arrow \textcolor{red}{लोटो लङ्वत्‌} (पा॰सू॰~३.४.८५)~\arrow ङिद्वत्त्वम्~\arrow \textcolor{red}{आतो ङितः} (पा॰सू॰~७.२.८१)~\arrow परा~अय्~अ~इय्~थाम्~\arrow \textcolor{red}{लोपो व्योर्वलि} (पा॰सू॰~६.१.६६)~\arrow परा~अय्~अ~इ~थाम्~\arrow \textcolor{red}{आद्गुणः} (पा॰सू॰~६.१.८७)~\arrow परा~अय्~ए~थाम्~\arrow \textcolor{red}{उपसर्गस्यायतौ} (पा॰सू॰~८.२.१९)~\arrow पला~अय्~ए~थाम्~\arrow \textcolor{red}{अकः सवर्णे दीर्घः} (पा॰सू॰~६.१.१०१)~\arrow पलाय्~ए~थाम्~\arrow पलायेथाम्।}\end{sloppypar}
\section[अभिदुद्रुवे]{अभिदुद्रुवे}
\centering\textcolor{blue}{इत्युक्त्वा राक्षसः सीतामादातुमभिदुद्रुवे।\nopagebreak\\
रामश्चिच्छेद तद्बाहू शरेण प्रहसन्निव॥}\nopagebreak\\
\raggedleft{–~अ॰रा॰~३.१.३०}\\
\fontsize{14}{21}\selectfont\begin{sloppypar}\hyphenrules{nohyphenation}\justifying\noindent\hspace{10mm} \textcolor{red}{अभिदुद्राव}\footnote{यथा \textcolor{red}{वालिपुत्रं महावीर्यमभिदुद्राव वीर्यवान्} (वा॰रा॰~६.७०.२) इति वाल्मीकि\-प्रयोगे। अभि~\textcolor{red}{द्रु गतौ} (धा॰पा॰~९४५)~\arrow अभि~द्रु~\arrow \textcolor{red}{शेषात्कर्तरि परस्मैपदम्} (पा॰सू॰~१.३.७८)~\arrow \textcolor{red}{परोक्षे लिट्} (पा॰सू॰~३.२.११५)~\arrow अभि~द्रु~लिट्~\arrow अभि~द्रु~तिप्~\arrow \textcolor{red}{परस्मैपदानां णलतुसुस्थलथुसणल्वमाः} (पा॰सू॰~३.४.८२)~\arrow अभि~द्रु~णल्~\arrow \textcolor{red}{लिटि धातोरनभ्यासस्य} (पा॰सू॰~६.१.८)~\arrow अभि~द्रु~द्रु~अ~\arrow \textcolor{red}{हलादिः शेषः} (पा॰सू॰~७.४.६०)~\arrow अभि~दु~द्रु~अ~\arrow \textcolor{red}{अचो ञ्णिति} (पा॰सू॰~७.२.११५)~\arrow अभि~दु~द्रौ~अ~\arrow \textcolor{red}{एचोऽयवायावः} (पा॰सू॰~६.१.७८)~\arrow अभि~दु~द्राव्~अ~\arrow अभिदुद्राव।} इति प्रयोक्तव्ये \textcolor{red}{अभिदुद्रुवे} इति प्रयुक्तम्। अत्रापि क्रिया\-विनिमय एव आत्मनेपदम्।\footnote{\textcolor{red}{कर्तरि कर्म\-व्यतिहारे} (पा॰सू॰~१.३.१४) इत्यनेन। अभि \textcolor{red}{द्रु गतौ} (धा॰पा॰~९४५)~\arrow अभि~द्रु~\arrow \textcolor{red}{कर्तरि कर्म\-व्यतिहारे} (पा॰सू॰~१.३.१४)~\arrow \textcolor{red}{परोक्षे लिट्} (पा॰सू॰~३.२.११५)~\arrow अभि~द्रु~लिट्~\arrow अभि~द्रु~त~\arrow \textcolor{red}{लिटस्तझयोरेशिरेच्} (पा॰सू॰~३.४.८१)~\arrow अभि~द्रु~एश्~\arrow अभि~द्रु~ए~\arrow \textcolor{red}{लिटि धातोरनभ्यासस्य} (पा॰सू॰~६.१.८)~\arrow अभि~द्रु~द्रु~ए~\arrow \textcolor{red}{हलादिः शेषः} (पा॰सू॰~७.४.६०)~\arrow अभि~दु~द्रु~ए~\arrow \textcolor{red}{अचि श्नुधातुभ्रुवां य्वोरियङुवङौ} (पा॰सू॰~६.४.७७)~\arrow \textcolor{red}{ङिच्च} (पा॰सू॰~१.१.५३)~\arrow अभि~दु~द्रुवँङ्~ए~\arrow अभि~दु~द्रुव्~ए~\arrow अभिदुद्रुवे। \textcolor{red}{सार्वधातुकमपित्} (पा॰सू॰~१.२.४) इत्यनेन ङिद्वत्त्वम् \textcolor{red}{ग्क्ङिति च} (पा॰सू॰~१.१.५) इत्यनेन च गुणनिषेधः।} रामादृते स्पर्शस्य कस्यचिदप्यनधिकारत्वात्।\end{sloppypar}
\section[उपवेशयत्]{उपवेशयत्}
\centering\textcolor{blue}{अभिगम्य सुसम्पूज्य विष्टरेषूपवेशयत्।\nopagebreak\\
आतिथ्यमकरोत्तेषां कन्दमूलफलादिभिः॥}\nopagebreak\\
\raggedleft{–~अ॰रा॰~३.२.३}\\
\fontsize{14}{21}\selectfont\begin{sloppypar}\hyphenrules{nohyphenation}\justifying\noindent\hspace{10mm} \textcolor{red}{विश्‌}\-धातुः (\textcolor{red}{विशँ प्रवेशने} धा॰पा॰~१४२४) अत्र णिजन्तः। तस्य \textcolor{red}{उप}\-उपसर्गेण संयोजनम्। \textcolor{red}{पूर्वं धातुः साधनेन युज्यते पश्चादुपसर्गेण} (भा॰पा॰सू॰~२.२.१९, ६.१.१३५, ८.१.७०) इति नियमेनाडागम \textcolor{red}{उप}\-उपसर्ग\-संयोजने \textcolor{red}{उपावेशयत्}।\footnote{विश्~\arrow \textcolor{red}{हेतुमति च} (पा॰सू॰~३.१.२६)~\arrow विश्~णिच्~\arrow विश्~इ~\arrow \textcolor{red}{पुगन्त\-लघूपधस्य च} (पा॰सू॰~७.३.८६)~\arrow वेश्~इ~\arrow वेशि~\arrow \textcolor{red}{सनाद्यन्ता धातवः} (पा॰सू॰~३.१.३२)~\arrow धातु\-सञ्ज्ञा। उप~वेशि~\arrow \textcolor{red}{शेषात्कर्तरि परस्मैपदम्} (पा॰सू॰~१.३.७८)~\arrow \textcolor{red}{अनद्यतने लङ्} (पा॰सू॰~३.२.१११)~\arrow उप~वेशि~लङ्~\arrow \textcolor{red}{लुङ्लङ्लृङ्क्ष्वडुदात्तः} (पा॰सू॰~६.४.७१)~\arrow \textcolor{red}{आद्यन्तौ टकितौ} (पा॰सू॰~१.१.४६)~\arrow उप~अट्~वेशि~लङ्~\arrow उप~अ~वेशि~लङ्~\arrow उप~अ~वेशि~तिप्~\arrow उप~अ~वेशि~ति~\arrow \textcolor{red}{कर्तरि शप्‌} (पा॰सू॰~३.१.६८)~\arrow उप~अ~वेशि~शप्~ति~\arrow उप~अ~वेशि~अ~ति~\arrow \textcolor{red}{सार्वधातुकार्ध\-धातुकयोः} (पा॰सू॰~७.३.८४)~\arrow उप~अ~वेशे~अ~ति~\arrow \textcolor{red}{एचोऽयवायावः} (पा॰सू॰~६.१.७८)~\arrow उप~अ~वेशय्~अ~ति~\arrow \textcolor{red}{इतश्च} (पा॰सू॰~३.४.१००)~\arrow उप~अ~वेशय्~अ~त्~\arrow \textcolor{red}{अकः सवर्णे दीर्घः} (पा॰सू॰~६.१.१०१)~\arrow उपा~वेशय्~अ~त्~\arrow उपावेशयत्।} अत्र \textcolor{red}{विनाऽपि प्रत्ययं पूर्वोत्तर\-पद\-लोपो वक्तव्यः} (वा॰~५.३.८३) इत्यनेनाडागमस्य लोपः। आगम\-कार्यस्यानित्यत्वाद्वा।\footnote{\textcolor{red}{आगम\-शास्त्रमनित्यम्} (प॰शे॰~९३.२)।}\end{sloppypar}
\section[व्रजामि]{व्रजामि}
\centering\textcolor{blue}{अद्य मत्तपसा सिद्धं यत्पुण्यं बहु विद्यते।\nopagebreak\\
तत्सर्वं तव दास्यामि ततो मुक्तिं व्रजाम्यहं॥}\nopagebreak\\
\raggedleft{–~अ॰रा॰~३.२.५}\\
\fontsize{14}{21}\selectfont\begin{sloppypar}\hyphenrules{nohyphenation}\justifying\noindent\hspace{10mm} वर्तमान\-सामीप्याल्लट्।\footnote{\textcolor{red}{वर्तमान\-सामीप्ये वर्तमानवद्वा} (पा॰सू॰~३.३.१३१) इत्यनेन। \textcolor{red}{व्रजँ गतौ} (धा॰पा॰~२५३)~\arrow व्रज्~\arrow \textcolor{red}{शेषात्कर्तरि परस्मैपदम्} (पा॰सू॰~१.३.७८)~\arrow \textcolor{red}{वर्तमान\-सामीप्ये वर्तमानवद्वा} (पा॰सू॰~३.३.१३१)~\arrow \textcolor{red}{वर्तमाने लट्} (पा॰सू॰~३.२.१२३)~\arrow व्रज्~लट्~\arrow व्रज्~मिप्~\arrow व्रज्~मि~\arrow \textcolor{red}{कर्तरि शप्‌} (पा॰सू॰~३.१.६८)~\arrow व्रज्~शप्~मि~\arrow व्रज्~अ~मि~\arrow \textcolor{red}{अतो दीर्घो यञि} (पा॰सू॰~७.३.१०१)~\arrow व्रज्~आ~मि~\arrow व्रजामि।} \textcolor{red}{शीघ्रं मुक्तिं व्रजिष्यामि}\footnote{\textcolor{red}{व्रजँ गतौ} (धा॰पा॰~२५३)~\arrow व्रज्~\arrow \textcolor{red}{शेषात्कर्तरि परस्मैपदम्} (पा॰सू॰~१.३.७८)~\arrow \textcolor{red}{लृट् शेषे च} (पा॰सू॰~३.३.१३)~\arrow व्रज्~लृट्~\arrow व्रज्~मिप्~\arrow व्रज्~मि~\arrow \textcolor{red}{स्यतासी लृलुटोः} (पा॰सू॰~३.१.३३)~\arrow \textcolor{red}{वर्तमान\-सामीप्ये वर्तमानवद्वा} (पा॰सू॰~३.३.१३१)~\arrow व्रज्~स्य~मि~\arrow \textcolor{red}{आर्धधातुकस्येड्वलादेः} (पा॰सू॰~७.२.३५)~\arrow व्रज्~इट्~स्य~मि~\arrow व्रज्~इ~स्य~मि~\arrow \textcolor{red}{अतो दीर्घो यञि} (पा॰सू॰~७.३.१०१)~\arrow व्रज्~इ~स्या~मि~\arrow \textcolor{red}{आदेश\-प्रत्यययोः} (पा॰सू॰~८.३.५९)~\arrow व्रज्~इ~ष्या~मि~\arrow व्रजिष्यामि।} इति शरभङ्गस्य तात्पर्यम्।\end{sloppypar}
\section[बभूवुः]{बभूवुः}
\centering\textcolor{blue}{अद्य मे क्रतवः सर्वे बभूवुः सफलाः प्रभो।\nopagebreak\\
दीर्घकालं मया तप्तमनन्यमतिना तपः।\nopagebreak\\
तस्येह तपसो राम फलं तव यदर्चनम्॥}\nopagebreak\\
\raggedleft{–~अ॰रा॰~३.३.४३}\\
\fontsize{14}{21}\selectfont\begin{sloppypar}\hyphenrules{nohyphenation}\justifying\noindent\hspace{10mm} शरभङ्गः प्रेमविह्वलः श्रीरामं दृष्ट्वा कथयति \textcolor{red}{अद्य मे क्रतवः सफला अभूवन्}। \textcolor{red}{अभूवन्}\footnote{\textcolor{red}{भू सत्तायाम्} (धा॰पा॰~१)~\arrow भू~\arrow \textcolor{red}{शेषात्कर्तरि परस्मैपदम्} (पा॰सू॰~१.३.७८)~\arrow \textcolor{red}{लुङ्} (पा॰सू॰~३.२.११०)~\arrow भू~लङ्~\arrow भू~झि~\arrow \textcolor{red}{लुङ्लङ्लृङ्क्ष्वडुदात्तः} (पा॰सू॰~६.४.७१)~\arrow \textcolor{red}{आद्यन्तौ टकितौ} (पा॰सू॰~१.१.४६)~\arrow अट्~भू~झि~\arrow अ~भू~झि~\arrow \textcolor{red}{च्लि लुङि} (पा॰सू॰~३.१.४३)~\arrow अ~भू~च्लि~झि~\arrow \textcolor{red}{च्लेः सिच्} (पा॰सू॰~३.१.४४)~\arrow अ~भू~सिच्~झि~\arrow \textcolor{red}{गातिस्था\-घुपाभूभ्यः सिचः परस्मैपदेषु} (पा॰सू॰~२.४.७७)~\arrow अ~भू~झि~\arrow \textcolor{red}{झोऽन्तः} (पा॰सू॰~७.१.३)~\arrow अ~भू~अन्ति~\arrow \textcolor{red}{भुवो वुग्लुङ्लिटोः} (पा॰सू॰~६.४.८८)~\arrow अ~भू~वुँक्~अन्ति~\arrow अ~भू~व्~अन्ति~\arrow \textcolor{red}{इतश्च} (पा॰सू॰~३.४.१००)~\arrow अ~भू~व्~अन्त्~\arrow \textcolor{red}{संयोगान्तस्य लोपः} (पा॰सू॰~८.२.२३)~\arrow अ~भू~व्~अन्~\arrow अभूवन्।} इति प्रयोक्तव्ये \textcolor{red}{बभूवुः} इति परोक्ष\-प्रयोगोऽकार्यः। तथा हि \textcolor{red}{भू}\-धातोः (\textcolor{red}{भू सत्तायाम्} धा॰पा॰~१) \textcolor{red}{भूवादयो धातवः} (पा॰सू॰~१.३.१) इत्यनेन धातु\-सञ्ज्ञायां \textcolor{red}{परोक्षे लिट्} (पा॰सू॰~३.२.११५) इत्यनेन लिड्लकारे \textcolor{red}{तिप्तस्झि\-सिप्थस्थ\-मिब्वस्मस्ताताञ्झ\-थासाथान्ध्वमिड्वहिमहिङ्} (पा॰सू॰~३.४.७८) इत्यनेन प्रथम\-पुरुष\-बहु\-वचने \textcolor{red}{झि}\-प्रत्यये \textcolor{red}{परस्मैपदानां णलतुसुस्थलथुस\-णल्वमाः} (पा॰सू॰~३.४.८२) इत्यनेनोसादेशे \textcolor{red}{भुवो वुग्लुङ्लिटोः} (पा॰सू॰~६.४.८८) इत्यनेन वुगागमे \textcolor{red}{लिटि धातोरनभ्यासस्य} (पा॰सू॰~६.१.८) इत्यनेन द्वित्वे \textcolor{red}{पूर्वोऽभ्यासः} (पा॰सू॰~६.१.४) इत्यनेनाभ्यास\-सञ्ज्ञायां \textcolor{red}{हलादिः शेषः} (पा॰सू॰~७.४.६०) इत्यनेनाऽदिहल्शेषे \textcolor{red}{ह्रस्वः} (पा॰सू॰~७.४.५९) इत्यनेन ह्रस्वे \textcolor{red}{भवतेरः} (पा॰सू॰~७.४.७३) इत्यनेनोकारस्याकारे \textcolor{red}{अभ्यासे चर्च} (पा॰सू॰~८.४.५४) इत्यनेन जश्त्वे रुत्वे विसर्गे च \textcolor{red}{बभूवुः}।\footnote{\textcolor{red}{भू सत्तायाम्} (धा॰पा॰~१)~\arrow भू~\arrow \textcolor{red}{शेषात्कर्तरि परस्मैपदम्} (पा॰सू॰~१.३.७८)~\arrow \textcolor{red}{परोक्षे लिट्} (पा॰सू॰~३.२.११५)~\arrow (पा॰सू॰~३.२.११५)~\arrow भू~लिट्~\arrow भू~तिप्~\arrow \textcolor{red}{परस्मैपदानां णलतुसुस्थलथुस\-णल्वमाः} (पा॰सू॰~३.४.८२)~\arrow भू~उस्~\arrow \textcolor{red}{भुवो वुग्लुङ्लिटोः} (पा॰सू॰~६.४.८८)~\arrow \textcolor{red}{आद्यन्तौ टकितौ} (पा॰सू॰~१.१.४६)~\arrow भूवुँक्~अ~\arrow भूव्~उस्~\arrow \textcolor{red}{लिटि धातोरनभ्यासस्य} (पा॰सू॰~६.१.८)~\arrow भूव्~भूव्~उस्~\arrow \textcolor{red}{हलादिः शेषः} (पा॰सू॰~७.४.६०)~\arrow भू~भूव्~उस्~\arrow \textcolor{red}{ह्रस्वः} (पा॰सू॰~७.४.५९)~\arrow भु~भूव्~उस्~\arrow \textcolor{red}{भवतेरः} (पा॰सू॰~७.४.७३)~\arrow भ~भूव्~उस्~\arrow \textcolor{red}{अभ्यासे चर्च} (पा॰सू॰~८.४.५४)~\arrow ब~भूव्~उस्~\arrow बभूवुस्~\arrow \textcolor{red}{ससजुषो रुः} (पा॰सू॰~८.२.६६)~\arrow रुत्वम्~\arrow बभूवुरुँ~\arrow अनुबन्ध\-लोपः~\arrow बभूवुर्~\arrow \textcolor{red}{खरवसानयोर्विसर्जनीयः} (पा॰सू॰~८.३.१५)~\arrow बभूवुः।} प्रेम\-विह्वलतया भगवतश्चाक्षुष\-प्रत्यक्षे वा षट्सु प्रत्यक्षेषु सकल\-विस्मृततया पारोक्ष्याल्लिट्प्रयोगः।\end{sloppypar}
\section[वत्स्ये]{वत्स्ये}
\centering\textcolor{blue}{पञ्चवट्यामहं वत्स्ये तवैव प्रियकाम्यया।\nopagebreak\\
मृगयायां कदाचित्तु प्रयाते लक्ष्मणेऽपि च॥}\nopagebreak\\
\raggedleft{–~अ॰रा॰~३.४.५}\\
\fontsize{14}{21}\selectfont\begin{sloppypar}\hyphenrules{nohyphenation}\justifying\noindent\hspace{10mm} अत्र क्रिया\-विनिमय आत्मनेपदम्।\footnote{\textcolor{red}{कर्तरि कर्म\-व्यतिहारे} (पा॰सू॰~१.३.१४) इत्यनेन। \textcolor{red}{वसँ निवासे} (धा॰पा॰~१००४)~\arrow वस्~\arrow \textcolor{red}{कर्तरि कर्म\-व्यतिहारे} (पा॰सू॰~१.३.१४)~\arrow \textcolor{red}{लृट् शेषे च} (पा॰सू॰~३.३.१३)~\arrow वस्~लृट्~\arrow वस्~इट्~\arrow वस्~इ~\arrow \textcolor{red}{स्यतासी लृलुटोः} (पा॰सू॰~३.१.३३)~\arrow वस्~स्य~इ~\arrow \textcolor{red}{सः स्यार्धधातुके} (पा॰सू॰~७.४.४९)~\arrow वत्~स्य~इ~\arrow \textcolor{red}{टित आत्मनेपदानां टेरे} (पा॰सू॰~३.४.७९)~\arrow वत्~स्य~ए~\arrow \textcolor{red}{अतो गुणे} (पा॰सू॰~६.१.९७)~\arrow वत्~स्ये~\arrow वत्स्ये। सामान्यतस्तु \textcolor{red}{वत्स्यामि} इति रूपम्। यथा \textcolor{red}{चतुर्दश हि वर्षाणि वत्स्यामि विजने वने} (वा॰रा॰~२.२०.२९) इति वाल्मीकि\-प्रयोगे। \textcolor{red}{वसँ निवासे} (धा॰पा॰~१००४)~\arrow वस्~\arrow \textcolor{red}{शेषात्कर्तरि परस्मैपदम्} (पा॰सू॰~१.३.७८)~\arrow \textcolor{red}{लृट् शेषे च} (पा॰सू॰~३.३.१३)~\arrow वस्~लृट्~\arrow वस्~मिप्~\arrow वस्~मि~\arrow \textcolor{red}{स्यतासी लृलुटोः} (पा॰सू॰~३.१.३३)~\arrow वस्~स्य~मि~\arrow \textcolor{red}{सः स्यार्धधातुके} (पा॰सू॰~७.४.४९)~\arrow वत्~स्य~मि~\arrow \textcolor{red}{अतो दीर्घो यञि} (पा॰सू॰~७.३.१०१)~\arrow वत्~स्या~मि~\arrow वत्स्यामि।}\end{sloppypar}
\section[जागर्ति]{जागर्ति}
\centering\textcolor{blue}{आनीय प्रददौ रामसेवातत्परमानसः।\nopagebreak\\
धनुर्बाणधरो नित्यं रात्रौ जागर्ति सर्वतः॥}\nopagebreak\\
\raggedleft{–~अ॰रा॰~३.४.१३}\\
\fontsize{14}{21}\selectfont\begin{sloppypar}\hyphenrules{nohyphenation}\justifying\noindent\hspace{10mm} \textcolor{red}{स्म}\-योगे लट्।\footnote{\textcolor{red}{स्म} इत्यध्याहार्यमिति भावः। \textcolor{red}{जागृ निद्राक्षये} (धा॰पा॰~१०७२)~\arrow जागृ~\arrow \textcolor{red}{शेषात्कर्तरि परस्मैपदम्} (पा॰सू॰~१.३.७८)~\arrow \textcolor{red}{लट् स्मे} (पा॰सू॰~३.२.११८)~\arrow जागृ~लट्~\arrow जागृ~तिप्~\arrow जागृ~ति~\arrow \textcolor{red}{कर्तरि शप्‌} (पा॰सू॰~३.१.६८)~\arrow जागृ~शप्~ति~\arrow जागृ~अ~ति~\arrow \textcolor{red}{सार्वधातुकार्ध\-धातुकयोः} (पा॰सू॰~७.३.८४)~\arrow \textcolor{red}{उरण् रपरः} (पा॰सू॰~१.१.५१)~\arrow जागर्~अ~ति~\arrow \textcolor{red}{अदिप्रभृतिभ्यः शपः} (पा॰सू॰~२.४.७२)~\arrow जागर्~ति~\arrow जागर्ति।}\end{sloppypar}
\section[सेवते]{सेवते}
\centering\textcolor{blue}{आनीय सलिलं नित्यं लक्ष्मणः प्रीतमानसः।\nopagebreak\\
सेवतेऽहरहः प्रीत्या एवमासन् सुखं त्रयः॥}\nopagebreak\\
\raggedleft{–~अ॰रा॰~३.४.१५}\\
\fontsize{14}{21}\selectfont\begin{sloppypar}\hyphenrules{nohyphenation}\justifying\noindent\hspace{10mm} \textcolor{red}{स्म}\-योगे लट्।\footnote{\textcolor{red}{स्म} इत्यध्याहार्यमिति भावः। \textcolor{red}{षेवृँ सेवने} (धा॰पा॰~५०१)~\arrow षेव्~\arrow \textcolor{red}{धात्वादेः षः सः} (पा॰सू॰~६.१.६४)~\arrow सेव्~\arrow \textcolor{red}{अनुदात्तङित आत्मने\-पदम्} (पा॰सू॰~१.३.१२)~\arrow \textcolor{red}{लट् स्मे} (पा॰सू॰~३.२.११८)~\arrow सेव्~लट्~\arrow सेव्~त~\arrow \textcolor{red}{कर्तरि शप्‌} (पा॰सू॰~३.१.६८)~\arrow सेव्~शप्~त~\arrow सेव्~अ~त~\arrow \textcolor{red}{टित आत्मनेपदानां टेरे} (पा॰सू॰~३.४.७९)~\arrow सेव्~अ~ते~\arrow सेवते।}\end{sloppypar}
\section[सङ्गच्छावः]{सङ्गच्छावः}
\centering\textcolor{blue}{भ्रातुराज्ञां पुरस्कृत्य सङ्गच्छावोऽद्य मा चिरम्।\nopagebreak\\
इत्याह राक्षसी घोरा लक्ष्मणं काममोहिता॥}\nopagebreak\\
\raggedleft{–~अ॰रा॰~३.५.१५}\\
\fontsize{14}{21}\selectfont\begin{sloppypar}\hyphenrules{nohyphenation}\justifying\noindent\hspace{10mm} इह \textcolor{red}{समो गम्यृच्छिभ्याम्} (पा॰सू॰~१.३.२९) इत्यनेनाऽत्मनेपदत्वात् \textcolor{red}{सङ्गच्छावहे}।\footnote{सम् \textcolor{red}{गमॢँ गतौ} (धा॰पा॰~९८२)~\arrow सम्~गम्~\arrow \textcolor{red}{समो गम्यृच्छिभ्याम्} (पा॰सू॰~१.३.२९)~\arrow \textcolor{red}{वर्तमाने लट्} (पा॰सू॰~३.२.१२३)~\arrow सम्~गम्~लट्~\arrow सम्~गम्~वहिङ्~\arrow सम्~गम्~वहि~\arrow \textcolor{red}{कर्तरि शप्‌} (पा॰सू॰~३.१.६८)~\arrow सम्~गम्~शप्~वहि~\arrow सम्~गम्~अ~वहि~\arrow \textcolor{red}{इषुगमियमां छः} (पा॰सू॰~७.३.७७)~\arrow सम्~गछ्~अ~वहि~\arrow \textcolor{red}{छे च} (पा॰सू॰~६.१.७३)~\arrow \textcolor{red}{आद्यन्तौ टकितौ} (पा॰सू॰~१.१.४६)~\arrow सम्~गतुँक्~छ्~अ~वहि~\arrow सम्~गत्~छ्~अ~वहि~\arrow \textcolor{red}{स्तोः श्चुना श्चुः} (पा॰सू॰~८.४.४०)~\arrow सम्~गच्~छ्~अ~वहि~\arrow \textcolor{red}{अतो दीर्घो यञि} (पा॰सू॰~७.३.१०१)~\arrow सम्~गच्~छ्~आ~वहि~\arrow \textcolor{red}{टित आत्मनेपदानां टेरे} (पा॰सू॰~३.४.७९)~\arrow सम्~गच्~छ्~आ~वहे~\arrow \textcolor{red}{मोऽनुस्वारः} (पा॰सू॰~८.३.२३)~\arrow सं~गच्~छ्~आ~वहे~\arrow \textcolor{red}{वा पदान्तस्य} (पा॰सू॰~८.४.५९)~\arrow सङ्~गच्~छ्~आ~वहे~\arrow सङ्गच्छावहे।} किन्तु अत्र \textcolor{red}{शम्} इति पृथक्पदम्। अर्थात् \textcolor{red}{शं शान्तिं गच्छावः} इति तात्पर्यम्।\footnote{\textcolor{red}{गमॢँ गतौ} (धा॰पा॰~९८२)~\arrow गम्~\arrow \textcolor{red}{शेषात्कर्तरि परस्मैपदम्} (पा॰सू॰~१.३.७८)~\arrow \textcolor{red}{वर्तमाने लट्} (पा॰सू॰~३.२.१२३)~\arrow गम्~लट्~\arrow गम्~वस्~\arrow \textcolor{red}{कर्तरि शप्‌} (पा॰सू॰~३.१.६८)~\arrow गम्~शप्~वस्~\arrow गम्~अ~वस्~\arrow \textcolor{red}{इषुगमियमां छः} (पा॰सू॰~७.३.७७)~\arrow गछ्~अ~वस्~\arrow \textcolor{red}{छे च} (पा॰सू॰~६.१.७३)~\arrow \textcolor{red}{आद्यन्तौ टकितौ} (पा॰सू॰~१.१.४६)~\arrow गतुँक्~छ्~अ~वस्~\arrow गत्~छ्~अ~वस्~\arrow \textcolor{red}{स्तोः श्चुना श्चुः} (पा॰सू॰~८.४.४०)~\arrow गच्~छ्~अ~वस्~\arrow \textcolor{red}{अतो दीर्घो यञि} (पा॰सू॰~७.३.१०१)~\arrow गच्~छ्~आ~वस्~\arrow \textcolor{red}{ससजुषो रुः} (पा॰सू॰~८.२.६६)~\arrow गच्~छ्~आ~वरुँ~\arrow \textcolor{red}{खरवसानयोर्विसर्जनीयः} (पा॰सू॰~८.३.१५)~\arrow गच्~छ्~आ~वः~\arrow गच्छावः।} तालव्य\-शकारो लेखनप्रमादान्मुद्रण\-प्रमादाच्च दन्त्यो लिखितो हस्तलेखेषु पुस्तकेषु च।\end{sloppypar}
\section[अनुधावति]{अनुधावति}
\centering\textcolor{blue}{इत्युक्त्वा विकटाकरा जानकीमनुधावति।\nopagebreak\\
ततो रामाज्ञया खड्गमादाय परिगृह्य ताम्॥}\nopagebreak\\
\raggedleft{–~अ॰रा॰~३.५.१९}\\
\fontsize{14}{21}\selectfont\begin{sloppypar}\hyphenrules{nohyphenation}\justifying\noindent\hspace{10mm} इहापि \textcolor{red}{स्म}\-योगे लट्।\footnote{\textcolor{red}{स्म} इत्यध्याहार्यमिति भावः। अनु~\textcolor{red}{धावुँ गतिशुद्धयोः} (धा॰पा॰~६०१)~\arrow अनु~धाव्~\arrow \textcolor{red}{शेषात्कर्तरि परस्मैपदम्} (पा॰सू॰~१.३.७८)~\arrow \textcolor{red}{लट् स्मे} (पा॰सू॰~३.२.११८)~\arrow अनु~धाव्~लट्~\arrow अनु~धाव्~तिप्~\arrow अनु~धाव्~ति~\arrow \textcolor{red}{कर्तरि शप्‌} (पा॰सू॰~३.१.६८)~\arrow अनु~धाव्~शप्~ति~\arrow अनु~धाव्~अ~ति~\arrow अनु~धावति।} एवमेवैतादृशेषु सर्वेषु प्रयोगेषूह्यम्।\end{sloppypar}
\section[पास्ये]{पास्ये}
\centering\textcolor{blue}{तयोस्तु रुधिरं पास्ये भक्षयैतौ सुदुर्मदौ।\nopagebreak\\
नो चेत्प्राणान्परित्यज्य यास्यामि यमसादनम्॥}\nopagebreak\\
\raggedleft{–~अ॰रा॰~३.५.२५}\\
\fontsize{14}{21}\selectfont\begin{sloppypar}\hyphenrules{nohyphenation}\justifying\noindent\hspace{10mm} अत्र कर्म\-विनिमयादात्मने\-पदम्।\footnote{\textcolor{red}{पा पाने} (धा॰पा॰~९२५)~\arrow \textcolor{red}{कर्तरि कर्म\-व्यतिहारे} (पा॰सू॰~१.३.१४)~\arrow \textcolor{red}{लृट् शेषे च} (पा॰सू॰~३.३.१३)~\arrow पा~लृँट्~\arrow पा~इट्~\arrow पा~इ~\arrow \textcolor{red}{स्यतासी लृलुटोः} (पा॰सू॰~३.१.३३)~\arrow पा~स्य~इ~\arrow \textcolor{red}{टित आत्मनेपदानां टेरे} (पा॰सू॰~३.४.७९)~\arrow पा~स्य~ए~\arrow \textcolor{red}{अतो गुणे} (पा॰सू॰~६.१.९७)~\arrow पा~स्ये~\arrow पास्ये।} एवमन्यत्रापि विभाव्यम्।\end{sloppypar}
\section[आनयिष्यामि]{आनयिष्यामि}
\centering\textcolor{blue}{अतस्त्वया सहायेन गत्वा तत्प्राणवल्लभाम्।\nopagebreak\\
आनयिष्यामि विपिने रहिते राघवेण ताम्॥}\nopagebreak\\
\raggedleft{–~अ॰रा॰~३.६.१२}\\
\centering\textcolor{blue}{यदि मारीच एवायं तदा हन्मि न संशयः।\nopagebreak\\
मृगश्चेदानयिष्यामि सीताविश्रमहेतवे॥\\
गमिष्यामि मृगं बद्ध्वा ह्यानयिष्यामि सत्वरः।\nopagebreak\\
त्वं प्रयत्नेन सन्तिष्ठ सीतासंरक्षणोद्यतः॥}\nopagebreak\\
\raggedleft{–~अ॰रा॰~३.७.१०–११}\\
\fontsize{14}{21}\selectfont\begin{sloppypar}\hyphenrules{nohyphenation}\justifying\noindent\hspace{10mm} \textcolor{red}{नी}\-धातोः (\textcolor{red}{णीञ् प्रापणे} धा॰पा॰~९०१) अनिट्त्वाल्लृड्लकार उत्तम\-पुरुषैक\-वचने \textcolor{red}{सार्वधातुकार्धधातुकयोः} (पा॰सू॰~७.३.८४) इत्यनेन गुणे \textcolor{red}{अतो दीर्घो यञि} (पा॰सू॰~७.३.१०१) इत्यनेन दीर्घे \textcolor{red}{आदेश\-प्रत्ययोः} (पा॰सू॰~८.३.५९) इत्यनेन षत्वे \textcolor{red}{आनेष्यामि} इति पाणिनीयम्।\footnote{आ~णीञ्~\arrow णी~\arrow \textcolor{red}{णो नः} (पा॰सू॰~६.१.६५)~\arrow आ~नी~\arrow \textcolor{red}{शेषात्कर्तरि परस्मैपदम्} (पा॰सू॰~१.३.७८)~\arrow \textcolor{red}{लृट् शेषे च} (पा॰सू॰~३.३.१३)~\arrow आ~नी~लृँट्~\arrow आ~नी~मिप्~\arrow आ~नी~मि~\arrow \textcolor{red}{स्यतासी लृलुटोः} (पा॰सू॰~३.१.३३)~\arrow आ~नी~स्य~मि~\arrow \textcolor{red}{सार्वधातुकार्ध\-धातुकयोः} (पा॰सू॰~७.३.८४)~\arrow आ~ने~स्य~मि~\arrow \textcolor{red}{अतो दीर्घो यञि} (पा॰सू॰~७.३.१०१)~\arrow आ~ने~स्या~मि~\arrow \textcolor{red}{आदेश\-प्रत्यययोः} (पा॰सू॰~८.३.५९)~\arrow आ~ने~ष्या~मि~\arrow आनेष्यामि।} परम् \textcolor{red}{आनीयत इत्यानयः}।\footnote{\textcolor{red}{एरच्} (पा॰सू॰~३.३.५६) इत्यनेन भावेऽचि। आङ्~\textcolor{red}{णीञ् प्रापणे} धा॰पा॰~९०१)~\arrow आ~णी~\arrow \textcolor{red}{णो नः} (पा॰सू॰~६.१.६५)~\arrow आ~नी~\arrow \textcolor{red}{एरच्} (पा॰सू॰~३.३.५६)~\arrow आ~नी~अच्~\arrow आ~नी~अ~\arrow \textcolor{red}{सार्वधातुकार्ध\-धातुकयोः} (पा॰सू॰~७.३.८४)~\arrow आ~ने~अच्~\arrow \textcolor{red}{एचोऽयवायावः} (पा॰सू॰~६.१.७८)~\arrow आ~नय्~अ~\arrow आनय~\arrow विभक्तिकार्यम्~\arrow आनयः।} \textcolor{red}{आनयमाचरिष्याम्यानयिष्यामि} इत्यानय\-धातोर्लृटीट्सम्भवः।\footnote{प्रयोगस्यास्य सिद्धिः \textcolor{red}{कृष्णिष्यति} (बा॰म॰~२६६५) इतिवत्। आनय~\arrow \textcolor{red}{सर्वप्राति\-पदिकेभ्य आचारे क्विब्वा वक्तव्यः} (वा॰~३.१.११)~\arrow आनय~क्विँप्~\arrow आनय~व्~\arrow \textcolor{red}{वेरपृक्तस्य} (पा॰सू॰~६.१.६७)~\arrow आनय~\arrow \textcolor{red}{सनाद्यन्ता धातवः} (पा॰सू॰~३.१.३२)~\arrow \textcolor{red}{शेषात्कर्तरि परस्मैपदम्} (पा॰सू॰~१.३.७८)~\arrow \textcolor{red}{लृट् शेषे च} (पा॰सू॰~३.३.१३)~\arrow आनय~लृट्~\arrow आनय~मिप्~\arrow आनय~मि~\arrow \textcolor{red}{स्यतासी लृलुटोः} (पा॰सू॰~३.१.३३)~\arrow आनय~स्य~मि~\arrow \textcolor{red}{आर्धधातुकस्येड्वलादेः} (पा॰सू॰~७.२.३५)~\arrow आनय~इट्~स्य~मि~\arrow आनय~इ~स्य~मि~\arrow \textcolor{red}{अतो लोपः} (पा॰सू॰~६.४.४८)~\arrow आनय्~इ~स्य~मि~\arrow \textcolor{red}{अतो दीर्घो यञि} (पा॰सू॰~७.३.१०१)~\arrow आनय्~इ~स्या~मि~\arrow \textcolor{red}{आदेश\-प्रत्ययोः} (पा॰सू॰~८.३.५९)~\arrow आनय्~इ~ष्या~मि~\arrow आनयिष्यामि। यद्वा सप्तमाध्याये \textcolor{red}{आनय इवाऽचरिष्यामि आनयिष्यामि}। \textcolor{red}{आनयतीत्यानयः}। \textcolor{red}{नन्दि\-ग्रहि\-पचादिभ्यो ल्युणिन्यचः} (पा॰सू॰~३.१.१३४) इत्यनेन कर्तर्यच्। \textcolor{red}{नयतीति नयः} इतिवत्। यथा विष्णु\-सहस्र\-नाम\-स्तोत्रे \textcolor{red}{रामो विरामो विरतो मार्गो नेयो नयोऽनयः} (वि॰स॰ना॰~५६)। अत्र सत्यभाष्ये सत्यदेव\-वासिष्ठाः~– \textcolor{red}{‘णीञ् प्रापणे’ धातोरच्प्रत्ययो ‘नयतीति नयः’ सर्वस्य नेतेत्यर्थः} (वि॰स॰ना॰ स॰भा॰~५६)। \textcolor{red}{नय}\-शब्दो \textcolor{red}{नेतरि} इति शब्दकल्पद्रुम\-वाचस्पत्यौ च। एवं तर्हि \textcolor{red}{आनय इवाऽचरिष्यामि आनयिष्यामि}। सिद्धिः पूर्ववत्। \textcolor{red}{रामो न गच्छति न तिष्ठति नानुशोचत्याकाङ्क्षते त्यजति नो न करोति किञ्चित्} (अ॰रा॰~१.१.४३) इत्यस्मिन्नेव ग्रन्थ उक्तत्वात्।}\end{sloppypar}
\section[सन्तिष्ठ]{सन्तिष्ठ}
\centering\textcolor{blue}{गमिष्यामि मृगं बद्ध्वा ह्यानयिष्यामि सत्वरः।\nopagebreak\\
त्वं प्रयत्नेन सन्तिष्ठ सीतासंरक्षणोद्यतः॥}\nopagebreak\\
\raggedleft{–~अ॰रा॰~३.७.११}\\
\fontsize{14}{21}\selectfont\begin{sloppypar}\hyphenrules{nohyphenation}\justifying\noindent\hspace{10mm} \textcolor{red}{समवप्रविभ्यः स्थः} (पा॰सू॰~१.३.२२) इत्यनेनाऽत्मनेपदं प्राप्तम्। \textcolor{red}{सन्तिष्ठस्व}\footnote{सम्~\textcolor{red}{ष्ठा गतिनिवृत्तौ} (धा॰पा॰~९२८)~\arrow \textcolor{red}{धात्वादेः षः सः} (पा॰सू॰~६.१.६४)~\arrow निमित्तापाये नैमित्तिकस्याप्यपायः~\arrow सम्~स्था~\arrow \textcolor{red}{समवप्रविभ्यः स्थः} (पा॰सू॰~१.३.२२)~\arrow \textcolor{red}{लोट् च} (पा॰सू॰~३.३.१६२)~\arrow सम्~स्था~लोट्~\arrow सम्~स्था~थास्~\arrow \textcolor{red}{कर्तरि शप्} (पा॰सू॰~३.१.६८)~\arrow सम्~स्था~शप्~थास्~\arrow सम्~स्था~अ~थास्~\arrow \textcolor{red}{पाघ्रा\-ध्मास्थाम्ना\-दाण्दृश्यर्त्ति\-सर्त्तिशदसदां पिब\-जिघ्र\-धम\-तिष्ठ\-मन\-यच्छ\-पश्यर्च्छ\-धौ\-शीय\-सीदाः} (पा॰सू॰~७.३.७८)~\arrow सम्~तिष्ठ्~अ~थास्~\arrow \textcolor{red}{थासस्से} (पा॰सू॰~३.४.८०)~\arrow सम्~तिष्ठ्~अ~से~\arrow \textcolor{red}{सवाभ्यां वामौ} (पा॰सू॰~३.४.९१)~\arrow सम्~तिष्ठ्~अ~स्~व~\arrow \textcolor{red}{मोऽनुस्वारः} (पा॰सू॰~८.३.२३)~\arrow सं~तिष्ठ्~अ~स्~व~\arrow \textcolor{red}{वा पदान्तस्य} (पा॰सू॰~८.४.५९)~\arrow सन्~तिष्ठ्~अ~स्~व~\arrow सन्तिष्ठस्व।} इति प्रयोगः पाणिनीयः। किन्तु \textcolor{red}{तिष्ठ} इति यावत्संसाध्य\footnote{\textcolor{red}{ष्ठा गतिनिवृत्तौ} (धा॰पा॰~९२८)~\arrow \textcolor{red}{धात्वादेः षः सः} (पा॰सू॰~६.१.६४)~\arrow निमित्तापाये नैमित्तिकस्याप्यपायः~\arrow स्था~\arrow \textcolor{red}{शेषात्कर्तरि परस्मैपदम्} (पा॰सू॰~१.३.७८)~\arrow \textcolor{red}{लोट् च} (पा॰सू॰~३.३.१६२)~\arrow स्था~लोट्~\arrow स्था~सिप्~\arrow स्था~सि~\arrow \textcolor{red}{कर्तरि शप्} (पा॰सू॰~३.१.६८)~\arrow स्था~शप्~सि~\arrow स्था~अ~सि~\arrow \textcolor{red}{पाघ्रा\-ध्मास्थाम्ना\-दाण्दृश्यर्त्ति\-सर्त्तिशदसदां पिब\-जिघ्र\-धम\-तिष्ठ\-मन\-यच्छ\-पश्यर्च्छ\-धौ\-शीय\-सीदाः} (पा॰सू॰~७.३.७८)~\arrow तिष्ठ्~अ~सि~\arrow \textcolor{red}{सेर्ह्यपिच्च} (पा॰सू॰~३.४.८७)~\arrow तिष्ठ्~अ~हि~\arrow \textcolor{red}{अतो हेः} (पा॰सू॰~६.४.१०५)~\arrow तिष्ठ्~अ~\arrow तिष्ठ।} पादपूर्त्यर्थं \textcolor{red}{सम्} इति निपातः प्रयुक्तः। अतो नाऽत्मनेपदम्। यथा \textcolor{red}{ह हि नु ननु खलु किल हन्त} इत्यादयः।\footnote{यद्वा \textcolor{red}{सन्} इति शत्रन्तं पृथक्पदम्। \textcolor{red}{असँ भुवि} (धा॰पा॰~१०६५)~\arrow अस्~\arrow \textcolor{red}{शेषात्कर्तरि परस्मैपदम्} (पा॰सू॰~१.३.७८)~\arrow \textcolor{red}{वर्तमाने लट्} (पा॰सू॰~३.२.१२३)~\arrow अस्~लट्~\arrow \textcolor{red}{लटः शतृशानचावप्रथमा\-समानाधिकरणे} (पा॰सू॰~३.२.१२४)~\arrow अस्~शतृँ~\arrow अस्~अत्~\arrow \textcolor{red}{कर्तरि शप्‌} (पा॰सू॰~३.१.६८)~\arrow अस्~शप्~अत्~\arrow \textcolor{red}{अदिप्रभृतिभ्यः शपः} (पा॰सू॰~२.४.७२)~\arrow अस्~अत्~\arrow \textcolor{red}{श्नसोरल्लोपः} (पा॰सू॰~६.४.१११)~\arrow स्~अत्~\arrow सत्~\arrow \textcolor{red}{कृत्तद्धित\-समासाश्च} (पा॰सू॰~१.२.४६)~\arrow प्रातिपदिक\-सञ्ज्ञा~\arrow विभक्ति\-कार्यम्~\arrow सत्~सुँ~\arrow सत्~स्~\arrow \textcolor{red}{उगिदचां सर्वनामस्थानेऽधातोः} (पा॰सू॰~७.१.७०)~\arrow \textcolor{red}{मिदचोऽन्त्यात्परः} (पा॰सू॰~१.१.४७)~\arrow स~नुँम्~त्~स्~\arrow स~न्~त्~स्~\arrow \textcolor{red}{हल्ङ्याब्भ्यो दीर्घात्सुतिस्यपृक्तं हल्} (पा॰सू॰~६.१.६८)~\arrow स~न्~त्~\arrow \textcolor{red}{संयोगान्तस्य लोपः} (पा॰सू॰~८.२.२३)~\arrow स~न्~\arrow सन्। शतुः \textcolor{red}{तिङ्शित्सार्वधातुकम्} (पा॰सू॰~३.४.११३) इत्यनेन सार्वधातुकत्वात् \textcolor{red}{अस्तेर्भूः} (पा॰सू॰~२.४.५२) इत्यस्याप्रवृत्तिः। \textcolor{red}{हे लक्ष्मण त्वं प्रयत्नेन सीतासंरक्षणोद्यतः सन् तिष्ठ} इति श्रीरामतात्पर्यम्।}\end{sloppypar}
\section[विश्रमस्व]{विश्रमस्व}
\centering\textcolor{blue}{कन्दमूलफलादीनि दत्त्वा स्वागतमब्रवीत्।\nopagebreak\\
मुने भुङ्क्ष्व फलादीनि विश्रमस्व यथासुखम्॥}\nopagebreak\\
\raggedleft{–~अ॰रा॰~३.७.३९}\\
\fontsize{14}{21}\selectfont\begin{sloppypar}\hyphenrules{nohyphenation}\justifying\noindent\hspace{10mm} दिवादित्वात् \textcolor{red}{विश्राम्य} इति प्रयोगः पाणिनीयो दीर्घे श्यनि परस्मैपदे।\footnote{वि~\textcolor{red}{श्रमुँ तपसि खेदे च} (धा॰पा॰~१२०४)~\arrow वि~श्रम्~\arrow \textcolor{red}{शेषात्कर्तरि परस्मैपदम्} (पा॰सू॰~१.३.७८)~\arrow \textcolor{red}{लोट् च} (पा॰सू॰~३.३.१६२)~\arrow वि~श्रम्~लोट्~\arrow वि~श्रम्~सिप्~\arrow वि~श्रम्~सि~\arrow \textcolor{red}{दिवादिभ्यः श्यन्‌} (पा॰सू॰~३.१.६९)~\arrow वि~श्रम्~श्यन्~सि~\arrow वि~श्रम्~य~सि~\arrow \textcolor{red}{शमामष्टानां दीर्घः श्यनि} (पा॰सू॰~७.३.६४)~\arrow वि~श्राम्~य~सि~\arrow \textcolor{red}{सेर्ह्यपिच्च} (पा॰सू॰~३.४.८७)~\arrow वि~श्राम्~य~हि~\arrow \textcolor{red}{अतो हेः} (पा॰सू॰~६.४.१०५)~\arrow विश्राम्य।} अत्र त्रयोऽप्यंशा विमर्श\-कोटिमञ्चन्ति।\footnote{यतो ह्यत्र दीर्घाभावः श्यन्नभाव आत्मनेपदञ्च।} \textcolor{red}{विश्रमणं विश्रमः}।\footnote{वि~\textcolor{red}{श्रमुँ तपसि खेदे च} (धा॰पा॰~१२०४)~\arrow \textcolor{red}{भावे} (पा॰सू॰~३.३.१८)~\arrow वि~श्रम्~घञ्~\arrow वि~श्रम्~अ~\arrow \textcolor{red}{अत उपधायाः} (पा॰सू॰~७.२.११६)~\arrow वृद्धिप्राप्तिः~\arrow \textcolor{red}{नोदात्तोपदेशस्य मान्तस्यानाचमेः} (पा॰सू॰~७.३.३४)~\arrow वृद्धिनिषेधः~\arrow वि~श्रम्~अ~\arrow विश्रम~\arrow विभक्तिकार्यम्~\arrow विश्रमः।} \textcolor{red}{विश्रममाचरति विश्रमति}।\footnote{विश्रम~\arrow \textcolor{red}{सर्वप्राति\-पदिकेभ्य आचारे क्विब्वा वक्तव्यः} (वा॰~३.१.११)~\arrow विश्रम~क्विँप्~\arrow विश्रम~व्~\arrow \textcolor{red}{वेरपृक्तस्य} (पा॰सू॰~६.१.६७)~\arrow विश्रम~\arrow \textcolor{red}{सनाद्यन्ता धातवः} (पा॰सू॰~३.१.३२)~\arrow धातुसञ्ज्ञा~\arrow \textcolor{red}{शेषात्कर्तरि परस्मैपदम्} (पा॰सू॰~१.३.७८)~\arrow \textcolor{red}{वर्तमाने लट्} (पा॰सू॰~३.२.१२३)~\arrow विश्रम~लट्~\arrow विश्रम~तिप्~\arrow विश्रम~ति~\arrow \textcolor{red}{कर्तरि शप्‌} (पा॰सू॰~३.१.६८)~\arrow विश्रम~शप्~ति~\arrow विश्रम~अ~ति~\arrow \textcolor{red}{अतो गुणे} (पा॰सू॰~६.१.९७)~\arrow विश्रम~ति~\arrow विश्रमति।} तदेव लोटि \textcolor{red}{विश्रम}।\footnote{विश्रम~\arrow धातुसञ्ज्ञा (पूर्ववत्)~\arrow \textcolor{red}{शेषात्कर्तरि परस्मैपदम्} (पा॰सू॰~१.३.७८)~\arrow \textcolor{red}{लोट् च} (पा॰सू॰~३.३.१६२)~\arrow विश्रम~लोट्~\arrow विश्रम~सिप्~\arrow विश्रम~सि~\arrow \textcolor{red}{कर्तरि शप्‌} (पा॰सू॰~३.१.६८)~\arrow विश्रम~शप्~सि~\arrow विश्रम~अ~सि~\arrow \textcolor{red}{अतो गुणे} (पा॰सू॰~६.१.९७)~\arrow विश्रम~सि~\arrow \textcolor{red}{सेर्ह्यपिच्च} (पा॰सू॰~३.४.८७)~\arrow विश्रम~हि~\arrow \textcolor{red}{अतो हेः} (पा॰सू॰~६.४.१०५)~\arrow विश्रम।} \textcolor{red}{हे स्व आत्मीय विश्रम}।\end{sloppypar}
\section[भक्षन्तु]{भक्षन्तु}
\centering\textcolor{blue}{शाद्वले प्राक्षिपद्रामः पृथक्पृथगनेकधा।\nopagebreak\\
भक्षन्तु पक्षिणः सर्वे तृप्तो भवतु पक्षिराट्॥}\nopagebreak\\
\raggedleft{–~अ॰रा॰~३.८.३९}\\
\fontsize{14}{21}\selectfont\begin{sloppypar}\hyphenrules{nohyphenation}\justifying\noindent\hspace{10mm} णिजन्त\-प्रसिद्धोऽयं \textcolor{red}{भक्षयन्तु} इति।\footnote{\textcolor{red}{भक्षँ अदने} (धा॰पा॰~१५५७) (\textcolor{red}{भ्लक्षँ/भक्षँ} इति माधवीयाधातुवृत्तिः, \textcolor{red}{भक्षँ} इति मैत्रेयः)~\arrow भक्ष्~\arrow \textcolor{red}{सत्यापपाश\-रूप\-वीणा\-तूल\-श्लोक\-सेना\-लोम\-त्वच\-वर्म\-वर्ण\-चूर्ण\-चुरादिभ्यो णिच् } (पा॰सू॰~३.१.२५)~\arrow भक्ष्~णिच्~\arrow भक्ष्~इ~\arrow भक्षि~\arrow \textcolor{red}{सनाद्यन्ता धातवः} (पा॰सू॰~३.१.३२)~\arrow धातु\-सञ्ज्ञा~\arrow \textcolor{red}{शेषात्कर्तरि परस्मैपदम्} (पा॰सू॰~१.३.७८)~\arrow \textcolor{red}{लोट् च} (पा॰सू॰~३.३.१६२)~\arrow भक्षि~लोट्~\arrow भक्षि~झि~\arrow \textcolor{red}{कर्तरि शप्‌} (पा॰सू॰~३.१.६८)~\arrow भक्षि~शप्~झि~\arrow भक्षि~अ~झि~\arrow \textcolor{red}{झोऽन्तः} (पा॰सू॰~७.१.३)~\arrow भक्षि~अ~अन्ति~\arrow \textcolor{red}{सार्वधातुकार्ध\-धातुकयोः} (पा॰सू॰~७.३.८४)~\arrow भक्षे~अ~अन्ति~\arrow \textcolor{red}{एचोऽयवायावः} (पा॰सू॰~६.१.७८)~\arrow भक्षय्~अ~अन्ति~\arrow \textcolor{red}{अतो गुणे} (पा॰सू॰~६.१.९७)~\arrow भक्षय्~अन्ति~\arrow \textcolor{red}{एरुः} (पा॰सू॰~३.४.८६)~\arrow भक्षय्~अन्तु~\arrow भक्षयन्तु।} किन्तु णिजन्ताः शुद्धा अपि भवन्त्यतो नापाणिनीयता।\footnote{\textcolor{red}{भक्षँ अदने} (धा॰पा॰~१५५७) (\textcolor{red}{भ्लक्षँ/भक्षँ} इति माधवीयाधातुवृत्तिः, \textcolor{red}{भक्षँ} इति मैत्रेयः)~\arrow भक्ष्~\arrow \textcolor{red}{शेषात्कर्तरि परस्मैपदम्} (पा॰सू॰~१.३.७८)~\arrow \textcolor{red}{लोट् च} (पा॰सू॰~३.३.१६२)~\arrow भक्ष्~लोट्~\arrow भक्ष्~झि~\arrow \textcolor{red}{कर्तरि शप्‌} (पा॰सू॰~३.१.६८)~\arrow भक्ष्~शप्~झि~\arrow भक्ष्~अ~झि~\arrow \textcolor{red}{झोऽन्तः} (पा॰सू॰~७.१.३)~\arrow भक्ष्~अ~अन्ति~\arrow \textcolor{red}{अतो गुणे} (पा॰सू॰~६.१.९७)~\arrow भक्ष्~अन्ति~\arrow \textcolor{red}{एरुः} (पा॰सू॰~३.४.८६)~\arrow भक्ष्~अन्तु~\arrow भक्षन्तु। यद्वा भक्षयन्तीति भक्षाः। \textcolor{red}{नन्दि\-ग्रहि\-पचादिभ्यो ल्युणिन्यचः} (पा॰सू॰~३.१.१३४) इत्यनेन कर्तरि पचाद्यचि \textcolor{red}{णेरनिटि} (पा॰सू॰~६.४.५१) इत्यनेन णिलोपे विभक्तिकार्ये। भक्षा इवाऽचरन्तु भक्षन्तु। भक्ष~\arrow \textcolor{red}{सर्वप्राति\-पदिकेभ्य आचारे क्विब्वा वक्तव्यः} (वा॰~३.१.११)~\arrow भक्ष~क्विँप्~\arrow भक्ष~व्~\arrow \textcolor{red}{वेरपृक्तस्य} (पा॰सू॰~६.१.६७)~\arrow भक्ष~\arrow \textcolor{red}{सनाद्यन्ता धातवः} (पा॰सू॰~३.१.३२)~\arrow धातुसञ्ज्ञा~\arrow \textcolor{red}{शेषात्कर्तरि परस्मैपदम्} (पा॰सू॰~१.३.७८)~\arrow \textcolor{red}{लोट् च} (पा॰सू॰~३.३.१६२)~\arrow भक्ष~लोट्~\arrow भक्ष~झि~\arrow \textcolor{red}{कर्तरि शप्‌} (पा॰सू॰~३.१.६८)~\arrow भक्ष~शप्~झि~\arrow भक्ष~अ~झि~\arrow \textcolor{red}{झोऽन्तः} (पा॰सू॰~७.१.३)~\arrow भक्ष~अ~अन्ति~\arrow \textcolor{red}{अतो गुणे} (पा॰सू॰~६.१.९७)~\arrow भक्ष~अन्ति~\arrow \textcolor{red}{अतो गुणे} (पा॰सू॰~६.१.९७)~\arrow भक्षन्ति~\arrow \textcolor{red}{एरुः} (पा॰सू॰~३.४.८६)~\arrow भक्षन्तु।}\end{sloppypar}
\vspace{2mm}
\centering ॥ इत्यरण्यकाण्डीयप्रयोगाणां विमर्शः ॥\nopagebreak\\
\vspace{4mm}
\pdfbookmark[2]{किष्किन्धाकाण्डम्}{Chap3Part2Kanda4}
\phantomsection
\addtocontents{toc}{\protect\setcounter{tocdepth}{2}}\addtocontents{toc}{\protect\setcounter{tocdepth}{2}}
\addcontentsline{toc}{subsection}{किष्किन्धाकाण्डीयप्रयोगाणां विमर्शः}
\addtocontents{toc}{\protect\setcounter{tocdepth}{0}}
\centering ॥ अथ किष्किन्धाकाण्डीयप्रयोगाणां विमर्शः ॥\nopagebreak\\
\section[आकाङ्क्षे]{आकाङ्क्षे}
\centering\textcolor{blue}{दाराः पुत्रा धनं राज्यं सर्वं त्वन्मायया कृतम्।\nopagebreak\\
अतोऽहं देवदेवेश नाकाङ्क्षेऽन्यत्प्रसीद मे॥}\nopagebreak\\
\raggedleft{–~अ॰रा॰~४.१.७८}\\
\fontsize{14}{21}\selectfont\begin{sloppypar}\hyphenrules{nohyphenation}\justifying\noindent\hspace{10mm} \textcolor{red}{आकाङ्क्षे} इत्यत्र परस्मैपदेन भवितव्यम्। धातोः परस्मैपदीयत्वात्। अतः \textcolor{red}{आकाङ्क्षामि} इति पाणिनीयम्।\footnote{आ~\textcolor{red}{काक्षिँ काङ्क्षायाम्} (धा॰पा॰~६६७)~\arrow आ~काक्ष्~\arrow \textcolor{red}{इदितो नुम् धातोः} (पा॰सू॰~७.१.५८)~\arrow \textcolor{red}{मिदचोऽन्त्यात्परः} (पा॰सू॰~१.१.४७)~\arrow आ~का~नुँम्~क्ष्~\arrow आ~कान्~क्ष्~\arrow \textcolor{red}{नश्चापदान्तस्य झलि} (पा॰सू॰~८.३.२४)~\arrow आ~कांक्ष्~\arrow \textcolor{red}{अनुस्वारस्य ययि परसवर्णः} (पा॰सू॰~८.४.५८)~\arrow आ~काङ्क्ष्~\arrow \textcolor{red}{शेषात्कर्तरि परस्मैपदम्} (पा॰सू॰~१.३.७८)~\arrow \textcolor{red}{वर्तमाने लट्} (पा॰सू॰~३.२.१२३)~\arrow आ~काङ्क्ष्~लट्~\arrow आ~काङ्क्ष्~मिप्~\arrow आ~काङ्क्ष्~मि~\arrow \textcolor{red}{कर्तरि शप्‌} (पा॰सू॰~३.१.६८)~\arrow आ~काङ्क्ष्~शप्~मि~\arrow आ~काङ्क्ष्~अ~मि~\arrow \textcolor{red}{अतो दीर्घो यञि} (पा॰सू॰~७.३.१०१)~\arrow आ~काङ्क्ष्~आ~मि~\arrow आकाङ्क्षामि।} \textcolor{red}{आकाङ्क्षे} इति तु \textcolor{red}{कर्तरि कर्म\-व्यतिहारे} (पा॰सू॰~१.३.१४) इति सूत्रेणाऽत्मनेपदम्।\footnote{आ~काङ्क्ष् (पूर्ववत्)~\arrow \textcolor{red}{कर्तरि कर्म\-व्यतिहारे} (पा॰सू॰~१.३.१४)~\arrow \textcolor{red}{वर्तमाने लट्} (पा॰सू॰~३.२.१२३)~\arrow आ~काङ्क्ष्~लट्~\arrow आ~काङ्क्ष्~इट्~\arrow आ~काङ्क्ष्~इ~\arrow \textcolor{red}{कर्तरि शप्‌} (पा॰सू॰~३.१.६८)~\arrow आ~काङ्क्ष्~शप्~इ~\arrow आ~काङ्क्ष्~अ~इ~\arrow \textcolor{red}{टित आत्मनेपदानां टेरे} (पा॰सू॰~३.४.७९)~\arrow आ~काङ्क्ष्~अ~ए~\arrow \textcolor{red}{अतो गुणे} (पा॰सू॰~६.१.९७)~\arrow आ~काङ्क्ष्~ए~\arrow आकाङ्क्षे।} कर्म\-व्यतिहारो नाम क्रिया\-विनिमयः।
अन्य\-करणीय\-कार्यस्यान्येन सम्पादनम्।
यथा कौमुद्यां \textcolor{red}{क्रिया\-विनिमये द्योत्ये कर्तर्यात्मनेपदं स्यात्} (वै॰सि॰कौ॰~२६८०)। कृत\-भगवद्दर्शनस्यान्यत्काङ्क्षणं स्पष्टं क्रिया\-विनिमयः। यद्वा \textcolor{red}{आकाङ्क्षणमाकाङ्क्षः} पचाद्यज्भावे।\footnote{\textcolor{red}{नन्दि\-ग्रहि\-पचादिभ्यो ल्युणिन्यचः} (पा॰सू॰~३.१.१३४) इत्यनेन। बाहुलकाद्भावे।} तस्मिन् \textcolor{red}{आकाङ्क्षे}। विषयत्वात्सप्तमी। ममाऽकाङ्क्षण इच्छायामन्यन्न। यद्वा \textcolor{red}{आकाङ्क्षणमाकाङ्क्षा}।\footnote{आ~काङ्क्ष् (पूर्ववत्)~\arrow \textcolor{red}{गुरोश्च हलः} (पा॰सू॰~३.३.१०३)~\arrow आ~काङ्क्ष्~अ~\arrow \textcolor{red}{अजाद्यतष्टाप्‌} (पा॰सू॰~४.१.४)~\arrow आ~काङ्क्ष्~अ~टाप्~\arrow आ~काङ्क्ष्~अ~आ~\arrow \textcolor{red}{अकः सवर्णे दीर्घः} (पा॰सू॰~६.१.१०१)~\arrow आ~काङ्क्ष्~आ~\arrow आकाङ्क्षा~\arrow \textcolor{red}{कृत्तद्धित\-समासाश्च} (पा॰सू॰~१.२.४६)~\arrow प्रातिपदिक\-सञ्ज्ञा~\arrow विभक्ति\-कार्यम्~\arrow आकाङ्क्षा~सुँ~\arrow \textcolor{red}{हल्ङ्याब्भ्यो दीर्घात्सुतिस्यपृक्तं हल्} (पा॰सू॰~६.१.६८)~\arrow आकाङ्क्षा। \textcolor{red}{अथ दोहदम्। इच्छा काङ्क्षा स्पृहेहा तृड्वाञ्छा लिप्सा मनोरथः॥ कामोऽभिलाषस्तर्षश्च सोऽत्यर्थं लालसा द्वयोः।} (अ॰को॰~१.७.२७–२८) इत्यमरः।} \textcolor{red}{आकाङ्क्षाऽस्त्यस्येत्याकाङ्क्षम्}। मनः। अर्शआदित्वादच्।\footnote{\textcolor{red}{अर्शआदिभ्योऽच्} (पा॰सू॰~५.२.१२७) इत्यनेन। आकाङ्क्षा~\arrow \textcolor{red}{अर्शआदिभ्योऽच्} (पा॰सू॰~५.२.१२७)~\arrow आकाङ्क्षा~अच्~\arrow आकाङ्क्षा~अ~\arrow \textcolor{red}{यचि भम्} (पा॰सू॰~१.४.१८)~\arrow भ\-सञ्ज्ञा~\arrow \textcolor{red}{यस्येति च} (पा॰सू॰~६.४.१४८)~\arrow आकाङ्क्ष्~अ~\arrow आकाङ्क्ष~\arrow \textcolor{red}{कृत्तद्धित\-समासाश्च} (पा॰सू॰~१.२.४६)~\arrow प्रातिपदिक\-सञ्ज्ञा~\arrow विभक्ति\-कार्यम्~\arrow आकाङ्क्ष~सुँ~\arrow \textcolor{red}{अतोऽम्} (पा॰सू॰~७.१.२४)~\arrow आकाङ्क्ष~अम्~\arrow \textcolor{red}{अमि पूर्वः} (पा॰सू॰~६.१.१०७)~\arrow आकाङ्क्षम्।} तस्मिन् \textcolor{red}{आकाङ्क्षे} त्वद्दर्शनेच्छावति मे मनसि नान्यत्किञ्चित्।\end{sloppypar}
\section[रक्षामहे]{रक्षामहे}
\centering\textcolor{blue}{चतुर्द्वारकपाटादीन् बद्ध्वा रक्षामहे पुरीम्।\nopagebreak\\
वानराणां तु राजानमङ्गदं कुरु भामिनि॥}\nopagebreak\\
\raggedleft{–~अ॰रा॰~४.३.३}\\
\fontsize{14}{21}\selectfont\begin{sloppypar}\hyphenrules{nohyphenation}\justifying\noindent\hspace{10mm} इहाऽपि कर्म\-व्यत्ययादात्मनेपदम्।\footnote{\textcolor{red}{कर्तरि कर्म\-व्यतिहारे} (पा॰सू॰~१.३.१४) इत्यनेन। पुरीरक्षणं सैन्यवानराणां कर्म न तु सर्वेषां वानराणाम्। अत्र सर्वे वानराः (\textcolor{red}{दुद्रुवुर्वानराः सर्वे किष्किन्धां भयविह्वलाः} अ॰रा॰~४.३.१) बाला वृद्धा विकलाङ्गा अपि रक्षाकर्म कुर्वन्तीति कर्मव्यतिहारः। \textcolor{red}{रक्षँ पालने} (धा॰पा॰~६५८)~\arrow रक्ष्~\arrow \textcolor{red}{कर्तरि कर्म\-व्यतिहारे} (पा॰सू॰~१.३.१४)~\arrow (पा॰सू॰~३.३.१३)~\arrow \textcolor{red}{वर्तमाने लट्} (पा॰सू॰~३.२.१२३)~\arrow रक्ष्~लृँट्~\arrow रक्ष्~महिङ्~\arrow रक्ष्~महि~\arrow \textcolor{red}{कर्तरि शप्‌} (पा॰सू॰~३.१.६८)~\arrow रक्ष्~शप्~महि~\arrow रक्ष्~अ~महि~\arrow \textcolor{red}{अतो दीर्घो यञि} (पा॰सू॰~७.३.१०१)~\arrow रक्ष्~आ~महि~\arrow \textcolor{red}{टित आत्मनेपदानां टेरे} (पा॰सू॰~३.४.७९)~\arrow रक्ष्~आ~महे~\arrow रक्षामहे।} यद्वा \textcolor{red}{हे} इति पृथक्पदम्। \textcolor{red}{हे तारे वयं नगरीं रक्षाम} इति प्रार्थनायां लोट्। सम्प्रश्ने वा। \textcolor{red}{लोट् च} (पा॰सू॰~३.३.१६२) इत्यनेन।\footnote{\textcolor{red}{रक्षँ पालने} (धा॰पा॰~६५८)~\arrow रक्ष्~\arrow \textcolor{red}{शेषात्कर्तरि परस्मैपदम्} (पा॰सू॰~१.३.७८)~\arrow \textcolor{red}{लोट् च} (पा॰सू॰~३.३.१६२)~\arrow रक्ष्~लोट्~\arrow रक्ष्~मस्~\arrow \textcolor{red}{कर्तरि शप्‌} (पा॰सू॰~३.१.६८)~\arrow रक्ष्~शप्~मस्~\arrow रक्ष्~अ~मस्~\arrow \textcolor{red}{आडुत्तमस्य पिच्च} (पा॰सू॰~३.४.९२)~\arrow रक्ष्~अ~आट्~मस्~\arrow \textcolor{red}{अकः सवर्णे दीर्घः} (पा॰सू॰~६.१.१०१)~\arrow रक्ष्~आ~मस्~\arrow \textcolor{red}{लोटो लङ्वत्‌} (पा॰सू॰~३.४.८५)~\arrow ङिद्वत्त्वम्~\arrow नित्यं ङितः~\arrow रक्ष्~आ~म~\arrow रक्षाम।}\end{sloppypar}
\section[लिप्यसे]{लिप्यसे}
\centering\textcolor{blue}{ध्यात्वा मद्रूपमनिशमालोचय मयोदितम्।\nopagebreak\\
प्रवाहपतितं कार्यं कुर्वत्यपि न लिप्यसे॥}\nopagebreak\\
\raggedleft{–~अ॰रा॰~४.३.३५}\\
\fontsize{14}{21}\selectfont\begin{sloppypar}\hyphenrules{nohyphenation}\justifying\noindent\hspace{10mm} \textcolor{red}{वर्तमान\-सामीप्ये वर्तमानवद्वा} (पा॰सू॰~३.३.१३१) इत्यनेनाऽसन्न\-भविष्यत्काले वर्तमानम्।\footnote{\textcolor{red}{लिपँ उपदेहे} (धा॰पा॰~१४३३)~\arrow लिप्~\arrow \textcolor{red}{भावकर्मणोः} (पा॰सू॰~१.३.१३)~\arrow \textcolor{red}{वर्तमान\-सामीप्ये वर्तमानवद्वा} (पा॰सू॰~३.३.१३१)~\arrow \textcolor{red}{वर्तमाने लट्} (पा॰सू॰~३.२.१२३)~\arrow लिप्~लट्~\arrow लिप्~थास्~\arrow \textcolor{red}{सार्वधातुके यक्} (पा॰सू॰~३.१.६७)~\arrow लिप्~यक्~थास्~\arrow लिप्~य~थास्~\arrow \textcolor{red}{ग्क्ङिति च} (पा॰सू॰~१.१.५)~\arrow लघूपध\-गुण\-निषेधः~\arrow लिप्~य~थास्~\arrow \textcolor{red}{थासस्से} (पा॰सू॰~३.४.८०)~\arrow लिप्~य~से~\arrow लिप्यसे।}\end{sloppypar}
\section[अनुसेविरे]{अनुसेविरे}
\centering\textcolor{blue}{चरन्तं परमात्मानं ज्ञात्वा सिद्धगणा भुवि।\nopagebreak\\
मृगपक्षिगणा भूत्वा राममेवानुसेविरे॥}\nopagebreak\\
\raggedleft{–~अ॰रा॰~४.४.५}\\
\fontsize{14}{21}\selectfont\begin{sloppypar}\hyphenrules{nohyphenation}\justifying\noindent\hspace{10mm}
लिड्लकार\-प्रयोगात् \textcolor{red}{अनुसिषेविरे}\footnote{अनु~\textcolor{red}{षेवृँ सेवने} (धा॰पा॰~५०१)~\arrow अनु~षेव्~\arrow \textcolor{red}{धात्वादेः षः सः} (पा॰सू॰~६.१.६४)~\arrow अनु~सेव्~\arrow \textcolor{red}{अनुदात्तङित आत्मने\-पदम्} (पा॰सू॰~१.३.१२)~\arrow \textcolor{red}{परोक्षे लिट्} (पा॰सू॰~३.२.११५)~\arrow अनु~सेव्~लिट्~\arrow अनु~सेव्~झ~\arrow \textcolor{red}{लिटि धातोरनभ्यासस्य} (पा॰सू॰~६.१.८)~\arrow अनु~सेव्~सेव्~झ~\arrow \textcolor{red}{हलादिः शेषः} (पा॰सू॰~७.४.६०)~\arrow अनु~से~सेव्~झ~\arrow \textcolor{red}{ह्रस्वः} (पा॰सू॰~७.४.५९)~\arrow \textcolor{red}{एच इग्घ्रस्वादेशे} (पा॰सू॰~१.१.४८)~\arrow अनु~सि~सेव्~झ~\arrow \textcolor{red}{लिटस्तझयोरेशिरेच्} (पा॰सू॰~३.४.८१)~\arrow अनु~सि~सेव्~इरेच्~\arrow अनु~सि~सेव्~इरे~\arrow \textcolor{red}{आदेश\-प्रत्यययोः} (पा॰सू॰~८.३.५९)~\arrow अनु~सि~षेव्~इरे~\arrow अनुसिषेविरे।} इति प्रयोक्तव्ये सिलोपाच्च सिद्धम् \textcolor{red}{अनुसेविरे} इति। 
अर्थात् \textcolor{red}{विनाऽपि प्रत्ययं पूर्वोत्तर\-पद\-लोपो वक्तव्यः} (वा॰~५.३.८३) इत्यनेनाऽदेर्हलोऽचश्च लोपे\footnote{बाहुलकादपदस्यापि लोप इति शेषः।} 
\textcolor{red}{अनुसेविरे} इति। यद्वा \textcolor{red}{अनुसेवां कृतवन्तः} इति भूत\-काल औणादिके \textcolor{red}{डिरच्} प्रत्यये भावे सप्तम्यन्ते च \textcolor{red}{अनुसेविरे}।\footnote{\textcolor{red}{चेरुः} इति चाध्याहार्यम्। \textcolor{red}{कार्याद्विद्यादनूबन्धम्} (भा॰पा॰सू॰~३.३.१) \textcolor{red}{केचिदविहिता अप्यूह्याः} (वै॰सि॰कौ॰~३१६९) इत्यनुसारमूह्योऽ\-यमविहितो \textcolor{red}{डिरच्} प्रत्ययः। अनुसेवा~\arrow \textcolor{red}{उणादयो बहुलम्} (पा॰सू॰~३.३.१)~\arrow अनुसेवा~डिरच्~\arrow अनुसेवा~इर~\arrow \textcolor{red}{डित्यभस्याप्यनु\-बन्धकरण\-सामर्थ्यात्} (वा॰~६.४.१४३)~\arrow अनुसेव्~इर~\arrow अनुसेविर~\arrow \textcolor{red}{कृत्तद्धित\-समासाश्च} (पा॰सू॰~१.२.४६)~\arrow प्रातिपदिक\-सञ्ज्ञा~\arrow विभक्ति\-कार्यम्~\arrow अनुसेविर~ङि~\arrow अनुसेविर~इ~\arrow \textcolor{red}{आद्गुणः} (पा॰सू॰~६.१.८७)~\arrow अनुसेविरे।} यद्वा \textcolor{red}{अनुसेवनमनु\-सेवा}।\footnote{\textcolor{red}{अनु}\-पूर्वकात् \textcolor{red}{षेवृँ सेवने} (धा॰पा॰~५०१) धातोः \textcolor{red}{गुरोश्च हलः} (पा॰सू॰~३.३.१०३) इत्यनेन भावे स्त्रियां \textcolor{red}{अ}प्रत्यये ततश्च \textcolor{red}{अजाद्यतष्टाप्‌} (पा॰सू॰~४.१.४) इत्यनेन टापि विभक्तिकार्ये।} \textcolor{red}{अनुसेवामाचरत्यनु\-सेवयति}।\footnote{अनुसेवा~\arrow \textcolor{red}{तत्करोति तदाचष्टे} (धा॰पा॰ ग॰सू॰~१८७)~\arrow अनुसेवा~णिच्~\arrow अनुसेवा~इ~\arrow \textcolor{red}{णाविष्ठवत्प्राति\-पदिकस्य पुंवद्भाव\-रभाव\-टिलोप\-यणादि\-परार्थम्} (वा॰~६.४.४८)~\arrow अनुसेव्~इ~\arrow अनुसेवि~\arrow \textcolor{red}{सनाद्यन्ता धातवः} (पा॰सू॰~३.१.३२)~\arrow धातुसञ्ज्ञा~\arrow \textcolor{red}{शेषात्कर्तरि परस्मैपदम्} (पा॰सू॰~१.३.७८)~\arrow \textcolor{red}{वर्तमाने लट्} (पा॰सू॰~३.२.१२३)~\arrow अनुसेवि~तिप्~\arrow अनुसेवि~ति~\arrow \textcolor{red}{कर्तरि शप्‌} (पा॰सू॰~३.१.६८)~\arrow अनुसेवि~शप्~ति~\arrow अनुसेवि~अ~ति~\arrow \textcolor{red}{सार्वधातुकार्ध\-धातुकयोः} (पा॰सू॰~७.३.८४)~\arrow अनुसेवे~अ~ति~\arrow \textcolor{red}{एचोऽयवायावः} (पा॰सू॰~६.१.७८)~\arrow अनुसेवय्~अ~ति~\arrow अनुसेवयति।} पुनः \textcolor{red}{अनुसेव्यत\footnote{अनुसेवि~\arrow धातु\-सञ्ज्ञा (पूर्ववत्)~\arrow \textcolor{red}{भावकर्मणोः} (पा॰सू॰~१.३.१३)~\arrow \textcolor{red}{वर्तमाने लट्} (पा॰सू॰~३.२.१२३)~\arrow अनुसेवि~लट्~\arrow अनुसेवि~त~\arrow \textcolor{red}{सार्वधातुके यक्} (पा॰सू॰~३.१.६७)~\arrow अनुसेवि~यक्~त~\arrow अनुसेवि~य~त~\arrow \textcolor{red}{णेरनिटि} (पा॰सू॰~६.४.५१)~\arrow अनुसेव्~य~त~\arrow \textcolor{red}{टित आत्मनेपदानां टेरे} (पा॰सू॰~३.४.७९)~\arrow अनुसेव्~य~ते~\arrow अनुसेव्यते।} इति अनुसेव्} हलन्त\-नपुंसक\-लिङ्गे।\footnote{\textcolor{red}{अनुसेवि}\-धातोः \textcolor{red}{सम्पदादिभ्‍यः क्विप्‌} (वा॰~३.३.१०८) इत्यनेन भावे क्विपि \textcolor{red}{णेरनिटि} (पा॰सू॰~६.४.५१) इति णिलोपे। बाहुलकान्नपुंसकलिङ्गम्।} लिड्लकारे \textcolor{red}{इण्‌}\-धातोः (\textcolor{red}{इण् गतौ} धा॰पा॰~१०४५) \textcolor{red}{कर्तरि कर्म\-व्यतिहारे} (पा॰सू॰~१.३.१४) इत्यनेनाऽत्मनेपदम्। \textcolor{red}{लिटस्तझयोरेशिरेच्} (पा॰सू॰~३.४.८१) इत्यनेन \textcolor{red}{इरेच्} प्रत्यये \textcolor{red}{ईयिरे}।\footnote{\textcolor{red}{इण् गतौ} (धा॰पा॰~१०४५)~\arrow इ~\arrow \textcolor{red}{कर्तरि कर्म\-व्यतिहारे} (पा॰सू॰~१.३.१४)~\arrow \textcolor{red}{परोक्षे लिट्} (पा॰सू॰~३.२.११५)~\arrow इ~लिट्~\arrow इ~झ~\arrow \textcolor{red}{लिटि धातोरनभ्यासस्य} (पा॰सू॰~६.१.८)~\arrow इ~इ~झ~\arrow \textcolor{red}{लिटस्तझयोरेशिरेच्} (पा॰सू॰~३.४.८१)~\arrow इ~इ~इरेच्~\arrow इ~इ~इरे~\arrow \textcolor{red}{असंयोगाल्लिट् कित्} (पा॰सू॰~१.२.५)~\arrow \textcolor{red}{दीर्घ इणः किति} (पा॰सू॰~७.४.६९)~\arrow ई~इ~इरे~\arrow \textcolor{red}{इणो यण्} (पा॰सू॰~६.४.८१)~\arrow ई~य्~इरे~\arrow ईयिरे।}
\textcolor{red}{विनाऽपि प्रत्ययं पूर्वोत्तर\-पद\-लोपो वक्तव्यः} (वा॰~५.३.८३) इत्यनेन धात्विकार\-लोपे\footnote{बाहुलकादपदस्यापि लोप इति शेषः। धातोरिकारौ धात्विकारौ तयोर्लोपः धात्विकारलोपस्तस्मिन् धात्विकारलोपे।} \textcolor{red}{इरे}। 
\textcolor{red}{अनुसेव् इरे} इति \textcolor{red}{अनुसेविरे}।\end{sloppypar}
\section[समारभेत्]{समारभेत्}
\centering\textcolor{blue}{किं पुनर्भक्ष्यभोज्यादि गन्धपुष्पाक्षतादिकम्।\nopagebreak\\
पूजाद्रव्याणि सर्वाणि सम्पाद्यैवं समारभेत्॥}\nopagebreak\\
\raggedleft{–~अ॰रा॰~४.४.२०}\\
\fontsize{14}{21}\selectfont\begin{sloppypar}\hyphenrules{nohyphenation}\justifying\noindent\hspace{10mm} \textcolor{red}{समा}\-पूर्वकात् \textcolor{red}{रभ्‌}\-धातोः (\textcolor{red}{रभँ राभस्ये} धा॰पा॰~९७४) विधिलिङि \textcolor{red}{समारभेत}\footnote{सम्~आङ्~\textcolor{red}{रभँ राभस्ये} (धा॰पा॰~९७४)~\arrow सम्~आ~रभ्~\arrow \textcolor{red}{अनुदात्तङित आत्मने\-पदम्} (पा॰सू॰~१.३.१२)~\arrow \textcolor{red}{विधि\-निमन्‍त्रणामन्‍त्रणाधीष्‍ट\-सम्प्रश्‍न\-प्रार्थनेषु लिङ्} (पा॰सू॰~३.३.१६१)~\arrow सम्~आ~रभ्~लिङ्~\arrow सम्~आ~रभ्~त~\arrow \textcolor{red}{कर्तरि शप्‌} (पा॰सू॰~३.१.६८)~\arrow समारभ~शप्~त~\arrow समारभ~अ~त~\arrow \textcolor{red}{अतो गुणे} (पा॰सू॰~६.१.९७)~\arrow समारभ~त~\arrow \textcolor{red}{यासुट् परस्मैपदेषूदात्तो ङिच्च} (पा॰सू॰~३.४.१०३)~\arrow \textcolor{red}{आद्यन्तौ टकितौ} (पा॰सू॰~१.१.४६)~\arrow समारभ~यासुँट्~त~\arrow समारभ~यास्~त~\arrow \textcolor{red}{सुट् तिथोः} (पा॰सू॰~३.४.१०७)~\arrow \textcolor{red}{आद्यन्तौ टकितौ} (पा॰सू॰~१.१.४६)~\arrow समारभ~यास्~सुँट्~त~\arrow समारभ~यास्~स्~त~\arrow \textcolor{red}{लिङः सलोपोऽनन्त्यस्य} (पा॰सू॰~७.२.७९)~\arrow समारभ~या~त~\arrow \textcolor{red}{अतो येयः} (पा॰सू॰~७.२.८०)~\arrow समारभ~इय्~त~\arrow \textcolor{red}{लोपो व्योर्वलि} (पा॰सू॰~६.१.६६)~\arrow समारभ~इ~त~\arrow \textcolor{red}{आद्गुणः} (पा॰सू॰~६.१.८७)~\arrow समारभे~त~\arrow समारभेत।} इति प्रयोक्तव्ये \textcolor{red}{समारभेत्} इति प्रयुक्तम्। \textcolor{red}{समारभत इति समारभः}। पचाद्यच्।\footnote{\textcolor{red}{नन्दि\-ग्रहि\-पचादिभ्यो ल्युणिन्यचः} (पा॰सू॰~३.१.१३४) इत्यनेन।} \textcolor{red}{समारभ इवाचरेत्} इति विधिलिङि \textcolor{red}{भवेत्} इव।\footnote{समारभ~\arrow \textcolor{red}{सर्वप्राति\-पदिकेभ्य आचारे क्विब्वा वक्तव्यः} (वा॰~३.१.११)~\arrow समारभ~क्विँप्~\arrow समारभ~व्~\arrow \textcolor{red}{वेरपृक्तस्य} (पा॰सू॰~६.१.६७)~\arrow समारभ~\arrow \textcolor{red}{सनाद्यन्ता धातवः} (पा॰सू॰~३.१.३२)~\arrow धातुसञ्ज्ञा~\arrow \textcolor{red}{शेषात्कर्तरि परस्मैपदम्} (पा॰सू॰~१.३.७८)~\arrow \textcolor{red}{विधि\-निमन्‍त्रणामन्‍त्रणाधीष्‍ट\-सम्प्रश्‍न\-प्रार्थनेषु लिङ्} (पा॰सू॰~३.३.१६१)~\arrow समारभ~लिङ~\arrow समारभ~तिप्~\arrow समारभ~ति~\arrow \textcolor{red}{कर्तरि शप्‌} (पा॰सू॰~३.१.६८)~\arrow समारभ~शप्~ति~\arrow समारभ~अ~ति~\arrow \textcolor{red}{अतो गुणे} (पा॰सू॰~६.१.९७)~\arrow समारभ~ति~\arrow \textcolor{red}{यासुट् परस्मैपदेषूदात्तो ङिच्च} (पा॰सू॰~३.४.१०३)~\arrow \textcolor{red}{आद्यन्तौ टकितौ} (पा॰सू॰~१.१.४६)~\arrow समारभ~यासुँट्~ति~\arrow समारभ~यास्~ति~\arrow \textcolor{red}{सुट् तिथोः} (पा॰सू॰~३.४.१०७)~\arrow \textcolor{red}{आद्यन्तौ टकितौ} (पा॰सू॰~१.१.४६)~\arrow समारभ~यास्~सुँट्~ति~\arrow समारभ~यास्~स्~ति~\arrow \textcolor{red}{लिङः सलोपोऽनन्त्यस्य} (पा॰सू॰~७.२.७९)~\arrow समारभ~या~ति~\arrow\textcolor{red}{अतो येयः} (पा॰सू॰~७.२.८०)~\arrow समारभ~इय्~ति~\arrow \textcolor{red}{लोपो व्योर्वलि} (पा॰सू॰~६.१.६६)~\arrow समारभ~इ~ति~\arrow \textcolor{red}{आद्गुणः} (पा॰सू॰~६.१.८७)~\arrow समारभे~ति~\arrow \textcolor{red}{इतश्च} (पा॰सू॰~३.४.१००)~\arrow समारभे~त्~\arrow समारभेत्। यद्वा \textcolor{red}{अनुदात्तेत्त्व\-लक्षणमात्मने\-पदमनित्यम्} (प॰शे॰~९३.४) इत्यपि समाधानम्।
}\end{sloppypar}
\section[प्रकारयेत्]{प्रकारयेत्}
\centering\textcolor{blue}{दशावरणपूजां वै ह्यागमोक्तां प्रकारयेत्।\nopagebreak\\
नीराजनैर्धूपदीपैर्नैवेद्यैर्बहुविस्तरैः॥}\nopagebreak\\
\raggedleft{–~अ॰रा॰~४.४.२९}\\
\fontsize{14}{21}\selectfont\begin{sloppypar}\hyphenrules{nohyphenation}\justifying\noindent\hspace{10mm} \textcolor{red}{प्र}\-पूर्वकात् \textcolor{red}{कृ}\-धातोः (\textcolor{red}{डुकृञ् करणे} धा॰पा॰~१४७२) विधिलिङि \textcolor{red}{प्रकुर्यात्}\footnote{प्र~\textcolor{red}{डुकृञ् करणे} (धा॰पा॰~१४७२)~\arrow प्र~कृ~\arrow \textcolor{red}{शेषात्कर्तरि परस्मैपदम्} (पा॰सू॰~१.३.७८)~\arrow \textcolor{red}{विधि\-निमन्‍त्रणामन्‍त्रणाधीष्‍ट\-सम्प्रश्‍न\-प्रार्थनेषु लिङ्} (पा॰सू॰~३.३.१६१)~\arrow प्र~कृ~लिङ्~\arrow प्र~कृ~तिप्~\arrow प्र~कृ~ति~\arrow \textcolor{red}{तनादि\-कृञ्भ्य उः} (पा॰सू॰~३.१.७९)~\arrow प्र~कृ~उ~ति~\arrow \textcolor{red}{सार्वधातुकार्ध\-धातुकयोः} (पा॰सू॰~७.३.८४)~\arrow \textcolor{red}{उरण् रपरः} (पा॰सू॰~१.१.५१)~\arrow प्र~कर्~उ~ति~\arrow \textcolor{red}{यासुट् परस्मैपदेषूदात्तो ङिच्च} (पा॰सू॰~३.४.१०३)~\arrow \textcolor{red}{आद्यन्तौ टकितौ} (पा॰सू॰~१.१.४६)~\arrow प्र~कर्~उ~यासुँट्~ति~\arrow प्र~कर्~उ~यास्~ति~\arrow \textcolor{red}{सुट् तिथोः} (पा॰सू॰~३.४.१०७)~\arrow \textcolor{red}{आद्यन्तौ टकितौ} (पा॰सू॰~१.१.४६)~\arrow प्र~कर्~उ~यास्~सुँट्~ति~\arrow प्र~कर्~उ~यास्~स्~ति~\arrow 
\textcolor{red}{लिङः सलोपोऽनन्त्यस्य} (पा॰सू॰~७.२.७९)~\arrow प्र~कर्~उ~या~ति~\arrow \textcolor{red}{अत उत्सार्वधातुके} (पा॰सू॰~६.४.११०)~\arrow प्र~कुर्~उ~या~ति~\arrow \textcolor{red}{ये च} (पा॰सू॰~६.४.१०९)~\arrow प्र~कुर्~या~ति~\arrow \textcolor{red}{इतश्च} (पा॰सू॰~३.४.१००)~\arrow प्र~कुर्~या~त्~\arrow प्रकुर्यात्।} इति प्रयोक्तव्ये \textcolor{red}{प्रकारयेत्} इति प्रयुक्तम्। \textcolor{red}{प्रकारमाचक्षीत} इति \textcolor{red}{प्रकारयेत्}। आचक्षाण\-णिजन्ताद्विधि\-लिङ्।\footnote{अत्र हेतुमण्णिञ्न। अर्थानुपपत्तेः। प्रकार~\arrow \textcolor{red}{तत्करोति तदाचष्टे} (धा॰पा॰ ग॰सू॰~१८७)~\arrow प्रकार~णिच्~\arrow प्रकार~इ~\arrow \textcolor{red}{णाविष्ठवत्प्राति\-पदिकस्य पुंवद्भाव\-रभाव\-टिलोप\-यणादि\-परार्थम्} (वा॰~६.४.४८)~\arrow प्रकार्~इ~\arrow प्रकारि~\arrow \textcolor{red}{सनाद्यन्ता धातवः} (पा॰सू॰~३.१.३२)~\arrow धातुसञ्ज्ञा~\arrow \textcolor{red}{शेषात्कर्तरि परस्मैपदम्} (पा॰सू॰~१.३.७८)~\arrow \textcolor{red}{विधि\-निमन्‍त्रणामन्‍त्रणाधीष्‍ट\-सम्प्रश्‍न\-प्रार्थनेषु लिङ्} (पा॰सू॰~३.३.१६१)~\arrow प्रकारि~लिङ~\arrow प्रकारि~तिप्~\arrow प्रकारि~ति~\arrow \textcolor{red}{कर्तरि शप्‌} (पा॰सू॰~३.१.६८)~\arrow प्रकारि~शप्~ति~\arrow प्रकारि~अ~ति~\arrow \textcolor{red}{सार्वधातुकार्ध\-धातुकयोः} (पा॰सू॰~७.३.८४)~\arrow प्रकारे~अ~ति~\arrow \textcolor{red}{यासुट् परस्मैपदेषूदात्तो ङिच्च} (पा॰सू॰~३.४.१०३)~\arrow \textcolor{red}{आद्यन्तौ टकितौ} (पा॰सू॰~१.१.४६)~\arrow प्रकारे~अ~यासुँट्~ति~\arrow प्रकारे~अ~यास्~ति~\arrow \textcolor{red}{सुट् तिथोः} (पा॰सू॰~३.४.१०७)~\arrow \textcolor{red}{आद्यन्तौ टकितौ} (पा॰सू॰~१.१.४६)~\arrow प्रकारे~अ~यास्~सुँट्~ति~\arrow प्रकारे~अ~यास्~स्~ति~\arrow \textcolor{red}{लिङः सलोपोऽनन्त्यस्य} (पा॰सू॰~७.२.७९)~\arrow प्रकारे~अ~या~ति~\arrow \textcolor{red}{अतो येयः} (पा॰सू॰~७.२.८०)~\arrow प्रकारे~अ~इय्~ति~\arrow \textcolor{red}{लोपो व्योर्वलि} (पा॰सू॰~६.१.६६)~\arrow प्रकारे~अ~इ~ति~\arrow \textcolor{red}{एचोऽयवायावः} (पा॰सू॰~६.१.७८)~\arrow प्रकारय्~अ~इ~ति~\arrow \textcolor{red}{आद्गुणः} (पा॰सू॰~६.१.८७)~\arrow प्रकारय्~ए~ति~\arrow \textcolor{red}{इतश्च} (पा॰सू॰~३.४.१००)~\arrow प्रकारय्~ए~त्~\arrow प्रकारयेत्।}\end{sloppypar}
\section[हन्यसे]{हन्यसे}
\centering\textcolor{blue}{करोमीति प्रतिज्ञाय सीतायाः परिमार्गणम्।\nopagebreak\\
न करोषि कृतघ्नस्त्वं हन्यसे वालिवद्द्रुतम्॥}\nopagebreak\\
\raggedleft{–~अ॰रा॰~४.४.४८}\\
\fontsize{14}{21}\selectfont\begin{sloppypar}\hyphenrules{nohyphenation}\justifying\noindent\hspace{10mm} शीघ्रता\-बोधनार्थं वर्तमान\-सामीप्ये लट्।\footnote{\textcolor{red}{वर्तमान\-सामीप्ये वर्तमानवद्वा} (पा॰सू॰~३.३.१३१) इत्यनेन। \textcolor{red}{हनँ हिंसागत्योः} (धा॰पा॰~१०१२)~\arrow हन्~\arrow \textcolor{red}{भावकर्मणोः} (पा॰सू॰~१.३.१३)~\arrow \textcolor{red}{वर्तमान\-सामीप्ये वर्तमानवद्वा} (पा॰सू॰~३.३.१३१)~\arrow \textcolor{red}{वर्तमाने लट्} (पा॰सू॰~३.२.१२३)~\arrow हन्~लट्~\arrow हन्~थास्~\arrow \textcolor{red}{सार्वधातुके यक्} (पा॰सू॰~३.१.६७)~\arrow हन्~यक्~थास्~\arrow हन्~य~थास्~\arrow \textcolor{red}{थासस्से} (पा॰सू॰~३.४.८०)~\arrow हन्~य~से~\arrow हन्यसे। }\end{sloppypar}
\section[वधयिष्यति]{वधयिष्यति}
\label{sec:vadhayisyati}
\centering\textcolor{blue}{सुग्रीवः स्वयमागत्य सर्ववानरयूथपैः।\nopagebreak\\
वधयिष्यति दैत्यौघान् रावणं च हनिष्यति॥}\nopagebreak\\
\raggedleft{–~अ॰रा॰~४.५.४७}\\
\fontsize{14}{21}\selectfont\begin{sloppypar}\hyphenrules{nohyphenation}\justifying\noindent\hspace{10mm} \textcolor{red}{वधमाचरन्तीति वधन्ति}।\footnote{वध~\arrow \textcolor{red}{सर्वप्राति\-पदिकेभ्य आचारे क्विब्वा वक्तव्यः} (वा॰~३.१.११)~\arrow वध~क्विँप्~\arrow वध~व्~\arrow \textcolor{red}{वेरपृक्तस्य} (पा॰सू॰~६.१.६७)~\arrow वध~\arrow \textcolor{red}{सनाद्यन्ता धातवः} (पा॰सू॰~३.१.३२)~\arrow धातुसञ्ज्ञा~\arrow \textcolor{red}{शेषात्कर्तरि परस्मैपदम्} (पा॰सू॰~१.३.७८)~\arrow \textcolor{red}{वर्तमाने लट्} (पा॰सू॰~३.२.१२३)~\arrow वध~लट्~\arrow वध~झि~\arrow \textcolor{red}{झोऽन्तः} (पा॰सू॰~७.१.३)~\arrow वध~अन्ति~\arrow \textcolor{red}{कर्तरि शप्‌} (पा॰सू॰~३.१.६८)~\arrow वध~शप्~अन्ति~\arrow वध~अ~अन्ति~\arrow \textcolor{red}{अतो गुणे} (पा॰सू॰~६.१.९७)~\arrow वध~अन्ति~\arrow \textcolor{red}{अतो गुणे} (पा॰सू॰~६.१.९७)~\arrow वधन्ति।} \textcolor{red}{तान् प्रेरयति वधयति}।\footnote{वध~\arrow धातुसञ्ज्ञा (पूर्ववत्)~\arrow \textcolor{red}{हेतुमति च} (पा॰सू॰~३.१.२६)~\arrow वध~णिच्~\arrow वध~इ~\arrow \textcolor{red}{णाविष्ठवत्प्राति\-पदिकस्य पुंवद्भाव\-रभाव\-टिलोप\-यणादि\-परार्थम्} (वा॰~६.४.४८)~\arrow वध्~इ~\arrow वधि~\arrow \textcolor{red}{सनाद्यन्ता धातवः} (पा॰सू॰~३.१.३२)~\arrow धातु\-सञ्ज्ञा~\arrow \textcolor{red}{शेषात्कर्तरि परस्मैपदम्} (पा॰सू॰~१.३.७८)~\arrow \textcolor{red}{वर्तमाने लट्} (पा॰सू॰~३.२.१२३)~\arrow वधि~लट्~\arrow वधि~तिप्~\arrow वधि~ति~\arrow \textcolor{red}{कर्तरि शप्‌} (पा॰सू॰~३.१.६८)~\arrow वधि~शप्~ति~\arrow वधि~अ~ति~\arrow \textcolor{red}{सार्वधातुकार्ध\-धातुकयोः} (पा॰सू॰~७.३.८४)~\arrow वधे~अ~ति~\arrow \textcolor{red}{एचोऽयवायावः} (पा॰सू॰~६.१.७८)~\arrow वधय्~अ~ति~\arrow वधयति।} तद्भविष्यत्काले \textcolor{red}{वधयिष्यति}।\footnote{वधि~\arrow \textcolor{red}{धातु\-सञ्ज्ञा} (पूर्ववत्)~\arrow \textcolor{red}{शेषात्कर्तरि परस्मैपदम्} (पा॰सू॰~१.३.७८)~\arrow \textcolor{red}{लृट् शेषे च} (पा॰सू॰~३.३.१३)~\arrow वधि~लृट्~\arrow वधि~तिप्~\arrow वधि~ति~\arrow \textcolor{red}{स्यतासी लृलुटोः} (पा॰सू॰~३.१.३३)~\arrow वधि~स्य~ति~\arrow \textcolor{red}{आर्धधातुकस्येड्वलादेः} (पा॰सू॰~७.२.३५)~\arrow वधि~इट्~स्य~ति~\arrow वधि~इ~स्य~ति~\arrow \textcolor{red}{सार्वधातुकार्ध\-धातुकयोः} (पा॰सू॰~७.३.८४)~\arrow वधे~इ~स्य~ति~\arrow \textcolor{red}{एचोऽयवायावः} (पा॰सू॰~६.१.७८)~\arrow वधय्~इ~स्य~ति~\arrow \textcolor{red}{आदेश\-प्रत्यययोः} (पा॰सू॰~८.३.५९)~\arrow वधय्~इ~ष्य~ति~\arrow वधयिष्यति।} यद्वा \textcolor{red}{वध्} इति स्वतन्त्रश्चौरादिको धातुः। तस्माल्लृटि \textcolor{red}{वधयिष्यति}। \textcolor{red}{मितां ह्रस्वः} (पा॰सू॰~६.४.९२) इति ह्रस्वः।\footnote{\textcolor{red}{बहुलमेतन्निदर्शनम्} (धा॰पा॰ ग॰सू॰~१९३८) \textcolor{red}{आकृतिगणोऽयम्} (धा॰पा॰ ग॰सू॰~१९९२) \textcolor{red}{भूवादिष्वेतदन्तेषु दशगणीषु धातूनां पाठो निदर्शनाय तेन स्तम्भुप्रभृतयः सौत्राश्चुलुम्पादयो वाक्यकारीयाः प्रयोगसिद्धा विक्लवत्यादयश्च} (मा॰धा॰वृ॰~१०.३२८) इत्यनुसारमाकृति\-गणत्वाच्चुरादि\-गण ऊह्योऽयं नामधातुः। \textcolor{red}{सङ्ग्राम युद्धे} (धा॰पा॰~१९२२) \textcolor{red}{संवर संवरणे} (धा॰पा॰~१९९२) इतिवत्। वृद्ध्यभावात् \textcolor{red}{ज्ञपँ ज्ञान\-ज्ञापन\-मारण\-तोषण\-निशान\-निशामनेषु} (धा॰पा॰~१६२४) इतिवन्मिदप्ययम्। वध्~\arrow \textcolor{red}{सत्याप\-पाश\-रूप\-वीणा\-तूल\-श्लोक\-सेना\-लोम\-त्वच\-वर्म\-वर्ण\-चूर्ण\-चुरादिभ्यो णिच्} (पा॰सू॰~३.१.२५)~\arrow वध्~णिच्~\arrow वध्~इ~\arrow \textcolor{red}{अत उपधायाः} (पा॰सू॰~७.२.११६)~\arrow वाध्~इ~\arrow \textcolor{red}{मितां ह्रस्वः} (पा॰सू॰~६.४.९२)~\arrow वध्~इ~\arrow वधि~\arrow \textcolor{red}{सनाद्यन्ता धातवः} (पा॰सू॰~३.१.३२)~\arrow धातु\-सञ्ज्ञा~\arrow शेषा प्रक्रिया पूर्ववत्।} यद्वा \textcolor{red}{अयनमय्}। भावे क्विप्।\footnote{\textcolor{red}{अयँ गतौ} (धा॰पा॰~४७४)~\arrow अय्~\arrow \textcolor{red}{सम्पदादिभ्‍यः क्विप्} (वा॰~३.३.१०८)~\arrow अय्~क्विँप्~\arrow अय्~व्~\arrow \textcolor{red}{वेरपृक्तस्य} (पा॰सू॰~६.१.६७)~\arrow अय्~\arrow विभक्ति\-कार्यम्~\arrow अय्~सुँ~\arrow \textcolor{red}{हल्ङ्याब्भ्यो दीर्घात्सुतिस्यपृक्तं हल्} (पा॰सू॰~६.१.६८)~\arrow \textcolor{red}{अय्}।} \textcolor{red}{वधस्य अय् इति वधय्} शकन्ध्वादित्वात्पर\-रूपे।\footnote{\textcolor{red}{शकन्ध्वादिषु पररूपं वाच्यम्} (वा॰~६.१.९४) इत्यनेन।} \textcolor{red}{इषुँ इच्छायाम्} (धा॰पा॰~१३५१) दिवादिः। \textcolor{red}{वधय् वध\-प्राप्तिमिच्छतीष्यति वधयिष्यति}।\footnote{\textcolor{red}{लिङ्गमशिष्यं लोकाश्रयत्वाल्लिङ्गस्य} (भा॰पा॰सू॰~२.१.३६) इति नियमेन \textcolor{red}{अय्} इति शब्दं नपुंसक\-लिङ्गे पठित्वा द्वितीयायां विभक्तौ अय्~अम् इति स्थिते \textcolor{red}{स्वमोर्नपुंसकात्‌} (पा॰सू॰~७.१.२३) इत्यनेनाम्लुकि \textcolor{red}{अय्} इत्येव। \pageref{sec:vadhisyami}तमे पृष्ठे \ref{sec:vadhisyami} \nameref{sec:vadhisyami} इति प्रयोगस्य विमर्शमपि पश्यन्तु।}\end{sloppypar}
\section[प्रधर्षथ]{प्रधर्षथ}
\centering\textcolor{blue}{कुतो वा कस्य दूता वा मत्स्थानं किं प्रधर्षथ।\nopagebreak\\
तच्छ्रुत्वा हनुमानाह श्रृणु वक्ष्यामि देवि ते॥}\nopagebreak\\
\raggedleft{–~अ॰रा॰~४.६.४२}\\
\fontsize{14}{21}\selectfont\begin{sloppypar}\hyphenrules{nohyphenation}\justifying\noindent\hspace{10mm} ण्यजन्तत्यागे रूपमिदम्।\footnote{\textcolor{red}{धृष्‌}\-धातुः (\textcolor{red}{धृषँ प्रसहने} धा॰पा॰~१८५०) आधृषीयान्तर्गणे पठितः। तत्र \textcolor{red}{आ धृषाद्वा} (धा॰पा॰ ग॰सू॰~१८०६) इति गणसूत्रस्याधिकाराद्वैकल्पिक\-णिच्प्रत्ययः। णिजभावे \textcolor{red}{प्रधर्षथ} इति रूपम्। प्र~\textcolor{red}{धृषँ प्रसहने}~\arrow प्र~धृष्~\arrow \textcolor{red}{शेषात्कर्तरि परस्मैपदम्} (पा॰सू॰~१.३.७८)~\arrow \textcolor{red}{वर्तमाने लट्} (पा॰सू॰~३.२.१२३)~\arrow प्र~धृष्~लट्~\arrow प्र~धृष्~थ~\arrow \textcolor{red}{कर्तरि शप्‌} (पा॰सू॰~३.१.६८)~\arrow प्र~धृष्~शप्~थ~\arrow प्र~धृष्~अ~थ~\arrow \textcolor{red}{पुगन्त\-लघूपधस्य च} (पा॰सू॰~७.३.८६)~\arrow \textcolor{red}{उरण् रपरः} (पा॰सू॰~१.१.५१)~\arrow प्र~धर्ष्~अ~थ~\arrow प्रधर्षथ। णिचि तु \textcolor{red}{प्रधर्षयथ} इति रूपम्। धृष्~\arrow \textcolor{red}{सत्याप\-पाश\-रूप\-वीणा\-तूल\-श्लोक\-सेना\-लोम\-त्वच\-वर्म\-वर्ण\-चूर्ण\-चुरादिभ्यो णिच्} (पा॰सू॰~३.१.२५)~\arrow धृष्~णिच्~\arrow धृष्~इ~\arrow \textcolor{red}{पुगन्त\-लघूपधस्य च} (पा॰सू॰~७.३.८६)~\arrow \textcolor{red}{उरण् रपरः} (पा॰सू॰~१.१.५१)~\arrow धर्ष्~इ~\arrow धर्षि~\arrow \textcolor{red}{सनाद्यन्ता धातवः} (पा॰सू॰~३.१.३२)~\arrow धातु\-सञ्ज्ञा। प्र~धर्षि~\arrow \textcolor{red}{शेषात्कर्तरि परस्मैपदम्} (पा॰सू॰~१.३.७८)~\arrow \textcolor{red}{वर्तमाने लट्} (पा॰सू॰~३.२.१२३)~\arrow प्र~धर्षि~लट्~\arrow प्र~धर्षि~थ~\arrow \textcolor{red}{कर्तरि शप्‌} (पा॰सू॰~३.१.६८)~\arrow प्र~धर्षि~शप्~थ~\arrow प्र~धर्षि~अ~थ~\arrow \textcolor{red}{सार्वधातुकार्धधातुकयोः} (पा॰सू॰~७.३.८४)~\arrow प्र~धर्षे~अ~थ~\arrow \textcolor{red}{एचोऽयवायावः} (पा॰सू॰~६.१.७८)~\arrow प्र~धर्षय्~अ~थ~\arrow प्रधर्षयथ।}\end{sloppypar}
\section[मृगयध्वम्]{मृगयध्वम्}
\centering\textcolor{blue}{मृगयध्वमिति प्राह ततो वयमुपागताः।\nopagebreak\\
ततो वनं विचिन्वन्तो जानकीं जलकाङ्क्षिणः॥}\nopagebreak\\
\raggedleft{–~अ॰रा॰~४.६.४६}\\
\fontsize{14}{21}\selectfont\begin{sloppypar}\hyphenrules{nohyphenation}\justifying\noindent\hspace{10mm} अत्र श्रीरामेणाऽत्मीयत्वात्क्रिया\-फलस्य वानर\-रूप\-कर्तृ\-गामित्व आत्मनेपदम्।\footnote{\textcolor{red}{मृग अन्वेषणे} (धा॰पा॰~१९००) चौरादिक\-धातुरागर्वीयत्वादात्मने\-पदी। परन्तु \textcolor{red}{मृगणां कुरुत} इति प्राकृतेऽर्थे णिचि कृते तूभयपदी। यथा – \textcolor{red}{निवृत्तप्रेषणाद्धातोः प्राकृतेऽर्थे णिजुच्यते} (वा॰प॰~३.७.६०)। तत्रात्मनेपद\-प्राप्तेः कारणमिदं प्रादर्शि ग्रन्थकारैः। मृग~णिच्~\arrow मृग~इ~\arrow \textcolor{red}{अतो लोपः} (पा॰सू॰~६.४.४८)~\arrow मृग्~इ~\arrow \textcolor{red}{सनाद्यन्ता धातवः} (पा॰सू॰~३.१.३२)~\arrow \textcolor{red}{णिचश्च} (पा॰सू॰~१.३.७४)~\arrow \textcolor{red}{लोट् च} (पा॰सू॰~३.३.१६२)~\arrow मृग्~इ~लोट्~\arrow मृग्~इ~ध्वम्~\arrow \textcolor{red}{कर्तरि शप्} (पा॰सू॰~३.१.६८)~\arrow मृग्~इ~शप्~ध्वम्~\arrow मृग्~इ~अ~ध्वम्~\arrow \textcolor{red}{सार्वधातुकार्ध\-धातुकयोः} (पा॰सू॰~७.३.८४)~\arrow मृग्~ए~अ~ध्वम्~\arrow \textcolor{red}{एचोऽयवायावः} (पा॰सू॰~६.१.७८)~\arrow मृग्~अय्~अ~ध्वम्~\arrow मृगयध्वम्।}\end{sloppypar}
\section[स्थास्यामहे]{स्थास्यामहे}
\centering\textcolor{blue}{पुनर्वैकुण्ठमासाद्य सुखं स्थास्यामहे वयम्।\nopagebreak\\
इत्यङ्गदमथाऽश्वास्य गता विन्ध्यं महाचलम्॥}\nopagebreak\\
\raggedleft{–~अ॰रा॰~४.७.२२}\\
\fontsize{14}{21}\selectfont\begin{sloppypar}\hyphenrules{nohyphenation}\justifying\noindent\hspace{10mm} सीतान्वेषणेऽसफलान् मरणे कृत\-निश्चयान् वानरान् हनुमान् प्रतिबोधयति \textcolor{red}{वैकुण्ठे सुखं स्थास्यामहे}। इह \textcolor{red}{उपान्मन्त्र\-करणे} (पा॰सू॰~१.३.२५) इत्यनेनाऽत्मनेपदम्।\footnote{उप~\textcolor{red}{ष्ठा गतिनिवृत्तौ} (धा॰पा॰~९२८)~\arrow \textcolor{red}{धात्वादेः षः सः} (पा॰सू॰~६.१.६४)~\arrow निमित्तापाये नैमित्तिकस्याप्यपायः~\arrow उप~स्था~\arrow \textcolor{red}{उपान्मन्त्र\-करणे} (पा॰सू॰~१.३.२५)~\arrow \textcolor{red}{लृट् शेषे च} (पा॰सू॰~३.३.१३)~\arrow उप~लृँट्~\arrow उप~स्था~महिङ्~\arrow उप~स्था~महि~\arrow \textcolor{red}{स्यतासी लृलुटोः} (पा॰सू॰~३.१.३३)~\arrow उप~स्था~स्य~महि~\arrow \textcolor{red}{अतो दीर्घो यञि} (पा॰सू॰~७.३.१०१)~\arrow उप~स्था~स्या~महि~\arrow \textcolor{red}{टित आत्मनेपदानां टेरे} (पा॰सू॰~३.४.७९)~\arrow उप~स्था~स्या~महे~\arrow उपस्थास्यामहे।} उप\-शब्दस्य लोपः।\footnote{\textcolor{red}{विनाऽपि प्रत्ययं पूर्वोत्तर\-पद\-लोपो वक्तव्यः} (वा॰~५.३.८३) इत्यनेन।}\end{sloppypar}
\section[नयध्वम्]{नयध्वम्}
\centering\textcolor{blue}{वाक्यसाहाय्यं करिष्येऽहं भवतां प्लवगेश्वराः।\nopagebreak\\
भ्रातुः सलिलदानाय नयध्वं मां जलान्तिकम्॥}\nopagebreak\\
\raggedleft{–~अ॰रा॰~४.७.४८}\\
\fontsize{14}{21}\selectfont\begin{sloppypar}\hyphenrules{nohyphenation}\justifying\noindent\hspace{10mm} इह \textcolor{red}{सम्माननोत्सञ्जनाचार्य\-करण\-ज्ञान\-भृति\-विगणन\-व्ययेषु नियः} (पा॰सू॰~१.३.३६) इत्यनेनाऽत्मनेपदम्। सम्माननेऽयं प्रयोगः। \textcolor{red}{नयध्वम्}।\footnote{\textcolor{red}{णीञ् प्रापणे} (धा॰पा॰~९०१)~\arrow \textcolor{red}{णो नः} (पा॰सू॰~६.१.६५)~\arrow नीञ्~\arrow नी~\arrow \textcolor{red}{सम्माननोत्सञ्जनाचार्य\-करण\-ज्ञान\-भृति\-विगणन\-व्ययेषु नियः} (पा॰सू॰~१.३.३६)~\arrow \textcolor{red}{लोट् च} (पा॰सू॰~३.३.१६२)~\arrow नी~लोट्~\arrow नी~ध्वम्~\arrow \textcolor{red}{कर्तरि शप्} (पा॰सू॰~३.१.६८)~\arrow नी~शप्~ध्वम्~\arrow नी~अ~ध्वम्~\arrow \textcolor{red}{सार्वधातुकार्ध\-धातुकयोः} (पा॰सू॰~७.३.८४)~\arrow ने~अ~ध्वम्~\arrow \textcolor{red}{एचोऽयवायावः} (पा॰सू॰~६.१.७८)~\arrow नय्~अ~ध्वम्~\arrow नयध्वम्।} \textcolor{red}{सम्मानं कुरुत}। तदिच्छापालनमेव सम्माननम्। उत्सञ्जने वाऽऽत्मनेपदम्।\end{sloppypar}
\section[इयात्]{इयात्}
\centering\textcolor{blue}{तत्पुनः पञ्चरात्रेण बुद्बुदाकारतामियात्।\nopagebreak\\
सप्तरात्रेण तदपि मांसपेशित्वमाप्नुयात्॥}\nopagebreak\\
\raggedleft{–~अ॰रा॰~४.८.२३}\\
\fontsize{14}{21}\selectfont\begin{sloppypar}\hyphenrules{nohyphenation}\justifying\noindent\hspace{10mm} \textcolor{red}{एति}\footnote{\textcolor{red}{इण् गतौ} (पा॰सू॰~१०४५)~\arrow इ~\arrow \textcolor{red}{शेषात्कर्तरि परस्मैपदम्} (पा॰सू॰~१.३.७८)~\arrow \textcolor{red}{वर्तमाने लट्} (पा॰सू॰~३.२.१२३)~\arrow इ~लट्~\arrow इ~तिप्~\arrow इ~ति~\arrow \textcolor{red}{कर्तरि शप्‌} (पा॰सू॰~३.१.६८)~\arrow इ~अ~ति~\arrow \textcolor{red}{अदिप्रभृतिभ्यः शपः} (पा॰सू॰~२.४.७२)~\arrow इ~ति~\arrow \textcolor{red}{सार्वधातुकार्ध\-धातुकयोः} (पा॰सू॰~७.३.८४)~\arrow ए~ति~\arrow एति।} इति प्रयोक्तव्ये \textcolor{red}{इयात्}\footnote{\textcolor{red}{इण् गतौ} (पा॰सू॰~१०४५)~\arrow इ~\arrow \textcolor{red}{शेषात्कर्तरि परस्मैपदम्} (पा॰सू॰~१.३.७८)~\arrow \textcolor{red}{आशंसावचने लिङ्} (पा॰सू॰~३.३.१३४)~\arrow इ~लिङ्~\arrow इ~तिप्~\arrow इ~ति~\arrow \textcolor{red}{यासुट् परस्मैपदेषूदात्तो ङिच्च} (पा॰सू॰~३.४.१०३)~\arrow \textcolor{red}{आद्यन्तौ टकितौ} (पा॰सू॰~१.१.४६)~\arrow इ~यासुँट्~ति~\arrow इ~यास्~ति~\arrow \textcolor{red}{सुट् तिथोः} (पा॰सू॰~३.४.१०७)~\arrow \textcolor{red}{आद्यन्तौ टकितौ} (पा॰सू॰~१.१.४६)~\arrow इ~यास्~सुँट्~ति~\arrow इ~यास्~स्~ति~\arrow \textcolor{red}{लिङः सलोपोऽनन्त्यस्य} (पा॰सू॰~७.२.७९)~\arrow इ~या~ति~\arrow \textcolor{red}{ग्क्ङिति च} (पा॰सू॰~१.१.५)~\arrow गुणनिषेधः~\arrow इ~या~ति~\arrow \textcolor{red}{इतश्च} (पा॰सू॰~३.४.१००)~\arrow इ~या~त्~\arrow इयात्।} इति। \textcolor{red}{आशंसावचने लिङ्} (पा॰सू॰~३.३.१३४) इत्यनेन लिङ्लकारः।\footnote{एवमेव \textcolor{red}{आप्नुयात्} इत्यत्रापि बोध्यम्।}\end{sloppypar}
\section[म्रियेत]{म्रियेत}
\centering\textcolor{blue}{नाभिसूत्राल्परन्ध्रेण मातृभुक्तान्नसारतः।\nopagebreak\\
वर्धते गर्भतः पिण्डो न म्रियेत स्वकर्मतः॥}\nopagebreak\\
\raggedleft{–~अ॰रा॰~४.८.३२}\\
\fontsize{14}{21}\selectfont\begin{sloppypar}\hyphenrules{nohyphenation}\justifying\noindent\hspace{10mm} \textcolor{red}{म्रियते}\footnote{\textcolor{red}{मृङ् प्राणत्यागे} (धा॰पा॰~१४०३)~\arrow मृ~\arrow \textcolor{red}{म्रियतेर्लुङ्‌लिङोश्च} (पा॰सू॰~१.३.६१)~\arrow \textcolor{red}{वर्तमाने लट्} (पा॰सू॰~३.२.१२३)~\arrow मृ~लट्~\arrow मृ~त~\arrow \textcolor{red}{तुदादिभ्यः शः} (पा॰सू॰~३.१.७७)~\arrow मृ~श~त~\arrow मृ~अ~त~\arrow \textcolor{red}{रिङ् शयग्लिङ्क्षु} (पा॰सू॰~७.४.२८)~\arrow \textcolor{red}{ङिच्च} (पा॰सू॰~१.१.५३)~\arrow म्~रिङ्~अ~त~\arrow म्~रि~अ~त~\arrow \textcolor{red}{अचि श्नुधातुभ्रुवां य्वोरियङुवङौ} (पा॰सू॰~६.४.७७)~\arrow \textcolor{red}{ङिच्च} (पा॰सू॰~१.१.५३)~\arrow म्~र्~इयँङ्~अ~त~\arrow म्~र्~इय्~अ~त~\arrow \textcolor{red}{टित आत्मनेपदानां टेरे} (पा॰सू॰~३.४.७९)~\arrow म्~र्~इय्~अ~ते~\arrow म्रियते।} इति प्रयोक्तव्ये \textcolor{red}{म्रियेत} इति प्रयोगस्तु \textcolor{red}{शकि लिङ् च} (पा॰सू॰~३.३.१७२) इति सूत्रेण शक्यार्थे लिङ्प्रयोगे साधु।\footnote{\textcolor{red}{मृङ् प्राणत्यागे} (धा॰पा॰~१४०३)~\arrow मृ~\arrow \textcolor{red}{म्रियतेर्लुङ्‌लिङोश्च} (पा॰सू॰~१.३.६१)~\arrow \textcolor{red}{शकि लिङ् च} (पा॰सू॰~३.३.१७२)~\arrow मृ~लिङ्~\arrow मृ~त~\arrow \textcolor{red}{तुदादिभ्यः शः} (पा॰सू॰~३.१.७७)~\arrow मृ~श~त~\arrow मृ~अ~त~\arrow \textcolor{red}{रिङ् शयग्लिङ्क्षु} (पा॰सू॰~७.४.२८)~\arrow \textcolor{red}{ङिच्च} (पा॰सू॰~१.१.५३)~\arrow म्~रिङ्~अ~त~\arrow म्~रि~अ~त~\arrow \textcolor{red}{अचि श्नुधातुभ्रुवां य्वोरियङुवङौ} (पा॰सू॰~६.४.७७)~\arrow \textcolor{red}{ङिच्च} (पा॰सू॰~१.१.५३)~\arrow म्~र्~इयँङ्~अ~त~\arrow म्~र्~इय्~अ~त~\arrow \textcolor{red}{लिङः सीयुट्} (पा॰सू॰~३.४.१०२)~\arrow म्~र्~इय्~अ~सीयुँट्~त~\arrow म्~र्~इय्~अ~सीय्~त~\arrow \textcolor{red}{सुट् तिथोः} (पा॰सू॰~३.४.१०७)~\arrow \textcolor{red}{आद्यन्तौ टकितौ} (पा॰सू॰~१.१.४६)~\arrow म्~र्~इय्~अ~सीय्~सुँट्~त~\arrow म्~र्~इय्~अ~सीय्~स्~त~\arrow \textcolor{red}{लिङः सलोपोऽनन्त्यस्य} (पा॰सू॰~७.२.७९)~\arrow म्~र्~इय्~अ~ईय्~त~\arrow \textcolor{red}{लोपो व्योर्वलि} (पा॰सू॰~६.१.६६)~\arrow म्~र्~इय्~अ~ई~त~\arrow \textcolor{red}{आद्गुणः} (पा॰सू॰~६.१.८७)~\arrow म्~र्~इय्~ए~त~\arrow म्रियेत।}\end{sloppypar}
\section[क्षिपामि]{क्षिपामि}
\centering\textcolor{blue}{लङ्कां सपर्वतां धृत्वा रामस्याग्रे क्षिपाम्यहम्।\nopagebreak\\
यद्वा दृष्ट्वैव यास्यामि जानकीं शुभलक्षणाम्॥}\nopagebreak\\
\raggedleft{–~अ॰रा॰~४.९.२४}\\
\fontsize{14}{21}\selectfont\begin{sloppypar}\hyphenrules{nohyphenation}\justifying\noindent\hspace{10mm} \textcolor{red}{वर्तमान\-सामीप्ये वर्तमानवद्वा} (पा॰सू॰~३.३.१३१) इत्यनेन लट्प्रयोगः।\footnote{\textcolor{red}{क्षिपँ प्रेरणे} (धा॰पा॰~१२८५)~\arrow क्षिप्~\arrow \textcolor{red}{शेषात्कर्तरि परस्मैपदम्} (पा॰सू॰~१.३.७८)~\arrow \textcolor{red}{वर्तमान\-सामीप्ये वर्तमानवद्वा} (पा॰सू॰~३.३.१३१)~\arrow \textcolor{red}{वर्तमाने लट्} (पा॰सू॰~३.२.१२३)~\arrow क्षिप्~लट्~\arrow क्षिप्~मिप्~\arrow क्षिप्~मि~\arrow \textcolor{red}{तुदादिभ्यः शः} (पा॰सू॰~३.१.७७)~\arrow क्षिप्~श~मि~\arrow क्षिप्~अ~मि~\arrow \textcolor{red}{सार्वधातुकमपित्} (पा॰सू॰~१.२.४)~\arrow ङित्त्वम्~\arrow \textcolor{red}{ग्क्ङिति च} (पा॰सू॰~१.१.५)~\arrow लघूपध\-गुण\-निषेधः~\arrow क्षिप्~अ~मि~\arrow \textcolor{red}{अतो दीर्घो यञि} (पा॰सू॰~७.३.१०१)~\arrow क्षिप्~आ~मि~\arrow क्षिपामि।}\end{sloppypar}
\vspace{2mm}
\centering ॥ इति किष्किन्धाकाण्डीयप्रयोगाणां विमर्शः ॥\nopagebreak\\
\vspace{4mm}
\pdfbookmark[2]{सुन्दरकाण्डम्}{Chap3Part2Kanda5}
\phantomsection
\addtocontents{toc}{\protect\setcounter{tocdepth}{2}}
\addcontentsline{toc}{subsection}{सुन्दरकाण्डीयप्रयोगाणां विमर्शः}
\addtocontents{toc}{\protect\setcounter{tocdepth}{0}}
\centering ॥ अथ सुन्दरकाण्डीयप्रयोगाणां विमर्शः ॥\nopagebreak\\
\section[पश्यामि]{पश्यामि}
\centering\textcolor{blue}{अमोघं रामनिर्मुक्तं महाबाणमिवाखिलाः।\nopagebreak\\
पश्याम्यद्यैव रामस्य पत्नीं जनकनन्दिनीम्॥}\nopagebreak\\
\raggedleft{–~अ॰रा॰~५.१.३}\\
\centering\textcolor{blue}{कृतार्थोऽहं कृतार्थोऽहं पुनः पश्यामि राघवम्।\nopagebreak\\
प्राणप्रयाणसमये यस्य नाम सकृत्स्मरन्॥}\nopagebreak\\
\raggedleft{–~अ॰रा॰~५.१.४}\\
\fontsize{14}{21}\selectfont\begin{sloppypar}\hyphenrules{nohyphenation}\justifying\noindent\hspace{10mm} अत्र शीघ्रता\-द्योतनाय वर्तमान\-समीपे भविष्यति \textcolor{red}{वर्तमान\-सामीप्ये वर्तमानवद्वा} (पा॰सू॰~३.३.१३१) इत्यनेन लट्।\footnote{\textcolor{red}{दृशिँर प्रेक्षणे} (धा॰पा॰~९८८)~\arrow दृश्~\arrow \textcolor{red}{शेषात्कर्तरि परस्मैपदम्} (पा॰सू॰~१.३.७८)~\arrow \textcolor{red}{वर्तमान\-सामीप्ये वर्तमानवद्वा} (पा॰सू॰~३.३.१३१)~\arrow \textcolor{red}{वर्तमाने लट्} (पा॰सू॰~३.२.१२३)~\arrow दृश्~लट्~\arrow दृश्~मिप्~\arrow दृश्~मि~\arrow \textcolor{red}{कर्तरि शप्‌} (पा॰सू॰~३.१.६८)~\arrow दृश्~शप्~मि~\arrow दृश्~अ~मि~\arrow \textcolor{red}{पाघ्रा\-ध्मास्थाम्ना\-दाण्दृश्यर्त्ति\-सर्त्तिशदसदां पिब\-जिघ्र\-धम\-तिष्ठ\-मन\-यच्छ\-पश्यर्च्छ\-धौ\-शीय\-सीदाः} (पा॰सू॰~७.३.७८)~\arrow पश्य्~अ~मि~\arrow \textcolor{red}{अतो दीर्घो यञि} (पा॰सू॰~७.३.१०१)~\arrow पश्य्~आ~मि~\arrow पश्यामि।}\end{sloppypar}
\section[विवेक्ष्ये]{विवेक्ष्ये}
\centering\textcolor{blue}{विवेक्ष्ये\footnote{\textcolor{red}{निवेक्ष्ये} इति पाठभेदः। तत्र तु \textcolor{red}{नेर्विशः} (पा॰सू॰~१.३.१७) इत्यनेनात्मनेपदम्।} देहि मे मार्गं सुरसायै नमोऽस्तु ते।\nopagebreak\\
इत्युक्त्वा पुनरेवाह सुरसा क्षुधितास्म्यहम्॥}\nopagebreak\\
\raggedleft{–~अ॰रा॰~५.१.१६}\\
\fontsize{14}{21}\selectfont\begin{sloppypar}\hyphenrules{nohyphenation}\justifying\noindent\hspace{10mm} कर्म\-व्यतिहार आत्मनेपदम्।\footnote{\textcolor{red}{कर्तरि कर्मव्यतिहारे} (पा॰सू॰~१.३.१४) इत्यनेन। वि~\textcolor{red}{विशँ प्रवेशने} (धा॰पा॰~१४२४)~\arrow वि~विश्~\arrow \textcolor{red}{कर्तरि कर्म\-व्यतिहारे} (पा॰सू॰~१.३.१४)~\arrow \textcolor{red}{लृट् शेषे च} (पा॰सू॰~३.३.१३)~\arrow वि~विश्~लृट्~\arrow वि~विश्~इट्~\arrow वि~विश्~इ~\arrow \textcolor{red}{स्यतासी लृलुटोः} (पा॰सू॰~३.१.३३)~\arrow वि~विश्~स्य~इ~\arrow \textcolor{red}{पुगन्त\-लघूपधस्य च} (पा॰सू॰~७.३.८६)~\arrow वि~वेश्~स्य~इ~\arrow \textcolor{red}{व्रश्चभ्रस्ज\-सृजमृज\-यजराज\-भ्राजच्छशां षः} (पा॰सू॰~८.२.३६)~\arrow वि~वेष्~स्य~इ~\arrow \textcolor{red}{षढोः कः सि} (पा॰सू॰~८.२.४१)~\arrow वि~वेक्~स्य~इ~\arrow \textcolor{red}{आदेश\-प्रत्यययोः} (पा॰सू॰~८.३.५९)~\arrow वि~वेक्~ष्य~इ~\arrow \textcolor{red}{टित आत्मनेपदानां टेरे} (पा॰सू॰~३.४.७९)~\arrow वि~वेक्~ष्य~ए~\arrow \textcolor{red}{अतो गुणे} (पा॰सू॰~६.१.९७)~\arrow वि~वेक्~ष्ये~\arrow विवेक्ष्ये।}\end{sloppypar}
\section[भक्षयेत्]{भक्षयेत्}
\centering\textcolor{blue}{सिंहिका नाम सा घोरा जलमध्ये स्थिता सदा।\nopagebreak\\
आकाशगामिनां छायामाक्रम्याऽकृष्य भक्षयेत्॥}\nopagebreak\\
\raggedleft{–~अ॰रा॰~५.१.३५}\\
\fontsize{14}{21}\selectfont\begin{sloppypar}\hyphenrules{nohyphenation}\justifying\noindent\hspace{10mm} अत्र \textcolor{red}{विधि\-निमन्त्रणामन्त्रणाधीष्ट\-सम्प्रश्न\-प्रार्थनेषु लिङ्} (पा॰सू॰~३.३.१६१) इत्यनेन \textcolor{red}{हेतु\-हेतुमतोर्लिङ्} (पा॰सू॰~३.३.१५६) इत्यनेन वा लिङ्लकारः।\footnote{भक्षँ \textcolor{red}{अदने} (धा॰पा॰~१५५७)~\arrow भक्ष्~\arrow \textcolor{red}{सत्याप\-पाश\-रूप\-वीणा\-तूल\-श्लोक\-सेना\-लोम\-त्वच\-वर्म\-वर्ण\-चूर्ण\-चुरादिभ्यो णिच्} (पा॰सू॰~३.१.२५)~\arrow भक्ष्~णिच्~\arrow भक्ष्~इ~\arrow भक्षि~\arrow \textcolor{red}{सनाद्यन्ता धातवः} (पा॰सू॰~३.१.३२)~\arrow धातु\-सञ्ज्ञा~\arrow \textcolor{red}{शेषात्कर्तरि परस्मैपदम्} (पा॰सू॰~१.३.७८)~\arrow \textcolor{red}{विधि\-निमन्त्रणामन्त्रणाधीष्ट\-सम्प्रश्न\-प्रार्थनेषु लिङ्} (पा॰सू॰~३.३.१६१) \textcolor{red}{हेतु\-हेतुमतोर्लिङ्} (पा॰सू॰~३.३.१५६) वा~\arrow भक्षि~लिङ~\arrow भक्षि~तिप्~\arrow भक्षि~ति~\arrow \textcolor{red}{कर्तरि शप्‌} (पा॰सू॰~३.१.६८)~\arrow भक्षि~शप्~ति~\arrow भक्षि~अ~ति~\arrow \textcolor{red}{सार्वधातुकार्ध\-धातुकयोः} (पा॰सू॰~७.३.८४)~\arrow भक्षे~अ~ति~\arrow \textcolor{red}{यासुट् परस्मैपदेषूदात्तो ङिच्च} (पा॰सू॰~३.४.१०३)~\arrow \textcolor{red}{आद्यन्तौ टकितौ} (पा॰सू॰~१.१.४६)~\arrow भक्षे~अ~यासुँट्~ति~\arrow भक्षे~अ~यास्~ति~\arrow \textcolor{red}{सुट् तिथोः} (पा॰सू॰~३.४.१०७)~\arrow \textcolor{red}{आद्यन्तौ टकितौ} (पा॰सू॰~१.१.४६)~\arrow भक्षे~अ~यास्~सुँट्~ति~\arrow भक्षे~अ~यास्~स्~ति~\arrow \textcolor{red}{अतो येयः} (पा॰सू॰~७.२.८०)~\arrow भक्षे~अ~इय्~ति~\arrow \textcolor{red}{लोपो व्योर्वलि} (पा॰सू॰~६.१.६६)~\arrow भक्षे~अ~इ~ति~\arrow \textcolor{red}{एचोऽयवायावः} (पा॰सू॰~६.१.७८)~\arrow भक्षय्~अ~इ~ति~\arrow \textcolor{red}{आद्गुणः} (पा॰सू॰~६.१.८७)~\arrow भक्षय्~ए~ति~\arrow \textcolor{red}{इतश्च} (पा॰सू॰~३.४.१००)~\arrow भक्षय्~ए~त्~\arrow भक्षयेत्।}\end{sloppypar}
\label{sec:prasarayat}
\section[प्रसारयत्]{प्रसारयत्}
\centering\textcolor{blue}{दृश्यते नैव कोऽप्यत्र विस्मयो मे प्रजायते।\nopagebreak\\
एवं विचिन्त्य हनूमानधो दृष्टिं प्रसारयत्॥}\nopagebreak\\
\raggedleft{–~अ॰रा॰~५.१.३७}\\
\fontsize{14}{21}\selectfont\begin{sloppypar}\hyphenrules{nohyphenation}\justifying\noindent\hspace{10mm} अत्र \textcolor{red}{विनाऽपि प्रत्ययं पूर्वोत्तर\-पद\-लोपो वक्तव्यः} (वा॰~५.३.८३) इत्यनेनाकार\-लोपः। आगम\-कार्यस्यानित्यत्वाद्वा।\footnote{\textcolor{red}{आगम\-शास्त्रमनित्यम्} (प॰शे॰~९३.२)। अडागमे कृते सामान्यतः \textcolor{red}{प्रासारयत्} इति रूपम्। यथा \textcolor{red}{ततो॒ वै स प्र॒जाना॒न्दक्षि॑णं बा॒हुं प्रासा॑रयत्} (कृ॰य॰ तै॰ब्रा~१.६.४.२) इत्यत्र। \textcolor{red}{सृ गतौ} (धा॰पा॰~९३५)~\arrow \textcolor{red}{हेतुमति च}~\arrow सृ~णिच्~\arrow सृ~इ~\arrow \textcolor{red}{अचो ञ्णिति}~\arrow \textcolor{red}{उरण् रपरः}~\arrow सार्~इ~\arrow सारि~\arrow \textcolor{red}{सनाद्यन्ता धातवः}~\arrow धातुसञ्ज्ञा। प्र~सारि~\arrow \textcolor{red}{शेषात्कर्तरि परस्मैपदम्} (पा॰सू॰~१.३.७८)~\arrow \textcolor{red}{अनद्यतने लङ्} (पा॰सू॰~३.२.१११)~\arrow प्र~सारि~तिप्~\arrow प्र~सारि~ति~\arrow \textcolor{red}{लुङ्लङ्लृङ्क्ष्वडुदात्तः} (पा॰सू॰~६.४.७१)~\arrow \textcolor{red}{आद्यन्तौ टकितौ} (पा॰सू॰~१.१.४६)~\arrow प्र~अट्~सारि~ति~\arrow प्र~अ~सारि~ति~\arrow \textcolor{red}{कर्तरि शप्‌} (पा॰सू॰~३.१.६८)~\arrow प्र~अ~सारि~शप्~ति~\arrow प्र~अ~सारि~अ~ति~\arrow \textcolor{red}{सार्वधातुकार्ध\-धातुकयोः} (पा॰सू॰~७.३.८४)~\arrow प्र~अ~सारे~अ~ति~\arrow प्र~अ~सारय्~अ~ति~\arrow \textcolor{red}{इतश्च} (पा॰सू॰~३.४.१००)~\arrow प्र~अ~सारय्~अ~त्~\arrow \textcolor{red}{अकः सवर्णे दीर्घः} (पा॰सू॰~६.१.१०१)~\arrow प्रा~सारय्~अ~त्~\arrow प्रासारयत्।} प्रमाणं चात्र \textcolor{red}{इको यणचि} (पा॰सू॰~६.१.७७) इत्यत्र ङमुडागमाभावः।\footnote{यद्यागमकार्यं नित्यमभविष्यत्तर्हि \textcolor{red}{इको यणचि} (पा॰सू॰~६.१.७७) इति सूत्रे \textcolor{red}{यण् अचि} इति स्थिते \textcolor{red}{ङमो ह्रस्वादचि ङमुण्नित्यम्} (पा॰सू॰~८.३.३२) इत्यनेन ङमुडागमं कृत्वा \textcolor{red}{इको यण्णचि} इत्येवासूत्रयिष्यन् सूत्रकाराः। अत एव वाल्मीकीय\-रामायणे सुन्दर\-काण्डे नव\-व्याकरणार्थ\-वेत्ता हनुमान् लङि \textcolor{red}{प्रविशम्} इति प्रयुङ्क्ते~– \textcolor{red}{प्रदोषकाले प्रविशं भीतयाऽहं तयोदितः} (वा॰रा॰~५.५८.५०)। अत्र तिलक\-शिरोमणि\-टीका\-कारौ च~– \textcolor{red}{प्रविशं प्राविशम्} (वा॰रा॰ ति॰टी॰~५.५८.५०, वा॰रा॰ शि॰टी॰~५.५८.५०)।}\end{sloppypar}
\section[प्रसीदताम्]{प्रसीदताम्}
\centering\textcolor{blue}{धन्याहमप्यद्य चिराय राघव स्मृतिर्ममासीद्भवपाशमोचिनी।\nopagebreak\\
तद्भक्तसङ्गोऽप्यतिदुर्लभो मम प्रसीदतां दाशरथिः सदा हृदि॥}\nopagebreak\\
\raggedleft{–~अ॰रा॰~५.१.५७}\\
\fontsize{14}{21}\selectfont\begin{sloppypar}\hyphenrules{nohyphenation}\justifying\noindent\hspace{10mm} \textcolor{red}{प्रसीदतु}\footnote{प्र~\textcolor{red}{षद्ऌँ विशरण\-गत्यवसादनेषु} (धा॰पा॰~८५४, १४२७)~\arrow प्र~षद्~\arrow \textcolor{red}{धात्वादेः षः सः} (पा॰सू॰~६.१.६४)~\arrow प्र~सद्~\arrow \textcolor{red}{शेषात्कर्तरि परस्मैपदम्} (पा॰सू॰~१.३.७८)~\arrow \textcolor{red}{लोट् च} (पा॰सू॰~३.३.१६२)~\arrow प्र~सद्~लोट्~\arrow प्र~सद्~तिप्~\arrow प्र~सद्~ति~\arrow \textcolor{red}{कर्तरि शप्‌} (पा॰सू॰~३.१.६८)~\arrow प्र~सद्~शप्~ति~\arrow प्र~सद्~अ~ति~\arrow \textcolor{red}{पाघ्रा\-ध्मास्थाम्ना\-दाण्दृश्यर्त्ति\-सर्त्तिशदसदां पिब\-जिघ्र\-धम\-तिष्ठ\-मन\-यच्छ\-पश्यर्च्छ\-धौ\-शीय\-सीदाः} (पा॰सू॰~७.३.७८)~\arrow प्र~सीद्~अ~ति~\arrow \textcolor{red}{एरुः} (पा॰सू॰~३.४.८६)~\arrow प्र~सीद्~अ~तु~\arrow प्रसीदतु।} इति प्रयोक्तव्ये \textcolor{red}{प्रसीदताम्}\footnote{प्र~\textcolor{red}{षद्ऌँ विशरण\-गत्यवसादनेषु} (धा॰पा॰~८५४, १४२७)~\arrow प्र~षद्~\arrow \textcolor{red}{धात्वादेः षः सः} (पा॰सू॰~६.१.६४)~\arrow प्र~सद्~\arrow \textcolor{red}{कर्तरि कर्म\-व्यतिहारे} (पा॰सू॰~१.३.१४)~\arrow \textcolor{red}{लोट् च} (पा॰सू॰~३.३.१६२)~\arrow प्र~सद्~लोट्~\arrow प्र~सद्~त~\arrow \textcolor{red}{कर्तरि शप्‌} (पा॰सू॰~३.१.६८)~\arrow प्र~सद्~शप्~त~\arrow प्र~सद्~अ~त~\arrow \textcolor{red}{पाघ्रा\-ध्मास्थाम्ना\-दाण्दृश्यर्त्ति\-सर्त्तिशदसदां पिब\-जिघ्र\-धम\-तिष्ठ\-मन\-यच्छ\-पश्यर्च्छ\-धौ\-शीय\-सीदाः} (पा॰सू॰~७.३.७८)~\arrow प्र~सीद्~अ~त~\arrow \textcolor{red}{टित आत्मनेपदानां टेरे} (पा॰सू॰~३.४.७९)~\arrow प्र~सीद्~अ~ते~\arrow \textcolor{red}{आमेतः} (पा॰सू॰~३.४.९०)~\arrow प्र~सीद्~अ~ताम्~\arrow प्रसीदताम्।} इति प्रयोगः।
कर्म\-व्यतिहार आत्मनेपदम्।\footnote{\textcolor{red}{कर्तरि कर्म\-व्यतिहारे} (पा॰सू॰~१.३.१४) इत्यनेन। प्रसन्नताभावस्तु जीवस्यैव धर्मो न ब्रह्मण इति ध्वनयितुं कर्मव्यतिहार आत्मनेपदप्रयोग इति भावः। मानसे गोस्वामि\-पादाश्च~– \textcolor{red}{प्रसन्नतां या न गताऽभिषेकतस्तथा न मम्लौ वनवास\-दुःखतः। मुखाम्बुजश्री रघुनन्दनस्य मे सदाऽस्तु सा मञ्जुलमङ्गलप्रदा॥} (रा॰च॰मा॰~२.म॰२)। एतेन श्रीमद्भागवतेऽष्टम\-स्कन्धे पञ्चमाध्याये ब्रह्मस्तुतौ द्वादशवारं प्रयुक्तं \textcolor{red}{प्रसीदताम्} इत्यपि व्याख्यातम्। यथा~– \textcolor{red}{प्रसीदतां ब्रह्म महाविभूतिः} (भा॰पु॰~८.५.३२) \textcolor{red}{प्रसीदतां नः स महाविभूतिः} (भा॰पु॰~८.५.३३.४३)।}\end{sloppypar}
\section[भोक्ष्यति]{भोक्ष्यति}
\centering\textcolor{blue}{द्विमासाभ्यन्तरे सीता यदि मे वशगा भवेत्।\nopagebreak\\
तदा सर्वसुखोपेता राज्यं भोक्ष्यति सा मया॥}\nopagebreak\\
\raggedleft{–~अ॰रा॰~५.२.४१}\\
\fontsize{14}{21}\selectfont\begin{sloppypar}\hyphenrules{nohyphenation}\justifying\noindent\hspace{10mm} अत्र \textcolor{red}{भुजोऽनवने} (पा॰सू॰~१.३.६६) इत्यनेनाऽत्मनेपदे सति \textcolor{red}{भोक्ष्यते} इति पाणिनीयः।\footnote{भुज्~\arrow \textcolor{red}{भुजोऽनवने} (पा॰सू॰~१.३.७८)~\arrow \textcolor{red}{लृट् शेषे च} (पा॰सू॰~३.३.१३)~\arrow भुज्~लृँट्~\arrow भुज्~त~\arrow \textcolor{red}{स्यतासी लृलुटोः} (पा॰सू॰~३.१.३३)~\arrow भुज्~स्य~त~\arrow \textcolor{red}{एकाच उपदेशेऽनुदात्तात्‌} (पा॰सू॰~७.२.१०)~\arrow इडागम\-निषेधः~\arrow \textcolor{red}{पुगन्त\-लघूपधस्य च} (पा॰सू॰~७.३.८६)~\arrow भोज्~स्य~त~\arrow \textcolor{red}{चोः कुः} (पा॰सू॰~८.२.३०)~\arrow भोग्~स्य~त~\arrow \textcolor{red}{आदेश\-प्रत्यययोः} (पा॰सू॰~८.३.५९)~\arrow भोग्~ष्य~त~\arrow \textcolor{red}{खरि च} (पा॰सू॰~८.४.५५)~\arrow भोक्~ष्य~त~\arrow \textcolor{red}{टित आत्मनेपदानां टेरे} (पा॰सू॰~३.४.७९)~\arrow भोक्~ष्य~ते~\arrow भोक्ष्यते।} किन्तु भोजनं \textcolor{red}{भोजः}।\footnote{\textcolor{red}{भुज्‌}\-धातोः \textcolor{red}{नन्दि\-ग्रहि\-पचादिभ्यो ल्युणिन्यचः} (पा॰सू॰~३.१.१३४) इत्यनेन \textcolor{red}{अच्‌}\-प्रत्यये विभक्ति\-कार्ये। बाहुलकाद्भावेऽच्। तेन \textcolor{red}{भोजः} \textcolor{red}{भोगः} इति द्वावपि समानार्थकौ। यथा भागवते \textcolor{red}{याचिष्णवे भूर्यपि भूरिभोजः} (भा॰पु॰~१०.८१.३४) इत्यत्र। अत्र टीकाकाराः~– \textcolor{red}{भूरिभोजो बहुभोगवान्} (भा॰पु॰ गू॰दी॰~१०.८१.३४, भा॰पु॰ अ॰प्र॰~१०.८१.३४) \textcolor{red}{भूरिभोजो बहुभोगः} (भा॰पु॰ नि॰प्र॰~१०.८१.३४)। \textcolor{red}{भोगः} इति तु \textcolor{red}{भुज्‌}\-धातोः \textcolor{red}{भावे} (पा॰सू॰~३.३.१८) इत्यनेन घञि \textcolor{red}{पुगन्त\-लघूपधस्य च} (पा॰सू॰~७.३.८६) इत्यनेन गुणे \textcolor{red}{चजोः कु घिण्ण्यतोः} (पा॰सू॰~७.३.५२) इत्यनेन कुत्वे विभक्तिकार्ये सिद्धम्।} \textcolor{red}{भोजमाचरतीति भोजति}।\footnote{भोज~\arrow \textcolor{red}{सर्वप्रातिपतिकेभ्य आचारे क्विब्वा वक्तव्यः} (वा॰~३.१.११)~\arrow भोज~क्विँप्~\arrow भोज~व्~\arrow \textcolor{red}{वेरपृक्तस्य} (पा॰सू॰~६.१.६७)~\arrow भोज~\arrow \textcolor{red}{सनाद्यन्ता धातवः} (पा॰सू॰~३.१.३२)~\arrow धातु\-सञ्ज्ञा~\arrow \textcolor{red}{शेषात्कर्तरि परस्मैपदम्} (पा॰सू॰~१.३.७८)~\arrow \textcolor{red}{वर्तमाने लट्} (पा॰सू॰~३.२.१२३)~\arrow भोज~लट्~\arrow भोज~तिप्~\arrow भोज~ति~\arrow \textcolor{red}{कर्तरि शप्‌} (पा॰सू॰~३.१.६८)~\arrow भोज~शप्~ति~\arrow भोज~अ~ति~\arrow\textcolor{red}{अतो गुणे} (पा॰सू॰~६.१.९७)~\arrow भोज~ति~\arrow भोजति।} तस्य लृड्लकारे \textcolor{red}{भोक्ष्यति}।\footnote{\textcolor{red}{आगम\-शास्त्रमनित्यम्} (प॰शे॰~९३.२) इति परिभाषयाऽऽगम\-कार्यस्यानित्यत्वाद्बाहुलकादिडभावः। यद्वा भोजनं \textcolor{red}{भोक्}। णिजन्तात् \textcolor{red}{भुज्‌}\-धातोः (भुज्~णिच्~\arrow भोजि) \textcolor{red}{सम्पदादिभ्‍यः क्विप्} (वा॰~३.३.१०८) इत्यनेन भावे क्विपि \textcolor{red}{णेरनिटि} (पा॰सू॰~६.४.५१) इत्यनेन णिलोपे \textcolor{red}{भोज्} इति प्रातिपदिकं निष्पन्नम्। तस्य प्रथमाविभक्तावेकवचने \textcolor{red}{भोक्} \textcolor{red}{भोग्} इति रूपद्वयम्। भोज्~सुँ~\arrow \textcolor{red}{हल्ङ्याब्भ्यो दीर्घात्सुतिस्यपृक्तं हल्} (पा॰सू॰~६.१.६८)~\arrow भोज्~\arrow \textcolor{red}{चोः कुः} (पा॰सू॰~८.२.३०)~\arrow भोग्~\arrow \textcolor{red}{वाऽवसाने} (पा॰सू॰~८.४.५६)~\arrow भोक्, भोग्। यथर्ग्वेद\-संहितायाम् \textcolor{red}{युवा॑नो रु॒द्रा अ॒जरा॑ अभो॒ग्घनो॑} (ऋ॰वे॰सं॰~१.६४.३) इति मन्त्रे पदपाठे \textcolor{red}{अ॒भो॒क्ऽहन॑} सायण\-भाष्ये \textcolor{red}{अभोग्घनः। भोजयन्तीति भोजः। न भोजः अभोजः। तेषां हन्तारः। ‘बहुलं छन्दसि’ (पा॰सू॰~३.२.८८) इति हन्तेः क्विप्। ‘झयो होऽन्यतरस्याम्’ (पा॰सू॰~८.४.६२) इति हकारस्य घत्वम्। ‘इन्हन्पूषार्यम्णां शौ’ (पा॰सू॰~६.४.१२) इति नियमाद्दीर्घाभावः}। तदाचरिष्यति भोक्ष्यति। नामधातोरनु\-दात्तैकाच्त्वात् \textcolor{red}{एकाच उपदेशेऽनुदात्तात्‌} (पा॰सू॰~७.२.१०) इति सूत्रेणेडागम\-निषेधः।} यद्वाऽत्रावनार्थ\-\textcolor{red}{भुज्‌}\-धातुः।\footnote{\textcolor{red}{भुजँ पालनाभ्यवहारयोः} (धा॰पा॰~१४५४)। भुज्~\arrow \textcolor{red}{शेषात्कर्तरि परस्मैपदम्} (पा॰सू॰~१.३.७८)~\arrow \textcolor{red}{लृट् शेषे च} (पा॰सू॰~३.३.१३)~\arrow भुज्~लृँट्~\arrow भुज्~तिप्~\arrow भुज्~ति~\arrow \textcolor{red}{स्यतासी लृलुटोः} (पा॰सू॰~३.१.३३)~\arrow भुज्~स्य~ति~\arrow \textcolor{red}{एकाच उपदेशेऽनुदात्तात्‌} (पा॰सू॰~७.२.१०)~\arrow इडागम\-निषेधः~\arrow \textcolor{red}{पुगन्त\-लघूपधस्य च} (पा॰सू॰~७.३.८६)~\arrow भोज्~स्य~ति~\arrow \textcolor{red}{चोः कुः} (पा॰सू॰~८.२.३०) भोग्~स्य~ति~\arrow \textcolor{red}{आदेश\-प्रत्यययोः} (पा॰सू॰~८.३.५९)~\arrow भोग्~ष्य~ति~\arrow \textcolor{red}{खरि च} (पा॰सू॰~८.४.५५)~\arrow भोक्~ष्य~ति~\arrow भोक्ष्यति।} रावणस्याभिप्रायोऽयम् \textcolor{red}{मद्वशगा सती सीता राज्यं भोक्ष्यत्यधिष्ठात्री भूत्वा तत्प्रतिपालयिष्यति}।\end{sloppypar}
\section[अगाहत्]{अगाहत्}
\centering\textcolor{blue}{अगाहत्पुत्रपौत्रैश्च कृत्वा वदनमालिकाम्।\nopagebreak\\
विभीषणस्तु रामस्य सन्निधौ हृष्टमानसः॥}\nopagebreak\\
\raggedleft{–~अ॰रा॰~५.२.५२}\\
\fontsize{14}{21}\selectfont\begin{sloppypar}\hyphenrules{nohyphenation}\justifying\noindent\hspace{10mm} आत्मनेपदस्यानित्यत्वात्प्रयोगोऽयं परस्मैपदी।\footnote{\textcolor{red}{अनुदात्तेत्त्व\-लक्षणमात्मने\-पदमनित्यम्} (प॰शे॰~९३.४) इत्यनेन। \textcolor{red}{गाह्‌}\-धातुः (\textcolor{red}{गाहूँ विलोडने} धा॰पा॰~६४९) आत्मने\-पदी। तस्य लङ्लकारे प्रथमपुरुष एकवचने \textcolor{red}{अगाहत} इति रूपम्। \textcolor{red}{गाहूँ विलोडने} (धा॰पा॰~६४९)~\arrow गाह्~\arrow \textcolor{red}{अनुदात्तङित आत्मने\-पदम्} (पा॰सू॰~१.३.१२)~\arrow \textcolor{red}{अनद्यतने लङ्} (पा॰सू॰~३.२.१११)~\arrow गाह्~लङ्~\arrow गाह्~त~\arrow \textcolor{red}{लुङ्लङ्लृङ्क्ष्वडुदात्तः} (पा॰सू॰~६.४.७१)~\arrow अट्~गाह्~त~\arrow अ~गाह्~त~\arrow \textcolor{red}{कर्तरि शप्‌} (पा॰सू॰~३.१.६८)~\arrow अ~गाह्~शप्~त~\arrow अ~गाह्~अ~त~\arrow अगाहत। स्वीकृत आत्मने\-पदस्यानित्यत्वे परस्मैपदे \textcolor{red}{अगाहत्} इति रूपम्। \textcolor{red}{गाहूँ विलोडने} (धा॰पा॰~६४९)~\arrow गाह्~\arrow \textcolor{red}{अनुदात्तेत्त्व\-लक्षणमात्मने\-पदमनित्यम्} (प॰शे॰~९३.४)~\arrow \textcolor{red}{शेषात्कर्तरि परस्मैपदम्} (पा॰सू॰~१.३.७८)~\arrow \textcolor{red}{अनद्यतने लङ्} (पा॰सू॰~३.२.१११)~\arrow गाह्~लङ्~\arrow गाह्~तिप्~\arrow गाह्~ति~\arrow \textcolor{red}{लुङ्लङ्लृङ्क्ष्वडुदात्तः} (पा॰सू॰~६.४.७१)~\arrow अट्~गाह्~ति~\arrow अ~गाह्~ति~\arrow \textcolor{red}{कर्तरि शप्‌} (पा॰सू॰~३.१.६८)~\arrow अ~गाह्~शप्~ति~\arrow अ~गाह्~अ~ति~\arrow \textcolor{red}{इतश्च} (पा॰सू॰~३.४.१००)~\arrow अगाहत्। एतेन \textcolor{red}{यूयं विवस्त्रा यदपो धृतव्रता व्यगाहतैतत्तदु देवहेलनम्} (भा॰पु॰~१०.२२.१८) इति भागवते रासपञ्चाध्याय्यां गोपीवस्त्रापहारे मध्यमपुरुष\-बहुवचन\-विवक्षायां कृतः \textcolor{red}{व्यगाहत} इति परस्मैपदप्रयोगोऽपि व्याख्यातः।}\end{sloppypar}
\section[निर्दहिष्यति]{निर्दहिष्यति}
\centering\textcolor{blue}{निर्दहिष्यति रक्षौघांस्त्वत्कृते नात्र संशयः।\nopagebreak\\
अनुज्ञां देहि मे देवि गच्छामि त्वरयाऽन्वितः॥}\nopagebreak\\
\raggedleft{–~अ॰रा॰~५.३.४९}\\
\fontsize{14}{21}\selectfont\begin{sloppypar}\hyphenrules{nohyphenation}\justifying\noindent\hspace{10mm} \textcolor{red}{निर्धक्ष्यति}\footnote{निस्~\textcolor{red}{दहँ भस्मीकरणे} (धा॰पा॰~९९१)~\arrow निस्~दह्~\arrow \textcolor{red}{शेषात्कर्तरि परस्मैपदम्} (पा॰सू॰~१.३.७८)~\arrow \textcolor{red}{लृट् शेषे च} (पा॰सू॰~३.३.१३)~\arrow निस्~दह्~लृट्~\arrow निस्~दह्~तिप्~\arrow निस्~दह्~ति~\arrow \textcolor{red}{स्यतासी लृलुटोः} (पा॰सू॰~३.१.३३)~\arrow निस्~दह्~स्य~ति~\arrow \textcolor{red}{दादेर्धातोर्घः} (पा॰सू॰~८.२.३२)~\arrow निस्~दघ्~स्य~ति~\arrow \textcolor{red}{एकाचो बशो भष् झषन्तस्य स्ध्वोः} (पा॰सू॰~८.२.३७)~\arrow निस्~धघ्~स्य~ति~\arrow \textcolor{red}{खरि च} (पा॰सू॰~८.४.५५)~\arrow निस्~धक्~स्य~ति~\arrow \textcolor{red}{आदेश\-प्रत्यययोः} (पा॰सू॰~८.३.५९)~\arrow निस्~धक्~ष्य~ति~\arrow ससजुषो रुः~\arrow (पा॰सू॰~८.२.६६)~\arrow निरुँ~धक्~ष्य~ति~\arrow निर्~धक्~ष्य~ति~\arrow निर्धक्ष्यति।} इति प्रयोक्तव्ये \textcolor{red}{निर्दहिष्यति} इति प्रयुक्तम्। \textcolor{red}{निर्दहतीति निर्दहः}।\footnote{\textcolor{red}{नन्दि\-ग्रहि\-पचादिभ्यो ल्युणिन्यचः} (पा॰सू॰~३.१.१३४) इत्यनेन।} \textcolor{red}{निर्दह इवाऽचरिष्यतीति निर्दहिष्यति}\footnote{प्रयोगस्यास्य सिद्धिः \textcolor{red}{कृष्णिष्यति} (बा॰म॰~२६६५) इतिवत्। निर्दह~\arrow \textcolor{red}{सर्वप्राति\-पदिकेभ्य आचारे क्विब्वा वक्तव्यः} (वा॰~३.१.११)~\arrow निर्दह~क्विँप्~\arrow निर्दह~व्~\arrow \textcolor{red}{वेरपृक्तस्य} (पा॰सू॰~६.१.६७)~\arrow निर्दह~\arrow \textcolor{red}{सनाद्यन्ता धातवः} (पा॰सू॰~३.१.३२)~\arrow \textcolor{red}{शेषात्कर्तरि परस्मैपदम्} (पा॰सू॰~१.३.७८)~\arrow \textcolor{red}{लृट् शेषे च} (पा॰सू॰~३.३.१३)~\arrow निर्दह~लृट्~\arrow निर्दह~तिप्~\arrow निर्दह~ति~\arrow \textcolor{red}{स्यतासी लृलुटोः} (पा॰सू॰~३.१.३३)~\arrow निर्दह~स्य~ति~\arrow \textcolor{red}{आर्धधातुकस्येड्वलादेः} (पा॰सू॰~७.२.३५)~\arrow निर्दह~इट्~स्य~ति~\arrow निर्दह~इ~स्य~ति~\arrow \textcolor{red}{अतो लोपः} (पा॰सू॰~६.४.४८)~\arrow निर्दह्~इ~स्य~ति~\arrow \textcolor{red}{आदेश\-प्रत्ययोः} (पा॰सू॰~८.३.५९)~\arrow निर्दह्~इ~ष्य~ति~\arrow निर्दहिष्यति।} इति प्रयोगे परिहारः।\end{sloppypar}
\section[निद्राति]{निद्राति}
\centering\textcolor{blue}{अभिज्ञानार्थमन्यच्च वदामि तव सुव्रत।\nopagebreak\\
चित्रकूटगिरौ पूर्वमेकदा रहसि स्थितः।\nopagebreak\\
मदङ्के शिर आधाय निद्राति रघुनन्दनः॥}\nopagebreak\\
\raggedleft{–~अ॰रा॰~५.३.५३}\\
\fontsize{14}{21}\selectfont\begin{sloppypar}\hyphenrules{nohyphenation}\justifying\noindent\hspace{10mm} इह \textcolor{red}{स्म} इति योजनीयम्। एवं \textcolor{red}{लट् स्मे} (पा॰सू॰~३.२.११८) इत्यनेन भूत\-काले लड्लकारः।\footnote{निपूर्वकात् \textcolor{red}{द्रा}\-धातोः (\textcolor{red}{द्रा कुत्सायां गतौ} धा॰पा॰~१०५४) लड्लकारे प्रथमपुरुष एकवचने \textcolor{red}{निद्राति}। नि~द्रा~\arrow \textcolor{red}{शेषात्कर्तरि परस्मैपदम्} (पा॰सू॰~१.३.७८)~\arrow \textcolor{red}{वर्तमाने लट्} (पा॰सू॰~३.२.१२३)~\arrow नि~द्रा~लट्~\arrow नि~द्रा~तिप्~\arrow नि~द्रा~ति~\arrow \textcolor{red}{कर्तरि शप्} (पा॰सू॰~३.१.६८)~\arrow नि~द्रा~शप्~ति~\arrow \textcolor{red}{अदिप्रभृतिभ्यः शपः} (पा॰सू॰~२.४.७२)~\arrow नि~द्रा~ति~\arrow निद्राति।}\end{sloppypar}
\section[वदस्व]{वदस्व}
\centering\textcolor{blue}{ततः प्रहस्तो हनुमन्तमादरात्पप्रच्छ केन प्रहितोऽसि वानर।\nopagebreak\\
भयं च ते माऽस्तु विमोक्ष्यसे मया सत्यं वदस्वाखिलराजसन्निधौ॥}\nopagebreak\\
\raggedleft{–~अ॰रा॰~५.४.६}\\
\fontsize{14}{21}\selectfont\begin{sloppypar}\hyphenrules{nohyphenation}\justifying\noindent\hspace{10mm} अत्र \textcolor{red}{भासनोपसम्भाषा\-ज्ञान\-यत्न\-विमत्युपमन्त्रणेषु वदः} (पा॰सू॰~१.३.४७) इत्यनेनोपसम्भाषायां ज्ञाने वाऽऽत्मनेपदम्।\footnote{\textcolor{red}{वदँ व्यक्तायां वाचि} (धा॰पा॰~१००९)~\arrow वद्~\arrow \textcolor{red}{भासनोपसम्भाषा\-ज्ञान\-यत्न\-विमत्युपमन्त्रणेषु वदः} (पा॰सू॰~१.३.४७)~\arrow \textcolor{red}{लोट् च} (पा॰सू॰~३.३.१६२)~\arrow वद्~लोट्~\arrow वद्~थास्~\arrow \textcolor{red}{कर्तरि शप्} (पा॰सू॰~३.१.६८)~\arrow वद्~शप्~थास्~\arrow वद्~अ~थास्~\arrow \textcolor{red}{थासस्से} (पा॰सू॰~३.४.८०)~\arrow वद्~अ~से~\arrow \textcolor{red}{सवाभ्यां वामौ} (पा॰सू॰~३.४.९१)~\arrow वद्~अ~स्~व~\arrow वदस्व।}\end{sloppypar}
\section[पश्यध्वम्]{पश्यध्वम्}
\centering\textcolor{blue}{शब्देनैव विजानीमः कृतकार्यः समागतः।\nopagebreak\\
हनूमानेव पश्यध्वं वानरा वानरर्षभम्॥}\nopagebreak\\
\raggedleft{–~अ॰रा॰~५.५.१३}\\
\fontsize{14}{21}\selectfont\begin{sloppypar}\hyphenrules{nohyphenation}\justifying\noindent\hspace{10mm} अत्र \textcolor{red}{कर्तरि कर्म\-व्यतिहारे} (पा॰सू॰~१.३.१४) इत्यनेन क्रिया\-विनिमय आत्मनेपदम्।\footnote{परस्मै\-पदिनः \textcolor{red}{दृश्‌}\-धातोः (\textcolor{red}{दृशिँर् प्रेक्षणे} धा॰पा॰~९८८) लोड्लकारे मध्यम\-पुरुषे बहुवचने \textcolor{red}{पश्यत} इति रूपम्। दृश्~\arrow \textcolor{red}{शेषात्कर्तरि परस्मैपदम्} (पा॰सू॰~१.३.७८)~\arrow \textcolor{red}{लोट् च} (पा॰सू॰~३.३.१६२)~\arrow दृश्~लोट्~\arrow दृश्~थ~\arrow \textcolor{red}{कर्तरि शप्} (पा॰सू॰~३.१.६८)~\arrow दृश्~शप्~थ~\arrow दृश्~अ~थ~\arrow \textcolor{red}{पाघ्रा\-ध्मास्थाम्ना\-दाण्दृश्यर्त्ति\-सर्त्तिशदसदां पिब\-जिघ्र\-धम\-तिष्ठ\-मन\-यच्छ\-पश्यर्च्छ\-धौ\-शीय\-सीदाः} (पा॰सू॰~७.३.७८)~\arrow पश्य्~अ~थ~\arrow \textcolor{red}{लोटो लङ्वत्‌} (पा॰सू॰~३.४.८५)~\arrow ङिद्वत्त्वम्~\arrow \textcolor{red}{तस्थस्थमिपां तान्तन्तामः} (पा॰सू॰~३.४.१०१)~\arrow पश्य्~अ~त~\arrow पश्यत। आत्मनेपदे च \textcolor{red}{पश्यध्वम्} इति। दृश्~\arrow \textcolor{red}{कर्तरि कर्मव्यतिहारे} (पा॰सू॰~१.३.१४)~\arrow \textcolor{red}{लोट् च} (पा॰सू॰~३.३.१६२)~\arrow दृश्~लोट्~\arrow दृश्~ध्वम्~\arrow \textcolor{red}{कर्तरि शप्} (पा॰सू॰~३.१.६८)~\arrow दृश्~शप्~ध्वम्~\arrow दृश्~अ~ध्वम्~\arrow \textcolor{red}{पाघ्रा\-ध्मास्थाम्ना\-दाण्दृश्यर्त्ति\-सर्त्तिशदसदां पिब\-जिघ्र\-धम\-तिष्ठ\-मन\-यच्छ\-पश्यर्च्छ\-धौ\-शीय\-सीदाः} (पा॰सू॰~७.३.७८)~\arrow पश्य्~अ~ध्वम्~\arrow पश्यध्वम्।}\end{sloppypar}
\vspace{2mm}
\centering ॥ इति सुन्दरकाण्डीयप्रयोगाणां विमर्शः ॥\nopagebreak\\
\vspace{4mm}
\pdfbookmark[2]{युद्धकाण्डम्}{Chap3Part2Kanda6}
\phantomsection
\addtocontents{toc}{\protect\setcounter{tocdepth}{2}}
\addcontentsline{toc}{subsection}{युद्धकाण्डीयप्रयोगाणां विमर्शः}
\addtocontents{toc}{\protect\setcounter{tocdepth}{0}}
\centering ॥ अथ युद्धकाण्डीयप्रयोगाणां विमर्शः ॥\nopagebreak\\
\section[निवसस्व]{निवसस्व}
\centering\textcolor{blue}{भुङ्क्ष्व चेमानि पक्वानि फलानि तदनन्तरम्।\nopagebreak\\
निवसस्व सुखेनात्र निद्रामेहि त्वराऽस्तु मा॥}\nopagebreak\\
\raggedleft{–~अ॰रा॰~६.७.१६}\\
\fontsize{14}{21}\selectfont\begin{sloppypar}\hyphenrules{nohyphenation}\justifying\noindent\hspace{10mm} अत्र \textcolor{red}{कर्तरि कर्म\-व्यतिहारे} (पा॰सू॰~१.३.१४) इत्यनेनाऽत्मनेपदम्।\footnote{\textcolor{red}{नि}पूर्वकात्परस्मै\-पदिनो \textcolor{red}{वस्‌}\-धातोः (\textcolor{red}{वसँ निवासे} धा॰पा॰~१००५) लोड्लकारे मध्यम\-पुरुष एकवचने \textcolor{red}{निवस} इति रूपम्। नि~वस्~\arrow \textcolor{red}{शेषात्कर्तरि परस्मैपदम्} (पा॰सू॰~१.३.७८)~\arrow \textcolor{red}{लोट् च} (पा॰सू॰~३.३.१६२)~\arrow नि~वस्~लोट्~\arrow नि~वस्~सिप्~\arrow नि~वस्~सि~\arrow \textcolor{red}{कर्तरि शप्} (पा॰सू॰~३.१.६८)~\arrow नि~वस्~शप्~सि~\arrow नि~वस्~अ~सि~\arrow \textcolor{red}{सेर्ह्यपिच्च} (पा॰सू॰~३.४.८७)~\arrow नि~वस्~अ~हि~\arrow \textcolor{red}{अतो हेः} (पा॰सू॰~६.४.१०५)~\arrow नि~वस्~अ~\arrow निवस। आत्मनेपदे च \textcolor{red}{निवसस्व} इति। नि~वस्~\arrow \textcolor{red}{कर्तरि कर्मव्यतिहारे} (पा॰सू॰~१.३.१४)~\arrow \textcolor{red}{लोट् च} (पा॰सू॰~३.३.१६२)~\arrow नि~वस्~लोट्~\arrow नि~वस्~थास्~\arrow \textcolor{red}{कर्तरि शप्} (पा॰सू॰~३.१.६८)~\arrow नि~वस्~शप्~थास्~\arrow नि~वस्~अ~थास्~\arrow \textcolor{red}{थासस्से} (पा॰सू॰~३.४.८०)~\arrow नि~वस्~अ~से~\arrow \textcolor{red}{सवाभ्यां वामौ} (पा॰सू॰~३.४.९१)~\arrow नि~वस्~अ~स्~व~\arrow निवसस्व। अपि च~– \textcolor{red}{त्वराऽस्तु मा} इति कथम् \textcolor{red}{माङि लुङ्} (पा॰सू॰~३.३.१७५) इत्यनेन सर्व\-लकारापवादत्वेन लुङो विधानात्। अस्य समाधानं दीक्षितैर्बाल\-मनोरमायामुक्तम्~– \textcolor{red}{‘माऽस्तु’ इत्यादौ तु ‘मा’ इत्यव्ययान्तरं प्रतिषेधार्थकमित्याहुः। ‘आङ्माङोश्च’ (पा॰सू॰~६.१.७४) इति सूत्रभाष्ये तु ङितो माशब्दस्य निर्देशात्प्रमाच्छन्द इत्यत्र तु न भवतीत्युक्तम्। ‘मा’\-शब्दस्याव्ययान्तरस्य सत्त्वे तु तदेवोदाह्रियेत। ‘माऽस्तु’ इत्यत्र तु ‘अस्तु’ इति विभक्ति\-प्रतिरूपकमव्ययमित्यन्ये} (बा॰म॰~२२१९)।}\end{sloppypar}
\section[निबोध]{निबोध}
\centering\textcolor{blue}{तमाह रावणो राजा भ्रातरं दीनया गिरा।\nopagebreak\\
कुम्भकर्ण निबोध त्वं महत्कष्टमुपस्थितम्॥}\nopagebreak\\
\raggedleft{–~अ॰रा॰~६.७.५१}\\
\fontsize{14}{21}\selectfont\begin{sloppypar}\hyphenrules{nohyphenation}\justifying\noindent\hspace{10mm} अत्र \textcolor{red}{नि}\-पूर्वकोऽवगमार्थको \textcolor{red}{बुध्‌}\-धातुः (\textcolor{red}{बुधँ अवगमने} धा॰पा॰~११७२)। तत्राऽत्मनेपदत्वाल्लोड्\-लकारे मध्यम\-पुरुषैक\-वचने \textcolor{red}{निबुध्यस्व} इति पाणिनीयम्।\footnote{नि~बुध्~\arrow \textcolor{red}{अनुदात्तङित आत्मनेपदम्} (पा॰सू॰~१.३.१२)~\arrow \textcolor{red}{लोट् च} (पा॰सू॰~३.३.१६२)~\arrow नि~बुध्~लोट्~\arrow नि~बुध्~थास्~\arrow \textcolor{red}{दिवादिभ्यः श्यन्} (पा॰सू॰~३.१.६९)~\arrow नि~बुध्~श्यन्~थास्~\arrow नि~बुध्~य~थास्~\arrow \textcolor{red}{थासस्से} (पा॰सू॰~३.४.८०)~\arrow नि~बुध्~य~से~\arrow \textcolor{red}{सवाभ्यां वामौ} (पा॰सू॰~३.४.९१)~\arrow नि~बुध्~य~स्~व~\arrow निबुध्यस्व।} किन्तु \textcolor{red}{बोधनं बोधः} भावे घञि\footnote{\textcolor{red}{भावे} (पा॰सू॰~३.३.१८) इत्यनेन।} ततश्च गुणे।\footnote{\textcolor{red}{पुगन्त\-लघूपधस्य च} (पा॰सू॰~७.३.८६) इत्यनेन।} \textcolor{red}{नितरां बोधो निबोधः}। \textcolor{red}{निबोधमाचर} इति \textcolor{red}{निबोध} इत्थमाचार\-क्विबन्ताद्धातोर्लोट्। मध्यम\-पुरुष एकवचने \textcolor{red}{सेर्ह्यपिच्च} (पा॰सू॰~३.४.८७) इत्थं \textcolor{red}{हि} आदेशे \textcolor{red}{अतो हेः} (पा॰सू॰~६.४.१०५) इत्यनेन हेर्लुकि \textcolor{red}{निबोध} इत्यपि पाणिनीयम्।\footnote{निबोध~\arrow \textcolor{red}{सर्वप्राति\-पदिकेभ्य आचारे क्विब्वा वक्तव्यः} (वा॰~३.१.११)~\arrow निबोध~क्विँप्~\arrow निबोध~व्~\arrow \textcolor{red}{वेरपृक्तस्य} (पा॰सू॰~६.१.६७)~\arrow निबोध~\arrow \textcolor{red}{सनाद्यन्ता धातवः} (पा॰सू॰~३.१.३२)~\arrow धातुसञ्ज्ञा~\arrow \textcolor{red}{लोट् च} (पा॰सू॰~३.३.१६२)~\arrow निबोध~लोट्~\arrow निबोध~सिप्~\arrow निबोध~सि~\arrow \textcolor{red}{कर्तरि शप्} (पा॰सू॰~३.१.६८)~\arrow निबोध~शप्~सि~\arrow निबोध~अ~सि~\arrow \textcolor{red}{अतो गुणे} (पा॰सू॰~६.१.९७)~\arrow निबोध~सि~\arrow \textcolor{red}{सेर्ह्यपिच्च} (पा॰सू॰~३.४.८७)~\arrow निबोध~हि~\arrow \textcolor{red}{अतो हेः}~\arrow निबोध।} यद्वाऽत्र \textcolor{red}{बुध्‌}\-धातुर्भ्वादि\-परस्मैपदी (\textcolor{red}{बुधँ अवगमने} धा॰पा॰~८५८)। तस्य लोड्लकारे मध्यमपुरुष एकवचन\-रूपमिदम् \textcolor{red}{निबोध}।\footnote{नि~बुध्~\arrow \textcolor{red}{शेषात्कर्तरि परस्मैपदम्} (पा॰सू॰~१.३.७८)~\arrow \textcolor{red}{लोट् च} (पा॰सू॰~३.३.१६२)~\arrow नि~बुध्~लोट्~\arrow नि~बुध्~सिप्~\arrow नि~बुध्~सि~\arrow \textcolor{red}{कर्तरि शप्} (पा॰सू॰~३.१.६८)~\arrow नि~बुध्~शप्~सि~\arrow नि~बुध्~अ~सि~\arrow \textcolor{red}{पुगन्त\-लघूपधस्य च} (पा॰सू॰~७.३.८६)~\arrow नि~बोध्~अ~सि~\arrow \textcolor{red}{सेर्ह्यपिच्च} (पा॰सू॰~३.४.८७)~\arrow नि~बोध्~अ~हि~\arrow \textcolor{red}{अतो हेः} (पा॰सू॰~६.४.१०५)~\arrow नि~बोध्~अ~\arrow निबोध।} एतेन \textcolor{red}{तान्निबोध द्विजोत्तम} (भ॰गी॰~१.७) इति गीता\-वचनमपि व्याख्यातम्।\end{sloppypar}
\section[न्यहनन्]{न्यहनन्}
\centering\textcolor{blue}{दशकोट्यः प्लवङ्गानां गत्वा मन्दिररक्षकान्।\nopagebreak\\
चूर्णयामासुरश्वांश्च गजांश्च न्यहनन् क्षणात्॥}\nopagebreak\\
\raggedleft{–~अ॰रा॰~६.१०.१७}\\
\fontsize{14}{21}\selectfont\begin{sloppypar}\hyphenrules{nohyphenation}\justifying\noindent\hspace{10mm} \textcolor{red}{न्यघ्नन्}\footnote{नि~\textcolor{red}{हनँ हिंसागत्योः} (धा॰पा॰~१०१२)~\arrow नि~हन्~\arrow \textcolor{red}{शेषात्कर्तरि परस्मैपदम्} (पा॰सू॰~१.३.७८)~\arrow \textcolor{red}{अनद्यतने लङ्} (पा॰सू॰~३.२.१११)~\arrow नि~हन्~लङ्~\arrow \textcolor{red}{लुङ्लङ्लृङ्क्ष्वडुदात्तः} (पा॰सू॰~६.४.७१)~\arrow \textcolor{red}{आद्यन्तौ टकितौ} (पा॰सू॰~१.१.४६)~\arrow नि~अट्~हन्~लङ्~\arrow नि~अ~हन्~लङ्~\arrow नि~अ~हन्~झि~\arrow \textcolor{red}{तिङ्शित्सार्व\-धातुकम्} (पा॰सू॰~३.४.११३)~\arrow \textcolor{red}{कर्तरि शप्} (पा॰सू॰~३.१.६८)~\arrow \textcolor{red}{अदि\-प्रभृतिभ्यः शपः} (पा॰सू॰~२.४.७२)~\arrow नि~अ~हन्~झि~\arrow \textcolor{red}{सार्वधातुकमपित्} (पा॰सू॰~१.२.४)~\arrow ङित्त्वम्~\arrow \textcolor{red}{गम\-हन\-जन\-खन\-घसां लोपः क्ङित्यनङि} (पा॰सू॰~६.४.९८)~\arrow नि~अ~ह्~न्~झि~\arrow \textcolor{red}{हो हन्तेर्ञ्णिन्नेषु} (पा॰सू॰~७.३.५४)~\arrow नि~अ~घ्~न्~झि~\arrow \textcolor{red}{झोऽन्तः} (पा॰सू॰~७.१.३)~\arrow नि~अ~घ्~न्~अन्ति~\arrow \textcolor{red}{इतश्च} (पा॰सू॰~३.४.१००)~\arrow नि~अ~घ्~न्~अन्त्~\arrow \textcolor{red}{संयोगान्तस्य लोपः} (पा॰सू॰~८.२.२३)~\arrow नि~अ~घ्~न्~अन्~\arrow नि~अघ्नन्~\arrow \textcolor{red}{इको यणचि} (पा॰सू॰~६.१.७७)~\arrow न्यघ्नन्।} इति प्रयोक्तव्ये \textcolor{red}{न्यहनन्} इति प्रयोगस्तु \textcolor{red}{गण\-कार्यमनित्यम्} (प॰शे॰~९३.३) इति नियमात् \textcolor{red}{नि}\-पूर्वकात् \textcolor{red}{हन्‌}\-धातोः (\textcolor{red}{हनँ हिंसागत्योः} धा॰पा॰~१०१२) लङि \textcolor{red}{झि}\-प्रत्ययेऽडागमे शपि \textcolor{red}{झोऽन्तः} (पा॰सू॰~७.१.३) इत्यनेनान्तादेशे \textcolor{red}{अतो गुणे} (पा॰सू॰~६.१.९७) इत्यनेन पर\-रूपे \textcolor{red}{इतश्च} (पा॰सू॰~३.४.१००) इत्यनेनेकार\-लोपे \textcolor{red}{संयोगान्तस्य लोपः} (पा॰सू॰~८.२.२३) इत्यनेन तकार\-लोपे \textcolor{red}{न्यहनन्}।\footnote{\textcolor{red}{गण\-कार्यमनित्यम्} (प॰शे॰~९३.३) इति नियमादत्र \textcolor{red}{अदि\-प्रभृतिभ्यः शपः} (पा॰सू॰~२.४.७२) इति सूत्रं न प्रवर्तते। शपः पित्वात्त् \textcolor{red}{सार्वधातुकमपित्} (पा॰सू॰~१.२.४) इत्यस्याप्रवृत्तौ शपो ङित्त्वं न। अङिति शपि परे \textcolor{red}{गम\-हन\-जन\-खन\-घसां लोपः क्ङित्यनङि} (पा॰सू॰~६.४.९८) इति सूत्रं न प्रवर्तते यतो हनोऽकार\-लोपाभावः। अकार\-लोपाभावे \textcolor{red}{हो हन्तेर्ञ्णिन्नेषु} (पा॰सू॰~७.३.५४) इत्यस्यापि प्रवृत्तिर्न। अतः कुत्वाभावः। नि~\textcolor{red}{हनँ हिंसागत्योः} (धा॰पा॰~१०१२)~\arrow नि~हन्~\arrow \textcolor{red}{शेषात्कर्तरि परस्मैपदम्} (पा॰सू॰~१.३.७८)~\arrow \textcolor{red}{अनद्यतने लङ्} (पा॰सू॰~३.२.१११)~\arrow नि~हन्~लङ्~\arrow \textcolor{red}{लुङ्लङ्लृङ्क्ष्वडुदात्तः} (पा॰सू॰~६.४.७१)~\arrow \textcolor{red}{आद्यन्तौ टकितौ} (पा॰सू॰~१.१.४६)~\arrow नि~अट्~हन्~लङ्~\arrow नि~अ~हन्~लङ्~\arrow नि~अ~हन्~झि~\arrow \textcolor{red}{तिङ्शित्सार्व\-धातुकम्} (पा॰सू॰~३.४.११३)~\arrow \textcolor{red}{कर्तरि शप्} (पा॰सू॰~३.१.६८)~\arrow \textcolor{red}{गण\-कार्यमनित्यम्} (प॰शे॰~९३.३)~\arrow शब्लुगभावः~\arrow नि~अ~हन्~अ~झि~\arrow \textcolor{red}{झोऽन्तः} (पा॰सू॰~७.१.३)~\arrow नि~अ~हन्~अ~अन्ति~\arrow \textcolor{red}{इतश्च} (पा॰सू॰~३.४.१००)~\arrow नि~अ~हन्~अ~अन्त्~\arrow \textcolor{red}{अतो गुणे} (पा॰सू॰~६.१.९७)~\arrow नि~अ~हन्~अन्त्~\arrow \textcolor{red}{संयोगान्तस्य लोपः} (पा॰सू॰~८.२.२३)~\arrow नि~अ~हन्~अन्~\arrow नि~अहनन्~\arrow \textcolor{red}{इको यणचि} (पा॰सू॰~६.१.७७)~\arrow न्यहनन्।}\end{sloppypar}
\section[योत्स्यामि]{योत्स्यामि}
\centering\textcolor{blue}{घातयित्वा राघवेण जीवामि वनगोचरः।\nopagebreak\\
रामेण सह योत्स्यामि रामबाणैः सुशीघ्रगैः॥}\nopagebreak\\
\raggedleft{–~अ॰रा॰~६.१०.५६}\\
\fontsize{14}{21}\selectfont\begin{sloppypar}\hyphenrules{nohyphenation}\justifying\noindent\hspace{10mm} आत्मने\-पदस्यानित्यत्वात्प्रयोगोऽयम्।\footnote{\textcolor{red}{अनुदात्तेत्त्व\-लक्षणमात्मने\-पदमनित्यम्} (प॰शे॰~९३.४)। आत्मनेपदे तु \textcolor{red}{युधँ सम्प्रहारे} (धा॰पा॰~११७३) इति धातोर्लृट्युत्तम\-पुरुष एकवचने \textcolor{red}{योत्स्ये} इति रूपम्। यथा गीतायाम्~– \textcolor{red}{न योत्स्य इति गोविन्दमुक्त्वा तूष्णीं बभूव ह} (भ॰गी॰~२.९)। युध्~\arrow \textcolor{red}{अनुदात्तङित आत्मनेपदम्} (पा॰सू॰~१.३.१२)~\arrow \textcolor{red}{लृट् शेषे च} (पा॰सू॰~३.३.१३)~\arrow युध्~लृँट्~\arrow युध्~इट्~\arrow युध्~इ~\arrow \textcolor{red}{स्यतासी लृलुटोः} (पा॰सू॰~३.१.३३)~\arrow युध्~स्य~इ~\arrow \textcolor{red}{एकाच उपदेशेऽनुदात्तात्‌} (पा॰सू॰~७.२.१०)~\arrow इडागम\-निषेधः~\arrow \textcolor{red}{पुगन्त\-लघूपधस्य च} (पा॰सू॰~७.३.८६)~\arrow योध्~स्य~इ~\arrow \textcolor{red}{खरि च} (पा॰सू॰~८.४.५५)~\arrow चर्त्वम्~\arrow योत्~स्य~इ~\arrow \textcolor{red}{टित आत्मनेपदानां टेरे} (पा॰सू॰~३.४.७९)~\arrow योत्~स्य~ए~\arrow \textcolor{red}{अतो गुणे} (पा॰सू॰~६.१.९७)~\arrow योत्~स्ये~\arrow योत्स्ये। युध्~\arrow \textcolor{red}{अनुदात्तेत्त्व\-लक्षणमात्मने\-पदमनित्यम्} (प॰शे॰~९३.४)~\arrow \textcolor{red}{शेषात्कर्तरि परस्मैपदम्} (पा॰सू॰~१.३.७८)~\arrow \textcolor{red}{लृट् शेषे च} (पा॰सू॰~३.३.१३)~\arrow युध्~लृँट्~\arrow युध्~मिप्~\arrow युध्~मि~\arrow \textcolor{red}{स्यतासी लृलुटोः} (पा॰सू॰~३.१.३३)~\arrow युध्~स्य~मि~\arrow \textcolor{red}{अतो दीर्घो यञि} (पा॰सू॰~७.३.१०१)~\arrow युध्~स्या~मि~\arrow \textcolor{red}{एकाच उपदेशेऽनुदात्तात्‌} (पा॰सू॰~७.२.१०)~\arrow इडागम\-निषेधः~\arrow \textcolor{red}{पुगन्त\-लघूपधस्य च} (पा॰सू॰~७.३.८६)~\arrow लघूपध\-गुणः~\arrow योध्~स्या~मि~\arrow \textcolor{red}{खरि च} (पा॰सू॰~८.४.५५)~\arrow योत्~स्या~मि~\arrow योत्स्यामि।}\end{sloppypar}
\section[ससृजे]{ससृजे}
\centering\textcolor{blue}{अस्त्रं राक्षसराजस्य जघान परमास्त्रवित्।\nopagebreak\\
ततस्तु ससृजे घोरं राक्षसं चास्त्रमस्त्रवित्।\nopagebreak\\
क्रोधेन महताऽऽविष्टो रामस्योपरि रावणः॥}\nopagebreak\\
\raggedleft{–~अ॰रा॰~६.११.२८}\\
\fontsize{14}{21}\selectfont\begin{sloppypar}\hyphenrules{nohyphenation}\justifying\noindent\hspace{10mm} \textcolor{red}{सृजँ विसर्गे} (धा॰पा॰~१४१४) परस्मैपदी धातुः। ततः \textcolor{red}{ससर्ज} इति रूपम्।\footnote{\textcolor{red}{सृजँ विसर्गे} (धा॰पा॰~१४१४)~\arrow सृज्~\arrow \textcolor{red}{शेषात्कर्तरि परस्मैपदम्} (पा॰सू॰~१.३.७८)~\arrow \textcolor{red}{परोक्षे लिट्} (पा॰सू॰~३.२.११५)~\arrow सृज्~लिँट्~\arrow सृज्~तिप्~\arrow \textcolor{red}{परस्मैपदानां णलतुसुस्थलथुस\-णल्वमाः} (पा॰सू॰~३.४.८२)~\arrow सृज्~णल्~\arrow सृज्~अ~\arrow \textcolor{red}{लिटि धातोरनभ्यासस्य} (पा॰सू॰~६.१.८)~\arrow सृज्~सृज्~अ~\arrow \textcolor{red}{उरत्} (पा॰सू॰~७.४.६६)~\arrow \textcolor{red}{उरण् रपरः} (पा॰सू॰~१.१.५१)~\arrow सर्ज्~सृज्~अ~\arrow \textcolor{red}{हलादिः शेषः} (पा॰सू॰~७.४.६०)~\arrow स~सृज्~अ~\arrow \textcolor{red}{पुगन्त\-लघूपधस्य च} (पा॰सू॰~७.३.८६)~\arrow \textcolor{red}{उरण् रपरः} (पा॰सू॰~१.१.५१)~\arrow स~सर्ज्~अ~\arrow ससर्ज।} अत्र कर्म\-व्यतिहारादात्मनेपदम्।\footnote{\textcolor{red}{कर्तरि कर्मव्यतिहारे} (पा॰सू॰~१.३.१४) इत्यनेन। प्राप्त आत्मने\-पदे लिटि प्रथम\-पुरुष एक\-वचने \textcolor{red}{ससृजे} इति रूपम्। सृज्~\arrow \textcolor{red}{कर्तरि कर्मव्यतिहारे} (पा॰सू॰~१.३.१४)~\arrow \textcolor{red}{परोक्षे लिट्} (पा॰सू॰~३.२.११५)~\arrow सृज्~लिँट्~\arrow सृज्~तिप्~\arrow \textcolor{red}{लिटस्तझयोरेशिरेच्} (पा॰सू॰~३.४.८१)~\arrow सृज्~एश्~\arrow सृज्~ए~\arrow \textcolor{red}{लिटि धातोरनभ्यासस्य} (पा॰सू॰~६.१.८)~\arrow सृज्~सृज्~ए~\arrow \textcolor{red}{हलादिः शेषः} (पा॰सू॰~७.४.६०)~\arrow सृ~सृज्~ए~\arrow \textcolor{red}{उरत्} (पा॰सू॰~७.४.६६)~\arrow \textcolor{red}{उरण् रपरः} (पा॰सू॰~१.१.५१)~\arrow सर्~सृज्~ए~\arrow \textcolor{red}{हलादिः शेषः} (पा॰सू॰~७.४.६०)~\arrow स~सृज्~ए~\arrow \textcolor{red}{सार्वधातुकमपित्} (पा॰सू॰~१.२.४)~\arrow ङित्त्वम्~\arrow \textcolor{red}{ग्क्ङिति च} (पा॰सू॰~१.१.५)~\arrow लघूपध\-गुण\-निषेधः~\arrow स~सृज्~ए~\arrow ससृजे।}\end{sloppypar}
\section[लिप्यसे]{लिप्यसे}
\centering\textcolor{blue}{भूतं भविष्यदभजन्वर्तमानमथाचरन्।\nopagebreak\\
विहरस्व यथान्यायन् भवदोषैर्न लिप्यसे॥}\nopagebreak\\
\raggedleft{–~अ॰रा॰~६.१२.२७}\\
\fontsize{14}{21}\selectfont\begin{sloppypar}\hyphenrules{nohyphenation}\justifying\noindent\hspace{10mm} अत्र \textcolor{red}{लिप्‌}\-धातोः (\textcolor{red}{लिपँ उपदेहे} धा॰पा॰~१४३३) कर्मवाच्ये लड्लकारे मध्यमपुरुष एकवचनरूपम्।\footnote{लिप्~\arrow \textcolor{red}{भावकर्मणोः} (पा॰सू॰~१.३.१३)~\arrow \textcolor{red}{वर्तमान\-सामीप्ये वर्तमानवद्वा} (पा॰सू॰~३.३.१३१)~\arrow \textcolor{red}{वर्तमाने लट्} (पा॰सू॰~३.२.१२३)~\arrow लिप्~लँट्~\arrow लिप्~थास्~\arrow \textcolor{red}{सार्वधातुके यक्} (पा॰सू॰~३.१.६७)~\arrow लिप्~यक्~थास्~\arrow लिप्~य~थास्~\arrow \textcolor{red}{थासस्से} (पा॰सू॰~३.४.८०)~\arrow लिप्~य~से~\arrow लिप्यसे।} वर्तमानसामीप्याद्भविष्यति वर्तमानता।\footnote{\textcolor{red}{वर्तमान\-सामीप्ये वर्तमानवद्वा} (पा॰सू॰~३.३.१३१) इत्यनेन।}\end{sloppypar}
\section[गमिष्यामहे]{गमिष्यामहे}
\centering\textcolor{blue}{अलङ्कृत्य सह भ्राता श्वो गमिष्यामहे वयम्।\nopagebreak\\
विभीषणवचः श्रुत्वा प्रत्युवाच रघूत्तमः॥}\nopagebreak\\
\raggedleft{–~अ॰रा॰~६.१३.४२}\\
\fontsize{14}{21}\selectfont\begin{sloppypar}\hyphenrules{nohyphenation}\justifying\noindent\hspace{10mm} \textcolor{red}{गमेरिट् परस्मैपदेषु} (पा॰सू॰~७.२.५८) इत्यनेनेड्विधानात् \textcolor{red}{गम्‌}\-धातोश्च (\textcolor{red}{गमॢँ गतौ} धा॰पा॰~९८२) परस्मैपदत्वात् \textcolor{red}{गमिष्यामः}।\footnote{गम्~\arrow \textcolor{red}{शेषात्कर्तरि परस्मैपदम्} (पा॰सू॰~१.३.७८)~\arrow \textcolor{red}{लृट् शेषे च} (पा॰सू॰~३.३.१३)~\arrow गम्~लृँट्~\arrow गम्~मस्~\arrow \textcolor{red}{स्यतासी लृलुटोः} (पा॰सू॰~३.१.३३)~\arrow गम्~स्य~मस्~\arrow \textcolor{red}{गमेरिट् परस्मैपदेषु} (पा॰सू॰~७.२.५८)~\arrow गम्~इट्~स्य~मस्~\arrow गम्~इ~स्य~मस्~\arrow \textcolor{red}{अतो दीर्घो यञि} (पा॰सू॰~७.३.१०१)~\arrow गम्~इ~स्या~मस्~\arrow \textcolor{red}{आदेश\-प्रत्यययोः} (पा॰सू॰~८.३.५९)~\arrow गम्~इ~ष्या~मस्~\arrow \textcolor{red}{ससजुषो रुः} (पा॰सू॰~८.२.६६)~\arrow गम्~इ~ष्या~मरुँ~\arrow \textcolor{red}{खरवसानयोर्विसर्जनीयः} (पा॰सू॰~८.३.१५)~\arrow गम्~इ~ष्या~मः~\arrow गमिष्यामः।} समुपसर्ग\-संयोजने तु \textcolor{red}{समो गम्यृच्छिभ्याम्} (पा॰सू॰~१.३.२९) इत्यनेनाऽत्मनेपदे \textcolor{red}{सङ्गंस्यामहे}\footnote{सम्~गम्~\arrow \textcolor{red}{समो गम्यृच्छिभ्याम्} (पा॰सू॰~१.३.२९)~\arrow \textcolor{red}{लृट् शेषे च} (पा॰सू॰~३.३.१३)~\arrow सम्~गम्~लृँट्~\arrow सम्~गम्~महिङ्~\arrow सम्~गम्~महि~\arrow \textcolor{red}{स्यतासी लृलुटोः} (पा॰सू॰~३.१.३३)~\arrow सम्~गम्~स्य~महि~\arrow \textcolor{red}{गमेरिट् परस्मैपदेषु} (पा॰सू॰~७.२.५८)~\arrow इडभावः~\arrow \textcolor{red}{अतो दीर्घो यञि} (पा॰सू॰~७.३.१०१)~\arrow सम्~गम्~स्या~महि~\arrow \textcolor{red}{टित आत्मनेपदानां टेरे} (पा॰सू॰~३.४.७९)~\arrow सम्~गम्~स्या~महे~\arrow \textcolor{red}{मोऽनुस्वारः} (पा॰सू॰~८.३.२३)~\arrow सं~गं~स्या~महे~\arrow \textcolor{red}{अनुस्वारस्य ययि परसवर्णः} (पा॰सू॰~८.४.५८)~\arrow सङ्गंस्यामहे।} उपसर्ग\-लोपेऽपि \textcolor{red}{गंस्यामहे} किन्तु \textcolor{red}{गमिष्यामहे} इति त्वत्यन्तमसामञ्जस्यावहमिति चेत्। \textcolor{red}{गमनं गमि} इत्यत्र \textcolor{red}{घिनुण्} प्रत्ययः। \textcolor{red}{शमित्यष्टाभ्यो घिनुण्} (पा॰सू॰~३.२.१४१) इत्यत्र \textcolor{red}{इति}\-शब्द\-प्रयोगाद्गमेरपि घिनुण्। घिनुण्प्रत्ययेऽनुबन्ध\-कार्ये नपुंसक\-लिङ्गे क्रिया\-विशेषणत्वाद्द्वितीया। \textcolor{red}{हे} इति सम्बोधनम्। \textcolor{red}{स्याम} इति \textcolor{red}{अस्‌}\-धातोः (\textcolor{red}{असँ भुवि} धा॰पा॰~१०६५) विध्यर्थे लिङ्लकार उत्तम\-पुरुष\-बहु\-वचनम्।\footnote{\textcolor{red}{असँ भुवि} (धा॰पा॰~१०६५)~\arrow अस्~\arrow \textcolor{red}{शेषात्कर्तरि परस्मैपदम्} (पा॰सू॰~१.३.७८)~\arrow \textcolor{red}{आशिषि लिङ्लोटौ} (पा॰सू॰~३.३.१७३)~\arrow अस्~लिङ्~\arrow अस्~मस्~\arrow \textcolor{red}{यासुट् परस्मै\-पदेषूदात्तो ङिच्च} (पा॰सू॰~३.४.१०३)~\arrow अस्~यासुँट्~मस्~\arrow अस्~यास्~मस्~\arrow \textcolor{red}{लिङः सलोपोऽनन्त्यस्य} (पा॰सू॰~७.२.७९)~\arrow अस्~या~मस्~\arrow \textcolor{red}{श्नसोरल्लोपः} (पा॰सू॰~६.४.१११)~\arrow स्~या~मस्~\arrow \textcolor{red}{नित्यं ङितः} (पा॰सू॰~३.४.९९)~\arrow स्~या~म~\arrow स्याम।} अर्थात् \textcolor{red}{हे जना अयोध्यां प्रति वयं गमि गमनं प्रति स्यामोद्यता भवेम}। यद्वा \textcolor{red}{गम्} इति \textcolor{red}{ड}\-प्रत्ययान्तम्।\footnote{\textcolor{red}{गमॢँ}\-धातोः \textcolor{red}{अन्येष्वपि दृश्यते} (पा॰सू॰~३.२.१०१) इत्यनेन भावे \textcolor{red}{ड}\-प्रत्यये \textcolor{red}{डित्यभस्याप्यनु\-बन्धकरण\-सामर्थ्यात्} (वा॰~६.४.१४३) इत्यनेन टिलोपे विभक्ति\-कार्ये \textcolor{red}{गम्}। \textcolor{red}{अन्येष्वपि} इत्यनेन बाहुलकाद्भावेऽपि। \textcolor{red}{अपिशब्दः सर्वोपाधि\-व्यभिचारार्थः} (का॰वृ॰~३.२.१०१)। गं गीतमितिवत्। यथा शब्दकल्पद्रुमे – \textcolor{red}{गं, क्ली॰, गीयते इति (गै गाने + भावे बाहुलकात् डः) गीतम्, इत्येकाक्षरकोषः}। गम्~\arrow \textcolor{red}{अन्येष्वपि दृश्यते} (पा॰सू॰~३.२.१०१)~\arrow गम्~ड~\arrow गम्~अ~\arrow \textcolor{red}{टेः} (पा॰सू॰~६.४.१४३)~\arrow \textcolor{red}{डित्त्व\-सामर्थ्यादभस्यापि टेर्लोपः} (ल॰सि॰कौ॰~३४३)~\arrow ग्~अ~\arrow ग~\arrow विभक्ति\-कार्यम्~\arrow ग~अम्~\arrow \textcolor{red}{अमि पूर्वः} (पा॰सू॰~६.१.१०७)~\arrow गम्।} \textcolor{red}{इषु}\-धातुः (\textcolor{red}{इषँ गतौ} धा॰पा॰~११२७) दिवादिः। तस्य वर्तमान\-काले कर्म\-व्यतिहार आत्मनेपदम्।\footnote{\textcolor{red}{कर्तरि कर्मव्यतिहारे} (पा॰सू॰~१.३.१४) इत्यनेन।} \textcolor{red}{श्वः} इति सन्निधानेनापि वर्तमान\-समीपे लट्।\footnote{\textcolor{red}{वर्तमान\-सामीप्ये वर्तमानवद्वा} (पा॰सू॰~३.३.१३१) इत्यनेन।} \textcolor{red}{श्वो गं गमनमिष्यामहेऽभिलषामः}\footnote{इष्~\arrow \textcolor{red}{कर्तरि कर्मव्यतिहारे} (पा॰सू॰~१.३.१४)~\arrow \textcolor{red}{वर्तमान\-सामीप्ये वर्तमानवद्वा} (पा॰सू॰~३.३.१३१)~\arrow \textcolor{red}{वर्तमाने लट्} (पा॰सू॰~३.२.१२३)~\arrow इष्~लट्~\arrow इष्~महिङ्~\arrow इष्~महि~\arrow \textcolor{red}{दिवादिभ्यः श्यन्} (पा॰सू॰~३.१.६९)~\arrow इष्~श्यन्~महि~\arrow इष्~य~महि~\arrow \textcolor{red}{अतो दीर्घो यञि} (पा॰सू॰~७.३.१०१)~\arrow इष्~या~महि~\arrow \textcolor{red}{टित आत्मनेपदानां टेरे} (पा॰सू॰~३.४.७९)~\arrow इष्~या~महे~\arrow इष्यामहे।} प्रयोगेऽस्मिन्नियमेव मे मनीषा।\end{sloppypar}
\section[प्रार्थयामि]{प्रार्थयामि}
\label{sec:prarthayami}
\centering\textcolor{blue}{प्रार्थयामि जगन्नाथ पवित्रं कुरु मे गृहम्।\nopagebreak\\
स्थित्वाद्य भुक्त्वा सबलः श्वो गमिष्यसि पत्तनम्॥}\nopagebreak\\
\raggedleft{–~अ॰रा॰~६.१४.३६}\\
\fontsize{14}{21}\selectfont\begin{sloppypar}\hyphenrules{nohyphenation}\justifying\noindent\hspace{10mm} \textcolor{red}{अनुदात्तेत्त्व\-लक्षणमात्मने\-पदमनित्यम्} (प॰शे॰~९३.४) इति \textcolor{red}{चक्षिङ्} (धा॰पा॰~१०१७) इत्यत्र ङित्करणेन ज्ञाप्यते। अतः \textcolor{red}{प्रार्थयामि}।\footnote{\textcolor{red}{अनुदात्तेत्त्व\-लक्षणमात्मने\-पदमनित्यम्} (प॰शे॰~९३.४) इत्यनेन। प्र~\textcolor{red}{अर्थँ उपयाच्ञायाम्} (धा॰पा॰~१९०५)~\arrow प्र~अर्थ्~\arrow \textcolor{red}{सत्याप\-पाश\-रूप\-वीणा\-तूल\-श्लोक\-सेना\-लोम\-त्वच\-वर्म\-वर्ण\-चूर्ण\-चुरादिभ्यो णिच्} (पा॰सू॰~३.१.२५)~\arrow प्र~अर्थ्~णिच्~\arrow प्र~अर्थ्~इ~\arrow प्र~अर्थि~\arrow \textcolor{red}{सनाद्यन्ता धातवः} (पा॰सू॰~३.१.३२)~\arrow धातु\-सञ्ज्ञा~\arrow \textcolor{red}{अनुदात्तेत्त्व\-लक्षणमात्मने\-पदमनित्यम्} (प॰शे॰~९३.४)~\arrow \textcolor{red}{शेषात्कर्तरि परस्मैपदम्} (पा॰सू॰~१.३.७८)~\arrow \textcolor{red}{वर्तमाने लट्} (पा॰सू॰~३.२.१२३)~\arrow प्र~अर्थि~लट्~\arrow प्र~अर्थि~मिप्~\arrow प्र~अर्थि~मि~\arrow \textcolor{red}{कर्तरि शप्‌} (पा॰सू॰~३.१.६८)~\arrow प्र~अर्थि~शप्~मि~\arrow प्र~अर्थि~अ~मि~\arrow \textcolor{red}{सार्वधातुकार्ध\-धातुकयोः} (पा॰सू॰~७.३.८४)~\arrow प्र~अर्थे~अ~मि~\arrow \textcolor{red}{एचोऽयवायावः} (पा॰सू॰~६.१.७८)~\arrow प्र~अर्थय्~अ~मि~\arrow \textcolor{red}{अतो दीर्घो यञि} (पा॰सू॰~७.३.१०१)~\arrow प्र~अर्थय्~आ~मि~\arrow प्र~अर्थयामि~\arrow \textcolor{red}{अकः सवर्णे दीर्घः} (पा॰सू॰~६.१.१०१)~\arrow प्रार्थयामि। \pageref{sec:prarthaya}तमे पृष्ठे \ref{sec:prarthaya} \nameref{sec:prarthaya} इति प्रयोगस्य विमर्शमपि पश्यन्तु।}\end{sloppypar}
\vspace{2mm}
\centering ॥ इति युद्धकाण्डीयप्रयोगाणां विमर्शः ॥\nopagebreak\\
\vspace{4mm}
\pdfbookmark[2]{उत्तरकाण्डम्}{Chap3Part2Kanda7}
\phantomsection
\addtocontents{toc}{\protect\setcounter{tocdepth}{2}}
\addcontentsline{toc}{subsection}{उत्तरकाण्डीयप्रयोगाणां विमर्शः}
\addtocontents{toc}{\protect\setcounter{tocdepth}{0}}
\centering ॥ अथोत्तरकाण्डीयप्रयोगाणां विमर्शः ॥\nopagebreak\\
\section[निजघ्ने]{निजघ्ने}
\centering\textcolor{blue}{ब्राह्मणानृषिमुख्यांश्च देवदानवकिन्नरान्।\nopagebreak\\
देवश्रियो मनुष्यांश्च निजघ्ने समहोरगान्॥}\nopagebreak\\
\raggedleft{–~अ॰रा॰~७.२.४७}\\
\fontsize{14}{21}\selectfont\begin{sloppypar}\hyphenrules{nohyphenation}\justifying\noindent\hspace{10mm} \textcolor{red}{आङो यमहनः} (पा॰सू॰~१.३.२८) इत्यनेनात्मनेपदे \textcolor{red}{न्याजघ्ने}।\footnote{न्या~हन्~\arrow \textcolor{red}{आङो यमहनः} (पा॰सू॰~१.३.२८)~\arrow \textcolor{red}{परोक्षे लिट्} (पा॰सू॰~३.२.११५)~\arrow न्या~हन्~लिँट्~\arrow न्या~हन्~त~\arrow \textcolor{red}{लिटस्तझयोरेशिरेच्} (पा॰सू॰~३.४.८१)~\arrow न्या~हन्~एश्~\arrow न्या~हन्~ए~\arrow \textcolor{red}{लिटि धातोरनभ्यासस्य} (पा॰सू॰~६.१.८)~\arrow न्या~हन्~हन्~ए~\arrow \textcolor{red}{हलादिः शेषः} (पा॰सू॰~७.४.६०)~\arrow न्या~ह~हन्~ए~\arrow \textcolor{red}{कुहोश्चुः} (पा॰सू॰~७.४.६२)~\arrow न्या~झ~हन्~ए~\arrow \textcolor{red}{अभ्यासे चर्च} (पा॰सू॰~८.४.५४)~\arrow न्या~ज~हन्~ए~\arrow \textcolor{red}{असंयोगाल्लिट्कित्} (पा॰सू॰~६.४.९८)~\arrow कित्त्वम्~\arrow \textcolor{red}{गम\-हन\-जन\-खन\-घसां लोपः क्ङित्यनङि} (पा॰सू॰~६.४.९८)~\arrow न्या~ज~ह्~न्~ए~\arrow \textcolor{red}{हो हन्तेर्ञ्णिन्नेषु} (पा॰सू॰~७.३.५४)~\arrow न्या~ज~घ्~न्~ए~\arrow न्याजघ्ने।} \textcolor{red}{विनाऽपि प्रत्ययं पूर्वोत्तर\-पद\-लोपो वक्तव्यः} (वा॰~५.३.८३) इत्यनेन \textcolor{red}{आङ्} इत्यस्य लोपे \textcolor{red}{निजघ्ने}।\footnote{परस्मैपदे तु \textcolor{red}{निजघान} इति रूपम्। नि~हन्~\arrow \textcolor{red}{शेषात्कर्तरि परस्मैपदम्} (पा॰सू॰~१.३.७८)~\arrow \textcolor{red}{परोक्षे लिट्} (पा॰सू॰~३.२.११५)~\arrow नि~हन्~लिँट्~\arrow नि~हन्~तिप्~\arrow \textcolor{red}{परस्मैपदानां णलतुसुस्थलथुस\-णल्वमाः} (पा॰सू॰~३.४.८२)~\arrow नि~हन्~णल्~\arrow नि~हन्~अ~\arrow \textcolor{red}{लिटि धातोरनभ्यासस्य} (पा॰सू॰~६.१.८)~\arrow नि~हन्~हन्~अ~\arrow \textcolor{red}{हलादिः शेषः} (पा॰सू॰~७.४.६०)~\arrow नि~ह~हन्~अ~\arrow \textcolor{red}{कुहोश्चुः} (पा॰सू॰~७.४.६२)~\arrow नि~झ~हन्~अ~\arrow \textcolor{red}{अभ्यासे चर्च} (पा॰सू॰~८.४.५४)~\arrow नि~ज~हन्~अ~\arrow \textcolor{red}{हो हन्तेर्ञ्णिन्नेषु} (पा॰सू॰~७.३.५४)~\arrow नि~ज~घन्~अ~\arrow \textcolor{red}{अत उपधायाः} (पा॰सू॰~७.२.११६)~\arrow नि~ज~घान्~अ~\arrow निजघान।}\end{sloppypar}
\section[काङ्क्षे]{काङ्क्षे}
\centering\textcolor{blue}{सत्येन च शपे नाहं त्वां विना दिवि वा भुवि।\nopagebreak\\
काङ्क्षे राज्यं रघुश्रेष्ठ शपे त्वत्पादयोः प्रभो॥}\nopagebreak\\
\raggedleft{–~अ॰रा॰~७.९.५}\\
\fontsize{14}{21}\selectfont\begin{sloppypar}\hyphenrules{nohyphenation}\justifying\noindent\hspace{10mm} \textcolor{red}{काङ्क्षामि}\footnote{\textcolor{red}{काक्षिँ काङ्क्षायाम्} (धा॰पा॰~६६७)~\arrow काक्ष्~\arrow \textcolor{red}{इदितो नुम् धातोः} (पा॰सू॰~७.१.५८)~\arrow \textcolor{red}{मिदचोऽन्त्यात्परः} (पा॰सू॰~१.१.४७)~\arrow का~नुँम्~क्ष्~\arrow कान्~क्ष्~\arrow \textcolor{red}{नश्चापदान्तस्य झलि} (पा॰सू॰~८.३.२४)~\arrow कांक्ष्~\arrow \textcolor{red}{अनुस्वारस्य ययि परसवर्णः} (पा॰सू॰~८.४.५८)~\arrow काङ्क्ष्~\arrow \textcolor{red}{शेषात्कर्तरि परस्मैपदम्} (पा॰सू॰~१.३.७८)~\arrow \textcolor{red}{वर्तमाने लट्} (पा॰सू॰~३.२.१२३)~\arrow काङ्क्ष्~लट्~\arrow काङ्क्ष्~मिप्~\arrow काङ्क्ष्~मि~\arrow \textcolor{red}{कर्तरि शप्} (पा॰सू॰~३.१.६८)~\arrow काङ्क्ष्~शप्~मि~\arrow काङ्क्ष्~अ~मि~\arrow \textcolor{red}{अतो दीर्घो यञि} (पा॰सू॰~७.३.१०१)~\arrow काङ्क्ष्~आ~मि~\arrow काङ्क्षामि।} इति प्रयोक्तव्ये \textcolor{red}{काङ्क्षे}\footnote{\textcolor{red}{काक्षिँ काङ्क्षायाम्} (धा॰पा॰~६६७)~\arrow काङ्क्ष् (पूर्ववत्)~\arrow \textcolor{red}{कर्तरि कर्म\-व्यतिहारे} (पा॰सू॰~१.३.१४)~\arrow \textcolor{red}{वर्तमाने लट्} (पा॰सू॰~३.२.१२३)~\arrow काङ्क्ष्~लट्~\arrow काङ्क्ष्~इट्~\arrow काङ्क्ष्~इ~\arrow \textcolor{red}{कर्तरि शप्} (पा॰सू॰~३.१.६८)~\arrow काङ्क्ष्~शप्~इ~\arrow काङ्क्ष्~अ~इ~\arrow \textcolor{red}{टित आत्मनेपदानां टेरे} (पा॰सू॰~३.४.७९)~\arrow काङ्क्ष्~अ~ए~\arrow \textcolor{red}{अतो गुणे} (पा॰सू॰~६.१.९७)~\arrow काङ्क्ष्~ए~\arrow काङ्क्षे।} इति प्रयोगस्तु \textcolor{red}{कर्तरि कर्म\-व्यतिहारे} (पा॰सू॰~१.३.१४) इत्यनेनात्मनेपदे सति।\footnote{एवमेव \textcolor{red}{न काङ्क्षे विजयं कृष्ण न च राज्यं सुखानि च} (भ॰गी॰~१.३२) इत्यत्र। विजय\-राज्य\-सुखानामकाङ्क्षा न राजोचिता।}\end{sloppypar}
\vspace{2mm}
\centering ॥ इत्युत्तरकाण्डीयप्रयोगाणां विमर्शः ॥\nopagebreak\\
\vspace{4mm}
\centering इत्यध्यात्म\-रामायणेऽपाणिनीय\-प्रयोगाणां\-विमर्श\-नामके शोध\-प्रबन्धे तृतीयाध्याये द्वितीय\-परिच्छेदः।\nopagebreak\\
\vspace{4mm}
\centering इत्यध्यात्म\-रामायणेऽपाणिनीय\-प्रयोगाणां\-विमर्श\-नामके शोध\-प्रबन्धे तृतीयोऽध्यायः।\\
\vspace{4mm}
\centering\textcolor{blue}{\fontsize{16}{24}\selectfont बुद्ध्या श्रीगुरुपादपद्मरजसा संशुद्धया सादरं\nopagebreak\\
कृत्वा लेखकमाप्तशीलयशसं शिष्यं शिशुं राघवम्।\nopagebreak\\
बालो नष्टविलोचनो गिरिधरः शब्दान् विभाव्याऽत्मना\nopagebreak\\
बध्नाति स्म निबन्धमेतममलं तोषाय सीतापतेः॥}\nopagebreak\\
\vspace{4mm}
\centering इत्यध्यात्म\-रामायणेऽपाणिनीय\-प्रयोगाणां\-विमर्शः।
