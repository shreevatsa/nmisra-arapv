% Nityanand Misra: LaTeX code to typeset a book in Sanskrit
% Copyright (C) 2016 Nityanand Misra
%
% This program is free software: you can redistribute it and/or modify it under
% the terms of the GNU General Public License as published by the Free Software
% Foundation, either version 3 of the License, or (at your option) any later
% version.
%
% This program is distributed in the hope that it will be useful, but WITHOUT
% ANY WARRANTY; without even the implied warranty of  MERCHANTABILITY or FITNESS
% FOR A PARTICULAR PURPOSE. See the GNU General Public License for more details.
%
% You should have received a copy of the GNU General Public License along with
% this program.  If not, see <http://www.gnu.org/licenses/>.

\renewcommand\chaptername{अथ प्रथमोऽध्यायः}
\chapter[\texorpdfstring{सन्धिकारकसमासप्रकरणम्‌}{प्रथमोऽध्यायः}]{सन्धिकारकसमासप्रकरणम्‌}
\vspace{-5mm}
\fontsize{16}{24}\selectfont\centering\hyphenrules{nohyphenation}\textcolor{blue}{अथ नत्वा घनश्यामं रामं सीतासमन्वितम्।\nopagebreak\\
अपाणिनीयानध्यात्मरामायणगतान् किल॥\nopagebreak\\
शब्दान् श्रीगुरुपादाब्जरजसा बुद्धया धिया।\nopagebreak\\
विमर्शये\footnote{\textcolor{red}{विमर्शये} इत्यत्र स्वार्थे णिच्। \textcolor{red}{निवृत्त\-प्रेषणाद्धातोः प्राकृतेऽर्थे णिजुच्यते} (वा॰प॰~३.७.६०)। स्वान्तःसुखाय विमृशामीति कर्त्रभिप्रायं ध्वनयितुमात्मने\-पदप्रयोगः। \textcolor{red}{णिचश्च} (पा॰सू॰~१.३.७४) इत्यनेन। वि~\textcolor{red}{मृशँ आमर्शने} (धा॰पा॰~१४२५)~\arrow वि~मृश्~\arrow स्वार्थे णिच्~\arrow वि~मृश्~णिच्~\arrow वि~मृश्~इ~\arrow \textcolor{red}{पुगन्त\-लघूपधस्य च} (पा॰सू॰~७.३.८६)~\arrow \textcolor{red}{उरण् रपरः} (पा॰सू॰~१.१.५१)~\arrow वि~मर्श्~इ~\arrow विमर्शि~\arrow \textcolor{red}{सनाद्यन्ता धातवः} (पा॰सू॰~३.१.३२)~\arrow धातु\-सञ्ज्ञा~\arrow \textcolor{red}{णिचश्च} (पा॰सू॰~१.३.७४)~\arrow \textcolor{red}{वर्तमाने लट्} (पा॰सू॰~३.२.१२३)~\arrow विमर्शि~इट्~\arrow विमर्शि~इ~\arrow \textcolor{red}{कर्तरि शप्‌} (पा॰सू॰~३.१.६८)~\arrow विमर्शि~शप्~इ~\arrow विमर्शि~अ~इ~\arrow \textcolor{red}{सार्वधातुकार्ध\-धातुकयोः} (पा॰सू॰~७.३.८४)~\arrow विमर्शे~अ~इ~\arrow \textcolor{red}{एचोऽयवायावः} (पा॰सू॰~६.१.७८)~\arrow विमर्शय्~अ~इ~\arrow \textcolor{red}{आद्गुणः} (पा॰सू॰~६.१.८७)~\arrow विमर्शय्~ए~\arrow विमर्शये। यद्वा \textcolor{red}{विमर्शं कुर्वे} इति विग्रहे \textcolor{red}{विमर्शये}। अत्रापि कर्त्रभिप्राये \textcolor{red}{णिचश्च} (पा॰सू॰~१.३.७४) इत्यनेनाऽत्मने\-पदम्। विमर्श~\arrow \textcolor{red}{तत्करोति तदाचष्टे} (धा॰पा॰ ग॰सू॰)~\arrow विमर्श~णिच्~\arrow विमर्श~इ~\arrow \textcolor{red}{णाविष्ठवत्प्राति\-पदिकस्य पुंवद्भाव\-रभाव\-टिलोप\-यणादि\-परार्थम्} (वा॰~६.४.४८)~\arrow विमर्श्~इ~\arrow विमर्शि~\arrow \textcolor{red}{सनाद्यन्ता धातवः} (पा॰सू॰~३.१.३२)~\arrow धातु\-सञ्ज्ञा~\arrow \textcolor{red}{णिचश्च} (पा॰सू॰~१.३.७४)~\arrow \textcolor{red}{वर्तमाने लट्} (पा॰सू॰~३.२.१२३)~\arrow विमर्शि~इट्~\arrow शेषं पूर्ववत्।} यथाबुद्धि सुधियो मर्षयन्त्वघम्॥}\nopagebreak\\
\vspace{4mm}
\fontsize{14}{21}\selectfont
\begin{sloppypar}\hyphenrules{nohyphenation}\justifying\noindent\hspace{10mm} अथ प्रकृतमनुसरामि। तत्र पूर्वं \textcolor{red}{रामायण}\-शब्द एव साधना\-प्रकारं प्रदर्शये। रामा सीता रामो रामचन्द्रो रामा च रामश्चेति विग्रहे \textcolor{red}{पुमान् स्त्रिया} (पा॰सू॰~१.२.६७) \textcolor{red}{स्त्रिया सहोक्तौ पुमाञ्छिष्यते} (वै॰सि॰कौ॰~९३३) इतिवृत्तिकेन सूत्रेणानेन स्त्री\-वाचकस्य लोपे \textcolor{red}{यः शिष्यते स लुप्यमानार्थाभिधायी}\footnote{मूलं मृग्यम्।} इति नियमाद्राम\-शब्दस्यैव द्वित्व\-बोधकतया \textcolor{red}{रामौ} इति शब्दः सीताराम\-बोधको \textcolor{red}{हंसी च हंसश्च हंसौ} इतिवत्ततो \textcolor{red}{रामयोः} सीता\-रामयोः \textcolor{red}{अयनम्‌} इति विग्रहे \textcolor{red}{इण् गतौ} (धा॰पा॰~१.१०४५) इत्यस्माद्धातोरीयते गम्यत इति कर्म\-व्युत्पत्त्या बाहुलकात्कर्मणि ल्युट्।\footnote{\textcolor{red}{कृत्यल्युटो बहुलम्‌} (पा॰सू॰~३.३.११३) इत्यनेन।}\end{sloppypar}
\begin{sloppypar}\hyphenrules{nohyphenation}\justifying\noindent\hspace{10mm} सम्प्रति \textcolor{red}{अध्यात्म\-रामायण}\-शब्दस्य सिद्धिं प्रदर्शये। अधिगतोऽन्तर्यामित्वेनाधिश्रित आत्मा मनो बुद्धिरहङ्कारश्चित्तं प्रत्यगात्मा वा याभ्यां तौ \textcolor{red}{अध्यात्मानौ} इत्यत्र \textcolor{red}{प्रादिभ्यो धातुजस्य वाच्यो वा चोत्तर\-पद\-लोपश्च} (वा॰~२.२.२२) इति वार्त्तिक\-बलेन \textcolor{red}{अनेकमन्य\-पदार्थे} (पा॰सू॰~२.२.२४) इति सूत्रेण बहुव्रीहि\-समासेऽथवाऽऽत्मानमधिगताविति \textcolor{red}{अध्यात्मानौ} एवं \textcolor{red}{अत्यादयः क्रान्ताद्यर्थे द्वितीयया} (वा॰~२.२.१८) इति वार्त्तिकेन तत्पुरुष\-समासे पुनरध्यात्म\-शब्दस्य द्विवचनान्त\-राम\-शब्देन सहाध्यात्मानौ च तौ रामौ चेति विग्रहे सति \textcolor{red}{विशेषणं विशेष्येण बहुलम्‌} (पा॰सू॰~२.१.५७) इति सूत्रेण कर्मधारय\-समासे पश्चादध्यात्म\-राम\-शब्दस्यायन\-शब्देन सह षष्ठी\-तत्पुरुषः। यद्वाऽध्यात्म\-रामाभ्यामयनमिति विग्रहे चतुर्थी\-तत्पुरुषः। \textcolor{red}{चतुर्थी तदर्थार्थ\-बलि\-हित\-सुख\-रक्षितैः} (पा॰सू॰~२.१.३६) इति सूत्रेण। उक्त\-सूत्रे वर्णित\-समस्यमान\-लक्षणतावच्छेदकत्वावच्छिन्न\-प्रतियोगितावच्छेदकताभाव\-विरहात्कथमयन\-शब्देन सह समास इति न शङ्क्यम्। पूर्वं \textcolor{red}{चतुर्थी} इति योगेन विभज्यताम्। अर्थश्च भवेत्। चतुर्थ्यन्तं सुबन्तेन समर्थेन समस्यते। इत्यर्थे समासः। पश्चात् \textcolor{red}{पूर्वपदात्सञ्ज्ञायामगः} (पा॰सू॰~८.४.३) इत्यनेन णत्वे \textcolor{red}{अध्यात्म\-रामायणम्‌}। तस्मिन् \textcolor{red}{अध्यात्म\-रामायणे}। पाणिनिना प्रोक्ताः पाणिनीयाः। \textcolor{red}{तेन प्रोक्तम्‌} (पा॰सू॰~४.३.१०१) इत्यनेन तृतीयान्त\-पाणिनि\-शब्दाच्छप्रत्यये\footnote{\textcolor{red}{पाणिनि}\-शब्दस्य \textcolor{red}{वृद्धिर्यस्याचामादिस्तद्वृद्धम्} (पा॰सू॰~१.१.७३) इत्यनेन वृद्धत्वात् \textcolor{red}{वृद्धाच्छः} (पा॰सू॰~४.२.११४) इत्यनुसारं \textcolor{red}{तेन प्रोक्तम्‌} (पा॰सू॰~४.३.१०१) इत्यनेन \textcolor{red}{छ}\-प्रत्ययः। अवृद्धात्तु \textcolor{red}{तेन प्रोक्तम्‌} (पा॰सू॰~४.३.१०१) इत्यनेन \textcolor{red}{अण्‌}\-प्रत्ययः। यथा \textcolor{red}{चन्द्रेण प्रोक्तं चान्द्रम्}।} \textcolor{red}{आयनेयीनीयियः फढखच्छघां प्रत्ययादीनाम्‌} (पा॰सू॰~७.१.२) इत्यनेन \textcolor{red}{ईय्‌}\-आदेशे भत्वात्पाणिनि\-घटकेकार\-लोपे\footnote{\textcolor{red}{यचि भम्‌} (पा॰सू॰~१.४.१८) इत्यनेन भत्वम्। \textcolor{red}{यस्येति च} (पा॰सू॰~६.४.१४८) इत्यनेनेकार\-लोपः। \textcolor{red}{सुपो धातु\-प्रातिपदिकयोः} (पा॰सू॰~२.४.७१) इत्यनेन तृतीया\-विभक्ति\-लोपः।} \textcolor{red}{जस्‌} विभक्तौ \textcolor{red}{पाणिनीयाः}। \textcolor{red}{न पाणिनीया इत्यपाणिनीया} इति नञ्समासः।\footnote{\textcolor{red}{नञ्‌} (पा॰सू॰~२.२.६) इत्यनेन समासो \textcolor{red}{नलोपो नञः} (पा॰सू॰~६.३.७३) इत्यनेन नकार\-लोपश्च।} \textcolor{red}{प्रकर्षेण युज्यन्त इति प्रयोगाः} इति प्रपूर्वो \textcolor{red}{युज्‌}\-धातोः (\textcolor{red}{युजिँर् योगे} धा॰पा॰~१४४४)
कर्मणि \textcolor{red}{घञ्‌}।\footnote{\textcolor{red}{अकर्तरि च कारके सञ्ज्ञायाम्} (पा॰सू॰~३.३.१९) इत्यनेन।} \textcolor{red}{प्रकर्षेण युज्यतेऽर्थो यैः} वेति करणे \textcolor{red}{घञ्‌}।\footnote{सोऽपि \textcolor{red}{अकर्तरि च कारके सञ्ज्ञायाम्} (पा॰सू॰~३.३.१९) इत्यनेन। \textcolor{red}{प्रयोग। पु०। प्र~{\englishfont +}~युज् भावकर्मकरणेषु यथायथं घञ्‌} इति वाचस्पत्यम्।} ततः \textcolor{red}{लशक्वतद्धिते} (पा॰सू॰~१.३.८) इति सूत्रेणेत्सञ्ज्ञायां लोपे ञकारस्य चानुबन्धकार्ये च \textcolor{red}{पुगन्तलघूपधस्य च} (पा॰सू॰~७.३.८६) इत्यनेन गुणे \textcolor{red}{चजोः कु घिण्ण्यतोः} (पा॰सू॰~७.३.५३) इत्यनेन कुत्वे जस्विभक्तौ \textcolor{red}{प्रयोगाः}। ततः \textcolor{red}{अपाणिनीयाश्चामी प्रयोगा इत्यपाणिनीयप्रयोगाः} इति कर्मधारय\-समासः। पुनः विशेषेण मृश्यत इति \textcolor{red}{विमर्शः}। वि\-पूर्वक\-\textcolor{red}{मृश्‌}\-धातोः (\textcolor{red}{मृशँ आमर्शने} धा॰पा॰~१४२५) भावे घञि\footnote{\textcolor{red}{भावे} (पा॰सू॰~३.३.१८) इत्यनेन।} पुनरनुबन्ध\-कार्ये गुणे\footnote{\textcolor{red}{पुगन्त\-लघूपधस्य च} (पा॰सू॰~७.३.८६) इत्यनेन।} रपरत्वे\footnote{\textcolor{red}{उरण् रपरः} (पा॰सू॰~१.१.५१) इत्यनेन।} विभक्ति\-कार्ये \textcolor{red}{विमर्शः}।\footnote{वि~\textcolor{red}{मृशँ आमर्शने} (धा॰पा॰~१४२५)~\arrow वि~मृश्~\arrow \textcolor{red}{भावे} (पा॰सू॰~३.३.१८)~\arrow वि~मृश्~घञ्~\arrow वि~मृश्~अ~\arrow \textcolor{red}{पुगन्त\-लघूपधस्य} (पा॰सू॰~७.३.८६)~\arrow \textcolor{red}{उरण् रपरः} (पा॰सू॰~१.१.५१)~\arrow वि~मर्श्~अ~\arrow विमर्श~\arrow विभक्तिकार्यम्~\arrow विमर्शः।} पुनः \textcolor{red}{कृदतिङ्‌} (पा॰सू॰~३.१.९३) इत्यनेन घञ्प्रत्ययत्वे कृत्सञ्ज्ञायां ततः \textcolor{red}{कर्तृ\-कर्मणोः कृति} (पा॰सू॰~२.३.६५) इत्यनेन कर्मणि षष्ठी। \textcolor{red}{अपाणिनीय\-प्रयोग}\-पदादामि नुडागमे दीर्घे णत्वे \textcolor{red}{अपाणिनीय\-प्रयोगाणाम्‌}।\footnote{अपाणिनीय\-प्रयोग~आम्~\arrow \textcolor{red}{ह्रस्वनद्यापो नुट्‌} (पा॰सू॰~७.१.५४)~\arrow \textcolor{red}{आद्यन्तौ टकितौ} (पा॰सू॰~१.१.४६)~\arrow अपाणिनीय\-प्रयोग~नुँट्~आम्~\arrow अपाणिनीय\-प्रयोग~न्~आम्~\arrow अपाणिनीय\-प्रयोग~नाम्~\arrow \textcolor{red}{नामि} (पा॰सू॰~६.४.३)~\arrow अपाणिनीय\-प्रयोगा~नाम्~\arrow \textcolor{red}{अट्कुप्वाङ्नुम्व्यवायेऽपि} (पा॰सू॰~८.४.२)~\arrow अपाणिनीय\-प्रयोगा~णाम्~\arrow अपाणिनीय\-प्रयोगाणाम्।} एवम् \textcolor{red}{अपाणिनीय\-प्रयोगाणां विमर्शः} इति। एवमध्यात्म\-रामायणोत्तर\-सप्तम्या वैषयिकतयाऽनुयोगिता\-रूपेऽर्थे स्वीकृते निष्ठा\-रूपे वाऽर्थे स्वीकृते प्रयोग\-पदोत्तर\-षष्ठ्याः कर्मता\-रूपेऽर्थेऽङ्गीकृते प्रयोग\-शब्दस्य निरूपितत्व\-सम्बन्धेनान्वयेऽध्यात्म\-रामायणानुयोगितावच्छेदक\-पाणिनीय\-प्रयोग\-भिन्न\-प्रयोग\-निष्ठ\-निरूपित\-कर्मतावच्छेदक\-विमर्शोऽथवाऽध्यात्म\-रामायण\-निष्ठतावच्छेदक\-पाणिनीय\-प्रयोग\-भिन्न\-प्रयोग\-निष्ठ\-निरूपित\-कर्मतावच्छेदक\-विमर्श इत्यर्थं प्रयच्छति \textcolor{red}{अध्यात्म\-रामायणेऽपाणिनीय\-प्रयोगाणां विमर्शः}। शोध\-प्रबन्धेऽस्मिन्नध्यात्म\-रामायणे बाल\-बुद्ध्याऽपाणिनीयताप्रतीतानां शब्दानां प्रायः सन्धि\-कारक\-समास\-लिङ्ग\-कृत्तद्धितान्त\-धातु\-सम्बन्धिनां विमर्शं कर्तुमहं निर्दिष्टः। अत्र त्रयोऽध्यायाः। प्रत्येकमध्याये द्वौ द्वौ परिच्छेदौ कल्पितौ। साम्प्रतं प्रकृते प्रथमाध्याये सन्धि\-कारक\-समास\-सम्बन्धिनोऽध्यात्म\-रामायणीया अपाणिनीयाः प्रयोगा विमृश्यन्ते।\end{sloppypar}
\vspace{4mm}
\pdfbookmark[1]{प्रथमः परिच्छेदः}{Chap1Part1}
\phantomsection
\addtocontents{toc}{\protect\setcounter{tocdepth}{1}}
\addcontentsline{toc}{section}{प्रथमः परिच्छेदः}
\addtocontents{toc}{\protect\setcounter{tocdepth}{0}}
\centering ॥ अथ प्रथमाध्याये प्रथमः परिच्छेदः ॥\nopagebreak\\
\vspace{4mm}
\pdfbookmark[2]{बालकाण्डम्‌}{Chap1Part1Kanda1}
\phantomsection
\addtocontents{toc}{\protect\setcounter{tocdepth}{2}}
\addcontentsline{toc}{subsection}{बालकाण्डीयप्रयोगाणां विमर्शः}
\addtocontents{toc}{\protect\setcounter{tocdepth}{0}}
\centering ॥ अथ बालकाण्डीयप्रयोगाणां विमर्शः ॥\nopagebreak\\
\section[जगताम्]{जगताम्‌}
\centering\textcolor{blue}{यः पृथिवीभरवारणाय दिविजैः सम्प्रार्थितश्चिन्मयः\nopagebreak\\
सञ्जातः पृथिवीतले रविकुले मायामनुष्योऽव्ययः।\nopagebreak\\
निश्चक्रं हतराक्षसः पुनरगाद्ब्रह्मत्वमाद्यं स्थिरां\nopagebreak\\
कीर्तिं पापहरां विधाय जगतां तं जानकीशं भजे॥}\nopagebreak\\
\raggedleft{–~अ॰रा॰~१.१.१}\\
\begin{sloppypar}\hyphenrules{nohyphenation}\justifying\noindent\hspace{10mm} अयं प्रयोगोऽध्यात्म\-रामायणस्य बाल\-काण्डस्य प्रथम\-सर्गस्य प्रथमे मङ्गलाचरणात्मके श्लोके \textcolor{red}{कीर्तिं पाप\-हरां विधाय जगताम्‌} इति चतुर्थ\-चरणांश उद्धृतः। निर्विघ्न\-ग्रन्थ\-समाप्तये शिवो नमस्कारात्मकं मङ्गलमाचरन् श्रीरामं स्तौति यद्भू\-भार\-हरणाय देवैः प्रार्थितो यः श्रीरामो भूतले रघु\-कुलेऽवतीर्य राक्षसान्निहत्य जगत्सु पाप\-हरां कीर्तिं व्यवस्थाप्य पुनो ब्रह्मत्वमगमत्तमेव जानकीशमहं वन्दे। अत्र \textcolor{red}{कीर्तिं पाप\-हरां विधाय} इत्यत्र प्रयुक्तो \textcolor{red}{ल्यप्‌}\-प्रत्ययान्तो वि\-पूर्वक\textcolor{red}{डुधाञ्‌}\-धातुर्धारणार्थः। यद्यपि \textcolor{red}{उपसर्गेण धात्वर्थो बलादन्यत्र नीयते। प्रहाराहार\-संहार\-विहार\-परिहारवत्॥}\footnote{मूलं मृग्यम्।} इति कारिकानुरोधेन \textcolor{red}{वि}\-उपसर्गात् \textcolor{red}{विधाय} इत्यस्य \textcolor{red}{कृत्वा} इत्यनेनार्थेन भवितव्यं किन्तु प्रकृते करण\-रूपस्यार्थस्योप\-योगो नास्त्यतोऽयं धातुरन्तर्भावित\-ण्यर्थो \textcolor{red}{विधाप्य} इत्यर्थ\-सूचकः स्वीकरणीयः। एवं \textcolor{red}{विधाय} इत्यस्य व्यवस्थाप्येत्यस्मिन्नर्थे व्यवस्थापनस्याधारे सम्भवात् \textcolor{red}{जगताम्‌} इत्यत्राधिकरण\-बोधिका सप्तम्युचिता। करणार्थेऽपि स्वीकृते करोतेश्चोत्पत्त्यर्थतयाऽत्र
सप्तम्येवोचिता। अतो \textcolor{red}{जगताम्‌} इति षष्ठ्यन्त\-प्रयोगोऽपाणिनीय इति। किन्तु विमर्शे कृत इदमपि पाणिनीय\-सिद्धान्तानुरूपम्। \textcolor{red}{विवक्षाधीनानि कारकाणि भवन्ति} इति हि भाष्य\-वचनम्।\footnote{मूलं विविध\-भाष्य\-संस्करणेषु मृग्यम्। यद्वा \textcolor{red}{कर्मादीनामविवक्षा शेषः} (भा॰पा॰सू॰~२.३.५०, २.३.५२, २.३.६७) इत्यस्य तात्पर्यमिदम्।} \textcolor{red}{विवक्षा नाम श्रोताऽर्थं बुध्येतेति वक्तुर्वक्तुमिच्छा}। अत्र 
सम्बन्ध\-विवक्षया षष्ठी। यतो हि भगवतो रामस्य कीर्तिः शाश्वत्यतस्तस्याः संसारेण सह शाश्वतः सम्बन्धः। अतः सप्तम्यपेक्षया सम्बन्ध\-विवक्षायां \textcolor{red}{षष्ठी शेषे} (पा॰सू॰~२.३.५०) इति षष्ठी विभक्तिर्वरीयसी। बहुवचनं च चतुर्दशानां भुवनानामित्यभिप्रायेण। इत्थं \textcolor{red}{जगताम्‌} इति प्रयोगः पाणिनि\-सिद्धान्तानुकूलः।\end{sloppypar}
\section[अध्यात्मरामगङ्गा]{अध्यात्मरामगङ्गा}
\centering\textcolor{blue}{पुरारिगिरिसम्भूता श्रीरामार्णवसङ्गता।\nopagebreak\\
अध्यात्मरामगङ्गेयं पुनाति भुवनत्रयम्॥}\nopagebreak\\
\raggedleft{–~अ॰रा॰~१.१.५}\\
\begin{sloppypar}\hyphenrules{nohyphenation}\justifying\noindent\hspace{10mm} अयं प्रयोगोऽध्यात्म\-रामायणस्य बाल\-काण्डस्य प्रथमे सर्गे ग्रन्थ\-प्रशंसायां कृतो वर्तते। रूपक\-विधयाऽध्यात्म\-रामायणं गङ्गात्वेन संस्मृतम्। अध्यात्म\-रामायण\-गङ्गा शङ्कर\-हिमालयात्प्रादुर्भूय श्रीराम\-रूपेण सागरेण सङ्गम्य सकल\-भुवनानि पुनातीति तात्पर्यम्। \textcolor{red}{अध्यात्म\-रामायण\-गङ्गा} इति वक्तव्ये \textcolor{red}{अध्यात्म\-राम\-गङ्गा} इत्युक्तम्। अध्यात्म\-रामायण\-गङ्गेत्यस्याध्यात्म\-राम\-गङ्गेति नैवार्थ\-बोधे समर्थः शब्दः।
पाणिनि\-मते शब्दार्थयोर्वाच्य\-वाचक\-भावः।\footnote{\textcolor{red}{तस्मात्पद\-पदार्थयोः सम्बन्धान्तरमेव शक्तिर्वाच्य\-वाचक\-भावापर\-पर्याया} (प॰ल॰म॰~१०)।} राम\-शब्दस्तद्वाच्यं दाशरथि\-नियतमर्थं बोधयिष्यति किन्त्वेकाक्षर\-कोषं विना मकारोऽकारो रकारो वा न तद्बोधयितुं समर्थः। अस्मन्मते वर्ण\-स्फोटस्य गतेवोपयोगिता। \textcolor{red}{वाक्य\-स्फोटोऽतिनिष्कर्षे तिष्ठतीति मत\-स्थितिः} (वै॰सि॰का॰~५९) इति प्राचीनोक्तेः। किं बहुना \textcolor{red}{उच्चारित एव शब्दः प्रत्यायको भवति नानुच्चारितः}\footnote{मूलं विविध\-भाष्य\-संस्करणेषु मृग्यम्। शब्द\-रत्ने च \textcolor{red}{इको यणचि} सूत्रे~– \textcolor{red}{ननु व्यक्तिपक्ष इक्पदोप\-स्थाप्योकारादिभिर्दीर्घादि\-ग्रहणं न स्यात् ‘उच्चारित एव शब्दः प्रत्यायको नानुच्चारितः’ इति ‘अणुदित्’ सूत्रे भाष्योक्तेः} (श॰र॰~४७)। यद्वा \textcolor{red}{उच्चार्यमाणः शब्दः सम्प्रत्यायको भवति न सम्प्रतीयमानः} (भा॰पा॰सू॰~१.१.६९) इत्यस्य तात्पर्यमिदम्।} इति भाष्य\-वचनादप्यध्यात्म\-राम\-गङ्गा\-शब्दोऽध्यात्म\-रामायण\-गङ्गार्थं कथं बोधयिष्यतीति चेत्। उच्यते। अत्राध्यात्म\-रामायण\-शब्देऽध्यात्म\-राम\-शब्दस्य लक्षणा। यदि चेत्सा नैव वैयाकरणैरङ्गीकृता लक्षणा\-खण्डनं विस्तरशो नागोजिभट्ट\-विरचित\-वैयाकरण\-सिद्धान्त\-लघु\-मञ्जूषायां विलसितं तदा शक्यतावच्छेदकता स्वीक्रियताम्। अथवा \textcolor{red}{विनाऽपि प्रत्ययं पूर्वोत्तर\-पद\-लोपो वक्तव्यः} (वा॰~५.३.८३)। यथा \textcolor{red}{सत्या भामा सत्यभामा भामा सत्या} इत्यादि।\footnote{\textcolor{red}{अथवा पूर्वपदलोपोऽत्र द्रष्टव्यः – अत्यन्तसिद्धः सिद्ध इति। तद्यथा देवदत्तो दत्तः सत्यभामा भामेति॥} (भा॰प॰)। \textcolor{red}{अथवा दृश्यन्ते हि वाक्येषु वाक्यैकदेशान् प्रयुञ्जानाः पदेषु च पदैकदेशान्। वाक्येषु तावद्वाक्यैकदेशान् – प्रविश पिण्डीं प्रविश तर्पर्णम्। पदेषु पदैकदेशान् – देवदत्तः दत्तः सत्यभामा भामेति} (भा॰पा॰सू॰~१.१.४५)। तत्रत्या श्रीभार्गव\-शास्त्रिणष्टिप्पणी – \textcolor{red}{अत्र वाक्यैकदेशाश्चत्वार उदाहृता इति बहुवचनमुपपद्यते। पदैकदेशाश्च द्वावेव प्रदर्शितौ। ‘देवः सत्या’ इति नोदाहृतावपि ज्ञेयाविति बहुवचनोपपत्तिः} (पाणिनीय\-व्याकरण\-महाभाष्यम्, प्रथमः खण्डः, चौखम्भा संस्कृत प्रतिष्ठान, दिल्ली, १९८७, ३९४तमे पृष्ठे)।} अनेन नियमेनायन\-शब्दस्य लोपः। तस्मादध्यात्म\-राम\-गङ्गा\-शब्दोऽध्यात्म\-रामायण\-गङ्गार्थ\-परो लुप्तेऽप्ययन\-शब्दे तदर्थ\-बोधकत्वात्।\footnote{\textcolor{red}{यः शिष्यते स लुप्यमानार्थाभिधायी} इति नियमात्।}\end{sloppypar}
\section[भक्तेषु]{भक्तेषु}
\centering\textcolor{blue}{गोप्यं यदत्यन्तमनन्यवाच्यं वदन्ति भक्तेषु महानुभावाः।\nopagebreak\\
तदप्यहोऽहं तव देव भक्ता प्रियोऽसि मे त्वं वद यत्तु पृष्टम्॥}\nopagebreak\\
\raggedleft{–~अ॰रा॰~१.१.८}\\
\begin{sloppypar}\hyphenrules{nohyphenation}\justifying\noindent\hspace{10mm}\noindent\hspace{10mm} एष प्रयोगोऽपि बाल\-काण्डस्य प्रथम\-सर्ग एव। पार्वती शिवं प्रत्यकथयद्यन्महा\-नुभावा अत्यन्त\-गोप्यमपि भक्तेषु वदन्ति। अत्र \textcolor{red}{भक्तेषु} इति सप्तमी चिन्त्या। यतो हि सप्तम्यधिकरणे भवति \textcolor{red}{सप्तम्यधिकरणे च} (पा॰सू॰~२.३.३६) इति सूत्रेण। अधिकरणं ह्याधारस्य सञ्ज्ञा \textcolor{red}{आधारोऽधिकरणम्‌} (पा॰सू॰~१.४.४५) इति सूत्रात्। अत्राऽधारस्य सम्भावनैव नास्ति।\footnote{पूर्वपक्षोऽयम्।} तथा चात्र \textcolor{red}{अकथितं च} (पा॰सू॰~१.४.५१) इत्यनेन कर्म\-सञ्ज्ञैवोचिता। एवं \textcolor{red}{भक्तेषु} इति सप्तम्या आधारः पाणिनि\-विरुद्ध इव भाति। परं विचारे कृतेऽविरुद्धमेतत्। \textcolor{red}{अकथितं च} (पा॰सू॰~१.४.५१) इत्यस्यार्थो हि सिद्धान्त\-कौमुद्यां कारक\-प्रकरणे लिखितो भट्टोजिदिक्षित\-महाभागैर्यत् \textcolor{red}{अपादानादि\-विशेषैरविवक्षितं कारकं कर्म\-सञ्ज्ञं स्यात्‌} (वै॰सि॰कौ॰~५३९)। तत्र षोडश\-धातूनां परिगणनं कारिकायामकारि~–\end{sloppypar}
\centering\textcolor{red}{दुह्याच्पच्दण्ड्रुधिप्रच्छिचिब्रूशासुजिमथ्मुषाम्।\nopagebreak\\
कर्मयुक्स्यादकथितं तथा स्यान्नीहृकृष्वहाम्॥}\nopagebreak\\
\raggedleft{–~वै॰सि॰कौ॰~५३९}\\
\begin{sloppypar}\hyphenrules{nohyphenation}\justifying\noindent इति। \textcolor{red}{अर्थ\-निबन्धनेयं सञ्ज्ञा} (वै॰सि॰कौ॰~५३९) इति नियमेनापि परिगणित\-धातु\-समानार्थकानामपि सङ्ग्रहो यथा \textcolor{red}{बलिं भिक्षते वसुधाम्‌}। अत्र हि \textcolor{red}{भिक्ष्‌}\-धातुः (\textcolor{red}{भिक्षँ भिक्षायामलाभे लाभे च} धा॰पा॰~६०६) कारिका\-परिगणित\-याच्समानार्थकः। तेनापादानेनाविवक्षितस्य बलिरित्यस्य कर्म\-सञ्ज्ञा। तथैवात्रापि धातुर्ब्रू\-समानार्थः। तस्मात् \textcolor{red}{वदन्ति} इत्यस्य योगेन \textcolor{red}{भक्तेषु} इत्यत्र द्वितीयया भवितव्यमासीत्। किन्तु यदाऽपादानादिभिरविवक्षा तदाऽयं नियम इत्येव \textcolor{red}{अकथित}\-शब्दाज्ज्ञायते। अत्र तु वैषयिकस्याधारस्य सम्भावनयाऽधिकरण\-कारकस्य विवक्षैव। अर्थाद्भक्त\-विषये वदन्ति। विषयता चात्रोपस्थिति\-रूपा। यद्वा संस्थेषु विद्यमानेषु वेत्यध्याहार्यम्। एवं च भक्तेषु विद्यमानेषु वदन्तीति निर्गलितम्। पश्चात् \textcolor{red}{गम्यमानाऽपि क्रिया कारक\-विभक्तौ प्रयोजिका} (वै॰सि॰कौ॰~५६८) इति नियमेन \textcolor{red}{गोषु दुह्यमानासु गतः} इतिवदत्रापि \textcolor{red}{यस्य च भावेन भाव\-लक्षणम्‌} (पा॰सू॰~२.३.३७) इति सूत्रेण सप्तमी। इति पाणिन्यनुकूलम्।\end{sloppypar}
\section[मे]{मे}
\centering\textcolor{blue}{अत्रोत्तरं किं विदितं भवद्भिस्तद्ब्रूत मे संशयभेदि वाक्यम्॥}\nopagebreak\\
\raggedleft{–~अ॰रा॰~१.१.१५}\\
\begin{sloppypar}\hyphenrules{nohyphenation}\justifying\noindent\hspace{10mm} अयं प्रयोगोऽपि बाल\-काण्डस्य प्रथम\-सर्गीय एव। अत्र पार्वती भगवती श्रीशिवं प्रार्थयमाना कथयति यद्यदुत्तरं ज्ञातं संशय\-भेदि वाक्यं तन्मे ब्रूतेति। \textcolor{red}{मे} इति मह्यं मम वेत्यस्याऽदेशः। अत्र द्वितीया\-स्थाने चतुर्थी\-प्रयोगः षष्ठी\-प्रयोगो वा पाणिनीयं विरुणद्धीव परं विचारे कृतेऽविरोधः। मह्यमित्यत्र \textcolor{red}{क्रियार्थोपपदस्य च कर्मणि स्थानिनः} (पा॰सू॰~२.३.१४) इत्यनेन चतुर्थी। \textcolor{red}{नमस्कुर्मो नृसिंहाय} इतिवत्। मामनुकूलयितुं मां बोधयितुं वा ब्रुवन्तु। तत्राप्रयुज्यमान\-धातु\-कर्म\-भूते \textcolor{red}{माम्‌} इत्यर्थे चतुर्थी तस्य च \textcolor{red}{मे} इत्यादेशः। यद्वा कर्मणोऽपि सम्बन्ध\-विवक्षायां षष्ठी \textcolor{red}{मातुः स्मरति} इतिवत्। अत्रापि \textcolor{red}{माम्‌} इति कर्मणः सम्बन्ध\-विवक्षा तस्मात्षष्ठी \textcolor{red}{मम} इति तस्य च \textcolor{red}{मे} इत्यादेश\footnote{\textcolor{red}{तेमयावेकवचनस्य} (पा॰सू॰~८.१.२२) इत्यनेन।} इति द्वितीयः कल्पः। अथवा \textcolor{red}{मे} इत्यस्य \textcolor{red}{संशय}\-शब्देनान्वयः। अर्थान्मत्प्रतियोगिक\-संशयस्य भेदकं वाक्यं मत्सम्बन्धि\-संशयस्येति तात्पर्यम्। सम्बन्धे षष्ठी। तत्र पार्वती प्रतियोगी संशयश्चानुयोगी विषयि\-विषय\-भाव\-सम्बन्धः। यदि चेदाशङ्का स्यात् \textcolor{red}{संशय}\-शब्दः समस्तः स च \textcolor{red}{मे}\-शब्देन सापेक्षः स च \textcolor{red}{सापेक्षः असमर्थवत्‌} इति न्यायेनासमर्थः सन् कथं समस्येदिति चेन्नित्य\-सापेक्षा\-स्थले नैवास्य नियमस्य प्रसरः \textcolor{red}{देवदत्तस्य गुरुकुलम्‌} इतिवत्। अतोऽत्र सुतरां षष्ठी।\end{sloppypar}
\section[ते कथयिष्यामि]{ते कथयिष्यामि}
\centering\textcolor{blue}{अत्र ते कथयिष्यामि रहस्यमपि दुर्लभम्।\nopagebreak\\
सीताराममरुत्सूनुसंवादं मोक्षसाधनम्॥}\nopagebreak\\
\raggedleft{–~अ॰रा॰~१.१.२५}\\
\begin{sloppypar}\hyphenrules{nohyphenation}\justifying\noindent\hspace{10mm} अत्र \textcolor{red}{त्वां कथयिष्यामि} इत्युचितम्। \textcolor{red}{कथ्‌}\-धातोः (\textcolor{red}{कथँ वाक्य\-प्रबन्धने} धा॰पा॰~१८५१) अकथित\-कर्मक\-परिगणित\-\textcolor{red}{ब्रू}\-धातोः (\textcolor{red}{ब्रूञ् व्यक्तायां वाचि} धा॰पा॰~१०४४) समानार्थकत्वात्। अतश्चतुर्थी वा षष्ठी वोभे अपि पाणिनीय\-विरुद्धे इव। \textcolor{red}{ते} इति \textcolor{red}{तुभ्यम्‌} इत्यस्य \textcolor{red}{तव} इत्यस्य वा विकरणम्।\footnote{\textcolor{red}{तेमयावेकवचनस्य} (पा॰सू॰~८.१.२२) इत्यनेन।} परमत्रोभे अपि साध्व्यौ। चतुर्थी तु \textcolor{red}{क्रियार्थोपपदस्य च कर्मणि स्थानिनः} (पा॰सू॰~२.३.१४) इत्यनेन साध्वी। अर्थात् \textcolor{red}{त्वां पार्वतीं प्रतिबोधयितुं कथयिष्यामि} अतश्चतुर्थी गम्यमान\-क्रियायाः प्रयोजकत्वात्। यद्वाऽत्र \textcolor{red}{हित}\-शब्दोऽध्याहार्यस्तथाऽत्र \textcolor{red}{ते हिताय हितं वेति कथयिष्यामि} एवं \textcolor{red}{हित\-योगे च} (वा॰~२.३.१३) इति वार्त्तिकेन चतुर्थी। यद्वा
\textcolor{red}{क्रियया यमभिप्रैति सोऽपि सम्प्रदानम्‌} (वा॰~१.४.३२) इति वार्त्तिकेनात्र चतुर्थी \textcolor{red}{पत्ये शेते} इतिवत्। अत्र शङ्करः कथन\-क्रियया पार्वतीमभि\-प्रैत्यतोऽत्र चतुर्थी। यद्वा \textcolor{red}{तादर्थ्ये चतुर्थी वाच्या} (वा॰~२.३.१३) इत्यनेन \textcolor{red}{मुक्तये हरिं भजति} इतिवदत्र चतुर्थी। अथवा \textcolor{red}{मातुः स्मरति} इतिवत्कर्मणि सम्बन्ध\-विवक्षायां षष्ठी। अथवा सम्प्रदानेन सम्बन्धेन च विवक्षितत्वाद्द्वितीयाया अवसर एव न।\footnote{\textcolor{red}{अपादानादि\-विशेषैरविवक्षितं कारकं कर्मसञ्ज्ञं स्यात्‌} (वै॰सि॰कौ॰~५३९)।}\end{sloppypar}
\section[ब्रूहि तत्त्वं हनूमते]{ब्रूहि तत्त्वं हनूमते}
\centering\textcolor{blue}{रामः सीतामुवाचेदं ब्रूहि तत्त्वं हनूमते।\nopagebreak\\
निष्कल्मषोऽयं ज्ञानस्य पात्रं नौ नित्यभक्तिमान्॥}\nopagebreak\\
\raggedleft{–~अ॰रा॰~१.१.३०}\\
\begin{sloppypar}\hyphenrules{nohyphenation}\justifying\noindent\hspace{10mm} अत्रापि स्पष्टं \textcolor{red}{ब्रू}\-धातुः (\textcolor{red}{ब्रूञ् व्यक्तायां वाचि} धा॰पा॰~१०४४)। स चाकथित\-कर्मक\-धातु\-गणना\-सूचक\-कारिकायां प्रामुख्येन गणितः। अतोऽत्र तु द्वितीया दुर्वारैव। \textcolor{red}{ब्रूहि तत्त्वं हनूमन्तम्‌} इत्युचितम्। चतुर्थ्यपाणिनीयेव। अत्र विमृश्यते। यदाऽपादानादिभिरभिधेयैरविवक्षितं सत्कारकं परिगणित\-धातुभिः सह युज्येत तदा कर्म\-सञ्ज्ञम्।\footnote{\textcolor{red}{अपादानादि\-विशेषैरविवक्षितं कारकं कर्म\-सञ्ज्ञं स्यात्‌} (वै॰सि॰कौ॰~५३९)।} इदं तु सम्प्रदानेन विवक्षितम्। सम्प्रदानादयश्च बुद्धिकृताः। अत एव \textcolor{red}{ध्रुवमपायेऽपादानम्‌} (पा॰सू॰~१.४.२४) इत्येव सूत्रं व्यवस्थाप्येतः परमपादान\-सञ्ज्ञा\-सूत्राणि सर्वाण्यपि बुद्धि\-कृतमपादानं कल्पयित्वा भाष्य\-कृता प्रत्याख्यातानि।\footnote{\textcolor{red}{अयं योगः शक्योऽवक्तुम्। ... स बुद्ध्या सम्प्राप्य निवर्तयति। तत्र “ध्रुवमपायेऽपादानम्” इत्येव सिद्धम्‌} (भा॰पा॰सू॰~१.४.२५)। \textcolor{red}{अयमपि योगः शक्योऽवक्तुम्। ... स बुद्ध्या सम्प्राप्य निवर्तते। तत्र “ध्रुवमपायेऽपादानम्” इत्येव सिद्धम्‌} (भा॰पा॰सू॰~१.४.२६)। \textcolor{red}{अयमपि योगः शक्योऽवक्तुम्। ... स बुद्ध्या सम्प्राप्य निवर्तयति। तत्र “ध्रुवमपायेऽपादानम्” इत्येव सिद्धम्‌} (भा॰पा॰सू॰~१.४.२७)। \textcolor{red}{अयमपि योगः शक्योऽवक्तुम्। ... स बुद्ध्या सम्प्राप्य निवर्तते। तत्र “ध्रुवमपायेऽपादानम्” इत्येव सिद्धम्‌} (भा॰पा॰सू॰~१.४.२८)। \textcolor{red}{अयमपि योगः शक्योऽवक्तुम्‌} (भा॰पा॰सू॰~१.४.२९)। \textcolor{red}{अयमपि योगः शक्योऽवक्तुम्‌} (भा॰पा॰सू॰~१.४.३०)। \textcolor{red}{अयमपि योगः शक्योऽवक्तुम्‌} (भा॰पा॰सू॰~१.४.३१)।} तस्मादत्रापि सम्प्रदानं विवक्षितमेव कल्प्यताम्। महा\-सञ्ज्ञानां प्रायो लक्ष्यानुरूपोऽर्थो भवत्येवान्यथा लाघव\-प्रियः पाणिनिः \textcolor{red}{घु} \textcolor{red}{टि} \textcolor{red}{घि} इत्यादि सञ्ज्ञा इवापादानाधिकरण\-सम्प्रदान\-सञ्ज्ञा अपि लघ्वीः कुर्यात्। तस्मान्महा\-सञ्ज्ञा\-करणादासां व्यवस्थितोऽर्थः। स च लक्ष्योपयोगी। प्रकृते \textcolor{red}{सम्यक्प्रकर्षेण दीयते यस्मै स सम्प्रदानम्} इति विग्रहे \textcolor{red}{सम्प्र}\-पूर्वक\-\textcolor{red}{दा}\-धातोः (\textcolor{red}{डुदाञ् दाने} धा॰पा॰~१०९१) \textcolor{red}{कृत्य\-ल्युटो बहुलम्‌} (पा॰सू॰~३.३.११३) इत्यनेन \textcolor{red}{दानीयो विप्रः} (ल॰सि॰कौ॰~७७२) इतिवत्सम्प्रदाने ल्युट्। अनुबन्ध\-कार्येऽनादेशे सम्प्रदानमिति सिद्ध्यति।\footnote{सम्~प्र~दा~\arrow \textcolor{red}{कृत्य\-ल्युटो बहुलम्‌} (पा॰सू॰~३.३.११३)~\arrow सम्~प्र~दा~ल्युँट्~\arrow सम्~प्र~दा~युँ~\arrow \textcolor{red}{युवोरनाकौ} (पा॰सू॰~७.१.१)~\arrow सम्~प्र~दा~अन~\arrow \textcolor{red}{अकः सवर्णे दीर्घः} (पा॰सू॰~६.१.१०१)~\arrow सम्~प्र~दान~\arrow सम्प्रदान~\arrow विभक्ति\-कार्यम्~\arrow सम्प्रदानम्।} हनूमांश्चात्र तत्त्व\-जिज्ञासुतया कृत\-प्रश्नः प्रबोधयितुं सीतयोपक्रम्यते। श्रीरामेण च हनूमन्तमुपदेष्टुं सीता प्रेर्यतेऽत उपदेश\-दान\-क्रियाया उद्देश्यं श्रीहनुमान्। अत्रैव \textcolor{red}{दा}\-धातोर्वाच्य\-दानस्य सम्यक्त्वं प्रकृष्टत्वञ्च सञ्जाघट्यते श्रीमहावीरे। यथाऽग्रिम\-चरणे श्रीरामः समर्थयते \textcolor{red}{निष्कल्मषोऽयं ज्ञानस्य पात्रं नौ नित्यभक्तिमान्‌}। तस्माच्चतुर्थी। अथवा श्रीरामो हनूमते हितं चिकीर्षति चिकारयिषति च। हितं च राम\-रहस्य\-तत्त्व\-ज्ञानोपदेशादेव। अतो \textcolor{red}{हिताय} इति योजनीयं ततः \textcolor{red}{हित\-योगे च} (वा॰~२.३.१३) इति वार्त्तिकेन चतुर्थी। इति नापाणिनीयता। अथवा धातूनामनेकार्थत्वात् \textcolor{red}{ब्रू}धातोर्दानार्थता। अतः \textcolor{red}{ब्रूहि तत्त्वं हनूमते} इत्यस्य \textcolor{red}{देहि तत्त्वं हनूमते} इत्यर्थः। ततः \textcolor{red}{कर्मणा यमभिप्रैति स सम्प्रदानम्‌} (पा॰सू॰~१.४.३२) इत्यनेन हनूमतः सम्प्रदान\-सञ्ज्ञा। \textcolor{red}{चतुर्थी सम्प्रदाने} (पा॰सू॰~२.३.१३) इत्यनेन \textcolor{red}{हनूमते} इत्यत्र चतुर्थी।\end{sloppypar}
\section[मया]{मया}
\centering\textcolor{blue}{मत्पाणिग्रहणं पश्चाद्भार्गवस्य मदक्षयः।\nopagebreak\\
अयोध्यानगरे वासो मया द्वादशवार्षिकः॥}\nopagebreak\\
\raggedleft{–~अ॰रा॰~१.१.३७}\\
\begin{sloppypar}\hyphenrules{nohyphenation}\justifying\noindent\hspace{10mm} अत्र \textcolor{red}{मया} इति प्रयोगो विभाव्यते। \textcolor{red}{वास}\-शब्दो भाव\-घञन्तः।\footnote{\textcolor{red}{वसँ निवासे} (धा॰पा॰~१००४) इति धातोः \textcolor{red}{भावे} (पा॰सू॰~३.३.१८) इत्यनेन घञ्। \textcolor{red}{अत उपधायाः} (पा॰सू॰~७.२.११६) इत्यनेनोपधावृद्धिः।} \textcolor{red}{घञ्‌}\-प्रत्ययश्च कृदन्तीयः।\footnote{\textcolor{red}{कृदतिङ्‌} (पा॰सू॰~३.१.९३) इत्यनेन। \textcolor{red}{कृदन्तीयः} इत्यत्र \textcolor{red}{तस्मै हितम्} (पा॰सू॰~५.१.५) इत्यनेन \textcolor{red}{छ}प्रत्ययः। \textcolor{red}{कृदन्तेभ्यो हितः} इति भावः।} तथा \textcolor{red}{कर्तृ\-कर्मणोः कृति} (पा॰सू॰~२.३.६५) इति सूत्रेण \textcolor{red}{मम} इति प्रसक्तम्। उच्यते। अत्र पूर्व\-प्रसङ्गे रामस्य लीला\-मञ्चे दृश्यमानं कर्तृत्वमौपचारिकमिति भगवती सीतोपक्रम्य निर्गुणे ब्रह्मणि रामचन्द्रे कर्तृत्वासम्भवं प्रदर्शयन्ती तत्तद्राम\-कर्तृक\-घटनासु राम\-कर्तृत्वाभासं मूल\-प्रकृतित्वात्स्वस्याः कर्तृत्वं व्यवस्थापयन्ती श्रीरघु\-नाथ\-कर्तृत्वं चाध्यारोपापवाद\-न्यायेन निराकरोति। अतोऽत्र सर्वासु घटनासु श्रीरामस्य कर्तृत्वारोपस्तस्माद्द्वादश\-वार्षिकस्यायोध्या\-वासस्य कर्ता श्रीराम एव। सीता च वास\-रूप\-क्रिया\-सिद्धौ प्रकृष्टोप\-कारकतया \textcolor{red}{साधकतमं करणम्‌} (पा॰सू॰~१.४.४२) इत्यनेन करण\-सञ्ज्ञा\-भाग्। अतः सीता\-बोधक उत्तम\-पुरुषैक\-वचन\-बोधकोऽस्मच्छब्दोऽपि \textcolor{red}{कर्तृ\-करणयोस्तृतीया} (पा॰सू॰~२.३.१८) इति सूत्रेण तृतीयामलभत। यद्वा \textcolor{red}{विनाऽपि तद्योगं तृतीया। वृद्धो यूनेत्यादिनिर्देशात्‌} (वै॰सि॰कौ॰~५६४) इति नियमेन सहशब्दाभावेऽपि तृतीया। यथा \textcolor{red}{वृद्धो यूना तल्लक्षणश्चेदेव विशेषः} (पा॰सू॰~१.२.६५) इति सूत्रे भगवान् पाणिनिः सहशब्दाभावेऽपि \textcolor{red}{यूना} इति निर्दिशति। तथैवात्रापि सह\-शब्द\-विरहेऽपि तृतीया। \textcolor{red}{मया सीतया सह रामस्य द्वादश\-वार्षिकोऽयोध्यायां वासः} इति न पाणिनि\-विरोधः।\end{sloppypar}
\section[विभीषणे राज्यदानम्]{विभीषणे राज्यदानम्‌}
\centering\textcolor{blue}{रावणस्य वधो युद्धे सपुत्रस्य दुरात्मनः।\nopagebreak\\
विभीषणे राज्यदानं पुष्पकेण मया सह॥}\nopagebreak\\
\raggedleft{–~अ॰रा॰~१.१.४१}\\
\begin{sloppypar}\hyphenrules{nohyphenation}\justifying\noindent\hspace{10mm} प्रयोगोऽयं बाल\-काण्डस्य प्रथम\-सर्गीय एव। अत्रापि \textcolor{red}{विभीषणाय राज्य\-दानम्‌} इत्यनेन भवितव्यमासीत्। यतो हि दान\-क्रियाया उद्देश्यता विभीषण एव। अस्माद्दान\-क्रियोद्देश्यस्य विभीषणस्य सम्प्रदानता निर्बाधैव। किन्त्वत्राधिकरण\-कारकम्। एवमपाणिनीयं प्रतीयते। परं विभीषण आधारत्वं परिकल्प्याधिकरण\-कारकमुचितम्। विभीषणो राम\-भक्तस्तस्य च भगवते सर्वस्व\-समर्पणम्। सम्प्रदानं हि \textcolor{red}{स्व\-स्वत्व\-निवृत्तिपूर्वकं पर\-स्वत्वोत्पादनम्‌} (त॰बो॰~५६९)। इदमत्र सम्भविष्यति नहि राम\-भक्तस्य सर्वथा स्वत्व\-हीनत्वात्। अतोऽत्र वैषयिक आधारो विभीषणः। आधारे चाधिकरण\-सञ्ज्ञा। अधिकरणे च सप्तमी निर्बाधा। \textcolor{red}{सप्तम्यधिकरणे च} (पा॰सू॰~२.३.३६) इत्यनेन। अतो \textcolor{red}{विभीषणे राज्यदानम्‌} सङ्गतम्। यद्वा \textcolor{red}{विवक्षाधीनानि कारकाणि भवन्ति}\footnote{मूलं मृग्यम्। यद्वा \textcolor{red}{कर्मादीनामविवक्षा शेषः} (भा॰पा॰सू॰~२.३.५०, २.३.५२, २.३.६७) इत्यस्य तात्पर्यमिदम्।} इति सिद्धान्तेनात्राधिकरणे विवक्षा। अथवाऽऽर्ष\-सिद्धान्त\-निरूपण\-क्रमे सूत्र\-वर्णन\-पराङ्गत्वादत्र दर्शन\-सूत्राणि वर्तन्ते। अतो यथा सूत्रे विभक्तीनां स्वातन्त्र्येण प्रयोगस्तथाऽत्रापि सीता सूत्र\-हेतुं सिद्धान्तं हनुमते कथयत्यतोऽत्र सौत्री सप्तमी। यथा पाणिनीयमधिकार\-सूत्रं \textcolor{red}{कारके} (पा॰सू॰~१.४.२३) इति। अत्र सप्तम्या उपयोगः। प्रथमार्थे सप्तमीति भाष्यकारा अपि स्वीकुर्वन्ति।\footnote{\textcolor{red}{किमिदं ‘कारके’ इति। सञ्ज्ञानिर्देशः} (भा॰पा॰सू॰~१.४.२३)। तत्र \textcolor{red}{सञ्ज्ञानिर्देश इति। सुपां सुपो भवन्तीति प्रथमायाः स्थाने सप्तमी कृतेति भावः} इति कैयटः।} तथैवात्रापि।\end{sloppypar}
\section[पूर्णेन एकत्वम्]{पूर्णेन एकत्वम्‌}
\centering\textcolor{blue}{अविच्छिन्नस्य पूर्णेन एकत्वं प्रतिपाद्यते।\nopagebreak\\
तत्त्वमस्यादिवाक्यैश्च साभासस्याहमस्तथा॥}\nopagebreak\\
\raggedleft{–~अ॰रा॰~१.१.४९}\\
\begin{sloppypar}\hyphenrules{nohyphenation}\justifying\noindent\hspace{10mm} अत्र तृतीया चिन्त्या। करणेऽपि तस्या असम्भवात्। यतो हि कारकाणि प्रायशः क्रियामेवाभ्यन्त्यनुयन्त्यत एव प्राचीनानां मते \textcolor{red}{क्रियान्वयित्वं कारकत्वम्‌} इत्येव सिद्धान्तः। यद्यपि \textcolor{red}{मातुः स्मरति} इत्यादावति\-व्याप्ति\-वारणाय \textcolor{red}{साक्षात्क्रियान्वयित्वं कारकत्वम्‌} इति प्राचीना आमनन्ति नवीनास्तु लक्षणेऽस्मिन् साक्षात्पद\-निवेशमसहमानाः \textcolor{red}{क्रिया\-जनकत्वं कारकत्वम्‌} इत्येव व्यवस्थापयन्ति। तथा च कृ\-धातुं (\textcolor{red}{डुकृञ् करणे} धा॰पा॰~१४७२) जन्यर्थकं मत्वा \textcolor{red}{करोति क्रियां निर्वर्तयतीति कारकम्‌} अस्मिन् विग्रहे कृ\-धातोः \textcolor{red}{ण्वुल्तृचौ} (पा॰सू॰~३.१.१३३) इति सूत्रेण \textcolor{red}{ण्वुल्‌}\-प्रत्ययः। \textcolor{red}{लशक्वतद्धिते} (पा॰सू॰~१.३.८) इत्यनेन लस्येत्सञ्ज्ञायां लोपे \textcolor{red}{चुटू} (पा॰सू॰~१.३.७) इत्यनेन णकारस्येत्सञ्ज्ञायां लोपे च \textcolor{red}{अचो ञ्णिति} (पा॰सू॰~७.२.११५) इत्यनेन वृद्धौ \textcolor{red}{उरण् रपरः} (पा॰सू॰~१.१.५१) इत्यनेन रपरत्वे \textcolor{red}{युवोरनाकौ} (पा॰सू॰~७.१.१) इत्यनेनाकादेशे विभक्ति\-कार्ये कारकमिति सिद्धम्। अतः पूर्वस्मिन् लक्षणे स्वीकृते क्रियान्वयित्वं कारकत्वमिति लक्षणस्य जागरूकतया \textcolor{red}{माणवकस्य पितरं पन्थानं पृच्छति} अत्र परम्परा\-सम्बन्धेन माणवकेनाऽपि क्रियान्वयः। तेनात्रापि षष्ठ्यन्ते कारकत्व\-प्रसङ्गः स्यात्। यद्यपि साक्षात्पद\-निवेशने न दोषस्तथाऽपि \textcolor{red}{मातुः स्मरति} इत्यादावपि स्थितिः सामान्या। अतो व्युत्पत्त्यनुरोधेन गौरवं चानुमाय क्रिया\-जनकत्वं कारकत्वं लक्षणं नवीनानां मते सार्वभौमतया सुस्थिरं परन्तु लक्षणेऽस्मिन् व्यवस्थितेऽपि सर्वेषां कारकाणां साक्षात्क्रिया\-जनकत्वाभावात्परम्परा\-पद\-निवेशोऽत्रापि प्राचीन\-लक्षण\-सम एव। क्रियां साक्षाद्रूपेण तु केवलं कारक\-द्वयं जनयति कर्ता कर्म च। तत्र व्यापार\-रूपां क्रियां कर्ता स्वाश्रयतयैवं फल\-रूपां क्रियां कर्म स्वानुकूलतया जनयति। शेषाणि कारकाणि यद्यप्यानुकूल्यमञ्चन्ति किन्त्वधिकरण\-कारकन्तु परम्परयैव व्यापारांशे कर्तारं सहायकं कृत्वा फलांशे च कर्म सहायकं मत्वा क्रियामुत्पादयति। अतः पक्ष\-द्वयेऽपि समान\-गौरव\-लाघवतया पूर्व\-लक्षणमेव स्फुटार्थतया श्रेयो लगति। अत एव प्रौढ\-मनोरमायामन्योऽन्याश्रय\-दोष\-परिहाराय दीक्षितेनायमेव सिद्धान्तो ध्वनितः। तत्र हीयं परिस्थितिः। अष्टाध्याय्यां प्रथमाध्याय इत्सञ्ज्ञा\-विधायक\-सूत्र\-सूत्रण\-प्रसङ्गे भगवान् पाणिनिः पपाठ \textcolor{red}{हलन्त्यम्‌} (पा॰सू॰~१.३.३) इति सूत्रम्। \textcolor{red}{वाक्यार्थ\-बोधे पदार्थ\-ज्ञानं कारणम्‌}\footnote{मूलं मृग्यम्।} अयं हि नियमः। \textcolor{red}{हलन्त्यम्‌} इति सूत्रस्य वाक्यार्थो ह्युपदेशेऽन्त्यं हलित्स्यात्। अत्र पदार्थ\-बोधोऽपि विचार्यताम्। \textcolor{red}{उपदेशे इत्‌} इति पद\-द्वितयम् \textcolor{red}{उपदेशेऽजनुनासिक इत्‌} (पा॰सू॰~१.३.२) अस्मात्सूत्रादनुवृत्तम्। \textcolor{red}{सूत्रेष्वदृष्टं पदं सूत्रान्तरादनुवर्तनीयं सर्वत्र} (ल॰सि॰कौ॰~१) इति वरदराजोक्तेः। एवं \textcolor{red}{हल्‌} \textcolor{red}{अन्त्यम्‌} इति मुख्ये सूत्रस्थे पदे। तत्र हल्शब्दस्य कोऽर्थ इति जिज्ञासायां हल्सञ्ज्ञा\-विधायकस्य \textcolor{red}{आदिरन्त्येन सहेता} (पा॰सू॰~१.१.७१) इति सूत्रस्य वाक्यार्थे विचारयिष्यमाणे तत्र हेतुत्वात्पदार्थ\-बोधस्य पूर्वं पदार्थ\-चिन्तनमनिवार्यतां गतम्। तत्र \textcolor{red}{आदिः} \textcolor{red}{अन्त्येन} \textcolor{red}{सह} \textcolor{red}{इता} इमानि चत्वारि पदानि। सूत्रेऽस्मिन्नेवं समुदाये सञ्ज्ञाया अनुपयोगादवयवेषु विश्रामः। तथा च \textcolor{red}{अन्त्येन} इत्यप्रधान\-तृतीया\-निर्देशान्मध्य\-वर्तिभिः सहाऽदि\-वाच्यस्यापि सङ्ग्रहः। तथा च \textcolor{red}{अन्त्येनेता सहित आदिर्मध्यगानां स्वस्य च सञ्ज्ञा स्यात्‌} (ल॰सि॰कौ॰~४)। अत्र \textcolor{red}{अन्त्य}\-पदस्य त्वर्थः सुस्पष्टः किन्तु \textcolor{red}{इत्‌}\-शब्दस्यार्थोऽनवगतः। स चेत्सञ्ज्ञा\-विधायकात् \textcolor{red}{हलन्त्यम्‌} (पा॰सू॰~१.३.३) इति सूत्रादवगन्तुं शक्यते। तदपि तदैवेत्पदार्थं बोधयिष्यति यदा तस्य हल्पदार्थो बोधितो भविष्यति। हल्पदार्थ\-बोधश्च तत्सञ्ज्ञा\-विधायक \textcolor{red}{आदिरन्त्येन सहेता} (पा॰सू॰~१.१.७१) इति सूत्राधीनः। स चेत्पदार्थ\-ज्ञानमन्तरेण हल्पदार्थं बोधयितुं शक्नोत्येव नहि। इत्थं हल्पदार्थ\-ज्ञानमित्पदार्थाधीनम्। इत्पदार्थ\-ज्ञानञ्च हल्पदार्थाधीनम्। अत एकैकमेकैकाधीनमित्येव परस्परापेक्षत्व\-रूपोऽन्योऽन्याश्रयः। अन्योऽन्याश्रयाणि कार्याणि न प्रकल्पन्ते। यथा नावि बद्धा नौर्नैव गतिशीला भवतीति भाष्ये\footnote{\textcolor{red}{इतरेतराश्रयाणि च कार्याणि न प्रकल्पन्ते। तद्यथा नौर्नावि बद्धा नेतरत्राणाय भवति} (भा॰पा॰सू॰~१.१.१)।} सिद्धान्तितत्वान्महान् विप्लवः समुपस्थितः। \textcolor{red}{अन्योऽन्याश्रयत्वं हि तद्ग्रह\-सापेक्ष\-ग्रह\-सापेक्ष\-ग्रह\-विषयत्वम्‌}। तद्ग्रहो हल्पदार्थ\-ग्रहस्तत्सापेक्ष\-ग्रह इद्ग्रहस्तत्सापेक्ष\-ग्रहो हल्ग्रहस्तद्विषयत्वमित्सञ्ज्ञा\-सूत्र एवमेव तद्ग्रह इद्ग्रहस्तद्विषयत्वं हल्सञ्ज्ञा\-सूत्र इत्युभयतस्पाशा रज्जुः। इत्थमन्योऽन्याश्रयमाशङ्क्य श्रीदीक्षितेन \textcolor{red}{हलन्त्यम्‌} इति सूत्रमेवावर्तितम्। विवेकोऽयं यन्मूल\-सूत्रमसमस्तम्। समासे कृते सति \textcolor{red}{अन्त्य}\-शब्दस्य विशेषणतया पूर्व\-निपातापत्तेः। किन्त्वावृत्त\-सूत्रं \textcolor{red}{हलन्त्यम्‌} इति समस्तम्। हल्शब्दस्यान्त्य\-शब्देन सह कः समासो भवेदित्येव विचारयितुमुपक्रान्तं वैयाकरण\-सिद्धान्त\-कौमुदी\-टीका\-प्रौढमनोरमा\-सञ्ज्ञा\-प्रकरणे। यथा प्रौढ\-मनोरमायाम्~–\end{sloppypar}
\centering\textcolor{red}{“हलि अन्त्यम्” इति विग्रहे “सप्तमी” (पा॰सू॰~२.१.४०) इति योग\-विभागात् “सुप्सुपा” (पा॰सू॰~२.१.४) इति वा समासः। यद्वा षष्ठी\-तत्पुरुषोऽयम्॥}\nopagebreak\\
\raggedleft{–~प्रौ॰म॰~१}\\
\begin{sloppypar}\hyphenrules{nohyphenation}\justifying\noindent इति। तत्र \textcolor{red}{सप्तमी} इति प्रतीकमादाय शब्दरत्ने व्याचक्षते श्रीहरिदीक्षित\-महाभागाः \textcolor{red}{अधिकरण\-कारकस्य कर्त्राद्यन्वय\-द्वारा क्रियान्वयादस्ति सामर्थ्यमिति भावः} (श॰र॰~१)। \textcolor{red}{सप्तमी शौण्डैः} (पा॰सू॰~२.१.४०) इति सूत्रे योग\-विभागं मत्वा सप्तमी\-समासो दर्शितः। तत्रायं पूर्वपक्षस्याक्षेपो यदधिकरण\-कारकं क्रियया नान्वेति। असति क्रियान्वये क्रियान्वय\-लक्षण\-कारकत्वस्याभावात्तस्मिन् कथं सामर्थ्यं सामर्थ्याभावे च कथं समास इति चेत्। हरिदीक्षितः कथयति यत्कर्तारं कर्म च द्वारीकृत्याधिकरण\-कारकं क्रियायामन्वेति क्रिया च ते एव माध्यमं कृत्वाऽधिकरण\-कारकेणान्वेति। यदाऽकर्मका क्रिया तदा कर्ता माध्यमो यदा च सकर्मिका तदा कर्म माध्यममित्यपि क्रियते। तद्यथा \textcolor{red}{रामः शय्यायां शेते} इत्यत्र स्व\-वृत्ति\-वृत्तित्व\-सम्बन्धेन क्रियाऽन्वेति। स्वं शय्या तद्वृत्ती रामस्तद्वृत्तिः शयनानुकूलो व्यापारः। एवमेव \textcolor{red}{स्थाल्यां तण्डुलं पचति} इत्यादौ स्वं स्थाली तत्र तद्वृत्तिस्तण्डुलस्तद्वृत्तिः पाकः। एवमेव क्रियाऽधिकरण\-कारकेन स्वाश्रयाश्रयत्व\-सम्बन्धेनान्वेति। यथा \textcolor{red}{सीता वाटिकायां वर्तते} इत्यत्र स्वं वर्तनानुकूला क्रिया तदाश्रयः सीता तदाश्रयश्च वाटिकेति।\end{sloppypar}
\begin{sloppypar}\hyphenrules{nohyphenation}\justifying\noindent\hspace{10mm} इत्थं क्रियान्वयित्व\-रूप\-कारकत्व\-सार्वजनीनत्वात् \textcolor{red}{पूर्णेन एकत्वम्‌} इत्येव साधयितुं नैव किमपि कारकं सङ्घटते। अस्मादत्रोपपद\-विभक्तिः। तस्मादुपपदं विना तृतीयाऽपाणिनीयेति चेत्। विभक्तिर्द्विधा कारक\-विभक्तिरुपपद\-विभक्तिश्च। क्रियामाश्रित्य जायमाना विभक्तिः कारक\-विभक्तिः। एवं पदमाश्रित्य जायमाना विभक्तिरुपपद\-विभक्तिः। कारक\-विभक्तिर्यथा \textcolor{red}{रामं नमति} अत्र हि नमन\-क्रियामाश्रित्यैव द्वितीया\-विभक्तिरुत्पन्ना। द्वितीया च याऽपरा सा पदमाश्रित्यैव। यथा \textcolor{red}{नमः शिवाय} अत्र क्रियान्वयाभावात्कारक\-विभक्तिर्नास्ति। एवं पद\-प्रयोगाभावादतोऽत्र तृतीया किमाधारा इति \textcolor{red}{वृद्धो यूना तल्लक्षणश्चेदेव विशेषः} (पा॰सू॰~१.२.६५) इत्यत्र सह\-प्रयोगं विनाऽपि सहोक्त्या तृतीया दरीदृश्यते।\footnote{\textcolor{red}{विनाऽपि तद्योगं तृतीया। वृद्धो यूनेत्यादिनिर्देशात्‌} (वै॰सि॰कौ॰~५६४)।} तस्मादत्र गम्यमान\-पदस्य कारक\-विभक्तीनां प्रयोजकत्वात् \textcolor{red}{पूर्णेन एकत्वम्‌} इत्यत्र पाणिनीयताऽक्षतैव।\end{sloppypar}
\section[साक्षात्कथितं तव]{साक्षात्कथितं तव}
\centering\textcolor{blue}{इदं रहस्यं हृदयं ममात्मनो मयैव साक्षात्कथितं तवानघ।\nopagebreak\\
मद्भक्तिहीनाय शठाय न त्वया दातव्यमैन्द्रादपि राज्यतोऽधिकम्॥}\nopagebreak\\
\raggedleft{–~अ॰रा॰~१.१.५२}\\
\begin{sloppypar}\hyphenrules{nohyphenation}\justifying\noindent\hspace{10mm} अत्रापि \textcolor{red}{अकथितं च} (पा॰सू॰~१.४.५१) इत्यनेन कर्म\-सञ्ज्ञा तथा च द्वितीयोचितैव। किन्तु \textcolor{red}{विवक्षाधीनानि कारकाणि भवन्ति}\footnote{मूलं मृग्यम्। यद्वा \textcolor{red}{कर्मादीनामविवक्षा शेषः} (भा॰पा॰सू॰~२.३.५०, २.३.५२, २.३.६७) इत्यस्य तात्पर्यमिदम्।} इति न्यायाङ्गीकारेण कर्मणि सम्बन्ध\-विवक्षया षष्ठी। यद्वा \textcolor{red}{समक्षम्‌} इत्यध्याहार्यम्। \textcolor{red}{तव समक्षं कथितम्‌} इति तात्पर्यम्। एवमत्र तु सुतरां सम्बन्धो निर्बाध एव। तेनात्र निसर्गतः षष्ठी।
अथवाऽत्र \textcolor{red}{पृष्ट}\-शब्दस्याध्याहारः। \textcolor{red}{तव पृष्टस्य कथितम्‌}। ततो \textcolor{red}{यस्य च भावेन भाव\-लक्षणम्‌} (पा॰सू॰~२.३.३७) इत्यनेन पृष्ट\-शब्देन कथन\-रूप\-क्रियान्तरस्य द्योतनात्षष्ठी कारकीया।\footnote{\textcolor{red}{दूरान्तिकार्थैः षष्ठ्यन्यतरस्याम्‌} (पा॰सू॰~२.३.३४) इत्यतः \textcolor{red}{षष्ठी} इत्यनुवर्त्य \textcolor{red}{षष्ठी चानादरे} (पा॰सू॰~२.३.३८) इत्यतः \textcolor{red}{षष्ठी} इत्यपकृष्य वाऽऽदरेऽपि भावलक्षणा षष्ठीति भावः।}\end{sloppypar}
\section[तेऽभिहितम्]{तेऽभिहितम्‌}
\centering\textcolor{blue}{एतत्तेऽभिहितं देवि श्रीरामहृदयं मया।\nopagebreak\\
अतिगुह्यतमं हृद्यं पवित्रं पापशोधनम्॥}\nopagebreak\\
\raggedleft{–~अ॰रा॰~१.१.५३}\\
\begin{sloppypar}\hyphenrules{nohyphenation}\justifying\noindent\hspace{10mm} अत्रापि \textcolor{red}{ब्रू}\-धातु\-समानार्थक\-\textcolor{red}{अभिधा}\-प्रकृतिकस्य\footnote{\textcolor{red}{ब्रूञ् व्यक्तायां वाचि} (धा॰पा॰~१०४४)। \textcolor{red}{अभि}\-पूर्वको \textcolor{red}{धा}\-धातुः (\textcolor{red}{डुधाञ् धारण\-पोषणयोः} धा॰पा॰~१०९२) कथनार्थः। \textcolor{red}{क्त}प्रत्यये \textcolor{red}{दधातेर्हिः} (पा॰सू॰~७.४.४२) इत्यनेन \textcolor{red}{हि}आदेशे विभक्तिकार्ये \textcolor{red}{अभिहितम्}।} \textcolor{red}{अभिहित}\-शब्दस्य प्रयोगेण \textcolor{red}{अकथितं च} (पा॰सू॰~१.४.५१) इति कर्म\-सञ्ज्ञया द्वितीया दुर्वारा किन्त्वत्र सम्प्रदान\-विवक्षया चतुर्थी। किं वा \textcolor{red}{त्वां सन्तोषयितुमभिहितम्‌} क्रियार्थोप\-पदस्याप्रयुज्यमानस्य कर्मणि चतुर्थी \textcolor{red}{क्रियार्थोपपदस्य च कर्मणि स्थानिनः} (पा॰सू॰~२.३.१४) इत्यनेन। यद्वा \textcolor{red}{ते हिताय} इति हित\-शब्दमध्याहार्यम्। ततः \textcolor{red}{हित\-योगे च} (वा॰~२.३.१३) इति वार्त्तिकेन चतुर्थी। यद्वा \textcolor{red}{सुख}पदमध्याहार्यम्। \textcolor{red}{ते तुभ्यं हनुमते सुखायाभिहितम्‌}। ततः \textcolor{red}{चतुर्थी तदर्थार्थ\-बलि\-हित\-सुख\-रक्षितैः} (पा॰सू॰~२.१.३६) इत्यत्र पठित\-सुख\-शब्दस्य चतुर्थी\-परत्व\-सूचनादत्र चतुर्थी।\footnote{यद्वा \textcolor{red}{तादर्थ्ये चतुर्थी वाच्या} (वा॰~२.३.१३) इति वार्त्तिकेनात्र चतुर्थी। त्वदर्थमभिहितम् इत्यर्थः।} यद्वा \textcolor{red}{ते} इति षष्ठ्यन्त्यम्। एवमत्र सम्बन्धे षष्ठी। यद्वा \textcolor{red}{प्रश्नोत्तर}\-शब्दोऽध्याहार्यः। \textcolor{red}{तव प्रश्नोत्तरमभिहितम्‌}।\footnote{अत्र \textcolor{red}{प्रश्न}\-शब्दस्य साकाङ्क्षतया \textcolor{red}{उत्तर}\-शब्देन कथं समास इति न शङ्क्यम्। नित्य\-सापेक्षस्थलेष्वस्य नियमस्याप्रसरात्। यद्वा \textcolor{red}{प्रश्नोत्तर}\-शब्देन पार्ष्ठिकोऽन्वयः।}\end{sloppypar}
\section[रामेणोक्तं पुरा मम]{रामेणोक्तं पुरा मम}
\centering\textcolor{blue}{शृणु देवि प्रवक्ष्यामि गुह्याद्गुह्यतरं महत्।\nopagebreak\\
अध्यात्मरामचरितं रामेणोक्तं पुरा मम॥}\nopagebreak\\
\raggedleft{–~अ॰रा॰~१.२.४}\\
\begin{sloppypar}\hyphenrules{nohyphenation}\justifying\noindent\hspace{10mm} अयं प्रयोगोऽध्यात्म\-रामायणस्य बाल\-काण्डस्य द्वितीय\-सर्गस्य चतुर्थे श्लोके शिवेन कृतो वर्तते। कृत\-राम\-विषयक\-प्रश्नां पार्वतीं सम्बोधयन् शिवः प्राह यत् \textcolor{red}{यत्कथा\-वस्तु पुरा रामेण ममोक्तम्‌}। अत्र तु \textcolor{red}{उक्तम्‌} इति शब्दोऽकथित\-गणित\-ब्रू\-धातु\-प्रकृतिक एव। यतो हि \textcolor{red}{ब्रू}\-धातोः (\textcolor{red}{ब्रूञ् व्यक्तायां वाचि} धा॰पा॰~१०४४) कर्मणि \textcolor{red}{क्त}\-प्रत्ययः। \textcolor{red}{ब्रुवो वचिः} (पा॰सू॰~२.४.५३) इत्यनेन \textcolor{red}{वच्‌}\-आदेशः। \textcolor{red}{वचि\-स्वपि\-यजादीनां किति} (पा॰सू॰~६.१.१५) इत्यनेन सम्प्रसारणम्। \textcolor{red}{सम्प्रसारणाच्च} (पा॰सू॰~६.१.१०८) इत्यनेन पूर्वरूपैकादेशः। \textcolor{red}{चोः कुः} (पा॰सू॰~८.२.३०) इत्यनेन कुत्वम्। अत्र साक्षात्कारिका\-परिगणित\-\textcolor{red}{ब्रू}\-धातोरुपस्थितौ द्वितीयाया अवश्यम्भावितया षष्ठीति पाणिनि\-विरुद्धेव किन्तु वस्तुतस्त्वनुरुद्धाऽत्र सम्बन्ध\-विवक्षया षष्ठी। रामायण\-शिवयोः प्रतिपाद्य\-प्रतिपादक\-रूप\-शाश्वत\-सम्बन्धस्य वक्तुमिष्टत्वात्।
\textcolor{red}{मम पुरः} इति वा \textcolor{red}{मम हितार्थम्‌} इति वा। \textcolor{red}{मम श्रवणे} इति वा। अत्रास्मच्छब्दस्य श्रवण\-शब्देनावयवावयवि\-भाव\-सम्बन्धस्य सिद्धत्वात्सम्बन्धे षष्ठी।\end{sloppypar}
\section[ब्रह्मणे प्राह]{ब्रह्मणे प्राह}
\centering\textcolor{blue}{भूमिर्भारेण मग्ना दशवदनमुखाशेषरक्षोगणानां\nopagebreak\\
धृत्वा गोरूपमादौ दिविजमुनिजनैः साकमब्जासनस्य।\nopagebreak\\
गत्वा लोकं रुदन्ती व्यसनमुपगतं ब्रह्मणे प्राह सर्वं\nopagebreak\\
ब्रह्मा ध्यात्वा मुहूर्तं सकलमपि हृदावेदशेषात्मकत्वात्॥}\nopagebreak\\
\raggedleft{–~अ॰रा॰~१.२.६}\\
\begin{sloppypar}\hyphenrules{nohyphenation}\justifying\noindent\hspace{10mm} अत्रापि \textcolor{red}{अकथितं च} (पा॰सू॰~१.४.५१) इत्यनेन द्वितीयैव। तत्स्थाने चतुर्थी तु \textcolor{red}{ब्रह्माणं मोदयितुं प्राह} इत्यप्रयुज्यमान\-मोदन\-क्रिया\-कर्मीभूत\-\textcolor{red}{ब्रह्म}\-शब्दात्।\footnote{\textcolor{red}{क्रियार्थोपपदस्य च कर्मणि स्थानिनः} (पा॰सू॰~२.३.१४) इत्यनेन।} यद्वा \textcolor{red}{ब्रह्मणे हिताय} इत्यध्याहारे \textcolor{red}{हित\-योगे च} (वा॰~२.३.१३) इत्यनेन चतुर्थी। यद्वा सुखमित्यध्याहृत्य \textcolor{red}{चतुर्थी तदर्थार्थबलिहितसुखरक्षितैः} (पा॰सू॰~२.१.३६) इति चतुर्थी\-समास\-सङ्केत\-सूचनाच्चतुर्थी। यद्वा \textcolor{red}{ब्रह्म परमात्मानं नयति धरा\-धाम प्रापयतीति ब्रह्मणः} इति ब्रह्मोपपदे नी\-धातोर्व्युत्पन्नम्।\footnote{ब्रह्म नयतीति ब्रह्मणः। \textcolor{red}{ब्रह्म}\-उपपदे \textcolor{red}{नी}\-धातोः (\textcolor{red}{णीञ् प्रापणे} धा.पा. ९०१) \textcolor{red}{अन्येष्वपि दृश्यते} (पा॰सू॰~३.२.१०१) इत्यनेन \textcolor{red}{ड}\-प्रत्ययः। ब्रह्मन्~अम्~नी~ड~\arrow ब्रह्मन्~अम्~नी~अ~\arrow \textcolor{red}{डित्यभस्याप्यनु\-बन्धकरण\-सामर्थ्यात्‌} (वा॰~६.४.१४३)~\arrow ब्रह्मन्~अम्~न्~अ~\arrow \textcolor{red}{सुपो धातुप्रातिपदिकयोः} (पा॰सू॰~२.४.७१)~\arrow ब्रह्मन्~न्~अ~\arrow \textcolor{red}{नलोपः प्रातिपदिकान्तस्य} (पा॰सू॰~८.२.७)~\arrow ब्रह्म~न्~अ~\arrow ब्रह्मन~\arrow \textcolor{red}{पूर्वपदात्सञ्ज्ञायामगः} (पा॰सू॰~८.४.३)~\arrow ब्रह्मण~\arrow विभक्ति\-कार्यम्~\arrow ब्रह्मणः।} ततश्च सप्तमी \textcolor{red}{उपस्थिते} इति शब्देऽध्याहृते \textcolor{red}{यस्य च भावेन भाव\-लक्षणम्‌} (पा॰सू॰~२.३.३७) इत्यनेन। यद्वा \textcolor{red}{नमस्कुर्मो नृसिंहाय} इतिवद्ब्रह्माणमनुकूलयितुं प्राह। यद्वाऽत्र \textcolor{red}{पत्ये शेते} इतिवत् \textcolor{red}{क्रियया यमभिप्रैति सोऽपि सम्प्रदानम्‌} (वा॰~१.४.३२) इति वार्त्तिकेन कथन\-क्रियया ब्रह्मणोऽभिप्रेतत्वात्सम्प्रदाने ततश्चतुर्थी।\end{sloppypar}
\section[कश्यपस्य वरो दत्तः]{कश्यपस्य वरो दत्तः}
\centering\textcolor{blue}{कश्यपस्य वरो दत्तस्तपसा तोषितेन मे॥}\nopagebreak\\
\raggedleft{–~अ॰रा॰~१.२.२५}\\
\begin{sloppypar}\hyphenrules{nohyphenation}\justifying\noindent\hspace{10mm} अत्र भाराक्रान्तया गो\-रूप\-धारिण्या पृथिव्या सह देवैः क्षीर\-सागरमभिगम्य स्तुवन्तं ब्रह्माणं प्रति स्वकीयावतरण\-प्रकारं प्रकटयन् भगवान् प्रणिगदति यन्मया पूर्वं कश्यपाय वरो दत्तो वर्तते। अतस्तस्यैव गृहे पुत्र\-रूपेणावतरिष्यामि। तत्र दत्त\-पद\-प्रयोगेण चतुर्थ्युचिता \textcolor{red}{तस्मै चपेटां ददाति} (भा॰पा॰सू॰~१.१.१) इति भाष्य\-प्रयोगात्किन्तु \textcolor{red}{कश्यपस्य} इति षष्ठी तन्त्रविरुद्धेव। परं नैतत्। दानस्य कर्मणा यमभिप्रैति स सम्प्रदानम्।\footnote{\textcolor{red}{कर्मणा यमभिप्रैति स सम्प्रदानम्} (पा॰सू॰~१.४.३२)। \textcolor{red}{दानस्य कर्मणा यमभिप्रैति स सम्प्रदान\-सञ्ज्ञः स्यात्} (वै॰सि॰कौ॰~५६९, ल॰सि॰कौ॰~८९६)।} अत्र सम्प्रदानस्य न विवक्षा। यतो हि प्रभुरात्मानं न सम्यक्प्रददाति। मङ्गलाचरण एव स्व\-धाम\-गमन\-सङ्केतात्। यथा \textcolor{red}{रजकस्य वस्त्रं ददाति} इत्यत्र क्षालयितुं वस्त्राणि दीयन्ते पुनश्च परावर्त्यन्ते तथैवात्रापि सप्तविंशति\-वर्षाणां कृते पुत्र\-रूपेणाऽगतः\footnote{कथं तर्हि \pageref{text:exileage1}\-तमे पृष्ठे ग्रन्थ\-प्रस्तावनायाम्~– \textcolor{red}{जन्मतो विवाहं यावद्द्वादशाब्दावधिस्ततो द्वादश\-वर्षं यावदयोध्यायां वास एवं पञ्चविंशे वर्षे सीता\-लक्ष्मणाभ्यां सह वन\-गमनम्‌} इति। कल्पभेदेन।} कश्यपावतारस्य दशरथस्य समक्षम्। पुनः श्रीराम\-वियोगानल\-दग्ध\-शरीरः सनयन\-नीरो धीरो दशरथ एव कश्यपतां गतः। अतोऽत्र षष्ठी दशरथस्याल्प\-कालिकत्वं सूचयति। सम्प्रदानं हि \textcolor{red}{स्व\-स्वत्व\-निवृत्तिपूर्वकं पर\-स्वत्वोत्पादनम्‌} (त॰बो॰~५६९) इति तत्त्वबोधिनी। अत्र भगवान् स्व\-स्वत्वं निवर्तयत्येव नहि स्थले स्थले लोकोत्तर\-कौतुक\-प्रदर्शनाय। यद्वा \textcolor{red}{कृते} इत्यध्याहार्यम्। \textcolor{red}{कश्यपस्य कृते वरो दत्तः} इति कृते\-योगे षष्ठी।
यद्वा \textcolor{red}{भार्यायै} इत्यध्याहार्यम्। \textcolor{red}{कश्यपस्य भार्यायै वरो दत्तः} अतो दाम्पत्य\-भाव\-रूपे सम्बन्धे षष्ठी।\end{sloppypar}
\section[दृष्टं मे]{दृष्टं मे}
\centering\textcolor{blue}{त्वं ममोदरसम्भूत इति लोकान्विडम्बसे।\nopagebreak\\
भक्तेषु पारवश्यं ते दृष्टं मेऽद्य रघूत्तम॥}\nopagebreak\\
\raggedleft{–~अ॰रा॰~१.३.२६}\\
\begin{sloppypar}\hyphenrules{nohyphenation}\justifying\noindent\hspace{10mm} अत्र परम\-पितरं परमात्मानं स्वपुर ईश्वर\-रूपेण प्रस्तुतं विलोक्य भगवती कौसल्या स्तौति यत् \textcolor{red}{हे रघूत्तम अद्य भक्तेषु ते पारवश्यं मे दृष्टम्‌}। \textcolor{red}{दृश्‌}\-धातोः (\textcolor{red}{दृशिँर् प्रेक्षणे} धा॰पा॰~९८८) कर्मणि क्त\-प्रत्यये कृते तेन च कर्मणोऽभिहितत्वात्कर्तुश्चानभिहितत्वात् \textcolor{red}{कर्तृ\-करणयोस्तृतीया} (पा॰सू॰~२.३.१८) इत्यनेनानभिहितेऽस्मत्पद\-वाच्य\-कौसल्या\-रूपिणि कर्तरि तृतीया। अत्र षष्ठी\-विचार\-विषयतामाटीकते। कर्तरि सम्बन्ध\-विवक्षायां षष्ठी। यद्वा \textcolor{red}{दृष्टम्‌} इति भावे कृत्प्रत्ययः \textcolor{red}{नपुंसके भावे क्तः} (पा॰सू॰~३.३.११४) इत्यनेन। ततश्च भावस्य विवक्षयाऽविवक्षितत्वाच्च कर्तुः \textcolor{red}{क्तस्य च वर्तमाने नपुंसके भाव उपसङ्ख्यानम्‌} (वा॰~२.३.६५) इत्यनेन \textcolor{red}{मम} इत्यत्र षष्ठी\footnote{यथा \textcolor{red}{छात्रस्य हसितम्‌। नटस्य भुक्तम्‌। मयूरस्य नृत्तम्‌। कोकिलस्य वयाहृतम्‌} (भा॰पा॰सू॰~२.३.६७) इत्यादि\-भाष्योदाहरणेषु \textcolor{red}{गतं तिरश्चीनमनूरुसारथेः} (शि॰व॰~१.२) \textcolor{red}{हसितं मधुरम् ... मधुराधिपतेः} (म॰अ॰~१) इत्यादि\-शिष्ट\-प्रयोगेषु च। \textcolor{red}{अद्य मे यद्दृष्टं दर्शनं तद्भक्तेषु ते पारवश्यम्‌} इति तात्पर्यम्।} तस्य च \textcolor{red}{मे} इत्यादेशः।\footnote{\textcolor{red}{तेमयावेकवचनस्य} (पा॰सू॰~८.१.२२) इत्यनेन।} यद्वा \textcolor{red}{मे} इत्यस्मात्परं \textcolor{red}{पुरतः} इत्यध्याहार्यम्। \textcolor{red}{मे पुरतो दृष्टम्‌} अत्र सम्बन्धे षष्ठी स्वारसिकी। यद्वा \textcolor{red}{मे} शब्दस्य \textcolor{red}{रघूत्तम}\-शब्देन अन्वयः। अर्थात् \textcolor{red}{हे मे मम रघूत्तम भक्तेषु ते पारवश्यमथ दृष्टम्‌}। अत्र पुत्र\-भावनया \textcolor{red}{मे रघूत्तम} इति व्याहरति। यथाऽयोध्या\-काण्डे स्वयमेव कौसल्या कथयति यत् \textcolor{red}{पुत्रः सभार्यो वनमेव यातः सलक्ष्मणो मे रघुरामचन्द्रः} (अ॰रा॰~२.७.८५)। \textcolor{red}{अयं मम पुत्रः} इति कौसल्या\-वचनं माधुर्यं सूचयति। अत्र च जन्य\-जनक\-भाव\-रूपे सम्बन्धे षष्ठीत्यनेन सूचितं यद्यदा जीवः श्रीरामं प्रत्येव \textcolor{red}{मम} इति सम्बोधयति तदा तस्य संसार\-ममता\-जालं नश्यति।\end{sloppypar}
\section[रामेति]{रामेति}
\centering\textcolor{blue}{यस्मिन् रमन्ते मुनयो विद्यया ज्ञानविप्लवे।\nopagebreak\\
तं गुरुः प्राह रामेति रमणाद्राम इत्यपि॥}\nopagebreak\\
\raggedleft{–~अ॰रा॰~१.३.४०}\\
\begin{sloppypar}\hyphenrules{nohyphenation}\justifying\noindent\hspace{10mm} एष प्रयोगोऽध्यात्म\-रामायण\-बाल\-काण्ड\-तृतीय\-सर्गे चत्वारिंशे श्लोके श्रीमता वसिष्ठेन कृतो भगवतो नाम\-करण\-प्रसङ्गे। अत्र राम\-शब्दस्य व्युत्पत्ति\-प्रकार\-द्वयं दर्शयति। एकोऽधिकरण\-घञन्तोऽपरः कर्त्रजन्तश्च। यस्मिन्मुनयो रमन्ते स रामो यश्च रमयति रमते वा स राम इति। अत्र प्रथमैक\-वचनान्तो \textcolor{red}{रामः} इति शब्दः। ततश्च \textcolor{red}{इति}\-शब्देन सह संहिता\-काले \textcolor{red}{राम सुँ} इति स्थिते पश्चात् \textcolor{red}{ससजुषो रुः} (पा॰सू॰~८.२.६६) इत्यनेन रुत्वे \textcolor{red}{भो\-भगो\-अघो\-अपूर्वस्य योऽशि} (पा॰सू॰~८.३.१७) इत्यनेन रोर्यत्वे \textcolor{red}{लोपः शाकल्यस्य} (पा॰सू॰~८.३.१९) इत्यनेन यकारलोपे पुना राम\-घटकाकारस्येति\-घटकेकारेण सहाऽशङ्क्यमाने गुणे \textcolor{red}{पूर्वत्रासिद्धम्‌} (पा॰सू॰~८.२.१) इत्यनेन त्रिपादीत्वाद्यलोपासिद्धौ गुणानवसरे \textcolor{red}{राम इति} इत्येव पाणिनीयम्। \textcolor{red}{पूर्वत्रासिद्धम्‌} (पा॰सू॰~८.२.१) इति सूत्रस्य जागरूकतायां गुणाभावे कथं \textcolor{red}{रामेति} इति चेत्। \textcolor{red}{मतौ च्छः सूक्तसाम्नोः} (पा॰सू॰~५.२.५९) इति सूत्र\-ज्ञापनात् \textcolor{red}{अनुकरणानु\-कार्ययोर्भेदाभेद\-विवक्षा च}\footnote{मूलं मृग्यम्। \textcolor{red}{मतौ च्छः सूक्तसाम्नोः} (पा॰सू॰~५.२.५९) इत्यस्य भाष्ये प्रदीपोद्द्योतयोश्च स्पष्टमिदम्। अभेदपक्षे \textcolor{red}{प्रकृतिवदनुकरणं भवति} (भा॰शि॰सू॰~२) इति महाभाष्ये \textcolor{red}{ऋऌक्‌} (शि॰सू॰~२) शिवसूत्र उक्तम्। \textcolor{red}{अनुकरणं ह्यनुकार्याद्भिन्नम्‌} इत्यपि महाभाष्ये \textcolor{red}{मतौ च्छः सूक्तसाम्नोः} (पा॰सू॰~५.२.४९) सूत्र उक्तमिति वैयाकरण\-भूषण\-सारस्य दर्पण\-व्याख्यायां चन्द्रिका\-प्रसाद\-द्विवेदाः। अस्माभिर्भाष्य\-संस्करणेषु \textcolor{red}{अनुकरणं ह्यनुकार्याद्भिन्नम्‌} इति नोपलब्धम्।} इति
परिभाषया तावद्भेद\-विवक्षाऽभेद\-विवक्षा च क्रियते। इति\-शब्द\-समभिव्याहरणेनात्र द्विःप्रयुक्तो राम\-शब्दोऽनुकरण\-परः। इति\-शब्दो ह्यनुकरण\-द्योतकः। तथा च पाणिनेः सूत्रम् \textcolor{red}{अव्यक्तानुकरणस्यात इतौ} (पा॰सू॰~६.१.९८)। \textcolor{red}{ध्वनेरनुकरणस्य योऽच्छब्दस्तस्मादितौ परे पररूपमेकादेशः स्यात्। पटत् इति पटिति} (वै॰सि॰कौ॰~८१)। अत्रत्या तत्त्वबोधिनी च~– \textcolor{red}{यद्यपि “अतो गुणे” इति पूर्व\-सूत्रादत इत्यनुवर्त्यातो\-ग्रहणमिह त्यक्तुं शक्यं तथाऽपि पूर्व\-सूत्रे अत इति तपर\-करणाद्ध्रस्वाकारस्य ग्रहणमिह तु शब्दाधिकार\-पक्षाश्रयणादच्छब्दस्य ग्रहणमिति व्याख्याने क्लेशः स्यादिति पुनरत्रातो\-ग्रहणं कृतम्। अव्यक्त\-शब्दं व्याचष्टे “ध्वनेरिति”। “अनुकरणस्येति”। परिस्फुटाकारादि\-वर्णस्येति भावः। तस्य चानुकरणत्वं किञ्चित्साम्येन बोध्यम्। पर\-रूपस्यास्य नित्यत्वेऽपि संहितायामविवक्षितायां तदभावादाह। “पटदितीति”} (त॰बो॰~८१)। अतोऽत्राभेद\-विवक्षायां विभक्त्यभावः।\footnote{तथा च \textcolor{red}{गवित्ययमाह। अत्रानु\-कार्यानु\-करणयोर्भेदस्याविवक्षितत्वादसत्यर्थवत्त्वे विभक्तिर्न भवति} (का॰वृ॰~१.१.१६)।} एवं च \textcolor{red}{आद्गुणः} (पा॰सू॰~६.१.८७) इत्यनेन गुणः।\footnote{अत्रोक्तम्~– \textcolor{red}{अत्र द्विःप्रयुक्तो राम\-शब्दोऽनुकरण\-परः}। यद्युभयत्रानुकरण\-परो रामशब्दः कथं तर्हि \textcolor{red}{रमणाद्राम इत्यपि} इति चेत्। संहिताया अविवक्षणात्।} ज्ञानाधिकरण\-ज्ञान\-स्वरूप इति न्याय\-वेदान्त\-बोध्य\-निर्गुण\-ब्रह्माभिन्न एव दशरथ\-पुत्रो राम इत्येव वसिष्ठ\-तात्पर्यं द्योतयितुमत्राभेद\-विवक्षायां संहिता। \textcolor{red}{संहितात्वं नामार्ध\-मात्रा\-कालातिरिक्त\-काल\-व्यवधान\-शून्यत्वम्‌}। यद्वा \textcolor{red}{अपदं न प्रयुञ्जीत} इति व्याकरण\-प्रसिद्धेर्द्विपदमनुकुर्वतो द्विपदस्य परम\-पदस्य भगवतः श्रीरामचन्द्रस्य कृते किमपदं प्रयुञ्जीत शिव इत्यपेक्षायामुच्यते। \textcolor{red}{सह सुपा} (पा॰सू॰~२.१.४) इति हि सूत्रम्। अत्र हि योग\-विभागः \textcolor{red}{सह} इति पृथक्पदं \textcolor{red}{सुपा} इति च पृथक्। सहेति समर्थेन सह समस्यते इत्यर्थक\-योग\-विभाग\-प्रथमांशेनात्र समासः \textcolor{red}{राम}\-शब्दस्य \textcolor{red}{इति}\-शब्देन। पश्चात् \textcolor{red}{सुपो धातु\-प्रातिपादिकयोः} (पा॰सू॰~२.४.७१) इत्यनेन विभक्ति\-लोपः। पश्चाद्गुणः। प्रत्यय\-लक्षणमाश्रित्य पुनर्गुण\-व्यवधानं न शक्यं यथा \textcolor{red}{गो\-हितम्‌} इत्यत्रान्तर्वर्तिनीं \textcolor{red}{ङे}\-विभक्तिमाश्रित्य न \textcolor{red}{अव्‌}\-आदेशस्तथैव \textcolor{red}{वर्णाश्रये नास्ति प्रत्यय\-लक्षणम्‌} (प॰शे॰~२०) इत्यनेनात्रापि प्रत्यय\-लक्षणं निषेध्यम्। अतो \textcolor{red}{रामेति} अयं शब्दः पाणिनीय एव।\end{sloppypar}
\section[मुनीन्द्राहम्]{मुनीन्द्राहम्‌}
\label{sec:munindraham}
\centering\textcolor{blue}{अभिवाद्य मुनिं राजा प्राञ्जलिर्भक्तिनम्रधीः।\nopagebreak\\
कृतार्थोऽस्मि मुनीन्द्राहं त्वदागमनकारणात्॥}\nopagebreak\\
\raggedleft{–~अ॰रा॰~१.४.३}\\
\begin{sloppypar}\hyphenrules{nohyphenation}\justifying\noindent\hspace{10mm} अयं च प्रयोगोऽध्यात्म\-रामायणस्य बाल\-काण्डस्य चतुर्थ\-सर्गीयः। अत्रायोध्या\-समागतं श्रीराघव\-समेतं विश्वामित्रं प्रणम्य प्राञ्जलिर्योग\-राजो दशरथः कथयति \textcolor{red}{मुनीन्द्र अहं कृतार्थः} एवम्। \textcolor{red}{दूराद्धूते च} (पा॰सू॰~८.२.८४) इत्यनेन प्लुतस्ततश्च \textcolor{red}{प्लुत\-प्रगृह्या अचि नित्यम्‌} (पा॰सू॰~६.१.१२५) इत्यनेन प्रकृति\-भावस्तस्मादतः \textcolor{red}{मुनीन्द्र३ अहम्‌} इत्येव पाणिनीयम्। \textcolor{red}{मुनीन्द्राहम्‌} इत्यार्षप्रयोगो नैव पाणिनीय इति न भ्रमितव्यम्। अत्र विश्वामित्रमभिगम्यैव दशरथः प्रणमति तेन दूर\-सम्बोधनत्वाभावान्न प्लुतावसरः। यद्वा \textcolor{red}{गुरोरनृतोऽ\-नन्त्यस्याप्येकैकस्य प्राचाम्‌} (पा॰सू॰~८.२.८६) इत्यत्र \textcolor{red}{प्राचाम्‌} इति योग\-विभागः। तथोक्तं सिद्धान्त\-कौमुद्याम् \textcolor{red}{इह प्राचामिति योगो विभज्यते। तेन सर्वः प्लुतो विकल्प्यते} (वै॰सि॰कौ॰~९७)। अनेन योग\-विभागेन सर्वेषां प्लुतानां वैकल्पिकता। प्लुताभावे \textcolor{red}{अकः सवर्णे दीर्घः} (पा॰सू॰~६.१.१०१) इत्यनेन सवर्णदीर्घः। अथवा \textcolor{red}{अकः सवर्णे दीर्घः} (पा॰सू॰~६.१.१०१) इति सूत्रं सपाद\-सप्ताध्यायीस्थं प्लुत\-विधायकञ्च त्रिपादीस्थं \textcolor{red}{दूराद्धूते च} (पा॰सू॰~८.२.८४)। ततः \textcolor{red}{पूर्वत्रासिद्धम्‌} (पा॰सू॰~८.२.१) इति सूत्र\-बलेन सपाद\-सप्ताध्याय्या दीर्घ\-शास्त्र\-कर्तव्यतायां प्लुत\-विधायकं त्रिपादी\-शास्त्रमसिद्धम्। एवं तन्निमित्तक\-प्रकृति\-भावस्य नास्ति प्रसरः। अतः \textcolor{red}{मुनीन्द्राहम्‌} इत्यत्र दीर्घः पाणिनि\-तन्त्रशोऽनुकूलः। एवमेव \textcolor{red}{तदप्यहोऽहं तव देव भक्ता} (अ॰रा॰~१.१.८) इत्यत्रापि समाधेयम्। यतो हि \textcolor{red}{अहो} इत्योदन्त\-निपातः। \textcolor{red}{ओत्‌} (पा॰सू॰~१.१.१५) इत्यनेन सूत्रेण \textcolor{red}{अहो} इत्यस्यौदन्त\-निपातत्वात्प्रगृह्य\-सञ्ज्ञा। ततोऽकारेऽचि परे \textcolor{red}{प्लुतप्रगृह्या अचि नित्यम्‌} (पा॰सू॰~६.१.१२५) इत्यनेन प्रकृतिभावे \textcolor{red}{अहो अहम्‌} इत्येव पाणिनीयम् \textcolor{red}{अहोऽहम्‌} इत्यपाणिनीयं प्रतीयत इति चेत्। अत्रापि \textcolor{red}{प्राचाम्‌} (पा॰सू॰~८.२.८६) इति योग\-विभागेन प्लुतस्य विकल्प इव प्रगृह्यस्यापि विकल्पता। उपलक्षणत्वात्। यद्वा \textcolor{red}{प्लुतप्रगृह्या अचि नित्यम्‌} (पा॰सू॰~६.१.१२५) इत्यत्र नित्य\-ग्रहणं प्रायिकार्थे। अर्थात्क्वचिदचि परे प्रगृह्यः प्रकृति\-भाव\-भाङ्न। यथा \textcolor{red}{ङमो ह्रस्वादचि ङमुण्नित्यम्‌} (पा॰सू॰~८.३.३२) इत्यत्र नित्यग्रहणस्य प्रायिकत्वात् \textcolor{red}{इको यणचि} (पा॰सू॰~६.१.७७) \textcolor{red}{सुप्तिङन्तं पदम्‌} (पा॰सू॰~१.४.१४) इत्यादौ न ङुण्मुटौ। तथैवात्र नित्य\-ग्रहणस्य प्रायिकत्वात् \textcolor{red}{अहो} इत्यस्मात्परेऽप्यकारेऽचि न प्रकृतिभावः।\end{sloppypar}
\section[मह्यम्]{मह्यम्‌}
\centering\textcolor{blue}{प्रत्याख्यातो यदि मुनिः शापं दास्यत्यसंशयः।\nopagebreak\\
कथं श्रेयो भवेन्मह्यमसत्यं चापि न स्पृशेत्॥}\nopagebreak\\
\raggedleft{–~अ॰रा॰~१.४.११}\\
\begin{sloppypar}\hyphenrules{nohyphenation}\justifying\noindent\hspace{10mm} विश्वामित्रो दशरथमभिगम्य राक्षस\-वधार्थी सलक्ष्मणं श्रीरामचन्द्रं याचते। तत्र पुत्र\-वत्सलतया रामचन्द्रं न दित्सन्महाराज\-दशरथो विकल्पते यच्छ्रीराम\-विसर्जनेऽहं जीवनमपि नैव धारयितुं शक्नोमि। एवं च प्रत्याख्यातः स मुनिः शापं दास्यति। एवमसत्य\-भाषण\-जनित\-पातकमपि लगिष्यति। तदानीमयं प्रयोगः \textcolor{red}{कथं श्रेयो भवेन्मह्यम्‌} इति।
अत्रास्मच्छब्दस्य चतुर्थ्येक\-वचनान्त\-रूपं \textcolor{red}{मह्यम्‌} इति।\footnote{अस्मद् ङे~\arrow \textcolor{red}{ङे प्रथमयोरम्‌} (पा॰सू॰~७.१.२८)~\arrow अस्मद्~अम्~\arrow \textcolor{red}{तुभ्यमह्यौ ङयि} (पा॰सू॰~७.२.९५)~\arrow मह्य~अद्~अम्~\arrow \textcolor{red}{अतो गुणे} (पा॰सू॰~६.१.९७)~\arrow मह्यद्~अम्~\arrow \textcolor{red}{शेषे लोपः} (पा॰सू॰~७.२.९०)~\arrow मह्य्~अम्~\arrow मह्यम्।} \textcolor{red}{मह्यम्‌} शब्दस्य श्रेयः\-प्रतियोगितया साकाङ्क्षत्वात्प्रतियोगिनः षष्ठ्यन्तत्वमेव। दृष्टि\-दृष्ट\-प्रायत्वादत्र सम्बन्धे षष्ठी भवेत्। किन्तु चतुर्थीयं महाराज एव श्रेयसः सम्प्रदानमिति विवक्षया। यद्वा \textcolor{red}{मह्यम्‌} इत्यस्य \textcolor{red}{शापं दास्यति} इत्यत्रान्वयः। अर्थात्प्रत्याख्यातो मुनिरसंशयं मह्यं शापं दास्यत्येवं कथं श्रेयो भवेदसत्यं च न स्पृशेदित्यत्रान्वय\-प्रकारः। इत्थं \textcolor{red}{मह्यम्‌} शब्दस्य \textcolor{red}{श्रेयः} शब्देनासत्यन्वये परिहारः। यद्वा \textcolor{red}{मह्यम्‌} न शब्द\-रूपमपि तु कृत्य\-प्रत्ययान्तम्। \textcolor{red}{महँ पूजायाम्‌} (धा॰पा॰~७३०, १८६७) इत्यस्माद्धातोः \textcolor{red}{कृत्य\-ल्युटो बहुलम्‌} (पा॰सू॰~३.३.११३) इति सूत्रानुरोधेनानुबन्ध\-लोपे सति \textcolor{red}{मह्यत इति मह्यम्‌} इत्यस्मिन् विग्रहे
कर्मणि यत्प्रत्ययः। अनुबन्ध\-लोपे विभक्ति\-कार्ये सौ सोरमादेशे\footnote{\textcolor{red}{अतोऽम्‌} (पा॰सू॰~३.३.११३) इत्यनेन} \textcolor{red}{मह्यम्‌} इति सिद्धम्। अत्र \textcolor{red}{मह्यम्‌} शब्दः \textcolor{red}{श्रेयः} शब्दस्य विशेषणम्। अर्थात् \textcolor{red}{मह्यं महत्त्व\-पूर्णम्‌}।\end{sloppypar}
\section[रामाय]{रामाय}
\centering\textcolor{blue}{विश्वामित्रोऽपि रामाय तां योजयितुमागतः।\nopagebreak\\
एतद्गुह्यतमं राजन्न वक्तव्यं कदाचन॥}\nopagebreak\\
\raggedleft{–~अ॰रा॰~१.४.१९}\\
\begin{sloppypar}\hyphenrules{nohyphenation}\justifying\noindent\hspace{10mm} वसिष्ठ\-वाक्यमिदम्। अत्र \textcolor{red}{रामेण योजयितुं} इति पाणिनीयं \textcolor{red}{रामाय} इत्यार्ष\-चतुर्थी। विचारे कृत इयमपि पाणिनीया। अत्र \textcolor{red}{हित}\-शब्दोऽध्याहार्यः। ततश्च \textcolor{red}{हित\-योगे च} (वा॰~२.३.१३) इत्यनेन चतुर्थी। यद्वा \textcolor{red}{रामं सुखयितुं तां योजयितुमागतः} इति \textcolor{red}{क्रियार्थोपपदस्य च कर्मणि स्थानिनः} (पा॰सू॰~२.३.१४) इत्यनेन चतुर्थी।\end{sloppypar}
\section[रामरामेति]{रामरामेति}
\centering\textcolor{blue}{आहूय रामरामेति लक्ष्मणेति च सादरम्।\nopagebreak\\
आलिङ्ग्य मूर्ध्न्यवघ्राय कौशिकाय समर्पयत्॥\footnote{\textcolor{red}{समार्पयत्} इति प्रयोक्तव्ये \textcolor{red}{समर्पयत्} इति प्रयोगः। अस्मिन् विषये \pageref{sec:prasarayat}तमे पृष्ठे \ref{sec:prasarayat} \nameref{sec:prasarayat} इति प्रयोगस्य विमर्शं पश्यन्तु।}}\nopagebreak\\
\raggedleft{–~अ॰रा॰~१.४.२२}\\
\begin{sloppypar}\hyphenrules{nohyphenation}\justifying\noindent\hspace{10mm} अत्र \textcolor{red}{राम}\-शब्देन सह \textcolor{red}{इति}\-शब्दस्य संहिता। तत्र \textcolor{red}{राम}\-घटकाकारस्य \textcolor{red}{इति}\-घटकेकारेण सह संहितायामुभयोः स्थाने गुणः। स एव विचार्यः। \textcolor{red}{दूराद्धूते च} (पा॰सू॰~८.२.८४) इति सूत्रेण दूर\-सम्बोधन\-वाक्यस्य \textcolor{red}{राम३ राम३} इत्यस्य टेर्मकारोत्तराकारस्य प्लुतः। एवं \textcolor{red}{प्लुत\-प्रगृह्या अचि नित्यम्‌} (पा॰सू॰~६.१.१२५) इति प्रकृति\-भावः स्यात्। अत्र च \textcolor{red}{राम३ राम३ इति} इत्येव पाणिनि\-तन्त्रीयः। अत्र सन्धिरार्षः।\footnote{पूर्वपक्षोऽयम्।} इत्थमेव प्रथम\-श्रीमद्भागवत\-टीका\-कारैः श्रीधर\-स्वामि\-पादैः~–\end{sloppypar}
\centering\textcolor{red}{यं प्रव्रजन्तमनुपेतमपेतकृत्यं द्वैपायनो विरहकातर आजुहाव।\nopagebreak\\
पुत्रेति तन्मयतया तरवोऽभिनेदुस्तं सर्वभूतहृदयं मुनिमानतोऽस्मि॥}\nopagebreak\\
\raggedleft{–~भा॰पु॰~१.२.२}\\
\begin{sloppypar}\hyphenrules{nohyphenation}\justifying\noindent इत्यत्र \textcolor{red}{पुत्रेति} अयं प्रयोगोऽप्यार्षत्वेन समाहितः।\footnote{\textcolor{red}{द्वैपायनो व्यासो विरहात्कातरो भीतः सन्पुत्र३ इति पलुतेन आजुहाव आहूतवान्। दूरादाह्वाने प्लुते सत्यपि सन्धिरार्षः} (भा॰पु॰~श्री॰टी॰~१.२.२)।} तथैवात्रापि। वस्तुतस्त्वयं पाणिनीय एव। अत्र चत्वारः पक्षाः प्रदर्श्यन्ते। प्रथमः \textcolor{red}{गुरोरनृतोऽनन्त्यस्याप्येकैकस्य प्राचाम्‌} (पा॰सू॰~८.२.८६) इत्यत्र \textcolor{red}{प्राचाम्‌} इति योग\-विभागस्तेन सर्वेषां प्लुतानां विकल्पः।\footnote{\textcolor{red}{इह प्राचामिति योगो विभज्यते। तेन सर्वः प्लुतो विकल्प्यते} (वै॰सि॰कौ॰~९७)। अत्रत्या तत्त्वबोधिनी~– \textcolor{red}{सर्वः प्लुतो विकल्प्यत इति। एतेन “द्वैपायनो विरहकातर आजुहाव पुत्रेति” इति भागवतं व्याख्यातम्। प्लुतस्य वैकल्पिकत्वात् ‘आर्षः प्रयोगः’ इति श्रीधराचार्योक्तिस्तु नादर्तव्या} (त॰बो॰~९७)।} अतोऽत्रापि प्लुताभावाद्गुणः सुकरः। द्वितीयः पक्षः \textcolor{red}{अप्लुतवदुपस्थिते} (पा॰सू॰~६.१.१२९) इत्यनेन प्लुतस्याप्लुतवद्भाव उपस्थित\-शब्देऽवैदिक\-\textcolor{red}{इति}\-परे। अर्थादवैदिक \textcolor{red}{इति}\-परे प्लुतोऽप्लुत\-वद्भवतीत्यनेनाप्लुत\-वद्भावस्ततश्च गुणोऽतो \textcolor{red}{रामरामेति} अयं प्रयोगः पाणिनीयः। पक्षान्तरेऽपि \textcolor{red}{सम्बुद्धौ शाकल्यस्येतावनार्षे} (पा॰सू॰~१.१.१६) इति सूत्रेण वैकल्पिक\-प्रकृति\-भावः। चतुर्थे पक्षे प्लुत\-प्रकरणमेवानित्यम्। अथ मया पञ्चम\-पक्षः प्रस्तूयते यद्गुण\-विधौ कर्तव्ये प्लुत\-शास्त्रमेवासिद्धमतो \textcolor{red}{रामरामेति} पाणिन्यनुकूल एव।\end{sloppypar}
\section[लक्ष्मणेति]{लक्ष्मणेति}
\centering\textcolor{blue}{आहूय रामरामेति लक्ष्मणेति च सादरम्।\nopagebreak\\
आलिङ्ग्य मूर्ध्न्यवघ्राय कौशिकाय समर्पयत्॥}\nopagebreak\\
\raggedleft{–~अ॰रा॰~१.४.२२}\\
\begin{sloppypar}\hyphenrules{nohyphenation}\justifying\noindent\hspace{10mm} अयमपि प्रयोगस्तथैव। \textcolor{red}{लक्ष्मण इति} स्थिते प्लुते जाते प्रकृति\-भावो नित्यत्वेन प्राप्तः पुनः \textcolor{red}{प्राचाम्‌} (पा॰सू॰~८.२.८६) इति योग\-विभाग\-सामर्थ्यात्प्लुत\-विकल्पे सन्धिः साधीयान्।\end{sloppypar}
\section[दीक्षां प्रविश्यताम्]{दीक्षां प्रविश्यताम्‌}
\centering\textcolor{blue}{पूजां च महतीं चक्रू रामलक्ष्मणयोर्द्रुतम्।\nopagebreak\\
श्रीरामः कौशिकं प्राह मुने दीक्षां प्रविश्यताम्॥}\nopagebreak\\
\raggedleft{–~अ॰रा॰~१.५.३}\\
\begin{sloppypar}\hyphenrules{nohyphenation}\justifying\noindent\hspace{10mm} कौशिकेन स्वाश्रमं नीतो भगवान् श्रीरामो विश्वामित्रं कथयति \textcolor{red}{हे मुने भवता दीक्षां प्रविश्यताम्‌}। अत्र \textcolor{red}{दीक्षां प्रविश} इति कर्तृ\-वाच्य\-रूपम्। कर्म\-वाच्ये च \textcolor{red}{लः कर्मणि च भावे चाकर्मकेभ्यः} (पा॰सू॰~३.४.६९) इति सूत्रानुसारं \textcolor{red}{भाव\-कर्मणोः} (पा॰सू॰~१.३.१३) इत्यनेनाऽत्मने\-पदम्। \textcolor{red}{त}\-प्रत्ययः \textcolor{red}{टित आत्मनेपदानां टेरे} (पा॰सू॰~३.४.७९) इत्यनेन \textcolor{red}{ते} इति जातः। ततश्च \textcolor{red}{आमेतः} (पा॰सू॰~३.४.९०) इत्यनेन लोड्लकारे \textcolor{red}{आम्‌} आदेशे \textcolor{red}{प्रविश्यताम्‌}।\footnote{प्र~\textcolor{red}{विशँ प्रवेशने} (धा॰पा॰~१४२४)~\arrow प्र~विश्~\arrow \textcolor{red}{भाव\-कर्मणोः} (पा॰सू॰~१.३.१३)~\arrow \textcolor{red}{लोट् च} (पा॰सू॰~३.३.१६२)~\arrow प्र~विश्~लोट्~\arrow प्र~विश्~त~\arrow \textcolor{red}{सार्वधातुके यक्‌} (पा॰सू॰~३.१.६७)~\arrow प्र~विश्~यक्~त~\arrow प्र~विश्~य~त~\arrow \textcolor{red}{टित आत्मनेपदानां टेरे} (पा॰सू॰~३.४.७९)~\arrow प्र~विश्~य~ते~\arrow \textcolor{red}{आमेतः} (पा॰सू॰~३.४.९०)~\arrow प्र~विश्~य~ताम्~\arrow प्रविश्यताम्।} एवं \textcolor{red}{यस्मिन्नर्थे प्रत्ययः स उक्तो} महा\-सञ्ज्ञा\-करणाच्च \textcolor{red}{प्रत्याययत्यर्थं यः स प्रत्यय} इत्युभयोर्नियमयोर्जागरूकत्वे कर्मणि प्रत्ययस्य विहितत्वात्प्रथमैवोचिता। एवं च \textcolor{red}{अनभिहिते} (पा॰सू॰~२.३.१) इति सूत्रं ह्यधिकारः। अधिकारो नामोत्तरोत्तर\-सम्बन्धित्वम्। ततः \textcolor{red}{कर्मणि द्वितीया} (पा॰सू॰~२.३.२) इत्यनेनाप्येतस्य सम्बन्धः। अथ च \textcolor{red}{अनभिहिते कर्मणि द्वितीया} इत्येव सूत्रार्थः। तस्मात्कर्मणोऽभिहितत्वादत्र प्रथमैव। तथा चोक्तं सिद्धान्त\-कौमुद्यां कारक\-प्रकरणे \textcolor{red}{अभिधानं च प्रायेण तिङ्कृत्तद्धित\-समासैः} (वै॰सि॰कौ॰~५३७) तेन \textcolor{red}{हरिः सेव्यते} इति प्रयोग इवात्रापि तिङ्। तिङोक्तं कर्म। अतो \textcolor{red}{दीक्षाम्‌} इत्यत्र द्वितीया पाणिनि\-विरुद्धेव। विचारे कृतेऽत्र \textcolor{red}{प्रविश्यताम्‌} इतिघटक\-धातुरकर्मकः। यद्यपि \textcolor{red}{विश्‌}\-धातुः (\textcolor{red}{विशँ प्रवेशने} धा॰पा॰~१४२४) सकर्मकः सर्व\-जन\-विदितः कथमत्राकर्मकतेत्यपेक्षायां तत्रैव बाल\-मनोरमायामात्मनेपद\-प्रक्रिया\-प्रकरणे दीक्षित\-महाभागैरेका कारिका निरटङ्कि यत्कस्यां कस्यां परिस्थितावकर्मिकाः क्रिया भवन्ति यथा~–\end{sloppypar}
\centering\textcolor{red}{धातोरर्थान्तरे वृत्तेर्धात्वर्थेनोपसङ्ग्रहात्।\nopagebreak\\
प्रसिद्धेरविवक्षातः कर्मणोऽकर्मिका क्रिया॥}\nopagebreak\\
\raggedleft{–~बा॰म॰~२६९५, वै॰सि॰कौ॰~२७०१, वा॰प॰~३.७.८८}\\
\begin{sloppypar}\hyphenrules{nohyphenation}\justifying\noindent यदा धातोरर्थान्तरं भवति तदाऽकर्मता। यथा \textcolor{red}{वहँ प्रापणे} (धा॰पा॰~१००४) इत्यस्य \textcolor{red}{वहँ स्यन्दने}। अयं \textcolor{red}{भारं वहति} इत्यत्र सकर्मकः \textcolor{red}{नदी वहति} इत्यत्र अकर्मकः। यदा धात्वर्थेनैव सङ्ग्रहो भवति तदैवाकर्मकः। यथा \textcolor{red}{जीवनं धारयति} इत्यस्य \textcolor{red}{जीवति} इत्यत्रोपसङ्ग्रहः। प्रसिद्धेरप्यकर्मकता। यथा \textcolor{red}{मेघो वर्षति} अत्र जल\-रूपस्य कर्मणः प्रसिद्धिस्तस्मादकर्मकः। एवमेव कर्मणोऽविवक्षातोऽकर्मकता। तस्मादत्र कर्मणो न विवक्षा। तस्मादकर्मक\-\textcolor{red}{विश्‌}\-धातोर्भावे लोड्लकारे त\-प्रत्ययः। एवं \textcolor{red}{दीक्षाम्‌} इत्यस्य \textcolor{red}{आश्रित्य} इत्यध्याहृता क्रिया। तस्मादाश्रयानुकूल\-व्यापारस्य कर्मणः क्त्वा\-प्रत्यय\-स्थानापन्न\-ल्यप्प्रत्यनेनानुक्तत्वम्। तस्य कर्तर्येवं विधानम्। अतोऽनुक्ते कर्मणि द्वितीया पाणिन्यनुकूला। अतो \textcolor{red}{दीक्षां प्रविश्यताम्‌} अयं प्रयोगः सुकरः। अथवा \textcolor{red}{गण\-कार्यमनित्यम्‌} (प॰शे॰~९३.३) इति नियमात् \textcolor{red}{प्र}\-पूर्वक\-\textcolor{red}{विश्‌}\-धातुर्दैवादिकः कल्प्यताम्।\footnote{\textcolor{red}{बहुलमेतन्निदर्शनम्‌} (धा॰पा॰ ग॰सू॰~१९३८) \textcolor{red}{आकृतिगणोऽयम्‌} (धा॰पा॰ ग॰सू॰~१९९२) \textcolor{red}{भूवादिष्वेतदन्तेषु दशगणीषु धातूनां पाठो निदर्शनाय तेन स्तम्भुप्रभृतयः सौत्राश्चुलुम्पादयो वाक्यकारीयाः प्रयोगसिद्धा विक्लवत्यादयश्च} (मा॰धा॰वृ॰~१०.३२८) इत्यनुसारमाकृति\-गणत्वाद्दिवादि\-गण ऊह्योऽयं धातुः।} ततश्च \textcolor{red}{दिवादिभ्यः श्यन्‌} (पा॰सू॰~३.१.६९) इत्यनेन प्र\-पूर्वक\-\textcolor{red}{विश्‌}\-धातोः \textcolor{red}{श्यन्‌} विकरणः। एवं \textcolor{red}{हे मुने तां पूर्व\-निर्दिष्टां दीक्षां त्वं प्रविश्य प्रविष्टो भव} इति नेयं कर्म\-वाच्य\-क्रियाऽपि तु \textcolor{red}{प्रविश्य} इति कर्तृ\-वाच्य\-क्रिया। इदं नव्यं समाधानम्। अथवा \textcolor{red}{दीक्षां प्रविश्य ताम्‌} इत्यत्र \textcolor{red}{प्रविश्य} इति ल्यबन्त\-प्रयोगः। तस्य च परवर्ति\-श्लोकस्थेन \textcolor{red}{दर्शयस्व} इत्यनेन क्रिया\-पदेनान्वयः। अर्थात् \textcolor{red}{हे महाभाग मुने तां पूर्व\-निर्दिष्टां दीक्षां प्रविश्य कुतस्तौ राक्षसाधमौ दर्शयस्व}।\end{sloppypar}
\section[पक्वफलादिभिः]{पक्वफलादिभिः}
\centering\textcolor{blue}{भोजयित्वा सह भ्रात्रा रामं पक्वफलादिभिः।\nopagebreak\\
पुराणवाक्यैर्मधुरैर्निनाय दिवसत्रयम्॥}\nopagebreak\\
\raggedleft{–~अ॰रा॰~१.५.११}\\
\begin{sloppypar}\hyphenrules{nohyphenation}\justifying\noindent\hspace{10mm} अत्र श्रीरामः फलं भुङ्क्ते विश्वामित्रः प्रेरयतीत्यर्थे \textcolor{red}{विश्वामित्रो रामं पक्व\-फलादीनि भोजयति}। अस्यामेवावस्थायां \textcolor{red}{क्त्वा}\-प्रत्यये \textcolor{red}{पक्वफलादीनि भोजयित्वा} इत्येव सामान्यतः पाणिनीयानुरूपम्। \textcolor{red}{पक्व\-फलादिभिः} इति कथम्। विमर्शे सति \textcolor{red}{विवक्षाधीनानि कारकाणि भवन्ति}\footnote{मूलं मृग्यम्। यद्वा \textcolor{red}{कर्मादीनामविवक्षा शेषः} (भा॰पा॰सू॰~२.३.५०, २.३.५२, २.३.६७) इत्यस्य तात्पर्यमिदम्।} इति नियमेनात्र करणत्व\-विवक्षा। करणं हि \textcolor{red}{साधकतमं करणम्‌} (पा॰सू॰~१.४.४२) इति पाणिनीय\-सूत्रानुसारं क्रिया\-सिद्धौ प्रकृष्टोप\-कारकं कारकम्। वाक्य\-पदीये च करण\-लक्षणमित्थम्~–\end{sloppypar}
\centering\textcolor{red}{क्रियायाः परिनिष्पत्तिर्यद्व्यापारादनन्तरम्।\nopagebreak\\
विवक्ष्यते यदा यत्र करणं तत्तदा स्मृतम्॥}\nopagebreak\\
\raggedleft{–~वा॰प॰~३.७.९०}\\
\begin{sloppypar}\hyphenrules{nohyphenation}\justifying\noindent अतः \textcolor{red}{कर्तृ\-करणयोस्तृतीया} (पा॰सू॰~२.३.१८) इति सूत्रेण तृतीया। यद्वा \textcolor{red}{हेतौ} (पा॰सू॰~२.३.२३) इति सूत्रेण तृतीया पक्व\-फलादौ हेतुत्व\-विवक्षणात्। सिद्धान्त\-कौमुद्यां हेतु\-करणयोरन्तर\-प्रतिपादन\-पुरः\-सरे लक्षणे व्याचष्ट दीक्षितो यत् \textcolor{red}{द्रव्यादि\-साधारणं निर्व्यापार\-साधारणं च हेतुत्वम्। करणत्वं तु क्रियामात्र\-विषयं व्यापारनियतं च} (वै॰सि॰कौ॰~५६८) इति। अतः तृतीयायां नानुपपत्तिः। यद्वा \textcolor{red}{प्रकृत्यादिभ्य उपसङ्ख्यानम्‌} (वा॰~२.३.१८) इति वार्त्तिक\-बलेन तृतीया। सा चाभेदे।\footnote{\textcolor{red}{धान्येन धनवान्‌} (भा॰पा॰सू॰~२.१.३०) इतिवत्। \textcolor{red}{मेधया तद्वान् धनेन धनवान् इत्यादि में प्रकृत्यादि होने से अभेद में तृतीया है} इति व्याकरण\-चन्द्रोदयस्य प्रथम\-खण्डे कारक\-प्रकरणे ४६तमे पृष्ठे चारुदेव\-शास्त्रिणः (मोतीलाल बनारसीदास, {\englishfont ISBN 978-81-2082-518-5})।} प्रकृत्यादिश्चाऽकृति\-गणः।\footnote{\textcolor{red}{“प्रकृत्यादिभ्य” इति। आकृतिगणोऽयम्। तेन “नाम्ना सुतीक्ष्णश्चरितेन दान्तः” (र॰वं॰~१३.४१) इत्यादि सिद्धम्‌} (त॰बो॰~४९६)।} पक्व\-फलाद्यभिन्नं भोजनं श्रीरामं भोजयित्वेति तात्पर्यम्।\end{sloppypar}
\section[यत्राहल्यास्थिता तपः]{यत्राहल्यास्थिता तपः}
\centering\textcolor{blue}{गौतमस्याश्रमं पुण्यं यत्राहल्यास्थिता तपः।\nopagebreak\\
दिव्यपुष्पफलोपेतपादपैः परिवेष्टितम्॥}\nopagebreak\\
\raggedleft{–~अ॰रा॰~१.५.१५}\\
\begin{sloppypar}\hyphenrules{nohyphenation}\justifying\noindent\hspace{10mm} अत्र \textcolor{red}{स्था}\-धातुर्गति\-निवृत्तौ (\textcolor{red}{ष्ठा गति\-निवृत्तौ} धा॰पा॰~९२८)। अस्यैव \textcolor{red}{क्त}\-प्रत्ययान्तं रूपमिदम्। अत्र तप इति वैषयिक आधारः। अत एव \textcolor{red}{आधारोऽधिकरणम्‌} (पा॰सू॰~१.४.४५) इत्यनेनाधिकरण\-सञ्ज्ञा। ततश्च \textcolor{red}{सप्तम्यधिकरणे च} (पा॰सू॰~१.४.४५) इत्यनेन सप्तमी सङ्गता। \textcolor{red}{तपसि स्थिता} इत्यनेन भवितव्यं किन्तु \textcolor{red}{यत्राऽहल्या} इत्यत्र \textcolor{red}{आकारः} प्रश्लिष्टः। स च \textcolor{red}{आश्रित्य} इत्यस्य सूचकः। अर्थात् \textcolor{red}{तप आश्रित्य स्थिता अहल्या}। यद्वाऽत्राधिरुपसर्गः स च \textcolor{red}{विनाऽपि प्रत्ययं पूर्वोत्तर\-पद\-लोपो वक्तव्यः} (वा॰~५.३.८३) इति वार्त्तिकेन लुप्तत्वात्सम्प्रति न दृश्यते किन्तु पूर्वमासीत्। तत्पूर्वकस्य स्था\-धातोर्योगे \textcolor{red}{तपः अधिष्ठिता} इति स्थिते \textcolor{red}{अधिशीङ्स्थासां कर्म} (पा॰सू॰~१.४.४६) इत्यनेन कर्म\-सञ्ज्ञायां \textcolor{red}{कर्मणि द्वितीया} (पा॰सू॰~२.३.२) इत्यनेन द्वितीया विभक्तौ पश्चादधीत्यस्य लोपे द्वितीया न निवर्तते। \textcolor{red}{जात\-संस्कारो न निवर्तते} इति परिभाषा\-बलेन। अथवा \textcolor{red}{तपःस्थिता} इत्येकं समस्तं पदम्। अत्र \textcolor{red}{सप्तमी शौण्डैः} (पा॰सू॰~२.१.४०) इति सूत्रेण \textcolor{red}{तपसि स्थिता} इति विग्रहे लौकिके \textcolor{red}{तपस् ङि स्थिता सुँ} इत्यलौकिक\-विग्रहे सप्तमी\-तत्पुरुषे \textcolor{red}{कृत्तद्धित\-समासाश्च} (पा॰सू॰~१.२.४६) इत्यनेन प्रातिपदिक\-सञ्ज्ञायां \textcolor{red}{सुपो धातु\-प्रातिपादिकयोः} (पा॰सू॰~२.४.७१) इत्यनेन विभक्ति\-लुकि विभक्त्यादि\-कार्ये \textcolor{red}{तपःस्थिता} इति।\footnote{\textcolor{red}{यत्राहल्या तपःस्थिता} इत्येव मूलपाठ इति तात्पर्यम्।} अथवा \textcolor{red}{तपसि} इति पृथक्पदं \textcolor{red}{स्थिता} इत्यपि पृथगुभे पदे च व्यस्ते तथा \textcolor{red}{सुपां सुलुक्पूर्व\-सवर्णाच्छेयाडाड्यायाजालः} (पा॰सू॰~७.१.३९) इति सूत्रेण छान्दसतया सप्तम्या लुक्। अथवाऽनेनैव सूत्रेण \textcolor{red}{सु}आदेशे पुनर्विसर्गादिः।\footnote{तपस्~ङि~\arrow \textcolor{red}{सुपां सुलुक्पूर्व\-सवर्णाच्छेयाडाड्यायाजालः} (पा॰सू॰~७.१.३९)~\arrow स्वादेशः~\arrow तपस्~सुँ~\arrow \textcolor{red}{स्वमोर्नपुंसकात्} (पा॰सू॰~७.१.२३)~\arrow तपस्~\arrow \textcolor{red}{ससजुषो रुः} (पा॰सू॰~८.२.६६)~\arrow तपरुँ~\arrow तपर्~\arrow \textcolor{red}{खरवसानयोर्विसर्जनीयः} (पा॰सू॰~८.३.१५)~\arrow तपः।} यद्वा \textcolor{red}{तप आस्थिता} इति विग्रहः। \textcolor{red}{आङ्‌}\-पूर्वस्य \textcolor{red}{स्था}\-धातोः करणार्थः। \textcolor{red}{तप आस्थिता} इत्यस्य \textcolor{red}{तपः कुर्वत्यासीत्‌} इत्यर्थः। यथा श्रीमद्भागवते \textcolor{red}{आतिष्ठ तत्तात विमत्सरस्त्वमुक्तं समात्राऽपि यदव्यलीकम्‌} (भा॰पु॰~४.८.१९) इत्यत्र टीकायां श्रीधर\-स्वामिनः \textcolor{red}{आतिष्ठ कुरु} (भा॰पु॰ श्री॰टी॰~४.८.१९) वंशीधराश्च \textcolor{red}{यत्तप उक्तं तत्कुरु} (भा॰पु॰ वं॰टी॰~४.८.१९) इति। \end{sloppypar}
\section[देवराजानम्]{देवराजानम्‌}
\label{sec:devarajanam}
\centering\textcolor{blue}{योनिलम्पट दुष्टात्मन्सहस्रभगवान्भव।\nopagebreak\\
शप्त्वा तं देवराजानं प्रविश्य स्वाश्रमं द्रुतम्॥}\nopagebreak\\
\raggedleft{–~अ॰रा॰~१.५.२६}\\
\begin{sloppypar}\hyphenrules{nohyphenation}\justifying\noindent\hspace{10mm} अत्राहल्याभिमर्श\-सञ्जात\-रोषो गौतमः कृत\-किल्बिषं पुरन्दरं क्रुद्धः शपति। अत्र \textcolor{red}{देवराजानम्‌} इति प्रयोगः कथम्। यतो हि \textcolor{red}{देवानां राजा} इति विग्रहे षष्ठी\-तत्पुरुषे \textcolor{red}{राजाऽहस्सखिभ्यष्टच्‌} (पा॰सू॰~५.४.९१) इत्यनेन टच्प्रत्यये \textcolor{red}{चुटू} (पा॰सू॰~१.३.७) इत्यनेन टकारेत्सञ्ज्ञायां लोपे चकारस्याप्यनुबन्ध\-कार्ये भत्वात् \textcolor{red}{अन्‌} इत्यस्य लोपेऽमि \textcolor{red}{देवराजम्‌}।\footnote{देवानां राजा देवराजस्तम्~\arrow देव~आम्~राजन्~अम्~\arrow \textcolor{red}{सुपो धातुप्रातिपदिकयोः} (पा॰सू॰~२.४.७१)~\arrow देव~राजन्~अम्~\arrow देवराजन्~अम्~\arrow \textcolor{red}{राजाऽहस्सखिभ्यष्टच्‌} (पा॰सू॰~५.४.९१)~\arrow देवराजन्~टच्~अम्~\arrow देवराजन्~अ~अम्~\arrow \textcolor{red}{यचि भम्‌} (पा॰सू॰~१.४.१८)~\arrow भसञ्ज्ञा~\arrow \textcolor{red}{नस्तद्धिते} (पा॰सू॰~६.४.४४)~\arrow देवराज्~अ~अम्~\arrow देवराज~अम्~\arrow \textcolor{red}{अमि पूर्वः} (पा॰सू॰~६.१.१०७)~\arrow देवराजम्।} परञ्च विचारे कृत इदमपि साधु। \textcolor{red}{साधुत्वञ्चाऽत्र वृत्त्यप्रवृत्त\-नित्य\-विध्युद्देश्यतावच्छेदकतानाक्रान्तत्वम्‌}। एवं हि \textcolor{red}{देवानां राजा} इति विग्रहेऽपि कथं न टच् प्रत्यय इत्यपेक्षायां समासान्त\-प्रत्यया अनित्या इत्येव समाधानम्।\footnote{\pageref{sec:sthapya}तमे पृष्ठे \ref{sec:sthapya} \nameref{sec:sthapya} इति प्रयोगस्य विमर्शं पश्यन्तु~– “समासान्त\-प्रत्यय\-प्रकरणं ह्यनित्यम्। प्रमाणं चात्र \textcolor{red}{यचि भम्‌} (पा॰सू॰~१.४.१८) इति सूत्रम्। अत्र \textcolor{red}{यश्चाच्च यच्‌} इति समाहार\-द्वन्द्वः। इह \textcolor{red}{द्वन्द्वाच्चु\-दषहान्तात्समाहारे} (पा॰सू॰~५.४.१०६) इत्यनेन चान्तत्वाट्टच्प्रत्ययः प्रयोक्तव्य आसीत्। तस्मिन् प्रयुक्ते \textcolor{red}{यचे भम्‌} इति स्यात्। यतो न प्रयुक्तोऽतः समासान्त\-प्रत्यस्यानित्यता ज्ञायते।”} \end{sloppypar}
\begin{sloppypar}\hyphenrules{nohyphenation}\justifying\noindent\hspace{10mm} यद्वा \textcolor{red}{राजृँ दीप्तौ} (धा॰पा॰~८२२) इत्यस्माद्धातोः \textcolor{red}{राजनं राट्‌} इति विग्रहे भावे क्विप्।\footnote{\textcolor{red}{सम्पदादिभ्‍यः क्विप्‌} (वा॰~३.३.१०८) इत्यनेन।} तस्य सर्वापहारि\-लोपः। पश्चात् \textcolor{red}{व्रश्च\-भ्रस्ज\-सृज\-मृज\-यज\-राज\-भ्राजच्छशां षः} (पा॰सू॰~८.२.३६) इति सूत्रेण मूर्धन्य\-षकारो \textcolor{red}{झलां जशोऽन्ते} (पा॰सू॰~८.२.३९) इत्यनेन जश्त्वं \textcolor{red}{वाऽवसाने} (पा॰सू॰~८.४.५६) इत्यनेन वैकल्पिक\-चर्त्वमित्थं \textcolor{red}{देवराट्‌} इति निष्पन्न\-प्रयोग\-स्थितिः। किन्तु \textcolor{red}{राजनं राट्‌} इति विग्रहे क्विप्प्रत्ययान्तस्य \textcolor{red}{राज्‌} शब्दस्य \textcolor{red}{देव} शब्देन सह समासे तेन च \textcolor{red}{देव\-राजाऽऽसमन्तादनिति} देव\-शासनेन निश्वसिति पद\-लोलुपतया सम्मानाकाङ्क्षिततया सुखं निश्वसितीति विग्रहे \textcolor{red}{आङ्‌}\-उपसर्ग\-पूर्वकात् \textcolor{red}{अन्‌}\-धातोः (\textcolor{red}{अनँ प्राणने}, धा॰पा॰~१०७०) \textcolor{red}{नन्दि\-ग्रहि\-पचादिभ्यो ल्युणिन्यचः} (पा॰सू॰~३.१.१३४) इत्यनेन \textcolor{red}{अच्‌}\-प्रत्ययः पश्चात् \textcolor{red}{तृतीया तत्कृतार्थेन गुण\-वचनेन} (पा॰सू॰~२.१.३०) इति सूत्रे \textcolor{red}{तृतीया} इति योग\-विभाग\-बलात् \textcolor{red}{देवराजा} इति शब्दस्य \textcolor{red}{आन} इति शब्देन तृतीया\-तत्पुरुषः। अथवा \textcolor{red}{देवराजे सुर\-शासनायाऽनिति} इति विग्रहे चतुर्थी\-तत्पुरुषः। अथवा \textcolor{red}{देवराज्यानिति} निर्भरतया जीवतीति विग्रहे सप्तमी\-तत्पुरुषः। अथवा \textcolor{red}{देवराज इदं देवराजार्थम्‌} इति विग्रहे \textcolor{red}{अर्थेन नित्य\-समासो विशेष्य\-लिङ्गता चेति वक्तव्यम्‌} (वा॰~२.१.३६) इति वार्त्तिक\-बलेन चतुर्थ्यन्त \textcolor{red}{देवराजे} शब्दस्य \textcolor{red}{अर्थ} शब्देन नित्य\-चतुर्थी\-समासः। पश्चात् \textcolor{red}{देवराजार्थमानिति} इति विग्रहे \textcolor{red}{देवराजार्थ} शब्दस्य \textcolor{red}{आन} शब्देन सह \textcolor{red}{सुप्सुपा} (पा॰सू॰~२.१.४) इति सूत्रेण समासः। एवं शब्दमिममाकृति\-गणत्वाच्छाक\-पार्थिवादि\-गणे मत्वा \textcolor{red}{शाकप्रियः पार्थिवः शाकपार्थिवः} इतिवत् \textcolor{red}{शाक\-पार्थिवादीनां सिद्धय उत्तर\-पद\-लोपस्योप\-सङ्ख्यानम्‌} (वा॰~२.१.६०) इति वार्त्तिकेन \textcolor{red}{अर्थ}शब्दस्य लोपे विभक्ति\-कार्ये \textcolor{red}{देवराजानम्‌} इति पूर्णतया पाणिन्यनुकूलम्। एतादृक्समास\-प्रकारस्तु पस्पशाह्निकेऽन्वमूमुदन्महा\-भाष्यकाराः सामोदम्। तत्रायं विचारः समुपस्थितो यद्यदि सिद्धः शब्दोऽर्थः सम्बन्धश्चेति लोकतो ज्ञायते। यदि लोक एषु प्रमाणं तर्हि किं शास्त्रेण क्रियत इत्यपेक्षायां वार्त्तिकमव\-तारयामासुः प्राञ्जलयः पतञ्जलयः यत् \textcolor{red}{लोकतोऽर्थप्रयुक्ते शब्द\-प्रयोगे शास्त्रेण धर्मनियमः}। धर्म\-नियम\-शब्दे बहवः समास\-प्रकाराः प्रदर्शयाम्बभूविरे। \textcolor{red}{किमिदं धर्मनियम इति। धर्माय नियमो धर्मनियमः। धर्मार्थो वा नियमो धर्मनियमः। धर्मप्रयोजनो वा नियमो धर्मनियमः} (भा॰प॰) इत्यादि। इत्थं \textcolor{red}{देवराजानम्‌} इति त्रि\-मुनि\-सम्मतम्। यद्वा \textcolor{red}{अन्‌} धातुमन्तर्भावित\-णिजन्तार्थमङ्गीकृत्य \textcolor{red}{देव\-राजं देव\-शासनमनित्यानयति} श्वासयतीति भावो निज\-प्रताप\-बलेन जीवयतीति हार्दं विग्रहेऽस्मिन्। \textcolor{red}{तत्रोपपदं सप्तमीस्थम्‌} (पा॰सू॰~३.१.९२) इत्यनेन उपपद\-सञ्ज्ञायां \textcolor{red}{कर्मण्यण्‌} (पा॰सू॰~३.२.१) इत्यनेन \textcolor{red}{अण्‌} प्रत्यये \textcolor{red}{अत उपधायाः} (पा॰सू॰~७.२.११६) इत्यनेन वृद्धौ \textcolor{red}{उपपदमतिङ्‌} (पा॰सू॰~२.२.१९) इति सूत्र\-बलेन समासे विभक्ति\-कार्ये \textcolor{red}{देवराजानम्‌} इति सम्यक्सिद्धम्।\end{sloppypar}
\section[पुलकाङ्कितसर्वाङ्गा]{पुलकाङ्कितसर्वाङ्गा}
\centering\textcolor{blue}{उत्थाय च पुनर्दृष्ट्वा रामं राजीवलोचनम्।\nopagebreak\\
पुलकाङ्कितसर्वाङ्गा गिरा गद्गदयैलत॥}\nopagebreak\\
\raggedleft{–~अ॰रा॰~१.५.४२}\\
\begin{sloppypar}\hyphenrules{nohyphenation}\justifying\noindent\hspace{10mm} श्रीराम\-चरणारविन्द\-रजः\-संस्पर्श\-लब्ध\-ललित\-ललना\-शरीरा कलित\-लोचन\-नीरा धीरा विगत\-शल्याऽहल्या कौसल्या\-सुतं श्रीरामं पुलक\-पूर्णाङ्गी स्तौति। अत्रैव \textcolor{red}{पुलकाङ्कित\-सर्वाङ्गा} इति प्रयोगोऽपाणिनीय इव। \textcolor{red}{पुलकेन अङ्कितानि सर्वाणि अङ्गानि यस्याः} इति विग्रहे \textcolor{red}{अनेकमन्य\-पदार्थे} (पा॰सू॰~२.२.२४) इत्यनेन चतुष्पदे बहुव्रीहौ
\textcolor{red}{स्वाङ्गाच्चोप\-सर्जनादसंयोगोपधात्‌} (पा॰सू॰~४.१.५४) इत्यस्योपरि \textcolor{red}{अङ्ग\-गात्र\-कण्ठेभ्य इति वक्तव्यम्} (का॰वृ॰वा॰~४.१.५४) इत्यनेन प्राप्तः \textcolor{red}{ङीष्‌} दुर्वार एव।\footnote{\textcolor{red}{अङ्ग\-गात्र\-कण्ठेभ्य इति वक्तव्यम्} इति वार्त्तिकं \textcolor{red}{स्वाङ्गाच्चोप\-सर्जनादसंयोगोपधात्‌} (पा॰सू॰~४.१.५४) इति सूत्रे काशिकायां पठितम्। \textcolor{red}{भाष्यादर्शनादप्रमाणमिदम्} इति \textcolor{red}{नासिकोदरौष्ठ\-जङ्घादन्त\-कर्णशृङ्गाच्च} (पा॰सू॰~४.१.५५) सूत्रे भट्टोजि\-दीक्षिताः~– \textcolor{red}{अत्र वृत्तिः। अङ्गगात्र\-कण्ठेभ्य इति वक्तव्यम्। स्वङ्गी स्वङ्गेत्यादि। एतच्चानुक्त\-समुच्चयार्थेन चकारेण सङ्ग्राह्यमिति केचित्। भाष्याद्यनुक्तत्वादप्रमाणमिति प्रामाणिकाः} (वै॰सि॰कौ॰~५११)। बाल\-मनोरमायामपि \textcolor{red}{एवञ्च तन्वङ्गी सुगात्री कलकण्ठी इत्यपभ्रंशा एवेति भावः} (बा॰म॰~५११)। परन्त्वेतेन \textcolor{red}{अनवद्याङ्गि} (म॰भा॰~३.६४.७२) \textcolor{red}{अवनताङ्गि} (कु॰स॰~५.८६) इत्यादयः शिष्ट\-प्रयोगा न सङ्गच्छन्ते। अत एव पदमञ्जर्यां हरदत्ताः~– \textcolor{red}{अङ्गगात्रेत्यादि भाष्येऽनुक्तमप्येतत्प्रयोग\-बाहुल्याद्वृत्तिकारेणोक्तम्} (प॰म॰~४.१.५४)।}
एवञ्च \textcolor{red}{पुलकाङ्कित\-सर्वाङ्गा} इत्यत्र \textcolor{red}{टाप्‌} कथमिति चेत्। उच्यते। अत्र हि सिद्धान्त\-कौमुदी\-वर्णनम्~– \textcolor{red}{स्वाङ्गं त्रिधा। अद्रवं मूर्तिमत्स्वाङ्गं प्राणिस्थमविकारजम्‌} (वै॰सि॰कौ॰~५१०)।\footnote{अस्य भाष्ये मूलम्~– \textcolor{red}{किं स्वाङ्गं नाम। अद्रवं मूर्तिमत्स्वाङ्गं प्राणिस्थमविकारजम्। अतत्स्थं तत्र दृष्टं च तस्य चेत्तत्तथा युतम्॥} (भा॰पा॰सू॰~४.१.५४)।} तथाऽपि \textcolor{red}{ङीष्‌}\-भावो वैकल्पिकः। \textcolor{red}{स्वाङ्गाच्चोप\-सर्जनादसंयोगोपधात्‌} (पा॰सू॰~४.१.५४) इति सूत्रं\footnote{तेन \textcolor{red}{अङ्ग\-गात्र\-कण्ठेभ्य इति वक्तव्यम्} इति वार्त्तिकं च।} हि वैकल्पिकं \textcolor{red}{ङीष्‌}\-प्रत्ययं करोति।\footnote{\textcolor{red}{वाग्रहणम् अनुवर्तते। स्वाङ्गं यदुपसर्जनमसंयोगोपधं तदन्तात्प्राति\-पदिकात्स्त्रियां वा ङीष् प्रत्ययो भवति} (का॰वृ॰~४.१.५४)। \textcolor{red}{असंयोगोपधमुपसर्जनं यत्स्वाङ्गं तदन्ताददन्तात्प्राति\-पदिकाद्वा ङीष्} (वै॰सि॰कौ॰~५११)। \textcolor{red}{असंयोगोपधमुपसर्जनं यत्स्वाङ्गं तदन्ताददन्तात् ङीष् वा स्यात्} (ल॰सि॰कौ॰~१२६८)।} यद्यपि~–\end{sloppypar}
\centering\textcolor{red}{अद्य प्रभृत्यवनताङ्गि तवास्मि दासः\nopagebreak\\
क्रीतस्तपोभिरिति वादिनि चन्द्रमौलौ।}\nopagebreak\\
\raggedleft{–~कु॰स॰~५.८६}\\
\begin{sloppypar}\hyphenrules{nohyphenation}\justifying\noindent इति कुमारसम्भवे कालिदासः \textcolor{red}{ङीष्‌}\-प्रत्ययान्तमेवाश्रयति।\footnote{एवमेव भारते वनपर्वणि दमयन्तीं प्रति महात्मनः~– \textcolor{red}{ब्रूहि सर्वानवद्याङ्गि का त्वं किं च चिकीर्षसि} (म॰भा~३.६४.७२)।} अत्र कथं नाऽश्रित इति चेत्। कवीनां कामचारः। यद्वा विकल्प\-बुद्धिमतीमहल्यां वर्णयन् विकल्प\-पक्षमेवाश्रयति।\footnote{यद्वा भाष्येऽनुक्तत्वात् \textcolor{red}{अङ्ग\-गात्र\-कण्ठेभ्य इति वक्तव्यम्} इत्यप्रमाणम्। ततः संयोगोपधात् \textcolor{red}{स्वाङ्गाच्चोप\-सर्जनादसंयोगोपधात्‌} (पा॰सू॰~४.१.५४) इत्यस्याप्रवृत्तौ \textcolor{red}{अजाद्यतष्टाप्‌} (पा॰सू॰~४.१.४) इत्यनेन टाबेव।} यद्वा नायं स्त्री\-प्रत्ययान्तोऽपि तु कर्मधारयान्मत्वर्थीयः। \textcolor{red}{पुलकेनाङ्कितानि पुलकाङ्कितानि पुलकाङ्कितानि च तानि सर्वाण्यङ्गानि} इति कर्मधारयः। \textcolor{red}{विशेषणं विशेष्येण बहुलम्‌} (पा॰सू॰~२.१.५७) इत्यनेन पुलकाङ्कित\-शब्दस्य विशेषणतया पूर्व\-निपातः। \textcolor{red}{विशेषणत्वं नाम विद्यमानत्वे सति विधेयान्वयित्वे सतीतर\-व्यावर्तकत्वम्‌}। एवं \textcolor{red}{पुलकाङ्कित\-सर्वाङ्गाणि सन्ति यस्याः}~– \textcolor{red}{प्रशस्तानि} इति शेषः~– इति विग्रहे \textcolor{red}{अर्शआदिभ्योऽच्‌} (पा॰सू॰~५.२.१२७) इत्यनेन \textcolor{red}{अच्‌}\-प्रत्यये टापि प्रत्यये\footnote{\textcolor{red}{अजाद्यतष्टाप्‌} (पा॰सू॰~४.१.४) इत्यनेन टाप्।} पुलकाङ्कितसर्वाङ्गा इति दिक्।\footnote{बहुव्रीहौ “प्रशस्तानि” इत्यार्थस्याम्भवात् \textcolor{red}{न कर्मधारयान्मत्वर्थीयो बहुव्रीहिश्चेत्तदर्थ\-प्रतिपत्तिकरः} इति नियमस्य न प्रसरः।}\end{sloppypar}
\section[रमणीयदेहिनम्]{रमणीयदेहिनम्‌}
\centering\textcolor{blue}{मर्त्यावतारे मनुजाकृतिं हरिं रामाभिधेयं रमणीयदेहिनम्।\nopagebreak\\
धनुर्धरं पद्मविशाललोचनं भजामि नित्यं न परान्भजिष्ये॥}\nopagebreak\\
\raggedleft{–~अ॰रा॰~१.५.४६}\\
\begin{sloppypar}\hyphenrules{nohyphenation}\justifying\noindent\hspace{10mm} शाप\-मुक्ता भक्ति\-युक्ता विगत\-शल्याऽहल्या कौशल्यानन्द\-वर्धनं स्तुवती\footnote{अत्र नुमागमो न। \textcolor{red}{ष्टुञ् स्तुतौ} (धा॰पा॰~१०४३)~\arrow \textcolor{red}{धात्वादेः षः सः} (पा॰सू॰~६.१.६४)~\arrow \textcolor{red}{निमित्तापाये नैमित्तिकस्याप्यपायः}~\arrow स्तु~\arrow \textcolor{red}{वर्तमाने लँट्‌} (पा॰सू॰~३.२.१२३)~\arrow स्तु~लँट्~\arrow \textcolor{red}{लटः शतृ\-शानचावप्रथमा\-समानाधिकरणे} (पा॰सू॰~३.२.१२४)~\arrow स्तु~शतृँ~\arrow \textcolor{red}{तिङ्शित्सार्वधातुकम्‌} (पा॰सू॰~३.४.११३)~\arrow \textcolor{red}{सार्वधातुकमपित्‌} (पा॰सू॰~१.२.४)~\arrow ङिद्वत्त्वम्~\arrow स्तु~अत्~\arrow \textcolor{red}{कर्तरि शप्‌} (पा॰सू॰~३.१.६८)~\arrow स्तु~शप्~अत्~\arrow \textcolor{red}{अदिप्रभृतिभ्यः शपः} (पा॰सू॰~२.४.७२)~\arrow स्तु~अत्~\arrow \textcolor{red}{ग्क्ङिति च} (पा॰सू॰~१.१.५)~\arrow गुणनिषेधः~\arrow \textcolor{red}{अचि श्नुधातुभ्रुवां य्वोरियँङुवँङौ} (पा॰सू॰~६.४.७७)~\arrow स्त्~उवँङ्~अत्~\arrow स्त्~उव्~अत्~\arrow स्तुवत्~\arrow \textcolor{red}{उगितश्च}~\arrow स्तुवत्~ङीप्~\arrow स्तुवत्~ई~\arrow स्तुवती~\arrow विभक्तिकार्यम्~\arrow स्तुवती~सुँ~\arrow \textcolor{red}{हल्ङ्याब्भ्यो दीर्घात्सुतिस्यपृक्तं हल्‌} (पा॰सू॰~६.१.६८)~\arrow स्तुवती।} श्रीराम\-विशेषणं \textcolor{red}{रमणीय\-देहिनम्‌} इति\-शब्दं प्रायुङ्क्त। अत्र \textcolor{red}{रमणीयश्चासौ देहश्चेति रमणीय\-देहः}। \textcolor{red}{विशेषणं विशेष्येण बहुलम्‌} (पा॰सू॰~२.१.५७) इति कर्मधारयः। \textcolor{red}{रमणीय\-देहो नित्यत्वेन प्राशस्त्येन वाऽस्ति यस्य स रमणीय\-देही} तं रमणीय\-देहिनम्। विचारणीयमिदं यत् \textcolor{red}{रमणीयो देहो यस्य स रमणीयदेहस्तथाभूतम्‌} इत्थं विग्रहेऽपि रमणीय\-देहवत्त्व\-रूपोऽर्थोऽवगम्येत। कर्मधारयानन्तरं मत्वर्थीयस्तु पाणिनीय\-तन्त्र\-विरुद्धः। \textcolor{red}{न कर्मधारयान्मत्वर्थीयो बहुव्रीहिश्चेत्तदर्थ\-प्रतिपत्ति\-करः} इति नियमेनात्र कर्मधारयादनन्तरं बहुव्रीहिरनुचित इति चेत्सत्यं किन्तु यदि बहुव्रीहौ विवक्षितार्थस्य प्रतीतिः स्यात्तदा कर्मधारयान्मत्वर्थीयो न। परमत्र मत्वर्थीयो नित्ययोगे प्राशस्त्ये च। तथा च कारिकां पेठुर्महाभाष्य\-कारा यत्~–\end{sloppypar}
\centering\textcolor{red}{भूमनिन्दाप्रशंसासु नित्ययोगेऽतिशायने।\nopagebreak\\
सम्बन्धेऽस्तिविवक्षायां भवन्ति मतुबादयः॥}\nopagebreak\\
\raggedleft{–~भा॰पा॰सू॰~५.२.९४}\\
\begin{sloppypar}\hyphenrules{nohyphenation}\justifying\noindent\hspace{10mm} भगवतो रमणीयो देहो नित्यः। कौशल्यायाः समक्षन्तु केवलं प्रकटस्तथा वाल्मीकीयेऽपि \textcolor{red}{कौसल्याजनयद्रामम्‌} (वा॰रा॰~१.१८.१०) इत्येव लिखितं न तु \textcolor{red}{बालम्‌} इत्यनेन पुराऽपि राम आसीदित्येव सूच्यते। अध्यात्म\-रामायणे तु शङ्ख\-चक्र\-गदा\-पद्म\-वन\-माला\-विभूषितं चतुर्भुज\-रूपमेवादर्शयत्।\footnote{\textcolor{red}{शङ्ख\-चक्र\-गदा\-पद्म\-वन\-माला\-विराजितः} (अ॰रा॰~१.३.१७)।} श्रीमानसे तु मनु\-शतरूपा\-समक्षं शर\-चाप\-युक्त\-द्वि\-भुजरूप एवाऽगच्छत्। भगवतो द्विभुजं रूपं शाश्वतं चेति श्रुति\-पुराण\-शास्त्र\-सम्मतम्। भागवतेऽपि ब्रह्मा कथयति~–\end{sloppypar}
\centering\textcolor{red}{अस्याऽपि देव वपुषो मदनुग्रहस्य स्वेच्छामयस्य न तु भूतमयस्य कोऽपि।}\nopagebreak\\
\raggedleft{–~भा॰पु॰~१०.१४.२}\\
\begin{sloppypar}\hyphenrules{nohyphenation}\justifying\noindent तुलसीदासोऽपि साटोपं कथयति~–\end{sloppypar}
\centering\textcolor{red}{चिदानन्दमय देह तुम्हारी। बिगत बिकार जान अधिकारी॥}\footnote{एतद्रूपान्तरम्–\textcolor{red}{भवदीयं चिदानन्द\-मयमस्ति कलेवरम्। तथा विकार\-रहितं विजानान्त्यधि\-कारिणः॥} (मा॰भा॰~२.१२७.५)।}\nopagebreak\\
\raggedleft{–~रा॰च॰मा॰~२.१२७.५}\\
\begin{sloppypar}\hyphenrules{nohyphenation}\justifying\noindent इति। इत्थं नित्य\-सिद्धस्य भगवतो देहस्य प्रतीतिर्बहु\-व्रीहावसम्भवान्मत्वर्थीयमन्तरेण प्राशस्त्यमपि नैवावगन्तुं शक्यते। अतोऽसत्यां तादृशार्थ\-प्रतीतौ मत्वर्थीयो वरीयान्। यद्वा \textcolor{red}{दिहँ उपचये} (धा॰पा॰~१०१५) इत्यस्माद्धातोस्ताच्छील्ये णिनिः। तथा च \textcolor{red}{रमणीयं देग्धुं तच्छीलः} इति विग्रहे कर्म\-भूत\-रमणीयोपपदे \textcolor{red}{सुप्यजातौ णिनिस्ताच्छील्ये} (पा॰सू॰~३.२.७८) इत्यनेन णिनि\-प्रत्ययेऽनुबन्ध\-कार्ये गुणेऽमि \textcolor{red}{रमणीय\-देहिनम्‌} इति।\footnote{रमणीयं देग्धुं तच्छीलो रमणीय\-देही तं रमणीय\-देहिनम्। रमणीय~अम्~दिह्~\arrow \textcolor{red}{सुप्यजातौ णिनिस्ताच्छील्ये} (पा॰सू॰~३.२.७८)~\arrow रमणीय~अम्~दिह्~णिनिँ~\arrow रमणीय~अम्~दिह्~इन्~\arrow \textcolor{red}{पुगन्त\-लघूपधस्य च} (पा॰सू॰~७.३.८६)~\arrow \textcolor{red}{रमणीय~अम्~देह्~इन्‌}~\arrow \textcolor{red}{कृत्तद्धित\-समासाश्च} (पा॰सू॰~१.२.४६)~\arrow प्रातिपदिक\-सञ्ज्ञा~\arrow \textcolor{red}{सुपो धातु\-प्रातिपदिकयोः} (पा॰सू॰~२.४.७१)~\arrow रमणीय~देह्~इन्~\arrow रमणीयदेहिन्। विभक्तिकार्ये रमणीयदेहिन् अम्~\arrow रमणीयदेहिनम्।} अथवा \textcolor{red}{रमणीया देहिनो मत्स्यादयश्चतुर्विंशतिरवतारा यस्य स रमणीय\-देही तं रमणीय\-देहिनम्‌}। श्रीरामस्यैव सर्वावतारित्वात्।\footnote{यथा \textcolor{red}{सर्वेषामवताराणामवतारी रघूत्तमः। रामपादनखज्योत्स्ना परब्रह्मेति गीयते॥} (अग॰सं॰) \textcolor{red}{तस्मिन्साकेतलोके विधिहरहरिभिः सन्ततं सेव्यमाने दिव्ये सिंहासने स्वे जनकतनयया राघवः शोभमानः। युक्तो मत्स्यैरनेकैः करिभिरपि तथा नारसिंहैरनन्तैः कूर्मैः श्रीनन्दनन्दैर्हयगलहरिभिर्नित्यमाज्ञोन्मुखैश्च॥ यज्ञः केशववामनौ नरवरो नारायणो धर्मजः श्रीकृष्णो हलधृक् तथा मधुरिपुः श्रीवासुदेवोऽपरः। एते नैकविधा महेन्द्रविधयो दुर्गादयः कोटिशः श्रीरामस्य पुरो निदेशसुमुखा नित्यास्तदीये पदे॥} (बृ॰ब्र॰सं॰) \textcolor{red}{नारायणोऽपि रामांशः शङ्खचक्रगदाधरः} (वा॰सं॰) इत्यादिषु स्पष्टम्। एते वाल्मीकीय\-रामायण\-शिरोमणि\-टीकाया मङ्गलाचरणे शिवसहाय\-महाभागैरुद्धृताः।} अथवा \textcolor{red}{कृत्य\-ल्युटो बहुलम्‌} (पा॰सू॰~३.१.११३) इति सूत्रेण \textcolor{red}{रमन्ते सर्वाकृतयो येषु ते रमणीयाः} इति विग्रहे \textcolor{red}{रमुँ क्रीडायाम्‌} (धा॰पा॰~८५३) इत्यस्माद्धातोरधिकरणे \textcolor{red}{अनीयर्‌} प्रत्ययः। एवं \textcolor{red}{रमणीया देहिनो जीवात्मानो यस्य स रमणीयदेही तम्‌}। षष्ठ्यर्थश्चांशांशि\-भाव\-रूपः। जीवात्मा परमात्मनोंऽश इति सार्वजनीनत्वात्। \textcolor{red}{ममैवांशो जीव\-लोके जीवभूतः सनातनः} (भ॰गी॰~१५.७) इति गीतोक्तेः। अथवा \textcolor{red}{रमन्ते इति रमणीयाः}। \textcolor{red}{कृत्य\-ल्युटो बहुलम्‌} (पा॰सू॰~३.१.११३) इत्यनेन कर्तरि \textcolor{red}{अनीयर्‌}। \textcolor{red}{नित्यं रमणशीला जीवात्मानो देहिनो मुक्त\-नित्या यस्मिन्स रमणीय\-देही तं रमणीय\-देहिनम्‌}।\end{sloppypar}
\centering\textcolor{blue}{यस्मिन् रमन्ते मुनयो विद्ययाऽज्ञानविप्लवे।\nopagebreak\\
तं गुरुः प्राह रामेति रमणाद्राम इत्यपि॥}\nopagebreak\\
\raggedleft{–~अ॰रा॰~१.३.४०}\\
\begin{sloppypar}\hyphenrules{nohyphenation}\justifying\noindent इत्यत्रैव ग्रन्थ उक्तत्वात्। तस्मादिदं पाणिनीयमेव।\end{sloppypar}
\section[तस्मात्ते]{तस्मात्ते}
\centering\textcolor{blue}{योषिन्मूढाहमज्ञा ते तत्त्वं जाने कथं विभो।\nopagebreak\\
तस्मात्ते शतशो राम नमस्कुर्यामनन्यधीः॥}\nopagebreak\\
\raggedleft{–~अ॰रा॰~१.५.५७}\\
\begin{sloppypar}\hyphenrules{nohyphenation}\justifying\noindent\hspace{10mm} \textcolor{red}{ते नमस्कुर्याम्‌} इत्यन्वयः। अत्र \textcolor{red}{उपपद\-विभक्तेः कारक\-विभक्तिर्बलीयसी} (भा॰पा॰सू॰~१.४.९५, २.३.४, ३.१.१९) इति परिभाषा\-बलेन द्वितीयैव सम्भवति। कदाचित्समासाभावः स्यात्तदा विसर्गस्य सत्वं न स्यात्। \textcolor{red}{नमस्पुरसोर्गत्योः} (पा॰सू॰~८.३.४०) इति सूत्रेण साम्प्रतं \textcolor{red}{साक्षात्प्रभृतीनि च} (पा॰सू॰~१.४.७४) इत्यनेन गति\-सञ्ज्ञायां विसर्गस्य सत्वेऽत्रैकार्थी\-भावः। \textcolor{red}{एकार्थी\-भावत्वञ्च पृथगर्थानामेकोपस्थित्या बोध\-जनकत्वम्‌}। यथा \textcolor{red}{राज\-पुरुषः} इत्यादौ समासात्प्राग्व्यस्तावस्थायां \textcolor{red}{राज्ञः} इत्यस्य पृथगर्थः \textcolor{red}{पुरुषः} इत्यस्य च पृथक्। सति समास उभयोरपि सत्ता समाप्ता साम्प्रतं हि \textcolor{red}{स्व\-स्वामि\-भाव\-सम्बन्धावच्छिन्न\-राज\-विशिष्ट\-पुरुषः} इत्येकोऽर्थः। तथा चोक्तं वैयाकरण\-भूषण\-सारे समास\-शक्ति\-निर्णये~–\end{sloppypar}
\centering\textcolor{red}{समासे खलु भिन्नैव शक्तिः पङ्कजशब्दवत्॥}\nopagebreak\\
\raggedleft{–~वै॰भू॰सा॰~५.३१, वै॰सि॰का॰~३१}\\
\begin{sloppypar}\hyphenrules{nohyphenation}\justifying\noindent\hspace{10mm} एवमत्रापि \textcolor{red}{कृ}धातोः (\textcolor{red}{डुकृञ् करणे} धा॰पा॰~१४७२) करणानुकूल\-व्यापारे शक्तिर्नास्ति न च \textcolor{red}{नमस्‌} इति शब्देऽवनत्यर्थ\-बोधिका सति समासे \textcolor{red}{परनिष्ठोत्कृष्टत्व\-विशिष्ट\-स्वनिष्ठापकृष्टत्वानुकूल\-व्यापारो नमस्कार\-पदार्थः}। अत्र च कारक\-विभक्तेर्बलवत्वाद्द्वितीयैव। \textcolor{red}{ते} तुभ्यं \textcolor{red}{नमस्कुर्याम्‌} इति कथम्। अत्रोच्यते। \textcolor{red}{स्वयम्भुवे नमस्कृत्य} (म॰स्मृ॰~१.१, भा॰पु॰~४.६.२) इत्यादिवत् \textcolor{red}{त्वामनुकूलयितुं प्रसादयितुं स्तोतुं वा नमस्कुर्याम्‌} इति \textcolor{red}{क्रियार्थोपपदस्य च कर्मणि स्थानिनः} (पा॰सू॰~२.३.१४) इत्यनेन चतुर्थी। यद्वा \textcolor{red}{चरणौ} इत्यध्याहृत्यावयावयवि\-भाव\-मूलक\-सम्बन्धे षष्ठी।
इति शम्।\end{sloppypar}
\section[कुटुम्बहानिः]{कुटुम्बहानिः}
\centering\textcolor{blue}{पादाम्बुजं ते विमलं हि कृत्वा पश्चात्परं तीरमहं नयामि।\nopagebreak\\
नो चेत्तरी सद्युवती मलेन स्याच्चेद्विभो विद्धि कुटुम्बहानिः॥}\nopagebreak\\
\raggedleft{–~अ॰रा॰~१.६.४}\\
\begin{sloppypar}\hyphenrules{nohyphenation}\justifying\noindent\hspace{10mm} अहल्योद्धारानन्तरं भू\-भार\-हारी हरिर्गङ्गामुत्तर्तुं निषादं नौकां याचते। कैवर्तकः केशवमप्रस्तुत\-प्रशंसाभङ्ग्याऽऽक्षिपन्नाह यत्त्वच्चरण\-कमल\-रजसा शिला नारी\-रूपेण परिणता। तत्स्पर्शेन मम नौकाऽपि नारी मा भूदिति कृत्वा चरणौ क्षालयित्वैव गङ्गा\-पारं नेष्यामि। अत्रैव कथयति \textcolor{red}{स्याच्चेद्विभो विद्धि कुटुम्ब\-हानिः}। अत्र ज्ञानार्थको \textcolor{red}{विद्‌}\-धातुः (\textcolor{red}{विदँ ज्ञाने} धा॰पा॰~१०६४)। स च सकर्मकः। ज्ञानस्य सविषयकत्वाद्विषयतया कर्मणः संस्कृते पाश्चात्य\-भाषायां सङ्कीर्तनात्। कर्मैव पाश्चात्य\-भाषायाम् \textcolor{red}{ऑब्जेक्ट्‌} शब्देन व्यवह्रियते। अतः \textcolor{red}{कुटुम्बहानिम्‌} इति द्वितीयया भवितव्यमिति चेत्सत्यम्। वैयाकरण\-सिद्धान्त\-कौमुद्याः कारके द्वितीया\-विभक्ति\-प्रकरणेऽभिधान\-चर्चायां दीक्षित\-महाभागाः संलिखन्ति यत् \textcolor{red}{अभिधानञ्च प्रायेण तिङ्कृत्तद्धित\-समासैः। तिङ् हरिः सेव्यते। कृत् लक्ष्म्या सेवितः। तद्धितः शतेन क्रीतः शत्यः। समासे प्राप्त आनन्दो यं स प्राप्तानन्दः। क्वचिन्निपातेनाप्यभिधानम्। विष\-वृक्षोऽपि संवर्ध्य स्वयं छेत्तुमसाम्प्रतम्। असाम्प्रतमित्यस्य हि न युज्यते इत्यर्थः} (वै॰सि॰कौ॰~५३७)। अत्र प्रायेणापि शब्देनैव निपातेन कर्मोक्तं मन्यन्तेऽन्यथा \textcolor{red}{संवर्ध्य} इति ल्यबन्तपद\-समभिव्याहारेण \textcolor{red}{विष\-वृक्षः} इत्यत्र द्वितीया दुर्वारेति सर्व\-विदितम्।\footnote{मल्लिनाथोऽपि~– \textcolor{red}{असाम्प्रतमित्यमेन निपातेनाभि\-हितत्वाद्‌वृक्ष इति द्वितीयान्तो न भवत्यनभिहिते कर्मणि द्वितीयाभिधानात्। यथाऽह वामनः~– “निपातेनाप्यभिहिते कर्मणि न विभक्तिः परिगणनस्य प्रायिकत्वात्” इति} (कु॰स॰ स॰व्या॰~२.५५)।} किन्त्वस्मत्सम्प्रदाये त्वस्मद्गुरु\-चरणा \textcolor{red}{असाम्प्रतम्‌}\-घटक\-\textcolor{red}{साम्प्रतम्‌}इति\-निपातेनैवोक्तं कर्म व्यवस्थापयन्ति। अत्रैवार्थ\-बोधन\-सामर्थ्यात्। सा रीतिरत्राप्यनु\-श्रियते। \textcolor{red}{स्याच्चेद्विभो विद्धि कुटुम्ब\-हानिः} इति छन्दश्चतुर्थ\-चरण\-घटक\-\textcolor{red}{चेत्‌}\-इति\-निपातेनैवोक्तं कर्म। अत उक्ते कर्मणि प्रथमा \textcolor{red}{विष\-वृक्षोऽपि संवर्ध्य} (कु॰स॰~२.५५) इतिवत् \textcolor{red}{विद्धि कुटुम्ब\-हानिः} इति।
यद्वा \textcolor{red}{कुटुम्ब\-हानिः} इति पदं \textcolor{red}{विद्धि} इति पदेन नान्वीयताम्। तथा चाविवक्षया कर्मणः \textcolor{red}{विद्धि} इति\-घटक\-\textcolor{red}{विद्‌}\-धातुम् (\textcolor{red}{विदँ ज्ञाने} धा॰पा॰~१०६४) अकर्मकं मत्वा वाक्यं विभिद्य व्याख्येयम्। अर्थात् \textcolor{red}{नौकायां गतायां मम कुटुम्ब\-हानिः स्याद्धे विभ इति त्वं विद्धि जानीहि} इति तात्पर्यम्।
इत्थं नापाणिनीयता।\footnote{\textcolor{red}{पश्य मृगो धावति} इतिवत्क्रिया कर्मेति भावः।}\end{sloppypar}
\section[मे मनःप्रीतिकरौ]{मे मनःप्रीतिकरौ}
\centering\textcolor{blue}{कस्यैतौ नरशार्दूलौ पुत्रौ देवसुतोपमौ।\nopagebreak\\
मनःप्रीतिकरौ मेऽद्य नरनारायणाविव॥}\nopagebreak\\
\raggedleft{–~अ॰रा॰~१.६.९}\\
\begin{sloppypar}\hyphenrules{nohyphenation}\justifying\noindent\hspace{10mm} अत्र लोकाभिरामं श्रीरामं सुमित्रा\-सुताभिरामं दृष्ट्वा
योगिराजो जनको विश्वामित्रं पृच्छति यत् \textcolor{red}{मे मनः\-प्रीति\-करौ एतौ कस्य सुतौ}। अत्र \textcolor{red}{मनसः प्रीति\-करौ} इति समासः। मनः\-शब्दस्य हि मम\-स्थानापन्न\-\textcolor{red}{मे}\-इति\-शब्देन साकाङ्क्षता। कस्य मन इत्यपेक्षायां मे जनकस्य यन्मनस्तस्य प्रीतिकरौ। \textcolor{red}{मे} इति मनः\-शब्दस्य विशेषणम्। इतरस्माद्व्यावर्तकत्वात्तत्र भाष्ये प्रतिपादितं \textcolor{red}{सविशेषणानां वृत्तिर्न वृत्तस्य वा विशेषण\-योगो न} (भा॰पा॰सू॰~२.१.१)। अर्थाद्विशेषणवतः शब्दस्य विशेषणं विहाय पञ्चसु वृत्तिषु कृत्तद्धित\-समासैक\-शेष\-सनाद्यन्त\-धातु\-रूपासु कतमाऽपि वृत्तिर्न भवति। यथा \textcolor{red}{ऋद्धस्य राज्ञः पुरुषः} इत्यत्र ऋद्धस्येति राज्ञ इत्यस्य विशेषणम्। एवं \textcolor{red}{ऋद्धस्य} इति पदं त्यक्त्वा \textcolor{red}{राज्ञः} इति शब्दस्य \textcolor{red}{पुरुषः} इति पदेन कथमपि न समासः।\footnote{\textcolor{red}{न च षष्ठी\-तत्पुरुषादि\-स्थलेऽपि लुप्त\-विभक्ति\-स्मरण एव चेदन्वयबोधस्तदा ‘ऋद्धस्य राजमातङ्गाः’ इत्यादि\-प्रयोगापत्तिः} (व्यु॰वा॰ का॰प्र॰)।} एवं \textcolor{red}{राज्ञः} इत्यस्य \textcolor{red}{पुरुषः} इत्यनेन समासे \textcolor{red}{ऋद्धस्य} इति विशेषणं कथमपि न योजयितुं शक्यते। तस्मात् \textcolor{red}{मे} इत्यस्य विशेषणतया जागरूकतायां \textcolor{red}{मनः} इत्यस्य \textcolor{red}{प्रीति\-करौ} इति शब्देन कथं समासः। \textcolor{red}{सापेक्षमसमर्थवत्‌} इति न्यायात्सामर्थ्य\-विरहात् \textcolor{red}{समर्थः पद\-विधिः} (पा॰सू॰~२.१.१) इति सूत्रस्य प्रसरण\-गन्धोऽपि नेत्यपेक्षायामुच्यते। नित्य\-सापेक्ष\-स्थले नियमोऽयमपोद्यते। \textcolor{red}{चैत्रस्य दास\-भार्या} इतिवत् \textcolor{red}{देवदत्तस्य गुरु\-कुलम्‌} इतिवच्च। मनसो व्यक्ति\-विशेषेण नित्य\-साकाङ्क्षतया व्यक्ताविव तन्निष्ठत्वादस्य नियमस्यापवादः। यद्वा \textcolor{red}{मनः\-प्रीति\-करौ} पृथक्समासः पश्चात् \textcolor{red}{मे} इत्यनेन सह पार्ष्ठिकोऽन्वय इति न विरोधः।\end{sloppypar}
\section[नोदितो मे]{नोदितो मे}
\centering\textcolor{blue}{मखसंरक्षणार्थाय मयाऽऽनीतौ पितुः पुरात्।\nopagebreak\\
आगच्छन् राघवो मार्गे ताटकां विश्वघातिनीम्॥\\
शरेणैकेन हतवान्नोदितो मेऽतिविक्रमः।\nopagebreak\\
ततो ममाश्रमं गत्वा मम यज्ञविहिंसकान्॥}\nopagebreak\\
\raggedleft{–~अ॰रा॰~१.६.११-१२}\\
\begin{sloppypar}\hyphenrules{nohyphenation}\justifying\noindent\hspace{10mm} विश्वामित्रो राम\-पराक्रमं प्रशंसन् कथयति \textcolor{red}{मे नोदितः शरेणैकेन श्रीरामस्ताटकां हतवान्‌}। अत्र \textcolor{red}{अहं राममनोदयम्‌} इति कर्तृ\-वाच्यार्थे \textcolor{red}{मया रामोऽनोद्यत} इति कर्म\-वाच्ये तस्मिन्नेव \textcolor{red}{तयोरेव कृत्यक्तखलर्थाः} (पा॰सू॰~३.४.७०) इति सूत्रानुसारं \textcolor{red}{निष्ठा} (पा॰सू॰~२.२.३६) इत्यनेन कर्मणि \textcolor{red}{क्त}\-प्रत्ययेन कर्मण उक्तत्वादनुक्ते कर्तरि तृतीया\-सम्भवात्\footnote{\textcolor{red}{कर्तृ\-करणयोस्तृतीया} (पा॰सू॰~२.३.१८) इत्यनेन।} \textcolor{red}{मया नोदितः} इत्येव वाच्यम्।
\textcolor{red}{मे} इति तु सम्बन्ध\-विवक्षायां कर्तरि षष्ठी।
\textcolor{red}{मत्सम्बन्धि\-नोदनाश्रयो रामः} इति तात्पर्यम्। यद्वा \textcolor{red}{नोदनं नोदः}। भावघञन्तः।\footnote{\textcolor{red}{भावे} (पा॰सू॰~३.३.१८) इत्यनेन।} स सञ्जातोऽस्य स नोदितः। \textcolor{red}{मे} इत्यस्य \textcolor{red}{राघवः} (अ॰रा॰~१.६.११) इत्यनेन सम्बन्धः। स च गुरु\-शिष्य\-भाव\-मूलको भावना\-पक्षे सेवक\-सेव्य\-भाव\-मूलकश्चेति मीमांसायां \textcolor{red}{तदस्य सञ्जातं तारकादिभ्य इतच्‌} (पा॰सू॰~५.२.३६) इति सूत्रेणाऽकृति\-गणतया तारकादि\-गणे मत्वा \textcolor{red}{नोद}\-शब्दात् \textcolor{red}{इतच्‌}\-प्रत्यय एवं भत्वाट्टिलोपे\footnote{\textcolor{red}{यचि भम्} (पा॰सू॰~१.४.१८) इत्यनेन भत्वम्। \textcolor{red}{यस्येति च} (पा॰सू॰~६.४.१४८) इत्यनेनाकार\-लोपः।} \textcolor{red}{नोदितः}। प्रेरणावाल्लीँला\-शक्त्या मे मम विश्वामित्रस्य शिष्य आराध्य इष्ट\-देवः श्रीराम\-चन्द्रस्ताटकामेकेनैव बाणेन हतवानिति पाणिनीयतयाऽर्थ\-माधुर्यमपि समायातम्।\end{sloppypar}
\section[रामाय दर्शय]{रामाय दर्शय}
\centering\textcolor{blue}{ततः सम्प्रेषयामास मन्त्रिणं बुद्धिमत्तरम्।\nopagebreak\\
जनक उवाच\nopagebreak\\
शीघ्रमानय विश्वेशचापं रामाय दर्शय॥}\nopagebreak\\
\raggedleft{–~अ॰रा॰~१.६.१८}\\
\begin{sloppypar}\hyphenrules{nohyphenation}\justifying\noindent\hspace{10mm} अत्र महर्षि\-विश्वामित्रः श्रीराम\-पराक्रम\-परिचय\-प्रस्तावनया जनकस्य हृदि श्रीरामस्य शैव\-कोदण्ड\-खण्डन\-सामर्थ्य\-विषयक\-विश्वासं सबलमुत्पाद्य \textcolor{red}{रामाय धनुर्दर्शय} इत्यादिशति। अत्र \textcolor{red}{रामाय} इति चतुर्थी विचार\-विषयः। वस्तुतस्तु रामश्चापं पश्यतु त्वं च प्रेरय इत्यर्थे \textcolor{red}{हेतुमति च} (पा॰सू॰~३.१.२६) इत्यनेन ण्यन्त\-\textcolor{red}{दृश्‌}\-धातुर्ज्ञान\-सामान्यार्थकः। एवं च \textcolor{red}{गति\-बुद्धि\-प्रत्यवसानार्थ\-शब्द\-कर्माकर्मकाणामणि कर्ता स णौ} (पा॰सू॰~१.४.५२) इति सूत्रेणाण्यन्तः कर्तृ\-भूतो रामोऽधुना ण्यन्तावस्थायां कर्म स्यात्। \textcolor{red}{रामं दर्शय} इत्येवाऽपातत उचितम्। पुनर्ण्यन्ते सति \textcolor{red}{रामम्‌} इत्येव भवितव्यं \textcolor{red}{रामाय} इति कथं तदुपर्युच्यते। \textcolor{red}{तादर्थ्ये चतुर्थी वाच्या} (वा॰~२.३.१३) इति वार्त्तिकेनात्र चतुर्थी। \textcolor{red}{रामार्थं धनुर्दर्शय} इति तात्पर्यम्। तवानेन किमपि प्रयोजनं न सेत्स्यतीति व्यज्यते। यद्वाऽत्र सम्प्रदानस्य विवक्षा। अर्थादिदं रामाय सम्प्रदेहि स्व\-स्वत्व\-निवर्तनं कुरु। रामः सज्जीकृत्य त्रोटयेद्वा रक्षेद्वा। तव कोऽप्यधिकारो नापेक्षते। एतावत्कालं यावद्धनुः पूजितवानधुना धनुर्धरं पूजय समुद्रं प्राप्य नदीमिव धनुर्धरे ममतां कुरु धनुर्ममतां च त्यजेत्येव विश्वामित्र\-तात्पर्यं प्रतिभाति। यद्वा \textcolor{red}{रामाय दातुं धनुर्दर्शय} इति गम्यमान\-दान\-क्रिया। तद्बलेन सुतरां चतुर्थी। यद्वा \textcolor{red}{रामं परीक्षितुं धनुर्दर्शय} इति \textcolor{red}{क्रियार्थोपपदस्य च कर्मणि स्थानिनः} (पा॰सू॰~२.३.१४) इत्यनेन चतुर्थी। यद्वा \textcolor{red}{रामाय} इति न चतुर्थ्यन्तमपि तु जनकस्य सम्बोधनमिदम्। विश्वामित्राद्रामं सगुणं ब्रह्म श्रुत्वा तमेवात्मना प्रविशन्तं समाधि\-मग्नं श्रीराम\-रूप\-माधुरी\-चोरित\-हृदयं पावके द्रवीभूतं कनकमिव जनकं विलोक्य विश्वामित्रो \textcolor{red}{रामाय} इति शब्देन जनकं सम्बोधयति। \textcolor{red}{अयँ गतौ} (धा॰पा॰~४७४) इति धातुः। गत्यर्थस्य ज्ञानार्थतया प्राप्त्यर्थतया च प्रसिद्धत्वात्\footnote{\textcolor{red}{गत्यर्थका ज्ञानार्थकाः प्राप्त्यर्थका अपि भवन्ति}।} \textcolor{red}{रामं श्रीरामचन्द्रमयते गच्छति प्रविशति जानाति वा स रामायः} इति विग्रहे \textcolor{red}{कर्मण्यण्‌} (पा॰सू॰~३.२.१) इत्यनेन \textcolor{red}{अण्‌}\-प्रत्यये विभक्तिकार्ये \textcolor{red}{रामायः}। तत्सम्बुद्धौ \textcolor{red}{हे रामाय}। \textcolor{red}{हे राम\-तत्त्वज्ञ}। \textcolor{red}{अधुना विश्वस्तः सन्नहङ्कार\-प्रतीकं तमःप्रतीकं वा शिव\-धनुः श्रीरामं दर्शय}। तस्मिन् सज्जीकृते सति यथा श्रीसीता\-राम\-विवाहो विलोक्येत।\end{sloppypar}
\section[दर्शयामास रामाय]{दर्शयामास रामाय}
\centering\textcolor{blue}{दर्शयामास रामाय मन्त्री मन्त्रयतां वरः।\nopagebreak\\
दृष्ट्वा रामः प्रहृष्टात्मा बद्ध्वा परिकरं दृढम्॥}\nopagebreak\\
\raggedleft{–~अ॰रा॰~१.६.२३}\\
\begin{sloppypar}\hyphenrules{nohyphenation}\justifying\noindent\hspace{10mm} अत्रापि \textcolor{red}{रामं दर्शयामास} इति प्रयोक्तव्ये \textcolor{red}{रामाय} इति प्रायुङ्क्त। यतो ह्यत्र सम्प्रदानस्य विवक्षा। \textcolor{red}{रामाय दातुं दर्शयामास} इति तात्पर्यम्।\footnote{\textcolor{red}{दातुम्} इत्यध्याहार्यमिति भावः।} यद्वा तादर्थ्ये चतुर्थी।\footnote{\textcolor{red}{तादर्थ्ये चतुर्थी वाच्या} (वा॰~२.३.१३) इत्यनेन।} \textcolor{red}{रामार्थं दर्शयामास} इति।\footnote{यद्वा पूर्ववत् \textcolor{red}{रामं परीक्षितुं धनुर्दर्शयामास} इति \textcolor{red}{क्रियार्थोपपदस्य च कर्मणि स्थानिनः} (पा॰सू॰~२.३.१४) इत्यनेन चतुर्थी।}\end{sloppypar}
\section[दीयते मे सुता]{दीयते मे सुता}
\centering\textcolor{blue}{दीयते मे सुता तुभ्यं प्रीतो भव रघूत्तम।\nopagebreak\\
इति प्रीतेन मनसा सीतां रामकरेऽर्पयन्॥}\nopagebreak\\
\raggedleft{–~अ॰रा॰~१.६.५४}\\
\begin{sloppypar}\hyphenrules{nohyphenation}\justifying\noindent\hspace{10mm} अत्र \textcolor{red}{मया दीयते} इति प्रयोक्तव्ये \textcolor{red}{मे दीयते} इति प्रयुक्तम्। यतो ह्यत्र सम्बन्ध\-विवक्षायां षष्ठी। स च सम्बन्धः पालक\-पाल्य\-रूपः। यद्वा \textcolor{red}{मे} इत्यस्य सुतया साकमन्वयः। मत्सम्बन्धिनी सुता तुभ्यं दीयते।
इत्थमन्वय\-विपरिणाम आपत्ति\-निरासः।\end{sloppypar}
\section[वसिष्ठाय विश्वामित्राय]{वसिष्ठाय विश्वामित्राय}
\centering\textcolor{blue}{ततोऽब्रवीद्वसिष्ठाय विश्वामित्राय मैथिलः।\nopagebreak\\
जनकः स्वसुतोदन्तं नारदेनाभिभाषितम्॥}\nopagebreak\\
\raggedleft{–~अ॰रा॰~१.६.५८}\\
\begin{sloppypar}\hyphenrules{nohyphenation}\justifying\noindent\hspace{10mm} अत्र \textcolor{red}{वसिष्ठाय विश्वामित्राय} चेति प्रयोगद्वयमपि द्वितीयायां प्रयोक्तव्यमासीत्किन्तु \textcolor{red}{विवक्षाधीनानि कारकाणि भवन्ति}\footnote{मूलं मृग्यम्। यद्वा \textcolor{red}{कर्मादीनामविवक्षा शेषः} (भा॰पा॰सू॰~२.३.५०, २.३.५२, २.३.६७) इत्यस्य तात्पर्यमिदम्।} इति नियममनुसृत्य चतुर्थी प्रयुक्ता। यद्वा \textcolor{red}{वसिष्ठं विश्वामित्रञ्च तोषयितुम्‌} इति \textcolor{red}{क्रियार्थोपपदस्य च कर्मणि स्थानिनः} (पा॰सू॰~२.३.१४) इत्यनेन चतुर्थी। यद्वा \textcolor{red}{स्व\-सुतोदन्तम्‌} इत्यत्र \textcolor{red}{रुचितम्‌} इत्यध्याहार्यम्। सीता\-कथायाः सर्वेभ्योऽपि रुचितत्वौचित्यात्। अतः \textcolor{red}{रुचितं स्व\-सुतोदन्तम्‌}। काभ्याम्। वसिष्ठाय विश्वामित्राय। द्वयोरेवर्षिवर्ययोर्नाम।\footnote{“द्वयोः एव ऋर्षिवर्ययोः” इत्यत्र सम्बन्ध\-विवक्षायां षष्ठी।} कथम्। यतो हि द्वावपि वैदिकावृषी। वसिष्ठो राम\-मन्त्रार्थ\-तत्त्वज्ञो विश्वामित्रश्च सीता\-मन्त्रार्थ\-तत्त्ववित्। इत्थं \textcolor{red}{राममन्त्रे स्थिता सीता सीतामन्त्रे च राघवः}\footnote{मूलं मृग्यम्।} इति प्राचीनोक्तेरुभयोरपि वेद\-मन्त्र\-साक्षात्\-कारित्वम्। विश्वामित्रस्तु साक्षाद्ब्रह्म\-गायत्र्या ऋषिः सीतैव गायत्री\-मन्त्रार्थः। वसिष्ठश्च महा\-मृत्युञ्जय\-मन्त्र\-द्रष्टर्षिर्महा\-मृत्युञ्जयश्च सीताराम\-गुण\-गान\-परः। तथा हि~–\end{sloppypar}
\centering\textcolor{red}{त्र्य॑म्बकं यजामहे सु॒गन्धिं॑ पुष्टि॒वर्ध॑नम्।\nopagebreak\\
उ॒र्वा॒रु॒कमि॑व॒ बन्ध॑नान्मृ॒त्योर्मु॑क्षीय॒ मामृता॑त्॥}\nopagebreak\\
\raggedleft{–~ऋ॰वे॰सं॰~७.५९.१२}\\
\begin{sloppypar}\hyphenrules{nohyphenation}\justifying\noindent अस्यार्थः। तिस्रः कौसल्या\-कैकेयी\-सुमित्रा अरुन्धत्यहल्यानसूयाः कौसल्या\-सुनयना\-शबर्योऽम्बा यस्य स त्र्यम्बको रामचन्द्रः। तं सुगन्धिं सुरभि\-युक्तं पुष्टिवर्धनमनुग्रह\-वर्धन\-कर्तारमुर्वारुकं दूर्वादलमिव श्यामलं श्रीरामचन्द्रं यजामहे यथा मृत्योर्मुक्षीय मुक्तो भवेयं माऽमृतादमृतान्मुक्तो न भवेयम्। यद्वा त्रीण्यम्बकानि सीता\-लक्ष्मण\-भरताख्यानि यस्य स त्र्यम्बको रामस्तं त्र्यम्बकं शेषं समानम्। अत एव द्वाभ्यां सीता\-वृत्तं रोचेतैव। तस्मात् \textcolor{red}{वसिष्ठाय विश्वामित्राय रुचितम्‌} इति वाक्ये \textcolor{red}{रुच्यर्थानां प्रीयमाणः} (पा॰सू॰~१.४.३३) इत्यनेन चतुर्थी।\end{sloppypar}
\vspace{2mm}
\centering ॥ इति बालकाण्डीयप्रयोगाणां विमर्शः ॥\nopagebreak\\
\vspace{4mm}
\pdfbookmark[2]{अयोध्याकाण्डम्‌}{Chap1Part1Kanda2}
\phantomsection
\addtocontents{toc}{\protect\setcounter{tocdepth}{2}}
\addcontentsline{toc}{subsection}{अयोध्याकाण्डीयप्रयोगाणां विमर्शः}
\addtocontents{toc}{\protect\setcounter{tocdepth}{0}}
\centering ॥ अथायोध्याकाण्डीयप्रयोगाणां विमर्शः ॥\nopagebreak\\
\section[प्रतिज्ञा ते कृता]{प्रतिज्ञा ते कृता}
\centering\textcolor{blue}{यदि राज्याभिसंसक्तो रावणं न हनिष्यसि।\nopagebreak\\
प्रतिज्ञा ते कृता राम भूभारहरणाय वै॥}\nopagebreak\\
\raggedleft{–~अ॰रा॰~२.१.३४}\\
\begin{sloppypar}\hyphenrules{nohyphenation}\justifying\noindent\hspace{10mm} एष प्रयोगोऽयोध्या\-काण्डस्य प्रथम\-सर्गीयः। अत्र विविध\-विद्या\-विशारदो भगवान्नारदः श्रीराममभिगम्य कथयति यद्राज्य\-लोभासक्तो रावणं न हनिष्यसि तदा ते कृता प्रतिज्ञा व्यर्था भविष्यति। \textcolor{red}{त्वया कृता} इति प्रयोक्तव्ये \textcolor{red}{ते कृता} इति प्रयुक्तम्। अत्र सम्बन्ध\-विवक्षायां षष्ठी। यद्वा \textcolor{red}{ते प्रतिज्ञा} इत्यन्वये सम्बन्धे षष्ठी।\end{sloppypar}
\section[शृणु मे]{शृणु मे}
\centering\textcolor{blue}{शृणु नारद मे किञ्चिद्विद्यतेऽविदितं क्वचित्।\nopagebreak\\
प्रतिज्ञातं च यत्पूर्वं करिष्ये तन्न संशयः॥}\nopagebreak\\
\raggedleft{–~अ॰रा॰~२.१.३६}\\
\begin{sloppypar}\hyphenrules{nohyphenation}\justifying\noindent\hspace{10mm} अत्र \textcolor{red}{शृणु मत्‌} इति प्रयोक्तव्ये \textcolor{red}{शृणु मे} इति सम्बन्ध\-विवक्षायां षष्ठी। यद्वा \textcolor{red}{मे वाक्यं शृणु} इत्यन्वये षष्ठी साधु। सा च सम्बन्ध\-सामान्ये। सम्बन्ध\-मीमांसायां न्याय\-व्याकरणयोरीषदन्तरम्। नैयायिकाः सम्बन्धमेक\-निष्ठं मन्यन्ते तदपि प्रतियोगित्व\-समानाधिकरणेन व्यवस्थापयन्ति किन्तु वयं सम्बन्धं द्विष्ठं मन्यामहे। एवं \textcolor{red}{सम्यग्बध्नाति प्रतियोग्यनुयोगिनौ यः स सम्बन्धः}। अत एव \textcolor{red}{सम्बन्धत्वं नाम सम्बन्धि\-भिन्नत्वे सति द्विष्ठत्वे सति विशिष्ट\-बुद्धि\-नियामकतावच्छेदकत्वम्}।\footnote{\textcolor{red}{“सम्बन्धो हि सम्बन्धि\-द्वय\-भिन्नत्वे सति द्विष्ठत्वे च सत्याश्रयतया विशिष्ट\-बुद्धि\-नियामकः” इत्यभियुक्त\-व्यवहारात्‌} (प॰ल॰म॰~११)।} यथा \textcolor{red}{राज\-पुरुषः} इत्यत्र स्व\-स्वामि\-भाव\-रूपः सम्बन्धः स च राजनि पुरुषे चेति द्वयोस्तिष्ठति राज\-पुरुषाभ्यां भिन्न एवं \textcolor{red}{स्वस्वामि\-भाव\-सम्बन्धावच्छिन्न\-राज\-विशिष्टः पुरुषः} इति विशिष्ट\-बुद्धि\-नियामकश्च। तस्माच्छक्ति\-निर्णय\-प्रसङ्गे \textcolor{red}{को नाम शक्ति\-पदार्थः} इति जिज्ञासायां \textcolor{red}{बोध\-जनकता शक्तिः}\footnote{\textcolor{red}{इन्द्रियाणां चक्षुरादीनां स्वविषयेषु चाक्षुषेषु घटादिषु यथाऽनादिर्योग्यता तदीय\-चाक्षुषादि\-कारणता तथा शब्दानामप्यर्थैः सह तद्बोध\-कारणतैव योग्यता सैव शक्तिरित्यर्थः} (वै॰भू॰सा॰~६.३७)।} इति यत्प्राचीनानां मतं तदप्यनयैवापत्त्या खण्डित\-प्रायम्। शक्तिर्हि शब्दार्थयोः सम्बन्धः। स च बोध\-जनकत्व\-रूपश्चेदसङ्गतः। यतो हि बोध\-जनकता समवायतया शब्दमधितिष्ठति नार्थम्। सम्बन्धस्य द्विष्ठत्वं सकल\-वैयाकरण\-सम्मतं नोपपद्येत। तस्माद्वाच्य\-वाचक\-भावापर\-पर्याया शक्तिः।\footnote{\textcolor{red}{तस्मात्पद\-पदार्थयोः सम्बन्धान्तरमेव शक्तिर्वाच्य\-वाचक\-भावापर\-पर्याया} (प॰ल॰म॰~११)।} स एव सम्बन्धो द्वावपि शब्दार्थावधितिष्ठति। सम्बन्धोऽयं तादात्म्य\-मूलकः शब्दार्थयोः। तत्रापि तादात्म्ये मूलमितरेतराध्यासः। तादात्म्यं नामाद्वैत\-वादि\-वेदान्ति\-दृष्ट्या तु तदभिन्नत्वे सति तद्भेदेन प्रतीयमानतावच्छेदकत्वम्। अस्मन्मते \textcolor{red}{तद्भिन्नत्वे सति तदभेदेन प्रतीयमानतावच्छेदकत्वम्}।\footnote{\textcolor{red}{तादात्म्यञ्च तद्भिन्नत्वे सति तदभेदेन प्रतीयमानत्वम्} (ल॰म॰, प॰ल॰म॰~१६)।} अद्वैत\-वेदान्तिनोऽभेदं पारमार्थिकं मन्यन्ते वयञ्च तमौपचारिकं स्वीकुर्महे। इत्थं वैयाकरण\-सिद्धान्तितः सम्बन्ध एव सूत्रकारैः शेष\-शब्देन व्यवह्रियते। तथा च सूत्रं \textcolor{red}{षष्ठी शेषे} (पा॰सू॰~२.३.५०)। अत्र कौमुदी\-कारो लिखति \textcolor{red}{कारक\-प्रातिपदिकार्थादि\-व्यतिरिक्तः स्व\-स्वामि\-भावादि\-सम्बन्धः शेषस्तत्र षष्ठी स्यात्‌} (वै॰सि॰कौ॰~६०६)। भाष्य\-कारास्तु कथयन्ति यत् \textcolor{red}{एकशतं हि षष्ठ्यर्थाः} (भा॰पा॰सू॰~१.१.४९)। तथाऽपि सम्बन्ध\-सामान्ये षष्ठी प्रसिद्धा। कुत्रचित्तत्तत्कारकेष्वपि षष्ठी भवति तत्तत्सूत्र\-विहित\-षष्ठी। सामान्य\-सम्बन्ध\-षष्ठ्योर्बाह्यतस्तु न कोऽपि भेदः किन्तु शाब्द\-बोधे वैलक्षण्यम्। तत्तल्लक्षणेष्वपि समासाद्यभावरूपं वैलक्षण्यमवगम्यत एव। यथा सम्बन्ध\-षष्ठ्यां \textcolor{red}{षष्ठी} (पा॰सू॰~२.२.८) इति सूत्रेण तत्पुरुष\-समासो भवति \textcolor{red}{राज\-पुरुषः} इतिवत्। कृत्षष्ठ्यां च \textcolor{red}{कृद्योगा च षष्ठी समस्यत इति वक्तव्यम्‌} (वा॰~२.२.८) इति वार्त्तिकेन तत्पुरुष\-समासो भवति \textcolor{red}{इध्मप्रव्रश्चनः} इतिवत्। परञ्च प्रतिपद\-विहित\-षष्ठ्यां समासं नैवेच्छन्ति भाष्यकारा यथा \textcolor{red}{प्रतिपद\-विधाना षष्ठी च न समस्यते} (भा॰पा॰सू॰~२.२.८)।\footnote{यथा~– \textcolor{red}{सर्पिषो ज्ञानम्। मधुनो ज्ञानम्} (का॰वृ॰~२.२.१०)। अत्र \textcolor{red}{ज्ञोऽविदर्थस्य करणे} (पा॰सू॰~२.३.५१) इत्यनेन प्रतिपद\-विधाना\-षष्ठी। ततः समासाभावः।} तस्मादत्रापादाने सम्बन्ध\-विवक्षया षष्ठी। यद्वा \textcolor{red}{मे वाक्यं श्रृणु} इति योजनया वाच्य\-वाचक\-भाव\-मूलिका षष्ठी।\end{sloppypar}
\section[तवाहितं कर्ता]{तवाहितं कर्ता}
\centering\textcolor{blue}{को वा तवाहितं कर्ता नारी वा पुरुषोऽपि वा।\nopagebreak\\
स मे दण्ड्यश्च वध्यश्च भविष्यति न संशयः॥}\nopagebreak\\
\raggedleft{–~अ॰रा॰~२.३.९}\\
\begin{sloppypar}\hyphenrules{nohyphenation}\justifying\noindent\hspace{10mm} अत्र श्वोभावि\-राम\-राज्याभिषेक\-सूचनादित्सया चक्रवर्ती दशरथः कोप\-भवने शयानां कैकेयीं श्रुत्वा तामनुनयन् कथयति यद्भवत्याः \textcolor{red}{कः अहितं कर्ता}। अत्र कृ\-धातोः (\textcolor{red}{डुकृञ् करणे} धा॰पा॰~१४७२) \textcolor{red}{ण्वुल्तृचौ} (पा॰सू॰~३.१.१३३) इत्यनेन विहिते तृचि ततः \textcolor{red}{कर्तृकर्मणोः कृति} (पा॰सू॰~२.३.६५) इत्यनेन कृतायां षष्ठ्याम् \textcolor{red}{अहितस्य कर्ता} इत्यनेन भवितव्यमासीत्। \textcolor{red}{अहितं कर्ता} इत्यापाततोऽपाणिनीयं लगति। परं तृन्नन्तमिदम्। अर्थात् \textcolor{red}{तृन्‌} (पा॰सू॰~३.२.१३५) इति सूत्रेण कृ\-धातोस्तृन्। रपरत्वेन गुणे विभक्ति\-कार्ये सौ \textcolor{red}{ऋदुशनस्पुरुदंसोऽनेहसां च} (पा॰सू॰~७.१.९४) इत्यनेनानङि \textcolor{red}{अप्तृन्तृच्स्वसृ\-नप्तृ\-नेष्टृ\-त्वष्टृ\-क्षत्तृ\-होतृ\-पोतृ\-प्रशास्तॄणाम्‌} (पा॰सू॰~६.४.११) इत्यनेन दीर्घे सुलोपे नलोपे चेति \textcolor{red}{कर्ता}।\footnote{\textcolor{red}{डुकृञ् करणे} (धा॰पा॰~१४७२)~\arrow कृ~\arrow \textcolor{red}{तृन्‌} (पा॰सू॰~३.२.१३५)~\arrow कृ~तृन्~\arrow कृ~तृ~\arrow \textcolor{red}{सार्वधातुकार्ध\-धातुकयोः} (पा॰सू॰~७.३.८४)~\arrow \textcolor{red}{उरण् रपरः} (पा॰सू॰~१.१.५१)~\arrow कर्~तृ~\arrow कर्तृ~\arrow विभक्तिकार्यम्~\arrow कर्तृ~सुँ~\arrow कर्तृ~स्~\arrow \textcolor{red}{ऋदुशनस्पुरुदंसोऽनेहसां च} (पा॰सू॰~७.१.९४)~\arrow कर्त्~अनँङ्~स्~\arrow कर्त्~अन्~स्~\arrow कर्तन्~स्~\arrow \textcolor{red}{अप्तृन्तृच्स्वसृ\-नप्तृ\-नेष्टृ\-त्वष्टृ\-क्षत्तृ\-होतृ\-पोतृ\-प्रशास्तॄणाम्} (पा॰सू॰~६.४.११)~\arrow कर्तान्~स्~\arrow \textcolor{red}{हल्ङ्याब्भ्यो दीर्घात्सुतिस्यपृक्तं हल्} (पा॰सू॰~६.१.६८)~\arrow कर्तान्~\arrow \textcolor{red}{न लोपः प्रातिपदिकान्तस्य} (पा॰सू॰~८.२.७)~\arrow कर्ता।} एवं प्राप्तायां षष्ठ्यां \textcolor{red}{न लोकाव्यय\-निष्ठा\-खलर्थ\-तृनाम्‌} (पा॰सू॰~२.३.६९) इत्यनेन निषेधेऽनुक्तत्वाच्च कर्मणस्तत्रैव \textcolor{red}{कर्मणि द्वितीया} (पा॰सू॰~२.३.२) इत्यनेन द्वितीया विभक्तिः। यद्वा \textcolor{red}{कर्ता} इति तिङन्त\-रूपं न कृदन्तम्। एवं हि कृ\-धातोः \textcolor{red}{अनद्यतने लुट्‌} (पा॰सू॰~३.३.१५) इत्यनेन लुड्\-लकारे \textcolor{red}{तिप्तस्झि\-सिप्थस्थ\-मिब्वस्मस्ताताञ्झ\-थासाथान्ध्वमिड्वहिमहिङ्‌} (पा॰सू॰~३.४.७८) इत्यनेन प्रथम\-पुरुषैकवचने \textcolor{red}{तिप्‌}\-प्रत्यये \textcolor{red}{स्यतासी लृलुटोः} (पा॰सू॰~३.१.३३) इत्यनेन \textcolor{red}{तासि}\-प्रत्यये गुणे रपरत्वे \textcolor{red}{लुटः प्रथमस्य डारौरसः} (पा॰सू॰~२.४.८५) इत्यनेन \textcolor{red}{डा} आदेशे \textcolor{red}{चुटू} (पा॰सू॰~१.३.७) इत्यनेन डकारेत्सञ्ज्ञायां \textcolor{red}{डित्त्वसामर्थ्यादभस्यापि टेर्लोपः} (ल॰सि॰कौ॰~३४३) इति नियमेन\footnote{\textcolor{red}{डित्यभस्याप्यनु\-बन्धकरण\-सामर्थ्यात्‌} (वा॰~६.४.१४३)।} टि\-लोपे \textcolor{red}{कर्ता} इति सिद्धम्।\footnote{\textcolor{red}{डुकृञ् करणे} (धा॰पा॰~१४७२)~\arrow कृ~\arrow \textcolor{red}{शेषात्कर्तरि परस्मैपदम्} (पा॰सू॰~१.३.७८)~\arrow \textcolor{red}{अनद्यतने लुट्‌} (पा॰सू॰~३.३.१५)~\arrow कृ~लुट्~\arrow कृ~तिप्~\arrow कृ~ति~\arrow \textcolor{red}{स्यतासी लृलुटोः} (पा॰सू॰~३.१.३३)~\arrow कृ~तासिँ~ति~\arrow कृ~तास्~ति~\arrow \textcolor{red}{एकाच उपदेशेऽनुदात्तात्‌} (पा॰सू॰~७.२.१०)~\arrow इडागम\-निषेधः~\arrow \textcolor{red}{सार्वधातुकार्ध\-धातुकयोः} (पा॰सू॰~७.३.८४)~\arrow \textcolor{red}{उरण् रपरः} (पा॰सू॰~१.१.५१)~\arrow कर्~तास्~ति~\arrow \textcolor{red}{लुटः प्रथमस्य डारौरसः} (पा॰सू॰~२.४.८५)~\arrow कर्~तास्~डा~\arrow कर्~तास्~आ~\arrow \textcolor{red}{डित्यभस्याप्यनु\-बन्धकरण\-सामर्थ्यात्‌} (वा॰~६.४.१४३)~\arrow कर्~त्~आ~\arrow कर्ता।} अतः कर्मणि द्वितीया निरुपद्रवा निर्भ्रान्ता।\end{sloppypar}
\section[मम]{मम}
\centering\textcolor{blue}{रामः प्राह न मे मातर्भोजनावसरः कृतः।\nopagebreak\\
दण्डकागमने शीघ्रं मम कालोऽद्य निश्चितः॥}\nopagebreak\\
\raggedleft{–~अ॰रा॰~२.४.४}\\
\begin{sloppypar}\hyphenrules{nohyphenation}\justifying\noindent\hspace{10mm} अत्र \textcolor{red}{नि}उपसर्ग\-पूर्वकाच्चयनार्थक\textcolor{red}{चिञ्‌}\-धातोः (धा॰पा॰~१२५१) भूते कर्मणि \textcolor{red}{क्तः}। कर्मण्युक्ते सति तत्र प्रथमा किन्त्वनुक्ते कर्तरि तृतीया। अतः \textcolor{red}{मया कालः निश्चितः} इत्येव। \textcolor{red}{मम कालः निश्चितः} इति कथम्। कर्तरि सम्बन्ध\-विवक्षया षष्ठी। यद्वा \textcolor{red}{मम} इत्यस्य \textcolor{red}{कालः} इत्यनेनान्वयः। तेन सम्बन्ध\-सामान्ये षष्ठी। अर्थात्स्वयं कालातीतः सन् भक्त\-वत्सलो राघवः कालमेव नियन्तारं मत्वा मातरं प्राह यदद्य मन्निरूपकः कालो दण्डकारण्य\-गमनाय पित्रा निश्चितः। यद्वा निरुपसर्ग\-चिञ्‌\-धातोः \textcolor{red}{नपुंसके भावे क्तः} (पा॰सू॰~३.३.११४) इत्यनेन भावार्थे क्त\-प्रत्ययः। ततश्च \textcolor{red}{निश्चितमस्त्यस्मिन्‌} इति विग्रहे \textcolor{red}{अर्शआदिभ्योऽच्‌} (पा॰सू॰~५.२.१२७) इत्यनेन \textcolor{red}{अच्‌}\-प्रत्ययः। अर्थाद्दण्डकारण्य\-गमनाय कालोऽयं निश्चयवानिति राघवेन्द्रस्य तात्पर्यं प्रतिभाति। यद्वा \textcolor{red}{निश्चिनोतीति निश्चितः} इति विग्रहे कर्तर्येव \textcolor{red}{क्त}\-प्रत्ययः।\footnote{\textcolor{red}{गत्यर्थाकर्मक\-श्लिष\-शीङ्स्थाऽऽस\-वस\-जन\-रुह\-जीर्यतिभ्यश्च} (पा॰सू॰~३.४.७२) इत्यनेन कर्तरि क्तः। \textcolor{red}{धातोरर्थान्तरे वृत्तेर्धात्वर्थेनोपसङ्ग्रहात्। प्रसिद्धेरविवक्षातः कर्मणोऽकर्मिका क्रिया॥} (वा॰प॰~३.७.८८)। कर्मणोऽविवक्षणादकर्मकत्वम्। यथा रघुवंशे कालिदासोऽपि प्रायुङ्क्त~– \textcolor{red}{निर्ययावथ पौलस्त्यः पुनर्युद्धाय मन्दिरात्। अरावणमरामं वा जगदद्येति निश्चितः॥} (र॰वं॰~१२.८३)। अत्र मल्लिनाथः~– \textcolor{red}{अद्य जगदरावणं रावणशून्यमरामं रामशून्यं वा भवेदेति निश्चितो निश्चितवान्। कर्तरि क्तः} (र॰वं॰ स॰व्या॰~१२.८३)।} अर्थात्साम्प्रतमयं कालो मम विश्रामं न ह्यनुमन्यते। तस्मादाज्ञां देहि। यद्वा \textcolor{red}{मम} शब्दस्य \textcolor{red}{दण्डकागमने} इति शब्देनान्वयः।\end{sloppypar}
\section[शरीरम्]{शरीरम्‌}
\centering\textcolor{blue}{प्रतिक्षणं क्षरत्येतदायुरामघटाम्बुवत्।\nopagebreak\\
सपत्ना इव रोगौघाः शरीरं प्रहरन्त्यहो॥}\nopagebreak\\
\raggedleft{–~अ॰रा॰~२.४.२८}\\
\begin{sloppypar}\hyphenrules{nohyphenation}\justifying\noindent\hspace{10mm} अत्र श्रीरामचन्द्रः क्रुद्धं लक्ष्मणं सान्त्वयन्नुपदिशति यत्काम\-क्रोधादयः षट्सपत्नाः शरीरं प्रहरन्ति। \textcolor{red}{शरीरे प्रहरन्ति} इति प्रयोक्तव्ये \textcolor{red}{शरीरम्‌} इति प्रयुक्तम्। \textcolor{red}{शरीरेऽरिः प्रहरति हृदये स्वजनस्तथा} (प्र॰ना॰~१.१२) इति प्रतिमानाटके प्रयुक्तत्वात्प्रहारस्याधार\-तयाऽधिकरणत्वं परित्यज्य कर्मत्वमुक्तमिति चेत्। कथ्यते। \textcolor{red}{शरीरमवलोक्य प्रहरन्ति} इति व्याख्यायताम्।\footnote{पूर्वपक्षोऽयम्।} न चास्मिन् व्याख्याने \textcolor{red}{ल्यब्लोपे कर्मण्यधिकरणे च} (वा॰~२.३.२८) इत्यनेन पञ्चम्याशङ्क्या। तदा \textcolor{red}{शरीरं लक्षयन्तः शरीरं घातयन्तो वा प्रहरन्ति}।\footnote{उत्तरपक्षोऽयम्।} यद्वा \textcolor{red}{परौ भुवोऽवज्ञाने} (पा॰सू॰~३.३.५५) इत्यत्रावज्ञान\-ग्रहणेन धातोरनेकार्थत्वं सूच्यते। यतो हि \textcolor{red}{परि}\-पूर्वकस्य \textcolor{red}{भू}\-धातोः (\textcolor{red}{भू सत्तायाम्} धा॰पा॰~१) निष्पन्नः \textcolor{red}{परिभव}\-शब्दोऽपमान\-सूचकः। अपमानं ह्यवज्ञानम्। \textcolor{red}{अनादरः परिभवः परीभावस्तिरस्क्रिया} (अ॰को॰~१.७.२२) इति कोष\-प्रामाण्यात्। \textcolor{red}{परि}\-पूर्वक\-\textcolor{red}{भू}\-धातोर्निसर्गतोऽप\-मानार्थे सिद्धेऽवज्ञानार्थ एवं प्रत्यय\-विधानेनावज्ञान\-ग्रहणं व्यर्थम्। तदेव व्यर्थं सज्ज्ञापयति यत् \textcolor{red}{अनेकार्था हि धातवः}। तेनात्रावज्ञान\-ग्रहणं चरितार्थं ज्ञापकञ्च। तदेव परं यत् \textcolor{red}{स्वांशे चरितार्थं वचन\-सिद्धिः फलमन्यत्र} इति नियमेनावज्ञान\-ग्रहणं स्वांशे चरितार्थमन्यस्मिन्नर्थे प्रत्यय\-व्यावर्तकत्वात् \textcolor{red}{अनेकार्था हि धातवः} इति वचन\-सिद्धिः \textcolor{red}{परिभाव्य} इत्यादौ
सत्क्रिया\-रूपं फलम्।\footnote{यथा~– \textcolor{red}{अस्यां वृणीष्व मनसा परिभाव्य कञ्चित्} (नै॰च॰~११.८) \textcolor{red}{मुनित्रयं नमस्कृत्य तदुक्तीः परिभाव्य च} (वै॰सि॰कौ॰ मङ्गलाचरणे~१) \textcolor{red}{एवं स परिभाव्य बिलद्वारं गत्वा तमाहूतवान्} (प॰त॰~४) इत्यादि\-शिष्ट\-प्रयोगेषु \textcolor{red}{परिभाव्य} इत्यस्य \textcolor{red}{विचार्य पर्यालोच्य} वेत्यर्थः। एवमेव ग्रन्थकारस्य श्रीभार्गव\-राघवीय\-महाकाव्ये \textcolor{red}{मनसा परिभाव्य भामिनीभयमाशङ्क्य भवाय भावुकः} (भा॰रा॰~५.६१) इति प्रयोगे।} तथैवात्रापि \textcolor{red}{प्र}\-पूर्वस्य \textcolor{red}{हृ}\-धातोः (\textcolor{red}{हृञ् हरणे} धा॰पा॰~८९९) हनन\-रूपोऽर्थः। यद्यपि प्रहार\-शब्दस्य सामान्यतो हिंसार्थ\-शस्त्रादि\-प्रक्षेपक\-रूपोऽर्थः स्वीक्रियते। प्रक्षेपणञ्च कस्मिंश्चिदाधारे सम्भवात्प्रक्षिप्त\-वस्तु\-संयोगाश्रयतयौपश्लेषिक आधारो जागरूकतामापद्यते। अतो मनन्ति प्राचीनाः~–\end{sloppypar}
\centering\textcolor{red}{उपसर्गेण धात्वर्थो बलादन्यत्र नीयते।\nopagebreak\\
प्रहाराहारसंहारविहारपरिहारवत्॥}\\
\begin{sloppypar}\hyphenrules{nohyphenation}\justifying\noindent\hspace{10mm} तथाऽप्यत्र 
लक्ष्यानुरोधेन प्रहार\-शब्दस्य हननार्थ\-स्वीकारे तत्कर्मतया \textcolor{red}{शरीरम्‌} इत्यत्र द्वितीया। यद्वाऽधिकरणेऽविवक्षिते सति \textcolor{red}{अकथितं च} (पा॰सू॰~१.४.५१) इत्यनेन कर्म। न च परिगणित\-षोडश\-धातुभ्योऽतिरिक्तस्य कथं कर्मतेति चेत् \textcolor{red}{तथा स्यान्नीहृकृष्वहाम्‌} (वै॰सि॰कौ॰~५३९) इत्युत्तरार्धे \textcolor{red}{हृ}\-धातोरपि गणनात्। ननु \textcolor{red}{प्र}\-पूर्वस्य \textcolor{red}{हृ}\-धातोर्धात्वन्तरतया न सङ्ग्रहो भविष्यति। उपसर्ग\-संयोजनेन धातुर्न परिवर्तते। यथा नील\-वस्त्र\-धरस्य बालकस्य कदाचिच्छ्वेत\-वस्त्र\-धारणेऽपि नैव परिवर्तनं लोके। \textcolor{red}{न हि शास्त्रं लोकाद्भिद्यते} (भा॰पा॰सू॰~१.१.३, ४.१.९३) इति महा\-भाष्य\-वचनाल्लोक\-मतमप्यादर्तव्यम्। न च \textcolor{red}{प्र}\-उपसर्ग\-संयोजनेन धातोरर्थान्तरं तथा च \textcolor{red}{अर्थ\-निबन्धनेयं सञ्ज्ञा} (वै॰सि॰कौ॰~५३९) इति वचनेन हृधात्वर्थाभावान्न कर्म\-संज्ञेति वाच्यम्। प्रापण\-रूपस्यैवार्थस्य \textcolor{red}{हृ}\-धातोर्वास्तविकत्वमिति नैव राजाज्ञा। धातोरनेकार्थकत्वस्यानुपदमेव निरूपितत्वात्। प्रायो धातूनामर्था भवन्त्युपसर्गास्तु केवलं प्रस्फोटयन्ति। निपाता द्योतका वाचका वेति पश्चाद्वक्ष्यते। यथा \textcolor{red}{भू}\-धातोः (\textcolor{red}{भू सत्तायाम्} धा॰पा॰~१) अनुभव\-रूपोऽर्थः शाश्वतः। अनुरुपसर्गो द्योतयति तद्यथा \textcolor{red}{अनुभवति} इत्यादि।
अत एवात्र हननार्थतया शरीरं कर्म। ततो द्वितीया।\end{sloppypar}
\section[तव]{तव}
\centering\textcolor{blue}{अहमग्रे गमिष्यामि वनं पश्चात्त्वमेष्यसि।\nopagebreak\\
इत्याह मां विना गन्तुं तव राघव नोचितम्॥}\nopagebreak\\
\raggedleft{–~अ॰रा॰~२.४.६३}\\
\begin{sloppypar}\hyphenrules{nohyphenation}\justifying\noindent\hspace{10mm} अत्र भगवती सीता श्रीरामं प्रार्थयमानाऽनुनयन्ती ब्रवीति हे राघव \textcolor{red}{मां विना तव गन्तुं न उचितम्‌}। अत्र \textcolor{red}{त्वया} इति प्रयोक्तव्ये \textcolor{red}{तव} इति प्रयुक्तम्। अत्र कर्तृ\-शेषत्व\-विवक्षायां षष्ठी। यद्वा \textcolor{red}{उचित}\-शब्देन सहान्वयात्सम्बन्ध\-षष्ठी। यद्वा \textcolor{red}{भावे तुमुन्‌} (भा॰पा॰सू॰~३.३.१०, ३.४.९)। ततः कर्तुरनुक्तत्वादनुक्ते कर्तरि \textcolor{red}{कर्तृ\-कर्मणोः कृति} (पा॰सू॰~२.३.६५) इति षष्ठी।
यद्वा \textcolor{red}{तव कृते} इत्यध्याहार्यम्। अतः षष्ठ्युचिता।\end{sloppypar}
\section[श्रुतानि बहुभिर्द्विजैः]{श्रुतानि बहुभिर्द्विजैः}
\centering\textcolor{blue}{अन्यत्किञ्चित्प्रवक्ष्यामि श्रुत्वा मां नय काननम्।\nopagebreak\\
रामायणानि बहुशः श्रुतानि बहुभिर्द्विजैः॥}\nopagebreak\\
\raggedleft{–~अ॰रा॰~२.४.७७}\\
\begin{sloppypar}\hyphenrules{nohyphenation}\justifying\noindent\hspace{10mm} वन\-गमनायानुरोधं कुर्वती भगवती सीता कथयति यत् \textcolor{red}{द्विजैर्बहूनि रामायणानि श्रुतानि}। अत्र शब्दानां पृथग्भवनतः \textcolor{red}{ध्रुवमपायेऽपादानम्‌} (पा॰सू॰~१.४.२४) इत्यनेनापादान\-सञ्ज्ञा स्यात्। यतो हि \textcolor{red}{अपायो विश्लेषस्तस्मिन्साध्ये यद्ध्रुवमवधि\-भूतं कारकं तदपादान\-सञ्ज्ञं स्यात्‌} (ल॰सि॰कौ॰~८९९) इत्यर्थः। तदनु विश्लिष्ट\-रामायणात्मक\-शब्द\-समूहस्यावधिभूत\-द्विजानामपादान\-सञ्ज्ञा। ततश्च \textcolor{red}{अपादाने पञ्चमी} (पा॰सू॰~२.३.२८) इत्यनेन पञ्चमी विभक्तिः। एवं \textcolor{red}{द्विजेभ्यः} इत्युचितं कथं \textcolor{red}{द्विजैः} इति। अत्रोच्यते। अत्र करणत्व\-विवक्षा। एवं करण\-तृतीया। यद्वा हेतुत्व\-विवक्षायां \textcolor{red}{हेतौ} (पा॰सू॰~२.३.२३) इत्यनेन तृतीया। यद्वा \textcolor{red}{कथितानि निगदितानि} इत्यध्याहार्यम्। एवं \textcolor{red}{द्विजैः कथितानि निगदितानि वा रामायणानि श्रुतानि} इत्यनुक्ते कर्तरि तृतीया। यद्वा भवता तु श्रुतान्येव किन्तु द्विजैरपि श्रुतानि। अतो रामायणस्य श्रवणस्य द्विजेषु कर्तृत्वम्। तात्पर्यमिदं यद्भवतश्चरित्रं श्लाघा\-विपर्ययतः कदाचिद्भवान्न शृणोति किन्तु संस्कार\-शीला द्विजा बहुशः शृण्वन्त्यश्रौषुश्च। शुद्ध\-संस्कारतया ते न विस्मरन्ति। तेषां वाक्यस्य प्रामाण्यं भवानपि मन्यतेऽतस्तान् पृच्छतु यत्कस्मिंश्चिद्रामायणे कस्मिंश्चिद्वा कल्पे भवान्मां विना वनमगच्छत्। श्रीसीताया हार्दमिदं यद्यद्यपि प्रतिकल्पं भवानवतरति मया सह तेषु तेषु कल्पेषु भिन्नानि भिन्नानि चरित्राणि समाचरति भवानतश्च तच्चारु\-चारु\-चरित्र\-प्रतिपादन\-परतया सहस्त्रशो रामायणानि व्याचक्षत मुनीन्द्राः। सत्सु प्रति\-रामायणे भिन्नेषु भवच्चरित्रेषु जन्म\-लीला विवाह\-लीला वन\-लीला रण\-लीला राज्य\-लीलाश्च सर्वत्र समाना एव। तत्र वन\-लीलायां मया सह भवद्वन\-गमनं सर्वैरपि रामायणैरनु\-मोदितं श्रुतवन्तो द्विजास्तत्र प्रमाणम्। अतो \textcolor{red}{द्विजैः} इत्यनुक्त\-कर्तरि तृतीया।\end{sloppypar}
\section[सजानकिम्]{सजानकिम्‌}
\centering\textcolor{blue}{आयान्तं नागरा दृष्ट्वा मार्गे रामं सजानकिम्।\nopagebreak\\
लक्ष्मणेन समं वीक्ष्य ऊचुः सर्वे परस्परम्॥}\nopagebreak\\
\raggedleft{–~अ॰रा॰~२.५.१}\\
\begin{sloppypar}\hyphenrules{nohyphenation}\justifying\noindent\hspace{10mm} \textcolor{red}{जानक्या सह वर्तमानं रामम्‌} इति विग्रहे \textcolor{red}{तेन सहेति तुल्य\-योगे} (पा॰सू॰~२.२.२८) इत्यनेन समासे सहस्य सादेशे\footnote{\textcolor{red}{वोपसर्जनस्य} (पा॰सू॰~६.३.८२) इत्यनेन।} अमि च \textcolor{red}{सजानकीम्‌} इत्येव।\footnote{पूर्वपक्षोऽयम्।} ह्रस्वः कथम्।
\textcolor{red}{जानकीवाऽचरति जानकयति}।\footnote{जानकी~\arrow \textcolor{red}{सर्वप्राति\-पदिकेभ्य आचारे क्विब्वा वक्तव्यः} (वा॰~३.१.११)~\arrow जानकी~क्विँप्~\arrow जानकी~व्~\arrow \textcolor{red}{वेरपृक्तस्य} (पा॰सू॰~६.१.६७)~\arrow जानकी~\arrow \textcolor{red}{सनाद्यन्ता धातवः} (पा॰सू॰~३.१.३२)~\arrow धातु\-सञ्ज्ञा~\arrow \textcolor{red}{शेषात्कर्तरि परस्मैपदम्} (पा॰सू॰~१.३.७८)~\arrow \textcolor{red}{वर्तमाने लट्} (पा॰सू॰~३.२.१२३)~\arrow जानकी~लट्~\arrow जानकी~तिप्~\arrow जानकी~ति~\arrow \textcolor{red}{कर्तरि शप्} (पा॰सू॰~३.१.६८)~\arrow जानकी~शप्~ति~\arrow जानकी~अ~ति~\arrow \textcolor{red}{सार्वधातुकार्ध\-धातुकयोः} (पा॰सू॰~७.३.८४)~\arrow जानके~अ~ति~\arrow \textcolor{red}{एचोऽयवायावः} (पा॰सू॰~६.१.७८)~\arrow जानकय्~अ~ति~\arrow जानकयति।} \textcolor{red}{जानकयतीति जानकिः} इति विग्रहे कर्तरि क्विप्।
पृषोदरादित्वाद्ध्रस्वः।
ततः समासे \textcolor{red}{सजानकिम्‌} इति। यद्वा \textcolor{red}{जानक्या सह वर्तमानम्‌} इति विग्रहेऽपि \textcolor{red}{गोस्त्रियोरुप\-सर्जनस्य} (पा॰सू॰~१.२.४८) इत्यनेन ह्रस्वः।\footnote{अत्र \textcolor{red}{तेन सहेति तुल्ययोगे} (पा॰सू॰~२.२.२८) इत्यनेन बहुव्रीहि\-समासः। \textcolor{red}{अनेकमन्यपदार्थे} (पा॰सू॰~२.२.२४) इति निर्देशनेन \textcolor{red}{सर्वोपसर्जनो बहुव्रीहिः}। तस्मात् \textcolor{red}{सह} इत्यस्य \textcolor{red}{जानकी} इति स्त्रीप्रत्ययान्तस्य चोपसर्जन\-सञ्ज्ञा। \textcolor{red}{वोपसर्जनस्य} (पा॰सू॰~६.३.८२) इत्यनेन \textcolor{red}{सह} इत्यस्य \textcolor{red}{स} इत्यादेशः। \textcolor{red}{गोस्त्रियोरुप\-सर्जनस्य} (पा॰सू॰~१.२.४८) इत्यनेन \textcolor{red}{जानकी} इत्यस्य \textcolor{red}{जानकि} इति ह्रस्वः। यथा \textcolor{red}{गोभिः सह वर्तमानः} इति विग्रहे \textcolor{red}{सगुः} इत्यत्र। यथा तैत्तिरीयारण्यके~– \textcolor{red}{इन्द्र॑स्य गृ॒हो॑सि॒ तं त्वा॒ प्रप॑द्ये॒ सगु॒ साश्व॑ स॒ह यन्मे॒ अस्ति॒ तेन॑} (कृ॰य॰ तै॰आ॰~४.४२.९)। अत्र सायणाचार्याः~– \textcolor{red}{सगुर्गोसहितः}। एवमेवापस्तम्ब\-श्रौत\-सूत्रे \textcolor{red}{ब्रह्म वर्म ममान्तरं तं त्वेन्द्रग्रह प्रविशानि सगुः साश्वः सपूरुषः} (आ॰श्रौ॰सू॰~१४.२६.१) इत्यत्रापि \textcolor{red}{सगुः}।} अनित्यत्वाच्च न कप्प्रत्ययः।\footnote{\textcolor{red}{नद्यृतश्च} (पा॰सू॰~५.४.१५३) इत्यनेन बहुव्रीहि\-समासान्ते \textcolor{red}{कप्‌}\-प्रत्ययः प्राप्तः। यथा \textcolor{red}{तेन सहेति तुल्ययोगे} (पा॰सू॰~२.२.२८) इत्यनेन जातेषु \textcolor{red}{सदेवीकः} (क॰स॰सा॰~७.१.९६) \textcolor{red}{सपत्नीकः} (र॰वं॰~१.८१, म॰पु॰~५८.२०, ७१.२) \textcolor{red}{सश्रीकः} (अ॰शा॰~५.९, ह॰व॰~२२.२९) \textcolor{red}{सश्रीकम्} (भा॰पु॰~९.६.१९) \textcolor{red}{ससाध्वीकाः} (बृ॰सं॰~१३.४) \textcolor{red}{ससुन्दरीकः} (क॰स॰सा॰~८.६.२५०) \textcolor{red}{सस्त्रीकाः} (भा॰पु॰~१०.३३.५) \textcolor{red}{सहपत्नीकाः} (आ॰श्रौ॰सू॰~७.२१.६, ७.२७.१६, १३.२०.५, १५.१३.१०) \textcolor{red}{सारुन्धतीकाः} (कु॰स॰~६.४) इत्यादि\-समासेषु। अत्र कबभावः। समासान्त\-प्रत्यय\-प्रकरणस्यानित्यत्वात्। \pageref{sec:sthapya}तमे पृष्ठे \ref{sec:sthapya} \nameref{sec:sthapya} इति प्रयोगस्य विमर्शं पश्यन्तु~– “समासान्त\-प्रत्यय\-प्रकरणं ह्यनित्यम्। प्रमाणं चात्र \textcolor{red}{यचि भम्‌} (पा॰सू॰~१.४.१८) इति सूत्रम्। अत्र \textcolor{red}{यश्चाच्च यच्‌} इति समाहार\-द्वन्द्वः। इह \textcolor{red}{द्वन्द्वाच्चु\-दषहान्तात्समाहारे} (पा॰सू॰~५.४.१०६) इत्यनेन चान्तत्वाट्टच्प्रत्ययः प्रयोक्तव्य आसीत्। तस्मिन् प्रयुक्ते \textcolor{red}{यचे भम्‌} इति स्यात्। यतो न प्रयुक्तोऽतः समासान्त\-प्रत्यस्यानित्यता ज्ञायते।”}\end{sloppypar}
\section[वनम्]{वनम्‌}
\centering\textcolor{blue}{गन्ताऽद्यैव वनं रामो लक्ष्मणेन सहायवान्।\nopagebreak\\
एषा सीता हरेर्माया सृष्टिस्थित्यन्तकारिणी॥}\nopagebreak\\
\raggedleft{–~अ॰रा॰~२.५.२३}\\
\begin{sloppypar}\hyphenrules{nohyphenation}\justifying\noindent\hspace{10mm} एष प्रयोगोऽध्यात्म\-रामायणस्यायोध्या\-काण्डे पञ्चमसर्गीयः। अत्र सीता\-लक्ष्मणाभ्यां सह दण्डकावनं गच्छन्तं श्रीरामं विलोक्य शोक\-सन्तप्त\-हृदयान् विषाद\-जलधौ निमज्जतः कोशल\-पुर\-वासिनो निरीक्ष्य भगवान् वामदेव आध्यात्मिक\-श्रीराम\-तत्त्व\-वर्णन\-माध्यमेन सर्वेषां शोकापनोदाय यतमानो ब्रवीति यत् \textcolor{red}{लक्ष्मणेन सह श्रीरामोऽद्यैव वनं गन्ता}। अत्र \textcolor{red}{गन्ता} इति तृजन्त\-प्रयोगः। \textcolor{red}{गम्‌}धातोः (\textcolor{red}{गमॢँ गतौ} धा.पा. ९८२) \textcolor{red}{गच्छतीति गन्ता} इति विग्रहे \textcolor{red}{तृच्‌}प्रत्यये\footnote{\textcolor{red}{ण्वुल्तृचौ} (पा॰सू॰~३.१.१३३) \textcolor{red}{कर्तरि कृत्‌} (पा॰सू॰~३.४.६७) इत्याभ्याम्।} \textcolor{red}{नश्चापदान्तस्य झलि} (पा॰सू॰~८.३.२४) इत्यनेनानुस्वारे \textcolor{red}{अनुस्वारस्य ययि परसवर्णः} (पा॰सू॰~८.४.५८) इत्यनेन परसवर्णे नकारे विभक्ति\-कार्ये सौ प्रत्यये \textcolor{red}{ऋदुशनस्पुरुदंसोऽनेहसां च} (पा॰सू॰~७.१.९४) इत्यनेन अनङि \textcolor{red}{अप्तृन्तृच्स्वसृ\-नप्तृ\-नेष्टृ\-त्वष्टृ\-क्षत्तृ\-होतृ\-पोतृ\-प्रशास्तॄणाम्‌} (पा॰सू॰~६.४.११) इत्यनेन दीर्घे सुलोप\-नलोपयोः\footnote{\textcolor{red}{हल्ङ्याब्भ्यो दीर्घात्सुतिस्यपृक्तं हल्‌} (पा॰सू॰~६.१.६८) \textcolor{red}{न लोपः प्रातिपदिकान्तस्य} (पा॰सू॰~८.२.७) इत्याभ्याम्।} \textcolor{red}{गन्ता} इति सिध्यति। \textcolor{red}{वनम्‌} इत्यत्र \textcolor{red}{कर्तृ\-कर्मणोः कृति} (पा॰सू॰~२.३.६५) इत्यनेन षष्ठ्युचिता। परञ्चात्र न \textcolor{red}{तृच्‌} अपि तु \textcolor{red}{तृन्‌}।\footnote{\textcolor{red}{तृन्‌} (पा॰सू॰~३.२.१३५) इत्यनेन तद्धर्मकर्तरि।} तृन्तृचोः स्वरभेदः।\footnote{शेषप्रक्रिया तु पूर्ववदिति भावः।} तृन्प्रत्ययेऽपि \textcolor{red}{कर्तृ\-कर्मणोः कृति} (पा॰सू॰~२.३.६५) इत्यनेन षष्ठी प्राप्ता किन्तु \textcolor{red}{न लोकाव्यय\-निष्ठा\-खलर्थ\-तृनाम्‌} (पा॰सू॰~२.३.६९) इत्यनेन षष्ठीनिषेधः। यद्वा \textcolor{red}{गन्ता} इति लुट्प्रथम\-पुरुषैक\-वचनान्तं तिङन्त\-रूपम्। अतस्तद्योगे \textcolor{red}{कर्तुरीप्सिततमं कर्म} (पा॰सू॰~१.४.४९) इत्यनेन कर्मसञ्ज्ञा \textcolor{red}{कर्मणि द्वितीया} (पा॰सू॰~२.३.२) इत्यनेन द्वितीया। अनद्यतनत्वस्य चाविवक्षा।\footnote{\textcolor{red}{अनद्यतने लुँट्‌} (पा॰सू॰~३.३.१५) इत्यनेनानद्यतने भविष्यति लुँट् विहितः। कथं तर्हि \textcolor{red}{अद्यैव} इत्यस्य योगेऽपि लुँट्। इत्यपेक्षायामुक्तम्~– \textcolor{red}{अनद्यतनत्वस्य चाविवक्षा}। यथा भारते \textcolor{red}{युवा षोडशवर्षो हि यदद्य भविता भवान्‌} (म॰भा॰~१४.५६.२२) इत्यत्रापि \textcolor{red}{अद्य}योगेऽपि लुँट्। यद्वाऽत्र परिदेवने भविष्यति लुँट्। \textcolor{red}{परिदेवने श्वस्तनीभविष्यन्त्यर्थे} (वा॰~३.३.१५) इत्यनेन। शोक\-सन्तप्तान् कोशल\-पुर\-वासिनो निरीक्ष्य वामदेवोऽपि शोकसन्तप्तः सन् निजपरिदेवनं द्योतयति।}\end{sloppypar}
\section[रामसीतयोः]{रामसीतयोः}
\centering\textcolor{blue}{य इदं चिन्तयेन्नित्यं रहस्यं रामसीतयोः।\nopagebreak\\
तस्य रामे दृढा भक्तिर्भवेद्विज्ञानपूर्विका॥}\nopagebreak\\
\raggedleft{–~अ॰रा॰~२.५.३१}\\
\begin{sloppypar}\hyphenrules{nohyphenation}\justifying\noindent\hspace{10mm} भगवान् वामदेवः श्रीरामचन्द्रं वल्कल\-धारिणं वनं प्रयान्तं दृष्ट्वा शोक\-कलित\-मनस्कानयोध्या\-वासिनः श्रीरामस्याऽध्यात्मिक\-तत्त्वं बोधयित्वा फल\-श्रुतिं श्रावयन्नाह \textcolor{red}{रहस्यं राम\-सीतयोः} इति। यो मानवो मयोदितं राम\-सीतयो रहस्यं चिन्तयिष्यति स निष्पापो भविष्यति। \textcolor{red}{रामश्च सीतेति रामसीते तयोः} इति द्वन्द्वः। विप्रतिपत्तिर्यत् \textcolor{red}{सीता\-रामयोः} इत्यनेन भवितव्यं यथा तुलसीदासोऽपि प्रयुङ्क्ते \textcolor{red}{सीता\-राम\-गुण\-ग्राम\-पुण्यारण्य\-विहारिणौ} (रा॰च॰मा॰~१/म॰~१.४)। सीता ह्यस्माकं जनन्यतः श्रीरामापेक्षयाऽभ्यर्हितैवं तस्या एव पूर्व\-प्रयोगो भवेत्।\footnote{\textcolor{red}{अभ्यर्हितम्} (वा॰~२.२.३४)। \textcolor{red}{अभ्यर्हितं पूर्वं निपततीति वक्तव्यम्। मातापितरौ श्रद्धामेधे} (भा॰पा॰सू॰~२.२.३४)।} उक्तञ्च स्मृतौ~–\end{sloppypar}
\centering\textcolor{red}{पितुर्दशगुणा माता गौरवेणातिरिच्यते।\nopagebreak\\
मातुर्दशगुणा मान्या विमाता धर्मभीरुणा॥}\footnote{मूलं स्मृति\-ग्रन्थेषु मृग्यम्। महाभारतस्य दाक्षिणात्य\-पाठे \textcolor{red}{पितुः शतगुणं माता गौरवेणातिरिच्यते} (म॰भा॰~१४.११०.६०) इति वर्तते। \textcolor{red}{“पितुः शतगुणं माता” इति स्मृतेश्च} (वा॰रा॰ भू॰टी॰~१.२३.२) इति \textcolor{red}{कौसल्या सुप्रजा राम} (वा॰रा॰~१.२३.२) इति श्लोके भूषण\-टीकायां गोविन्दराजाः।}\\
\begin{sloppypar}\hyphenrules{nohyphenation}\justifying\noindent वाल्मीकीय\-रामायणे बङ्ग\-संस्करणे चापि \textcolor{red}{जननी जन्मभूमिश्च स्वर्गादपि गरीयसी}।\footnote{{\englishfont Julius J. Lipner (Translator) (2005). Chatterji, Bankimcandra. \textit{Anandamath, or The Sacred Brotherhood}. Oxford University Press: Oxford, UK. ISBN 978-01-95346-33-6. p. 241: \textit{VMBS} (Sabyasachi Bhattacharya's \textit{Vande Mataram: The Biography of a Song}) says that this eulogy of the mother and the birthland ocurred in the version of Valmiki's Ramayana current in Bengal}.} प्रसिद्धं चापि \textcolor{red}{गौरी\-शङ्करौ लक्ष्मी\-नारायणौ शची\-पुरन्दरौ सीता\-रामौ राधा\-कृष्णौ} इत्याद्यत्रेयमेवापाणिनीयता प्रतिभाति। \textcolor{red}{आर्षत्वात्‌} इत्यपि न विचार्यताम्। ऐश्वर्य\-दृष्ट्या सीता सर्वेषां जननी। मुनिरत्र वामदेवो दशरथ\-पुरोहितो गुरु\-पुत्रो मन्त्री च। अयोध्यायाः सीता पुत्र\-वधूरिति। श्वशुरालये स्त्रियः प्राधान्यं वा श्रैष्ठ्यत्वं वा नाङ्गीक्रियते शास्त्रीय\-संस्कृतौ पत्न्याः पतिं प्रति दासीभाव\-व्यवस्थापनात्। अतो वामदेव\-दृष्टावयोध्या\-निवास\-कारणाच्छ्रीरामोऽभ्यर्हितः। अतः \textcolor{red}{रामश्च सीता चेति तयो रामसीतयोः} इति प्रयुक्तम्। यद्वा \textcolor{red}{सिनोति बध्नाति}। \textcolor{red}{प्रेम्णा सिनोति सीतां यः स सीतः}।\footnote{\textcolor{red}{षिञ् बन्धने} (धा॰पा॰~१४७७) इति धातोः \textcolor{red}{गत्यर्थाकर्मक\-श्लिष\-शीङ्स्थाऽऽस\-वस\-जन\-रुह\-जीर्यतिभ्यश्च} (पा॰सू॰~३.४.७२) इत्यनेन कर्तरि क्तः। पृषोदरादित्वाद्दीर्घः। अविवक्षित\-कर्मकत्वादकर्मकत्वम्। एवमेव ग्रन्थ\-कारैः श्रीसीता\-सुधा\-निधौ \textcolor{red}{सीता}\-शब्दस्य व्युत्पत्तिः प्रदर्शिता~– \textcolor{red}{यत्पाद\-पङ्कज\-पराग\-रसानुराग\-खड्गात् खडन्ति मुनयो भवबन्धनानि। रामं सिनोषि तमहो निजभाव\-रज्ज्वा सीतेति नाम समगास्त्वमतो जगत्याम्॥} (सी॰सु॰नि॰~१.१७)। अत्र भाषायां भक्तिटीका च~– \textcolor{red}{सिनोति इति सीता इस व्युत्पत्ति से षिञ् बन्धने धातु से कर्ता में क्त प्रत्यय और पृषोदरादित्वात् दीर्घ ईकार करके सीता शब्द की निष्पत्ति की जाती है} (सी॰सु॰नि॰ भ॰टी॰~१.१७)।} \textcolor{red}{सीता च सीतश्च} इति विग्रहे \textcolor{red}{पुमान् स्त्रिया} (पा॰सू॰~१.२.६७) इत्यनेनैक\-शेषः \textcolor{red}{सीतौ}। \textcolor{red}{रमयत इति रामौ}।\footnote{\textcolor{red}{रमयति सर्वान् गुणैरिति रामः। “रामो रमयतां वरः” (वा॰रा॰~७.४२.२१, ७.१०८.२५) इत्यार्ष\-निर्वचन\-बलात्कर्तर्यपि कारके घञ् वर्ण्यते} (वा॰रा॰ भू॰टी॰~१.१.८)। एवमेव रमयतः सर्वान् गुणैरिति रामौ।} \textcolor{red}{रामौ च तौ सीतौ चेति राम\-सीतौ तयो राम\-सीतयोः} इति। यद्वा पूर्व\-निपात\-प्रकरणमनित्यम्। \textcolor{red}{लक्षण\-हेत्वोः क्रियायाः} (पा॰सू॰~३.२.१२६) इति दर्शनात्।\footnote{\textcolor{red}{लक्षण\-हेत्वोः} इत्यत्र द्वन्द्व\-समासे \textcolor{red}{अल्पाच्तरम्} (पा॰सू॰~२.२.३४) इत्यनेन \textcolor{red}{हेतु}\-शब्दस्य पूर्व\-निपाते \textcolor{red}{हेतु\-लक्षणयोः} इत्यनेन भवितव्यमासीत्। \textcolor{red}{लक्षण\-हेत्वोः} इति पाणिनि\-प्रयोगः पूर्व\-निपात\-प्रकरणस्यानित्यत्वं ज्ञापयति। \textcolor{red}{लक्षण\-हेत्वोरिति निर्देशः पूर्वनिपात\-व्यभिचार\-लिङ्गम्} (का॰वृ॰~३.२.१२६)।}\end{sloppypar}
\section[मे]{मे}
\centering\textcolor{blue}{गृहाण फलमूलानि त्वदर्थं सञ्चितानि मे।\nopagebreak\\
अनुगृह्णीष्व भगवन् दासस्तेऽहं सुरोत्तम॥}\nopagebreak\\
\raggedleft{–~अ॰रा॰~२.५.६७}\\
\begin{sloppypar}\hyphenrules{nohyphenation}\justifying\noindent\hspace{10mm} अयोध्यातः प्रतिष्ठमानः सीता\-लक्ष्मण\-समेतः श्रीरामो निषादेन सह गतः। तदनु फल\-मूलानि श्रीरामाय निषादो न्यवेदयत यत् \textcolor{red}{मे सञ्चितानि फलमूलानि गृहाण}। अत्र सञ्चयनानुकूल\-व्यापारस्याऽश्रयः कर्ता निषादः प्रत्ययश्चात्र कर्मणि। अतः कर्तुरनुक्तत्वादत्र तृतीयया भवितव्यं\footnote{\textcolor{red}{कर्तृ\-करणयोस्तृतीया} (पा॰सू॰~२.३.१८) इत्यनेन।} किन्तु कर्तुः सम्बन्ध\-विवक्षायां षष्ठी। यद्वा \textcolor{red}{क्तस्य च वर्तमाने} (पा॰सू॰~२.३.६८) इति सूत्रेण क्तयोगा षष्ठी।\footnote{\textcolor{red}{मति\-बुद्धि\-पूजार्थेभ्यश्च} (पा॰सू॰~३.२.१८८) इत्यनेन वर्तमाने क्तः। \textcolor{red}{च}\-कारेणान्यत्रापि। \textcolor{red}{शीलितो रक्षितः क्षान्त आक्रुष्टो जुष्ट इत्यपि। रुष्टश्च रुषितश्चोभावभिव्याहृत इत्यपि॥ हृष्टतुष्टौ तथा कान्तस्तथोभौ संयतोद्यतौ। कष्टं भविष्यतीत्याहुरमृताः पूर्ववत्समृताः॥ न म्रियन्तेऽमृताः} (भा॰पा॰सू॰~३.२.१८८)। \textcolor{red}{अनुक्तसमुच्चयार्थश्चकारः। ... तथा सुप्तः शयित आशितो लिप्तस्तृप्त इत्येवमादयोऽपि वर्तमाने द्रष्टव्याः} (का॰वृ॰~३.२.१८८)। \textcolor{red}{चकारोऽनुक्त\-समुच्चयार्थः} (वै॰सि॰कौ॰~३०८९)।} यद्वा \textcolor{red}{मे} इति फलमूलैः सहान्वितम्। तेन \textcolor{red}{मत्सम्बन्धीनि फल\-मूलानि गृहाण} इति सामान्य\-सम्बन्धेन षष्ठी। यद्वा \textcolor{red}{मे} इति \textcolor{red}{अनुगृह्णीष्व} इत्यनेनान्वितम्। \textcolor{red}{मे ममोपर्यनुगृह्णीष्वानुग्रहं कुरुष्व} इति तात्पर्यम्।\end{sloppypar}
\section[आरुह्यतां नौकाम्]{आरुह्यतां नौकाम्‌}
\centering\textcolor{blue}{स्वयमेव दृढं नावमानिनाय सुलक्षणाम्।\nopagebreak\\
स्वामिन्नारुह्यतां नौकां सीतया लक्ष्मणेन च॥}\nopagebreak\\
\raggedleft{–~अ॰रा॰~२.६.१८}\\
\begin{sloppypar}\hyphenrules{nohyphenation}\justifying\noindent\hspace{10mm} गङ्गा\-पारं गन्तुकामं रामं राजीव\-लोचनं स्वामिनम् \textcolor{red}{आरुह्यतां नौकाम्‌} इत्युवाच निनाय नावं निषादः। अत्र \textcolor{red}{आरुह्यताम्‌} इति कर्म\-वाच्य\-प्रयोगः।\end{sloppypar}
\centering\textcolor{red}{कर्मवाच्यप्रयोगेषु प्रथमा कर्मकारके।\nopagebreak\\
तृतीयान्तो भवेत्कर्ता कर्माधीनं क्रियापदम्॥}\nopagebreak\\
\raggedleft{–~अस्मद्गुरुचरणाः}\\
\begin{sloppypar}\hyphenrules{nohyphenation}\justifying\noindent इति नियमानुसारमत्र \textcolor{red}{आरुह्यताम्‌} इति कर्म\-वाच्य\-प्रयोगः। अत्र \textcolor{red}{नौका} इति कर्म। एवमत्र प्रथमा\-विभक्त्या भवितव्यम् \textcolor{red}{आरुह्यतां नौका} इत्येव \textcolor{red}{आरुह्यतां नौकाम्‌} इत्यार्ष\-प्रयोगः प्रतिभाति। कारणमिदं यद्यस्मिन्नर्थे प्रत्ययो भवति स उक्तः। लकाराः प्रायस्त्रिषु कर्तरि कर्मणि भावे च भवन्ति। तत्र सूत्रम् \textcolor{red}{लः कर्मणि च भावे चाकर्मकेभ्यः} (पा॰सू॰~३.४.६९) इति। अकर्मकेभ्यो धातुभ्यः कर्तरि भावे चैवं सकर्मकेभ्यो धातुभ्यः कर्मणि कर्तरि च लकारा भवन्त्विति। अत्राऽशङ्क्यते यत् \textcolor{red}{भाव\-कर्मणोः} (पा॰सू॰~१.३.१३) इत्यस्मात्सूत्राद्भाव\-कर्मणोर्ज्ञाने \textcolor{red}{कर्तरि कर्मव्यतिहारे} (पा॰सू॰~१.३.१४) इत्यस्मात् \textcolor{red}{कर्तृ}\-पदस्यावगतौ भावे कर्तरि कर्मणि लकाराः स्युरिति द्वाभ्यां सूत्राभ्यामुपात्तं\footnote{एताभ्यां सूत्राभ्यां कर्मणि भावे कर्म\-व्यतिहारे कर्तरि चात्मने\-पद\-प्रत्ययाः स्युरित्युपात्तम्। \textcolor{red}{तिप्तस्झि\-सिप्थस्थमिब्वस्मस्ताताञ्झ\-थासाथान्ध्वमिड्वहिमहिङ्} (पा॰सू॰~३.४.७८) \textcolor{red}{लः परस्मैपदम्} (पा॰सू॰~१.४.९९) \textcolor{red}{तङानावात्मनेपदम्} (पा॰सू॰~१.४.१००) इत्येभिर्लादेशाः परमैपद\-सञ्ज्ञका आत्मने\-पद\-सञ्ज्ञका वा स्युरित्युपात्तम्। ततः कर्मणि भावे कर्तरि च लादेशाः स्युरित्युपात्तम्। तस्माद्वह्नि\-धूम\-न्यायेन कर्मणि भावे कर्तरि च लकाराः स्युरित्यप्युपात्तम्। तथा च \textcolor{red}{शेषात्कर्तरि परस्मैपदम्‌} (पा॰सू॰~१.३.७८) इत्यस्मात्कर्तरि परस्मैपद\-सञ्ज्ञका लादेशाः स्युरित्युपात्तम्। अपि च \textcolor{red}{कर्तरि कृत्‌} (पा॰सू॰~३.४.६७) इत्यस्माल्लकाराः कर्तरि स्युरित्युपात्तम्। तेषां कृत्त्वात्।} किमनेन सूत्रारम्भेण इति चेन्न। सूत्राभावे सकर्मकेभ्यो धातुभ्यो लकारा भावेऽपि भविष्यन्ति प्रतिरोधकाभावात्। एवं \textcolor{red}{घटं क्रियते} इत्यनिष्ट\-प्रयोगः स्यात्। अकर्मकेभ्यश्च धातुभ्यः कर्मणि लकारा भविष्यन्ति। तदर्थं सूत्रारम्भ आवश्यकः। न च \textcolor{red}{अकर्मकेभ्यो भावे लः} इत्येव न्यासोऽस्तु। एतन्न्यासे \textcolor{red}{असति बाधके सर्वं वाक्यं सावधारणं भवति}\footnote{मूलं मृग्यम्।} इति न्यायेन \textcolor{red}{अकर्मकेभ्यो भाव एव लः} इति नियमे जातेऽकर्मकेभ्यो भाव एव लकारा भविष्यन्ति न कर्तरि। तदा \textcolor{red}{रामः शेते} इत्यादि\-प्रयोगा न भविष्यन्ति \textcolor{red}{रामेण क्रीड्यते} इत्यादयो भाव\-प्रयोगा एव सेत्स्यन्ति। एव\-कारस्य त्रिधाऽर्थोऽस्मत्सम्प्रदाये प्रसिद्धोऽन्ययोग\-व्यवच्छेदोऽयोग\-व्यवच्छेदोऽत्यन्तायोग\-व्यवच्छेदश्च। 
अन्य\-योग\-व्यवच्छेदो नामान्य\-सम्बन्धि\-योग\-निवर्तकत्वम्। अयोग\-व्यवच्छेदो नाम योगाभाव\-निवर्तनम्। अत्यन्तायोग\-व्यवच्छेदो नाम योगत्वावच्छिन्ने प्रतियोगितावच्छेदकाभाव\-व्यावर्तकत्वम्। यत्र विशेषण\-सङ्गत एवकारो भवति तत्रान्य\-योगो व्यवच्छिद्यते यथा \textcolor{red}{नीलमेवोत्पलम्‌} अत्र नीलत्व\-प्रतिष्ठापनेन श्वेतत्वादीनां व्यवच्छेदः। विशेष्य\-सङ्गत एव\-कारे सत्ययोगो व्यवच्छिद्यते यथा \textcolor{red}{दाशरथी राम एव सर्वावतारि\-ब्रह्म}। अत्यन्तायोगो व्यावर्त्यते क्रिया\-सङ्गतेनैव\-कारेण यथा \textcolor{red}{दाशरथी रामः शराणागतान् रक्षत्येव}। इत्थम् \textcolor{red}{अकर्मकेभ्यो भाव एव लकारः} इत्येव\-कारेण कर्ता व्यवच्छिद्यते तथा \textcolor{red}{रामो राजते मुकुन्दः} इत्यादयः प्रयोगा न भविष्यन्ति। \textcolor{red}{लो भावे चाकर्मकेभ्यो} इति सूत्र्यतां चकारेण कर्ताऽऽगमिष्यति पुनर्द्वितीय\-चकारस्य किं प्रयोजनमिति चेत्सत्यम्। तथा सत्यकर्मकेभ्य एव कर्तरि प्रयोगो भविष्यति। एवम् \textcolor{red}{आरामे रमते रामः} इत्यादय एव प्रयोगाः सेत्स्यन्ति। एवं सकर्मक\-धातोः कर्तरि प्रयोगो नैव सम्पत्स्यते। तथा सति \textcolor{red}{लश्च भावे चाकर्मकेभ्यः} न्यासो भवतु। द्वितीय\-चकारेण सकर्मकेभ्योऽपि कर्तरि लकारो भविष्यति पुनः \textcolor{red}{कर्मणि} इति पदस्य काऽऽवश्यकता इति चेत्। \textcolor{red}{कर्मणि} पदाभावे \textcolor{red}{रामो रावणं हन्ति} इत्यादय एव प्रयोगाः सम्पत्स्यन्ते तथा च \textcolor{red}{रामेण रावणो हन्यते} इत्यादयः प्रयोगा न निष्पत्स्यन्ते। न च यथा द्वितीय\-चकारेण \textcolor{red}{कर्तरि} इत्यस्यानुकर्षणं\footnote{\textcolor{red}{कर्तरि कृत्‌} (पा॰सू॰~३.४.६७) इत्यस्मात्।} तथैव \textcolor{red}{कर्मणि} इत्यस्य\footnote{\textcolor{red}{उपमाने कर्मणि च} (पा॰सू॰~३.४.४५) इत्यस्मान्मण्डूकप्लुत्या।} कथं नेति चेत् \textcolor{red}{कर्मणि} इत्यस्य पुनर्वैयर्थ्यं सत्यम्। उत्तर\-सूत्रेऽनुवृत्त्यर्थमिदम्। न च \textcolor{red}{सकर्मकेभ्यः कर्तरि कर्मणि च लकाराः} इति कुत आयातं सूत्रे। अकर्मकेभ्यो भावे लकारविधानात् \textcolor{red}{कर्मणि} इति पद\-ग्रहणेन तत्र लकारस्य प्राप्तत्वात्कर्मणश्चाकर्मक\-धातुष्वसम्भवात् \textcolor{red}{सकर्मकेभ्यः} इत्यस्य सुतरां लाभः। \textcolor{red}{नास्ति कर्म येषां त अकर्मकाः कर्मणा सहिताः सकर्मकाः}। यत्र फलव्यापारौ द्वावपि धातु\-वाच्यावेकस्मिन्नधिकरणे तिष्ठतः स एवाकर्मको यत्र च फल\-व्यापारौ धात्वर्थौ विरुद्धमधिकरणमधि\-तिष्ठतस्तत्रैव सकर्मकत्वम्। यद्यपि प्राचीना बालोपलालनार्थमिमां कारिकामपाठिषुर्यत्~–\end{sloppypar}
\centering\textcolor{red}{लज्जासत्तास्थितिजागरणं वृद्धिक्षयभयजीवितमरणम्।\nopagebreak\\
शयनक्रीडारुचिदीप्त्यर्थं धातुगणं तमकर्मकमाहुः॥}\footnote{मूलं मृग्यम्।}\\
\begin{sloppypar}\hyphenrules{nohyphenation}\justifying\noindent इति। किन्त्विदं न्यूनम्। \textcolor{red}{रामः पश्यति}, \textcolor{red}{रावणो माद्यति}, \textcolor{red}{लक्ष्मणः परिश्राम्यति}, \textcolor{red}{रामो हसति} इत्यादीनामसङ्ग्रहात्। अतो व्यवस्थितं लक्षणम् \textcolor{red}{स्वार्थ\-व्यापार\-समानाधिकरण\-फल\-वाचकत्वमकर्मकत्वम्‌} इति। कर्म\-वाच्ये कर्मणि लकारे \textcolor{red}{नौका आरुह्यताम्‌} इत्येव वरमुक्ते कर्मणि प्रथमानियमादिति चेदुच्यते। \textcolor{red}{आरुह्यताम्‌} 
इत्यविवक्षित\-कर्माकर्मक\-धातु\-प्रयोगः। \textcolor{red}{नौकां दृष्ट्वा आरुह्यताम्‌} यद्वा \textcolor{red}{नौकां सनाथयिष्यताऽऽरुह्यतां भवता} इति तात्पर्यम्। यद्वा \textcolor{red}{आरुह्य} इति ल्यबन्तः प्रयोगः \textcolor{red}{तां नौकां आरुह्य स्वामिन् पारं व्रज} इति निदर्शयत्यन्यथा \textcolor{red}{स्वामिन्‌} इति कथयित्वा कर्तृ\-सम्बोधनं पुनः \textcolor{red}{आरुह्यताम्‌} इति कर्म\-वाच्यं कथं प्रयुञ्जीत। न च \textcolor{red}{स्वामिन्। भवता आरुह्यताम्} इति चेत्। \textcolor{red}{सम्भवत्येक\-वाक्यत्वे वाक्य\-भेदो न युज्यते} (श्लो॰वा॰~१.९) इति वचन\-बलेन निरर्थक\-वाक्य\-भेदस्यानौचित्यात् \textcolor{red}{स्वामिन्‌} इति शब्देन सह ल्यबन्त\-प्रयोगमेव रोचयामहे। तद्योगे च \textcolor{red}{तां नौकाम्‌} इति कर्मणि द्वितीया साध्वी। यद्वा \textcolor{red}{आरोहणमित्यारुट्‌} इति विग्रहे \textcolor{red}{आङ्‌}\-पूर्वक\-\textcolor{red}{रुह्‌}\-धातोः (\textcolor{red}{रुहँ बीजजन्मनि प्रादुर्भावे च} धा॰पा॰~८५९) क्विप्‌।\footnote{\textcolor{red}{आङ्‌}\-पूर्वकात् \textcolor{red}{रुह्‌}\-धातोः \textcolor{red}{सम्पदादिभ्‍यः क्विप्‌} (वा॰~३.३.१०८) इत्यनेन भावे क्विप्। सर्वापहारि\-लोपे कित्त्वाल्लघूपध\-गुण\-निषेधे \textcolor{red}{आरुह्} इति प्रातिपदिके जाते विभक्ति\-कार्ये सौ प्रत्यये \textcolor{red}{हल्ङ्याब्भ्यो दीर्घात्सुतिस्यपृक्तं हल्} (पा॰सू॰~६.१.६८) इत्यनेन सोर्लोपे \textcolor{red}{हो ढः} (पा॰सू॰~८.२.३१) इत्यनेन ढत्वे \textcolor{red}{झलां जशोऽन्ते} (पा॰सू॰~८.२.३९) इत्यनेन जश्त्वे \textcolor{red}{वाऽवसाने} (पा॰सू॰~८.४.५६) इत्यनेन वैकल्पिक\-चर्त्वे \textcolor{red}{आरुट्} इति सिध्यति।} ततः \textcolor{red}{सुप आत्मनः क्यच्‌} (पा॰सू॰~३.१.८) इत्यनेन \textcolor{red}{आत्मन आरुहमिच्छति} इति विग्रहे \textcolor{red}{आरुह्यति}। \textcolor{red}{सनाद्यन्ता धातवः} (पा॰सू॰~३.१.३२) इत्यनेन धातु\-सञ्ज्ञायां तस्यैव लोड्लकार\-प्रथम\-पुरुषैक\-वचनान्तं रूपम् \textcolor{red}{आरुह्य} अतो द्वितीया। यद्वाऽऽकृति\-गणत्वात् \textcolor{red}{आङ्‌}\-पूर्वको \textcolor{red}{रुह्‌}\-धातुर्दिवादि\-गणे पठ्यताम्।\footnote{\textcolor{red}{बहुलमेतन्निदर्शनम्‌} (धा॰पा॰ ग॰सू॰~१९३८) \textcolor{red}{आकृतिगणोऽयम्‌} (धा॰पा॰ ग॰सू॰~१९९२) \textcolor{red}{भूवादिष्वेतदन्तेषु दशगणीषु धातूनां पाठो निदर्शनाय तेन स्तम्भुप्रभृतयः सौत्राश्चुलुम्पादयो वाक्यकारीयाः प्रयोगसिद्धा विक्लवत्यादयश्च} (मा॰धा॰वृ॰~१०.३२८) इत्यनुसारमाकृति\-गणत्वाद्दिवादि\-गणेऽप्यूह्योऽयं धातुः।} तथा च धातु\-सञ्ज्ञायां लड्लकारे तिप्प्रत्यये \textcolor{red}{दिवादिभ्यः श्यन्‌} (पा॰सू॰~३.१.६९) इत्यनेन \textcolor{red}{श्यन्‌} विकरणेऽनुबन्ध\-कार्ये \textcolor{red}{आरुह्यति} इति रूपं भवति\footnote{आ~रुह्~\arrow \textcolor{red}{शेषात्कर्तरि परस्मैपदम्} (पा॰सू॰~१.३.७८)~\arrow \textcolor{red}{वर्तमाने लट्} (पा॰सू॰~३.२.१२३)~\arrow आ~रुह्~लट्~\arrow आ~रुह्~तिप्~\arrow आ~रुह्~ति~\arrow \textcolor{red}{दिवादिभ्यः श्यन्} (पा॰सू॰~३.१.६९)~\arrow आ~रुह्~श्यन्~ति~\arrow \textcolor{red}{सार्वधातुकमपित्} (पा॰सू॰~१.२.४)~\arrow ङित्त्वम्~\arrow \textcolor{red}{ग्क्ङिति च} (पा॰सू॰~१.१.५)~\arrow लघूपधगुणनिषेधः~\arrow आ~रुह्~य~ति~\arrow आरुह्यति।} तस्यैव लोड्लकारे प्रथम\-पुरुषैक\-वचने रूपम् \textcolor{red}{आरुह्य} इति।\footnote{आ~रुह्~\arrow \textcolor{red}{शेषात्कर्तरि परस्मैपदम्} (पा॰सू॰~१.३.७८)~\arrow \textcolor{red}{लोट् च} (पा॰सू॰~३.३.१६२)~\arrow आ~रुह्~लोट्~\arrow आ~रुह्~सिप्~\arrow आ~रुह्~सि~\arrow \textcolor{red}{दिवादिभ्यः श्यन्} (पा॰सू॰~३.१.६९)~\arrow आ~रुह्~श्यन्~सि~\arrow \textcolor{red}{सार्वधातुकमपित्} (पा॰सू॰~१.२.४)~\arrow ङित्त्वम्~\arrow \textcolor{red}{ग्क्ङिति च} (पा॰सू॰~१.१.५)~\arrow लघूपधगुणनिषेधः~\arrow आ~रुह्~य~सि~\arrow \textcolor{red}{सेर्ह्यपिच्च} (पा॰सू॰~३.४.८७)~\arrow आ~रुह्~य~हि~\arrow \textcolor{red}{अतो हेः} (पा॰सू॰~६.४.१०५)~\arrow आ~रुह्~य~\arrow आरुह्य।} यद्वाऽत्र \textcolor{red}{कर्तरि कर्म\-व्यतिहारे} (पा॰सू॰~१.३.१४) इत्यनेनात्मने\-पदम्।\footnote{अस्यार्थः। निषादो विवक्षति यत्स्वामिन् भवान् मां सुदृढनावमयाचत (\textcolor{red}{उवाच शीघ्रं सुदृढां नावमानय मे सखे} अ॰रा॰~२.६.१७)। सैवानीता मया (\textcolor{red}{स्वयमेव दृढं नावमानिनाय सुलक्षणाम्‌} अ॰रा॰~२.६.१८)। परन्त्वियं न राजनौः। भवान् हि ज्येष्ठ\-राजपुत्रः। अतो भवतः कृते राजनौरेवोचिता। ज्येष्ठ\-राजपुत्रो भूत्वाऽपि भवान् सुदृढनावमेवारोहति न राजनावम्। अयमेव कर्मव्यतिहारः। तं ध्वनयितुं निषाद आत्मनेपदं प्रयुङ्क्ते। न च निषादस्य राजनौर्न वर्तत इति चेत्। भरत\-शत्रुघ्न\-कौसल्या\-वसिष्ठानां कृते स एव निषादो राजनावमानेष्यति। यथाऽष्टमे सर्गे~– \textcolor{red}{इत्युक्त्वा त्वरितं गत्वा नावः पञ्चशतानि ह॥ समानयत्ससैन्यस्य तर्तुं गङ्गां महानदीम्। स्वयमेवानिनायैकां राजनावं गुहस्तदा॥ आरोप्य भरतं तत्र शत्रुघ्नं राममातरम्। वसिष्ठं च तथाऽन्यत्र कैकेयीं चान्ययोषितः॥} (अ॰रा॰~२.८.३८.४०) इत्यत्र तस्य राजनौकानयनं वर्णितमस्ति।} एवं लोड्लकारे प्रथम\-पुरुषैक\-वचने \textcolor{red}{त}\-प्रत्यये \textcolor{red}{टित आत्मने\-पदानां टेरे} (पा॰सू॰~३.४.७९) इत्यनेनैत्वे \textcolor{red}{आमेतः} (पा॰सू॰~३.४.९०) इत्यनेन \textcolor{red}{आम्‌} आदेशे \textcolor{red}{आरुह्यताम्‌}।\footnote{आ~रुह्~\arrow \textcolor{red}{कर्तरि कर्म\-व्यतिहारे} (पा॰सू॰~१.३.१४)~\arrow \textcolor{red}{लोट् च} (पा॰सू॰~३.३.१६२)~\arrow आ~रुह्~लोट्~\arrow आ~रुह्~त~\arrow \textcolor{red}{दिवादिभ्यः श्यन्} (पा॰सू॰~३.१.६९)~\arrow आ~रुह्~श्यन्~त~\arrow \textcolor{red}{सार्वधातुकमपित्} (पा॰सू॰~१.२.४)~\arrow ङित्त्वम्~\arrow \textcolor{red}{ग्क्ङिति च} (पा॰सू॰~१.१.५)~\arrow लघूपधगुणनिषेधः~\arrow आ~रुह्~य~त~\arrow \textcolor{red}{टित आत्मने\-पदानां टेरे} (पा॰सू॰~३.४.७९)~\arrow आ~रुह्~य~ते~\arrow \textcolor{red}{आमेतः} (पा॰सू॰~३.४.९०)~\arrow आ~रुह्~य~ताम्~\arrow आरुह्यताम्।} \textcolor{red}{स्वामिन् भवान् नौकाम् आरुह्यताम्‌} इत्यनुक्ते कर्मणि द्वितीया।\end{sloppypar}
\section[बहिर्वनस्य]{बहिर्वनस्य}
\label{sec:bahirvanasya}
\centering\textcolor{blue}{रामो दाशरथिः सीतालक्ष्मणाभ्यां समन्वितः।\nopagebreak\\
आस्ते बहिर्वनस्येति ह्युच्यतां मुनिसन्निधौ॥}\nopagebreak\\
\raggedleft{–~अ॰रा॰~२.६.३०}\\
\begin{sloppypar}\hyphenrules{nohyphenation}\justifying\noindent\hspace{10mm} नौकया जाह्नवीमवतीर्य सौमित्रि\-सीतानुचरः पद\-चरः श्रीरामो भरद्वाज\-चरण\-कमलं दिदृक्षुर्दृष्टिं पातयन् भरद्वाज\-शिष्यं प्रत्यात्मनो दर्शनेच्छां भारद्वाजाय ज्ञापयन् ब्रूते यत्सीता\-लक्ष्मण\-सहायो रामो भवद्दर्शनं चिकीर्षन् \textcolor{red}{बहिर्वनस्य} तिष्ठन् प्रतीक्षत इति। तत्रैव \textcolor{red}{बहिर्वनस्य} इति प्रयोगं करोति। बहिर्योगे पञ्चमीति \textcolor{red}{अपपरिबहिरञ्चवः पञ्चम्या} (पा॰सू॰~२.१.१२) इति सूत्रेण ज्ञापिता। अत्र पञ्चमी प्रयोक्तव्याऽऽसीत्किन्तु षष्ठी प्रयुक्ता। किमिदमार्षम्। वस्तुतस्त्विदमपि पाणिनीयम्। \textcolor{red}{आस्ते बहिर्वनस्य} इत्यत्र \textcolor{red}{वनस्य बहिःप्रदेश आस्ते} इति योजनीयम्। इत्थं चावयवावयवि\-भाव\-सम्बन्धमूलिका षष्ठी। यद्वा \textcolor{red}{बहिर्योगे पञ्चमी} इति वचनं ज्ञापक\-मूलकम्। यतो हि \textcolor{red}{अपपरिबहिरञ्चवः पञ्चम्या} (पा॰सू॰~२.१.१२) इति समास\-विधायकेन सूत्रेण \textcolor{red}{बहिः} शब्दस्य पञ्चम्यन्त\-प्रादिपदिकेन समासो ज्ञाप्यते। \textcolor{red}{वनाद्बहिरिति बहिर्वनम्‌}। तेनैव बहिर्योगे पञ्चमी ज्ञापिता। यदि बहिर्योगे पञ्चमी न स्यात्तर्हि तत्र योगे समास\-विधानमपि न स्याद्यतः समास\-विधानं ततः पञ्चमीत्यन्वय\-व्यतिरेकाभ्यां बहिर्योगे पञ्चमी साधिता। \textcolor{red}{ज्ञापक\-सिद्धं न सर्वत्र}\footnote{मूलं मृग्यम्।} इति नियमेन नियमोऽयं ज्ञापक\-सिद्धतया न सार्वत्रिकः।\footnote{\textcolor{red}{अपपरिबहिः। अपपरियोगे ‘पञ्चम्यपाङ्परिभिः’ (पा॰सू॰~२.३.१०) इति पञ्चमी विहिता। अञ्चूत्तरपदयोगेऽपि ‘अन्यारात्’ (पा॰सू॰~२.३.२९) इत्यादिना विहितैव। तेनाऽत्र ‘पञ्चम्या’ इति ग्रहणं ‘बहिर्योगे पञ्चमी भवति’ इति ज्ञापनार्थम्। ‘ज्ञापकसिद्धं न सर्वत्र’ इति ‘करस्य करभो बहिः’ (अ॰को॰~२.६.८०क) इत्यपि सिद्धम्‌} (त॰बो॰~६६६)।} अतः षष्ठी। यद्वा \textcolor{red}{विवक्षाधीनानि कारकाणि भवन्ति}\footnote{मूलं मृग्यम्। यद्वा \textcolor{red}{कर्मादीनामविवक्षा शेषः} (भा॰पा॰सू॰~२.३.५०, २.३.५२, २.३.६७) इत्यस्य तात्पर्यमिदम्।} इति नियमेन षष्ठी। यद्वा पञ्चम्या सह बहिःशब्दस्य समासो ज्ञापितः। किन्तु \textcolor{red}{बहिर्योगे पञ्चम्येव भवतु} इति नेतरा विभक्तयो निषिद्धाः। केवलं बहिः\-शब्दस्येतर\-विभक्त्यन्त\-योग\-समास एव नियन्त्रितस्तस्मादर्थानुरोधेनात्र षष्ठी साधीयसी। पञ्चम्यां सत्यां समासाग्रहो गले पतितः षष्ठ्यां तु समास आग्रहो नास्ति। इयान् विशेषः।\end{sloppypar}
\section[भरद्वाजाय मुनये]{भरद्वाजाय मुनये}
\centering\textcolor{blue}{सभार्यः सानुजः श्रीमानाह मां देवसन्निभः।\nopagebreak\\
भरद्वाजाय मुनये ज्ञापयस्व यथोचितम्॥}\nopagebreak\\
\raggedleft{–~अ॰रा॰~२.६.३२}\\
\begin{sloppypar}\hyphenrules{nohyphenation}\justifying\noindent\hspace{10mm} अत्र स्वकीयं सन्देश\-प्रकारं तन्वन् श्रीरामो ब्रवीति \textcolor{red}{भरद्वाजाय मुनये} ज्ञापयस्व। अत्र \textcolor{red}{भरद्वाजो जानातु त्वं प्रेरय} इत्यर्थे \textcolor{red}{भरद्वाजं ज्ञापयस्व}। यतो ह्यण्यन्तावस्थायाः कर्ता ण्यन्ते कर्म भवति। तथा च सूत्रं \textcolor{red}{गति\-बुद्धि\-प्रत्यवसानार्थ\-शब्दकर्माकर्मकाणामणि कर्ता स णौ} (पा॰सू॰~१.४.५२)। अत्र \textcolor{red}{ज्ञा}\-धातुः (\textcolor{red}{ज्ञा अवबोधने} धा॰पा॰~१५०७) बुद्ध्यर्थस्तस्याण्यन्तावस्थायां कर्ता भरद्वाज इति सम्प्रति ण्यन्तावस्थायां कर्म भवत्वित्येव पाणिनीयम्। परमत्र चतुर्थ्यपि पाणिनीयैव। \textcolor{red}{भरद्वाजाय मुनय आनन्दं दातुं ज्ञापयस्व} इति दा\-धातु\-योगे (\textcolor{red}{डुदाञ् दाने} धा॰पा॰~१०९१) चतुर्थी।\footnote{\textcolor{red}{आनन्दं दातुम्} इत्यध्याहार्यमिति भावः।} यद्वा \textcolor{red}{भरद्वाजाय मुनये रुचितं ज्ञापयस्व} इति रुचित\-पदाध्याहारे \textcolor{red}{रुच्यर्थानां प्रीयमाणः} (पा॰सू॰~१.४.३३) इत्यनेन सम्प्रदान\-सञ्ज्ञा ततश्चतुर्थी। यद्वा \textcolor{red}{भरद्वाजं मुनिं हर्षयितुं ज्ञापयस्व} इत्यप्रयुज्यमान\-तुमुन्\-कर्मणि चतुर्थी \textcolor{red}{क्रियार्थोपपदस्य च कर्मणि स्थानिनः} (पा॰सू॰~२.३.१४) इत्यनेन।\end{sloppypar}
\section[सीतया लक्ष्मणेन च]{सीतया लक्ष्मणेन च}
\label{sec:sitaya_lakmanena_ca}
\centering\textcolor{blue}{ततो राजा नमन्तं तं सुमन्त्रं विह्वलोऽब्रवीत्।\nopagebreak\\
सुमन्त्र रामः कुत्रास्ते सीतया लक्ष्मणेन च॥}\nopagebreak\\
\raggedleft{–~अ॰रा॰~२.७.३}\\
\begin{sloppypar}\hyphenrules{nohyphenation}\justifying\noindent\hspace{10mm} श्रीरामं रथेन प्रयागं यावत्प्रेष्य प्रत्यागतमयोध्यां सुमन्त्रं प्रति राजा पप्रच्छ यत्सीतया लक्ष्मणेन च रामः कुत्रास्ते। अत्र तृतीया
विमर्शास्पदम्। यतो हि नहि काचिदत्र क्रिया यां प्रत्यनयोः कर्तृत्वं संसाध्य तयोस्तृतीयोपकल्प्यताम्।\footnote{\textcolor{red}{कर्तृ\-करणयोस्तृतीया} (पा॰सू॰~२.३.१८) इत्यनेन।} न वा कश्चन क्रिया\-सिद्धौ व्यापारोऽवशिष्यते यद्व्यापारादनन्तरं क्रिया परिनिष्पद्येत येन करणत्वं साध्यताम्। न वा किमपि 
व्यापार\-गीतं विलोक्यते येन हेतुत्वं साधयित्वा ततश्च तृतीया क्रियताम्।\footnote{\textcolor{red}{हेतौ} (पा॰सू॰~२.३.२३) इत्यनेन।} अत्रोच्यते। विनाऽपि सहार्थ\-वाचक\-शब्द\-योगेन तृतीया भवति \textcolor{red}{वृद्धो यूना तल्लक्षणश्चेदेव विशेषः} (पा॰सू॰~१.२.६५) इत्यादि\-निर्देशात्। यद्वा प्रयुक्त\-सह\-शब्दस्य \textcolor{red}{विनाऽपि प्रत्ययं पूर्वोत्तर\-पद\-लोपो वक्तव्यः} (वा॰~५.३.८३) इति वार्त्तिकेन \textcolor{red}{सह} इत्यस्य लोपे \textcolor{red}{यः शिष्यते स लुप्यमानार्थामिधायी} इति नियमेन लुप्यमान\-सहार्थावगतेस्तृतीया। यद्वा \textcolor{red}{प्रकृत्यादिभ्य उप\-सङ्ख्यानम्‌} (वा॰~२.३.१८) इति वार्त्तिकेन प्रकृत्यादित्वादभेदार्थे तृतीया। \textcolor{red}{प्रकृत्या चारु} इतिवत्। सीतालक्ष्मणाभिन्नो राम इति भावः। सीतया लक्ष्मणेन च श्रीरामस्याभेद\-प्रसिद्धेः। वाल्मीकीय\-रामायणे द्वाभ्यामपि श्रीसीता\-रामाभ्यामभेदस्य प्रतिपादितत्वात्~–\end{sloppypar}
\centering\textcolor{red}{अनन्या हि मया सीता भास्करेण यथा प्रभा॥}\nopagebreak\\
\raggedleft{–~वा॰रा॰~६.११८.१९}\\
\begin{sloppypar}\hyphenrules{nohyphenation}\justifying\noindent इति रामवचनम्।\end{sloppypar}
\centering\textcolor{red}{अनन्या राघवेणाहं भास्करेण यथा प्रभा॥}\nopagebreak\\
\raggedleft{–~वा॰रा॰~५.२१.१५}\\
\begin{sloppypar}\hyphenrules{nohyphenation}\justifying\noindent इति सीतावचनम्। उभयत्र प्रयुक्तोऽनन्य\-शब्दोऽभावाभाव\-प्रतिपादन\-परः। अभावाभावो हि प्रतियोगि\-ज्ञानम्। यथा घटाभावाभावो घट एव। तथैव \textcolor{red}{न अन्या अनन्या}।\footnote{\textcolor{red}{नञ्‌} (पा॰सू॰~२.२.६) इत्यनेन समासे \textcolor{red}{नलोपो नञः} (पा॰सू॰~६.३.७३) इत्यनेन नकार\-लोपे \textcolor{red}{तस्मान्नुडचि} (पा॰सू॰~६.३.७४) इत्यनेन नुँट्।} \textcolor{red}{अन्या रामाद्भिन्ना}। \textcolor{red}{न अन्या अन्यभिन्ना अर्थादभिन्नैव}~–\end{sloppypar}
\centering\textcolor{red}{प्रभा जाइ कहँ भानु बिहाई। कहँ चन्द्रिका चन्द्र तजि जाई॥}\footnote{एतद्रूपान्तरम्–\textcolor{red}{कुत्र प्रगन्तुं शक्नोति प्रभा त्यक्त्वा प्रभाकरम्। त्यक्त्वा चन्द्रमसं कुत्र गन्तुं शक्नोति चन्द्रिका॥} (मा॰भा॰~२.९७.६)।}\nopagebreak\\
\raggedleft{–~रा॰च॰मा॰~२.९७.६}\\
\begin{sloppypar}\hyphenrules{nohyphenation}\justifying\noindent अन्यत्वं हि भेदे सिध्यत्यनन्यत्वमभेदे।\end{sloppypar}
\centering\textcolor{red}{गिरा अरथ जल बीचि सम देखियत भिन्न न भिन्न।}\footnote{एतद्रूपान्तरम्–\textcolor{red}{वागर्थतुल्यौ जलवीचितुल्यौ वाच्यौ पृथक्किन्तु विभेदशून्यौ} (मा॰भा॰~१.१८)।}\nopagebreak\\
\raggedleft{–~रा॰च॰मा॰~१.१८}\\
\begin{sloppypar}\hyphenrules{nohyphenation}\justifying\noindent इति गोस्वामिनाऽपि तत्र तत्र मानसेऽपि प्रतिपादितत्वात्। अतः प्रकृत्यादित्वादभेदे तृतीया स्वर्ण\-सौरभ\-संयोग इवोत्कर्षमाटीकते। करपात्र\-स्वामि\-चरणैरपि रामायण\-मीमांसा\-मङ्गलाचरणे लिखितं यत्~–\end{sloppypar}
\centering\textcolor{red}{सौन्दर्यसारसर्वस्वं माधुर्यगुणबृंहितम्।\nopagebreak\\
ब्रह्मैकमद्वितीयं तत्तत्त्वमेकं द्विधा कृतम्॥}\nopagebreak\\
\raggedleft{–~रा॰मी॰~मङ्गलाचरणे}\\
\begin{sloppypar}\hyphenrules{nohyphenation}\justifying\noindent लक्ष्मणस्याभेदत्वं तु शेषत्वात्। \textcolor{red}{शेषस्तु लक्ष्मणो राजन्‌} (अ॰रा॰~१.४.१७) इत्यस्मिन् ग्रन्थ एवोक्तत्वाच्छेषोंऽशः श्रीरामोंऽश्यंशांशिनोर्बिन्दु\-सिन्धोरिवाभेद\-प्रसिद्धेः। सीता\-लक्ष्मणयोः प्रकृत्यादि\-गणे पठितत्वौचित्यं हि सीताया मूल\-प्रकृतित्वाल्लक्ष्मणस्य च जीवाचार्यत्वात्। यद्वा \textcolor{red}{इत्थं\-भूत\-लक्षणे} (पा॰सू॰~२.३.२१) इदं हि सूत्रम्। अस्यार्थः किञ्चित्प्रकारस्य लक्षणे तृतीया। अर्थात् \textcolor{red}{इत्थं भूतं लक्षयतीति इत्थं\-भूत\-लक्षणं तस्मिन्‌}। कर्तरि बाहुलकाल्ल्युँट्।\footnote{\textcolor{red}{कृत्यल्युटो बहुलम्} (पा॰सू॰~३.३.११३) इत्यनेन।} ततश्चानुबन्ध\-कार्ये सति \textcolor{red}{युवोरनाकौ} (पा॰सू॰~७.१.१) इत्यनेनानादेशे \textcolor{red}{अट्कुप्वाङ्नुम्व्यवायेऽपि} (पा॰सू॰~८.४.२) इत्यनेन णत्वम्। यद्वा \textcolor{red}{लक्ष्यतेऽनेनेति लक्षणम्‌}। \textcolor{red}{करणाधिकरणयोश्च} (पा॰सू॰~३.३.११७) इति करणे ल्युट्। \textcolor{red}{इत्थं\-भूतस्य लक्षणमित्थं\-भूत\-लक्षणम्} इति षष्ठी\-तत्पुरुषः।\footnote{\textcolor{red}{कर्तृकर्मणोः कृति} (पा॰सू॰~२.३.६५) इत्यनेन कृद्योगा षष्ठी \textcolor{red}{कृद्योगा च षष्ठी समस्यत इति वक्तव्यम्‌} (वा॰~२.२.८) इत्यनेन समासः।} अर्थाद्येन व्यापक\-चिह्न\-विशेषेण व्यक्तेः कश्चित्प्रकारो द्योत्यते तत्र तृतीया स्यात्। एवं हि \textcolor{red}{इत्थं\-भूत\-लक्षणे} इत्यनेनैव तृतीया। रामचन्द्रस्य रामत्वं हि सीता\-लक्ष्मणाभ्यामेव सूच्यते। यथा जटाभिस्तापसः। मौञ्जी\-मेखलया ब्रह्मचारी। अत्र जटा\-ज्ञाप्य\-तापसत्व\-विशिष्टस्तापस एवं मौञ्जी\-मेखला\-ज्ञाप्य\-ब्रह्मचारित्व\-विशिष्टो ब्रह्मचारी। तथैवात्र सीता\-लक्ष्मण\-ज्ञाप्य\-रामत्व\-विशिष्टः श्रीरामः। रामत्वं हि सीता\-लक्ष्मणाभ्यामेव ज्ञाप्यते। रामेति नामन्यपि वर्ण\-त्रय\-विवरणं कृत्वा ब्रह्म\-माया\-जीवानिव राम\-सीता\-लक्ष्मणानवगच्छन्ति भावुक\-भक्ताः। एवं रकारे श्रीराम आकारे श्रीसीता मकारे च श्रीलक्ष्मणः। यथा~–\end{sloppypar}
\centering\textcolor{red}{रकारे रामचन्द्रः स्यान्मकारे लक्ष्मणः स्वराट्।\nopagebreak\\
तयोः संयोजनार्थाय सीता आकार उच्यते॥}\nopagebreak\\
\raggedleft{–~इति साम्प्रदायिकाः}\\
\begin{sloppypar}\hyphenrules{nohyphenation}\justifying\noindent इत्थं \textcolor{red}{सीतया लक्ष्मणेन च} इति द्वावपि पाणिनीयौ।\end{sloppypar}
\section[तयोः]{तयोः}
\centering\textcolor{blue}{तयोस्त्वमुदकं देहि शीघ्रमेवाविचारयन्।\nopagebreak\\
न चेत्त्वां भस्मसात्कुर्यात्पिता मे यदि कुप्यति॥}\nopagebreak\\
\raggedleft{–~अ॰रा॰~२.७.२८}\\
\begin{sloppypar}\hyphenrules{nohyphenation}\justifying\noindent\hspace{10mm} म्रियमाणो दशरथः शरीर\-रक्षायै राम\-मात्रा प्रार्थितः। मरणमपरि\-वर्तनीयमिति साधयन् प्रसङ्गोपात्त\-श्रमण\-कुमार\-कथां प्रस्तुवन् शब्द\-वेधि\-बाण\-विक्रियमाण\-महा\-प्रयाण\-श्रमण\-वचनमानुपूर्वीतयाऽनुवदति यत्प्राणांस्त्यजञ्छ्रमणो दशरथं कथयति \textcolor{red}{विचारं न कृत्वा तयोरुदकं देहि}।
अर्थान्मम पितृभ्यां जलमर्पयेति। अभीप्सिते \textcolor{red}{ताभ्याम्‌} इति चतुर्थ्यन्ते प्रयोक्तव्ये \textcolor{red}{तयोः} इति षष्ठ्यन्तं प्रयुक्तम्। अत्र \textcolor{red}{विवक्षाधीनानि कारकाणि भवन्ति}\footnote{मूलं मृग्यम्। यद्वा \textcolor{red}{कर्मादीनामविवक्षा शेषः} (भा॰पा॰सू॰~२.३.५०, २.३.५२, २.३.६७) इत्यस्य तात्पर्यमिदम्।} इति नियमेन सम्बन्ध\-विवक्षणात्षष्ठी। अथवा \textcolor{red}{तयोः} इत्यस्य \textcolor{red}{त्वम्‌} शब्देन सहान्वयः। अत्र च रक्ष्य\-रक्षक\-भाव\-सम्बन्ध\-मूलिका षष्ठी।\footnote{षड्विंशतितमे श्लोके दशरथं सम्बोधयन् श्रमणः कथयति~– \textcolor{red}{मा भैषीर्नृपसत्तम} (अ॰रा॰~२.७.२६)। नॄन् नरान् पातीति नृपः। यथाह मनुः~– \textcolor{red}{ब्राह्मं प्राप्तेन संस्कारं क्षत्त्रियेण यथाविधि। सर्वस्यास्य यथान्यायं कर्तव्यं परिरक्षणम्॥ अराजके हि लोकेऽस्मिन्सर्वतो विद्रुतो भयात्। रक्षार्थमस्य सर्वस्य राजानमसृजत्प्रभुः॥} (म॰स्मृ॰~७.२.३)। श्रमणमते दशरथस्तु नृपसत्तमः। अर्थान्नररक्षितृश्रेष्ठः। अत एव तेन \textcolor{red}{तयोस्त्वमुदकं देहि} इत्यत्र रक्ष्य\-रक्षक\-भाव\-सम्बन्ध\-मूलिका षष्ठी प्रयुक्ता।} यद्वा पित्रोर्म्रियमाणत्वात्तत्र सम्प्रदान\-विवक्षैव न। यतो हि पुत्र\-प्रदत्तोदकमेव पितरौ सम्यग्गृह्णीतः। अतः \textcolor{red}{तयोः} इति षष्ठी न पाणिनीय\-विरुद्धा।\end{sloppypar}
\section[देह्यावयोः]{देह्यावयोः}
\centering\textcolor{blue}{देह्यावयोः सुपानीयं पिब त्वमपि पुत्रक।\nopagebreak\\
इत्येवं लपतोर्भीत्या सकाशमगमं शनैः॥}\nopagebreak\\
\raggedleft{–~अ॰रा॰~२.७.३४}\\
\begin{sloppypar}\hyphenrules{nohyphenation}\justifying\noindent\hspace{10mm} अत्र श्रमण\-कुमारे मृते हस्ते कलशं गृहीत्वा निकटं गच्छतो दशरथस्य चरण\-सञ्चार\-ध्वनिं श्रुत्वा श्रमण\-कुमार\-बुद्ध्याऽन्धौ पितरौ प्राहतुर्यत् \textcolor{red}{आवयोः सुपानीयं देहि}। अत्र \textcolor{red}{दा}\-धातु\-प्रयोगे (\textcolor{red}{डुदाञ् दाने} धा॰पा॰~१०९१) \textcolor{red}{कर्मणा यमभिप्रैति स सम्प्रदानम्‌} (पा॰सू॰~१.४.३२) इत्यनेन सम्प्रदान\-सञ्ज्ञा तदनुगामिनी चतुर्थी\-विभक्तिश्च प्रयोक्तव्या। \textcolor{red}{आवाभ्यां सुपानीयं देहि} इत्येव वक्तव्यम्। परमत्र षष्ठी पाणिनीय\-विरुद्धेव। अथोच्यते। वात्सल्य\-धिया पितरौ सम्प्रदानं न विवक्षतः। अर्थाज्जलदानं तु तव कर्तव्यं नैव ते कृपा न वा नैमित्तिकं कर्म। अत आवयोरुदकं देहीति भावं प्रकटयितुं सम्बन्ध\-विवक्षा। अथवा \textcolor{red}{आवयोर्मुख उदकं देहि} इति मुख\-शब्दस्याध्याहारेणावयवावयवि\-भावे सम्बन्ध\-षष्ठी। यद्वा \textcolor{red}{आवयोः} शब्दस्य \textcolor{red}{पुत्रक} शब्देन अन्वयः। अर्थात् \textcolor{red}{हे आवयोः पुत्रक सुपानीयं देहि त्वमपि 
पिब} इत्यन्वये \textcolor{red}{आवयोः} इत्यत्र जन्य\-जनक\-सम्बन्धे षष्ठी।\end{sloppypar}
\section[मे वद]{मे वद}
\centering\textcolor{blue}{त्वया विना न मे तातः कदाचिद्रहसि स्थितः।\nopagebreak\\
इदानीं दृश्यते नैव कुत्र तिष्ठति मे वद॥}\nopagebreak\\
\raggedleft{–~अ॰रा॰~२.७.६३}\\
\begin{sloppypar}\hyphenrules{nohyphenation}\justifying\noindent\hspace{10mm} अत्र मृते दशरथे दूत\-द्वारा गुरु\-सन्देशं प्राप्यायोध्यां प्रत्यागतो भरतः कैकेयी\-सकाशं गत्वाऽदृष्ट्वा पितरं साश्चर्यं पृच्छति यत् \textcolor{red}{कुत्र तिष्ठति मे वद}। अत्र \textcolor{red}{ब्रू}\-धातु\-समानार्थकतया (\textcolor{red}{ब्रूञ् व्यक्तायां वाचि} धा॰पा॰~१०४४) \textcolor{red}{अकथितं च} (पा॰सू॰~१.४.५१) इत्यनेन कर्म\-सञ्ज्ञा प्राप्ता \textcolor{red}{कर्मणि द्वितीया} (पा॰सू॰~२.३.२) इत्यनेन च द्वितीया प्रयोक्तव्या। एवं च \textcolor{red}{मां वद} इत्यनेनैव भवितव्यम्। तदुपर्युच्यते। अत्र सम्प्रदान\-विवक्षया चतुर्थी। यद्वा \textcolor{red}{मामनुग्रहीतुं वद} इति गम्यमान\-तुमुन्प्रयोगेण तस्य कर्मणि चतुर्थी \textcolor{red}{मह्यम्‌} इति \textcolor{red}{क्रियार्थोपपदस्य च कर्मणि स्थानिनः} (पा॰सू॰~२.३.१४) इत्यनेन तस्य च \textcolor{red}{मे} इत्यादेशः।\footnote{\textcolor{red}{तेमयावेकवचनस्य} (पा॰सू॰~८.१.२२) इत्यनेन।} यद्वा \textcolor{red}{मातः} इत्यध्याहृत्य तेन सहान्वये \textcolor{red}{हे मे मातः वद} इत्यर्थे सम्बन्ध\-सामान्ये षष्ठी।
\end{sloppypar}
\section[हेऽम्ब]{हेऽम्ब}
\centering\textcolor{blue}{तामाह भरतो हेऽम्ब रामः सन्निहितो न किम्।\nopagebreak\\
तदानीं लक्ष्मणो वाऽपि सीता वा कुत्र ते गताः॥}\nopagebreak\\
\raggedleft{–~अ॰रा॰~२.७.७१}\\
\begin{sloppypar}\hyphenrules{nohyphenation}\justifying\noindent\hspace{10mm} अत्र शोकाकुलित\-हृदयो भरतो मातरं पृच्छति यत् \textcolor{red}{हेऽम्ब राम\-लक्ष्मण\-सीताः कुत्र गताः}। अत्र \textcolor{red}{हैहेप्रयोगे हैहयोः} (पा॰सू॰~८.२.८५) इति सूत्रेण \textcolor{red}{हे}\-घटकैकारस्य प्लुते \textcolor{red}{प्लुतप्रगृह्या अचि नित्यम्‌} (पा॰सू॰~६.१.१२५) इत्यनेन प्रकृति\-भावे \textcolor{red}{हे३ अम्ब} इत्येव पाणिनीयं \textcolor{red}{हेऽम्ब} इति कथम्। उच्यते। \textcolor{red}{गुरोरनृतोऽनन्त्यस्याप्येकैकस्य प्राचाम्‌} (पा॰सू॰~८.२.८६) इत्यत्र \textcolor{red}{प्राचाम्‌} इत्यस्य ग्रहणं व्यर्थं सज्ज्ञापयति यत्सर्वोऽपि प्लुतो विकल्प्यते।\footnote{\textcolor{red}{इह प्राचामिति योगो विभज्यते। तेन सर्वः प्लुतो विकल्प्यते} (वै॰सि॰कौ॰~९७)।} तस्मादत्रापि प्लुत\-विकल्पः। एवं \textcolor{red}{हे} इत्यव्ययं मत्वा \textcolor{red}{अव्ययादाप्सुपः} (पा॰सू॰~२.४.८२) इत्यनेन विभक्ति\-लोपे पदान्तत्वादेकाराकारयोः \textcolor{red}{एङः पदान्तादति} (पा॰सू॰~६.१.१०९) इत्यनेन पररूपे \textcolor{red}{हेऽम्ब} इति सिद्धम्। यद्वा त्रिपादीस्थत्वात्सपाद\-सप्ताध्यायी\-पर\-रूप\-विधायक\-शास्त्र\-दृष्ट्येदमसिद्धं ततः पर\-रूपे \textcolor{red}{हेऽम्ब} इति सिद्धम्। न च प्लुतारम्भ\-सामर्थ्यादसिद्धत्वं लोपयितुं न शक्यते। \textcolor{red}{हे३ राम राम है३} इत्यादौ प्लुत\-स्वरे चारितार्थ्यात्। न च \textcolor{red}{प्रतिलक्ष्यं लक्षणोपप्लवः}\footnote{मूलं मृग्यम्।}
इति नियमेन प्रकृति\-भाव\-रूपे लक्ष्येऽचारितार्थ्यात्तद्वैयर्थ्ये नैवासिद्धत्वं रोधयिष्यते। अन्यत्र \textcolor{red}{हे३ अरि\-सूदन} \textcolor{red}{हे३ अम्ब} इत्यादि\-प्रयोगेषु च चारितार्थ्यात्।\footnote{संहिताया अविवक्षायामिति शेषः।}
यद्वा \textcolor{red}{प्लुत\-प्रगृह्या अचि नित्यम्‌} (पा॰सू॰~६.१.१२५) इत्यत्र नित्य\-ग्रहणस्य प्रायिकत्वात्प्रकृति\-भावाभावः।\footnote{\pageref{sec:munindraham}तमे पृष्ठे \ref{sec:munindraham} \nameref{sec:munindraham} इति प्रयोगस्य विमर्शं पश्यन्तु~– “यथा \textcolor{red}{ङमो ह्रस्वादचि ङमुण्नित्यम्‌} (पा॰सू॰~८.३.३२) इत्यत्र नित्यग्रहणस्य प्रायिकत्वात् \textcolor{red}{इको यणचि} (पा॰सू॰~६.१.७७) \textcolor{red}{सुप्तिङन्तं पदम्‌} (पा॰सू॰~१.४.१४) इत्यादौ न ङुण्मुटौ।”}\end{sloppypar}
\section[तव राज्यप्रदानाय]{तव राज्यप्रदानाय}
\centering\textcolor{blue}{रामस्य यौवराज्यार्थं पित्रा ते सम्भ्रमः कृतः।\nopagebreak\\
तव राज्यप्रदानाय तदाहं विघ्नमाचरम्॥}\nopagebreak\\
\raggedleft{–~अ॰रा॰~२.७.७२}\\
\begin{sloppypar}\hyphenrules{nohyphenation}\justifying\noindent\hspace{10mm} कैकयी भरतं प्रबोधयन्ती प्राह यत् \textcolor{red}{तव राज्य\-प्रदानायाहमेव राम\-राज्ये विघ्नमाचरम्‌}। अत्र दानयोगे चतुर्थी वक्तव्याऽऽसीत् \textcolor{red}{तुभ्यं राज्य\-प्रदानाय} इति। \textcolor{red}{तव} इति सम्बन्ध\-विवक्षायां षष्ठी। अर्थात् \textcolor{red}{न च राज्यमिदं परकीयं मम विवाहे पितुः पण आसीद्यन्मम दौहित्र एव राज्याधिकारी भवेत्तत इदं राज्यमधिकारतस्तव}। अस्मादधि\-कार्याधि\-कारक\-भावे\footnote{अधिक्रियते यत्तत् \textcolor{red}{अधिकार्यम्‌}। \textcolor{red}{तयोरेव कृत्यक्तखलर्थाः} (पा॰सू॰~३.४.७०), \textcolor{red}{ऋहलोर्ण्यत्‌} (पा॰सू॰~३.१.१२४) इत्याभ्याम् \textcolor{red}{अधि}पूर्वक\textcolor{red}{कृ}धातोः (\textcolor{red}{डुकृञ् करणे} धा॰पा॰~१४७२) कर्मणि कृत्यसञ्ज्ञको ण्यत्। अधिकरोतीति \textcolor{red}{अधिकारकः}। \textcolor{red}{अधि}पूर्वक\textcolor{red}{कृ}धातोः \textcolor{red}{ण्वुल्तृचौ} (पा॰सू॰~३.१.१३३) इत्यनेन कर्तरि ण्वुल्।} सम्बन्धे षष्ठी। न च \textcolor{red}{सापेक्षमसमर्थवत्‌} इति नियमेन \textcolor{red}{राज्य}\-शब्दस्य \textcolor{red}{तव}\-शब्देन सह साकाङ्क्षतया \textcolor{red}{तव} इत्यस्य च विशेषणत्वात् \textcolor{red}{राज्य\-प्रदानाय} इति कथं तत्पुरुष इति वाच्यम्। नित्य\-साकाङ्क्षा\-स्थले तस्य नियमस्यानङ्गीकारात्। \textcolor{red}{देवदत्तस्य गुरु\-कुलम्‌} (भा॰पा॰सू॰~२.१.१) इत्यादि\-भाष्य\-प्रयोगाच्च। यद्वा \textcolor{red}{कृते} इत्यध्याहार्यम्।
अर्थात् \textcolor{red}{तव कृते राज्य\-प्रदानाय} इति। सम्बन्धे षष्ठी।\end{sloppypar}
\section[तवैव]{तवैव}
\centering\textcolor{blue}{राज्यं रामस्य चैकेन वनवासो मुनिव्रतम्।\nopagebreak\\
ततः सत्यपरो राजा राज्यं दत्त्वा तवैव हि॥}\nopagebreak\\
\raggedleft{–~अ॰रा॰~२.७.७४}\\
\begin{sloppypar}\hyphenrules{nohyphenation}\justifying\noindent\hspace{10mm} अत्र कैकेयी राम\-वन\-वास\-घटनां वर्णयति। \textcolor{red}{तवैव राज्यं दत्त्वा पिता दिवं गतः}। \textcolor{red}{दत्त्वा} इति स्पष्टो \textcolor{red}{दा}\-धातोः (\textcolor{red}{डुदाञ् दाने} धा॰पा॰~१०९१) क्त्वान्त\-प्रयोगः। अत्र \textcolor{red}{तुभ्यम्‌} इति सम्प्रदानाच्चतुर्थीप्रयोग एव सम्यग्भाति। \textcolor{red}{तव} इति षष्ठी तु सम्बन्ध\-सामान्ये। यद्वा \textcolor{red}{तव कृते राज्यं दत्त्वा}।
यद्वा \textcolor{red}{तव शासनाय राज्यं दत्त्वा} इति।\footnote{\textcolor{red}{शासनाय} इत्यध्याहार्यमिति भावः। एवं \textcolor{red}{तव} इत्यत्र \textcolor{red}{कर्तृकर्मणोः कृति} (पा॰सू॰~२.३.६५) इत्यनेन षष्ठी।} यद्वा कैकेयी\-मुखात्सरस्वत्येव भरतं सङ्केतयति यदिदं राज्यं तुभ्यं महाराजेन न दत्तम्। यावद्रामागमनं शासनाय दत्तम्। अतस्तव तत्र स्वत्वं नास्ति। स्वत्वं तु राघवस्यैव। तस्मात् \textcolor{red}{रजकस्य वस्त्रं ददाति} इतिवदत्र राज्य\-परावर्तन\-सङ्केते षष्ठी।\end{sloppypar}
\section[मे]{मे}
\centering\textcolor{blue}{असम्भाष्याऽसि पापे मे घोरे त्वं भर्तृघातिनी।\nopagebreak\\
पापे त्वद्गर्भजातोऽहं पापवानस्मि साम्प्रतम्।\nopagebreak\\
अहमग्निं प्रवेक्ष्यामि विषं वा भक्षयाम्यहम्॥}\nopagebreak\\
\raggedleft{–~अ॰रा॰~२.७.८०}\\
\begin{sloppypar}\hyphenrules{nohyphenation}\justifying\noindent\hspace{10mm} कैकेयी\-मुखाद्राम\-वन\-वास\-कथा\-श्रवणेन जात\-कारुण्य\-कोपो भरतः कथयति \textcolor{red}{हे पापे त्वं मे असम्भाष्या}। अथ \textcolor{red}{ऋहलोर्ण्यत्‌} (पा॰सू॰~३.१.१२४) इत्यनेन \textcolor{red}{तयोरेव कृत्य\-क्त\-खलर्थाः} (पा॰सू॰~३.४.७०) इत्यनेन च सम्पूर्वक\-\textcolor{red}{भाष्‌}\-धातोः (\textcolor{red}{भाषँ व्यक्तायां वाचि} धा॰पा॰~६१२) कर्मणि ण्यत्प्रत्यये \textcolor{red}{चुटू} (पा॰सू॰~१.३.७) इत्यनेन णकारेत्सञ्ज्ञायां \textcolor{red}{तस्य लोपः} (पा॰सू॰~१.३.९) इत्यनेन लोपे \textcolor{red}{हलन्त्यम्‌} (पा॰सू॰~१.३.३) चेत्यनेन तकारेत्सञ्ज्ञायां तेनैव सूत्रेण लोपे विभक्ति\-कार्ये \textcolor{red}{सम्भाष्या}। \textcolor{red}{न सम्भाष्येत्यसम्भाष्या}।\footnote{\textcolor{red}{नञ्‌} (पा॰सू॰~२.२.६) इत्यनेन नञ्तत्पुरुषसमासः। \textcolor{red}{नलोपो नञः} (पा॰सू॰~६.३.७३) इत्यनेन नञो नलोपे \textcolor{red}{असम्भाष्या}।} \textcolor{red}{असम्भाष्या} इत्यत्र कर्मणि प्रत्यय\-विधानादनुक्तत्वाच्च कर्तुः \textcolor{red}{कर्तृकरणयोस्तृतीया} (पा॰सू॰~२.३.१८) इत्यनेन तृतीयया भवितव्यम्। \textcolor{red}{मया असम्भाष्यासि}। \textcolor{red}{मे} इति \textcolor{red}{कृत्यानां कर्तरि वा} (पा॰सू॰~२.३.७१) इत्यनेन वैकल्पकी षष्ठी। यद्वा \textcolor{red}{मे} इत्यस्मादग्रे \textcolor{red}{दृष्टौ} इत्यध्याहार्यम्। \textcolor{red}{पापे मे दृष्टावसम्भाष्या}। अत्राप्यवयवावयवि\-भाव\-सम्बन्धे षष्ठी।\end{sloppypar}
\section[मे]{मे}
\centering\textcolor{blue}{हा राम हा मे रघुवंशनाथ जातोऽसि मे त्वं परतः परात्मा।\nopagebreak\\
तथाऽपि दुःखं न जहाति मां वै विधिर्बलीयानिति मे मनीषा॥}\nopagebreak\\
\raggedleft{–~अ॰रा॰~२.७.८६}\\
\begin{sloppypar}\hyphenrules{nohyphenation}\justifying\noindent\hspace{10mm} अत्र शोक\-सन्तप्त\-हृदया भगवती कौसल्याम्बा श्रीभरत\-समक्षं श्रीराममुद्दिश्य विलपति। \textcolor{red}{परतः परात्मा त्वं मे जातोऽसि}। अत्र \textcolor{red}{मे} इति षष्ठ्यन्तं विचाराय। यतो हि जायमानस्याऽधारो माता। एवं पुत्र\-जन्मनि पिता निमित्त\-कारणं रजः\-शुक्र\-संयोगोऽसमवायि\-कारणमेवं माता समवायि\-कारणम्।\footnote{\textcolor{red}{कारणं त्रिविधं समवाय्यसमवायि\-निमित्त\-भेदात्। यत्समवेतं कार्यमुत्पद्यते तत्समवायि\-कारणम्। यथा तन्तवः पटस्य पटश्च स्वगतरूपादेः। कार्येण कारणेन वा सहैकस्मिन्नर्थे समवेतं सत्कारणमसमवायिकारणम्। यथा तन्तुसंयोगः पटस्य। तन्तुरूपं पटरूपस्य। तदुभयभिन्नं कारणं निमित्तकारणम्। यथा तुरीवेमादिकं पटस्य} (त॰स॰~४०)।} तत्र मातरि समवाय\-सम्बन्धेन सन्ततिरुत्पद्यते।\footnote{\textcolor{red}{यस्मिन्समवेतं सत्समवायेन सम्बद्धं सत्कार्यमुत्पद्यते तत्समवायि\-कारणमित्यर्थः ... समवाय\-सम्बन्धावच्छिन्न\-कार्यता\-निरूपित\-तादात्म्य\-सम्बन्धावच्छिन्न\-कारणत्वं समवायि\-कारणतत्वमिति} (त॰स॰ न्या॰बो॰व्या॰~४०)।} तथा च तस्या एवाऽधारत्वात्सप्तम्युचिता। आधारो ह्यत्रौपश्लेषिको बोध्यः संयोग\-रूपो न तु समवायः।\footnote{अनयोरयं भेदः~– \textcolor{red}{अयुत\-सिद्धयोस्सम्बन्धः समवायः। अन्ययोस्तु संयोगः} (बा॰म॰~६३३)।} अतः \textcolor{red}{मयि जातः} इत्युचितं यथा भागवते~–\end{sloppypar}
\centering\textcolor{red}{निशीथे तमउद्भूते जायमाने जनार्दने।\nopagebreak\\
देवक्यां देवरूपिण्यां विष्णुः सर्वगुहाशयः॥\nopagebreak\\
आविरासीद्यथा प्राच्यां दिशीन्दुरिव पुष्कलः॥}\nopagebreak\\
\raggedleft{–~भा॰पु॰~१०.३.८}\\
\begin{sloppypar}\hyphenrules{nohyphenation}\justifying\noindent इति। अत्र \textcolor{red}{देवक्याम्‌} इति सप्तम्यन्त\-प्रयोगः। तस्मादत्रापि सप्तम्युचितेति चेत्। उच्यते। \textcolor{red}{मे कुक्षौ} इत्यध्याहारे \textcolor{red}{सम्बन्धे षष्ठी}। यद्वा \textcolor{red}{जातः} इत्यस्य पुत्रोऽर्थः।\footnote{यथा \textcolor{red}{अयि जात कथयितव्यं कथय} (उ॰रा॰च॰~४.२३) इति भवभूति\-प्रयोगे।} अर्थात् \textcolor{red}{त्वं मे जातः पुत्रोऽसि}। इत्यर्थान्तर\-सम्बन्धे षष्ठी। अथवा \textcolor{red}{मे} इति चतुर्थ्यन्तं \textcolor{red}{तादर्थ्ये चतुर्थी वाच्या} (वा॰~२.३.१३) इति वार्त्तिकेन चतुर्थी। \textcolor{red}{कौसल्या\-हित\-कारी}\footnote{एतद्रूपान्तरम्–\textcolor{red}{कोसलात्मजाहितावहो} (मा॰भा॰~१.१९२.१)।} (रा॰च॰मा॰~१.१९२.१) इति मानसेऽपि समर्थितत्वात्। अथवा भगवतः कौसल्यां निमित्ती\-कृत्य प्राकट्यान्नैव सप्तमी। ततः \textcolor{red}{मे जातः} इत्यस्य \textcolor{red}{मे पुरतः प्रकटः} इति नव्यं समाधानम्।\end{sloppypar}
\section[मे]{मे}
\centering\textcolor{blue}{यत्र रामस्त्वया दृष्टस्तत्र मां नय सुव्रत।\nopagebreak\\
सीतया सहितो यत्र सुप्तस्तद्दर्शयस्व मे॥}\nopagebreak\\
\raggedleft{–~अ॰रा॰~२.८.२५}\\
\begin{sloppypar}\hyphenrules{nohyphenation}\justifying\noindent\hspace{10mm} अत्र चित्रकूटं गच्छन् भरतो निषादेन सङ्गम्य भगवच्छयन\-स्थानं पृच्छति \textcolor{red}{यत्र भगवान् सुप्तस्तन्मे दर्शयस्व}। अत्र \textcolor{red}{अहं पश्यानि त्वं प्रेरय} इत्यर्थे \textcolor{red}{त्वं मां दर्शयस्व} इत्येव द्वितीयोचिता। चतुर्थी\-प्रयोगस्तु सम्प्रदान\-विवक्षया। भरतो दान\-रूपेण स्थान\-दर्शनं याचते। यद्वा \textcolor{red}{मां सुखयितुं दर्शयस्व} इति \textcolor{red}{क्रियार्थोपपदस्य च कर्मणि स्थानिनः} (पा॰सू॰~२.३.१४) इत्यनेन चतुर्थी।\end{sloppypar}
\vspace{2mm}
\centering ॥ इत्ययोध्याकाण्डीयप्रयोगाणां विमर्शः ॥\nopagebreak\\
\vspace{4mm}
\centering\textcolor{blue}{\fontsize{16}{24}\selectfont अध्यात्मरामायणसंसृतानां बहिःस्थितानामिव पाणिनीयात्।\nopagebreak\\
मया पदानां प्रथमो विमर्शे शोधे परिच्छेद इति व्यधायि॥}\nopagebreak\\
\vspace{4mm}
\centering इत्यध्यात्म\-रामायणेऽपाणिनीय\-प्रयोगाणां\-विमर्श\-नामके शोध\-प्रबन्धे प्रथमाध्याये प्रथम\-परिच्छेदः।\\
\pagebreak
\pdfbookmark[1]{द्वितीयः परिच्छेदः}{Chap1Part2}
\phantomsection
\addtocontents{toc}{\protect\setcounter{tocdepth}{1}}
\addcontentsline{toc}{section}{द्वितीयः परिच्छेदः}
\addtocontents{toc}{\protect\setcounter{tocdepth}{0}}
\centering ॥ अथ प्रथमाध्याये द्वितीयः परिच्छेदः ॥\nopagebreak\\
\vspace{4mm}
\pdfbookmark[2]{अरण्यकाण्डम्‌}{Chap1Part2Kanda3}
\phantomsection
\addtocontents{toc}{\protect\setcounter{tocdepth}{2}}
\addcontentsline{toc}{subsection}{अरण्यकाण्डीयप्रयोगाणां विमर्शः}
\addtocontents{toc}{\protect\setcounter{tocdepth}{0}}
\centering ॥ अथारण्यकाण्डीयप्रयोगाणां विमर्शः ॥\nopagebreak\\
\section[सहितेन मे]{सहितेन मे}
\centering\textcolor{blue}{इतः परं प्रयत्नेन गन्तव्यं सहितेन मे।\nopagebreak\\
धनुर्गुणेन संयोज्य शरानपि करे दधत्॥}\nopagebreak\\
\raggedleft{–~अ॰रा॰~३.१.१२}\\
\begin{sloppypar}\hyphenrules{nohyphenation}\justifying\noindent\hspace{10mm} प्रयोगोऽयमध्यात्म\-रामायणेऽरण्य\-काण्डे प्रथमे सर्गे श्रीरामेण विहितः। चित्रकूटं त्यक्त्वा पश्चात्सीता\-लक्ष्मणाभ्यां सह कियद्दूरं गत्वा घोर\-काननं निरीक्ष्य लक्ष्मणं सतर्कयति यत् \textcolor{red}{इतः परं मे सहितेन त्वया गन्तव्यम्‌}। \textcolor{red}{सहितेन} इति लक्ष्मणस्य विशेषणम्। तथा च \textcolor{red}{सह\-युक्तेऽप्रधाने} (पा॰सू॰~२.३.१९) इत्यनेनात्र तृतीयायां \textcolor{red}{मया सहितेन} त्वया गन्तव्यं \textcolor{red}{मे} इति कथमिति चेत्। सम्बन्ध\-विवक्षायां षष्ठी। यद्वा \textcolor{red}{हितेन सह वर्तमानः सहितः}। \textcolor{red}{तेन सहेति तुल्य\-योगे} (पा॰सू॰~२.२.२८) इत्यनेन समासः।\footnote{\textcolor{red}{वोपसर्जनस्य} (पा॰सू॰~६.३.८२) इत्यनेन \textcolor{red}{सह} इत्यस्य \textcolor{red}{स} इत्यादेशः।} तेनात्र \textcolor{red}{मे} इत्यस्य \textcolor{red}{हित}\-शब्देनान्वयः। तत्र \textcolor{red}{मे हित\-सहितेन त्वया गन्तव्यम्‌} इति श्रीरामस्य तात्पर्यम्। न च \textcolor{red}{सापेक्षमसमर्थवत्‌} इति वचनेन \textcolor{red}{हित}\-शब्दस्य \textcolor{red}{मे} इत्यनेन सापेक्षतया कथं समासः। नित्य\-सापेक्ष\-स्थल उक्त\-नियमस्यानादरात्। यद्वा गमन\-क्रियायाः कर्ता श्रीरामः। एवं \textcolor{red}{हितेन सह वर्तमानेनेति सहितेन त्वया लक्ष्मणेन सह इतो मे गन्तव्यम्‌}। अत्र तव्यत्प्रत्ययो भावे।\footnote{\textcolor{red}{तयोरेव कृत्य\-क्तखलर्थाः} (पा॰सू॰~३.४.७०) इत्यनेन।} तस्मादनुक्त\-कर्तरि यद्यपि तृतीया प्राप्नोति तथाऽपि \textcolor{red}{कृत्यानां कर्तरि वा} (पा॰सू॰~२.३.७१) इति सूत्रेण \textcolor{red}{कृत्याः} (पा॰सू॰~३.१.९५) इत्यधिकारे विहितस्य \textcolor{red}{तव्यत्‌}\-प्रत्ययस्य कर्तरि राम\-वाच्येऽस्मच्छब्दे तृतीया\-स्थाने षष्ठी \textcolor{red}{मम} इति तस्य \textcolor{red}{मे} इत्यादेशः।\footnote{\textcolor{red}{तेमयावेकवचनस्य} (पा॰सू॰~८.१.२२) इत्यनेन।} यद्वा \textcolor{red}{बन्धुना} इत्यध्याहार्यम्। \textcolor{red}{मे बन्धुना त्वया गन्तव्यम्‌} इति विग्रहे पाल्य\-पालक\-भावे स्व\-स्वामि\-भावे वा षष्ठी। यद्वा \textcolor{red}{कृते} इत्यध्याहार्यम्।
\textcolor{red}{मे मम कृते सहितेन त्वया गन्तव्यम्‌}। यद्वा \textcolor{red}{मे} इति चतुर्थ्यन्तं \textcolor{red}{मे मह्यं गन्तव्यम्‌}। \textcolor{red}{तादर्थ्ये चतुर्थी वाच्या} (वा॰~२.३.१३) इति चतुर्थी। अर्थात् \textcolor{red}{इतः पूर्वं मया साकं त्वं स्व\-सुखानुभूत्याऽयासीः किन्त्वधुनेतः परं राक्षस\-सङ्ग्रामे महत्त्व\-पूर्ण\-योग\-दानाय मदर्थमेव गन्तव्यम्‌} इति श्रीरामस्य तात्पर्यं प्रतिभाति। राम\-रावण\-सङ्ग्रामे मेघनादादि\-वधे लक्ष्मणस्य महत्त्व\-पूर्णा भूमिका सर्वैरपि ज्ञाता। 
अथवा \textcolor{red}{मया} इत्यर्थे \textcolor{red}{मे} इत्यव्ययम्।\end{sloppypar}
\section[तव दास्यामि]{तव दास्यामि}
\centering\textcolor{blue}{तव सन्दर्शनाकाङ्क्षी राम त्वं परमेश्वरः।\nopagebreak\\
अद्य मत्तपसा सिद्धं यत्पुण्यं बहु विद्यते।\nopagebreak\\
तत्सर्वं तव दास्यामि ततो मुक्तिं व्रजाम्यहम्॥}\nopagebreak\\
\raggedleft{–~अ॰रा॰~३.२.५}\\
\begin{sloppypar}\hyphenrules{nohyphenation}\justifying\noindent\hspace{10mm} श्रीसीता\-लक्ष्मण\-समेत\-श्रीरामेण शरभङ्गश्चितायां शरीरं दिधक्षुः प्राप्यते। अथात्र \textcolor{red}{सम्पूर्णस्य तपसः पुण्यं तुभ्यं दास्यामि}। श्लोकेऽस्मिन् \textcolor{red}{तुभ्यं दास्यामि} इति प्रयोक्तव्ये \textcolor{red}{तव दास्यामि} इति प्रयुक्तम्। अत्र सम्प्रदाने सम्बन्ध\-विवक्षायां षष्ठी। यतो हि पुण्यस्य त्वयैव सह शाश्वतः सम्बन्धः। त्वमेव पुण्य\-निधानं त्वत्तः किमपि परं नास्ति। अतस्तवैव वस्तु तवैव दास्यामि मम किमपि स्वामित्वं नास्ति। यथा \textcolor{red}{रजकस्य वस्त्रं ददाति} इत्यत्र दाता स्व\-स्वत्वं न त्यजति।\footnote{\textcolor{red}{दानं चापुनर्ग्रहणाय स्व\-स्वत्व\-निवृत्तिपूर्वकं पर\-स्वत्वोत्पादनम्‌। अत एव “रजकस्य वस्त्रं ददाति” इत्यादौ न भवति} (त॰बो॰~५६९)। रजक\-वस्त्र\-दाने वस्त्र\-परावर्तनात्स्व\-स्वत्व\-निवृत्तिर्न पर\-स्वत्वोत्पादनञ्च न। तस्मान्न सम्प्रदानत्वम्। एवमेवात्र भगवत्पुण्यदानेऽपि भगवत एव पुण्य\-स्वामित्वात्स्व\-स्वत्व\-निवृत्तिर्न पर\-स्वत्वोत्पादनञ्च न। तस्मान्न सम्प्रदानत्वम्। अयं भावः।} पुनर्न सम्प्रदानं विनियमः। विनिमयो नाम प्रतिदानम्। \textcolor{red}{तव प्रतिदास्यामि} इदमेव वाक्यम्।\footnote{अत्र शरभङ्गः पुण्यं ददाति मुक्तिं च प्राप्नोति। भगवांश्च पुण्यं प्राप्नोति मुक्तिं च ददाति। इदमेव प्रतिदानम्। भगवान् पुण्यात्प्रतियच्छति मुक्तिमिति भावः।} \textcolor{red}{प्रति} इत्यस्य \textcolor{red}{विनाऽपि प्रत्ययं पूर्वोत्तर\-पद\-लोपो वक्तव्यः} (वा॰~५.३.८३) इत्यनेन लोपः। अथवा \textcolor{red}{तव दास्यामि} येन त्वं राक्षसानां
संहारं कुरु। उपासकानामयं नियमो यत्सत्कर्माणि समाचरन्ति तेषां फलञ्च स्वस्मायिष्टदेवाय प्रयच्छन्ति यथा भागवते~–\end{sloppypar}
\centering\textcolor{red}{कायेन वाचा मनसेन्द्रियैर्वा बुद्ध्याऽऽत्मना वाऽनुसृतस्वभावात्।\nopagebreak\\
करोति यद्यत्सकलं परस्मै नारायणायेति समर्पयेत्तत्॥}\nopagebreak\\
\raggedleft{–~भा॰पु॰~११.२.३६}\\
\begin{sloppypar}\hyphenrules{nohyphenation}\justifying\noindent अत एव \textcolor{red}{जय जय} इति कथयन्ति जनाः। वाल्मीकिरप्यन्तिम\-वाक्यं लिखति \textcolor{red}{बलं विष्णोश्च वर्धताम्‌} (वा॰रा॰~६.१२८.१२१)।\end{sloppypar}
\section[समर्प्य रामस्य]{समर्प्य रामस्य}
\centering\textcolor{blue}{समर्प्य रामस्य महत्सुपुण्यफलं विरक्तः शरभङ्गयोगी।\nopagebreak\\
चितिं समारोहयदप्रमेयं रामं ससीतं सहसा प्रणम्य॥}\nopagebreak\\
\raggedleft{–~अ॰रा॰~३.२.६}\\
\begin{sloppypar}\hyphenrules{nohyphenation}\justifying\noindent\hspace{10mm} अत्र भगवाञ्छरभङ्गः समस्तं पुण्य\-जातं श्रीरामाय समर्पयति। अत्र \textcolor{red}{रामस्य समर्प्य} इति लिखितम्।\footnote{णिजन्तात् \textcolor{red}{सम्‌}\-पूर्वकात् \textcolor{red}{ऋ}\-धातोः (\textcolor{red}{ऋ गतिप्रापणयोः} धा॰पा॰~९३६) क्त्वा प्रत्यये \textcolor{red}{समर्प्य} इति रूपं सिध्यति। ऋ~\arrow \textcolor{red}{हेतुमति च} (पा॰सू॰~३.१.२६)~\arrow ऋ~णिच्~\arrow ऋ~इ~\arrow \textcolor{red}{अर्ति\-ह्री\-व्ली\-री\-क्नूयी\-क्ष्माय्यातां पुङ्णौ} (पा॰सू॰~७.३.३६)~\arrow \textcolor{red}{आद्यन्तौ टकितौ} (पा॰सू॰~१.१.४६)~\arrow ऋ~पुँक्~इ~\arrow ऋ~प्~इ~\arrow \textcolor{red}{पुगन्त\-लघूपधस्य च} (पा॰सू॰~७.३.८६)~\arrow \textcolor{red}{उरण् रपरः} (पा॰सू॰~१.१.५१)~\arrow अर्प्~इ~\arrow अर्पि~\arrow \textcolor{red}{सनाद्यन्ता धातवः} (पा॰सू॰~३.१.३२)~\arrow धातु\-सञ्ज्ञा। सम्~अर्पि~\arrow \textcolor{red}{समानकर्तृकयोः पूर्वकाले} (पा॰सू॰~३.४.२१)~\arrow समर्पि~क्त्वा~\arrow \textcolor{red}{कुगतिप्रादयः}~\arrow \textcolor{red}{समासेऽनञ्पूर्वे क्त्वो ल्यप्‌} (पा॰सू॰~७.१.३७)~\arrow समर्पि~ल्यप्~\arrow समर्पि~य~\arrow \textcolor{red}{णेरनिटि} (पा॰सू॰~६.४.५१)~\arrow समर्प्~य~\arrow समर्प्य।} \textcolor{red}{समर्प्य} इति पदस्य योगेन चतुर्थी प्रयोक्तव्या \textcolor{red}{दा}\-धातु\-समानार्थकत्वात्। किन्त्वत्र षष्ठी किमपि विशेषं वक्ति। \textcolor{red}{रामस्यैव पुण्य\-जातं न्यास\-रूपेण मम पार्श्व आसीत्साम्प्रतं परावर्तयामि} इमं भावं ध्वनयितुं \textcolor{red}{रामस्य} इति। यद्वा \textcolor{red}{रामस्य करे समर्प्य} इत्यध्याहृते सम्बन्ध\-सामान्य\-विवक्षायां षष्ठी।\footnote{अवयवावयवि\-भाव\-मूलकश्चात्र सम्बन्धः।}\end{sloppypar}
\section[शरभङ्गयोगी]{शरभङ्गयोगी}
\centering\textcolor{blue}{समर्प्य रामस्य महत्सुपुण्यफलं विरक्तः शरभङ्गयोगी।\nopagebreak\\
चितिं समारोहयदप्रमेयं रामं ससीतं सहसा प्रणम्य॥}\nopagebreak\\
\raggedleft{–~अ॰रा॰~३.२.६}\\
\begin{sloppypar}\hyphenrules{nohyphenation}\justifying\noindent\hspace{10mm} अत्र राघवेन्द्रं प्रणम्य पुण्य\-जातं समर्प्य महर्षिः शरभङ्गश्चितां समारोहत्। \textcolor{red}{शरभङ्ग एव योगीति शरभङ्ग\-योगी} इति विग्रहे कर्मधारयः। अत्र \textcolor{red}{योगि}\-शब्दः \textcolor{red}{शरभङ्ग}\-शब्दस्य विशेषणम्। अन्यः कोऽपि शरभङ्गो न योग्येवायं हि शरभङ्ग\-योगी संसार\-जनस्तु रस\-भङ्ग\-योगीति। \textcolor{red}{विशेषणत्वं नाम विद्यमानत्वे सति विधेयान्वयित्वे
सतीतर\-व्यावर्तकत्वम्}। अत्रेदं ध्येयं विशेषणमुप\-लक्षणमुपाधिरिमे त्रयः प्रायः समाना अतिसूक्ष्ममन्तरमेतेषु। विशेषण\-लक्षणमुक्तम्। \textcolor{red}{उपाधिरविद्यमानत्वे सति विधेयान्वयित्वे सतीतर\-व्यावर्तकत्वम्}। \textcolor{red}{उपलक्षणं ह्यविद्यमानत्वे सति विधेयानन्वयित्वे सतीतर\-व्यावर्तकत्वम्}। यथा \textcolor{red}{काकवन्तो देवदत्तस्य गृहाः} इति। एवमेवेतर\-व्यावर्तकतया योगीति विशेषणं शरभङ्गश्च विशेष्यम्। तथा \textcolor{red}{विशेषणं विशेष्येण बहुलम्‌} (पा॰सू॰~२.१.५७) इत्यनेन समासे \textcolor{red}{प्रथमा\-निर्दिष्टं समास उपसर्जनम्‌} (पा॰सू॰~१.२.४३) समास\-विधायके शास्त्रे प्रथमया निर्दिष्टं पदं समास उपसर्जन\-सञ्ज्ञं स्यात्। अत्र समास\-विधायकं शास्त्रं \textcolor{red}{विशेषणं विशेष्येण बहुलम्‌} इति। प्रथमया निर्दिष्टं \textcolor{red}{विशेषणम्‌} इति। अत्रानुपूर्व्यां न तात्पर्यं तस्या अनुपयोगात्। \textcolor{red}{विशेषणम्‌} इत्यानुपूर्व्याः \textcolor{red}{विशेष्येण} इत्यानुपूर्व्या सह समासाभावादनुप\-योगाच्च। तस्मादानुपूर्व्या आनुपूर्व्यभिधेये तात्पर्यमत्र। विशेषण\-वाचक\-पदमित्यर्थः। एवं \textcolor{red}{नीलम् उत्पलम्‌} इत्यत्र \textcolor{red}{नीलम्‌} इत्यस्योपसर्जन\-सञ्ज्ञा। न च समास\-विधायके शास्त्रे त्रीणि पदानि \textcolor{red}{विशेषणम् विशेष्येण बहुलम्‌} इत्येतेष्वेकं तृतीयान्तं द्वे च प्रथमान्ते। यथा \textcolor{red}{विशेषणम्‌} इति प्रथमा\-निर्दिष्टं तथैव \textcolor{red}{बहुलम्‌} इत्यपि। अस्य कथं नोपसर्जन\-सञ्ज्ञेति चेत्सत्यम्। उपसर्जन\-सञ्ज्ञा\-विधायक\-सूत्रं \textcolor{red}{प्रथमा\-निर्दिष्टं समास उपसर्जनम्‌} इति। अत्र \textcolor{red}{समासे} इति सप्तमी च वैषयिकी। विषयता च घटकत्व\-रूपा। अर्थात्समासे घटकतया सत्प्रविष्टं सत्समास\-विधायक\-सूत्रे प्रथमया निर्दिष्टमेवोपसर्जन\-सञ्ज्ञम्। \textcolor{red}{बहुलम्‌} इति प्रथमया निर्दिष्टं किन्तु समासे घटकतया न प्रविष्टम्। अतो न दोषः। तस्मादुपसर्जन\-सञ्ज्ञायाम् \textcolor{red}{उपसर्जनं पूर्वम्‌} (पा॰सू॰~२.२.३०) इत्यनेन पूर्व\-प्रयोगे सति \textcolor{red}{योगि\-शरभङ्गः} न तु \textcolor{red}{शरभङ्ग\-योगी} इति। अत्रोच्यते। पूर्व\-निपात\-प्रकरणमनित्यमिति पूर्वमेवोप\-पादितम्। तेन शरभङ्ग\-शब्दस्य पूर्व\-प्रयोगः। अथवा विशेषण\-विशेष्य\-भावे काम\-चारः। प्रत्येकं पदं प्रत्येकस्य विशेषणं विशेष्यञ्च भवति। को योगी योगिनस्तु बहवस्तदा शरभङ्ग इति। अत्रापि त्वितर\-व्यावर्तकता। तस्माच्छरभङ्ग\-शब्दस्येतर\-व्यावर्तकतया विशेषणता। ततश्च पूर्व\-प्रयोगः। यद्वाऽनेन सूत्रेण समासस्त्यज्यताम्। \textcolor{red}{मयूर\-व्यंसकादयश्च} (पा॰सू॰~२.१.७२) इत्यनेन समासः। \textcolor{red}{भाष्याब्धिः} (भा॰प्र॰~मङ्गलाचरणे ६) इति कैयटेनापि सरणि\-स्वीकारात् \textcolor{red}{शरभङ्ग एव योगी} इति विग्रहः। यद्वा \textcolor{red}{शरं चिता तस्मिन् शरीरभङ्ग इति शरभङ्गस्तमेव योजयितुं शीलमस्य} इति विग्रहे \textcolor{red}{सुप्यजातौ णिनिस्ताच्छील्ये} (पा॰सू॰~३.२.७८) इत्यनेन णिनि\-प्रत्ययः। अनुबन्ध\-लोपे गुणे\footnote{\textcolor{red}{पुगन्त\-लघूपधस्य च} (पा॰सू॰~७.३.८६) इत्यनेन।} \textcolor{red}{चजोः कु घिण्ण्यतोः} (पा॰सू॰~७.३.५२) इत्यनेन जकारस्य कुत्वे\footnote{कुत्वे सति \textcolor{red}{स्थानेऽन्तरतमः} (पा॰सू॰~१.१.५०) इत्यनेन गत्वम्।} विभक्ति\-कार्ये \textcolor{red}{शरभङ्ग\-योगी} इति।\end{sloppypar}
\section[आदौ ऋषीणाम्]{आदौ ऋषीणाम्‌}
\centering\textcolor{blue}{शेषांशं शङ्खचक्रे द्वे भरतं सानुजं तथा।\nopagebreak\\
अतश्चादौ ऋषीणां त्वं दुःखं मोक्तुमिहार्हसि॥}\nopagebreak\\
\raggedleft{–~अ॰रा॰~३.२.१६}\\
\begin{sloppypar}\hyphenrules{nohyphenation}\justifying\noindent\hspace{10mm} अत्र सीता\-लक्ष्मण\-समन्वितं श्रीरामं दृष्ट्वा तं ब्रह्म ज्ञात्वा कथयन्ति मुनीश्वरा यत् \textcolor{red}{ऋषीणामादौ त्वं दुःखं हर्तुमर्हसि}। अत्र \textcolor{red}{एचोऽयवायावः} (पा॰सू॰~६.१.७८) इत्यनेनाऽवादेशोऽनिवार्यः। \textcolor{red}{आदावृषीणाम्‌} इत्येव भवेत् \textcolor{red}{आदौ ऋषीणाम्‌} इति त्वपाणिनीयमिव। अत्रोच्यते। सन्धेरविवक्षया। साम्प्रतं भगवान् राक्षसैः सह विग्रहं चिकीर्षत्यतो राक्षसैः सह सन्धिरनुचित इति भावं ध्वनयितुमत्र सन्धिर्न। यथा पत्न्या सह विजिघृक्षुर्भर्तृहरिर्नीति\-शतकस्य द्वितीये श्लोके कथयति~–\end{sloppypar}
\centering\textcolor{red}{यां चिन्तयामि सततं मयि सा विरक्ता\nopagebreak\\
साऽप्यन्यमिच्छति जनं स जनोऽन्यसक्तः।\\
अस्मत्कृते च परिशुष्यति काचिदन्या\\
धिक्ताञ्च तञ्च मदनञ्च इमाञ्च माञ्च॥}\nopagebreak\\
\raggedleft{–~भ॰नी॰~२}\\
\begin{sloppypar}\hyphenrules{nohyphenation}\justifying\noindent अत्र सन्धिरिष्टो नाऽसीदतः श्लोकेऽपि \textcolor{red}{मदनञ्च इमाञ्च} इत्यत्र न सन्धिस्तथैवात्रापि राक्षसैः सह विग्रह इष्टोऽतोऽसन्धि\-प्रयोगः। सन्धिर्वाक्ये विवक्षाधीनो भवतीति व्याकरण\-सिद्धान्तः। एकपदे सन्धिर्नित्यो यथा \textcolor{red}{रामेण}। धातूपसर्गयोस्तथा यथा \textcolor{red}{प्रार्च्छति}। समासेऽपि सन्धिर्नित्यो यथा \textcolor{red}{रामानुजः}। किन्तु वाक्ये तु विवक्षामपेक्षते तथा चोक्तम्~–\end{sloppypar}
\centering\textcolor{red}{संहितैकपदे नित्या नित्या धातूपसर्गयोः।\nopagebreak\\
नित्या समासे वाक्ये तु सा विवक्षामपेक्षते॥}\nopagebreak\\
\raggedleft{–~वै॰सि॰कौ॰~२२३२}\\
\begin{sloppypar}\hyphenrules{nohyphenation}\justifying\noindent अतोऽत्र न विवक्षा। अतो न सन्धिः। \end{sloppypar}
\section[त्वन्मन्त्र\-साधन\-परेषु]{त्वन्मन्त्र\-साधन\-परेषु}
\centering\textcolor{blue}{त्वं सर्वभूतहृदयेषु कृतालयोऽपि\nopagebreak\\
त्वन्मन्त्रजाप्यविमुखेषु तनोषि मायाम्।\nopagebreak\\
त्वन्मन्त्रसाधनपरेष्वपयाति माया\nopagebreak\\
सेवानुरूपफलदोऽसि यथा महीपः॥}\nopagebreak\\
\raggedleft{–~अ॰रा॰~३.२.२९}\\
\begin{sloppypar}\hyphenrules{nohyphenation}\justifying\noindent\hspace{10mm} नीलोत्पल\-दल\-श्यामं रामं राजीव\-लोचनं जानकी\-लक्ष्मण\-युतं जटा\-वल्कल\-धारिणं सुतीक्ष्णस्तीक्ष्णया भक्त्या तुष्टाव रघु\-नन्दनम्। \textcolor{red}{त्वन्मन्त्रसाधन\-परेषु} इति प्रायुङ्क्त सुतीक्ष्णः। तत्र हि \textcolor{red}{अपयाति माया} इति पद\-द्वय\-समभिव्याहारेण \textcolor{red}{त्वन्मन्त्रसाधनपरेषु} इत्यत्र पञ्चम्या भवितव्यम्। \textcolor{red}{अपयाति} इत्यस्य हि पूर्व\-देश\-त्यागानुकूलो व्यापारोऽर्थः। तत्र च कस्मादपयातीत्यपेक्षायां विश्लेषे साध्यमाने विश्लिष्ट\-मायाया अवधि\-भूततया \textcolor{red}{त्वन्मन्त्र\-साधन\-परेभ्यः} इत्यपादान\-सञ्ज्ञा\-फल\-भूत\-पञ्चमी। तथा च सूत्रम् \textcolor{red}{ध्रुवमपायेऽपादानम्‌} (पा॰सू॰~१.४.२४)। \textcolor{red}{अपायो विश्लेषस्तस्मिन् साध्ये ध्रुवमवधि\-भूतं कारकमपादानं स्यात्‌} (वै॰सि॰कौ॰~५८६) इति। सप्तम्यत्र विमर्श\-विषयः। अत्र समाधानम्। \textcolor{red}{सत्सु} इत्यध्याहार्यं तथा च \textcolor{red}{त्वन्मन्त्र\-साधन\-परेषु सत्सु पश्चान्मायाऽपयाति} इति। \textcolor{red}{यस्य च भावेन भाव\-लक्षणम्‌} (पा॰सू॰~२.३.३७) इत्यनेन सप्तमी। यद्वा \textcolor{red}{सप्तम्यधिकरणे च} (पा॰सू॰~२.३.३६) इत्यत्र चकारात्सप्तमी।\footnote{\textcolor{red}{चकाराद्दूरान्तिकार्थेभ्यः} (का॰वृ॰~२.३.३५, वै॰सि॰कौ॰~६३३, ल॰सि॰कौ॰~९०६)।} यद्वा \textcolor{red}{त्वन्मन्त्र\-साधन\-परानपेक्ष्य मायाऽपयाति} इति ल्यबन्त\-लोपे सति तत्रानादरार्थे सप्तमी।\footnote{\textcolor{red}{षष्ठी चानादरे} (पा॰सू॰~२.३.३८) इत्यनेन।} यद्वा \textcolor{red}{असाधु}\-पदस्याध्याहारे \textcolor{red}{साध्व\-साधु\-प्रयोगे च} (वा॰~२.३.३६) इत्यनेन \textcolor{red}{त्वन्मन्त्र\-साधन\-परेष्वसाधु\-मायाऽप्यपयाति} इति सप्तमी।
\end{sloppypar}
\section[मनो मे त्वरयति]{मनो मे त्वरयति}
\centering\textcolor{blue}{देहान्ते मम सायुज्यं लप्स्यसे नात्र संशयः।\nopagebreak\\
गुरुं ते द्रष्टुमिच्छामि ह्यगस्त्यं मुनिनायकम्।\nopagebreak\\
किञ्चित्कालं तत्र वस्तुं मनो मे त्वरयत्यलम्॥}\nopagebreak\\
\raggedleft{–~अ॰रा॰~३.२.३९}\\
\begin{sloppypar}\hyphenrules{nohyphenation}\justifying\noindent\hspace{10mm} अत्र सुतीक्ष्णमभिगम्य श्रीरामोऽगस्त्यं द्रष्टुमिच्छुः कथयति यत् \textcolor{red}{मे मनस्त्वरयति}। अत्र णिजन्तात् \textcolor{red}{त्वर्‌}\-धातोः\footnote{अस्य धातोर्घटाद्यन्तर्गणे पाठात् \textcolor{red}{घटादयो मितः} (धा॰पा॰ ग॰सू॰) इत्यनेन मित्त्वाण्णिचि \textcolor{red}{अत उपधायाः} (पा॰सू॰~७.२.११६) इत्यनेनोपधा\-वृद्धौ \textcolor{red}{मितां ह्रस्वः} (पा॰सू॰~६.४.९२) इत्यनेन ह्रस्वे \textcolor{red}{त्वरि} इत्यस्य धातु\-सञ्ज्ञायां लटि तिपि शपि गुणेऽयादेशे \textcolor{red}{त्वरयति}।} (\textcolor{red}{ञित्वराँ सम्भ्रमे} धा॰पा॰~७७५) कर्मतया \textcolor{red}{माम्‌} इति प्रयोक्तव्ये \textcolor{red}{मे} इति प्रयुक्तम्। अत्र \textcolor{red}{कर्तुरीप्सित\-तमं कर्म} (पा॰सू॰~१.४.४९) इत्यनेन कर्म\-सञ्ज्ञा। ततः \textcolor{red}{कर्मणि द्वितीया} (पा॰सू॰~२.३.२) इत्यनेन द्वितीया प्राप्ता किन्त्वत्र षष्ठी। \textcolor{red}{कर्मादीनामपि सम्बन्ध\-मात्र\-विवक्षायां षष्ठ्येव} (वै॰सि॰कौ॰~६०६) तथैवात्राप्येवम्। \textcolor{red}{मनः\-कर्तृक\-वर्तमान\-कालिक\-श्रीराम\-कर्मक\-मुनि\-दर्शन\-विषयकोत्कण्ठानुकूल\-व्यापारः} इति शाब्दबोधः। यद्वा \textcolor{red}{मे मनः} इत्यन्वये \textcolor{red}{मत्सम्बन्धि मनः}।
सम्बन्ध\-सामान्ये षष्ठी। तथा च \textcolor{red}{मे मनः मां त्वरयति} इत्यध्याहार्यम्।\end{sloppypar}
\section[सुतीक्ष्णेन]{सुतीक्ष्णेन}
\centering\textcolor{blue}{अथ रामः सुतीक्ष्णेन जानक्या लक्ष्मणेन च।\nopagebreak\\
अगस्त्यस्यानुजस्थानं मध्याह्ने समपद्यत॥}\nopagebreak\\
\raggedleft{–~अ॰रा॰~३.३.१}\\
\begin{sloppypar}\hyphenrules{nohyphenation}\justifying\noindent\hspace{10mm} अत्र सह\-शब्दं विनाऽपि तृतीया। विनाऽपि सह\-शब्दं तृतीया\-विधानात्।\footnote{\textcolor{red}{विनाऽपि तद्योगं तृतीया। वृद्धो यूनेत्यादिनिर्देशात्‌} (वै॰सि॰कौ॰~५६४)।} यद्वा \textcolor{red}{इत्थं\-भूत\-लक्षणे} (पा॰सू॰~२.३.२१) इति सूत्रेण तृतीया।\footnote{सुतीक्ष्ण\-ज्ञाप्य\-रामत्व\-विशिष्टः श्रीरामोऽग्निजिह्व\-स्थानं समपद्यतेति भावः।} यद्वा \textcolor{red}{हेतौ} (पा॰सू॰~२.३.२३) इत्यनेन तृतीया।\footnote{\textcolor{red}{पुण्येन दृष्टो हरिः} (वै॰सि॰कौ॰~५६८) इतिवदत्राग्निजिह्व\-मुनि\-स्थान\-गमने सुतीक्ष्णो हेतुरिति भावः।} यद्वा \textcolor{red}{प्रकृत्यादिभ्य उप\-सङ्ख्यानम्‌} (पा॰सू॰~२.३.१८) इत्यनेनाभेदे तृतीया।\footnote{एतत्पूर्वसर्गे \textcolor{red}{निरपेक्षा नान्यगतास्तेषां दृश्योऽहमन्वहम्} (अ॰रा॰~३.२.३७) इत्यत्र श्रीरामेण सुतीक्ष्णस्य कृते स्वस्य नित्यदृश्यत्वमुक्तम्। \textcolor{red}{देहान्ते मां सायुज्यं लप्स्यसे नात्र संशयः} (अ॰रा॰~३.२.३९) इत्यत्राभेदश्चोक्तः। तस्मादभेदे तृतीयेति भावः।}\end{sloppypar}
\section[अगस्त्यमुनिवर्याय]{अगस्त्यमुनिवर्याय}
\centering\textcolor{blue}{सुतीक्ष्ण गच्छ त्वं शीघ्रमागतं मां निवेदय॥\\
अगस्त्यमुनिवर्याय सीतया लक्ष्मणेन च।\nopagebreak\\
महाप्रसाद इत्युक्त्वा सुतीक्ष्णः प्रययौ गुरोः॥}\nopagebreak\\
\raggedleft{–~अ॰रा॰~३.३.५-६}\\
\begin{sloppypar}\hyphenrules{nohyphenation}\justifying\noindent\hspace{10mm} अत्र श्रीरामः सुतीक्ष्णं प्रति कथयति \textcolor{red}{त्वमागतं मामगस्त्य\-मुनि\-वर्याय निवेदय}। \textcolor{red}{गति\-बुद्धि\-प्रत्यवसानार्थ\-शब्द\-कर्माकर्मकाणामणि कर्ता स णौ} (पा॰सू॰~१.४.५२) इत्यनेन नि\-पूर्वक\-\textcolor{red}{विद्‌}\-धातोः (\textcolor{red}{विदँ ज्ञाने} धा॰पा॰~१०६४) बुद्ध्यर्थतया शब्द\-कर्मतया च \textcolor{red}{अगस्त्यमुनिवर्य} इत्यस्य कर्म\-सञ्ज्ञा प्राप्ता। एवं तदनुगामिनी द्वितीया\-विभक्तिः प्राप्ता। किन्त्वत्र चतुर्थ्यपाणिनीयेव। परं विचारे कृत इयमपि पाणिन्यनुकूला। अत्र \textcolor{red}{अगस्त्य\-मुनि\-वर्यं तोषयितुं मां निवेदय} इति \textcolor{red}{क्रियार्थोपपदस्य च कर्मणि स्थानिनः} (पा॰सू॰~२.३.१४) इत्यनेन चतुर्थी। यद्वा \textcolor{red}{तादर्थ्ये चतुर्थी वाच्या} (वा॰~२.३.१३) इत्यनेन \textcolor{red}{मुक्तये हरिं भजति} इत्यादिवच्चतुर्थी। यद्वा \textcolor{red}{अगस्त्य\-मुनि\-वर्याय} इति सुतीक्ष्ण\-सम्बोधनम्। एवम् \textcolor{red}{अगस्त्य\-मुनि\-वर्यमयतेऽन्तेवासित्वेन सेवार्थं गच्छतीत्यगस्त्य\-मुनि\-वर्यायः} तत्सम्बुद्धौ \textcolor{red}{अगस्त्य\-मुनि\-वर्याय}। अत्र \textcolor{red}{कर्मण्यण्‌} (पा॰सू॰~३.२.१) इत्यनेन \textcolor{red}{अण्‌} प्रत्ययः।\footnote{\textcolor{red}{अयँ गतौ} (धा॰पा॰~४७४) इति धातोः \textcolor{red}{अगस्त्य\-मुनि\-वर्यम्‌} इति कर्मोपपदे।} ततश्च \textcolor{red}{उपपदमतिङ्‌} (पा॰सू॰~२.२.१९) इत्यनेन समासे प्राप्ते \textcolor{red}{कृत्तद्धित\-समासाश्च} (पा॰सू॰~१.२.४६) इत्यनेन विभक्तिः प्राप्ता। द्वयोर्मध्ये कतरेण भाव्यमिति सामञ्जस्ये \textcolor{red}{गति\-कारकोपपदानां कृद्भिः सह समास\-वचनं प्राक्सुबुत्पत्तेः} (भा॰पा॰सू॰~८.४.११) इत्यनेन समास उपक्रान्ते पूर्वं \textcolor{red}{कर्तृ\-कर्मणोः कृति} (पा॰सू॰~२.३.६५) इत्यनेन कृद्योगे षष्ठी पश्चात्समासे पश्चाद्विभक्ति\-लोपे दीर्घे पुनर्विभक्ति\-कार्ये \textcolor{red}{अगस्त्य\-मुनि\-वर्यायः} तत्सम्बुद्धौ \textcolor{red}{हे अगस्त्य\-मुनि\-वर्याय} इति न दोषः। अर्थात्त्वं निरन्तरमगस्त्य\-मुनि\-वर्यमुपगच्छसि सेवार्थमतस्त्वं तस्य स्वभावेन परिचितोऽतो मामप्यागतं निवेदय।\footnote{यद्वा \textcolor{red}{अयत इत्ययः}। \textcolor{red}{अयः अयते अच् गन्तरि} इति वाचस्पत्यकारः। \textcolor{red}{नन्दि\-ग्रहि\-पचादिभ्यो ल्युणिन्यचः} (पा॰सू॰~३.१.१३४) इत्यनेन कर्तर्यच्। अगस्त्य\-मुनि\-वर्यस्यायः \textcolor{red}{अगस्त्यमुनिवर्यायः} तत्सम्बुद्धौ \textcolor{red}{अगस्त्य\-मुनि\-वर्याय}। \textcolor{red}{कर्तृकर्मणोः कृति} (पा॰सू॰~२.३.६५) इत्यनेन कृत्षष्ठ्यां \textcolor{red}{कृद्योगा च षष्ठी समस्यत इति वक्तव्यम्‌} (वा॰~२.२.८) इति वार्त्तिकेन तत्पुरुष\-समासः \textcolor{red}{इध्मप्रव्रश्चनः} इतिवत्।}\end{sloppypar}
\section[गुरोः]{गुरोः}
\centering\textcolor{blue}{अगस्त्यमुनिवर्याय सीतया लक्ष्मणेन च।\nopagebreak\\
महाप्रसाद इत्युक्त्वा सुतीक्ष्णः प्रययौ गुरोः॥}\nopagebreak\\
\raggedleft{–~अ॰रा॰~३.३.६}\\
\begin{sloppypar}\hyphenrules{nohyphenation}\justifying\noindent\hspace{10mm} अत्र श्रीराम\-वचनं श्रुत्वा सुतीक्ष्णः सहर्षमगस्त्यं प्रति गतः। अत्र \textcolor{red}{प्रययौ गुरोः} इति षष्ठ्यन्तं प्रयुक्तम्। तदपाणिनीय\-भ्रमावहम्। यतो हि \textcolor{red}{कर्तुरीप्सिततमं कर्म} (पा॰सू॰~१.४.४९) इति सूत्रम्। अस्य सामान्योऽर्थः \textcolor{red}{कर्तुः क्रिययाऽऽप्तुमिष्टतमं कारकं कर्म\-सञ्ज्ञं स्यात्‌} (वै॰सि॰कौ॰~५३५)। विशेषस्तु \textcolor{red}{कारके} (पा॰सू॰~१.४.२३) इत्यधिकार\-सूत्रेण क्रिया\-पदस्य लाभः। एवं या या क्रिया सा सा कर्तारं विनाऽनुपपन्नेति क्रिययैव कर्तुराक्षेपे सिद्धे पुनः कर्तुर्ग्रहणं व्यर्थं सज्ज्ञापयति यत् 
\textcolor{red}{प्रकृति\-धातूपात्त\-कर्तृ\-वृत्ति\-व्यापार\-प्रयोज्य\-फलाश्रयः कर्म}। तेन \textcolor{red}{भक्तो रामं भजति} इत्यत्र
प्रकृति\-धातूपात्त\-कर्ता भक्तस्तद्वृत्ति\-व्यापारो भजनानुकूलस्तत्प्रयोज्य\-फलं भजनं तदाश्रयो राम इत्यत्र कर्म\-सञ्ज्ञा। असति कर्तृ\-ग्रहणे क्रियाया लब्धस्य कर्तृ\-पदस्य प्रकृति\-धातु\-कर्तेत्यर्थो नावगन्तुं शक्येत। एवं \textcolor{red}{माषेष्वश्वं बध्नाति} इत्यत्र माषाणामपि कर्म\-सञ्ज्ञा स्यात्तथा च कर्ताऽश्वस्तद्वृत्ति\-व्यापारो गल\-विलाधः\-संयोगरूपस्तत्प्रयोज्य\-फलं भक्षणानुकूलं तदाश्रयः कर्म माषा एवेति माषाणामेव कर्म\-सञ्ज्ञा स्यात्। सति \textcolor{red}{कर्तुः} इति पद\-ग्रहणे प्रकृति\-धातूपात्तो यो व्यापारस्तदाश्रयो यः कर्तेत्यर्थे सम्पन्ने प्रकृति\-धातुः \textcolor{red}{बन्ध्‌}\-धातुः (\textcolor{red}{बन्धँ बन्धने} धा॰पा॰~१५०८) तद्वृत्ति\-व्यापारः शङ्कु\-संयोगानुकूलस्तदाश्रयो देवदत्तस्तादृक्कर्तृ\-वृत्ति\-व्यापार\-प्रयोज्यं फलं बन्धन\-रूपं तदाश्रयः कर्माश्व एव। एवं \textcolor{red}{कर्तुः} इत्यत्र षष्ठी \textcolor{red}{क्तस्य च वर्तमाने} (पा॰सू॰~२.३.६७) इत्यनेन। साऽपि कर्तर्येव। अतः कर्तृ\-पदमपि कर्तारं बोधयति षष्ठ्यपि। \textcolor{red}{कर्तैव यः कर्ता} इत्यर्थे सति \textcolor{red}{प्रधान\-भूत\-व्यापाराश्रय\-कर्ता} इत्यर्थोऽवगम्यते। तथा च \textcolor{red}{ईप्सिततमम्‌} इति \textcolor{red}{आप्‌}\-धातोः (\textcolor{red}{आपॢँ लम्भने} धा॰पा॰~१८३९) सन्नन्त\-रूपम्।
\textcolor{red}{आप्तुमिष्टमीप्सितम्‌}।\footnote{\textcolor{red}{आप्तुमिष्यमाणमीप्सितम्} (बा॰म॰~५३५) इति बाल\-मनोरमा। \textcolor{red}{इष्टम्} इत्यत्रापि \textcolor{red}{मतिबुद्धि\-पूजार्थेभ्यश्च} (पा॰सू॰~३.२.१८८) इत्यनेनेच्छार्थे वर्तमान\-काल एव \textcolor{red}{क्त}\-प्रत्ययः। तस्मादुभे अप्युक्ती समानार्थिके। \textcolor{red}{आपॢँ लम्भने} (धा॰पा॰~१८३९)~\arrow आप्‌~\arrow \textcolor{red}{धातोः कर्मणः समानकर्तृकादिच्छायां वा} (पा॰सू॰~३.१.७)~\arrow आप्~सन्~\arrow आप्~स~\arrow \textcolor{red}{एकाच उपदेशेऽनुदात्तात्‌} (पा॰सू॰~७.२.१०)~\arrow इडागम\-निषेधः~\arrow \textcolor{red}{आप्ज्ञप्यृधामीत्‌} (पा॰सू॰~७.४.५५)~\arrow ईप्~स~\arrow \textcolor{red}{सन्यङोः} (पा॰सू॰~६.१.९)~\arrow ईप्~ईप्~स~\arrow \textcolor{red}{पूर्वोऽभ्यासः} (पा॰सू॰~६.१.४)~\arrow \textcolor{red}{अत्र लोपोऽभ्यासस्य} (पा॰सू॰~७.४.५८)~\arrow ईप्~स~\arrow \textcolor{red}{सनाद्यन्ता धातवः} (पा॰सू॰~३.१.३२)~\arrow धातु\-सञ्ज्ञा~\arrow \textcolor{red}{तयोरेव कृत्य\-क्तखलर्थाः} (पा॰सू॰~३.४.७०)~\arrow \textcolor{red}{मतिबुद्धि\-पूजार्थेभ्यश्च} (पा॰सू॰~३.२.१८८)~\arrow ईप्स~क्त~\arrow ईप्स~त~\arrow \textcolor{red}{आर्धधातुकस्येड्वलादेः} (पा॰सू॰~७.२.३५)~\arrow ईप्स~इट्~त~\arrow ईप्स~इ~त~\arrow ईप्सित~\arrow विभक्ति\-कार्यम्~\arrow ईप्सित~सुँ~\arrow \textcolor{red}{अतोऽम्} (पा॰सू॰~७.१.२४)~\arrow ईप्सित~अम्~\arrow \textcolor{red}{अमि पूर्वः} (पा॰सू॰~६.१.१०७)~\arrow ईप्सितम्।} \textcolor{red}{अतिशयेनेप्सितमितीप्सित\-तमम्‌}\footnote{\textcolor{red}{अतिशयेनेप्सितमीप्सित\-तमम्‌} (बा॰म॰~५३५)।} इति विग्रहे \textcolor{red}{अतिशायने तमबिष्ठनौ} (पा॰सू॰~५.३.५५) इत्यनेन \textcolor{red}{तमप्‌}\-प्रत्ययः। तथा च प्रकृति\-धातूपात्त\-प्रधान\-भूत\-व्यापाराश्रय\-कर्तृ\-वृत्ति\-व्यापार\-प्रयोज्य\-फलाश्रयता\-प्रकारिकेच्छा\-निरूपितोद्देश्यता\-फलाश्रयत्वावच्छेदकं कर्म। एवं च \textcolor{red}{भक्तो रामं भजति} इत्यत्र प्रकृति\-धातुः \textcolor{red}{भज्‌}\-धातुः (\textcolor{red}{भजँ सेवायाम्‌} धा॰पा॰~९९८) तत्प्रधान\-भूत\-व्यापारो भजनानुकूल\-व्यापारस्तदाश्रयः कर्ता भक्तस्तद्वृत्तिर्भजनानुकूल\-व्यापार\-प्रकारिकेच्छा यद्वृत्ति\-व्यापारेण रामः सन्तुष्टो भवतीत्याकारिका तन्निरूपितोद्देश्यता\-फलाश्रयत्वावच्छेदकं राम एव तस्यैव कर्म\-सञ्ज्ञा।
इत्थं \textcolor{red}{कर्तुरीप्सिततमं कर्म} (पा॰सू॰~१.४.४९) इत्यत्रेप्सिततम\-ग्रहणाभावे \textcolor{red}{कर्तुः कर्म} इति सूत्रे सम्पन्ने \textcolor{red}{कर्तुरुद्देश्यं कर्म} इत्यर्थे \textcolor{red}{पयसौदनं भुङ्क्ते} इति प्रत्युदाहरणम्। अर्थात्पयसा मिश्रमोदनं भुङ्क्ते। कृत\-भोजनं प्रति यदि कश्चिद्ब्रवीति यत् \textcolor{red}{ओदनं भुङ्क्ष्व तुभ्यं पयो दास्यामि} ततश्च स भोजने प्रवर्तते तर्ह्येव \textcolor{red}{पय ओदनं भुङ्क्ते} इति प्रयुज्यते। \textcolor{red}{पयसा मिश्रमोदनं भुङ्क्ते} इत्यर्थे \textcolor{red}{पयसौदनं भुङ्क्ते} इति प्रयुज्यते। अत्र तृतीया। \textcolor{red}{कर्तुरुद्देश्यं कर्म} इत्यर्थे कृतेऽत्रापि द्वितीया स्यात्। पयसोऽपि कर्तुरुद्देश्यत्वात्। पय उद्देश्यं किन्त्वीप्सिततमं नास्ति। न च \textcolor{red}{तमप्‌}\-ग्रहणं 
मा भूत् \textcolor{red}{ईप्सित}\-ग्रहणेनैव कार्ये सिद्धे तथा च \textcolor{red}{कर्तुरीप्सितं कर्म} इत्येव सूत्रं स्यात्। तथा सति \textcolor{red}{अग्नेर्माणवकं वारयति} इत्यत्र \textcolor{red}{अग्नेः} इत्यत्रापि कर्म\-सञ्ज्ञा स्यात्। \textcolor{red}{वारयति} इत्यस्यार्थः \textcolor{red}{अग्नि\-संयोगानुकूल\-व्यापाराभावानुकूल\-व्यापारः}। मद्वृत्ति\-व्यापारेणाग्निर्बालक\-संयोगाभाववान् भवत्वितीच्छाश्रयः। \textcolor{red}{अग्नेः} इत्यत्रापि कर्म\-सञ्ज्ञा मा भूत्तस्मात्तमब्ग्रहणम्।
न च \textcolor{red}{वारणार्थानामीप्सितः} (पा॰सू॰~१.४.२७) इत्यपादान\-सञ्ज्ञा कर्म\-सञ्ज्ञां बाधिष्यते सा च \textcolor{red}{माणवकम्} इत्यत्र स्यादियं कर्म\-सञ्ज्ञा च। परत्वात्कर्म\-सञ्ज्ञैव।\footnote{\textcolor{red}{“गाम्” इत्यत्रेप्सितत्व\-प्रयुक्ताऽपादानसञ्ज्ञा न भवति। ईप्सिततमत्व\-विवक्षायां परत्वात्कर्म\-संज्ञाप्रवृत्तेः} (त॰बो॰~५९०)।} सूत्राभावे \textcolor{red}{अग्नेः} इत्यत्र कर्म\-सञ्ज्ञां च को वारयिष्यति।\footnote{\textcolor{red}{कर्तुरीप्सिततमं कर्म} (पा॰सू॰~१.४.४९) इति सूत्रस्याभावे \textcolor{red}{अकथितं च} (पा॰सू॰~१.४.५१) इत्यनेनापि \textcolor{red}{माणवकम्} इत्यत्रेप्सित\-तमस्य कर्मसञ्ज्ञा प्राप्नोति परन्त्वेतेन सूत्रेणेप्सितस्याप्यकथितस्य कर्म\-सञ्ज्ञा\-प्राप्तौ \textcolor{red}{*अग्निं माणवकं वारयति} इत्यनिष्ट\-प्रयोगः स्यात्तस्मात्सूत्रारम्भ आवश्यक इति भावः।}
न च \textcolor{red}{अधि\-शीङ्\-स्थासां कर्म} (पा॰सू॰~१.४.४६) इत्यतः \textcolor{red}{कर्म} इति पदमनुवर्त्यतां पुनरत्र \textcolor{red}{कर्म}\-ग्रहणं किमर्थमिति चेत्। यदि ततः \textcolor{red}{कर्म} इति पदं तर्हि \textcolor{red}{एक\-योग\-निर्दिष्टानां सह वा प्रवृत्तिः सह वा निवृत्तिः} (प॰शे॰~१७) इति नियमेनाधारस्यैव कर्म\-सञ्ज्ञा स्यात् \textcolor{red}{गेहं प्रविशति} इत्यादावेव। तदर्थं कर्म\-ग्रहणम्। \textcolor{red}{क्वचिदेक\-देशोऽप्यनुवर्तते} (प॰शे॰~१८) इति नियमेनाधार\-निवृत्तौ कर्म\-ग्रहणमदृष्टार्थम्। इत्यर्थं \textcolor{red}{कर्तुरीप्सिततमं कर्म} इत्यनेन सञ्ज्ञाऽनिवार्या। तथा च \textcolor{red}{प्रययौ} इति \textcolor{red}{या}\-धातोः (\textcolor{red}{या प्रापणे} धा॰पा॰~१०४९) कर्ता सुतीक्ष्णस्तद्\-वृत्ति\-व्यापार उत्तर\-देश\-संयोगानुकूल\-व्यापारस्तत्प्रयोज्य\-फलाश्रयत्व\-प्रकारिकेच्छा मद्वृत्तिव्यापारेण गुरुर्मत्संयोगवान् भवतु तन्निरूपितोद्देश्यता गुरावेव ततः स एव कर्म। अतोऽत्र कथं न कर्म\-सञ्ज्ञेति चेत्। उच्यते। कर्मणः शेषत्व\-विवक्षायां षष्ठी। कर्म\-सञ्ज्ञा तु जातैव केवलं तत्फल\-भूता द्वितीया न।
ननु \textcolor{red}{या या सञ्ज्ञा सा सा फलवती भवति} इति न्यायेन फलाभावेन कर्म\-सञ्ज्ञायाः किं प्रयोजनम्।
असत्याञ्च कर्म\-सञ्ज्ञायां सूत्राप्रवृत्तौ न साधुताऽभावे साधुताया न पुण्य\-जनकतावच्छेदकतेति चेत्। कर्म\-सञ्ज्ञायाः फलं न केवलं द्वितीया। अन्यान्य\-फलानि सन्ति। अत्र किं फलमिति चेत्। \textcolor{red}{गुरोः प्रययौ} इत्यत्र हि \textcolor{red}{गुरु\-कर्मक\-भूतकालावच्छिन्नोत्तर\-देश\-संयोगानुकूल\-व्यापारः} इत्यत्र शाब्द\-बोधे कर्म\-मूलक\-सम्बन्ध\-बोधयोः फलम्। यद्वा \textcolor{red}{गुरोः सकाशं प्रययौ} इति पदाध्याहारे सम्बन्धे षष्ठी।\end{sloppypar}
\section[शिष्येभ्यः]{शिष्येभ्यः}
\centering\textcolor{blue}{व्याख्यातराममन्त्रार्थं शिष्येभ्यश्चातिभक्तितः।\nopagebreak\\
दृष्ट्वाऽगस्त्यं मुनिश्रेष्ठं सुतीक्ष्णः प्रययौ मुनेः॥}\nopagebreak\\
\raggedleft{–~अ॰रा॰~३.३.८}\\
\begin{sloppypar}\hyphenrules{nohyphenation}\justifying\noindent\hspace{10mm} अत्र शिष्यान् प्रति राम\-मन्त्रार्थं व्याचक्षाणस्यागस्त्यस्य स्थितिं वर्णयति। अत्र \textcolor{red}{शिष्यान् प्रति} इति प्रयोक्तव्यमासीत्। \textcolor{red}{शिष्येभ्यः} इति चतुर्थी\-प्रयोगस्तु \textcolor{red}{शिष्यान् बोधयितुम्‌} इत्यप्रयुज्यमान\-तुमुनः कर्मणि चतुर्थी।\footnote{\textcolor{red}{क्रियार्थोप\-पदस्य च कर्मणि स्थानिनः} (पा॰सू॰~२.३.१४) इत्यनेन।} यद्वा सुख\-योगे सति चतुर्थी।\footnote{\textcolor{red}{शिष्येभ्यः सुखाय} इत्यध्याहार्यमिति भावः।} यद्वा \textcolor{red}{शिष्येभ्यः हिताय} इत्यध्याहारे \textcolor{red}{हित\-योगे च} (वा॰~२.३.१३) इत्यनेन चतुर्थी। यद्वा \textcolor{red}{तादर्थ्ये चतुर्थी वाच्या} (वा॰~२.३.१३) इति वार्त्तिकेन चतुर्थी। न च केभ्यो राम\-मन्त्रार्थो व्याख्यात इत्यपेक्षायां \textcolor{red}{शिष्येभ्यः} इत्यस्य व्याख्यात\-घटक\-व्याख्यान\-क्रियाया अपेक्षितत्वात्कथं बहुव्रीहिरिति चेत्। नित्य\-साकाङ्क्ष\-स्थल उक्त\-नियमस्य प्रसक्त्यभावात्। यद्वा \textcolor{red}{शिष्यान् आहूय} वा \textcolor{red}{शिष्यान् अपेक्ष्य व्याख्यात\-राम\-मन्त्रार्थः} इति योजनया \textcolor{red}{ल्यब्लोपे कर्मण्यधिकरणे च} (वा॰~२.३.२८) इति वार्त्तिकेनात्र पञ्चमी। अस्य वार्त्तिकस्यार्थः \textcolor{red}{ल्यबन्तस्य लोपे सति तस्य कर्मण्यधिकरणे च पञ्चमी भवति}। अत्र हि ल्यबन्तम् \textcolor{red}{आहूय} अपेक्षतयाऽस्य कर्म \textcolor{red}{शिष्यान्‌} इति ततः पञ्चमी \textcolor{red}{श्वशुराज्जिह्रेति} (वै॰सि॰कौ॰~५९४) इतिवत्। इत्थं \textcolor{red}{शिष्येभ्यः} इति चतुर्थ्यन्तः पञ्चम्यन्तो वा पाणिनीय एव। \end{sloppypar}
\section[ब्रवीमि ते]{ब्रवीमि ते}
\centering\textcolor{blue}{किं राम बहुनोक्तेन सारं किञ्चिद्ब्रवीमि ते।\nopagebreak\\
साधुसङ्गतिरेवात्र मोक्षहेतुरुदाहृता॥}\nopagebreak\\
\raggedleft{–~अ॰रा॰~३.३.३६}\\
\begin{sloppypar}\hyphenrules{nohyphenation}\justifying\noindent\hspace{10mm} अत्रागस्त्यः श्रीरामं प्रति गोपनीय\-विषयं वर्णयति यत् \textcolor{red}{ते किञ्चित्सारं ब्रवीमि}। \textcolor{red}{ब्रू}\-धातुः (\textcolor{red}{ब्रूञ् व्यक्तायां वाचि} धा॰पा॰~१०४४) अत्राकथित\-परिगणित\-धात्वन्तर्गतः~– \textcolor{red}{चिब्रूशासुजिमथ्मुषाम्‌} (वै॰सि॰कौ॰~५३९)। अत्र \textcolor{red}{त्वां ब्रवीमि} इति पाणिनीयम्। \textcolor{red}{ते ब्रवीमि} इति कथम्। कर्मणः शेषत्व\-विवक्षायां षष्ठी। यद्वा \textcolor{red}{समक्षम्‌} इत्यध्याहार्यम्। एवं \textcolor{red}{ते} इति\-शब्दे सम्बन्ध\-विवक्षायां षष्ठी। अथवा \textcolor{red}{त्वां स्तोतुं ब्रवीमि} इति तुमुन्यप्रयुक्ते तत्कर्मणि \textcolor{red}{क्रियार्थोपपदस्य च कर्मणि स्थानिनः} (पा॰सू॰~२.३.१४) इति चतुर्थी। अथवा \textcolor{red}{ते} इत्यस्य \textcolor{red}{सारम्‌} इत्यनेनान्वयस्तेन सम्बन्धे षष्ठी। \textcolor{red}{त्वत्सम्बन्धि\-सारं ब्रवीमि} इत्यर्थः। अथवा \textcolor{red}{शृण्वतः} इति शत्रन्तमध्याहार्यम्। \textcolor{red}{ते शृण्वतः सारं ब्रवीमि} इत्यर्थे \textcolor{red}{यस्य च भावेन भाव\-लक्षणम्‌} (पा॰सू॰~२.३.३७) इति षष्ठी।\footnote{\textcolor{red}{दूरान्तिकार्थैः षष्ठ्यन्यतरस्याम्‌} (पा॰सू॰~२.३.३४) इत्यतः \textcolor{red}{षष्ठी} इत्यनुवर्त्य \textcolor{red}{षष्ठी चानादरे} (पा॰सू॰~२.३.३८) इत्यतः \textcolor{red}{षष्ठी} इत्यपकृष्य वाऽऽदरेऽपि भावलक्षणा षष्ठीति भावः।}\end{sloppypar}
\section[नेतव्यस्तत्र ते कालः]{नेतव्यस्तत्र ते कालः}
\centering\textcolor{blue}{अस्ति पञ्चवटीनाम्ना आश्रमो गौतमीतटे।\nopagebreak\\
नेतव्यस्तत्र ते कालः शेषो रघुकुलोद्वह॥}\nopagebreak\\
\raggedleft{–~अ॰रा॰~३.३.४८}\\
\begin{sloppypar}\hyphenrules{nohyphenation}\justifying\noindent\hspace{10mm} अत्र अगस्त्यः श्रीरामचन्द्रं दण्डक\-वासाय प्रेरयति यत् \textcolor{red}{रामभद्र तत्रैव ते भवतः कालो नेतव्यः}। आशङ्क्योऽयम्।\footnote{पूर्वपक्षोऽयम्।} \textcolor{red}{णीञ् प्रापणे} (धा॰पा॰~९०१) इत्यस्माद्धातोः \textcolor{red}{तयोरेव कृत्य\-क्त\-खलर्थाः} (पा॰सू॰~३.४.७०) इति सूत्र\-सहकारेण \textcolor{red}{तव्यत्तव्यानीयरः} (पा॰सू॰~३.१.९६) इत्यनेन तव्यत्प्रत्ययः स च कर्मणि। तेन कर्ताऽनुक्तः। तत्रैव तृतीया भाव्या।\footnote{\textcolor{red}{कर्तृ\-करणयोस्तृतीया} (पा॰सू॰~२.३.१८) इत्यनेन।} अत्र \textcolor{red}{कृत्यानां कर्तरि वा} (पा॰सू॰~२.३.७१) इत्यनेन वैकल्पिकी षष्ठी।\footnote{पक्षे \textcolor{red}{कर्तृ\-करणयोस्तृतीया} (पा॰सू॰~२.३.१८) इत्यनेन तृतीयाऽपि।} अथवा \textcolor{red}{ते} इत्यस्य \textcolor{red}{कालः} इत्यनेनान्वयः। अर्थात्त्वत्सम्बन्धी कालस्तत्रैव नेतव्यः।
\textcolor{red}{अकथितं च} (पा॰सू॰~१.४.५१) इत्यत्र परिगणित\-षोडश\-धातूनां मध्य उत्तरार्धे \textcolor{red}{तथा स्यान्नीहृकृष्वहाम्‌} (वै॰सि॰कौ॰~५३९) इत्यत्र \textcolor{red}{नी} इत्यस्य ग्रहणात्तथा च \textcolor{red}{ग्राममजां नयति} इतिवदत्रापि \textcolor{red}{तत्र} इत्यस्य कर्म\-सञ्ज्ञा भवेत्।\footnote{अयमपि पूर्वपक्षः।} न च कर्म\-सञ्ज्ञायामेवं त्रल्। \textcolor{red}{सप्तमी शौण्डैः} (पा॰सू॰~२.१.४०) इत्यत्र सप्तमी\-पदस्य ग्रहणात् \textcolor{red}{सप्तम्यास्त्रल्‌} (पा॰सू॰~५.३.१०) इत्यनेन \textcolor{red}{तत्र} इति सप्तम्यन्तस्य प्रयोगः।\footnote{तद्~ङि~\arrow \textcolor{red}{सप्तम्यास्त्रल्‌} (पा॰सू॰~५.३.१०)~\arrow तद्~ङि~त्रल्~\arrow \textcolor{red}{कृत्तद्धित\-समासाश्च} (पा॰सू॰~१.२.४६)~\arrow प्रातिपदिक\-सञ्ज्ञा~\arrow \textcolor{red}{सुपो धातु\-प्रातिपदिकयोः} (पा॰सू॰~२.४.७१)~\arrow तद्~त्र~\arrow \textcolor{red}{प्राग्दिशो विभक्तिः} (पा॰सू॰~५.३.१)~\arrow \textcolor{red}{त्यदादीनामः} (पा॰सू॰~७.२.१०२)~\arrow त~अ~त्र~\arrow \textcolor{red}{अतो गुणे} (पा॰सू॰~६.१.९७)~\arrow त~त्र~\arrow तत्र~\arrow \textcolor{red}{तद्धितश्चासर्व\-विभक्तिः} (पा॰सू॰~१.१.३८)~\arrow अव्यय\-सञ्ज्ञा~\arrow विभक्ति\-कार्यम्~\arrow तत्र~सुँ~\arrow \textcolor{red}{अव्ययादाप्सुपः} (पा॰सू॰~२.४.८२)~\arrow तत्र। पक्षे \textcolor{red}{तस्मिन्} इति। तद्~ङि~\arrow \textcolor{red}{ङसिङ्योः स्मात्स्मिनौ} (पा॰सू॰~७.१.१५)~\arrow तद्~स्मिन्~\arrow \textcolor{red}{त्यदादीनामः} (पा॰सू॰~७.२.१०२)~\arrow त~अ~स्मिन्~\arrow \textcolor{red}{अतो गुणे} (पा॰सू॰~६.१.९७)~\arrow त~स्मिन्~\arrow तस्मिन्।} सप्तमीं विना द्वितीयातस्त्रल् कथं सम्भवेत्। उच्यते। यदाऽपादानादिभिर्न विवक्षा स्यात्तदा कर्म\-सञ्ज्ञा भवति।\footnote{\textcolor{red}{अपादानादि\-विशेषैरविवक्षितं कारकं कर्म\-सञ्ज्ञं स्यात्‌} (वै॰सि॰कौ॰~५३९)।} अधुना त्वधिकरण\-विवक्षैव। आत्म\-समानाधिकरणस्य सकल\-जगदधिकरणस्य विविध\-दिव्याभरणस्य जगदाधारस्यापि रामस्याऽधार\-भूतत्वात् \textcolor{red}{नेतव्यस्तत्र ते कालः} इति पञ्चवटीं जगदाधारस्याऽधारभूतामिति साम्प्रदायिका ध्वनयितुं षष्ठीं युक्तिभिः साधयन्ति।\end{sloppypar}
\section[तयोः]{तयोः}
\centering\textcolor{blue}{अध्युवास सुखं रामो देवलोक इवापरः।\nopagebreak\\
कन्दमूलफलादीनि लक्ष्मणोऽनुदिनं तयोः॥\\
आनीय प्रददौ रामसेवातत्परमानसः।\nopagebreak\\
धनुर्बाणधरो नित्यं रात्रौ जागर्ति सर्वतः॥}\nopagebreak\\
\raggedleft{–~अ॰रा॰~३.४.१२-१३}\\
\begin{sloppypar}\hyphenrules{nohyphenation}\justifying\noindent\hspace{10mm} दण्डकारण्ये निवसतोः \textcolor{red}{तयोर्लक्ष्मणः कन्द\-मूल\-फलानि प्रददौ}। अत्र \textcolor{red}{दा}\-धातु\-योगे (\textcolor{red}{डुदाञ् दाने} धा॰पा॰~१०९१) चतुर्थ्यां \textcolor{red}{ताभ्याम्‌} इति भवेत्।\footnote{\textcolor{red}{चतुर्थी सम्प्रदाने} (पा॰सू॰~२.३.१३) इत्यनेन।} \textcolor{red}{तयोः} इति कथनं तु सम्प्रदाने सम्बन्ध\-विवक्षायाम्। यद्वा \textcolor{red}{तयोः पुरतः} इत्यध्याहारे सामान्य\-सम्बन्धे षष्ठी। यद्वा \textcolor{red}{निवसतोः तयोः} इति शत्रन्ते \textcolor{red}{यस्य च भावेन भाव\-लक्षणम्‌} (पा॰सू॰~२.३.३७) इति षष्ठीसप्तम्यौ।\footnote{\textcolor{red}{दूरान्तिकार्थैः षष्ठ्यन्यतरस्याम्‌} (पा॰सू॰~२.३.३४) इत्यतः \textcolor{red}{षष्ठी} इत्यनुवर्त्य \textcolor{red}{षष्ठी चानादरे} (पा॰सू॰~२.३.३८) इत्यतः \textcolor{red}{षष्ठी} इत्यपकृष्य वाऽऽदरेऽपि \textcolor{red}{यस्य च भावेन भाव\-लक्षणम्‌} (पा॰सू॰~२.३.३७) इत्यनेन भावलक्षणा षष्ठीति भावः।} अथवा \textcolor{red}{कन्द\-मूल\-फलानि} इत्यनेन साकमन्वित्य \textcolor{red}{तयोः एव कन्द\-मूल\-फलानि} इत्यर्थे सम्बन्धे षष्ठी। सीता च प्रकृतिः श्रीरामः पुरुषस्तयोः संयोगात्सृष्टिरिति सार्वभौमत्वादध्यात्म\-रामायणे च तत्र तत्र प्रतिपादितत्वाज्जन्य\-जनक\-भाव\-सम्बन्धे षष्ठ्यैश्वर्यं द्रढयितुम्।\end{sloppypar}
\section[मे]{मे}
\centering\textcolor{blue}{ज्ञानं विज्ञानसहितं भक्तिवैराग्यबृंहितम्।\nopagebreak\\
आचक्ष्व मे रघुश्रेष्ठ वक्ता नान्योऽस्ति भूतले॥}\nopagebreak\\
\raggedleft{–~अ॰रा॰~३.४.१८}\\
\begin{sloppypar}\hyphenrules{nohyphenation}\justifying\noindent\hspace{10mm} अत्र लक्ष्मणः श्रीरामं प्रार्थयते यत् \textcolor{red}{ज्ञानं विज्ञानञ्च मे आचक्ष्व}। अत्र \textcolor{red}{आचक्ष्व} इति प्रयुक्तम्। \textcolor{red}{चक्ष्‌}\-धातोः (\textcolor{red}{चक्षिङ् व्यक्तायां वाचि} धा॰पा॰~१०१७) लोड्\-लकारस्य मध्यम\-पुरुषस्यैक\-वचनान्तं रूपमिदम्। अयं हि धातुः कथनार्थः। कथनार्थक\-धातूनां योगे द्वितीया प्रसिद्धा। अत्र चतुर्थी तु \textcolor{red}{मे हिताय आचक्ष्व} इति तात्पर्य एवं \textcolor{red}{हित\-योगे च} (वा॰~२.३.१३) इति चतुर्थी। अथवा \textcolor{red}{मे मम} इति षष्ठ्यन्त\-प्रयोगः पश्चात् \textcolor{red}{परतः} इत्यध्याहार्यम्। अथवा \textcolor{red}{पृच्छतः} इति शत्रन्त\-प्रयोग एवं \textcolor{red}{यस्य च भावेन भाव\-लक्षणम्‌} (पा॰सू॰~२.३.३७) इत्यनेन षष्ठी।\footnote{\textcolor{red}{दूरान्तिकार्थैः षष्ठ्यन्यतरस्याम्‌} (पा॰सू॰~२.३.३४) इत्यतः \textcolor{red}{षष्ठी} इत्यनुवर्त्य \textcolor{red}{षष्ठी चानादरे} (पा॰सू॰~२.३.३८) इत्यतः \textcolor{red}{षष्ठी} इत्यपकृष्य वाऽऽदरेऽपि भावलक्षणा षष्ठीति भावः।}\end{sloppypar}
\section[ते]{ते}
\centering\textcolor{blue}{शृणु वक्ष्यामि ते वत्स गुह्याद्गुह्यतरं परम्।\nopagebreak\\
यद्विज्ञाय नरो जह्यात्सद्यो वैकल्पकं भ्रमम्॥}\nopagebreak\\
\raggedleft{–~अ॰रा॰~३.४.१९}\\
\begin{sloppypar}\hyphenrules{nohyphenation}\justifying\noindent\hspace{10mm} श्रीरामो लक्ष्मणं प्रति वक्ति यत् \textcolor{red}{ते गृह्याद्गुह्यतरं वक्ष्यामि}। \textcolor{red}{वक्ष्यामि} इति \textcolor{red}{ब्रू}\-धातु\-निष्पन्नो (\textcolor{red}{ब्रूञ् व्यक्तायां वाचि} धा॰पा॰~१०४४) लृडुत्तम\-पुरुषैक\-वचनान्तः।\footnote{\textcolor{red}{ब्रूञ् व्यक्तायां वाचि} (धा॰पा॰~१०४४)~\arrow ब्रू~\arrow \textcolor{red}{शेषात्कर्तरि परस्मैपदम्} (पा॰सू॰~१.३.७८)~\arrow \textcolor{red}{लृट् शेषे च} (पा॰सू॰~३.३.१३)~\arrow ब्रू~लृट्~\arrow ब्रू~मिप्~\arrow ब्रू~मि~\arrow \textcolor{red}{ब्रुवो वचिः} (पा॰सू॰~२.४.५३)~\arrow वच्~मि~\arrow \textcolor{red}{स्यतासी लृलुटोः} (पा॰सू॰~३.१.३३)~\arrow वच्~स्य~मि~\arrow \textcolor{red}{एकाच उपदेशेऽनुदात्तात्‌} (पा॰सू॰~७.२.१०)~\arrow इडागम\-निषेधः~\arrow वच्~स्य~मि~\arrow \textcolor{red}{चोः कुः} (पा॰सू॰~८.२.३०)~\arrow वक्~स्य~मि~\arrow \textcolor{red}{अतो दीर्घो यञि} (पा॰सू॰~७.३.१०१)~\arrow वक्~स्या~मि~\textcolor{red}{आदेश\-प्रत्यययोः} (पा॰सू॰~८.३.५९)~\arrow वक्~ष्या~मि~\arrow वक्ष्यामि।} अत्र \textcolor{red}{ब्रू}\-धातु\-योगे द्वितीया प्राप्तैव। चतुर्थी तु \textcolor{red}{त्वां बोधयितुं वक्ष्यामि} इत्यप्रयुज्यमान\-तुमुन्कर्मणि चतुर्थी।\footnote{\textcolor{red}{क्रियार्थोपपदस्य च कर्मणि स्थानिनः} (पा॰सू॰~२.३.१४) इत्यनेन।} \textcolor{red}{ते हितं वक्ष्यामि} इत्यध्याहारे \textcolor{red}{हित\-योगे च} (वा॰~२.३.१३) इत्यनेन वा। \textcolor{red}{पृष्टवतः ते वक्ष्यामि} इत्यध्याहारे भाव\-लक्षणा षष्ठी वा।\footnote{\textcolor{red}{दूरान्तिकार्थैः षष्ठ्यन्यतरस्याम्‌} (पा॰सू॰~२.३.३४) इत्यतः \textcolor{red}{षष्ठी} इत्यनुवर्त्य \textcolor{red}{षष्ठी चानादरे} (पा॰सू॰~२.३.३८) इत्यतः \textcolor{red}{षष्ठी} इत्यपकृष्य वाऽऽदरेऽपि \textcolor{red}{यस्य च भावेन भाव\-लक्षणम्‌} (पा॰सू॰~२.३.३७) इत्यनेन भावलक्षणा षष्ठीति भावः।} \textcolor{red}{ते समक्षं वक्ष्यामि} इत्यध्याहार\-बलेन षष्ठी वा।\end{sloppypar}
\section[मे]{मे}
\centering\textcolor{blue}{एतैर्विलक्षणो जीवः परमात्मा निरामयः।\nopagebreak\\
तस्य जीवस्य विज्ञाने साधनान्यपि मे शृणु॥}\nopagebreak\\
\raggedleft{–~अ॰रा॰~३.४.३०}\\
\begin{sloppypar}\hyphenrules{nohyphenation}\justifying\noindent\hspace{10mm} श्रीरामः श्रीलक्ष्मणमामन्त्रयति \textcolor{red}{मे शृणु} इति। अत्र विषयः श्रीराम\-मुखान्निर्गत्य लक्ष्मण\-कर्णौ प्रविशतीति शब्द\-विश्लेषे श्रीरामस्य ध्रुवत्वेनावधि\-भूतत्वात्तत्रापादान\-सञ्ज्ञा तन्मूलिका च पञ्चम्याशङ्क्यते। किन्त्वत्र शब्दानां विश्लेषोऽपेक्ष्यत एव नहि। वैयाकरणानां नये शब्दो ब्रह्म। \textcolor{red}{वाग्वै ब्रह्म} (बृ॰उ॰~१.३.२१) इति श्रुतेः। श्रीरामस्तु पर\-ब्रह्म शब्द\-ब्रह्म यस्य निःश्वास\-भूतम्। \textcolor{red}{अस्य महतो भूतस्य निःश्वसितमेतद्यदृग्वेदो यजुर्वेदः सामवेदोऽथर्वाङ्गिरस} (बृ॰उ॰~२.४.१०) इति श्रुतेः। \textcolor{red}{यस्य निश्श्वसितं वेदाः} (ऋ॰वे॰सं॰ सा॰भा॰ उ॰प्र॰~२) इति वचनाच्च। तर्हि तस्य शब्द\-ब्रह्मणः श्रीरामाद्विश्लेषः सम्भव एव नहि। ब्रह्मणो रामचन्द्रस्य सर्व\-व्यापित्वात्। लक्ष्मणश्च भगवदंशः। \textcolor{red}{शेषांशो लक्ष्मणः साक्षात्‌} इति वचनात्।\footnote{मूलं मृग्यम्।} भगवान् श्रीरामोंऽशी। अंशांशिनोरप्यभेदात्तेन सम्बन्ध\-विवक्षायामेव षष्ठ्यत्रत्य\-भाव\-गरिमाणं गरयति। यद्वा \textcolor{red}{मे सकाशात् शृणु} इत्यध्याहारेण सम्बन्ध\-षष्ठी। यद्वा \textcolor{red}{मे मह्यम्‌} इति चतुर्थी सा च \textcolor{red}{माम् तोषयितुम् माम् अनुकूलयितुम्‌} वेति तुमुन्कर्मणि चतुर्थी।\footnote{\textcolor{red}{क्रियार्थोपपदस्य च कर्मणि स्थानिनः} (पा॰सू॰~२.३.१४) इत्यनेन।} सावधान\-श्रोतुरुपरि वक्तुः प्रीतिर्वर्धत इति न केषामप्यविदितम्।\end{sloppypar}
\section[चक्षुष्मताम्]{चक्षुष्मताम्‌}
\centering\textcolor{blue}{चक्षुष्मतामपि तथा रात्रौ सम्यङ्न दृश्यते।\nopagebreak\\
पदं दीपसमेतानां दृश्यते सम्यगेव हि॥}\nopagebreak\\
\raggedleft{–~अ॰रा॰~३.४.४६}\\
\begin{sloppypar}\hyphenrules{nohyphenation}\justifying\noindent\hspace{10mm} अत्र \textcolor{red}{दृश्यते} इति हि भाव\-वाच्य\-प्रयोगः। अत्र कर्तुरनुक्तत्वात्तत्र तृतीयया भवितव्यम्।\footnote{\textcolor{red}{कर्तृ\-करणयोस्तृतीया} (पा॰सू॰~२.३.१८) इत्यनेन।} \textcolor{red}{चक्षुष्मद्भिर्न दृश्यते} इत्येव। षष्ठी तु \textcolor{red}{विवक्षाधीनानि कारकाणि भवन्ति}\footnote{मूलं मृग्यम्। यद्वा \textcolor{red}{कर्मादीनामविवक्षा शेषः} (भा॰पा॰सू॰~२.३.५०, २.३.५२, २.३.६७) इत्यस्य तात्पर्यमिदम्।} इत्यनेन षष्ठी सम्बन्ध\-विवक्षायाम्। यद्वा \textcolor{red}{दृष्ट्या} इत्यध्याहार्यम्। \textcolor{red}{चक्षुष्मतां दृष्ट्या न दृश्यते} इत्यवयवावयवि\-भावे सा।\end{sloppypar}
\section[राघवे]{राघवे}
\label{sec:raaghave_3_5_36}
\centering\textcolor{blue}{लक्ष्मणोऽपि गुहामध्यात्सीतामादाय राघवे।\nopagebreak\\
समर्प्य राक्षसान्दृष्ट्वा हतान्विस्मयमाययौ॥}\nopagebreak\\
\raggedleft{–~अ॰रा॰~३.५.३६}\\
\begin{sloppypar}\hyphenrules{nohyphenation}\justifying\noindent\hspace{10mm} खर\-दूषण\-वधानन्तरं लक्ष्मणो गुहातो निर्गत्य सीतां राघवाय समर्पयत्। तत्र \textcolor{red}{राघवे} इति सप्तमी\-प्रयोगस्तु शङ्कावहः। अत्र सम्प्रदानत्वस्यैव प्राप्तत्वात्। अत्रोच्यते। यत्र स्वकीयं वस्तु कस्मैचिद्दीयते तत्रैव सम्प्रदानम्।\footnote{\textcolor{red}{दानं चापुनर्ग्रहणाय स्व\-स्वत्व\-निवृत्तिपूर्वकं पर\-स्वत्वोत्पादनम्‌। अत एव “रजकस्य वस्त्रं ददाति” इत्यादौ न भवति} (त॰बो॰~५६९)।} अत्र तथाविध\-प्रसङ्गो नास्ति। यतो भगवाञ्छ्रीरामः सीतायाः प्राणाधारः। लौकिक\-व्यवहारेऽपि सीताया लता\-स्थानित्वाच्छ्रीरामस्य च तरु\-स्थानित्वात्। इमं पक्षमेव स्पष्टयितुमेकं राम\-कथा\-सम्बन्ध\-दृष्टान्तं प्रस्तूयते। चित्रकूटे निवासं कुर्वन् रममाणो रामचन्द्रः सीतया लक्ष्मणेन सह विनोद\-भङ्ग्या सीता\-त्याग\-गौरवं व्यञ्जनया वर्णयल्लँतामेकां सङ्केत्य समवोचत्प्रणतां वनितां यत् \textcolor{red}{प्रिये। इयं लता सम्मान\-योग्यतामर्हति या खलु नीरस\-तरुमपि विभूषयति सौन्दर्य\-लक्ष्म्या स्वकीय\-समालिङ्गन\-द्वारा}। अर्थाल्लतास्थानापन्ना त्वं तरु\-स्थानीयस्य मे श्रियं वर्धयसि। तत्क्षणं सीतोवाच \textcolor{red}{देव। नेत्थम्। लताश्रयस्तरुरेव धन्यो यः खलु निराधारां लतां सबहुमानं शाखा\-बाहुभिरालम्बते}। अर्थाल्लतास्थानीयाया म आधार\-भूतस्तरु\-स्थानीयो भवान्। उभयोर्वार्तां निशम्य निरीक्ष्य पावन\-प्रेम\-मण्डित\-परस्पर\-प्रशंसा\-निःस्पृहां सुमित्रा\-नन्दनो लक्ष्मणः प्राह यत् \textcolor{red}{प्रभो लता तरुर्वा नोभौ धन्यौ लक्ष्मण\-रूपः पथिक एव धन्यो यः सीता\-लता\-श्रीराम\-कल्प\-वृक्ष\-स्निग्ध\-छायामाश्रित्य समेधते}। अत्रत्य\-भाव\-सङ्ग्रह\-रूपमेकं शार्दूल\-विक्रीडितं प्रस्तुवन् धृष्टतामाधरामि यत्~–\end{sloppypar}
\centering\textcolor{red}{रामः प्राह लता प्रिये बहुमता सम्भूषयन्ती तरुं\nopagebreak\\
सीतोवाच लताश्रयस्तरुरयं धन्यो न चैषा लता।\nopagebreak\\
सौमित्रिर्निजगाद नाथ विटपो धन्यो न वल्ली तथा\nopagebreak\\
श्लाघ्योऽयं पथिको नितान्तमुभयोश्छायां श्रयन्मोदते॥}\\
\begin{sloppypar}\hyphenrules{nohyphenation}\justifying\noindent इत्थमनेन सीता\-राम\-लक्ष्मण\-वार्तालाप\-दृष्टान्त\-प्रस्तावेन सीताया आधारो रामः। अतः \textcolor{red}{आधारोऽधिकरणम्‌} (पा॰सू॰~१.४.४५) इत्यनेनाधिकरण\-सञ्ज्ञा। आधारश्च त्रिधौपश्लेषिको वैषयिकोऽभिव्यापकश्च। औपश्लेषिको नाम सामीप्य\-संयोगान्तरः सम्बन्धो यथा \textcolor{red}{कटे आस्ते} \textcolor{red}{गुरौ वसति}।\footnote{गुरोः समीपे वसतीत्यर्थ औपश्लेषिक आधारः।} वैषयिकः संयोग\-समवाय\-सम्बन्ध\-भिन्नो यथा \textcolor{red}{मोक्षे इच्छा}। अभिव्यापकः सर्वावयव\-व्यापको यथा \textcolor{red}{तिलेषु तैलम्‌} \textcolor{red}{सर्वस्मिन् आत्मा} \textcolor{red}{दधिनि सर्पिः} इत्यादि। इमे त्रयोऽप्याधार\-गुणाः श्रीरामेऽत्र। सीता\-विषयकमाधार\-त्रितयत्वं श्रीराम एव सङ्घटते। यथा \textcolor{red}{कटे आस्ते} इतिवत्सीता रामोपश्लिष्टा। सामीप्येनापि संयोगेनापि निरन्तरं सा भगवतोऽत्यन्त\-निकटा। एवमाह्लादिनी\-शक्ति\-रूपेण कृपा\-शक्ति\-रूपेण वा संयुक्ता। भावुकानां मते तु स्वयं मैथिली पीताम्बर\-रूपेण भगवतः श्रीविग्रहं निरन्तरमुपश्लिष्यति। सामीप्यञ्चात्राव्यवहितमेव यथा \textcolor{red}{इको यणचि} (पा॰सू॰~६.१.७७) इत्यत्र \textcolor{red}{अचि} औपश्लेषिक आधारः संसारे संसाधितः \textcolor{red}{संहितायाम्‌} (पा॰सू॰~६.१.७२) इति सूत्रे भगवता भाष्य\-कारेण।\footnote{\textcolor{red}{अयं योगः शक्योऽवक्तुम्। कथम्। अधिकरणं नाम त्रिःप्रकारम्। व्यापकम् औपश्लेषिकम् वैषयिकम् इति। शब्दस्य च शब्देन कोऽन्योऽभिसम्बन्धो भवितुमर्हत्यन्यदत उपश्लेषात्। इको यणचि। अच्युपश्लिष्टस्येति। तत्रान्तरेण संहिताग्रहणं संहितायामेव भविष्यति} (भा॰पा॰सू॰~६.१.७२)।} तथैव भगवती सीता शब्दार्थयोरिवाव्यवहित\-सामीप्यमञ्चति। वैषयिकश्चाप्याधारो राम एव। \textcolor{red}{आकाशे पक्षी} इतिवत्सीता रामे। अत्र 
संयोग\-समवायावान्तरेण निरूप्य\-निरूपक\-सम्बन्धतया श्रीसीताया वैषयिक आधारः श्रीराम एव। \textcolor{red}{मोक्षे इच्छा} इतिवत्। अभिव्यापकश्चाधारः श्रीराम एव कार्त्स्न्येन। \textcolor{red}{सर्वस्मिन् आत्मा} इत्यादिवत्।
सीता श्रीरामस्य प्रतिरामं रमते। एषु त्रिष्वप्याधारेषूपश्लेष एव मुख्यो भेद\-कल्पनया त्रयः। तथा कारक\-प्रकरणे भर्तृहरिः प्राह~–\end{sloppypar}
\centering\textcolor{red}{उपश्लेषस्य चाभेदस्तिलाकाशकटादिषु।}\nopagebreak\\
\raggedleft{–~वा॰प॰~३.७.१४९}\\
\begin{sloppypar}\hyphenrules{nohyphenation}\justifying\noindent हेलाराजोऽपि समर्थयति यद्वैषयिकाभि\-व्यापकयोरप्युप\-श्लेषस्यानुस्यूतत्वात्।\footnote{\textcolor{red}{उपश्लेष आधारस्याधेयेन सम्बन्धः। यद्वशावसावाधारः। तस्य त्रिष्वप्यधिकरणेष्वभेदः} (वा॰प॰ हे॰टी॰~३.७.१४९)।} तथैव सीतोपश्लेषस्य पक्षत्रयेऽपि सम्भवः। \textcolor{red}{अनन्या राघवेणाहं भास्करेण यथा प्रभा} (वा॰रा॰~५.२१.१५) इति प्रभा\-भास्करयोः समवाय\-सम्बन्धो यथा तथैव सीता\-रामयोः किन्तु \textcolor{red}{सीताऽप्यनुगता रामं शशिनं रोहिणी यथा} (वा॰रा॰~१.१.२८) इति वचनेनोपश्लेष एव सार्वत्रिकः। अतो \textcolor{red}{राघवे समर्प्य} इत्यत्र न्यास\-भूतां सीतां लक्ष्मणः पुना राघवमाधारं मत्वा समर्पयतीत्यर्थ\-बुबोधयिषया सप्तमीऽयं मधुर\-भाव\-बोधन\-पुरः\-सरं सर्वतो\-भावेन पाणिनीया।\end{sloppypar}
\section[निशायाम्]{निशायाम्‌}
\centering\textcolor{blue}{विचिन्त्यैवं निशायां स प्रभाते रथमास्थितः।\nopagebreak\\
रावणो मनसा कार्यमेकं निश्चित्य बुद्धिमान्॥}\nopagebreak\\
\raggedleft{–~अ॰रा॰~३.६.१}\\
\begin{sloppypar}\hyphenrules{nohyphenation}\justifying\noindent\hspace{10mm} \textcolor{red}{निशायाम्‌} इत्यत्राजन्त\-प्रयोगोऽपाणिनीय इव। प्रायशोऽयं हलन्त\-प्रयोगः। \textcolor{red}{निशायाम्‌} इति कथम्। यतो हि \textcolor{red}{अजाद्यतष्टाप्‌} (पा॰सू॰~४.१.४) इति सूत्रं ह्यजादीनामकारान्तस्य च टाप्प्रत्ययं करोत्ययं नाजादिर्न वाऽकारान्त इति चेत्। भागुरिर्हलन्त\-स्त्रीलिङ्गे पठितानां \textcolor{red}{टाप्‌}\-प्रत्ययं वदति। तथा च कारिका~–\end{sloppypar}
\centering\textcolor{red}{वष्टि भागुरिरल्लोपमवाप्योरुपसर्गयोः।\nopagebreak\\
आपं चैव हलन्तानां यथा वाचा दिशा निशा॥}\nopagebreak\\
\raggedleft{–~वै॰सि॰कौ॰ अव्ययप्रकरणान्ते कारिका २}\\
\begin{sloppypar}\hyphenrules{nohyphenation}\justifying\noindent न च भागुरि\-मतमपि त्वपाणिनीयमिति वाच्यम्। निशा\-शब्दस्य सूत्र उच्चारणात्पाणिनिनाऽपि भागुरि\-मतस्य कृत\-समादर\-दर्शनात्। यथा \textcolor{red}{विभाषा सेना\-सुराच्छाया\-शाला\-निशानाम्‌} (पा॰सू॰~२.४.२५) इत्यत्र \textcolor{red}{निशानाम्‌} षष्ठ्यन्त\-प्रयोगेणायं पाणिनीय एव। एवमेव \textcolor{red}{दिवा\-विभा\-निशा\-प्रभा\-भास्करान्तानन्तादि\-बहु\-नान्दीकिंलिपि\-लिबि\-बलि\-भक्ति\-कर्तृ\-चित्र\-क्षेत्र\-सङ्ख्या\-जङ्घा\-बाह्वहर्यत्तद्धनुररुष्षु} (पा॰सू॰~३.२.२१) \textcolor{red}{निशा\-प्रदोषाभ्यां च} (पा॰सू॰~४.३.१४) इत्यनयोरपि।\end{sloppypar}
\section[ब्रूहि मे]{ब्रूहि मे}
\centering\textcolor{blue}{ब्रूहि मे न हि गोप्यं चेत्करवाणि तव प्रियम्।\nopagebreak\\
न्याय्यं चेद्ब्रूहि राजेन्द्र वृजिनं मां स्पृशेन्नहि॥}\nopagebreak\\
\raggedleft{–~अ॰रा॰~३.६.६}\\
\begin{sloppypar}\hyphenrules{nohyphenation}\justifying\noindent\hspace{10mm} \textcolor{red}{माम्‌} इति वक्तव्ये \textcolor{red}{मे} इत्युक्तम्। कर्मणि सम्बन्ध\-विवक्षया षष्ठी। यद्वा \textcolor{red}{मे पुरतः} इत्यध्याहार\-बलेन षष्ठी। यद्वा \textcolor{red}{मे मह्यम्‌}। \textcolor{red}{मां ज्ञापयितुं ब्रूहि} इति तुमुन्कर्मणि चतुर्थी।\footnote{\textcolor{red}{क्रियार्थोपपदस्य च कर्मणि स्थानिनः} (पा॰सू॰~२.३.१४) इत्यनेन।}\end{sloppypar}
\section[आश्रमादपनेष्यसि]{आश्रमादपनेष्यसि}
\centering\textcolor{blue}{त्वं तु मायामृगो भूत्वा ह्याश्रमादपनेष्यसि।\nopagebreak\\
रामं च लक्ष्मणं चैव तदा सीतां हराम्यहम्॥}\nopagebreak\\
\raggedleft{–~अ॰रा॰~३.६.१३}\\
\begin{sloppypar}\hyphenrules{nohyphenation}\justifying\noindent\hspace{10mm} अत्र रावणो मारीचं प्रत्युपायं निर्दिशति। \textcolor{red}{अकथितं च} (पा॰सू॰~१.४.५१) इत्यनेनात्र कर्म\-सञ्ज्ञा। अकथित\-कर्मणां धातूनां क्रमे \textcolor{red}{नी}\-धातोरपि (\textcolor{red}{णीञ् प्रापणे} धा॰पा॰~९०१) पठितत्वात्।\footnote{\textcolor{red}{दुह्याच्पच्दण्ड्रुधिप्रच्छिचिब्रूशासुजिमथ्मुषाम्। कर्मयुक्स्यादकथितं तथा स्यान्नीहृकृष्वहाम्॥} (वै॰सि॰कौ॰~५३९)।} किन्त्वपादानादिभिरिविवक्षिते सतीयं व्यवस्था। अत्र त्वपादानस्य विवक्षाऽऽवश्यकी विश्लेषावधि\-भूतस्याश्रमस्य ध्रुवत्वं स्पष्टयितुम्।\end{sloppypar}
\section[देवायम्]{देवायम्‌}
\centering\textcolor{blue}{लक्ष्मणो राममाहेदं देवायं मृगरूपधृक्।\nopagebreak\\
मारीचोऽत्र न सन्देह एवंभूतो मृगः कुतः॥}\nopagebreak\\
\raggedleft{–~अ॰रा॰~३.७.९}\\
\begin{sloppypar}\hyphenrules{nohyphenation}\justifying\noindent\hspace{10mm} अत्र कपट\-कुरङ्ग\-वेष\-धारिणं मारीचमभिगन्तुकामं श्रीरामं लक्ष्मणो निर्दिशति \textcolor{red}{देव। अयं मृग\-रूप\-धृङ्मारीचः}। अत्र \textcolor{red}{देव} इति सम्बोधनम्। अत्र च \textcolor{red}{गुरोरनृतोऽनन्त्यस्याप्येकैकस्य प्राचाम्‌} (पा॰सू॰~८.२.८६) इत्यनेन प्लुतः। ततश्च \textcolor{red}{प्लुत\-प्रगृह्या अचि नित्यम्‌} (पा॰सू॰~६.१.१२५) इत्यनेन प्रकृति\-भावे कथं दीर्घ इति चेत्। \textcolor{red}{प्राचाम्‌} इति योग\-विभागेन \textcolor{red}{सर्वः प्लुतो विकल्प्यते} (वै॰सि॰कौ॰~९७) इत्यनेन प्लुतस्य वैकल्पिकत्वात्प्लुताभाव\-पक्षे दीर्घः।\footnote{\textcolor{red}{अकः सवर्णे दीर्घः} (पा॰सू॰~६.१.१०१) इत्यनेन।}\end{sloppypar}
\section[कालमेघसमद्युतिम्]{कालमेघसमद्युतिम्‌}
\centering\textcolor{blue}{स्वरूपं दर्शयामास महापर्वतसन्निभम्।\nopagebreak\\
दशास्यं विंशतिभुजं कालमेघसमद्युतिम्॥}\nopagebreak\\
\raggedleft{–~अ॰रा॰~३.७.५०}\\
\begin{sloppypar}\hyphenrules{nohyphenation}\justifying\noindent\hspace{10mm} सीता\-हरणाय यति\-वेषेणागतो
रावणः सीतां स्वरूपं दर्शयामास। अत्र \textcolor{red}{स्वरूपम्‌} इति नपुंसकलिङ्गं तस्य विशेषणं \textcolor{red}{काल\-मेघ\-सम\-द्युतिम्‌} इति नपुंसक\-लिङ्गस्य विशेषणं कथं यतो ह्यत्र \textcolor{red}{काल\-मेघेन समाना द्युतिर्यस्य तत्‌} इति विग्रहेऽन्यपदार्थे बहुव्रीहिः। अन्य\-पदार्थो हि \textcolor{red}{स्वरूपम्‌} इति। तस्य नपुंसक\-लिङ्गता सर्व\-विदिताऽतस्तद्विशेषणेनाप्यनिवार्या सा। व्युत्पत्ति\-वादेऽपि समान\-लिङ्गता समान\-वचनकत्वञ्च बहुशः प्रतिपादितम्। कथयन्ति च बुधाः~–\end{sloppypar}
\centering\textcolor{red}{या विशेष्येषु दृश्यन्ते लिङ्गसङ्ख्याविभक्तयः।\nopagebreak\\
प्रायस्ता एव कर्तव्याः समानार्थे विशेषणे॥}\footnote{मूलं मृग्यम्।}\\
\begin{sloppypar}\hyphenrules{nohyphenation}\justifying\noindent अतः सति नपुंसक\-लिङ्गे \textcolor{red}{कालमेघ\-समद्युतिम्‌} इत्यपाणिनीयतामावहतीव। यतो हि नपुंसक\-लिङ्गेऽमि विभक्तौ \textcolor{red}{स्वमोर्नपुंसकात्‌} (पा॰सू॰~७.१.२३) इति लुग्विधानात् \textcolor{red}{काल\-मेघ\-सम\-द्युति} इत्येव पाणिनीयम्। परं विमर्शे कृते विरोधः परिहर्तुं शक्यते। \textcolor{red}{अर्धर्चाः पुंसि च} (पा॰सू॰~२.४.३१) इति सूत्रेऽर्धर्चादि\-गणस्य पुल्लिँङ्गे नपुंसक\-लिङ्गे च विधानादाकृति\-गणत्वात् \textcolor{red}{स्वरूप}\-शब्दमर्धर्चादि\-गणे पठित्वा तत्र पुल्लिँङ्गता\-स्वीकारे तद्विशेषणेऽपि। तेन \textcolor{red}{हरिम्‌} इत्यादिवदत्रापि \textcolor{red}{अमि पूर्वः} (पा॰सू॰~६.१.१०७) इत्यनेन पूर्व\-रूपम्। यद्वा स्वरूपं वर्णयित्वा द्युतिं पृथग्वर्णयति। एवं चात्र कर्मधारयः \textcolor{red}{काल\-मेघ\-समा चासौ द्युतिश्च तां काल\-मेघ\-सम\-द्युतिम्‌} इति \textcolor{red}{मतिम्‌} इत्यादिवत्।\end{sloppypar}
\section[त्वरा]{त्वरा}
\centering\textcolor{blue}{त्वद्वाक्यसदृशं श्रुत्वा मां गच्छेति त्वराऽब्रवीत्।\nopagebreak\\
रुदन्ती सा मया प्रोक्ता देवि राक्षसभाषितम्।\nopagebreak\\
नेदं रामस्य वचनं स्वस्था भव शुचिस्मिते॥}\nopagebreak\\
\raggedleft{–~अ॰रा॰~३.८.११}\\
\begin{sloppypar}\hyphenrules{nohyphenation}\justifying\noindent\hspace{10mm} अत्र \textcolor{red}{त्वरया अब्रवीत्‌} इति वक्तव्ये \textcolor{red}{त्वरा अब्रवीत्‌} इति प्रयुक्तम्। यतो हि हलन्त\-स्त्री\-लिङ्गानां वैकल्पिक\-टाब्विधाने पाक्षिक\-हलन्तत्वे तृतीयायां \textcolor{red}{त्वर् टा} इति स्थिते टकारानुबन्ध\-लोपे \textcolor{red}{त्वरा} इति \textcolor{red}{गिरा} इव।\footnote{\textcolor{red}{ञित्वराँ सम्भ्रमे} (धा॰पा॰~७७५) इति धातोः \textcolor{red}{सम्पदादिभ्‍यः क्विप्} (वा॰~३.३.१०८) इत्यनेन स्त्रियां भावे क्विप्। तस्मात्तृतीयायां विभक्तौ \textcolor{red}{टा}\-प्रत्यये \textcolor{red}{त्वरा} इति भावः।} यद्वा \textcolor{red}{सह सुपा} (पा॰सू॰~२.१.४) इत्यनेन सुबन्तस्य धातुना सह समासे \textcolor{red}{त्वराब्रवीत्‌} इति।\footnote{अस्मिन् पक्षे \textcolor{red}{त्वर्‌}\-धातोः \textcolor{red}{घटादयः षितः} (धा॰पा॰ ग॰सू॰) इत्यनेन षित्त्वात् \textcolor{red}{षिद्भिदादिभ्योऽङ्} (पा॰सू॰~३.३.१०४) इत्यनेनाङि \textcolor{red}{अजाद्यतष्टाप्‌} (पा॰सू॰~४.१.४) इत्यनेन टापि \textcolor{red}{त्वरा} प्रातिपदिकम्। \textcolor{red}{सम्भ्रमस्त्वरा} (अ॰को॰~३.२.२६) इत्यमरः। तस्मात्तृतीयायां विभक्तौ~– त्वरा~टा~\arrow त्वरा~आ~\arrow \textcolor{red}{आङि चापः} (पा॰सू॰~७.३.१०५)~\arrow तवरे~आ~\arrow \textcolor{red}{एचोऽयवायावः} (पा॰सू॰~६.१.७८)~\arrow त्वरय्~आ~\arrow त्वरया। ततः \textcolor{red}{त्वरया अब्रवीत्‌} इति स्थिते सुपस्तिङा समासे \textcolor{red}{सुपो धातु\-प्रातिपदिकयोः} (पा॰सू॰~२.४.७१) इत्यनेन विभक्तेर्लुकि \textcolor{red}{त्वरा अब्रवीत्‌} इति जाते \textcolor{red}{अकः सवर्णे दीर्घः} (पा॰सू॰~६.१.१०१) इत्यनेन सवर्ण\-दीर्घे \textcolor{red}{त्वराऽब्रवीत्‌} इति।} समासस्य षड्विधत्वं प्रसिद्धमेव। यथा सिद्धान्त\-कौमुद्याम्~– \textcolor{red}{किञ्च}\end{sloppypar}
\centering\textcolor{red}{सुपां सुपा तिङा नाम्ना धातुनाऽथ तिङां तिङा।\nopagebreak\\
सुबन्तेनेति विज्ञेयः समासः षड्विधो बुधैः॥}\\
\begin{sloppypar}\hyphenrules{nohyphenation}\justifying\noindent \textcolor{red}{सुपां सुपा राजपुरुषः। तिङा पर्यभूषत्। नाम्ना कुम्भकारः। धातुना कटप्रूः। अजस्रम्। तिङां तिङा पिबतखादता। खादतमोदता। तिङां सुपा कृन्त विचक्षणेति यस्यां क्रियायां सा कृन्त\-विचक्षणा।} (वै॰सि॰कौ॰ सर्वसमासशेष\-प्रकरणे)। तस्मात्तृतीयान्त\-\textcolor{red}{त्वरया}\-इति\-शब्देन समासो विभक्ति\-लोपश्च।
\end{sloppypar}
\section[जायेति सीतेति]{जायेति सीतेति}
\label{sec:jaayeti_siiteti}
\centering\textcolor{blue}{निर्ममो निरहङ्कारोऽप्यखण्डानन्दरूपवान्।\nopagebreak\\
मम जायेति सीतेति विललापातिदुःखितः॥}\nopagebreak\\
\raggedleft{–~अ॰रा॰~३.८.२०}\\
\begin{sloppypar}\hyphenrules{nohyphenation}\justifying\noindent\hspace{10mm} अत्र सीता\-हरण\-सञ्जात\-शोकः श्रीरामोऽखण्डानन्द\-रूपवान् सन् \textcolor{red}{हे मम जाये इति हे सीते इति} व्यलपत्। अत्र \textcolor{red}{जायेति सीतेति} च प्रयोग\-द्वयं विभाव्यम्। \textcolor{red}{जाये सीते} च द्वावपि शब्दौ सम्बोधनान्तौ। एवं प्रातिपदिकात्सौ विभक्तौ सम्बोधनतया \textcolor{red}{संबुद्धौ च} (पा॰सू॰~७.३.१०६) इत्येकार्ये \textcolor{red}{एङ्ह्रस्वात्सम्बुद्धेः} (पा॰सू॰~६.१.६९) इत्यनेन सुलोपे \textcolor{red}{जाये सीते} इति सिध्यतः। तत \textcolor{red}{इति}\-घटकेकारेण सह सन्धावयादेशे\footnote{\textcolor{red}{एचोऽयवायावः} (पा॰सू॰~६.१.७८) इत्यनेन।} यलोपे\footnote{\textcolor{red}{लोपः शाकल्यस्य} (पा॰सू॰~८.३.१९) इत्यनेन।} \textcolor{red}{जाय इति सीत इति} च। \textcolor{red}{जायेति सीतेति} अपाणिनीयमिव। उच्यते। अत्र च \textcolor{red}{इति}\-शब्दार्थकः \textcolor{red}{ति}\-शब्दः।\footnote{\setcounter{dummy}{\value{footnote}}\addtocounter{dummy}{-1}\refstepcounter{dummy}\label{fn:ti}\textcolor{red}{ति}\-शब्स्यार्थो वाचस्पत्ये~– \textcolor{red}{इति + वेदे पृषो॰। इतिशब्दार्थे। “सहोवाचास्तीह प्रायश्चित्तिरित्यस्तीति का ति पिता ते वेदेति शत॰~ब्रा॰~११.६.१.३। का प्रायश्चित्तिस्ति इति प्रश्नः}। शतपथब्राह्मणे त्रिषु मन्त्रेषु \textcolor{red}{ति}\-शब्द \textcolor{red}{इति}\-अर्थे प्रयुक्तः – \textcolor{red}{का ति पिता ते वेदेति} (श॰ब्रा॰~११.६.१.३) \textcolor{red}{का ति पितैव ते वेदेति} (श॰ब्रा॰~११.६.१.४) \textcolor{red}{का ति पितैव ते वेदेति} (श॰ब्रा॰~११.६.१.५)।} \textcolor{red}{जाये ति सीते ति}। यथा \textcolor{red}{ईदूदेद्द्विवचनं प्रगृह्यम्‌} (पा॰सू॰~१.१.११) इत्यनेनेदन्तोदन्तैदन्तानां द्विवचनानां प्रगृह्य\-सञ्ज्ञा क्रियते। सत्यां च तस्यां प्रकृति\-भावो भवति। \textcolor{red}{हरी एतौ विष्णू इमौ गङ्गे अमू} (वै॰सि॰कौ॰~१००) इत्यादिवत्। तर्हि \textcolor{red}{मणीवोष्ट्रस्य लम्बेते प्रियौ वत्सतरौ मम} (म॰भा॰~१२.१७६.११) इति महाभारत\-श्लोक\-वाक्ये \textcolor{red}{मणी इव उष्ट्रस्य} इति स्थितेऽत्रापीदन्त\-द्विवचनतया प्रगृह्यत्वात्प्रकृति\-भावे कथं \textcolor{red}{मणीव} इत्यत्र दीर्घः। अतः कौमुद्यामाक्षिप्य समाहितं \textcolor{red}{‘मणी वोष्ट्रस्य लम्बेते प्रियौ वत्सतरौ मम’ इत्यत्र त्विवार्थे वशब्दो वाशब्दो वा बोद्ध्यः} (वै॰सि॰कौ॰~१००)। तत्रत्या तत्त्वबोधिनी च – \textcolor{red}{वशब्द इत्यादि। ‘वं प्रचेतसि जानीयादिवार्थे च तदव्ययम्’ (मे॰को॰~व॰~१) इति मेदिनी। ‘व वा यथा तथैवैवं साम्ये’ (अ॰को॰~३.४.९) इत्यमरः। ‘कादम्ब\-खण्डित\-दलानि व पङ्कजानि’ इत्यादि\-प्रयोग\-दर्शनाच्चेति भावः} (त॰बो॰~१००)। स एव पन्था अत्राप्यनुयातव्यः। \textcolor{red}{इति}\-अर्थे \textcolor{red}{ति}\-शब्दो बोद्धव्यः। एतेन \textcolor{red}{हे कृष्ण हे यादव हे सखे ति} (भ॰गी॰~११.४१) इति भगवद्गीतोक्तमप्यपाणिनीयं समाहितम्।\footnote{इदं प्रथमं समाधानम्। द्वितीयं पश्चाद्वक्ष्यन्ति।} यद्वा \textcolor{red}{जाया इति सीता इति} शुद्ध\-प्रथमान्तमेव रूप\-द्वयं ततो गुणे \textcolor{red}{जायेति सीतेति}। अथवा \textcolor{red}{न मु ने} (पा॰सू॰~८.२.३) इत्यत्र \textcolor{red}{न}\-ग्रहणमेव \textcolor{red}{पूर्वत्रासिद्धम्‌} (पा॰सू॰~८.२.१) इति सूत्रस्यासिद्ध\-करणानित्यत्वं ज्ञापयति क्वाचित्कम्। अन्यथा \textcolor{red}{अमुना} इत्येव ब्रूयात्। तत्र हि \textcolor{red}{न। मु ने} इति योग\-विभागः। \textcolor{red}{पूर्वत्रासिद्धं न} इति प्रथमोऽर्थः। \textcolor{red}{ने मु न असिद्धः} इति द्वितीयोऽर्थः। योग\-विभाग\-फलं क्वचित्क्वचित् \textcolor{red}{पूर्वत्रासिद्धम्‌} (पा॰सू॰~८.२.१) इति सूत्रस्य प्रसराभावः। तेन \textcolor{red}{न चीरमस्याः प्रविधीयतेति} (वा॰रा॰~२.३७.३४) इह \textcolor{red}{प्रविधीयते इति} इत्यवस्थायां \textcolor{red}{एचोऽयवायावः} (पा॰सू॰~६.१.७८) इत्यनेनायादेशे \textcolor{red}{प्रविधीयतय् इति} इति जाते \textcolor{red}{लोपः शाकल्यस्य} (पा॰सू॰~८.३.१९) इत्यनेन यकार\-लोपे \textcolor{red}{आद्गुणः} (पा॰सू॰~६.१.८७) इत्यनेन गुणे प्राप्ते यलोपस्य त्रिपादीत्वाद्गुण\-शास्त्रस्य सपाद\-सप्ताध्यायीत्वात् \textcolor{red}{पूर्वत्रासिद्धम्‌} (पा॰सू॰~८.२.१) इत्यनेन लोप\-कार्यस्यासिद्धत्वे गुणाभावे \textcolor{red}{प्रविधीयतेति} इति कथम्। ततो \textcolor{red}{न मु ने} इत्यनेनैव \textcolor{red}{पूर्वत्रासिद्धम्‌} (पा॰सू॰~८.२.१) इत्यस्याप्रवृत्तौ लोप\-कार्ये सिद्धे गुणे \textcolor{red}{प्रविधीयतेति}।\footnote{\textcolor{red}{प्रविधीयतेतीत्यत्र सन्धिस्तु ‘न मु ने’ इत्यत्र योगविभागेन क्वचित्त्रिपाद्या असिद्धत्वाभावज्ञापनादार्षत्वाद्वा} (वा॰रा॰ ति॰टी॰~२.३७.३४)।} एवमेव \textcolor{red}{हे कृष्ण हे यादव हे सखेति} (भ॰गी॰~११.४१) इत्यत्रापि कर्तव्ये लोपकार्ये सिद्धे गुणः।\footnote{इदं द्वितीयं समाधानम्।} एवमेव \textcolor{red}{प्रियः प्रियायार्हसि देव सोढुम्‌} (भ॰गी॰~११.४४) इत्यत्रापि \textcolor{red}{प्रियायास् अर्हसि} इति स्थिते \textcolor{red}{ससजुषो रुः} (पा॰सू॰~८.२.६६) इत्यनेन सकारस्य रुत्वे \textcolor{red}{भोभगोअघोअपूर्वस्य योऽशि} (पा॰सू॰~८.३.१७) इत्यनेन यत्वे \textcolor{red}{लोपः शाकल्यस्य} (पा॰सू॰~८.३.१९) इत्यनेन लोपे दीर्घे प्राप्ते पुनस्त्रि\-पादित्वाल्लोप\-कार्यस्यासिद्धत्वे प्राप्ते दीर्घे च सङ्कटापन्ने \textcolor{red}{न मु ने} (पा॰सू॰~८.२.३) इत्यनेन \textcolor{red}{पूर्वत्रासिद्धम्‌} (पा॰सू॰~८.२.१) इति सूत्रे निराकृते लोप\-कार्ये सिद्धे दीर्घे \textcolor{red}{प्रियायार्हसि} इति साधु। अनया मीमांसया \textcolor{red}{प्रियाय अर्हसि} इति चतुर्थ्यन्तं श्रीरामानुज\-कथनमपास्तम्।\footnote{\textcolor{red}{यस्मात्त्वं सर्वस्य पिता पूज्यतमो गुरुश्च कारुण्यादिगुणैश्च सर्वाधिकोऽसि तस्मात्त्वामीशमीड्यं प्रणम्य प्रणिधाय च कायं प्रसादये। यथा कृतापराधस्य अपि पुत्रस्य यथा च सख्युः प्रणामपूर्वकं प्रार्थितः पिता सखा वा प्रसीदति तथा त्वं परमकारुणिकः प्रियः प्रियाय मे सर्वं सोढुमर्हसि} (भ॰गी॰ रा॰भा॰~११.४४)।} एवमेव \textcolor{red}{मम जाये इति सीते इति} इत्युभयत्राप्येकारस्यायादेशे यकारस्य शाकल\-लोपे\footnote{शाकलश्चासौ लोपश्चेति शाकललोपः। शकलस्य गोत्रापत्यं पुमान् शाकल्यः। गर्गादित्वात् \textcolor{red}{यञ्‌}\-प्रत्ययः। शाकल्यस्यायं शाकलः। \textcolor{red}{लोपः शाकल्यस्य} (पा॰सू॰~८.३.१९) इत्यनेन विहितो लोपः शाकलो लोपः शाकललोपो वा। शकल~\arrow \textcolor{red}{गर्गादिभ्यो यञ्‌} (पा॰सू॰~४.१.१०५)~\arrow शकल~यञ्~\arrow शकल~य~\arrow \textcolor{red}{यचि भम्‌} (पा॰सू॰~१.४.१८)~\arrow भसञ्ज्ञा~\arrow \textcolor{red}{तद्धितेष्वचामादेः} (पा॰सू॰~७.२.११७)~\arrow शाकल~य~\arrow\textcolor{red}{यस्येति च} (पा॰सू॰~६.४.१४८)~\arrow शाकल्~य~\arrow शाकल्य~\arrow \textcolor{red}{तस्येदम्‌} (पा॰सू॰~४.३.१२०)~\arrow शाकल्य~अण्~\arrow शाकल्य~अ~\arrow \textcolor{red}{यचि भम्‌} (पा॰सू॰~१.४.१८)~\arrow भसञ्ज्ञा~\arrow \textcolor{red}{यस्येति च} (पा॰सू॰~६.४.१४८)~\arrow शाकल्य्~अ~\arrow \textcolor{red}{आपत्यस्य च तद्धितेऽनाति} (पा॰सू॰~६.४.१५१)~\arrow शाकल्~अ~\arrow शाकल~\arrow विभक्तिकार्यम्~\arrow शाकलः। ष्फाणौ प्रत्ययौ \textcolor{red}{शाकल्य}\-शब्दाद्भवत इति महाभाष्ये स्पष्टम्। यथा~– \textcolor{red}{कण्वात्तु शकलः पूर्वः कतादुत्तर इष्यते। पूर्वोत्तरौ तदन्तादी ष्फाणौ तत्र प्रयोजनम्॥} (भा॰पा॰सू॰~४.१.१८)। तत्रैव \textcolor{red}{शाकल्यायनी} शाकल्यस्य च्छात्राः \textcolor{red}{शाकलाः} इत्युदाहृतौ।} गुणे प्राप्ते \textcolor{red}{पूर्वत्रासिद्धम्‌} (पा॰सू॰~८.२.१) इति सूत्रेण यकार\-लोपस्यासिद्धत्वे प्राप्ते \textcolor{red}{न मु ने} (पा॰सू॰~८.२.३) इत्यनेन सूत्रेऽस्मिन्निराकृते गुण उभावपि प्रयोगौ निरवद्यौ। \textcolor{red}{न मु ने} इत्यत्र नकारयोगविभागं समर्थयन्ते वैयाकरण\-सिद्धान्त\-कौमुदी\-तत्त्व\-बोधिनी\-टीकाकारा ज्ञानेन्द्र\-सरस्वती\-महाभागाः। तथा च तत्रत्या तत्त्वबोधिनी~– \textcolor{red}{ननु “अधुना” इतिवत् “अमुना” इत्योवोच्यतां किमनेनासिद्धत्वनिषेधेनेति चेत्। अत्राहुः। “न मु ने” (पा॰सू॰~८.२.३) इत्युक्तिः “न” इति योगविभागार्था। तेन रामः रामेभ्य इत्यादि सिध्यति। अन्यथा हि रोरसिद्ध\-तयोकारस्येत्सञ्ज्ञा\-लोपौ कथं स्याताम्। न चानुनासिक\-निर्देश\-सामर्थ्यादित्सञ्ज्ञा\-लोपौ प्रति रुत्वं नासिद्धमिति वाच्यम्। तरुमूलं देवरुहीत्यादौ “हशि च” (पा॰सू॰~६.१.११४) इत्यस्य व्यावृत्तये “अतो रोरप्लुतादप्लुते” (पा॰सू॰~६.१.११३) इत्यत्रानु\-नासिकस्यैव निर्देशेन तत्रैव चरितार्थत्वात्। एवं च स्थानिवत्सूत्रस्यापि प्रवृतौ पदत्वाद्विसर्गो लभ्यते। “प्रत्ययः” (पा॰सू॰~३.१.१) “परश्च” (पा॰सू॰~३.१.२) इत्यादि\-निर्देशाश्चेह लिङ्गमिति दिक्} (त॰बो॰~४३९)।\end{sloppypar}
\section[जटायो]{जटायो}
\centering\textcolor{blue}{जटायो ब्रूहि मे भार्या केन नीता शुभानना।\nopagebreak\\
मत्कार्यार्थं हतोऽसि त्वमतो मे प्रियबान्धवः॥}\nopagebreak\\
\raggedleft{–~अ॰रा॰~३.८.३१}\\
\centering\textcolor{blue}{इत्युक्त्वा राघवः प्राह जटायो गच्छ मत्पदम्।\nopagebreak\\
मत्सारूप्यं भजस्वाद्य सर्वलोकस्य पश्यतः॥}\nopagebreak\\
\raggedleft{–~अ॰रा॰~३.८.४०}\\
\begin{sloppypar}\hyphenrules{nohyphenation}\justifying\noindent\hspace{10mm} \textcolor{red}{जटायामायुर्यस्य} इति विग्रहे व्यधिकरण\-बहुव्रीहौ विभक्ति\-कार्ये सम्बोधने \textcolor{red}{हे जटायुः} इति चेत्।\footnote{\textcolor{red}{जटायुस्} प्राति\-पदिकात्सम्बुद्धौ सौ विभक्तौ जटायुस्~सुँ~\arrow जटायुस्~स्~\arrow \textcolor{red}{हल्ङ्याब्भ्यो दीर्घात्सुतिस्यपृक्तं हल्} (पा॰सू॰~६.१.६८)~\arrow जटायुस्~\arrow \textcolor{red}{ससजुषो रुः} (पा॰सू॰~८.२.६६)~\arrow जटायुरुँ~\arrow जटायुर्~\arrow \textcolor{red}{खरवसानयोर्विसर्जनीयः} (पा॰सू॰~८.३.१५)~\arrow जटायुः।} \textcolor{red}{आयुस्‌} \textcolor{red}{आयु} शब्दश्च द्वावप्यायुष्य\-वाचकौ।\footnote{उकारान्त\-\textcolor{red}{आयु}\-शब्दो यथा~– \textcolor{red}{यथा श्रुतं मया पूर्वं वायुना जगदायुना} (वायु॰पु॰~१.५४.२) \textcolor{red}{स्थाणोः पश्चिमदिग्भागे वायुना जगदायुना} (वाम॰पु॰ स॰मा॰~२५.३९)। तत्त्व\-बोधिन्यामुणादि\-प्रकरणे \textcolor{red}{छन्दसीणः} (प॰उ॰~१.२) इति सूत्रे ज्ञानेन्द्र\-सरस्वती\-महाभागास्तु~– \textcolor{red}{उणनुवर्तते। एति गच्छतीत्यायुः। “मा न आयौ” इति। आयुशब्दो मनुष्यपर्यायेषु वैदिकनिघण्टौ पठितः। अत एव “त्वाम॑ग्ने प्रथ॒ममा॒युमा॒यवे॑” (ऋ॰वे॰सं॰~१.३१.११) “मा न॑स्तो॒के तन॑ये॒ मा न॑ आ॒यौ” (ऋ॰वे॰सं॰~१.११४.८) इत्यादिमन्त्रेषु वेदभाष्ये तथैव व्याख्यातम्। अर्वाचीनास्तु “छन्दसीणः” इति सूत्रं बहुल\-वचनाद्भाषायामपि प्रवर्तत इति स्वीकृत्य “आयुर्जीवितकालो ना” (अ॰को॰~२.८.१२०) इत्यमरग्रन्थे ‘आयु’\-शब्दमुकारान्तं व्याचख्युः। ननु “एतेर्णिच्च” (प॰उ॰~२.११८) इत्युस्प्रत्यये सकारान्तो वक्ष्यमाण ‘आयुः’\-शब्दस्तु लोक\-वेदयोर्नर्विवाद एव। अत एव ‘जटा आयुरस्य’ इति विग्रहे “गृध्रं हत्वा जटायुषम्” (वा॰रा॰~१.१.५२) इति रामायणप्रयोगः “यदि त्रिलोकी गणनापरा स्यात्तस्याः समाप्तिर्यदि नायुषः स्यात्” (नै॰च॰~३.४०) इति श्रीहर्ष\-प्रयोगश्च सङ्गच्छते। तथा च “आयुर्जीवितकालो ना” (अ॰को॰~२.८.१२०) इत्यत्रायुःशब्दः सकारान्त इत्येव व्याख्यायतां किमुकारान्ताभ्युप\-गमेनेति चेत्। अत्राहुः~– सकारान्त आयुःशब्दो नपुसंक इति तस्य पुँल्लिङ्गता नेत्याशयेन तथोक्तमिति। अन्ये तु “छन्दसीणः” (प॰उ॰~१.२) इति सूत्रस्य भाषायां प्रवृत्त्यभावे “मा वधीष्ट जटायुं माम्” (भ॰का॰~६.४१) इति भट्टिप्रयोगः “तटीं विन्ध्यस्याऽद्रेरभजत जटायोः प्रथमजः” इति विन्ध्यवर्णने अभिनन्दोक्त\-प्रयोगश्च न सङ्गच्छेतेत्याहुः। वस्तुतस्तु “जटां याति प्राप्नोतीति जटायुः”। मृगय्वादित्वात्कुः। आयातीत्यायुः। एवं च “जटायुषा जटायुं च विद्यादायुं तथायुषा” इति द्विरूपकोशः “वायुना जगदायुना” इति वर्णविवेकश्च सुसाध इति दिक्}। ऋग्वेद\-संहितायाम् \textcolor{red}{अग्ने॒ जर॑स्व स्वप॒त्य आयु॑न्यू॒र्जा पि॑न्वस्व॒ समिषो॑ दिदीहि नः} (ऋ॰वे॰सं॰~३.३.७) \textcolor{red}{व॒त्सं न पूर्व॒ आयु॑नि जा॒तं रि॑हन्ति मा॒तरः॑} (ऋ॰वे॰सं॰~सा॰भा॰~९.१००.१) इति मन्त्रयोरुकारान्तः \textcolor{red}{आयु}\-शब्द इति केचित्। अत्र सायणाः \textcolor{red}{आयुनि आयुषि} (ऋ॰वे॰सं॰~३.३.७) \textcolor{red}{आयुनि वयसि} (ऋ॰वे॰सं॰~सा॰भा॰~९.१००.१)।} अत्र \textcolor{red}{आयु}\-शब्द एव जटा\-शब्देन सह समस्तः। ततो प्रथमा\-विभक्तौ \textcolor{red}{एक\-वचनं सम्बुद्धिः} (पा॰सू॰~२.३.४९) इत्यनेन सम्बुद्धि\-सञ्ज्ञायां \textcolor{red}{ह्रस्वस्य गुणः} (पा॰सू॰~७.३.१०८) इत्यनेन गुणे सति \textcolor{red}{एङ्ह्रस्वात्सम्बुद्धेः} (पा॰सू॰~६.१.६९) इत्यनेन सोर्लोपे \textcolor{red}{जटायो} इति।\footnote{सकारान्त\-\textcolor{red}{जटायुस्‌}\-शब्दः सर्वविदितः। उकारान्तो \textcolor{red}{जटायु}\-शब्दोऽपि दृश्यते। यथा \textcolor{red}{जटायोस्तु वधं श्रुत्वा दुःखितः सोऽरुणात्मजः} (वा॰रा॰~५.३५.६५) इति वाल्मीकि\-रामायणे सुन्दर\-काण्डे पाठभेदः। अत्र \textcolor{red}{जटायुषो वधं श्रुत्वा} इति गोविन्द\-राज\-सम्मत\-पाठः। अरण्यकाण्डे तु \textcolor{red}{तलेनाभिजघानार्तो जटायुं क्रोधमूर्छितः} (वा॰रा॰~३.५१.३७) इत्यत्र गोविन्द\-राजा अपि~– \textcolor{red}{जटायुं जटायुरित्युकारान्तोऽप्यस्ति} (वा॰रा॰ भू॰टी॰~३.५१.३७)। अन्यच्च~– \textcolor{red}{जटायोः कीर्तनं चक्रू रामकार्ये मृतं पुरा} (आ॰रा॰~१.८.१११) \textcolor{red}{जघान तेन दुष्टात्मा जटायुं धर्मचारिणम्} (नर॰पु॰~४९.९६) \textcolor{red}{माऽऽवधिष्ठा जटायुं माम्} (भ॰का॰~६.४१) \textcolor{red}{जटायोः मोक्षप्राप्तिः} (सा॰द॰~६.६९) \textcolor{red}{ब्रह्मास्त्रे चापि दत्ते पथि पितृसुहृदं वीक्ष्य भूयो जटायुम्} (ना॰~९.३४.७) इत्यादिषु।}\end{sloppypar}
\section[वन्दितो मे]{वन्दितो मे}
\centering\textcolor{blue}{अष्टावक्रः पुनः प्राह वन्दितो मे दयापरः।\nopagebreak\\
शापस्यान्तं च मे प्राह तपसा द्योतितप्रभः॥}\nopagebreak\\
\raggedleft{–~अ॰रा॰~३.९.१८}\\
\begin{sloppypar}\hyphenrules{nohyphenation}\justifying\noindent\hspace{10mm} अत्र \textcolor{red}{मया वन्दितः} इत्येवोचितं यतो हि कर्मणि \textcolor{red}{क्त}\-प्रत्ययः कर्ता चानुक्तोऽतोऽनुक्ते कर्तरि तृतीया \textcolor{red}{मया} इति।\footnote{\textcolor{red}{कर्तृ\-करणयोस्तृतीया} (पा॰सू॰~२.३.१८) इत्यनेन।}
अत्र कर्मणि सम्बन्ध\-विवक्षायां षष्ठी। अथवा \textcolor{red}{क्तस्य च वर्तमाने} (पा॰सू॰~२.३.६७) इत्यनेन षष्ठी।\footnote{\textcolor{red}{मति\-बुद्धि\-पूजार्थेभ्यश्च} (पा॰सू॰~३.२.१८८) इत्यनेन पूजार्थे वर्तमाने क्तः। मे वन्दितः सन् वन्द्यमानो वा प्राह इत्यर्थो न त्वहं तमवन्दे पश्चात्स प्राहेति। एवमेव \textcolor{red}{त्वमव्ययः शाश्वतधर्मगोप्ता सनातनस्त्वं पुरुषो मतो मे} (भ॰गी॰~११.१८) इति गीतावचने \textcolor{red}{मतः} इत्यत्र \textcolor{red}{मति\-बुद्धि\-पूजार्थेभ्यश्च} (पा॰सू॰~३.२.१८८) इत्यनेन पूजार्थे वर्तमाने क्तः \textcolor{red}{क्तस्य च वर्तमाने} (पा॰सू॰~२.३.६७) इत्यनेन च षष्ठी।} यद्वा \textcolor{red}{मे} इत्यस्य \textcolor{red}{प्राह} इत्यनेनान्वयः। तुमुन्कर्मणि चतुर्थी।\footnote{\textcolor{red}{क्रियार्थोपपदस्य च कर्मणि स्थानिनः} (पा॰सू॰~२.३.१४) इत्यनेन। \textcolor{red}{मां बोधयितुमुद्धर्तुं वा प्राह} इत्यर्थः। एवमेवोत्तरार्धे \textcolor{red}{शापस्यान्तं च मे प्राह} इत्यत्र बोध्यम्।}\end{sloppypar}
\section[देवराजानम्]{देवराजानम्‌}
\centering\textcolor{blue}{कदाचिद्देवराजानमभ्याद्रवमहं रुषा।\nopagebreak\\
सोऽपि वज्रेण मां राम शिरोदेशेऽभ्यताडयत्॥}\nopagebreak\\
\raggedleft{–~अ॰रा॰~३.९.२१}\\
\begin{sloppypar}\hyphenrules{nohyphenation}\justifying\noindent\hspace{10mm} \textcolor{red}{देवानां राजेति देवराजः} इति विग्रहे षष्ठी\-तत्पुरुषे \textcolor{red}{राजाऽहस्सखिभ्यष्टच्‌} (पा॰सू॰~५.४.९१) इत्यनेन टच्प्रत्यये \textcolor{red}{देवराजम्‌} इत्युचितम्। \textcolor{red}{देव\-राजानम्‌} इति च समासान्त\-प्रत्ययानामनित्यत्व\-स्वीकारे सति। अथवा \textcolor{red}{राजनं राट्‌}। भावे क्विप्।\footnote{\textcolor{red}{सम्पदादिभ्‍यः क्विप्‌} (वा॰~३.३.१०८) इत्यनेन।} \textcolor{red}{देवानां राडिति देवराट्‌}। \textcolor{red}{देवराजाऽऽसमन्तादनिति} इति विग्रहे \textcolor{red}{देवराजानः} तं \textcolor{red}{देवराजानम्‌} इति पाणिनीयमेव।\footnote{विस्तृत\-व्याख्यानं \pageref{sec:devarajanam}तमे पृष्ठे \ref{sec:devarajanam} \nameref{sec:devarajanam} इति प्रयोगस्य विमर्शे पश्यन्तु।}\end{sloppypar}
\section[ते]{ते}
\centering\textcolor{blue}{भक्त्या त्वत्पादकमले भक्तिमार्गविशारदा।\nopagebreak\\
तां प्रयाहि महाभाग सर्वं ते कथयिष्यति॥}\nopagebreak\\
\raggedleft{–~अ॰रा॰~३.१०.२}\\
\begin{sloppypar}\hyphenrules{nohyphenation}\justifying\noindent\hspace{10mm} अत्र \textcolor{red}{त्वां बोधयितुं कथयिष्यति} इति तुमुन्कर्मणि चतुर्थी।\footnote{\textcolor{red}{क्रियार्थोपपदस्य च कर्मणि स्थानिनः} (वा॰~२.३.१४) इत्यनेन।} यद्वा \textcolor{red}{ते पुरतः कथयिष्यति} इति सम्बन्ध\-षष्ठी।\end{sloppypar}
\vspace{2mm}
\centering ॥ इत्यरण्यकाण्डीयप्रयोगाणां विमर्शः ॥\nopagebreak\\
\vspace{4mm}
\pdfbookmark[2]{किष्किन्धाकाण्डम्‌}{Chap1Part2Kanda4}
\phantomsection
\addtocontents{toc}{\protect\setcounter{tocdepth}{2}}\addtocontents{toc}{\protect\setcounter{tocdepth}{2}}
\addcontentsline{toc}{subsection}{किष्किन्धाकाण्डीयप्रयोगाणां विमर्शः}
\addtocontents{toc}{\protect\setcounter{tocdepth}{0}}
\centering ॥ अथ किष्किन्धाकाण्डीयप्रयोगाणां विमर्शः ॥\nopagebreak\\
\section[भाति मनो मम]{भाति मनो मम}
\centering\textcolor{blue}{युवां त्रैलोक्यकर्ताराविति भाति मनो मम।\nopagebreak\\
युवां प्रधानपुरुषौ जगद्धेतू जगन्मयौ॥}\nopagebreak\\
\raggedleft{–~अ॰रा॰~४.१.१३}\\
\begin{sloppypar}\hyphenrules{nohyphenation}\justifying\noindent\hspace{10mm} अत्र पम्पातीर ऋष्यमूक\-गिरेः पार्श्वे गच्छन्तौ राम\-लक्ष्मणौ विलोक्य भीतेन सुग्रीवेण प्रेषितो विप्र\-वेष\-धारी हनुमान् तौ पृच्छन् कथयति यत् \textcolor{red}{युवाम् ईश्वरौ इति मे मनो भाति}। \textcolor{red}{मनः} इति द्वितीयान्तम्। अर्थात् \textcolor{red}{मे मनः प्रतीत्थं प्रतीयते}।\footnote{\textcolor{red}{प्रति} इत्यध्याहार्यमिति भावः। ततः \textcolor{red}{अभितः\-परितः\-समया\-निकषा\-हा\-प्रति\-योगेऽपि} (वा॰~२.३.२) इत्यनेन द्वितीया।} यद्वा \textcolor{red}{मनः} इति प्रथमान्तमेव। \textcolor{red}{भाति} इत्यस्य प्रत्यायनमर्थो न तु भानम्। प्रतीत्युपसर्ग\-संयोजनेन धातोः प्रत्यायन\-रूपस्यार्थस्य स्वीकारात्।\footnote{यथा भारते विराट\-पर्वणि विराटोऽर्जुनं प्रति~– \textcolor{red}{नैवंविधाः क्लीबरूपा भवन्ति कथञ्चनेति प्रतिभाति मे मनः} (म॰भा॰~४.११.७)।} न च प्रतीत्यत्र न विलोक्यत इति चेत्। \textcolor{red}{विनाऽपि प्रत्ययं पूर्वोत्तर\-पद\-लोपो वक्तव्यः} (वा॰~५.३.८३) इति लुप्तत्वात्।\end{sloppypar}
\section[समागम्य रघूत्तमम्]{समागम्य रघूत्तमम्‌}
\centering\textcolor{blue}{ततोऽतिहर्षात्सुग्रीवः समागम्य रघूत्तमम्।\nopagebreak\\
वृक्षशाखां स्वयं छित्त्वा विष्टराय ददौ मुदा॥}\nopagebreak\\
\raggedleft{–~अ॰रा॰~४.१.३२}\\
\begin{sloppypar}\hyphenrules{nohyphenation}\justifying\noindent\hspace{10mm} \textcolor{red}{समागम्य} इत्यस्य मित्रता\-रूपेऽर्थे स्वीकृतेऽलिङ्गने वा तृतीयया भवितव्यं\footnote{यथा \textcolor{red}{राजभिस्तत्र वार्ष्णेयः समागच्छद्यथावयः} (म॰भा॰~५.९१.७) \textcolor{red}{त्वमात्मनस्तुल्यममुं वृणीष्व रत्नं समागच्छतु काञ्चनेन} (र॰वं॰~६.७९) \textcolor{red}{प्रहृष्टो भव विप्रर्षे समागच्छ मया सह} (म॰भा॰~१३.५०.८५) \textcolor{red}{सा त्वं मया समागच्छ} (म॰भा॰~३.३०७.२८) इत्यादिषु। अन्यान्युदाहरणानि चारुदेव\-शास्त्रि\-कृतायाम् \textcolor{red}{उपसर्गार्थ\-चन्द्रिकायाम्} द्वितीय\-खण्डे \textcolor{red}{समा–गम्} इत्यत्र पश्यन्तु।} किन्त्वागमन\-रूपार्थ\-स्वीकारे नापत्तिः।\footnote{यथा भारते \textcolor{red}{क्व नु नाम वयं सर्वाः कालेनाल्पेन तं नरम्। समागच्छेम यो नस्तद्रूपमापादयेत्पुनः॥} (म॰भा॰~१.२१७.१३) इत्यत्र।} यद्वा \textcolor{red}{रघूत्तमं प्रति समागम्य पश्चात्तेन मैत्रीं चकार} इति द्वितीया सङ्गता।\footnote{\textcolor{red}{प्रति} इत्यध्याहार्यमिति भावः। ततः \textcolor{red}{अभितः\-परितः\-समया\-निकषा\-हा\-प्रति\-योगेऽपि} (वा॰~२.३.२) इत्यनेन द्वितीया।}\end{sloppypar}
\section[हा सीतेति]{हा सीतेति (४.१.४१)}
\centering\textcolor{blue}{विमुच्य रामस्तद्दृष्ट्वा हा सीतेति मुहुर्मुहुः।\nopagebreak\\
हृदि निक्षिप्य तत्सर्वं रुरोद प्राकृतो यथा॥}\nopagebreak\\
\raggedleft{–~अ॰रा॰~४.१.४१}\\
\begin{sloppypar}\hyphenrules{nohyphenation}\justifying\noindent\hspace{10mm} अत्रापि \textcolor{red}{हा सीते इति} अस्यामवस्थायां गुणोऽसङ्गतः।\footnote{\textcolor{red}{हा सीते इति} इति स्थिते \textcolor{red}{एचोऽयवायावः} (पा॰सू॰~६.१.७८) इत्यनेन \textcolor{red}{हा सीतय् इति} इति जाते \textcolor{red}{लोपः शाकल्यस्य} (पा॰सू॰~८.३.१९) इत्यनेन यकारलोपे \textcolor{red}{हा सीत इति} जाते यकार\-लोप\-विधायि\-सूत्रस्य त्रिपादीस्थत्वात् \textcolor{red}{आद्गुणः} (पा॰सू॰~६.१.८७) इति गुण\-विधायि\-सूत्रस्य सपाद\-सप्ताध्यायी\-स्थत्वाद्यलोपेऽसिद्धे गुणोऽसङ्गत इति भावः।} किन्तु \textcolor{red}{इति}\-शब्द\-ग्राहक\-\textcolor{red}{ति}\-शब्द\-स्वीकारे न दोषः।\footnote{\pageref{fn:ti}तमे पृष्ठे \ref{fn:ti}तमीं पादटिप्पणीं पश्यन्तु।
} अथवा शुद्ध\-प्रथमान्त\-स्वीकारे गुणः सङ्गत एव। अथवा \textcolor{red}{अवङ् स्फोटायनस्य} (पा॰सू॰~६.१.१२३) इति व्यवस्थित\-विभाषया \textcolor{red}{क्वचिदन्यदेव} इति नियमेनात्राचि परेऽपि पूर्व\-रूपम्। अथवा \textcolor{red}{पृषोदरादीनि यथोपदिष्टम्‌} (पा॰सू॰~६.३.१०९) इति सूत्रेण तस्मिन् गणे पठित्वाऽत्र पूर्व\-रूपं \textcolor{red}{हा सीतेति}। अथवा पूर्वोक्त\-रीत्या \textcolor{red}{न मु ने} (पा॰सू॰~८.२.३) इत्यस्य नकारस्य मुभावादतिरिक्ते स्थलेऽपि प्रवृत्तत्वाद्यकारलोपे सिद्धे गुणे च \textcolor{red}{हा सीतेति} इति पाणिनीयम्।\footnote{\pageref{sec:jaayeti_siiteti}तमे पृष्ठे \ref{sec:jaayeti_siiteti} \nameref{sec:jaayeti_siiteti} इति प्रयोगस्य विमर्शमपि पश्यन्तु।}\end{sloppypar}
\section[ते]{ते}
\centering\textcolor{blue}{आश्वास्य राघवं भ्राता लक्ष्मणो वाक्यमब्रवीत्।\nopagebreak\\
अचिरेणैव ते राम प्राप्यते जानकी शुभा।\nopagebreak\\
वानरेन्द्रसहायेन हत्वा रावणमाहवे॥}\nopagebreak\\
\raggedleft{–~अ॰रा॰~४.१.४२}\\
\begin{sloppypar}\hyphenrules{nohyphenation}\justifying\noindent\hspace{10mm} अत्र \textcolor{red}{प्राप्यते} इति कर्म\-वाच्ये प्रयोगेऽनुक्तत्वात्कर्तरि तृतीया प्रयोक्तव्या।\footnote{\textcolor{red}{कर्तृ\-करणयोस्तृतीया} (पा॰सू॰~२.३.१८) इत्यनेन।} षष्ठी\-प्रयोगस्तु कर्मणि सम्बन्ध\-विवक्षया। यद्वा \textcolor{red}{ते} इत्यस्य \textcolor{red}{जानकी} इति शब्देन सह अन्वयः। अर्थात् \textcolor{red}{ते तव जानकी} इति सामान्य\-सम्बन्धे षष्ठी।\end{sloppypar}
\section[तव]{तव}
\centering\textcolor{blue}{सुग्रीवोऽप्याह हे राम प्रतिज्ञां करवाणि ते।\nopagebreak\\
समरे रावणं हत्वा तव दास्यामि जानकीम्॥}\nopagebreak\\
\raggedleft{–~अ॰रा॰~४.१.४३}\\
\begin{sloppypar}\hyphenrules{nohyphenation}\justifying\noindent\hspace{10mm} अत्र \textcolor{red}{दा}\-धातु\-प्रयोगे (\textcolor{red}{डुदाञ् दाने} धा॰पा॰~१०९१) \textcolor{red}{तुभ्यम्‌} इति चतुर्थी प्रयोक्तव्या।\footnote{\textcolor{red}{चतुर्थी सम्प्रदाने} (पा॰सू॰~२.३.१३) इत्यनेन।} षष्ठी\-प्रयोगस्तु सम्प्रदानेऽपि सम्बन्ध\-विवक्षायाम्। अथवा \textcolor{red}{तव सहचारिणीं जानकीं दास्यामि} इति दाम्पत्य\-भावे सम्बन्धे षष्ठी।\footnote{\textcolor{red}{सहचारिणीम्} इत्यध्याहार्यमिति भावः।}\end{sloppypar}
\section[रघुनायके]{रघुनायके}
\centering\textcolor{blue}{स्वोदन्तं कथयामास प्रणयाद्रघुनायके।\nopagebreak\\
सखे श्रृणु ममोदन्तं वालिना यत्कृतं पुरा॥}\nopagebreak\\
\raggedleft{–~अ॰रा॰~४.१.४६}\\
\begin{sloppypar}\hyphenrules{nohyphenation}\justifying\noindent\hspace{10mm} अत्र \textcolor{red}{कथयामास} इति \textcolor{red}{कथ}\-धातु\-प्रयोगे (\textcolor{red}{कथँ वाक्य\-प्रबन्धने} धा॰पा॰~१८५१) \textcolor{red}{रघुनायकम्‌} इति द्वितीयया भवितव्यमासीत्।\footnote{\textcolor{red}{अकथितं च} (पा॰सू॰~१.४.५१) इत्यनेन कर्म\-सञ्ज्ञायां \textcolor{red}{कर्मणि द्वितीया} (पा॰सू॰~२.३.२) इत्यनेन।} \textcolor{red}{रघुनायके} इति सप्तमी त्वपाणिनीयेव। अत्र वैषयिक आधारे सप्तमी। विषयता च प्रतिपाल्यता\-रूपा। यद्वा \textcolor{red}{रघुनायके शृण्वति कथयामास} इत्यध्याहारे \textcolor{red}{यस्य च भावेन भाव\-लक्षणम्‌} (पा॰सू॰~२.३.३७) इत्यनेन सप्तमी। यद्वा \textcolor{red}{रघुनायके साधुः सुग्रीवः कथयामास} इत्यध्याहारे \textcolor{red}{साध्व\-साधु\-प्रयोगे च} (वा॰~२.३.३६) इत्यनेन सप्तमी।\end{sloppypar}
\section[तदादि]{तदादि}
\centering\textcolor{blue}{तदादि मम भार्यां स स्वयं भुङ्क्ते विमूढधीः।\nopagebreak\\
अतो दुःखेन सन्तप्तो हृतदारो हृताश्रयः॥}\nopagebreak\\
\raggedleft{–~अ॰रा॰~४.१.५७}\\
\begin{sloppypar}\hyphenrules{nohyphenation}\justifying\noindent\hspace{10mm} अत्र प्रभृत्यर्थतया पञ्चमी युक्ता\footnote{\textcolor{red}{‘अपादाने पञ्चमी’ इति सूत्रे ‘कार्तिक्याः प्रभृति’ इति भाष्य\-प्रयोगात् प्रभृत्यर्थयोगे पञ्चमी} (वै॰सि॰कौ॰~५९५)।} किन्तु \textcolor{red}{आरभ्य\-योगे द्वितीया च}\footnote{मूलं मृग्यम्।} इति द्वितीयैवात्र।\footnote{\textcolor{red}{अत्राऽरभ्येति क्रियापेक्षया कर्मत्वविवक्षायां द्वितीयैव। ‘उपपदविभक्तेः कारकविभक्तिर्बलीयसी’ इत्युक्तेः। यथा ‘सूर्योदयमारभ्य आऽस्तमयाज्जपति’ इत्यादौ} (बा॰म॰~५९५)।}\end{sloppypar}
\section[कपीन्द्राय]{कपीन्द्राय}
\centering\textcolor{blue}{किन्तु लोका वदिष्यन्ति मामेवं रघुनन्दनः।\nopagebreak\\
कृतवान्किं कपीन्द्राय सख्यं कृत्वाऽग्निसाक्षिकम्॥}\nopagebreak\\
\raggedleft{–~अ॰रा॰~४.२.३}\\
\begin{sloppypar}\hyphenrules{nohyphenation}\justifying\noindent\hspace{10mm}
अत्र \textcolor{red}{कपीन्द्रस्य} इति षष्ठ्यापाततः किन्तु विचारे कृते \textcolor{red}{हिताय} इत्यध्याहारे \textcolor{red}{हित\-योगे च} (वा॰~२.३.१३) इत्यनेन चतुर्थी। यद्वा \textcolor{red}{तादर्थ्ये चतुर्थी वाच्या} (वा॰~२.३.१३) इति वार्त्तिकेन चतुर्थी।\end{sloppypar}
\section[मे]{मे}
\centering\textcolor{blue}{एवं मे प्रत्ययं कृत्वा सत्यवादिन् रघूत्तम।\nopagebreak\\
उपेक्षसे किमर्थं मां शरणागतवत्सल॥}\nopagebreak\\
\raggedleft{–~अ॰रा॰~४.२.१२}\\
\begin{sloppypar}\hyphenrules{nohyphenation}\justifying\noindent\hspace{10mm} \textcolor{red}{मे} इति षष्ठी\-प्रयोगोऽसङ्गतः। यतो हि रामः सुग्रीवे विश्वासमुत्पादितवान् वालि\-हनन\-प्रत्ययतः। वैषयिकाधारतया \textcolor{red}{मयि} इति प्रयोक्तव्यम्। परं \textcolor{red}{मे उपरि मे हृदये} वेत्यध्याहारे न दोषः। अथवा सुग्रीव आत्मीयतया कथयति \textcolor{red}{हे मे रघूत्तम} इति मत्सम्बन्धि\-रघूत्तम। सम्बन्धश्च रक्ष्य\-रक्षक\-भाव\-रूपः। तत्र षष्ठी।\end{sloppypar}
\section[त्वां शपे]{त्वां शपे}
\centering\textcolor{blue}{गत्वाऽह्वय पुनः शत्रुं हतं द्रक्ष्यसि वालिनम्।\nopagebreak\\
रामोऽहं त्वां शपे भ्रातर्हनिष्यामि रिपुं क्षणात्॥}\nopagebreak\\
\raggedleft{–~अ॰रा॰~४.२.१५}\\
\begin{sloppypar}\hyphenrules{nohyphenation}\justifying\noindent\hspace{10mm} अत्र \textcolor{red}{त्वाम्‌} इति प्रयोगः सन्दिग्धः। शापस्थ\-विषयतया सुग्रीवो वैषयिक आधारः। तद्वाचक\-युष्मच्छब्द एक\-वचने सप्तम्युचिता।\footnote{\textcolor{red}{सप्तम्यधिकरणे च} (पा॰सू॰~२.३.३६) इत्यनेन।} किन्तु मीमांस्यमाने प्रतिरध्याहार्यः। \textcolor{red}{त्वां प्रति शपे} इति योजने \textcolor{red}{अभितः\-परितः\-समया\-निकषा\-हा\-प्रति\-योगेऽपि} (वा॰~२.३.२) इत्यनेन द्वितीया।\end{sloppypar}
\section[ते]{ते}
\centering\textcolor{blue}{इदानीमेव ते भग्नः पुनरायाति सत्वरः।\nopagebreak\\
सहायो बलवांस्तस्य कश्चिन्नूनं समागतः॥}\nopagebreak\\
\raggedleft{–~अ॰रा॰~४.२.२१}\\
\centering\textcolor{blue}{इति निश्चित्य तौ यातौ निश्चितं श्रृणु मद्वचः।\nopagebreak\\
इदानीमेव ते भग्नः कथं पुनरुपागतः॥}\nopagebreak\\
\raggedleft{–~अ॰रा॰~४.२.३०}\\
\begin{sloppypar}\hyphenrules{nohyphenation}\justifying\noindent\hspace{10mm} \textcolor{red}{ते} इत्यनुचितो यतो हि \textcolor{red}{भग्नः} इति कर्मणि क्तान्तस्ततः कर्तर्यनुक्ते तृतीया\footnote{\textcolor{red}{कर्तृ\-करणयोस्तृतीया} (पा॰सू॰~२.३.१८) इत्यनेन।} किन्तु \textcolor{red}{ते मुष्टिना भग्नः} इत्यध्याहारेऽङ्गाङ्गि\-भाव\-सम्बन्धे षष्ठी।\end{sloppypar}
\section[एकामपि]{एकामपि}
\centering\textcolor{blue}{अधर्मकारिणं हत्वा सद्धर्मं पालयाम्यहम्।\nopagebreak\\
दुहिता भगिनी भ्रातुर्भार्या चैव तथा स्नुषा॥\\
समा यो रमते तासामेकामपि विमूढधीः।\nopagebreak\\
पातकी स तु विज्ञेयः स वध्यो राजभिः सदा॥}\nopagebreak\\
\raggedleft{–~अ॰रा॰~४.२.६०.६१}\\
\begin{sloppypar}\hyphenrules{nohyphenation}\justifying\noindent\hspace{10mm} रमणं सार्धं वाऽऽधारे वा भवति। यदि चेत्सार्धं तदा \textcolor{red}{एकया सह} इति \textcolor{red}{सह\-युक्तेऽप्रधाने} (पा॰सू॰~२.३.१९) इत्यनेन तृतीया। यदि चेदौपश्लेषिक आधारे रमणं चेत् \textcolor{red}{एकस्याम्‌} इति सप्तमी। किन्त्वत्र \textcolor{red}{एकाम्‌} इति द्वितीया तु \textcolor{red}{एकां गृहीत्वा} इत्यध्याहारेण ग्रहण\-कर्मणि।\end{sloppypar}
\section[चतुर्द्वारकपाटादीन्]{चतुर्द्वारकपाटादीन्‌}
\centering\textcolor{blue}{चतुर्द्वारकपाटादीन् बद्ध्वा रक्षामहे पुरीम्।\nopagebreak\\
वानराणां तु राजानमङ्गदं कुरु भामिनि॥}\nopagebreak\\
\raggedleft{–~अ॰रा॰~४.३.३}\\
\begin{sloppypar}\hyphenrules{nohyphenation}\justifying\noindent\hspace{10mm} अत्र श्रीरामभद्रेण वालिनि हते तारां प्रति भीत\-वानराः कथयन्ति यत् \textcolor{red}{चतुर्द्वारकपाटादीन् रुन्त्स्व}। अत्र बहुव्रीहिः। \textcolor{red}{चतुर्द्वार\-कपाटान्यादौ येषां तान्‌} इति विग्रहेऽत्र पदार्थस्य सामान्यतया \textcolor{red}{सामान्ये नपुंसकम्‌} (वा॰~२.४.३०) इत्यनेन नपुंसके \textcolor{red}{चतुर्द्वार\-कपाटादीनि} इत्येव पाणिनीयम्।\footnote{चतुर्द्वार\-कपाटादि~शस्~\arrow \textcolor{red}{जश्शसोः शिः} (पा॰सू॰~७.१.२०)~\arrow चतुर्द्वार\-कपाटादि~शि~\arrow \textcolor{red}{शि सर्वनामस्थानम्} (पा॰सू॰~१.१.४२)~\arrow सर्वनाम\-स्थान\-सञ्ज्ञा~\arrow चतुर्द्वार\-कपाटादि~इ~\arrow \textcolor{red}{इकोऽचि विभक्तौ} (पा॰सू॰~७.१.७३)~\arrow \textcolor{red}{आद्यन्तौ टकितौ} (पा॰सू॰~१.१.४६)~\arrow चतुर्द्वार\-कपाटादि~नुँम्~इ~\arrow चतुर्द्वार\-कपाटादि~न्~इ~\arrow \textcolor{red}{सर्वनामस्थाने चासम्बुद्धौ} (पा॰सू॰~६.४.८)~\arrow चतुर्द्वार\-कपाटादी~न्~इ~\arrow चतुर्द्वार\-कपाटादीनि।} किन्तु विशेषण\-वस्तु\-योजनान्न दोषः। यथा \textcolor{red}{चतुर्द्वार\-कपाटादीन् उपकरण\-विशेषान्‌} इत्यध्याहारे न दोषः।\footnote{चतुर्द्वार\-कपाटादि~शस्~\arrow \textcolor{red}{प्रथमयोः पूर्वसवर्णः} (पा॰सू॰~६.१.१०२)~\arrow चतुर्द्वार\-कपाटादी~शस्~\arrow \textcolor{red}{तस्माच्छसो नः पुंसि} (पा॰सू॰~६.१.१०३)~\arrow चतुर्द्वार\-कपाटादी~न्~\arrow चतुर्द्वार\-कपाटादीन्।}\end{sloppypar}
\section[पतिना]{पतिना}
\label{sec:patina}
\centering\textcolor{blue}{किमङ्गदेन राज्येन नगरेण धनेन वा।\nopagebreak\\
इदानीमेव निधनं यास्यामि पतिना सह॥}\nopagebreak\\
\raggedleft{–~अ॰रा॰~४.३.५}\\
\begin{sloppypar}\hyphenrules{nohyphenation}\justifying\noindent\hspace{10mm} अत्र रामेण वालिनि हते तारा विलपन्ती प्राह यत् \textcolor{red}{पतिना सह निधनं यास्यामि}। अत्र \textcolor{red}{पतिना} इति प्रयोगोऽपाणिनीय इव। यतो हि \textcolor{red}{पति}\-शब्दस्य तृतीयैक\-वचने \textcolor{red}{घि}\-सञ्ज्ञायाम् \textcolor{red}{आङो नाऽस्त्रियाम्‌} (पा॰सू॰~७.३.१२०) इत्यनेनाङो नादेशे \textcolor{red}{पतिना} इति सम्भवति किन्तु नादेशः सत्यां घिसञ्ज्ञायां 
सम्भवः।
सा च \textcolor{red}{घि}\-सञ्ज्ञा पति\-शब्दस्य समास एव 
सम्भवा।
\textcolor{red}{पतिः समास एव} (पा॰सू॰~१.४.८) इति सूत्रात्। अतः समासाभावे घि\-सञ्ज्ञाया असम्भवात्तदभावे च नादेशस्यासम्भवे \textcolor{red}{पतिना} इत्यनुपपन्न एवेति चेत्। अत्र नास्ति शुद्ध\-पति\-शब्दोऽपि तु \textcolor{red}{पतिरिवाऽचरति पतयति} इति विग्रह आचारे \textcolor{red}{क्विप्}।\footnote{पति~\arrow \textcolor{red}{सर्वप्राति\-पदिकेभ्य आचारे क्विब्वा वक्तव्यः} (वा॰~३.१.११)~\arrow पति~क्विँप्~\arrow पति~व्~\arrow \textcolor{red}{वेरपृक्तस्य} (पा॰सू॰~६.१.६७)~\arrow पति~\arrow \textcolor{red}{सनाद्यन्ता धातवः} (पा॰सू॰~३.१.३२)~\arrow धातुसञ्ज्ञा~\arrow \textcolor{red}{शेषात्कर्तरि परस्मैपदम्} (पा॰सू॰~१.३.७८)~\arrow \textcolor{red}{वर्तमाने लट्} (पा॰सू॰~३.२.१२३)~\arrow पति~लट्~\arrow पति~तिप्~\arrow पति~ति~\arrow \textcolor{red}{कर्तरि शप्‌} (पा॰सू॰~३.१.६८)~\arrow पति~शप्~ति~\arrow पति~अ~ति~\arrow \textcolor{red}{सार्वधातुकार्ध\-धातुकयोः} (पा॰सू॰~७.३.८४)~\arrow पते~अ~ति~\arrow \textcolor{red}{एचोऽयवायावः} (पा॰सू॰~६.१.७८)~\arrow पतय्~अ~ति~\arrow पतयति।} पुनः \textcolor{red}{पतयतीति पतिः}। कर्तरि \textcolor{red}{क्विप्}।\footnote{पति~\arrow पूर्ववद्धातु\-सञ्ज्ञा~\arrow \textcolor{red}{क्विप् च} (पा॰सू॰~३.२.७६)~\arrow पति~क्विँप्~\arrow पति~व्~\arrow \textcolor{red}{वेरपृक्तस्य} (पा॰सू॰~६.१.६७)~\arrow पति~\arrow विभक्तिकार्यम्~\arrow पतिः।} सर्वापहारि\-लोपे \textcolor{red}{गौण\-मुख्ययोर्मुख्ये कार्य\-सम्प्रत्ययः} (प॰शे॰~१५) इति परिभाषा\-बलेनात्र गौणे पति\-शब्दे \textcolor{red}{पतिः समास एव} (पा॰सू॰~१.४.८) इति सूत्रस्याप्रवृत्तौ \textcolor{red}{शेषो घ्यसखि} (पा॰सू॰~१.४.७) इति सूत्रेण घि\-सञ्ज्ञायाम् \textcolor{red}{आङो नाऽस्त्रियाम्‌} (पा॰सू॰~७.३.१२०) इत्यनेन नादेशे \textcolor{red}{पतिना} इति सिद्धं पाणिनीयमेव।\footnote{अपि च तत्त्वबोधिनीकाराः \textcolor{red}{पतिः समास एव} (पा॰सू॰~१.४.८) इति सूत्रे – \textcolor{red}{अथ कथं ‘सीतायाः पतये नमः’ (रा॰र॰स्तो॰~२७) इति ‘नष्टे मृते प्रव्रजिते क्लीबे च पतिते पतौ’ (प॰स्मृ॰~४.३०) इति पराशरश्च। अत्राहुः। पतिरित्याख्यातः पतिः ‘तत्करोति तदाचष्टे’ (धा॰पा॰ ग॰सू॰) इति णिचि टिलोपे ‘अच इः’ (प॰उ॰~४.१४८) इत्यौणादिक\-प्रत्यये ‘णेरनिटि’ (पा॰सू॰~६.४.५१) इति णिलोपे च निष्पन्नोऽयं पतिशब्दः ‘पतिः समास एव’ (पा॰सू॰~१.४.८) इत्यत्र न गृह्यते लाक्षणिकत्वादिति। एतेन ‘कृष्णस्य सखिरर्जुनः’ इति भारतम् ‘सखिना वानरेन्द्रेण’ इति रामायणं च व्याख्यातम्‌} (त॰बो॰~२५७)। अत्रत्या प्रक्रिया~– पति~\arrow \textcolor{red}{तत्करोति तदाचष्टे} (धा॰पा॰ ग॰सू॰~१८७)~\arrow पति~णिच्~\arrow पति~इ~\arrow \textcolor{red}{णाविष्ठवत्प्राति\-पदिकस्य पुंवद्भाव\-रभाव\-टिलोप\-यणादि\-परार्थम्} (वा॰~६.४.४८)~\arrow पत्~इ~\arrow पति~\arrow \textcolor{red}{सनाद्यन्ता धातवः} (पा॰सू॰~३.१.३२)~\arrow धातुसञ्ज्ञा~\arrow \textcolor{red}{शेषात्कर्तरि परस्मैपदम्} (पा॰सू॰~१.३.७८)~\arrow \textcolor{red}{वर्तमाने लट्} (पा॰सू॰~३.२.१२३)~\arrow पति~लट्~\arrow पति~तिप्~\arrow पति~ति~\arrow \textcolor{red}{कर्तरि शप्‌} (पा॰सू॰~३.१.६८)~\arrow पति~शप्~ति~\arrow पति~अ~ति~\arrow \textcolor{red}{सार्वधातुकार्ध\-धातुकयोः} (पा॰सू॰~७.३.८४)~\arrow पते~अ~ति~\arrow \textcolor{red}{एचोऽयवायावः} (पा॰सू॰~६.१.७८)~\arrow पतय्~अ~ति~\arrow पतयति। पति~\arrow पूर्ववद्धातु\-सञ्ज्ञा~\arrow \textcolor{red}{अच इः} (प॰उ॰~४.१४८)~\arrow पति~इ~\arrow \textcolor{red}{णेरनिटि} (पा॰सू॰~६.४.५१)~\arrow पत्~इ~\arrow पति~\arrow विभक्तिकार्यम्~\arrow पतिः।} यद्वा \textcolor{red}{पतिं स्वर्गं नयति} इति विग्रहे \textcolor{red}{उणादयो बहुलम्‌} (पा॰सू॰~३.३.१) इत्यनेन \textcolor{red}{डन्‌}\-प्रत्यये\footnote{\textcolor{red}{कार्याद्विद्यादनूबन्धम्} (भा॰पा॰सू॰~३.३.१) \textcolor{red}{केचिदविहिता अप्यूह्याः} (वै॰सि॰कौ॰~३१६९) इत्यनुसारमूह्योऽ\-त्राविहितो \textcolor{red}{डन्‌}\-प्रत्ययः।} \textcolor{red}{डित्यभस्याप्यनु\-बन्धकरण\-सामर्थ्यात्‌} (वा॰~६.४.१४३) इत्यनेन नी\-घटकेकारस्य लोपे \textcolor{red}{पतिनन्‌} इति जाते ततः प्रातिपदिक\-सञ्ज्ञायां सौ विभक्तौ \textcolor{red}{अलोऽन्त्यात्पूर्व उपधा} (पा॰सू॰~१.१.६५) इत्यनेनोपधा\-सञ्ज्ञायां \textcolor{red}{सर्वनामस्थाने चासम्बुद्धौ} (पा॰सू॰~६.४.८) इत्यनेन दीर्घे \textcolor{red}{हल्ङ्याब्भ्यो दीर्घात्सुतिस्यपृक्तं हल्‌} (पा॰सू॰~६.१.६८) इत्यनेन सुलोपे \textcolor{red}{नलोपः प्रातिपदिकान्तस्य} (पा॰सू॰~८.२.७) इत्यनेन नकार\-लोपे \textcolor{red}{पतिना} इति प्रथमान्त एव न तृतीयान्तः। अर्थात् \textcolor{red}{पतिना भवान्‌} अर्थात् \textcolor{red}{रामभद्र भवान् पतिं स्वर्गं नीतवानतोऽहमपि निधनं यास्यामि}।\footnote{\textcolor{red}{सह}\-शब्दान्वये \textcolor{red}{अहमपि सह युगपदेव निधनं यास्यामीत्यर्थः}। \textcolor{red}{सह}\-शब्दो यौगपद्ये इति वाचस्पत्य\-काराः। यथा \textcolor{red}{अस्तोदयौ सहैवासौ कुरुते नृपतिर्द्विषाम्} इत्यत्र।} निःशेषं धनं निधनं स्वकीयं सर्वस्वं पतिमनु\-यास्यामि। अथवा निचितं निखिलं भोगादि\-धनं मोक्ष\-रूपं वा धनं यस्मिन् तादृशं निधनं साकेतमहमपि यास्यामि। तारा प्रार्थयते यत् \textcolor{red}{भवान् पतिनाऽर्थात्पतितमपि मे पतिं यया कृपया साकेतं नीतवान् तयैवाहमपि निधनं साकेतं यास्यामि} इति पाणिनीय एव \textcolor{red}{पतिना} शब्दस्तेन नात्रासिद्धिः। अथवा \textcolor{red}{पतिं न असहते} अर्थात्सहत एवेति \textcolor{red}{षहँ मर्षणे} (धा॰पा॰~८५२, १८०९) इत्यस्मात् \textcolor{red}{अच्‌} प्रत्ययः पचादित्वात्।\footnote{\textcolor{red}{नन्दि\-ग्रहि\-पचादिभ्यो ल्युणिन्यचः} (पा॰सू॰~३.१.१३४) इत्यनेन।} अत्र \textcolor{red}{न सहः} इति \textcolor{red}{असहः} इति नञ्तत्पुरुषः।
\textcolor{red}{न असहः} इति \textcolor{red}{नासहः} इति सुप्सुपा\-समासः।\footnote{\textcolor{red}{नलोपो नञः} (पा॰सू॰~६.३.७३) इत्यनेन नलोपो नञ्तत्पुरुष\-समासे भवति परन्तु सुप्सुपा\-समासे न भवतीति व्याकरण\-चन्द्रोदये प्रथम\-खण्डे चारुदेव\-शास्त्रिणः। यथा नचिर (\textcolor{red}{यथा नचिरकालं नो निष्कृतिः स्यात्त्रिलोकगे} म॰भा॰~१.९६.१९, \textcolor{red}{मुनिर्ब्रह्म नचिरेणाधिगच्छति} भ॰गी॰~५.६, \textcolor{red}{भवामि नचिरात्पार्थ मय्यावेशितचेतसाम्‌} भ॰गी॰~१२.७), नान्तरीयक (\textcolor{red}{यत्र नान्तरीयकोऽलाश्रीयते नासावल्विधिः} भा॰पा॰सू॰~१.१.५६ \textcolor{red}{नान्तरीयकत्वात्} भा॰पा॰सू॰~१.२.३९, ३.३.१८, ३.४.२१, ४.१.९२) नाणक (\textcolor{red}{तुलाशासन\-मानानां कूटकृन्नाणकस्य च} या॰स्मृ॰~२.२०.२४०), नास्ति (\textcolor{red}{अस्ति नास्ति न जानन्ति देहि देहि पुनः पुनः} म॰सु॰स॰~५५४), नेष्ट (\textcolor{red}{अतिककुदाः कृशदेहा नेष्टा हीनाधिकाङ्ग्यश्च॥} बृ॰सं॰~६१.४), नसंहत (\textcolor{red}{नसंहतास्तस्य नभिन्नवृत्तयः} कि॰~१.१९), नभिन्न (\textcolor{red}{नसंहतास्तस्य नभिन्नवृत्तयः} कि॰~१.१९), नसुकर (\textcolor{red}{कृत्वा नसुकरं कर्म गता वैवस्वतक्षयम्} म॰भा॰~८.३.२१), नैकभेद (\textcolor{red}{उच्चावचं नैकभेदम्} अ॰को॰~३.१.८३) नैक (\textcolor{red}{सा ददर्श नगान्नैकान् नैकाश्च सरितस्तथा} म॰भा॰~३.६१.१०४, \textcolor{red}{एको नैकः सवः कः किं यत्तत्पदमनुत्तमम्‌} वि॰स॰ना॰~६१) इत्यादिषु। यथा वाचस्पतये – \textcolor{red}{नैक त्रि० न एकः नञर्थनशब्देन “सह सुपा” पा॰स॰~अनेके १ एकभिन्ने २ विष्णौ पु॰~“एको नैकः सवः कः किम्” विष्णुस॰~मायया बहुरूपत्वान्नैकः “इन्द्रो मायाभिः पुरुरूप ईयते इति श्रुतेः” भा॰}।} ततः \textcolor{red}{पत्युर्नासहः} इति \textcolor{red}{पति\-नासहः} इति षष्ठी\-तत्पुरुष\-समासस्तत्सम्बुद्धौ \textcolor{red}{हे पति\-नासह} अर्थात् \textcolor{red}{हे पति\-पाप\-सहन\-कर्तः}। अर्थात्प्रभो त्वमति\-करुणोऽसि कारुणिकोत्तमोऽसि यतो हि पर\-कलत्र\-गामिनं नितान्त\-कामिनं पापिनं मम स्वामिनमपि त्वं नासहसे। \textcolor{red}{अभावाभावः प्रतियोगि\-ज्ञानस्य कारणं भवति} इति न्यायेन तमपि क्षमसे स्मेदृशी ते क्षमा तयैव प्रभो मेऽपराधमपि क्षान्त्वा मामपि निधनं साकेतं नय। \textcolor{red}{ह} इति पाद\-पूर्तौ \textcolor{red}{पतिमपि नाऽसिनोषि} न कर्म\-बन्धनेन बध्नास्यपि तु तमपि कर्म\-बन्धनान्मोचयित्वा साकेतं नयसीति \textcolor{red}{पतिनासः} इति विग्रहे \textcolor{red}{न आसिनोतीति नासः} इति विग्रहे \textcolor{red}{आ}पूर्वकात् \textcolor{red}{षिञ् बन्धने} (धा॰पा॰~१२४८, १४७७) इत्यस्मात् \textcolor{red}{अन्येष्वपि दृश्यते} (पा॰सू॰~३.२.१०१) इत्यनेन \textcolor{red}{ड}\-प्रत्यये डित्त्वाट्टिलोपे\footnote{\textcolor{red}{डित्यभस्याप्यनु\-बन्धकरण\-सामर्थ्यात्‌} (वा॰~६.४.१४३) इत्यनेन।} \textcolor{red}{पतिनासः} तत्सम्बुद्धौ \textcolor{red}{हे पतिनास हे वालि\-मोक्ष\-दायिन्‌} मामपि साकेतं नय। अथवा \textcolor{red}{पतिश्चासाविनश्चेति पतिनः}।\footnote{\textcolor{red}{शकन्ध्वादिषु पर\-रूपं वाच्यम्‌} (वा॰~६.१.९४) इत्यनेन पररूपः।} \textcolor{red}{न सहते भक्ताभक्त\-पीडामित्यसहः}।\footnote{अत्रापि \textcolor{red}{नन्दि\-ग्रहि\-पचादिभ्यो ल्युणिन्यचः} (पा॰सू॰~३.१.१३४) इत्यनेनाच्।} \textcolor{red}{पतिनश्चासावसहश्चेति पतिनासहस्तत्सम्बुद्धौ हे पतिनासह} इति विग्रहेऽपि प्रयोगसिद्धिः। अत्र पूर्वं स्वामि\-वाचकेन \textcolor{red}{इन}\-शब्देन \textcolor{red}{पति}\-शब्दस्य कर्मधारयः \textcolor{red}{पतिरेवेनः} इति विग्रहः। अत्र पति\-शब्दः पालकार्थ\-वाचकः। सति कर्मधारये विभक्ति\-लोपे पति\-घटकेकारस्येन\-घटकेकारेण सह \textcolor{red}{अकः सवर्णे दीर्घः} (पा॰सू॰~६.१.१०१) इति दीर्घः स्यादिति वाच्यम्। \textcolor{red}{शकन्ध्वादिषु पर\-रूपं वाच्यम्‌} (वा॰~६.१.९४) इत्यनेन पर\-रूपे सति न दोषः। न च \textcolor{red}{पतिन}\-शब्दस्य शकन्ध्वादि\-गणे न पाठः। तस्याप्याकृति\-गणतया\footnote{\textcolor{red}{आकृतिगणोऽयम्} (वै॰सि॰कौ॰~७९, ल॰सि॰कौ॰~३९)।} पाठ\-स्वीकारे क्षति\-विरहः। एवं \textcolor{red}{न सहते} इति पचाद्यञ्ञिष्पन्नस्य\footnote{\textcolor{red}{नन्दि\-ग्रहि\-पचादिभ्यो ल्युणिन्यचः} (पा॰सू॰~३.१.१३४)इत्यनेन।} \textcolor{red}{असह}\-शब्दस्य \textcolor{red}{पतिन एवासहः} इति विग्रहे कर्मधारये दीर्घे सम्बोधने च \textcolor{red}{हे पतिनासह} इति सिद्धम्। यद्वा \textcolor{red}{पतिं नाशयतीति पति\-नाशः} इति विग्रहे \textcolor{red}{कर्मण्यण्‌} (पा॰सू॰~३.२.१) इत्यनेन \textcolor{red}{अण्‌}प्रत्यये \textcolor{red}{नश्‌}\-धातोः (\textcolor{red}{णशँ अदर्शने} धा॰पा॰~११९४) उपधा\-वृद्धौ\footnote{\textcolor{red}{अत उपधायाः} (पा॰सू॰~७.२.११६) इत्यनेन।} समासे पृषोदरादित्वाद्दन्त्य\-सकारे सति सम्बोधने \textcolor{red}{हे पति\-नास}। \textcolor{red}{ह} इति पाद\-पूर्तौ। अथवा \textcolor{red}{अनः शकटे जले} इति नानार्थ\-कोषात्।\footnote{मूलं मृग्यम्। \textcolor{red}{क्लीबेऽनः शकटोऽस्त्री स्यात्} (अ॰को॰~२.८.५२) इत्यमरः। \textcolor{red}{आ ते॑ कारो शृणवामा॒ वचां॑सि य॒याथ॑ दू॒रादन॑सा॒ रथे॑न} (ऋ॰वे॰सं॰~३.३३.१०) इति मन्त्रे \textcolor{red}{अनसा शकटेन} इति सायणाः।} अत्र \textcolor{red}{पत्युरनः शरीर\-रूपं शकटं हन्तीति पतिनासहः} तत्सम्बुद्धौ \textcolor{red}{हे पतिनासह} इति विग्रहे \textcolor{red}{पति}\-शब्दस्य \textcolor{red}{अनस्‌}\-शब्देन तत्पुरुषे पृषोदरादित्वादकार\-लोपे \textcolor{red}{डस}\-प्रत्यये\footnote{\textcolor{red}{कार्याद्विद्यादनूबन्धम्} (भा॰पा॰सू॰~३.३.१) \textcolor{red}{केचिदविहिता अप्यूह्याः} (वै॰सि॰कौ॰~३१६९) इत्यनुसारमूह्योऽ\-त्राविहितः समासान्तो \textcolor{red}{डस}\-प्रत्ययः।} डित्त्वसामर्थ्याट्टिलोप\footnote{\textcolor{red}{डित्यभस्याप्यनु\-बन्धकरण\-सामर्थ्यात्‌} (वा॰~६.४.१४३) इत्यनेन।} आकारादेशे\footnote{पृषोदरादित्वादाकारादेशः।} पुनः \textcolor{red}{हन्‌} धातोः (\textcolor{red}{हनँ हिंसा\-गतयोः} धा॰पा॰~१०१२)
\textcolor{red}{अन्येष्वपि दृश्यते} (पा॰सू॰~३.२.१०१) इत्यनेन \textcolor{red}{ड}\-प्रत्यये टिलोपे\footnote{\textcolor{red}{डित्यभस्याप्यनु\-बन्धकरण\-सामर्थ्यात्‌} (वा॰~६.४.१४३) इत्यनेन।} \textcolor{red}{पतिनासहः}\footnote{पति~अनस्‌~\arrow \textcolor{red}{पृषोदरादीनि यथोपदिष्टम्} (पा॰सू॰~६.३.१०९)~\arrow अलोपः~\arrow पति~नस्~\arrow पतिनस्~\arrow \textcolor{red}{उणादयो बहुलम्} (पा॰सू॰~३.३.१)~\arrow पतिनस्~डस~\arrow पतिनस्~अस~\arrow \textcolor{red}{डित्यभस्याप्यनु\-बन्धकरण\-सामर्थ्यात्‌} (वा॰~६.४.१४३)~\arrow पतिन्~अस~\arrow पतिनस~\arrow \textcolor{red}{पृषोदरादीनि यथोपदिष्टम्} (पा॰सू॰~६.३.१०९)~\arrow आकारादेशः~\arrow पतिनास। \textcolor{red}{पतिनासं हन्ति} इति विग्रहे \textcolor{red}{पतिनासम्} इत्युपपदे \textcolor{red}{अन्येष्वपि दृश्यते} (पा॰सू॰~३.२.१०१) इत्यनेन \textcolor{red}{ड}\-प्रत्ययः। पतिनास~अम्~हन्~ड~\arrow पतिनास~अम्~हन्~अ~\arrow \textcolor{red}{डित्यभस्याप्यनु\-बन्धकरण\-सामर्थ्यात्‌} (वा॰~६.४.१४३)~\arrow पतिनास~अम्~ह्~अ~\arrow पतिनास~अम्~ह~\arrow \textcolor{red}{सुपो धातु\-प्रातिपदिकयोः} (पा॰सू॰~२.४.७१)~\arrow पतिनास~ह~\arrow पतिनासह~\arrow विभक्ति\-कार्यम्~\arrow पतिनासहः।} तत्सम्बुद्धौ \textcolor{red}{हे पतिनासह} अर्थाद्धे मत्पति\-शरीर\-शकट\-हन्तः प्रभोऽहमपि त्वल्लोकं जिगमिषामि। अथवा \textcolor{red}{अपतत्त्वच्चरणारविन्दे यः स पतिर्वाली} इति \textcolor{red}{पत्‌}\-धातोः (\textcolor{red}{पतॢँ गतौ} धा॰पा॰~८४५) भूतकाल औणादिके \textcolor{red}{इच्‌}प्रत्यये\footnote{\textcolor{red}{कार्याद्विद्यादनूबन्धम्} (भा॰पा॰सू॰~३.३.१) \textcolor{red}{केचिदविहिता अप्यूह्याः} (वै॰सि॰कौ॰~३१६९) इत्यनुसारमूह्योऽ\-त्राविहितः \textcolor{red}{इच्}\-प्रत्ययः।} चकारानुबन्ध\-कार्ये पुनस्तस्यैव घि\-सञ्ज्ञायां तृतीया\-\textcolor{red}{टा}\-विभक्तौ \textcolor{red}{आङो नाऽस्त्रियाम्‌} (पा॰सू॰~७.३.१२०) इत्यनेन नादेशे\footnote{पूर्ववत् \textcolor{red}{गौण\-मुख्ययोर्मुख्ये कार्य\-सम्प्रत्ययः} (प॰शे॰~१५) इति परिभाषया गौणे पति\-शब्दे \textcolor{red}{पतिः समास एव} (पा॰सू॰~१.४.८) इति सूत्रस्याप्रवृत्तौ \textcolor{red}{शेषो घ्यसखि} (पा॰सू॰~१.४.७) इति सूत्रेण घि\-सञ्ज्ञा।} योगे \textcolor{red}{सह\-युक्तेऽप्रधाने} (पा॰सू॰~२.३.१९) इत्यनेन तृतीयाप्राप्तौ \textcolor{red}{पतिना सह} इति पाणिनीयपरम्परया सम्यक्साधु। अर्थात् \textcolor{red}{त्वच्चरणारविन्द\-पतितेन पतिना वालिना सहाहमपि निधनं त्वत्सालोक्यं यास्यामि}। अथवा \textcolor{red}{पत्युर्वालिनो नाशः कुकर्म\-कार्यतया नरक\-गमनमिति पतिनासः} पृषोदरादित्वाद्दन्त्य\-सकारः \textcolor{red}{तमेव हन्तीति पतिनासहस्तत्सम्बुद्धौ हे पतिनासह} इति विग्रहे 
\textcolor{red}{हन्‌} धातोः (\textcolor{red}{हनँ हिंसा\-गतयोः} धा॰पा॰~१०१२) \textcolor{red}{अन्येष्वपि दृश्यते} (पा॰सू॰~३.२.१०१) इत्यनेन
\textcolor{red}{ड}\-प्रत्ययेऽभत्वेऽपि
डित्त्व\-सामर्थ्याट्टिलोपे\footnote{\textcolor{red}{डित्यभस्याप्यनु\-बन्धकरण\-सामर्थ्यात्‌} (वा॰~६.४.१४३) इत्यनेन।} सम्बोधन एकवचने \textcolor{red}{पतिनासह}। अर्थान्निज\-कृत\-कुकर्म\-परिणामेन तु मम पत्युर्नरक\-गमनमनिवार्यमासीत्किन्तु रामभद्र त्वमेव दीन\-वत्सलतया निज\-कृपा\-मन्दाकिनी\-शीकर\-सम्पात\-सेचनेन तादृशं नरक\-रूपं पतिनासमपि निहत्य तस्मै परां गतिं दत्तवानतोऽहमपि तामेव गन्तुमीह इति तारा\-तात्पर्यम्।\end{sloppypar}
\section[नाथनाथेति]{नाथनाथेति}
\centering\textcolor{blue}{पतितं वालिनं दृष्ट्वा रक्तैः पांसुभिरावृतम्।\nopagebreak\\
रुदती नाथनाथेति पतिता तस्य पादयोः॥}\nopagebreak\\
\raggedleft{–~अ॰रा॰~४.३.७}\\
\begin{sloppypar}\hyphenrules{nohyphenation}\justifying\noindent\hspace{10mm} अत्र रामेण निहतं वालिनं \textcolor{red}{नाथ\-नाथेति} तारा रुदती व्याहरत्। \textcolor{red}{नाथ\-नाथ इति} अयं सम्बोधने प्रथमैक\-वचनान्तः।
अतोऽत्र प्लुतः।\footnote{\textcolor{red}{दूराद्धूते च} (पा॰सू॰~८.२.८४) इत्यनेन।} एवं च \textcolor{red}{प्लुत\-प्रगृह्या अचि नित्यम्‌} (पा॰सू॰~६.१.१२५) इत्यनेन प्रकृति\-भावे गुणोऽसङ्गत इव\footnote{\textcolor{red}{नाथ\-नाथ३ इति} इत्यनेन भवितव्यमिति भावः।} किन्तु \textcolor{red}{प्राचाम्‌} (पा॰सू॰~८.२.८६) इति योग\-विभागे प्लुतानां वैकल्पिकत्वाद्गुण\-सम्भवः।\footnote{\textcolor{red}{इह प्राचामिति योगो विभज्यते। तेन सर्वः प्लुतो विकल्प्यते} (वै॰सि॰कौ॰~९७)।} यद्वा \textcolor{red}{अनुकरणानु\-कार्ययोर्भेदाभेद\-विवक्षा च}\footnote{मूलं मृग्यम्। \textcolor{red}{मतौ च्छः सूक्तसाम्नोः} (पा॰सू॰~५.२.५९) इत्यस्य भाष्ये प्रदीपोद्द्योतयोश्च स्पष्टमिदम्। अभेदपक्षे \textcolor{red}{प्रकृतिवदनुकरणं भवति} (भा॰शि॰सू॰~२) इति महाभाष्ये \textcolor{red}{ऋऌक्‌} (शि॰सू॰~२) शिवसूत्र उक्तम्। \textcolor{red}{अनुकरणं ह्यनुकार्याद्भिन्नम्‌} इत्यपि महाभाष्ये \textcolor{red}{मतौ च्छः सूक्तसाम्नोः} (पा॰सू॰~५.२.४९) सूत्र उक्तमिति वैयाकरण\-भूषण\-सारस्य दर्पण\-व्याख्यायां चन्द्रिका\-प्रसाद\-द्विवेदाः। अस्माभिर्भाष्य\-संस्करणेषु \textcolor{red}{अनुकरणं ह्यनुकार्याद्भिन्नम्‌} इति नोपलब्धम्।} इति परिभाषयाऽस्य प्रयोगस्यानु\-करणतयाऽभेद\-विवक्षायां विभक्त्यभावे सति गुणः सङ्गत एव। यद्वा \textcolor{red}{नाथ एव नाथो यस्याः सा नाथ\-नाथा} अर्थान्नाथो वाल्येव नाथः पतिर्यस्याः सा नाथ\-नाथा तारा। वालि\-परिचयार्थं सम्बोधयति यत् \textcolor{red}{पते त्वमेव यस्याः स्वामी सा एति तव सम्मुखमागच्छत्यतो मां पश्य} इत्यभिप्राये नाथ\-नाथा\-शब्दे बहुव्रीहिरन्य\-पदार्थश्च तारा। एवं गच्छत्यर्थक\-\textcolor{red}{एति}\-शब्दो \textcolor{red}{नाथ\-नाथा} इत्यनेन सह सन्धीयतां ततः \textcolor{red}{एङि पर\-रूपम्‌} (पा॰सू॰~६.१.९४) इत्यनेन पररूपे \textcolor{red}{नाथ\-नाथेति}। न चोक्त\-सूत्रमुपसर्ग\-धातु\-सन्धि\-विषयकमिति वाच्यम्। 
लक्ष्यानुरोधेनोप\-सर्गांशानुवृत्ति\-मोषे न दोषः।\footnote{\textcolor{red}{उपसर्गादृति धातौ} (पा॰सू॰~६.१.९१) इत्यतोऽनुवृत्तम् \textcolor{red}{उपसर्गात्} इति पदं लक्ष्यानुरोधेनात्र मोषणीयमिति भावः।} यथा \textcolor{red}{कर्तुरीप्सिततमं कर्म} (पा॰सू॰~१.४.४९) इत्यत्र क्रियाक्षिप्त\-कर्त्रंशे जागरूके व्यर्थीभूते \textcolor{red}{कर्तुः} इति पदं
\textcolor{red}{प्रकृतिधातूपात्त\-प्रधान\-भूत\-व्यापाराश्रयो यः कर्ता} इति विशिष्टार्थं बोधयति। तत्र पुनः \textcolor{red}{ईप्सित} इति वर्तमाने क्तप्रयोगात् \textcolor{red}{क्तस्य च वर्तमाने} (पा॰सू॰~२.३.६७) इति षष्ठ्यपि कर्तरि।\footnote{\textcolor{red}{मति\-बुद्धि\-पूजार्थेभ्यश्च} (पा॰सू॰~३.२.१८८) इत्यनेन \textcolor{red}{ईप्सित} इत्यत्र मत्यर्थे पूजार्थे वा वर्तमाने क्तः। \textcolor{red}{मतिरिच्छा} (का॰वृ॰~३.२.१८८) इति काशिका। \textcolor{red}{मतिरिहेच्छा} (वै॰सि॰कौ॰~३०८९) इति सिद्धान्त\-कौमुदी।} तत इयमप्येवं प्रकृतिरपीत्युभे
कर्त्रर्थं बोधयतः। अतः
षष्ठी\-वाच्य\-कर्त्रर्थस्य मोषं कुर्वन्ति गुरवः शेमुषीजुषः। तथैवात्राप्युपसर्गांश\-मोषेऽपि न रोषः कर्तव्यः। इत्थं \textcolor{red}{नाथनाथेति} इत्यत्र \textcolor{red}{एङि पररूपम्‌} (पा॰सू॰~६.१.९४) इत्यनेन पररूपम्।\end{sloppypar}
\section[मे]{मे}
\centering\textcolor{blue}{पूर्वजन्मनि ते सुभ्रु कृता मद्भक्तिरुत्तमा।\nopagebreak\\
अतस्तव विमोक्षाय रूपं मे दर्शितं शुभे॥}\nopagebreak\\
\raggedleft{–~अ॰रा॰~४.३.३४}\\
\begin{sloppypar}\hyphenrules{nohyphenation}\justifying\noindent\hspace{10mm} अत्र \textcolor{red}{क्तस्य च वर्तमाने} (पा॰सू॰~२.३.६७) इत्यनेन षष्ठी तृतीयां प्रबाध्य।\footnote{\textcolor{red}{मति\-बुद्धि\-पूजार्थेभ्यश्च} (पा॰सू॰~३.२.१८८) इत्यनेन बुद्ध्यर्थे वर्तमाने क्तः। सूत्रेऽस्मिन् बुद्धेर्ज्ञानमर्थः। यथा काशिकाकाराः~– \textcolor{red}{बुद्धिर्ज्ञानम्‌} (का॰वृ॰~३.२.१८८)। भगवतो रूपदर्शनं च प्रत्यक्ष\-ज्ञापनम्। यतो हि \textcolor{red}{इन्द्रियार्थ\-सन्निकर्ष\-जन्यं ज्ञानं प्रत्यक्षम्‌} (त॰स॰~४२)। यद्वा \textcolor{red}{मति\-बुद्धि\-पूजार्थेभ्यश्च} (पा॰सू॰~३.२.१८८) इत्यत्र \textcolor{red}{च}कारेणान्यत्रापि। \textcolor{red}{चकारोऽनुक्त\-समुच्चयार्थः। “शीलितो रक्षितः क्षान्त आक्रुष्टो जुष्ट इत्यपि” इत्यादि} (वै॰सि॰कौ॰~३०८९) इति सिद्धान्त\-कौमुद्यां भट्टोजि\-दीक्षित\-महाभागाः।} यद्वा \textcolor{red}{मे} इत्यस्य \textcolor{red}{रूपम्‌} इत्यनेनान्वयः।\footnote{तर्हि सम्बन्ध\-सामान्ये षष्ठी। सम्बन्धश्च धर्मि\-धर्म\-भाव\-रूपः। पूर्वार्धे च \textcolor{red}{त्वया कृता} इति प्रयोक्तव्ये \textcolor{red}{ते कृता} इति प्रयुक्तम्। \textcolor{red}{ते} इत्यस्य \textcolor{red}{पूर्वजन्मनि} इत्यनेनान्वये शङ्कापरिहारः। \textcolor{red}{सुभ्रु तव पूर्वजन्मनि उत्तमा मद्भक्तिः कृता} इति भावः।}\end{sloppypar}
\section[कुण्डेन]{कुण्डेन}
\centering\textcolor{blue}{अगस्त्येनोक्तमार्गेण कुण्डेनाऽगमवित्तमः।\nopagebreak\\
जुहुयान्मूलमन्त्रेण पुंसूक्तेनाथवा बुधः॥}\nopagebreak\\
\raggedleft{–~अ॰रा॰~४.४.३१}\\
\begin{sloppypar}\hyphenrules{nohyphenation}\justifying\noindent\hspace{10mm} अत्र सप्तम्यां तृतीया प्रोक्ता। करणत्व\-विवक्षायां\footnote{\textcolor{red}{कर्तृकरणयोस्तृतीया} (पा॰सू॰~२.३.१८) इत्यनेन।} हेतुत्व\-विवक्षायां\footnote{\textcolor{red}{हेतौ} (पा॰सू॰~२.३.२३) इत्यनेन।} प्रकृत्यादित्वाद्वा\footnote{\textcolor{red}{प्रकृत्यादिभ्य उपसङ्ख्यानम्‌} (वा॰~२.३.१८) इत्यनेन।} इयं साध्वी।\footnote{यद्वा \textcolor{red}{कुण्डे नाऽऽगमवित्तमः} इति पदच्छेदे \textcolor{red}{आगमवित्तमो ना मनुष्यः कुण्डे जुहुयात्‌} इत्यपि समाधानम्।}\end{sloppypar}
\section[माम्]{माम्‌}
\centering\textcolor{blue}{मद्भक्तो यदि मामेवं पूजां चैव दिने दिने।\nopagebreak\\
करोति मम सारूप्यं प्राप्नोत्येव न संशयः॥}\nopagebreak\\
\raggedleft{–~अ॰रा॰~४.४.३९}\\
\begin{sloppypar}\hyphenrules{nohyphenation}\justifying\noindent\hspace{10mm} अत्र \textcolor{red}{पूजाम्‌} इति कृदन्त\-प्रयोगेण \textcolor{red}{कर्तृ\-कर्मणोः कृति} (पा॰सू॰~२.३.६५) इत्यनेन षष्ठ्युचिता किन्तु \textcolor{red}{प्रति} इत्यध्याहारेण \textcolor{red}{मां प्रति पूजां करोति} इति \textcolor{red}{अभितः\-परितः\-समया\-निकषा\-हाप्रति\-योगेऽपि} (वा॰~२.३.२) इति \textcolor{red}{प्रति}\-योगे द्वितीयेति सिद्धम्।\end{sloppypar}
\section[शेषांशाय]{शेषांशाय}
\centering\textcolor{blue}{एवं परात्मा श्रीरामः क्रियायोगमनुत्तमम्।\nopagebreak\\
पृष्टः प्राह स्वभक्ताय शेषांशाय महात्मने॥}\nopagebreak\\
\raggedleft{–~अ॰रा॰~४.४.४१}\\
\begin{sloppypar}\hyphenrules{nohyphenation}\justifying\noindent\hspace{10mm} अत्रापि पूर्व\-प्रकारेण चतुर्थी \textcolor{red}{हित\-योगे च} (वा॰~२.३.१३) इत्यनेन।\footnote{\textcolor{red}{हितम्} इत्यध्याहार्यमिति भावः। यद्वा \textcolor{red}{तादर्थ्ये चतुर्थी वाच्या} (वा॰~२.३.१३) इत्यनेन चतुर्थी।}\end{sloppypar}
\section[हा सीतेति]{हा सीतेति}
\centering\textcolor{blue}{पुनः प्राकृतवद्रामो मायामालम्ब्य दुःखितः।\nopagebreak\\
हा सीतेति वदन्नैव निद्रां लेभे कथञ्चन॥}\nopagebreak\\
\raggedleft{–~अ॰रा॰~४.४.४२}\\
\begin{sloppypar}\hyphenrules{nohyphenation}\justifying\noindent\hspace{10mm} \textcolor{red}{हा सीतेति} इत्यत्र \textcolor{red}{इति}\-शब्दार्थकं \textcolor{red}{ति}\-शब्दं स्वीकृत्य समाधेयम्।\footnote{विस्तराय \pageref{sec:jaayeti_siiteti}तमे पृष्ठे \ref{sec:jaayeti_siiteti} \nameref{sec:jaayeti_siiteti} इति प्रयोगस्य विमर्शं पश्यन्तु।}\end{sloppypar}
\section[मणिसानौ]{मणिसानौ}
\centering\textcolor{blue}{रामस्तु पर्वतस्याग्रे मणिसानौ निशामुखे।\nopagebreak\\
सीताविरहजं शोकमसहन्निदमब्रवीत्॥}\nopagebreak\\
\raggedleft{–~अ॰रा॰~४.५.१}\\
\begin{sloppypar}\hyphenrules{nohyphenation}\justifying\noindent\hspace{10mm} अत्र \textcolor{red}{सानु}\-शब्दस्य प्रायो नपुंसक\-लिङ्गे प्रयोगात्\footnote{यथा \textcolor{red}{या सानु॑नि॒ पर्व॑ताना॒मदा॑भ्या म॒हस्त॒स्थतु॒रर्व॑तेव सा॒धुना॑} (ऋ॰वे॰सं॰~१.१५५.१) \textcolor{red}{बृ॑ह॒तः सानु॑न॒स्परि॑} (ऋ॰वे॰सं॰~५.५९.७) \textcolor{red}{सानू॑नि दि॒वो अ॒मृत॑स्य के॒तुना॑} (ऋ॰वे॰सं॰~६.७.६) \textcolor{red}{सानूनि मृगपक्षिणः} (वा॰रा॰~२.३३.२३) \textcolor{red}{गिरेः सानूनि रम्याणि} (वा॰रा॰~२.९३.९) \textcolor{red}{तस्य शैलस्य सानूनि} (वा॰रा॰~३.६१.२१) \textcolor{red}{दक्षिणे गिरिसानुनि} (वा॰रा॰~४.१.७३) \textcolor{red}{अस्मिन्सानुनि} (वा॰रा॰~४.१.१०३) \textcolor{red}{सानूनि सुमहान्ति च} (वा॰रा॰~६.६७.४) \textcolor{red}{तस्मिन्मन्दरसानुनि} (ग॰सं॰~७.४३.१४, भा॰पु॰~४.२३.२४) \textcolor{red}{यत्र स्रुतक्षीरतया प्रसूतः सानूनि गन्धः सुरभीकरोति} (कु॰स॰~१.९) इत्यादिषु। \textcolor{red}{सानुगतां} (कु॰स॰~१.५) इत्यत्र मल्लिनाथाः~– \textcolor{red}{सानूनि मेघमण्डलादधस्तटानि गतां प्राप्ताम्} (कु॰स॰ स॰व्या॰~१.९)।} सप्तम्येकवचने \textcolor{red}{इकोऽचि विभक्तौ} (पा॰सू॰~७.१.७३) इत्यनेन नुम्यनुबन्ध\-कार्ये \textcolor{red}{मणि\-सानुनि} इत्येव पाणिनीयम्।\footnote{पूर्वपक्षोऽयम्।} परं \textcolor{red}{मणि\-सानु}\-शब्दम् \textcolor{red}{अर्धर्चादि}\-गणे (\textcolor{red}{अर्धर्चाः पुंसि च} पा॰सू॰~२.४.३१) आकृति\-गणतया मत्वा पुंस्त्वम्। यद्वा \textcolor{red}{लिङ्गमशिष्यं लोकाश्रयत्वाल्लिङ्गस्य} (भा॰पा॰सू॰~२.१.३६) इति भाष्य\-वचनेन लिङ्गानामनियमनतया पुल्लिँङ्गे प्रयोगोऽपि पाणिनीयः।\footnote{यद्वा लिङ्गानुशासने \textcolor{red}{मद्गु\-मधु\-सीधु\-शीधु\-सानु\-कमण्डलूनि नपुंसके च} (लि॰~५६) इत्यत्र चकारात्पुंस्यपि \textcolor{red}{सानु}\-शब्दः।  यथा \textcolor{red}{अ॒पाद॑ह॒स्तो अ॑पृतन्य॒दिन्द्र॒मास्य॒ वज्र॒मधि॒ सानौ॑ जघान} (ऋ॰वे॰सं॰~१.३२.७) \textcolor{red}{उ॒र्व्याः प॒दो नि द॑धाति॒ सानौ॑} (ऋ॰वे॰सं॰~१.१४६.२) \textcolor{red}{पृथि॒व्याः सानौ॒ जङ्घ॑नन्त पा॒णिभि॑} (ऋ॰वे॰सं॰~२.३१.२) \textcolor{red}{म॒ना॒नग्रेतो॑ जहतुर्वि॒यन्ता॒ सानौ॒} (ऋ॰वे॰सं॰~१०.६१.६) इत्यादिषु। \textcolor{red}{स्नुः प्रस्थः सानुरस्त्रियौ} (अ॰को॰~२.३.५) इत्यमरः। \textcolor{red}{‘सानु’आदि\-शब्दानां स्वत एव द्विलिङ्गता~– ‘सानुने’ ‘सानवे’। स्नुः प्रस्थः सानुरस्त्रियामित्यमरः} (ह॰ना॰व्या॰~२.९३) इति हरिनामामृत\-व्याकरणे जीवगोस्वामिनः।}\end{sloppypar}
\section[मम]{मम}
\centering\textcolor{blue}{जीवतीति मम ब्रूयात्कश्चिद्वा प्रियकृत्स मे।\nopagebreak\\
यदि जानामि तां साध्वीं जीवन्तीं यत्र कुत्र वा॥}\nopagebreak\\
\raggedleft{–~अ॰रा॰~४.५.३}\\
\begin{sloppypar}\hyphenrules{nohyphenation}\justifying\noindent\hspace{10mm} अत्र \textcolor{red}{ब्रू}\-धातु\-योगे (\textcolor{red}{ब्रूञ् व्यक्तायां वाचि} धा॰पा॰~१०४४) द्वितीया तूचितैव\footnote{\textcolor{red}{अकथितं च} (पा॰सू॰~१.४.५१) इत्यनेन कर्म\-सञ्ज्ञायां \textcolor{red}{कर्मणि द्वितीया} (पा॰सू॰~२.३.२) इत्यनेन।} किन्तु कर्मणि सम्बन्ध\-विवक्षायां षष्ठ्यपि साध्वी।\end{sloppypar}
\section[बहुऋक्षवानरैः]{बहुऋक्षवानरैः}
\centering\textcolor{blue}{भेरीमृदङ्गैर्बहुऋक्षवानरैः श्वेतातपत्रैर्व्यजनैश्च शोभितः।\nopagebreak\\
नीलाङ्गदाद्यैर्हनुमत्प्रधानैः समावृतो राघवमभ्यगाद्धरिः॥}\nopagebreak\\
\raggedleft{–~अ॰रा॰~४.५.६३}\\
\begin{sloppypar}\hyphenrules{nohyphenation}\justifying\noindent\hspace{10mm} अत्र 
लक्ष्मणेन मृत्योर्भीयमानः प्रमादी सुग्रीवः श्रीराममुपगच्छति। अत्रैव प्रयोगः \textcolor{red}{बहुऋक्षवानरैः}। अत्र \textcolor{red}{बहु}\-शब्द\-घटकोकारस्य \textcolor{red}{ऋक्ष}\-घटकर्कारेण सन्धौ \textcolor{red}{बह्वृक्ष\-वानरैः} इत्येव पाणिनीयम्। \textcolor{red}{बहु}\-शब्दस्य \textcolor{red}{ऋक्ष\-वानर}\-शब्देन तत्पुरुष\-कर्मधारये समासे संहिताया नित्यत्वात् \textcolor{red}{इको यणचि} (पा॰सू॰~६.१.७७) इत्यनेन \textcolor{red}{यण्‌} अनिवार्यः। किन्तु व्यस्तावस्थायां सन्धिरनिवार्यो नास्ति। अतः \textcolor{red}{बहु ऋक्षवानरैः} इति पाणिनीयमेव। न च समासं विना \textcolor{red}{बहु}\-शब्दस्य \textcolor{red}{ऋक्ष\-वानर}\-शब्द\-विशेषणतया \textcolor{red}{बहुभिः ऋक्ष\-वानरैः} इति सविभक्तिकः प्रयोगः स्यादेवं \textcolor{red}{बहु ऋक्ष\-वानरैः} इत्यसङ्गतमिति वाच्यम्। \textcolor{red}{सुपां सुलुक्पूर्व\-सवर्णाच्छेयाडाड्यायाजालः} (पा॰सू॰~७.१.३९) इत्यनेन भिस्विभक्तेर्लुकि सङ्गतम्। यद्वा \textcolor{red}{बहु} इति क्रियाविशेषणम्। क्रिया\-विशेषणानां द्वितीयात्वं नपुंसकत्वमेक\-वचनत्वमौत्पत्तिकं सर्व\-विदितमेव। तेन नात्र सन्धिः। यद्वा \textcolor{red}{ऋत्यकः} (पा॰सू॰~६.१.१२४) इत्यनेन शाकल्यमते वैकल्पिक\-प्रकृतिभावः \textcolor{red}{तस्यां वै भार्गवऋषेः} (भा॰पु॰~९.१५.१३) इतिवत्।\footnote{\textcolor{red}{समासेऽप्ययं प्रकृतिभावः। सप्तऋषीणाम्। सप्तर्षीणाम्} (वै॰सि॰कौ॰~९२)। अन्यच्च \textcolor{red}{तत्रासीनं सुरऋषिम्} (भा॰पु॰~७.१.१४) \textcolor{red}{सम्पूज्य देवऋषिवर्यम्} (भा॰पु॰~१०.६९.१६) \textcolor{red}{ब्रह्मऋषीनेतान्} (भा॰पु॰~१०.८६.५७) इत्यादिष्वपि।}\end{sloppypar}
\section[बहिर्गुहाम्]{बहिर्गुहाम्‌}
\centering\textcolor{blue}{यूयं पिदध्वमक्षीणि गमिष्यथ बहिर्गुहाम्।\nopagebreak\\
तथैव चक्रुस्ते वेगाद्गताः पूर्वस्थितं वनम्॥}\nopagebreak\\
\raggedleft{–~अ॰रा॰~४.६.५८}\\
\begin{sloppypar}\hyphenrules{nohyphenation}\justifying\noindent\hspace{10mm} अत्र बहिर्योगे पञ्चमी पाणिनीया\footnote{\textcolor{red}{अपपरिबहिरञ्चवः पञ्चम्या} (पा॰सू॰~२.१.१२) इति ज्ञापनेन।} किन्तु \textcolor{red}{गुहां प्रति बहिः} इति प्रति\-योगे द्वितीयाऽपि पाणिनि\-सम्मता।\footnote{\textcolor{red}{प्रति} इत्यध्याहार्यमिति भावः। ततः \textcolor{red}{अभितः\-परितः\-समया\-निकषा\-हाप्रति\-योगेऽपि} (वा॰~२.३.२) इत्यनेन द्वितीया।} यद्वा \textcolor{red}{बहिर्देशे स्थितां गुहाम्‌} इत्यध्याहारेण द्वितीया।\footnote{यद्वा \textcolor{red}{ज्ञापकसिद्धं न सर्वत्र} इत्यनेन \textcolor{red}{करस्य करभो बहिः} (अ॰को॰~२.६.८०क) इतिवत्पञ्चमीतर\-विभक्तिः। विशेषं \pageref{sec:bahirvanasya}तमे पृष्ठे \ref{sec:bahirvanasya} \nameref{sec:bahirvanasya} इति प्रयोगस्य विमर्शे पश्यन्तु।}\end{sloppypar}
\section[मे]{मे}
\centering\textcolor{blue}{दासी तवाहं राजेन्द्र दर्शनार्थमिहागता।\nopagebreak\\
बहुवर्षसहस्त्राणि तप्तं मे दुश्चरं तपः॥}\nopagebreak\\
\raggedleft{–~अ॰रा॰~४.६.६१}\\
\begin{sloppypar}\hyphenrules{nohyphenation}\justifying\noindent\hspace{10mm} अत्र \textcolor{red}{मया तप्तम्‌} इति कर्मणि \textcolor{red}{क्त}\-प्रत्यय\-विधानादनुक्त\-कर्तरि तृतीया पाणिनीया\footnote{\textcolor{red}{कर्तृ\-करणयोस्तृतीया} (पा॰सू॰~२.३.१८) इत्यनेन।} किन्तु \textcolor{red}{मे} इति कृद्योगा षष्ठी \textcolor{red}{क्तस्य च वर्तमाने} (पा॰सू॰~२.३.६७) इति विहितत्वात्पाणिनि\-सम्मता।\footnote{\textcolor{red}{मति\-बुद्धि\-पूजार्थेभ्यश्च} (पा॰सू॰~३.२.१८८) इत्यनेन वर्तमान\-विवक्षायां क्तः। \textcolor{red}{च}कारेणान्यत्रापि। \textcolor{red}{चकारोऽनुक्त\-समुच्चयार्थः। “शीलितो रक्षितः क्षान्त आक्रुष्टो जुष्ट इत्यपि” इत्यादि} (वै॰सि॰कौ॰~३०८९) इति सिद्धान्त\-कौमुद्यां भट्टोजि\-दीक्षित\-महाभागाः। यद्वा \textcolor{red}{मे} इत्यस्य \textcolor{red}{दुश्चरं तपः} इत्यनेनान्वये कर्मणि सम्बन्ध\-विवक्षायां षष्ठी। \textcolor{red}{मे दुश्चरं तपो बहुवर्षसहस्त्राणि तप्तम्‌} इति भावः।}\end{sloppypar}
\section[रक्षोगणविनाशने]{रक्षोगणविनाशने}
\centering\textcolor{blue}{ब्रह्मणा प्रार्थिताः सर्वे रक्षोगणविनाशने।\nopagebreak\\
मायामानुषभावेन जाता लोकैकरक्षकाः॥}\nopagebreak\\
\raggedleft{–~अ॰रा॰~४.७.१८}\\
\begin{sloppypar}\hyphenrules{nohyphenation}\justifying\noindent\hspace{10mm} अत्र \textcolor{red}{रक्षोगण\-विनाशनाय} इति तादर्थ्ये चतुर्थी सामान्यतः पाणिनीया\footnote{\textcolor{red}{तादर्थ्ये चतुर्थी वाच्या} (वा॰~२.३.१३) इत्यनेन।} किन्तु \textcolor{red}{निमित्तात्कर्म\-संयोगे} (वा॰~२.३.३६) इत्यनेन सप्तमी। निमित्तमिह फलं तस्य कर्मणा सह योगे तद्वाचकात्सप्तमी यथा भाष्य\-वचनम्~–\end{sloppypar}
\centering\textcolor{red}{चर्मणि द्वीपिनं हन्ति दन्तयोर्हन्ति कुञ्जरम्।\nopagebreak\\
केशेषु चमरीं हन्ति सीम्नि पुष्कलको हतः॥}\nopagebreak\\
\raggedleft{–~भा॰पा॰सू॰~२.३.३६}\\
\begin{sloppypar}\hyphenrules{nohyphenation}\justifying\noindent\hspace{10mm} तथा \textcolor{red}{रक्षोगण\-विनाशनं} हि फलम्। कर्म \textcolor{red}{लोके जन्म\-ग्रहणम्‌}। अतोऽत्र सप्तमी।\end{sloppypar}
\section[वानरवृन्दान्]{वानरवृन्दान्‌}
\centering\textcolor{blue}{सुग्रीवः प्रेषयामास सीतायाः परिमार्गणे।\nopagebreak\\
अस्मान्वानरवृन्दान्वै महासत्त्वान्महाबलः॥}\nopagebreak\\
\raggedleft{–~अ॰रा॰~४.७.४३}\\
\begin{sloppypar}\hyphenrules{nohyphenation}\justifying\noindent \textcolor{red}{वृन्द}\-शब्दस्य नपुंसकतया \textcolor{red}{वानर\-वृन्दानि} इति पाणिनि\-सम्मतं किन्तु \textcolor{red}{लिङ्गमशिष्यं लोकाश्रयत्वाल्लिङ्गस्य} (भा॰पा॰सू॰~२.१.३६) इति वचनेन पुँल्लिङ्गेऽप्यनुकूलता।\end{sloppypar}
\vspace{2mm}
\centering ॥ इति किष्किन्धाकाण्डीयप्रयोगाणां विमर्शः ॥\nopagebreak\\
\vspace{4mm}
\pdfbookmark[2]{सुन्दरकाण्डम्‌}{Chap1Part2Kanda5}
\phantomsection
\addtocontents{toc}{\protect\setcounter{tocdepth}{2}}
\addcontentsline{toc}{subsection}{सुन्दरकाण्डीयप्रयोगाणां विमर्शः}
\addtocontents{toc}{\protect\setcounter{tocdepth}{0}}
\centering ॥ अथ सुन्दरकाण्डीयप्रयोगाणां विमर्शः ॥\nopagebreak\\
\section[देवतावृन्दः]{देवतावृन्दः}
\centering\textcolor{blue}{अब्रवीद्देवतावृन्दः कौतूहलसमन्वितः।\nopagebreak\\
गच्छ त्वं वानरेन्द्रस्य किञ्चिद्विघ्नं समाचर॥}\nopagebreak\\
\raggedleft{–~अ॰रा॰~५.१.११}\\
\begin{sloppypar}\hyphenrules{nohyphenation}\justifying\noindent\hspace{10mm} अत्रापि नपुंसक\-लिङ्गे \textcolor{red}{वृन्द}\-शब्दस्य प्रचलित\-प्रयोगात्\footnote{यथा \textcolor{red}{वृन्दान्युत्सार्यमाणानि दूरमुत्ससृजुस्तदा} (वा॰रा॰~६.११७.२१)  \textcolor{red}{वृन्दानि ददृशे तदा} (म॰भा॰~३.१५०.१९)  \textcolor{red}{अभ्यकीर्यन्त वृन्दानि} (म॰भा॰~४.१८.३२)  \textcolor{red}{मासादयैतद्रथसिंहवृन्दम्} (म॰भा॰~४.४९.४)  \textcolor{red}{संशप्तानि च वृन्दानि} (म॰भा॰~५.५४.५५)  \textcolor{red}{मेघवृन्दानि} (म॰भा॰~५.१७८.८४)  \textcolor{red}{सदश्ववृन्दानि} (म॰भा॰~६.५६.१६)  \textcolor{red}{अश्ववृन्दानि} (म॰भा॰~७.१७२.२६)  \textcolor{red}{रथवृन्दानि} (म॰भा॰~६.५९.११, ७.१६१.४९, ७.१७२.२६)  \textcolor{red}{यो वृन्‍दानि त्‍वरयति पथि श्राम्‍यतां प्रोषितानां} (मे॰दू॰~२.३६)  इत्यादिषु।} पुँल्लिङ्ग\-प्रयोगस्तु \textcolor{red}{लिङ्गमशिष्यं लोकाश्रयत्वाल्लिङ्गस्य} (भा॰पा॰सू॰~२.१.३६) इति भाष्य\-नियमेन पाणिनीय एव।\footnote{यद्वा शब्दमिममर्धर्चादिगणे पठित्वा \textcolor{red}{अर्धर्चाः पुंसि च} (पा॰सू॰~२.४.३१) इत्यनेन पुँल्लिङ्ग\-प्रयोगः समर्थनीयः। एवमेव \textcolor{red}{देववृन्दः सदा त्वां तु स्मृत्वा विजयतेऽसुरान्} (म॰भा॰~१.२११.१९)  इत्यादिषु बोध्यम्।} \textcolor{red}{यथोत्तरं मुनीनां प्रामाण्यम्‌} इति वचनात्।\end{sloppypar}
\section[मे]{मे}
\centering\textcolor{blue}{सर्वं कथय रामाय यथा मे जायते दया।\nopagebreak\\
मासद्वयावधि प्राणाः स्थास्यन्ति मम सत्तम॥}\nopagebreak\\
\raggedleft{–~अ॰रा॰~५.३.४०}\\
\begin{sloppypar}\hyphenrules{nohyphenation}\justifying\noindent\hspace{10mm} अत्र \textcolor{red}{मयि} इति प्रयोगस्त्वर्थानुकूलः किन्तु \textcolor{red}{उपरि} इत्यध्याहारे सम्बन्ध\-विवक्षायां \textcolor{red}{मे} इत्यपि। यद्वा \textcolor{red}{मे} इति \textcolor{red}{मह्यम्‌} इति चतुर्थी।\footnote{\textcolor{red}{तेमयावेकवचनस्य} (पा॰सू॰~८.१.२२) इत्यनेन।} सा च \textcolor{red}{मामुद्धर्तुं दया जायते मे हिताय वा दया जायताम्‌} इत्युभयथाऽपि तुमुन्कर्मणि\footnote{\textcolor{red}{क्रियार्थोपपदस्य च कर्मणि स्थानिनः} (वा॰~२.३.१४) इत्यनेन।} हित\-योगे\footnote{\textcolor{red}{हित\-योगे च} (वा॰~२.३.१३) इत्यनेन।} वा चतुर्थी।\end{sloppypar}
\section[देव्यै]{देव्यै}
\centering\textcolor{blue}{श्रुत्वा तद्वचनं देव्यै पूर्वरूपमदर्शयत्।\nopagebreak\\
मेरुमन्दरसङ्काशं रक्षोगणविभीषणम्॥}\nopagebreak\\
\raggedleft{–~अ॰रा॰~५.३.६४}\\
\begin{sloppypar}\hyphenrules{nohyphenation}\justifying\noindent\hspace{10mm} अत्र \textcolor{red}{देवीं विश्वासयितुं पूर्व\-रूपमदर्शयत्‌} इति \textcolor{red}{क्रियार्थोपपदस्य च कर्मणि स्थानिनः} (पा॰सू॰~२.३.१४) इत्यनेनाप्रयुज्यमान\-तुमुन्कर्मणि चतुर्थी। \textcolor{red}{हित}शब्दस्याध्याहारे हित\-योगा वा।\footnote{\textcolor{red}{हित\-योगे च} (वा॰~२.३.१३) इत्यनेन।}\end{sloppypar}
\section[महाप्रियम्]{महाप्रियम्‌}
\centering\textcolor{blue}{प्रासादरक्षिणः सर्वान्हत्वा तत्रैव तस्थिवान्।\nopagebreak\\
तच्छ्रुत्वा तूर्णमुत्थाय वनभङ्गं महाप्रियम्॥}\nopagebreak\\
\raggedleft{–~अ॰रा॰~५.३.७७}\\
\begin{sloppypar}\hyphenrules{nohyphenation}\justifying\noindent\hspace{10mm} अत्र \textcolor{red}{महत् अप्रियम् इति महाप्रियम्‌} इति विग्रहे समासः। \textcolor{red}{सन्महत्परमोत्तमोत्कृष्टाः पूज्यमानैः} (पा॰सू॰~२.१.६१) इति सूत्रेण कर्मधारयः। \textcolor{red}{आन्महतः समानाधिकरणजातीययोः} (पा॰सू॰~६.३.४६) इत्यनेनाऽकारादेशः। \textcolor{red}{महाप्रियम्‌} इति \textcolor{red}{महाखलः} इतिवत्।\end{sloppypar}
\section[राघवे]{राघवे}
\centering\textcolor{blue}{राक्षसीनां तर्जनैस्तत्सर्वं कथय राघवे।\nopagebreak\\
मयोक्तं देवि रामोऽपि त्वच्चिन्तापरिनिष्ठितः॥}\nopagebreak\\
\raggedleft{–~अ॰रा॰~५.५.४९}\\
\begin{sloppypar}\hyphenrules{nohyphenation}\justifying\noindent\hspace{10mm} अत्र \textcolor{red}{कथ्‌}\-धातु\-प्रयोगात् (\textcolor{red}{कथँ वाक्य\-प्रबन्धने} धा॰पा॰~१८५१) द्वितीयोचितैव किन्तु \textcolor{red}{राघवे शृण्वति कथय} इत्यध्याहारे सति सप्तम्यपि सिद्धान्तानुकूला।\footnote{\textcolor{red}{यस्य च भावेन भाव\-लक्षणम्‌} (पा॰सू॰~२.३.३७) इत्यनेन।}\end{sloppypar}
\vspace{2mm}
\centering ॥ इति सुन्दरकाण्डीयप्रयोगाणां विमर्शः ॥\nopagebreak\\
\vspace{4mm}
\pdfbookmark[2]{युद्धकाण्डम्‌}{Chap1Part2Kanda6}
\phantomsection
\addtocontents{toc}{\protect\setcounter{tocdepth}{2}}
\addcontentsline{toc}{subsection}{युद्धकाण्डीयप्रयोगाणां विमर्शः}
\addtocontents{toc}{\protect\setcounter{tocdepth}{0}}
\centering ॥ अथ युद्धकाण्डीयप्रयोगाणां विमर्शः ॥\nopagebreak\\
\section[हनूमन्तम्]{हनूमन्तम्‌}
\centering\textcolor{blue}{आययुश्चानुपूर्व्येण समुद्रं भीमनिःस्वनम्।\nopagebreak\\
अवतीर्य हनूमन्तं रामः सुग्रीवसंयुतः॥}\nopagebreak\\
\raggedleft{–~अ॰रा॰~६.१.४२}\\
\begin{sloppypar}\hyphenrules{nohyphenation}\justifying\noindent\hspace{10mm} अत्र लङ्कां प्रतिष्ठमानो भगवाञ्छ्रीरामो हनूमतोऽवतीर्य समुद्र\-वेलायां सैन्यं निवेशयति। राम\-विश्लेषस्य ध्रुवत्वेनावधि\-भूतत्वात् \textcolor{red}{हनूमन्तम्‌} इति द्वितीया पाणिनीय\-विरुद्धेव। \textcolor{red}{ध्रुवमपायेऽपादानम्‌} (पा॰सू॰~१.४.२४) इत्यनेनापादान\-सञ्ज्ञायां पञ्चम्येव। किन्तु \textcolor{red}{हनूमन्तं त्यक्त्वा अवतीर्य} इत्यध्याहारेण \textcolor{red}{त्यज्‌}\-कर्मतया (\textcolor{red}{त्यजँ हानौ} धा॰पा॰~९८६) द्वितीया। यद्वा \textcolor{red}{दुह्याच्‌} (वै॰सि॰कौ॰~५३९) इति परिगणनस्य शब्दरत्नादौ
खण्डनेनापादानस्याविवक्षायां \textcolor{red}{वृक्षं पुष्पं चिनोति}\footnote{\textcolor{red}{वृक्षमवचिनोति फलानि} (भा॰पा॰सू॰~१.४.५१) इति महाभाष्य उदाहृतः।} 
इत्यादिवद्द्वितीया।
\textcolor{red}{अकथितं च} (पा॰सू॰~१.४.५१) इत्यनेन कर्म\-सञ्ज्ञा\-बलाद्द्वितीया सङ्गतेति पाणिनीयता।\end{sloppypar}
\section[रघुनायकस्य]{रघुनायकस्य}
\centering\textcolor{blue}{यावन्न रामस्य शिताः शिलीमुखा लङ्कामभिव्याप्य शिरांसि रक्षसाम्।\nopagebreak\\
छिन्दन्ति तावद्रघुनायकस्य भोस्तां जानकीं त्वं प्रतिदातुमर्हसि॥}\nopagebreak\\
\raggedleft{–~अ॰रा॰~६.२.२४}\\
\begin{sloppypar}\hyphenrules{nohyphenation}\justifying\noindent\hspace{10mm} अत्र यद्यपि \textcolor{red}{दा}\-धातु\-योगे (\textcolor{red}{डुदाञ् दाने} धा॰पा॰~१०९१) चतुर्थी पाणिनीया किन्तु प्रतीत्युपसर्ग\-योजनतया विनिमय\-रूपेऽर्थे जाते सम्प्रदानताभावे षष्ठी\-साहित्यम्। \textcolor{red}{प्रतिदातुम्‌} इत्यनेन विनिमय\-द्योतनात्। यद्वा \textcolor{red}{रघुनायकस्य} इत्यस्य \textcolor{red}{जानकीम्‌} इत्यनेनान्वये दाम्पत्य\-भाव\-सम्बन्धे षष्ठी। \end{sloppypar}
\section[मे]{मे}
\centering\textcolor{blue}{मन्त्रिभिः सायुधैरस्मान् विवरे निहनिष्यति।\nopagebreak\\
तदाज्ञापय मे देव वानरैर्हन्यतामयम्॥}\nopagebreak\\
\raggedleft{–~अ॰रा॰~६.३.८}\\
\begin{sloppypar}\hyphenrules{nohyphenation}\justifying\noindent\hspace{10mm} अत्र सुग्रीवो विभीषणं प्रत्याशङ्कमानः प्राह \textcolor{red}{मे आज्ञापय}। \textcolor{red}{मां पोषयितुमाज्ञापय} इति चतुर्थी।\footnote{\textcolor{red}{क्रियार्थोपपदस्य च कर्मणि स्थानिनः} (पा॰सू॰~२.३.१४) इत्यनेन।} यद्वा कारकाणां बुद्धि\-कल्पितत्वात्\footnote{यथा \textcolor{red}{अयं योगः शक्योऽवक्तुम्। ... स बुद्ध्या सम्प्राप्य निवर्तयति। तत्र “ध्रुवमपायेऽपादानम्” इत्येव सिद्धम्‌} (भा॰पा॰सू॰~१.४.२५)। \textcolor{red}{अयमपि योगः शक्योऽवक्तुम्। ... स बुद्ध्या सम्प्राप्य निवर्तते। तत्र “ध्रुवमपायेऽपादानम्” इत्येव सिद्धम्‌} (भा॰पा॰सू॰~१.४.२६)। \textcolor{red}{अयमपि योगः शक्योऽवक्तुम्। ... स बुद्ध्या सम्प्राप्य निवर्तयति। तत्र “ध्रुवमपायेऽपादानम्” इत्येव सिद्धम्‌} (भा॰पा॰सू॰~१.४.२७)। \textcolor{red}{अयमपि योगः शक्योऽवक्तुम्। ... स बुद्ध्या सम्प्राप्य निवर्तते। तत्र “ध्रुवमपायेऽपादानम्” इत्येव सिद्धम्‌} (भा॰पा॰सू॰~१.४.२८)। \textcolor{red}{अयमपि योगः शक्योऽवक्तुम्‌} (भा॰पा॰सू॰~१.४.२९)। \textcolor{red}{अयमपि योगः शक्योऽवक्तुम्‌} (भा॰पा॰सू॰~१.४.३०)। \textcolor{red}{अयमपि योगः शक्योऽवक्तुम्‌} (भा॰पा॰सू॰~१.४.३१) इत्यादिषु स्पष्टम्।} \textcolor{red}{दा}\-धातुं (\textcolor{red}{डुदाञ् दाने} धा॰पा॰~१०९१) विनाऽपि सम्प्रदाने चतुर्थी।\footnote{\textcolor{red}{चतुर्थी सम्प्रदाने} (पा॰सू॰~२.३.१३) इत्यनेन।}\end{sloppypar}
\section[रघूणां पतये]{रघूणां पतये}
\centering\textcolor{blue}{नमोऽनन्ताय शान्ताय रामायामिततेजसे।\nopagebreak\\
सुग्रीवमित्राय च ते रघूणां पतये नमः॥}\nopagebreak\\
\raggedleft{–~अ॰रा॰~६.३.१८}\\
\begin{sloppypar}\hyphenrules{nohyphenation}\justifying\noindent\hspace{10mm} अत्र शरणागत\-विभीषणो रघु\-कुल\-भूषणं सुग्रीव\-लक्ष्मणाभिरामं रामं समीडानो \textcolor{red}{रघूणां पतये नमः} इति प्रयुङ्क्ते। अत्र समासाभावे \textcolor{red}{रघूणां पत्ये} इत्येव पाणिनीयं \textcolor{red}{पतिः समास एव} (पा॰सू॰~१.४.८) इत्यनेनासमास\-सञ्ज्ञा\-निषेधात् \textcolor{red}{घेर्ङिति} (पा॰सू॰~७.३.१११) इति गुणाभावेऽयादेशाभावे च \textcolor{red}{रघूणां पतये} इति कथम्। परं \textcolor{red}{पतिरिवाऽचरतीति पतिः}। आचारे \textcolor{red}{क्विप्}।\footnote{पति~\arrow \textcolor{red}{सर्वप्राति\-पदिकेभ्य आचारे क्विब्वा वक्तव्यः} (वा॰~३.१.११)~\arrow पति~क्विँप्~\arrow पति~व्~\arrow \textcolor{red}{वेरपृक्तस्य} (पा॰सू॰~६.१.६७)~\arrow पति~\arrow \textcolor{red}{सनाद्यन्ता धातवः} (पा॰सू॰~३.१.३२)~\arrow धातुसञ्ज्ञा~\arrow \textcolor{red}{क्विप् च} (पा॰सू॰~३.२.७६)~\arrow पति~क्विँप्~\arrow पति~व्~\arrow \textcolor{red}{वेरपृक्तस्य} (पा॰सू॰~६.१.६७)~\arrow पति~\arrow विभक्ति\-कार्यम्~\arrow पतिः।} ततः सर्वापहारि\-लोपे \textcolor{red}{लक्षण\-प्रतिपदोक्तयोः प्रतिपदोक्तस्यैव ग्रहणम्‌}\footnote{\textcolor{red}{सिद्धं तु लक्षण\-प्रतिपदोक्तयोः प्रतिपदोक्तस्यैव ग्रहणात्} (वा॰~६.२.२)।} इति परिभाषया लाक्षणिक\-पति\-शब्दे \textcolor{red}{पतिः समास एव} (पा॰सू॰~१.४.८) इति सूत्रं न प्रवर्तते। अतो घि\-सञ्ज्ञायां गुणेऽयादेशे \textcolor{red}{रघूणां पतये} इति।\footnote{\pageref{sec:patina}तमे पृष्ठे \ref{sec:patina} \nameref{sec:patina} इति प्रयोगस्य विमर्शमपि पश्यन्तु।} यद्वा \textcolor{red}{अपां पतये नमः}, \textcolor{red}{तस्क॑राणां॒ पत॑ये॒ नमः॒} (कृ॰य॰ तै॰सं॰~४.५.३.१) इति छान्दस\-प्रयोग इव विभीषणेनाप्यत्र
ब्राह्मणत्व\-प्रदिदर्शयिषया भगवतो रामचन्द्रस्य छन्दोमयत्वाच्छान्दसः प्रयोगः कृतः। \textcolor{red}{षष्ठीयुक्तश्छन्दसि वा} (पा॰सू॰~१.४.९) इति सूत्रेण। \textcolor{red}{षष्ठीयुक्तश्छन्दसि वा} इत्यत्र \textcolor{red}{वा} ग्रहणेन लोकेऽपि। अथवा षष्ठ्याऽलुक्समासे चोक्त\-प्रयोगः पाणिनीयः \textcolor{red}{बहुलं छन्दसि} (पा॰सू॰~२.४.७३) इति सूत्रेण \textcolor{red}{सुपो धातु\-प्रातिपदिकयोः} (पा॰सू॰~२.४.७१) इति सूत्राप्रवृत्तेः। अथवा \textcolor{red}{तत्पुरुषे कृति बहुलम्‌} (पा॰सू॰~६.३.१४) इति सूत्रेण पूर्व\-साधित\-कृदन्त\-\textcolor{red}{पति}\-शब्दे परेऽलुकि \textcolor{red}{रघूणां पतये} इत्यपि पाणिनीयम्।\footnote{मण्डूक\-प्लुत्या \textcolor{red}{हलदन्तात्सप्तम्याः संज्ञायाम्‌} (पा॰सू॰~६.३.९) इत्यस्मादनुवृत्तस्य \textcolor{red}{सप्तम्याः} इति पदस्य \textcolor{red}{तत्पुरुषे कृति बहुलम्‌} (पा॰सू॰~६.३.१४) इत्यत्र निवृत्तौ षष्ठ्या अप्यलुक्। यद्वा \textcolor{red}{बहुलम्‌} इत्यस्य ग्रहणेन षष्ठ्या अप्यलुक्।}\end{sloppypar}
\section[विभीषणे]{विभीषणे}
\centering\textcolor{blue}{सीतां प्रयच्छ रामाय राज्यं देहि विभीषणे।\nopagebreak\\
वनं याहि महाबाहो रम्यं मुनिगणाश्रयम्॥}\nopagebreak\\
\raggedleft{–~अ॰रा॰~६.६.४६}\\
\begin{sloppypar}\hyphenrules{nohyphenation}\justifying\noindent\hspace{10mm} अत्र चिर\-जीवित्वादाधार\-विवक्षयाऽधिकरण\-सप्तम्यपि पाणिनीया।\footnote{\textcolor{red}{अश्वत्थामा बलिर्व्यासो हनूमांश्च विभीषणः। कृपः परशुरामश्च सप्तैते चिरजीविनः॥} (आ॰रा॰~९.७.११८)।}\end{sloppypar}
\section[परैः]{परैः}
\centering\textcolor{blue}{निहन्मि त्वां दुरात्मानं मच्छासनपराङ्मुखम्।\nopagebreak\\
परैः किञ्चिद्गृहीत्वा त्वं भाषसे रामकिंकरः॥}\nopagebreak\\
\raggedleft{–~अ॰रा॰~६.७.२}\\
\begin{sloppypar}\hyphenrules{nohyphenation}\justifying\noindent\hspace{10mm} अत्र ग्रहणस्यावधि\-भूतत्वात् \textcolor{red}{परेभ्यः किञ्चिद्गृहीत्वा} इति वक्तव्यं तथाऽपि \textcolor{red}{परैः दीयमानम्‌} इत्यध्याहारे कर्तरि तृतीया। उताहो \textcolor{red}{परैः} इति करणे तृतीया। दान\-क्रियायां पर\-दानस्याऽपि करणत्वात्।\end{sloppypar}
\section[वायुसूनोः]{वायुसूनोः}
\centering\textcolor{blue}{हतस्यापि शरैस्तीक्ष्णैर्वायुसूनोः स्वतेजसा।\nopagebreak\\
व्यवर्धत पुनस्तेजो ननर्द च महाकपिः॥}\nopagebreak\\
\raggedleft{–~अ॰रा॰~६.६.२४}\\
\begin{sloppypar}\hyphenrules{nohyphenation}\justifying\noindent\hspace{10mm} अत्र कर्मणि सम्बन्ध\-विवक्षायां षष्ठी।\footnote{\textcolor{red}{षष्ठी शेषे} (पा॰सू॰~२.३.५०) इत्यनेन।}\end{sloppypar}
\section[लक्ष्मणाय]{लक्ष्मणाय}
\centering\textcolor{blue}{चिकित्सां कारयामास लक्ष्मणाय महात्मने।\nopagebreak\\
ततः सुप्तोत्थित इव बुद्ध्वा प्रोवाच लक्ष्मण॥}\nopagebreak\\
\raggedleft{–~अ॰रा॰~६.७.३७}\\
\begin{sloppypar}\hyphenrules{nohyphenation}\justifying\noindent\hspace{10mm} \textcolor{red}{लक्ष्मणस्य महात्मनः} इति सम्बन्ध\-विवक्षया षष्ठ्युचिता किन्तु \textcolor{red}{लक्ष्मणं महात्मानं जीवयितुम्‌} इत्यप्रयुज्यमान\-तुमुन्कर्मणि चतुर्थी।\footnote{\textcolor{red}{क्रियार्थोपपदस्य च कर्मणि स्थानिनः} (पा॰सू॰~२.३.१४) इत्यनेन।}\end{sloppypar}
\section[रघूत्तमे]{रघूत्तमे}
\centering\textcolor{blue}{इतः परं वा वैदेहीं प्रेषयस्व रघूत्तमे।\nopagebreak\\
विभीषणाय राज्यं तु दत्त्वा गच्छामहे वनम्॥}\nopagebreak\\
\raggedleft{–~अ॰रा॰~६.१०.५४}\\
\begin{sloppypar}\hyphenrules{nohyphenation}\justifying\noindent\hspace{10mm} अत्राप्याधार\-विवक्षायां सप्तमी।\footnote{\textcolor{red}{आधारोऽधिकरणम्} (पा॰सू॰~१.४.४५) \textcolor{red}{सप्तम्यधिकरणे च} (पा॰सू॰~२.३.३६) इत्याभ्याम्।} रामस्याधारत्वं पूर्वं निरूपितम्।\footnote{\pageref{sec:raaghave_3_5_36}तमे पृष्ठे \ref{sec:raaghave_3_5_36} \nameref{sec:raaghave_3_5_36} इति प्रयोगस्य विमर्शं पश्यन्तु।}\end{sloppypar}
\section[कारुण्यभाजनाः]{कारुण्यभाजनाः}
\centering\textcolor{blue}{वयं तु सात्त्विका देवा विष्णोः कारुण्यभाजनाः।\nopagebreak\\
भयदुःखादिभिर्व्याप्ताः संसारे परिवर्तिनः॥}\nopagebreak\\
\raggedleft{–~अ॰रा॰~६.११.८०}\\
\begin{sloppypar}\hyphenrules{nohyphenation}\justifying\noindent\hspace{10mm} \textcolor{red}{कारुण्य\-भाजनमेषाम्‌} इत्यर्शआदित्वादच्।\footnote{\textcolor{red}{अर्शआदिभ्योऽच्‌} (पा॰सू॰~५.२.१२७) इत्यनेन।} अथवा \textcolor{red}{भजन्त इति भजनाः} कर्तरि ल्युट्।\footnote{\textcolor{red}{कृत्यल्युटो बहुलम्‌} (पा॰सू॰~३.३.११३) इत्यनेन।} \textcolor{red}{भजना एव भाजनाः} इति प्रज्ञादित्वादण्।\footnote{\textcolor{red}{प्रज्ञादिभ्यश्च} (पा॰सू॰~५.४.३८) इत्यनेन।} \textcolor{red}{कारुण्यस्य भाजनाः} इति षष्ठी\-समासः।\footnote{\textcolor{red}{कृद्योगा च षष्ठी समस्यत इति वक्तव्यम्‌} (वा॰~२.२.८) इत्यनेन। यद्वा \textcolor{red}{कारुण्यस्य भाजनं येषाम्‌} इति बहुव्रीहिः।} अतो न लिङ्ग\-दोषः।\end{sloppypar}
\section[शिबिकोत्तमे]{शिबिकोत्तमे}
\centering\textcolor{blue}{सर्वाभरणसम्पन्नामारोप्य शिबिकोत्तमे।\nopagebreak\\
याष्टीकैर्बहुभिर्गुप्तां कञ्चुकोष्णीषिभिः शुभाम्॥}\nopagebreak\\
\raggedleft{–~अ॰रा॰~६.१२.७०}\\
\begin{sloppypar}\hyphenrules{nohyphenation}\justifying\noindent\hspace{10mm} अस्याप्यत्र नपुंसक\-लिङ्गे पाठात् \textcolor{red}{शिबिकोत्तमायाम्‌} इत्यनुक्त्वेदं कथमुक्तमिति न भ्रमितव्यम्।\footnote{\textcolor{red}{लिङ्गमशिष्यं लोकाश्रयत्वाल्लिङ्गस्य} (भा॰पा॰सू॰~२.१.३६) इत्यनेन लिङ्गं शिष्ट\-प्रयोगाधीनमिति भावः। वाल्मीकीय\-रामायणे भारते तु \textcolor{red}{शिबिका} इति स्त्रीलिङ्ग एव प्रयोगः।}\end{sloppypar}
\section[देवताभ्यः]{देवताभ्यः}
\centering\textcolor{blue}{पश्यतां सर्वलोकानां देवराक्षसयोषिताम्।\nopagebreak\\
प्रणम्य देवताभ्यश्च ब्राह्मणेभ्यश्च मैथिली॥}\nopagebreak\\
\raggedleft{–~अ॰रा॰~६.१२.८०}\\
\begin{sloppypar}\hyphenrules{nohyphenation}\justifying\noindent\hspace{10mm} \textcolor{red}{देवता अनुकूलयितुं प्रसादयितुं वा प्रणम्य} इति तुमुन्कर्मणि चतुर्थी।\footnote{\textcolor{red}{क्रियार्थोपपदस्य च कर्मणि स्थानिनः} (पा॰सू॰~२.३.१४) इत्यनेन। एवमेव \textcolor{red}{ ब्राह्मणेभ्यः} इत्यत्रापि बोध्यम्।}\end{sloppypar}
\section[गन्धर्वाप्सरसोरगाः]{गन्धर्वाप्सरसोरगाः}
\centering\textcolor{blue}{ततः शक्रः सहस्राक्षो यमश्च वरुणस्तथा।\nopagebreak\\
कुबेरश्च महातेजाः पिनाकी वृषवाहनः॥\\
ब्रह्मा ब्रह्मविदां श्रेष्ठो मुनिभिः सिद्धचारणैः।\nopagebreak\\
ऋषयः पितरः साध्या गन्धर्वाप्सरसोरगाः॥\\
एते चान्ये विमानाग्र्यैराजग्मुर्यत्र राघवः।\nopagebreak\\
अब्रुवन् परमात्मानं रामं प्राञ्जलयश्च ते॥}\nopagebreak\\
\raggedleft{–~अ॰रा॰~६.१३.१-३}\\
\begin{sloppypar}\hyphenrules{nohyphenation}\justifying\noindent\hspace{10mm} हलन्त\-\textcolor{red}{अप्सरस्‌} इति पाठे तु \textcolor{red}{गन्धर्वाप्सरउरगाः} इत्येव।\footnote{गन्धर्वाश्चाप्सरसश्चोरगाश्चेति गन्धर्वाप्सरउरगाः। गन्धर्व~जस् अप्सरस्~जस् उरग~जस्~\arrow \textcolor{red}{चार्थे द्वन्द्वः} (पा॰सू॰~२.२.२९)~\arrow \textcolor{red}{सुपो धातु\-प्रातिपदिकयोः} (पा॰सू॰~२.४.७१)~\arrow गन्धर्व~अप्सरस्~उरग~\arrow \textcolor{red}{अकः सवर्णे दीर्घः} (पा॰सू॰~६.१.१०१)~\arrow गन्धर्वाप्सरस्~उरग~\arrow \textcolor{red}{ससजुषो रुः} (पा॰सू॰~८.२.६६)~\arrow गन्धर्वाप्सररुँ~उरग~\arrow \textcolor{red}{भो भोभगोअघोअपूर्वस्य योऽशि} (पा॰सू॰~८.३.१७)~\arrow गन्धर्वाप्सरय्~उरग~\arrow \textcolor{red}{लोपः शाकल्यस्य} (पा॰सू॰~८.३.१९)~\arrow गन्धर्वाप्सर~उरग~\arrow गन्धर्वाप्सरउरग~\arrow विभक्तिकार्यम्~\arrow गन्धर्वाप्सरउरग~जस्~\arrow गन्धर्वाप्सरउरग~अस्~\arrow \textcolor{red}{प्रथमयोः पूर्वसवर्णः} (पा॰सू॰~६.१.१०२)~\arrow गन्धर्वाप्सरउरगास्~\arrow \textcolor{red}{ससजुषो रुः} (पा॰सू॰~८.२.६६)~\arrow \textcolor{red}{खरवसानयोर्विसर्जनीयः} (पा॰सू॰~८.३.१५)~\arrow गन्धर्वाप्सरउरगाः।} अजन्ते \textcolor{red}{गन्धर्वाप्सरसोरगाः}।\footnote{\textcolor{red}{सर्वे सान्ता अदन्ताः स्युः} इत्युक्तेः \textcolor{red}{अप्सर} इति शब्दोऽपि स्वीकरणीय इति भावः। \textcolor{red}{अप्सर}\-शब्दस्य प्रयोगः स्कन्द\-पुराणेऽवन्ती\-खण्डे रेवाखण्डे कुसुमेश्वर\-तीर्थ\-माहात्म्य\-वर्णने कृतोऽस्ति~– \textcolor{red}{वसन्तमासे कुसुमाकराकुले मयूरदात्यूह\-सुकोकिलाकुले। प्रनृत्य\-देवाप्सरगीत\-सङ्कुले प्रवाति वाते यमनैरृताकुले॥} (स्क॰पु॰~रे॰ख॰~१५०.१४)। अत्र \textcolor{red}{प्रनृत्य\-देवाप्सरगीत\-सङ्कुले} इत्यत्र \textcolor{red}{अप्सर} इत्येव शब्दः। एवमेवात्र \textcolor{red}{अप्सर}\-शब्द\-स्वीकारे \textcolor{red}{स}\-शब्दः सर्पे (\textcolor{red}{“सः स्याद्विष्णौ हरे सर्पे” इति भरतैकार्थ\-सङ्ग्रहः} इति शब्द\-कल्प\-द्रुमः) \textcolor{red}{उरग}\-शब्दश्च नागे स्वीकरणीयः। \textcolor{red}{देवगन्धर्व\-मानुषोरग\-राक्षसान्} (न॰उ॰~१.२९) \textcolor{red}{गन्धर्वोरग\-रक्षसाम्} (म॰स्मृ॰~३.१९६) इत्यादिषु \textcolor{red}{उरग}\-शब्दो नागार्थ इत्याप्टे\-कोशः। अत्र पर्याययोः कथं पृथगुपादानमिति न भ्रमितव्यम्। सर्प\-नागयोरीषदन्तरम्। अत एव गीतायां \textcolor{red}{सर्पाणामस्मि वासुकिः} (भ॰गी॰~१०.२८) इत्युक्त्वाऽपि \textcolor{red}{अनन्तश्चास्मि नागानाम्} (भ॰गी॰~१०.२९) इत्युक्तं भगवता। नागा नागलोक\-वासिनो मानव\-मुखाः सर्पजाति\-विशेषाः। \textcolor{red}{सर्पा एकशिरसः} (भ॰गी॰ रा॰भा॰~१०.२९) \textcolor{red}{नागा बहुशिरसः} (भ॰गी॰ रा॰भा॰~१०.२९) इति भगवन्तो रामानुजाचार्या गीताभाष्ये। एवं तर्हि गन्धर्व~जस् अप्सर~जस् स~जस् उरग~जस्~\arrow \textcolor{red}{चार्थे द्वन्द्वः} (पा॰सू॰~२.२.२९)~\arrow \textcolor{red}{सुपो धातु\-प्रातिपदिकयोः} (पा॰सू॰~२.४.७१)~\arrow गन्धर्व~अप्सर~स~उरग~\arrow \textcolor{red}{अकः सवर्णे दीर्घः} (पा॰सू॰~६.१.१०१)~\arrow गन्धर्वाप्सर~स~उरग~\arrow \textcolor{red}{आद्गुणः} (पा॰सू॰~६.१.८७)~\arrow गन्धर्वाप्सर~सोरग~\arrow गन्धर्वाप्सरसोरग~\arrow विभक्तिकार्यम्~\arrow गन्धर्वाप्सरसोरग~जस्~\arrow गन्धर्वाप्सरसोरग~अस्~\arrow \textcolor{red}{प्रथमयोः पूर्वसवर्णः} (पा॰सू॰~६.१.१०२)~\arrow गन्धर्वाप्सरसोरगास्~\arrow \textcolor{red}{ससजुषो रुः} (पा॰सू॰~८.२.६६)~\arrow \textcolor{red}{खरवसानयोर्विसर्जनीयः} (पा॰सू॰~८.३.१५)~\arrow गन्धर्वाप्सरसोरगाः।}
अथवा \textcolor{red}{द्वन्द्वाच्चुदषहान्तात्समाहारे} (पा॰सू॰~५.४.१०६) इत्यत्र \textcolor{red}{द्वन्द्वात्} इति योग\-विभागेन \textcolor{red}{टच्‌}\-प्रत्ययः।\footnote{योगविभागे \textcolor{red}{चुदषहान्तात् समाहारे} इत्यनयोर्मोषे सान्तादितरेतर\-द्वन्द्वादपि कुत्रचिदिति भावः।}
न चायमन्ते करोति। तर्हि पूर्वं \textcolor{red}{गन्धर्वाप्सरसाः} इति खण्डवाक्यमेकं पश्चात् \textcolor{red}{उरग}\-शब्देन समासः।\footnote{यद्वाऽत्र न समासोऽपि तु \textcolor{red}{गन्धर्वाप्सरसः} इति पृथक् \textcolor{red}{उरगाः} इति च पृथक्। ततः संहितायां \textcolor{red}{ससजुषो रुः} (पा॰सू॰~८.२.६६) इत्यनेन रुत्वे \textcolor{red}{भो भोभगोअघोअपूर्वस्य योऽशि} (पा॰सू॰~८.३.१७) इत्यनेन यत्वे \textcolor{red}{लोपः शाकल्यस्य} (पा॰सू॰~८.३.१९) इत्यनेन यलोपे गुणे प्राप्ते त्रिपादीत्वाल्लोप\-कार्यस्यासिद्धत्वे सामान्यतः \textcolor{red}{गन्धर्वाप्सरस उरगाः} इत्येव। परन्त्वत्र \textcolor{red}{न मु ने} (पा॰सू॰~८.२.३) इत्यनेन \textcolor{red}{पूर्वत्रासिद्धम्‌} (पा॰सू॰~८.२.१) इति सूत्रे निराकृते सिद्धे यलोपे \textcolor{red}{आद्गुणः} (पा॰सू॰~६.१.८७) इत्यनेन गुणे \textcolor{red}{गन्धर्वाप्सरसोरगाः} इति सिद्धम्। विस्तराय \pageref{sec:jaayeti_siiteti}तमे पृष्ठे \ref{sec:jaayeti_siiteti} \nameref{sec:jaayeti_siiteti} इति प्रयोगस्य विमर्शे पश्यन्तु।}\end{sloppypar}
\vspace{2mm}
\centering ॥ इति युद्धकाण्डीयप्रयोगाणां विमर्शः ॥\nopagebreak\\
\vspace{4mm}
\pdfbookmark[2]{उत्तरकाण्डम्‌}{Chap1Part2Kanda7}
\phantomsection
\addtocontents{toc}{\protect\setcounter{tocdepth}{2}}
\addcontentsline{toc}{subsection}{उत्तरकाण्डीयप्रयोगाणां विमर्शः}
\addtocontents{toc}{\protect\setcounter{tocdepth}{0}}
\centering ॥ अथोत्तरकाण्डीयप्रयोगाणां विमर्शः ॥\nopagebreak\\
\section[मम]{मम}
\centering\textcolor{blue}{श्रृणु राम यथा वृत्तं रावणे रावणस्य च।\nopagebreak\\
जन्म कर्म वरादानं सङ्क्षेपाद्वदतो मम॥}\nopagebreak\\
\raggedleft{–~अ॰रा॰~७.१.२४}\\
\begin{sloppypar}\hyphenrules{nohyphenation}\justifying\noindent\hspace{10mm} अत्र पञ्चम्या भवितव्यमासीत्\footnote{\textcolor{red}{अपादाने पञ्चमी} (पा॰सू॰~२.३.२८) इत्यनेन। अत्र शब्दविश्लेषादपादानत्वम्। यद्वा \textcolor{red}{आख्यातोपयोगे} (पा॰सू॰~१.४.२९) इत्यनेनापादानत्वम्।} किन्तु \textcolor{red}{वदतो मम} इति भाव\-लक्षणा षष्ठी \textcolor{red}{यस्य च भावेन भाव\-लक्षणम्‌} (पा॰सू॰~२.३.३७) इति सूत्रेण।\footnote{\textcolor{red}{दूरान्तिकार्थैः षष्ठ्यन्यतरस्याम्‌} (पा॰सू॰~२.३.३४) इत्यतः \textcolor{red}{षष्ठी} इत्यनुवर्त्य \textcolor{red}{षष्ठी चानादरे} (पा॰सू॰~२.३.३८) इत्यतः \textcolor{red}{षष्ठी} इत्यपकृष्य वाऽऽदरेऽपि \textcolor{red}{यस्य च भावेन भाव\-लक्षणम्‌} (पा॰सू॰~२.३.३७) इत्यनेन भावलक्षणा षष्ठीति भावः।}\end{sloppypar}
\section[कालस्य]{कालस्य}
\centering\textcolor{blue}{स तत्र सुचिरं कालमुवास पितृसम्मतः।\nopagebreak\\
कस्यचित्त्वथ कालस्य सुमाली नाम राक्षसः॥}\nopagebreak\\
\raggedleft{–~अ॰रा॰~७.१.४५}\\
\begin{sloppypar}\hyphenrules{nohyphenation}\justifying\noindent\hspace{10mm} \textcolor{red}{आगतस्य} इति \textcolor{red}{व्यतीतस्य} वेत्यध्याहारे भाव\-लक्षणा षष्ठी।\footnote{\textcolor{red}{दूरान्तिकार्थैः षष्ठ्यन्यतरस्याम्‌} (पा॰सू॰~२.३.३४) इत्यतः \textcolor{red}{षष्ठी} इत्यनुवर्त्य \textcolor{red}{षष्ठी चानादरे} (पा॰सू॰~२.३.३८) इत्यतः \textcolor{red}{षष्ठी} इत्यपकृष्य वाऽऽदरेऽपि \textcolor{red}{यस्य च भावेन भाव\-लक्षणम्‌} (पा॰सू॰~२.३.३७) इत्यनेन भावलक्षणा षष्ठीति भावः।}\end{sloppypar}
\section[जगत्त्रयम्]{जगत्त्रयम्‌}
\centering\textcolor{blue}{भगवन्ब्रूहि मे योद्धुं कुत्र सन्ति महाबलाः।\nopagebreak\\
योद्धुमिच्छामि बलिभिस्त्वं ज्ञाताऽसि जगत्त्रयम्॥}\nopagebreak\\
\raggedleft{–~अ॰रा॰~७.४.२}\\
\begin{sloppypar}\hyphenrules{nohyphenation}\justifying\noindent\hspace{10mm} अत्र \textcolor{red}{ज्ञाता} इति तृन्प्रत्ययान्तः।\footnote{\textcolor{red}{तृन्} (पा॰सू॰~३.२.१३५) इत्यनेन ताच्छील्ये ताद्धर्म्ये वा \textcolor{red}{तृन्‌}\-प्रत्ययः।} तृन्प्रत्ययान्त\-प्रयोगात् \textcolor{red}{जगत्त्रयम्‌} इत्यत्र द्वितीया पाणिनीया।\footnote{\textcolor{red}{कर्ता कटान्। वदिता जनापवादान्} (का॰वृ॰~२.३.६९) \textcolor{red}{कर्ता लोकान्} (वै॰सि॰कौ॰~६२७) इतिवत्।} \textcolor{red}{न लोकाव्यय\-निष्ठा\-खलर्थ\-तृनाम्‌} (पा॰सू॰~२.३.६९) इत्यनेन षष्ठी\-निषेधात्।\end{sloppypar}
\section[जगाम ऋषिवाटस्य]{जगाम ऋषिवाटस्य}
\centering\textcolor{blue}{वाल्मीकिरपि सङ्गृह्य गायन्तौ तौ कुशीलवौ।\nopagebreak\\
जगाम ऋषिवाटस्य समीपं मुनिपुङ्गवः॥}\nopagebreak\\
\raggedleft{–~अ॰रा॰~७.६.३६}\\
\begin{sloppypar}\hyphenrules{nohyphenation}\justifying\noindent\hspace{10mm} अत्र संहिताया अविवक्षणान्न सन्धिः।\footnote{यद्वा \textcolor{red}{ऋत्यकः} (पा॰सू॰~६.१.१२८) इत्यनेन शाकल\-प्रकृति\-भावः।}\end{sloppypar}
\section[मह्यम्]{मह्यम्‌}
\centering\textcolor{blue}{एवमेतन्महाप्राज्ञ यथा वदसि सुव्रत।\nopagebreak\\
प्रत्ययो जनितो मह्यं तव वाक्यैरकिल्बिषैः॥}\nopagebreak\\
\raggedleft{–~अ॰रा॰~७.७.३४}\\
\begin{sloppypar}\hyphenrules{nohyphenation}\justifying\noindent\hspace{10mm} \textcolor{red}{मां विश्वासयितुं} प्रत्ययो जनितो \textcolor{red}{मां तोषयितुं} वेति तुमुन्कर्मणि चतुर्थी।\footnote{\textcolor{red}{क्रियार्थोपपदस्य च कर्मणि स्थानिनः} (पा॰सू॰~२.३.१४) इत्यनेन।}\end{sloppypar}
\section[मुनये]{मुनये}
\centering\textcolor{blue}{तस्मै स मुनये रामः पूजां कृत्वा यथाविधि।\nopagebreak\\
पृष्ट्वाऽनामयव्यग्रो रामः पृष्टोऽथ तेन सः॥}\nopagebreak\\
\raggedleft{–~अ॰रा॰~७.८.१५}\\
\begin{sloppypar}\hyphenrules{nohyphenation}\justifying\noindent\hspace{10mm} \textcolor{red}{मुनिमनुकूलयितुम्‌} इति तुमुन्कर्मणि चतुर्थी \textcolor{red}{क्रियार्थोपपदस्य च कर्मणि स्थानिनः} (पा॰सू॰~२.३.१४) इत्यनेन।\end{sloppypar}
\vspace{2mm}
\centering ॥ इत्युत्तरकाण्डीयप्रयोगाणां विमर्शः ॥\nopagebreak\\
\vspace{4mm}
\centering इत्यध्यात्म\-रामायणेऽपाणिनीय\-प्रयोगाणां\-विमर्श\-नामके शोध\-प्रबन्धे प्रथमाध्याये द्वितीय\-परिच्छेदः।\nopagebreak\\
\vspace{4mm}
\centering\textcolor{blue}{\fontsize{16}{24}\selectfont इत्थं महीजारघुवीरलेखकः अध्यात्मरामायणमध्यवर्तिनः।\nopagebreak\\
अपाणिनीयान् स्वधिया विमृश्य वै अध्यायमेतं प्रथमं व्यमर्शयम्॥}\nopagebreak\\
\vspace{4mm}
\centering इत्यध्यात्म\-रामायणेऽपाणिनीय\-प्रयोगाणां\-विमर्श\-नामके शोध\-प्रबन्धे प्रथमोऽध्यायः।
