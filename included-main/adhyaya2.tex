% Nityanand Misra: LaTeX code to typeset a book in Sanskrit
% Copyright (C) 2016 Nityanand Misra
%
% This program is free software: you can redistribute it and/or modify it under
% the terms of the GNU General Public License as published by the Free Software
% Foundation, either version 3 of the License, or (at your option) any later
% version.
%
% This program is distributed in the hope that it will be useful, but WITHOUT
% ANY WARRANTY; without even the implied warranty of  MERCHANTABILITY or FITNESS
% FOR A PARTICULAR PURPOSE. See the GNU General Public License for more details.
%
% You should have received a copy of the GNU General Public License along with
% this program.  If not, see <http://www.gnu.org/licenses/>.

\renewcommand\chaptername{अथ द्वितीयोऽध्यायः}
\chapter[\texorpdfstring{कृत्तद्धितप्रकरणम्}{द्वितीयोऽध्यायः}]{कृत्तद्धितप्रकरणम्}
\vspace{-5mm}
\fontsize{16}{24}\selectfont\centering\textcolor{blue}{नत्वा नीलाम्बुदश्यामं रामं तामरसाननम्।\nopagebreak\\
शोधे गिरिधरः प्रेम्णा द्वितीयाध्यायमारभे॥}\nopagebreak\\
\vspace{4mm}
\fontsize{14}{21}\selectfont\begin{sloppypar}\hyphenrules{nohyphenation}\justifying\noindent\hspace{10mm} अथाध्यात्म\-रामायणे समागतान् कृत्तद्धित\-सम्बन्धिनोऽपाणिनीयान् प्रयोगाननु\-सन्दधे।\end{sloppypar}
\vspace{4mm}
\pdfbookmark[1]{प्रथमः परिच्छेदः}{Chap2Part1}
\phantomsection
\addtocontents{toc}{\protect\setcounter{tocdepth}{1}}
\addcontentsline{toc}{section}{प्रथमः परिच्छेदः}
\addtocontents{toc}{\protect\setcounter{tocdepth}{0}}
\centering ॥ अथ द्वितीयाध्याये प्रथमः परिच्छेदः ॥\nopagebreak\\
\vspace{4mm}
\pdfbookmark[2]{बालकाण्डम्}{Chap2Part1Kanda1}
\phantomsection
\addtocontents{toc}{\protect\setcounter{tocdepth}{2}}
\addcontentsline{toc}{subsection}{बालकाण्डीयप्रयोगाणां विमर्शः}
\addtocontents{toc}{\protect\setcounter{tocdepth}{0}}
\centering ॥ अथ बालकाण्डीयप्रयोगाणां विमर्शः ॥\nopagebreak\\
\section[संविष्टम्]{संविष्टम्}
\centering\textcolor{blue}{कैलासाग्रे कदाचिद्रविशतविमले मन्दिरे रत्नपीठे\nopagebreak\\
संविष्टं ध्याननिष्ठं त्रिनयनमभयं सेवितं सिद्धसङ्घैः।\nopagebreak\\
देवी वामाङ्कसंस्था गिरिवरतनया पार्वती भक्तिनम्रा\nopagebreak\\
प्राहेदं देवमीशं सकलमलहरं वाक्यमानन्दकन्दम्॥}\nopagebreak\\
\raggedleft{–~अ॰रा॰~१.१.६}\\
\begin{sloppypar}\hyphenrules{nohyphenation}\justifying\noindent\hspace{10mm} अत्राध्यात्म\-रामायणस्य प्रारम्भ\-स्थितिं प्रस्तौति।\footnote{‘व्यासः’ इति शेषः।} कैलास\-गिरौ संविष्टं भगवन्तं शिवं पार्वती पृच्छति। अत्र सम्पूर्वकात् \textcolor{red}{विश्‌}\-धातोः (\textcolor{red}{विशँ प्रवेशने} धा॰पा॰~१६३९) कर्तरि \textcolor{red}{गत्यर्थाकर्मक\-श्लिष\-शीङ्स्थास\-वस\-जन\-रुह\-जी\-र्यतिभ्यश्च} (पा॰सू॰~३.४.७२) इत्यनेन \textcolor{red}{क्त}\-प्रत्ययः। \textcolor{red}{लशक्वतद्धिते} (पा॰सू॰~१.३.८) इत्यनेनेत्सञ्ज्ञायां लोपे\footnote{\textcolor{red}{तस्य लोपः} (पा॰सू॰~१.३.९) इत्यनेन} \textcolor{red}{व्रश्च\-भ्रस्ज\-सृज\-मृज\-यज\-राज\-भ्राजच्छशां षः} (पा॰सू॰~८.२.३६) इत्यनेन षत्वे \textcolor{red}{ष्टुना ष्टुः} (पा॰सू॰~८.४.४१) इत्यनेन ष्टुत्वे विभक्ति\-कार्ये \textcolor{red}{संविष्टम्} इति। सम्पूर्वकस्य \textcolor{red}{विश्‌}\-धातोः शयनमर्थः।\footnote{यथा~– \textcolor{red}{पश्चादग्नेरुदगग्रेषु दर्भेषु प्राक्शिराः संविशति} (गो॰गृ॰सू॰~२.६.१०) \textcolor{red}{क्रमेण सुप्तामनु संविवेश} (र॰वं॰~२.२४) \textcolor{red}{चरमं संविशति या प्रथमं प्रतिबुध्यते} (म॰भा॰~२.८८.३६) \textcolor{red}{आन्याय्यादुत्थानादान्याय्याच्च संवेशनादेषोऽद्यतनः कालः} (का॰वृ॰~१.२.५७) इत्यादिषु। अवलम्बः~– चारुदेवशास्त्रिकृता \textcolor{red}{उपसर्गार्थचन्द्रिका}।} तर्ह्युपवेशन\-रूपोऽर्थः कथमिति चेत्। \textcolor{red}{समुपविष्टम्} इत्येवात्र।
\textcolor{red}{उप}\-उपसर्गस्य लोपः।\footnote{\textcolor{red}{विनाऽपि प्रत्ययं पूर्वोत्तर\-पद\-लोपो वक्तव्यः} (वा॰~५.३.८३) इत्यनेन।} अत एव \textcolor{red}{संविष्टम्} इत्यस्य हि \textcolor{red}{समुपविष्टम्} इत्यर्थः।\footnote{अन्यत्रापि दृश्यते। \textcolor{red}{जघनार्धेन च पशुरुच्च तिष्ठति सं च विशति} (श॰ब्रा॰~८.२.४.२०, संविशति = निषीदति)। \textcolor{red}{पयटेत्कीटवद्भूमिं वर्षास्वेकत्र संविशेत्} (ल॰वि॰स्मृ॰~४.५, संविशेत् = तिष्ठेत् = वसेत्)। \textcolor{red}{अन्येनोत्थाप्यतेऽन्येन तथा संवेश्यते जरी} (वि॰पु॰~६.५.३३, संवेश्यते = उपवेश्यते)। अवलम्बः~– चारुदेवशास्त्रिकृता \textcolor{red}{उपसर्गार्थचन्द्रिका}।}\end{sloppypar}
\section[पुरा रामायणे रामः]{पुरा रामायणे रामः}
\centering\textcolor{blue}{पुरा रामायणे रामो रावणं देवकण्टकम्।\nopagebreak\\
हत्वा रणे रणश्लाघी सपुत्रबलवाहनम्॥\\
सीतया सह सुग्रीवलक्ष्मणाभ्यां समन्वितः।\nopagebreak\\
अयोध्यामगमद्रामो हनूमत्प्रमुखैर्वृतः॥}\nopagebreak\\
\raggedleft{–~अ॰रा॰~१.१.२६-२७}\\
\begin{sloppypar}\hyphenrules{nohyphenation}\justifying\noindent\hspace{10mm} अत्र भूतभावनो भगवान् शिवोऽध्यात्म\-रामायण\-कथायाः प्रस्तावं करोति यत् \textcolor{red}{पुरा रामायणे} श्रीरामो रावणं हत्वाऽयोध्यामगमत्। अत्र \textcolor{red}{रामायणे} इति हि कस्य विशेषणं किमभिप्रायकं वाऽत्र सप्तमी वा किन्निमित्तिका। यदि चेदाधारे सप्तमी तदा रामायणं पुस्तकमत्र राम\-निरूपिताऽऽधारता कथं सम्भवा। यदि चेल्लक्षणया रामायण\-लक्षितस्तस्या मूलमन्वयानुप\-पत्तिस्तात्पर्यानुप\-पत्तिश्च।\footnote{\textcolor{red}{अन्वयाद्यनुपपत्ति\-प्रतिसन्धानञ्च लक्षणाबीजम्। वस्तुतस्तु तात्पर्यानुपपत्ति\-सन्धानमेव तद्बीजम्} (प॰ल॰म॰~२३–२४)।} यथा \textcolor{red}{गङ्गायां घोषः} इत्यत्र तात्पर्यानुपपत्तिः। यतो हि घोष आभीरपल्ली। सा च भगीरथ\-रथ\-खातावच्छिन्न\-जल\-प्रवाहे सम्भवा नहि। अतोऽन्वयानुप\-पत्तिरपि तात्पर्यानुपपत्तिश्चेति चेत्सामीप्य\-सम्बन्धेन गङ्गा\-पदस्य गङ्गा\-तीरे लक्षणा। तथैवेत्यत्रापि \textcolor{red}{रामायणे} तात्पर्यानुपपत्तेः \textcolor{red}{रामायण}\-पदस्य \textcolor{red}{रामायणोपलक्षिते काले} लक्षणा। इयं च जघन्या वृत्तिर्वैयाकरण\-मते। एतस्या अस्तित्वमपि नास्तीति चेच्छक्यतावच्छेदकारोप इति चेत्।\footnote{\textcolor{red}{तन्न। सति तात्पर्ये सर्वे सर्वार्थवाचका इति भाष्याल्लक्षणाया अभावाद्वृत्ति\-द्वयावच्छेदक\-द्वय\-कल्पने गौरवात्। जघन्य\-वृत्ति\-कल्पनाया अन्याय्यत्वाच्च} (प॰ल॰म॰~२७)।} 
अलं गुरु\-गुरु\-कल्पनया। \textcolor{red}{रामायणमस्त्यस्मिन् स रामायणः} इति विग्रहे प्रथमान्ताद्रामायण\-शब्दात् \textcolor{red}{अर्श\-आदिभ्योऽच्} (पा॰सू॰~५.२.१२७) इत्यनेन \textcolor{red}{अच्} प्रत्यये \textcolor{red}{यचि भम्} (पा॰सू॰~१.४.१८) इत्यनेन भ\-सञ्ज्ञायां \textcolor{red}{यस्येति च} (पा॰सू॰~६.४.१४८) इत्यनेनाकार\-लोपे \textcolor{red}{रामायणः} तस्मिन् \textcolor{red}{रामायणे} इति साधु। अर्थाद्रामायणे काले।\end{sloppypar}
\section[हनूमन्तम्]{हनूमन्तम्}
\centering\textcolor{blue}{दृष्ट्वा तदा हनूमन्तं प्राञ्जलिं पुरतः स्थितम्।\nopagebreak\\
कृतकार्यं निराकाङ्क्षं ज्ञानापेक्षं महामतिम्॥}\nopagebreak\\
\raggedleft{–~अ॰रा॰~१.१.२९}\\
\centering\textcolor{blue}{ततो रामः स्वयं प्राह हनूमन्तमुपस्थितम्।\nopagebreak\\
शृणु तत्त्वं प्रवक्ष्यामि ह्यात्मानात्मपरात्मनाम्॥}\nopagebreak\\
\raggedleft{–~अ॰रा॰~१.१.४४}\\
\begin{sloppypar}\hyphenrules{nohyphenation}\justifying\noindent\hspace{10mm} \textcolor{red}{प्रशस्तो हनुर्यस्य स हनुमान्} इति विग्रहे \textcolor{red}{तदस्यास्त्यस्मिन्निति मतुप्} (पा॰सू॰~५.२.९४) इत्यनेन मतुप्प्रत्यये विभक्ति\-कार्ये \textcolor{red}{हनुमन्तम्} इति पाणिनीयम्। \textcolor{red}{हनूमन्तम्} इति हि कथम्। \textcolor{red}{ऊङुतः} (पा॰सू॰~४.१.६६) इत्यनेनोङि कृते दीर्घे \textcolor{red}{हनू} इति।\footnote{छान्दसत्वाद्बाहुलकाद्वा नोपधादमनुष्य\-जातेरप्यूङित्यर्थः। अत्र वाचस्पत्य\-काराः – \textcolor{red}{हनु(नू) पुंस्त्री॰~हन-उन् स्त्रीत्वे वा ऊङ्}। शब्दकल्पद्रुम\-काराश्च – \textcolor{red}{हनूः, स्त्री, हनु + पक्षे ऊञ्। हनुः। इत्यमरटीकायां भरतः}। भाष्ये तु \textcolor{red}{ऊङ्प्रकरणेऽप्राणिजातेश्चारज्ज्वादीनाम्} (वा॰~४.१.६६) इति वार्त्तिकानन्तरं \textcolor{red}{हनु}शब्दो रज्ज्वादिगणे पठितः। परन्तु दृश्यते हि \textcolor{red}{हनू}\-रूपमार्षग्रन्थेषु। यथा मूल\-रामायणे \textcolor{red}{भरतस्यान्तिकं रामो हनूमन्तं व्यसर्जयत्} (वा॰रा॰~१.१.८५)। अत्र गोविन्दराजाश्च~– \textcolor{red}{हनूशब्द ऊकारान्तोऽप्यस्ति} (वा॰रा॰ भू॰टी॰~१.१.८५)।} ततो मतुप्प्रत्यये \textcolor{red}{हनूमान्} इति। विभक्ति\-कार्ये \textcolor{red}{हनूमन्तम्}। यद्वा \textcolor{red}{अन्येषामपि दृश्यते} (पा॰सू॰~६.३.१३८) इत्यनेन दीर्घे \textcolor{red}{हनूमन्तम्} इति।\end{sloppypar}
\section[आनन्दम्]{आनन्दम्}
\centering\textcolor{blue}{आनन्दं निर्मलं शान्तं निर्विकारं निरञ्जनम्।\nopagebreak\\
सर्वव्यापिनमात्मानं स्वप्रकाशमकल्मषम्॥}\nopagebreak\\
\raggedleft{–~अ॰रा॰~१.१.३३}\\
\begin{sloppypar}\hyphenrules{nohyphenation}\justifying\noindent\hspace{10mm} अत्र \textcolor{red}{आनन्द}\-शब्दस्य \textcolor{red}{आत्मानम्} इत्यनेन सामानाधिकरण्यं कथम्। यतो ह्यात्माऽऽनन्दस्याधिकरणम्। तथा चात्र कथं न षष्ठीति चेत्। अद्वैत\-वेदान्त\-मत आत्मन आनन्द\-रूपत्वात्। न च \textcolor{red}{आनन्दमयोऽभ्यासात्} (ब्र॰सू॰~१.१.१३) इति ब्रह्मसूत्र आनन्द\-स्वरूपत्वं न प्रत्यपादि। अथ कस्मिन्नर्थे \textcolor{red}{मयट्}। किं \textcolor{red}{तस्य विकारः} (पा॰सू॰~४.३.१३४) इत्यनेन विकारार्थे। नहि तावदात्मा निर्विकारः \textcolor{red}{आनन्दं निर्मलं शान्तमविकारमकल्मषम्} इत्यत्रैवोक्तत्वात् \textcolor{red}{अविकार्योऽयमुच्यते} (भ॰गी॰~२.२५) इति गीतायामप्युक्तत्वादिति चेत्। \textcolor{red}{तत्प्रकृत\-वचने मयट्} (पा॰सू॰~५.४.२१) इत्यनेन प्राचुर्यार्थे। अद्वैतवादिनां मते स्वरूपे। अतः सामानाधिकरण्यं प्राचुर्यार्थे मयटि। परमात्माऽऽनन्दस्य निलयमित्यपेक्षायाम् \textcolor{red}{आनन्दोऽस्त्यस्मिन्} इत्यर्श\-आद्यच्।\footnote{\textcolor{red}{अर्शआदिभ्योऽच्} (पा॰सू॰~५.२.१२७) इत्यनेन।} यद्वा \textcolor{red}{आनन्द इवाऽचरतीत्यानन्दति}।\footnote{आनन्द~\arrow \textcolor{red}{सर्वप्राति\-पदिकेभ्य आचारे क्विब्वा वक्तव्यः} (वा॰~३.१.११)~\arrow आनन्द~क्विँप्~\arrow आनन्द~व्~\arrow \textcolor{red}{वेरपृक्तस्य} (पा॰सू॰~६.१.६७)~\arrow आनन्द~\arrow \textcolor{red}{सनाद्यन्ता धातवः} (पा॰सू॰~३.१.३२)~\arrow धातुसञ्ज्ञा~\arrow \textcolor{red}{शेषात्कर्तरि परस्मैपदम्} (पा॰सू॰~१.३.७८)~\arrow \textcolor{red}{वर्तमाने लट्} (पा॰सू॰~३.२.१२३)~\arrow आनन्द~लट्~\arrow आनन्द~तिप्~\arrow आनन्द~ति~\arrow \textcolor{red}{कर्तरि शप्‌} (पा॰सू॰~३.१.६८)~\arrow आनन्द~शप्~ति~\arrow आनन्द~अ~ति~\arrow \textcolor{red}{अतो गुणे} (पा॰सू॰~६.१.९७)~\arrow आनन्द~ति~\arrow आनन्दति।} \textcolor{red}{आनन्दतीत्यानन्दः}।\footnote{आनन्द~\arrow धातुसञ्ज्ञा (पूर्ववत्)~\arrow \textcolor{red}{क्विप् च} (पा॰सू॰~३.२.७६)~\arrow आनन्द~क्विँप्~\arrow आनन्द~व्~\arrow \textcolor{red}{वेरपृक्तस्य} (पा॰सू॰~६.१.६७)~\arrow आनन्द~\arrow विभक्तिकार्यम्~\arrow आनन्दः। यद्वा \textcolor{red}{नन्दि\-ग्रहि\-पचादिभ्यो ल्युणिन्यचः} (पा॰सू॰~३.१.१३४) इत्यनेन कर्तर्यचि। आनन्द~\arrow \textcolor{red}{धातुसञ्ज्ञा} (पूर्ववत्)~\arrow \textcolor{red}{नन्दि\-ग्रहि\-पचादिभ्यो ल्युणिन्यचः} (पा॰सू॰~३.१.१३४)~\arrow आनन्द~अच्~\arrow आनन्द~अ~\arrow \textcolor{red}{अतो गुणे} (पा॰सू॰~६.१.९७)~\arrow आनन्द~\arrow विभक्तिकार्यम्~\arrow आनन्दः।} कर्तरि क्विपि सर्वापहारि\-लोपे \textcolor{red}{कृत्तद्धित\-समासाश्च} (पा॰सू॰~१.२.४६) इत्यनेन प्रातिपदिक\-सञ्ज्ञायां \textcolor{red}{स्वौ\-जसमौट्छष्टा\-भ्याम्भिस्ङे\-भ्याम्भ्यस्ङसि\-भ्याम्भ्यस्ङसोसाम्ङ्योस्सुप्} (पा॰सू॰~४.१.२) इत्यनेन \textcolor{red}{अमि} विभक्तौ \textcolor{red}{अमि पूर्वः} (पा॰सू॰~६.१.१०७) इत्यनेन पूर्व\-रूपे \textcolor{red}{आनन्दम्} इति पाणिनीयमेव।\end{sloppypar}
\section[अविकारिणि]{अविकारिणि}
\centering\textcolor{blue}{साभासबुद्धेः कर्तृत्वमविच्छिन्नेऽविकारिणि।\nopagebreak\\
साक्षिण्यारोप्यते भ्रान्त्या जीवत्वं च तथा बुधैः॥}\nopagebreak\\
\raggedleft{–~अ॰रा॰~१.१.४७}\\
\begin{sloppypar}\hyphenrules{nohyphenation}\justifying\noindent\hspace{10mm} अत्र \textcolor{red}{न विकार इत्यविकारः}। \textcolor{red}{अविकारोऽस्त्यस्मिन्निति अविकारी} इति विग्रहे \textcolor{red}{अत इनिठनौ} (पा॰सू॰~५.२.११५) इत्यनेन \textcolor{red}{इनि}\-प्रत्यये विभक्ति\-लोपे भ\-सञ्ज्ञायां \textcolor{red}{यस्येति च} (पा॰सू॰~६.४.१४८) इत्यनेनाकार\-लोपे पुनः सप्तमी\-\textcolor{red}{ङि}\-विभक्तौ \textcolor{red}{लशक्वतद्धिते} (पा॰सू॰~१.३.८) इत्यनेन ङकारेत्सञ्ज्ञायामनुबन्ध\-लोपे \textcolor{red}{अट्कुप्वाङ्नुम्व्यवायेऽपि} (पा॰सू॰~८.४.२) इत्यनेन णत्वे \textcolor{red}{अविकारिणि}। अत्र \textcolor{red}{न कर्मधारयान्मत्वर्थीयो बहुव्रीहिश्चेत्तदर्थ\-प्रतिपत्ति\-करः}\footnote{मूलं मृग्यम्।} इति वचनेन हि मत्वर्थीय\-निषेधात् \textcolor{red}{इनिः} अपाणिनीय इति चेत्। अत्र कर्मधारयो नास्ति तदा कथमुक्त\-नियमस्य प्रसरः। अत्र कर्मधारयः सकल\-समासोपलक्षणमिति चेत्। अत्र न \textcolor{red}{इनिः} किन्तु \textcolor{red}{न विकर्तुं तच्छीलः} इति विग्रहे \textcolor{red}{वि}\-पूर्वकात् \textcolor{red}{कृ}\-धातोः (\textcolor{red}{डुकृञ् करणे} धा॰पा॰~१४७२) \textcolor{red}{सुप्यजातौ णिनिस्ताच्छील्ये} (पा॰सू॰~३.२.७८) इत्यनेन णिनिः। णकारानुबन्धे कार्ये \textcolor{red}{अचो ञ्णिति} (पा॰सू॰~७.२.११५) इत्यनेन वृद्धौ रपरत्वे\footnote{\textcolor{red}{उरण् रपरः} (पा॰सू॰~१.१.५१) इत्यनेन।} सप्तम्येकवचने णत्वे\footnote{\textcolor{red}{अट्कुप्वाङ्नुम्व्यवायेऽपि} (पा॰सू॰~८.४.२) इत्यनेन।} \textcolor{red}{अविकारिणि} इति पाणिनीयमेव।\end{sloppypar}
\section[जगत्येन]{जगत्येन}
\centering\textcolor{blue}{मायया गुणमय्या त्वं सृजस्यवसि लुम्पसि।\nopagebreak\\
जगत्येन न ते लेप आनन्दानुभवात्मनः॥}\nopagebreak\\
\raggedleft{–~अ॰रा॰~१.२.१५}\\
\begin{sloppypar}\hyphenrules{nohyphenation}\justifying\noindent\hspace{10mm} अत्र रावण\-कुकृत्य\-जन्य\-पाद\-भार\-पीडितया वसुमत्या सह सकल\-देव\-पुरःसरं क्षीर\-सागरं गतो ब्रह्मा भगवन्तं स्तुवन्नाह यत्त्रिगुणमय्या मायया हेतु\-भूतयोपलक्षितस्त्वं संसारं सृजसि पालयसि नाशयसि किन्तु त्वं जागतिक\-पदार्थेन न लिप्यसे। अत्र \textcolor{red}{जागतिकेन} इत्येव प्रयोक्तव्यं यतो हि \textcolor{red}{जगति भवं जागतिकम्} इति विग्रहे \textcolor{red}{तत्र भवः} (पा॰सू॰~४.३.५३) इत्यनेन \textcolor{red}{ठक्} प्रत्यये \textcolor{red}{तद्धितेष्वचामादेः} (पा॰सू॰~७.२.११७) इति वृद्धौ 
\textcolor{red}{इसुसुक्तान्तात्कः} (पा॰सू॰~७.३.५१) इत्यनेन कादेशे विभक्ति\-कार्ये \textcolor{red}{जागत्केन} इत्येव पाणिनीयम्। एवं \textcolor{red}{जगत्येन} इति विमृश्यते। तथा च \textcolor{red}{आत्मनो जगदिच्छतीति जगत्यति} इति विग्रहे \textcolor{red}{सुप आत्मनः क्यच्} (पा॰सू॰~३.१.८) इत्यनेन \textcolor{red}{क्यच्} प्रत्यये \textcolor{red}{लशक्वतद्धिते} (पा॰सू॰~१.३.८) इत्यनेनेत्सञ्ज्ञायामनु\-बन्ध\-लोपे \textcolor{red}{वर्तमाने लट्} (पा॰सू॰~३.२.१२३) इति लट्। ततः तिपि शपि \textcolor{red}{जगत्यति}।\footnote{\textcolor{red}{जगत्यति} इति भाष्ये \textcolor{red}{तुग्यणेकादेश\-गुण\-वृद्ध्यौत्त्व\-दीर्घत्वेत्वमुमेत्त्त्वरी\-विधिभ्यः} (वा॰~१.४.२) इति वार्तिक उदाहृतम्।} ततः \textcolor{red}{जगत्यतीति जगत्यः} इति विग्रहे \textcolor{red}{नन्दिग्रहिपचादिभ्यो ल्युणिन्यचः} (पा॰सू॰~३.१.१३४) इत्यनेन \textcolor{red}{अच्} प्रत्यये विभक्ति\-कार्ये तृतीयैक\-वचने \textcolor{red}{टा}\-विभक्तौ \textcolor{red}{टाङसिङसामिनात्स्याः} (पा॰सू॰~७.१.१२) इत्यनेनेनादेशे गुणे\footnote{\textcolor{red}{आद्गुणः} (पा॰सू॰~६.१.८७) इत्यनेन।} \textcolor{red}{जगत्येन} इति पाणिनीयमेव। यद्वा \textcolor{red}{जगत्संसारं याति गच्छति} इति विग्रहे जगदुपपदे \textcolor{red}{या}\-धातोः (\textcolor{red}{या प्रापणे} धा॰पा॰~१०४९) \textcolor{red}{आतोऽनुपसर्गे कः} (पा॰सू॰~३.२.३) इत्यनेन \textcolor{red}{क}\-प्रत्यये ककारस्यानुबन्ध\-कार्ये \textcolor{red}{आतो धातोः} (पा॰सू॰~६.४.१४०) इत्यनेनाऽकार\-लोपेऽपदत्वाज्जश्त्वाभावे\footnote{\textcolor{red}{पृषोदरादीनि यथोपदिष्टम्} (पा॰सू॰~६.३.१०९) इति सूत्रेणापदत्वम्। यद्वा \textcolor{red}{अयस्मयादीनि च्छन्दसि} (पा॰सू॰~१.४.२०) इति सूत्रेण छान्दसभत्वम्।} \textcolor{red}{जगत्यः} तेन \textcolor{red}{जगत्येन}।
यद्वा \textcolor{red}{जगदाचष्ट इति जगतयति}।\footnote{जगत्~\arrow \textcolor{red}{तत्करोति तदाचष्टे} (धा॰पा॰ ग॰सू॰)~\arrow जगत्~णिच्~\arrow जगत्~इ~\arrow जगति~\arrow \textcolor{red}{सनाद्यन्ता धातवः} (पा॰सू॰~३.१.३२)~\arrow \textcolor{red}{शेषात्कर्तरि परस्मैपदम्} (पा॰सू॰~१.३.७८)~\arrow \textcolor{red}{वर्तमाने लँट्} (पा॰सू॰~३.२.१२३)~\arrow जगति~लट्~\arrow जगति~तिप्~\arrow \textcolor{red}{कर्तरि शप्‌} (पा॰सू॰~३.१.६८)~\arrow जगति~शप्~तिप्~\arrow जगति~अ~ति~\arrow \textcolor{red}{सार्वधातुकार्धधातुकयोः} (पा॰सू॰~७.३.८४)~\arrow जगते~अ~ति~\arrow \textcolor{red}{एचोऽयवायावः} (पा॰सू॰~६.१.७८)~\arrow जगतय्~अ~ति~\arrow जगतयति।} \textcolor{red}{जगतयतीति जगत्} इत्याचक्षाणाण्णिजन्तात्क्विपि ककारस्य \textcolor{red}{लशक्वतद्धिते} (पा॰सू॰~१.३.८) इत्यनेनेत्सञ्ज्ञायां \textcolor{red}{तस्य लोपः} (पा॰सू॰~१.३.९) इत्यनेन लोप इकारस्य \textcolor{red}{उपदेशेऽजनुनासिक इत्} (पा॰सू॰~१.३.२) इत्यनेनेत्सञ्ज्ञायां लोपे पकारस्य \textcolor{red}{हलन्त्यम्} (पा॰सू॰~१.३.३) इत्यनेनेत्सञ्ज्ञायां लोपे \textcolor{red}{वेरपृक्तस्य} (पा॰सू॰~६.१.६७) इत्यनेन वकार\-लोपे \textcolor{red}{णेरनिटि} (पा॰सू॰~६.४.५१) इत्यनेन णिलोपे।\footnote{जगति~\arrow धातुसञ्ज्ञा (पूर्ववत्)~\arrow \textcolor{red}{क्विप् च} (पा॰सू॰~३.२.७६)~\arrow जगति~क्विँप्~\arrow जगति~व्~\arrow \textcolor{red}{वेरपृक्तस्य} (पा॰सू॰~६.१.६७)~\arrow जगति~\arrow \textcolor{red}{णेरनिटि} (पा॰सू॰~६.४.५१)~\arrow जगत्~\arrow विभक्तिकार्यम्~\arrow जगत्।} पुनः \textcolor{red}{यातीति यः}।\footnote{\textcolor{red}{या प्रापणे} (धा॰पा॰~१०४९)~\arrow या~\arrow \textcolor{red}{अन्येष्वपि दृश्यते} (पा॰सू॰~३.२.१०१)~\arrow या~ड~\arrow या~अ~\arrow \textcolor{red}{डित्यभस्याप्यनु\-बन्धकरण\-सामर्थ्यात्} (वा॰~६.४.१४३)~\arrow य्~अ~\arrow य~\arrow विभक्तिकार्यम्~\arrow यः।} \textcolor{red}{जगदेव य इति जगत्यः} इति विग्रहे कर्मधारय\-समासे \textcolor{red}{जगत्यः}। न च \textcolor{red}{झलां जशोऽन्ते} (पा॰सू॰~८.२.३९) इत्यनेन कथं न जश्त्वम्। पृषोदरादित्वात्\footnote{\textcolor{red}{पृषोदरादीनि यथोपदिष्टम्} (पा॰सू॰~६.३.१०९) इत्यनेन। यद्वा \textcolor{red}{अयस्मयादीनि च्छन्दसि} (पा॰सू॰~१.४.२०) इति सूत्रेण छान्दसभत्वम्।} तद्भाव\-कल्पनेनादोषात्। पुनः तेन \textcolor{red}{जगत्येन}।\end{sloppypar}
\section[सपत्निवत्]{सपत्निवत्}
\centering\textcolor{blue}{तवाङ्घ्रिपूजानिर्माल्यतुलसीमालया विभो।\nopagebreak\\
स्पर्धते वक्षसि पदं लब्ध्वाऽपि श्रीः सपत्निवत्॥}\nopagebreak\\
\raggedleft{–~अ॰रा॰~१.२.१९}\\
\begin{sloppypar}\hyphenrules{nohyphenation}\justifying\noindent\hspace{10mm} अत्र \textcolor{red}{सपत्न्या तुल्यम्} इति विग्रहे \textcolor{red}{तेन तुल्यं क्रिया चेद्वतिः} (पा॰सू॰~५.१.११५) इत्यनेन \textcolor{red}{वति} प्रत्यये \textcolor{red}{सपत्नीवत्}।\footnote{\textcolor{red}{समानः पतिरस्याः} इति विग्रहे \textcolor{red}{नित्यं सपत्न्यादिषु} (पा॰सू॰~४.१.३५) इत्यनेन निपातनात्समानस्य सादेशे \textcolor{red}{सपत्नी}\-शब्दो व्युत्पन्नः।} \textcolor{red}{सपत्निवत्} इति कथम्।
अत्र हि \textcolor{red}{ङ्यापोः सञ्ज्ञा\-छन्दसोर्बहुलम्} (पा॰सू॰~६.३.६३) इत्यनेन ह्रस्वः।\footnote{अपि च वाल्मीकीय\-रामायणेऽयोध्या\-काण्डे~– \textcolor{red}{साहं त्वदर्थे सम्प्राप्ता त्वं तु मां नावबुध्यसे। सपत्निवृद्धौ या मे त्वं प्रदेयं दातुमिच्छसि॥} (वा॰रा॰~२.८.२६)। अत्र टीकाकाराः – \textcolor{red}{ङ्यापोः सञ्ज्ञा\-छन्दसोर्बहुलम्} (पा॰सू॰~६.३.६३) इत्यनेनार्षत्वेन वा ह्रस्वत्वम्। यथा~– \textcolor{red}{सपत्निवृद्धाविति ‘ङ्यापोः’ इति ह्रस्वः} (वा॰रा॰ भू॰टी॰)। \textcolor{red}{सपत्न्याः वृद्धिः – सपत्निवृद्धिः ‘ङ्यापोः’ इति ह्रस्वः} (वा॰रा॰ क॰टी॰)। \textcolor{red}{सपत्निवृद्धावित्यत्र ह्रस्व आर्षः} (वा॰रा॰ शि॰टी॰)। \textcolor{red}{सपत्निवृद्धावित्यार्षो ह्रस्वः} (वा॰रा॰ ति॰टी॰)।}\end{sloppypar}
\section[अतिहर्षितः]{अतिहर्षितः}
\centering\textcolor{blue}{इति ब्रुवन्तं ब्रह्माणं बभाषे भगवान् हरिः।\nopagebreak\\
किं करोमीति तं वेधाः प्रत्युवाचातिहर्षितः॥}\nopagebreak\\
\raggedleft{–~अ॰रा॰~१.२.२२}\\
\begin{sloppypar}\hyphenrules{nohyphenation}\justifying\noindent\hspace{10mm} ब्रह्मणः स्तुतिं श्रुत्वाऽऽत्मानं प्रदर्श्य भगवान् बभाषे यत् \textcolor{red}{किं करोमि} इति। समाकर्ण्याति\-हृष्टो ब्रह्मा प्रत्युवाच। \textcolor{red}{अतिहृष्टः} इति प्रयोगतन्त्रे \textcolor{red}{अतिहर्षितः} इत्यपाणिनीय इव प्रयुक्तः। यतो हि \textcolor{red}{हृष्‌}\-धातोः (\textcolor{red}{हृषँ तुष्टौ} धा॰पा॰~१२२९) \textcolor{red}{गत्यर्थाकर्मक\-श्लिष\-शीङ्स्थास\-वस\-जन\-रुह\-जीर्यतिभ्यश्च} (पा॰सू॰~३.४.७२) इत्यनेनाकर्मक\-धातोः कर्तरि \textcolor{red}{क्त}\-प्रत्यये ककारानुबन्ध\-लोपे\footnote{\textcolor{red}{लशक्वतद्धिते} (पा॰सू॰~१.३.८) \textcolor{red}{तस्य लोपः} (पा॰सू॰~१.३.९) इत्याभ्याम्।} ष्टुत्वे\footnote{\textcolor{red}{ष्टुना ष्टुः} (पा॰सू॰~८.४.४१) इत्यनेन।} विभक्तिकार्ये \textcolor{red}{हृष्टः} इत्येव हि पाणिनीयम्। \textcolor{red}{हर्षितः} इति प्रयोगोऽपि पाणिनीयः। यतो हि \textcolor{red}{हर्षः सञ्जातोऽस्य} इति विग्रहे \textcolor{red}{तदस्य सञ्जातं तारकादिभ्य इतच्} (पा॰सू॰~५.२.३६) इत्यनेन \textcolor{red}{इतच्‌}\-प्रत्ययेऽनुबन्ध\-लोपे \textcolor{red}{यचि भम्} (पा॰सू॰~१.४.१८) इत्यनेन भसञ्ज्ञायां \textcolor{red}{यस्येति च} (पा॰सू॰~६.४.१४८) इत्यनेनाकार\-लोपे विभक्ति\-कार्ये \textcolor{red}{हर्षितः}। यद्वा \textcolor{red}{हर्षमितः} इति विग्रहे \textcolor{red}{द्वितीया श्रितातीत\-पतित\-गतात्यस्त\-प्राप्तापन्नैः} (पा॰सू॰~२.१.२४) इत्यनेन द्वितीया\-तत्पुरुष\-समासः। न च \textcolor{red}{इत}\-शब्दस्य श्रितादि\-बहिर्भूतत्वात्कथं समास इति वाच्यम्। \textcolor{red}{द्वितीया} इति योग\-विभागेन द्वितीया समर्थेन सुबन्तेन समस्यत इत्यर्थकरणे द्वितीया\-तत्पुरुषो नैव दोषावहः। तेन \textcolor{red}{गृहं यातो गृहयातः गृहमागतो गृहागतः} इत्यादि\-प्रयोगा अपि सङ्गच्छन्ते। कालिदासोऽपि प्रयुङ्क्ते यथा~–\end{sloppypar}
\centering\textcolor{red}{सुरेन्द्रमात्राश्रितगर्भगौरवात्प्रयत्नमुक्तासनया गृहागतः।\nopagebreak\\
तयोपचाराञ्जलिखिन्नहस्तया ननन्द पारिप्लवनेत्रया नृपः॥}\nopagebreak\\
\raggedleft{–~र॰वं॰~३.११}\\
\begin{sloppypar}\hyphenrules{nohyphenation}\justifying\noindent अत्र \textcolor{red}{गृहमागतो गृहागतः} इत्यसति योग\-विभागे कथं समासः सम्भवः। तस्मात् \textcolor{red}{हर्षमितः} इति विग्रहे द्वितीया\-तत्पुरुषे \textcolor{red}{कृत्तद्धित\-समासाश्च} (पा॰सू॰~१.२.४६) इत्यनेन प्रातिपदिक\-सञ्ज्ञायां \textcolor{red}{सुपो धातु\-प्रातिपदिकयोः} (पा॰सू॰~२.४.७१) इत्यनेन विभक्ति\-लुकि पुनस्तामेव प्रातिपदिक\-सञ्ज्ञामाश्रित्य \textcolor{red}{सौ} विभक्तौ \textcolor{red}{हर्षितः}। न च सति विभक्ति\-लोपे \textcolor{red}{हर्ष इत} इति स्थिते गुणः स्यादिति चेत्। अत्राऽकृति\-गणत्वात् \textcolor{red}{शकन्ध्वादिषु पर\-रूपं वाच्यम्} (वा॰~६.१.९४) इत्यनेन पर\-रूपम्। कुत्र स्यात्कस्य वेति चेत्। \textcolor{red}{तच्च टेः} (वै॰सि॰कौ॰~७९, ल॰सि॰कौ॰~३९)। अर्थात्तत्पर\-रूपं भसञ्ज्ञकस्य टेः स्थाने भवतु।\footnote{पूर्वपक्षोऽयम्।} असति च तस्मिन् \textcolor{red}{शक}\-घटकाकारस्य पर\-रूपतया \textcolor{red}{शक अन्धु} इति स्थिते तयोर्दीर्घो दुर्वार एव। एवं \textcolor{red}{मनस् ईषा} इति स्थिते \textcolor{red}{शकन्ध्वादिषु पररूपं वाच्यम्} (वा॰~६.१.९४) इत्यनेनाऽकृति\-गणत्वात् \textcolor{red}{मनीषा} इत्यत्र \textcolor{red}{टि} इत्यस्य पररूपे \textcolor{red}{मन्~ई~ईषा} इति स्थिते ततो \textcolor{red}{अज्झीनं वर्णपरेण संयोज्यम्} (स्व॰शि॰~८.१५) इति वचनात् \textcolor{red}{मनीषा} इत्यत्र दीर्घः। अनिष्टः प्रयोगः स्यात्किन्तु \textcolor{red}{पतत् अञ्जलौ} इति। \textcolor{red}{टि} इत्यस्य पररूपे दीर्घे \textcolor{red}{पताञ्जलिः} इत्यसङ्गतं स्यात्।\footnote{\textcolor{red}{यदि त्वादित्यधि\-कारादस्यैवेष्येत तर्हि मनीषा पतञ्जलिरिति न सिध्येत्। केचित्तु मनः\-पतच्छब्दयोः पृषोदरादित्वादन्त्य\-लोप अकारस्यैव पररूपमाहुः} (त॰बो॰~७९)।} अतः \textcolor{red}{‘टेः’ इति पञ्चम्यन्तम्। तदन्वचि परे पूर्व\-परयोः पर\-रूपम्} इत्यस्मद्गुरु\-चरणाः।\footnote{बालमनोरमायामपि~– \textcolor{red}{ततश्च शकादि\-शब्दानां टेरचि परे टेश्च परस्याचस्स्थाने पर\-रूपमेकादेश इत्यर्थाल्लभ्यते। आदित्यनु\-वृत्तौ शकन्ध्वादिगणे ‘सीमन्त’ इति कतिपय\-रूपाणामसिद्धेः} (बा॰म॰~७९)।} तथैवात्रापि पर\-रूपं करणीयम्।\end{sloppypar}
\section[मद्दत्तवरदर्पितः]{मद्दत्तवरदर्पितः}
\centering\textcolor{blue}{भगवन् रावणो नाम पौलस्त्यतनयो महान्।\nopagebreak\\
राक्षसानामधिपतिर्मद्दत्तवरदर्पितः॥}\nopagebreak\\
\raggedleft{–~अ॰रा॰~१.२.२३}\\
\begin{sloppypar}\hyphenrules{nohyphenation}\justifying\noindent\hspace{10mm} अत्र ब्रह्मा रावणस्योच्छृङ्खलतां वर्णयति। \textcolor{red}{मद्दत्तवर\-दर्पितः} इति। अत्र \textcolor{red}{मया दत्तो वर इति मद्दत्तवरस्तेन दर्पितः} इति विग्रहे \textcolor{red}{दृप्‌}\-धातोः (\textcolor{red}{दृपँ हर्षमोहनयोः} धा॰पा॰~११९६) अनिट्कत्वात्\footnote{\textcolor{red}{एकाच उपदेशेऽनुदात्तात्‌} (पा॰सू॰~७.२.१०) इति सूत्रेण धातोरनिट्कत्वम्। तद्बाधित्वा \textcolor{red}{रधादिभ्यश्च} (पा॰सू॰~७.२.४५) इत्यनेन वैकल्पिकेट्प्राप्तिः। परन्तु निष्ठायां \textcolor{red}{यस्य विभाषा} (पा॰सू॰~७.२.१५) इत्यनेनेडभावः।} \textcolor{red}{दृप्तः} इत्येव पाणिनीयं \textcolor{red}{दर्पितः} इति कथम्। \textcolor{red}{दर्पः सञ्जातोऽस्य} इति विग्रहे तारकादिगणे \textcolor{red}{दर्पित}\-शब्दस्य पाठात् \textcolor{red}{इतच्} प्रत्यये\footnote{\textcolor{red}{तदस्य सञ्जातं तारकादिभ्य इतच्} (पा॰सू॰~५.२.३६) इत्यनेन।} भत्वादकार\-लोपे\footnote{\textcolor{red}{यचि भम्} (पा॰सू॰~१.४.१८) इत्यनेन भत्वम्। \textcolor{red}{यस्येति च} (पा॰सू॰~६.४.१४८) इत्यनेनाकार\-लोपः।} \textcolor{red}{दर्पितः}। यद्वा \textcolor{red}{दर्पमितः} इति द्वितीया\-समासे पर\-रूपे\footnote{\textcolor{red}{शकन्ध्वादिषु पररूपं वाच्यम्} (वा॰~६.१.९४) इत्यनेन।} \textcolor{red}{दर्पितः} इति।\end{sloppypar}
\section[सहायम्]{सहायम्}
\label{sec:sahayam}
\centering\textcolor{blue}{यूयं सृजध्वं सर्वेऽपि वानरेष्वंशसम्भवान्।\nopagebreak\\
विष्णोः सहायं कुरुत यावत्स्थास्यति भूतले॥}\nopagebreak\\
\raggedleft{–~अ॰रा॰~१.२.३०}\\
\begin{sloppypar}\hyphenrules{nohyphenation}\justifying\noindent\hspace{10mm} अत्र भगवान् ब्रह्मा सर्वान् देवान् वानर\-शरीराणि स्रष्टुं प्रेरयति। \textcolor{red}{विष्णोः सहायं कुरुत} अत्र \textcolor{red}{सहायम्} इत्यपाणिनीयमिव। यतो हि \textcolor{red}{अयँ गतौ} (धा.पा. ४७४) इत्यस्मात्कर्तरि \textcolor{red}{अच्‌}\-प्रत्यये \textcolor{red}{सहायः} इति भवति\footnote{\textcolor{red}{सहायः, पुं॰, (सह अयते इति। अय + अच्)} इति शब्द\-कल्प\-द्रुमः।} किन्त्वत्र तु \textcolor{red}{सर्वे सहायं कुरुत} इति विवक्षायां सहाय\-शब्दो भाव\-साधनः प्रतीयते। भावे च \textcolor{red}{सहायस्य भावः साहाय्यम्} इति विग्रहे \textcolor{red}{गुण\-वचन\-ब्राह्मणादिभ्यः कर्मणि च} (पा.सू. ५.१.१२४) इत्यनेन \textcolor{red}{ष्यञ्‌}\-प्रत्यये वृद्धौ भत्वादकार\-लोपे\footnote{\textcolor{red}{तद्धितेष्वचामादेः} (पा॰सू॰~७.२.११७) इत्यनेनादिवृद्धिः। \textcolor{red}{यचि भम्} (पा॰सू॰~१.४.१८) इत्यनेन भत्वम्। \textcolor{red}{यस्येति च} (पा॰सू॰~६.४.१४८) इत्यनेनाकार\-लोपः।} \textcolor{red}{साहाय्यम्}। \textcolor{red}{अत्र सहायम्} इति हि। पञ्चवट्यां निवासे \textcolor{red}{भगवतः साह्यं कुर्वन्तु}\footnote{\textcolor{red}{साह्यम्, क्ली॰, सहस्य भावः (सह + ष्यञ्)} इति शब्दकल्पद्रुमः।} इति विवक्षायां \textcolor{red}{सहायम्} इति कथमिति चेत्। \textcolor{red}{सहायम्} इत्यत्र पृषोदरादित्वाद्वृद्ध्यभावे यकार\-लोपे \textcolor{red}{सहायम्}। यद्वा \textcolor{red}{सर्वे सहायं यथा स्यात्तथा कुर्वन्तु} इति क्रिया\-विशेषणत्वाद्द्वितीया। यद्वा \textcolor{red}{आत्मानं सहायं कुर्वन्तु} इत्यध्याहारेऽप्यपाणिनीयता\-परिहारः। यद्वा \textcolor{red}{अयनम् अयः} इति भावे \textcolor{red}{एरच्} (पा॰सू॰~३.३.५६) इत्यनेन \textcolor{red}{इ}\-धातोः (\textcolor{red}{इण् गतौ} धा॰पा॰~१०४५) अचि गुणेऽयादेशे विभक्ति\-कार्ये \textcolor{red}{अयः}। \textcolor{red}{सह अयः सहायस्तं सहायम्} इति साधनीयम्।\footnote{\textcolor{red}{सह अयः सहायः} इत्यत्र \textcolor{red}{सह सुपा} (पा॰सू॰~२.१.४) इत्यनेन सुप्सुपा\-समासः। \pageref{sec:sahayam_me}तमे पृष्ठे \ref{sec:sahayam_me} \nameref{sec:sahayam_me} इति प्रयोगस्य विमर्शमपि पश्यन्तु। अनेन \textcolor{red}{जानकिनाथ सहाय करैं जब कौन बिगाड़ करे नर तेरो} इति मुक्तके \textcolor{red}{सहाय करैं} इति गोस्वामि\-तुलसी\-दासकृतोऽवधी\-भाषा\-प्रयोगोऽपि व्याख्यातः।}\end{sloppypar}
\section[सहायार्थम्]{सहायार्थम्}
\centering\textcolor{blue}{देवाश्च सर्वे हरिरूपधारिणः स्थिताः सहायार्थमितस्ततो हरेः।\nopagebreak\\
महाबलाः पर्वतवृक्षयोधिनः प्रतीक्षमाणा भगवन्तमीश्वरम्॥}\nopagebreak\\
\raggedleft{–~अ॰रा॰~१.२.३२}\\
\begin{sloppypar}\hyphenrules{nohyphenation}\justifying\noindent\hspace{10mm} अत्र \textcolor{red}{सहायार्थम्} इति प्रयोगोऽपि तथैव। अत्र \textcolor{red}{अयनमयः}। भावे \textcolor{red}{अच्}।\footnote{\textcolor{red}{इण् गतौ} (धा॰पा॰~१.१०४५) इति धातोः \textcolor{red}{एरच्} (पा॰सू॰~३.३.५६) इत्यनेन।} \textcolor{red}{सह अयः सहायः}।\footnote{सुप्सुपासमासः।} तस्मायिदम् \textcolor{red}{सहायार्थम्}।\footnote{\textcolor{red}{अर्थेन नित्य\-समासो विशेष्य\-लिङ्गता चेति वक्तव्यम्‌} (वा॰~२.१.३६) इत्यनेन नित्यसमासः।} यद्वा भावेऽयं घञन्तः \textcolor{red}{आयः}।\footnote{\textcolor{red}{अयँ गतौ} (धा॰पा॰~४७४) इति धातोः \textcolor{red}{भावे} (पा॰सू॰~३.३.१८) इत्यनेन।} तेन सन्धौ कृते न दोषः। \textcolor{red}{सह आयः सहायः}। \textcolor{red}{तस्मा इदं सहायार्थम्}।
एवं निष्पन्न\-\textcolor{red}{सहाय}\-शब्दस्य \textcolor{red}{चतुर्थी तदर्थार्थ\-बलि\-हित\-सुख\-रक्षितैः} (पा॰सू॰~२.१.३६) इत्यनेन \textcolor{red}{अर्थेन नित्य\-समासो विशेष्य\-लिङ्गता चेति वक्तव्यम्} (वा॰~२.१.३६) इति\-वार्त्तिक\-बलेन नित्य\-समासे सिद्धमिदम्। न च \textcolor{red}{सहायाय इदम्} इति लौकिक\-विग्रहे श्रूयते पश्चात् \textcolor{red}{अर्थ}\-शब्देन सह समास इति चेत्। विग्रहो द्विधा स्वपद\-विग्रहोऽस्वपदविग्रहश्च। स्वपद\-विग्रहो नाम यत्र विग्रहे श्रुतानां पदानां समासः। यत्र विग्रहेऽश्रुतानामपि श्रुतार्थ\-बोध\-समर्थानां समासस्तत्रास्वपद\-विग्रहः। यथा \textcolor{red}{प्रातिपदिकार्थ\-लिङ्ग\-परिमाण\-वचन\-मात्रे प्रथमा} (पा॰सू॰~२.३.४६) इति सूत्रे। अत्र हि \textcolor{red}{प्रातिपदिकार्थश्च लिङ्गञ्च परिमाणञ्च वचनञ्चेति प्रातिपदिकार्थ\-लिङ्ग\-परिमाण\-वचनानि तान्येवेति प्रातिपदिकार्थ\-लिङ्ग\-परिमाण\-वचन\-मात्रं तस्मिन् प्रातिपदिकार्थ\-लिङ्ग\-परिमाण\-वचनमात्रे}। अत्र विग्रहे श्रुतस्तु \textcolor{red}{एव}\-शब्दः किन्त्वस्व\-पद\-विग्रहतया तदर्थ\-वाचक आगतो \textcolor{red}{मात्र}\-शब्दः। स च प्रत्येकमन्वितः। यतो हि \textcolor{red}{द्वन्द्वान्ते श्रूयमाणं पदं प्रत्येकमभिसम्बध्यते}। न च द्वन्द्व\-घटकोऽन्त्य\-शब्दस्तु वचनमिति तदेव प्रत्येकमभिसम्बध्यतामिति चेत्। अन्त्य\-शब्दस्य समीपार्थे लक्षणा। न च सा जघन्या वृत्तिर्वैयाकरणैरस्वीकृताऽपि।\footnote{\textcolor{red}{तन्न। सति तात्पर्ये सर्वे सर्वार्थवाचका इति भाष्याल्लक्षणाया अभावाद्वृत्ति\-द्वयावच्छेदक\-द्वय\-कल्पने गौरवात्। जघन्य\-वृत्ति\-कल्पनाया अन्याय्यत्वाच्च} (प॰ल॰म॰~२७)।} तथा च \textcolor{red}{गङ्गायां घोषः} इत्यत्राऽभीर\-पल्ल्यर्थ\-वाचकस्य घोष\-शब्दस्य गङ्गा\-पदाभिधेय\-भगीरथ\-रथ\-खातावच्छिन्न\-जल\-प्रवाह\-रूपे शक्यार्थेऽन्वयासम्भवात्तात्पर्यानुपपतेश्च। अत्र सामीप्य\-सम्बन्धेन गङ्गा\-पदस्य गङ्गा\-समीपवर्ति\-तटे लक्षणा। न च \textcolor{red}{गङ्गायाम्} इत्यत्र सप्तमी कथमिति चेत्। औपश्लेषिकस्याऽधारस्यात्र विवक्षा। औपश्लेषिको हि सम्बन्धः।
\textcolor{red}{उपश्लेषे भव औपश्लेषिकः} इति \textcolor{red}{तत्र जातः} (पा॰सू॰~४.३.२५) इत्यनेन \textcolor{red}{ठक्}।\footnote{उपश्लेष~\arrow \textcolor{red}{तत्र जातः} (पा॰सू॰~४.३.२५)~\arrow उपश्लेष~ठक्~\arrow उपश्लेष~ठ~\arrow \textcolor{red}{ठस्येकः} (पा॰सू॰~७.३.५०)~\arrow उपश्लेष~इक~\arrow \textcolor{red}{यचि भम्} (पा॰सू॰~१.४.१८)~\arrow भसञ्ज्ञा~\arrow \textcolor{red}{किति च} (पा॰सू॰~७.२.११८)~\arrow औपश्लेष~इक~\arrow \textcolor{red}{यस्येति च} (पा॰सू॰~६.४.१४८)~\arrow औपश्लेष्~इक~\arrow औपश्लेषिक~\arrow विभक्ति\-कार्यम्~\arrow औपश्लेषिकः।} उपश्लेषश्च सामीप्येन संयोगेन वा सम्बन्धः। संयोगे यथा \textcolor{red}{कटे शेते}। सामीप्ये यथा \textcolor{red}{गुरौ वसति}। न तु \textcolor{red}{गुरुमभिसंयुज्य वसति} अपि तु \textcolor{red}{गुरोः समीपे वसति}। तथैव \textcolor{red}{गङ्गायां घोषः}। न तु \textcolor{red}{गङ्गामभिसंयुज्य घोषः} अपि तु \textcolor{red}{गङ्गासमीपे घोषः}। ध्येयमेतत्~– शक्यार्थे स्वीकृते \textcolor{red}{गङ्गायां मीनः} इत्यादौ \textcolor{red}{कटे शेते} इत्यादिवत् संयोग\-सम्बन्ध\-रूप औपश्लेषिक आधारे सप्तमी किन्तु लक्ष्यार्थे \textcolor{red}{गङ्गायां घोषः} इत्यादौ सामीप्य\-सम्बन्ध औपश्लेषिक आधारे सप्तमी \textcolor{red}{गुरौ वसति} इत्यादिवत्। परञ्च \textcolor{red}{गङ्गायां मीन\-घोषौ} इत्यादौ लक्षणा\-स्वीकारे दोषः। यतो हि \textcolor{red}{मीनश्च घोषश्च मीन\-घोषौ} इति द्वन्द्वः। स च साहित्यमपेक्षते। \textcolor{red}{साहित्यं नामैक\-धर्मावच्छिन्नस्यैक\-धर्मावच्छिन्न\-संसर्गेणैक\-कर्मावच्छिन्नेऽन्वयः}। यथा \textcolor{red}{राम\-लक्ष्मणौ राक्षसान् हतः} इत्यत्र \textcolor{red}{राम\-लक्ष्मणौ} इत्यस्य कर्तृत्व\-रूप\-सम्बन्धावच्छिन्न\-कर्तृत्व\-रूप\-धर्मेण हन्तृत्व\-रूप\-धर्मावच्छिन्ने हनन\-कर्मण्यन्वयः। तथात्रासम्भवः। एकधर्माव\-च्छिन्नस्य \textcolor{red}{मीन\-घोषौ} इत्यस्यैक\-धर्मावच्छिन्न\-संसर्गेण स्वनिष्ठत्वावच्छिन्न\-स्वनिष्ठत्व\-संसर्गेण सत्तात्वावच्छिन्न\-सत्ता\-पदार्थेऽन्वयः। परमत्र \textcolor{red}{गङ्गायां मीन\-घोषौ} इत्यत्र मीन\-पदार्थाय गङ्गा\-पदस्य प्रवाहार्थो घोष\-कृते तीर\-रूपो लक्ष्यार्थस्तर्हि कथमन्वय इति चेत्। अत्र लक्षणामनङ्गीकृत्य शक्यतावच्छेदक\-धर्मारोपः।
तथा कृत उभयत्र प्रवाह\-रूपोऽर्थः समान एव। तस्मादत्र न लक्षणा। सामीप्ये षष्ठी च चेत् \textcolor{red}{द्वन्द्वस्यान्तं द्वन्द्वान्तं द्वन्द्व\-समीपान्तम्} इत्यर्थः। सामीप्य\-षष्ठ्यर्थस्यास्मन्नयेऽस्वीकारात्प्रौढ\-मनोरमा\-बृहच्छब्दरत्न\-लघुशब्दरत्नादौ साटोपं खण्डितत्वाच्च। तत्रेयं परिस्थितिर्यदावृत्त\-हलन्तमित्यत्रैव प्रौढमनोरमायां दीक्षित\-महाभागैः पूर्वपक्षः प्रदर्शितो यत् \textcolor{red}{हलित्येकदेशस्यैव तन्त्रावृत्त्येकशेषान्यतममस्तु} (प्रौ॰म॰~१)। \textcolor{red}{हस्य ल्} इति समासः करिष्यत एवं च षष्ठ्यर्थः सामीप्यं भविष्यति। तथा च \textcolor{red}{अन्त्य}\-शब्दो \textcolor{red}{ल्}शब्देन सहान्वेष्यते। तथा च \textcolor{red}{ह\-समीप\-वर्त्यन्त्यो लकार इत्} इत्यर्थ\-स्वीकारेऽन्योऽन्याश्रय\-दोष\-परिहारः किन्तु स्वीकृतेऽस्मिन् विकल्पे षष्ठ्यर्थो हि सामीप्यम्। तस्य च सर्वथाऽसिद्धत्वम्। यतो हि सामीप्य\-रूपस्य षष्ठ्यर्थस्य लोकेऽप्रसिद्धिः। नहि \textcolor{red}{नीलमम्बरं यस्य} इत्युक्ते नीलाम्बर\-समीप\-वर्ति\-भवनादिः प्रतीयते न वा \textcolor{red}{चित्रा गावो यस्य} इत्युक्ते चित्रगवीनां समीप\-वर्ति\-वृक्षादिः प्रतीयते। किं वा \textcolor{red}{चित्रा गावोऽन्तरा समीपे वा यस्य} इत्यर्थे बहुव्रीहि\-मत्वर्थीयमपि न भवति \textcolor{red}{अन्तरा}\-शब्दस्य सामीप्यस्य च षष्ठ्यर्थ\-स्वीकारे मानाभावात्। अत एव \textcolor{red}{अन्तरेण न समासः} इति भाष्योक्तिः।\footnote{\textcolor{red}{स्त्र्यन्तस्य प्रातिपदिकस्यो\-पसर्जनस्य ह्रस्वो भवतीत्युच्यते न चान्तरेण समासं स्त्र्यन्तं प्रातिपदिकमुप\-सर्जनमस्ति} (भा॰पा॰सू॰~१.२.४८)।} वस्तुतस्तु षष्ठ्यर्था यद्यप्येकशतम् \textcolor{red}{एक\-शतं हि षष्ठ्यर्थाः} (भा॰पा॰सू॰~१.१.४९) इति भाष्योक्तेः किन्तु सम्बन्ध\-भेदेनार्थात्सम्बन्ध\-वैचित्र्येण तदर्थ\-वैविध्यम्। यतो हि सूत्र\-कारोऽपि षष्ठ्याः सम्बन्धमेवार्थं कण्ठ\-रवेण कथयति। \textcolor{red}{षष्ठी शेषे} (पा॰सू॰~२.३.५०)। \textcolor{red}{शेष}\-शब्दो हि \textcolor{red}{शिष्‌}\-धातोः (\textcolor{red}{शिषॢँ विशेषणे} धा॰पा॰~१४५१) कर्तरि \textcolor{red}{अच्}प्रत्ययेन। \textcolor{red}{शिनष्टि कारकादीनि प्रातिपदिकार्थञ्च विशिनष्टि स्वस्माद्व्यावर्तयतीति शेषः}। पचादित्वादचि प्रत्यये\footnote{\textcolor{red}{नन्दि\-ग्रहि\-पचादिभ्यो ल्युणिन्यचः} (पा॰सू॰~३.१.१३४) इत्यनेन।} \textcolor{red}{पुगन्त\-लघूपधस्य च} (पा॰सू॰~७.३.८६) इत्यनेन गुणे विभक्तिकार्ये सिद्धः \textcolor{red}{शेषः} इति।
यद्वा \textcolor{red}{शिष्यते सर्वेभ्यः कारक\-प्रातिपदिकार्थेभ्यो व्यावर्तत इति शेषः} इति \textcolor{red}{अकर्तरि च कारके सञ्ज्ञायाम्} (पा॰सू॰~३.३.१९) इत्यनेन कर्तृ\-भिन्ने कर्मणि कारके घञि गुणे सिद्धम्। यद्वा \textcolor{red}{शेषणं शेषः} इति विग्रहे भाव\-घञन्तः।\footnote{\textcolor{red}{भावे} (पा॰सू॰~३.३.१८) इत्यनेन।} सोऽस्त्यस्मिन्निति मत्वर्थीयोऽर्शआद्यच्।\footnote{\textcolor{red}{अर्शआदिभ्योऽच्} पा॰सू॰~५.२.१२७) इत्यनेन।} न च करणे \textcolor{red}{घञ्} इति चेत्। \textcolor{red}{करणाधिकरणयोश्च} (पा॰सू॰~३.३.११७) इत्यनेन विधीयमान\-ल्युट्\-प्रत्यनेन बाधोऽत एव। \textcolor{red}{पुंसि सञ्ज्ञायां घः प्रायेण} (पा॰सू॰~३.३.११८) इत्यनेन \textcolor{red}{घः} प्रत्ययः क्रियताम्। सञ्ज्ञा\-भावे कथमत्र। \textcolor{red}{घः प्रायेण} इति हि \textcolor{red}{सञ्ज्ञायाम्} इत्यस्य विशेषणमिति चेत्। \textcolor{red}{प्रायेण} इति विशेषणं न करण\-घञ्व्यवस्थापकम्। \textcolor{red}{न ह्युपाधेरुपाधिर्भवति} (भा॰पा॰सू॰~१.३.२, ५.१.१६) इति भाष्यकारेण निषिद्धत्वात्। अत एव \textcolor{red}{हलश्च} (पा॰सू॰~३.३.१२१) इति घञपि न यतो ह्ययं घविषयेऽपवादत्वात्प्रवर्तते। \textcolor{red}{उत्सर्गापवादौ समान\-देशौ भवतः}\footnote{मूलं मृग्यम्।} इति नियमस्य सार्वत्रिकत्वात् \textcolor{red}{घ} एव बाध्यते तर्हि घञः का गतिः।
अगतिक\-गतिस्तस्माद्भाव\-घञन्त एव। पश्चात् \textcolor{red}{अच्} प्रत्यये \textcolor{red}{शेष}\-शब्दो निष्पाद्यते। इमा युक्तयः प्रौढमनोरमायाम् \textcolor{red}{उपदेश}\-शब्द\-निष्पत्तौ स्पष्टं निरूपिता दीक्षित\-महाभागैर्यथा~– \textcolor{red}{यद्यपि उपदिश्यतेऽनेनेति करण\-व्युत्पत्त्या शास्त्रमुपदेश इति भाष्य\-वृत्त्यादिषु व्याख्यातं तथाऽपि तत्प्रौढिवादमात्रं करणे घञो दुर्लभत्वाल्ल्युटा बाधात्। न च “घः” असञ्ज्ञत्वात्। प्रायेण सञ्ज्ञायामिति व्याख्यानस्य क्लिष्टत्वात्। न ह्युपाधेरुपाधिर्भवतीत्यादिना भाष्य\-कृतावहेलनाच्च। अत एव घापवादः “हलश्च” (पा॰सू॰~३.३.१२१) इति घञपीह न। बाहुलकं त्वगतिक\-गतिः। अत एव प्रक्रियन्ते शब्दा याभिरिति करण\-व्युत्पत्तिरपि परास्ता। तथा च वार्त्तिकं “अजब्भ्यां स्त्रीखलनाः” (वा॰~३.३.१२६) “स्त्रियाः खलनौ विप्रतिषेधेन” (वा॰~३.३.१२६) इति। अतो भाव एव प्रत्ययो न्याय्य इति भावः} इति (प्रौ॰म॰~३)।
किन्तु यन्निमित्तोपदेश\-प्रवृत्तावित्यादिषु स्थलेषु भाष्ये करण\-घञन्तस्य प्रतिपादितत्वात्सा सरणिरत्राप्यनु\-सर्तव्या। तथा च \textcolor{red}{शिष्यन्ते कारक\-प्रातिपदिकार्था अनेनेति शेषः}। स सम्बन्धः। तथा च सिद्धान्तकौमुद्यां \textcolor{red}{कारक\-प्रातिपदिकार्थ\-व्यतिरिक्तः स्व\-स्वामि\-भावादि\-सम्बन्धः शेषस्तत्र षष्ठी स्यात्} (वै॰सि॰कौ॰~६०६)। एवं हि सम्बन्धः षष्ठ्यर्थ इति राद्धान्तः। सम्पूर्वकात् \textcolor{red}{बन्ध्‌}\-धातोः (\textcolor{red}{बन्धँ बन्धने} धा॰पा॰~१५०८) \textcolor{red}{सम्यग्बध्नाति प्रतियोग्यनुयोगिनौ यः स सम्बन्धः} इति विग्रहे कर्तरि पचाद्यचि।\footnote{\textcolor{red}{नन्दि\-ग्रहि\-पचादिभ्यो ल्युणिन्यचः} (पा॰सू॰~३.१.१३४) इत्यनेन।} तथा च सम्बन्धो हि द्विष्ठः सम्बन्धि\-भिन्न आश्रयतया विशिष्ट\-बुद्धि\-नियामकश्चेति वैयाकरण\-सिद्धान्तः।\footnote{\textcolor{red}{“सम्बन्धो हि सम्बन्धि\-द्वय\-भिन्नत्वे सति द्विष्ठत्वे च सत्याश्रयतया विशिष्टबुद्धिनियामकः” इत्यभियुक्त\-व्यवहारात्} (प॰ल॰म॰~११)।} सामीप्यं हि नैव द्विष्ठम्। अतः कथं सम्बन्धो भवेदिति ममापि मनीषा। एवं षष्ठ्यर्थो न सामीप्यम्। अतः षष्ठी\-बलेन सामीप्यार्थो नावगन्तुं शक्यते। तेन
पद\-समीपेऽन्त\-पदस्य लक्षणा। साऽस्वीकृतेति चेच्छक्यतावच्छेदकता\-रूपः। इत्थमुपबृंहणेन सिद्धमिदं यदस्वपद\-विग्रह एव \textcolor{red}{प्रातिपदिकार्थ\-लिङ्ग\-परिमाण\-वचन\-मात्रे प्रथमा} (पा॰सू॰~२.३.४६) इति सूत्रे। \textcolor{red}{मात्र}\-शब्दस्य मयूर\-व्यंसकादित्वात्समासः।\footnote{\textcolor{red}{मयूरव्यंसकादयश्च} (पा॰सू॰~२.१.७२) इत्यनेन।} तथैव \textcolor{red}{सहायायेदं सहायार्थम्} इत्यत्राप्यस्व\-पद\-विग्रहे \textcolor{red}{अर्थ}\-शब्देन नित्य\-समासः\footnote{\textcolor{red}{अर्थेन नित्य\-समासो विशेष्य\-लिङ्गता चेति वक्तव्यम्‌} (वा॰~२.१.३६) इत्यनेन।} क्रिया\-विशेषणत्वाच्च द्वितीया। अथवा \textcolor{red}{सहायार्थम्} इति पदं \textcolor{red}{भगवन्तम्} इत्यस्य विशेषणम्। एवं हि \textcolor{red}{हरि\-रूप\-धारिणः सर्वे देवा हरेरितस्ततः स्थिताः सहायार्थमीश्वरं भगवन्तं प्रतीक्षमाणाः} इति योजनीयम्। एवं \textcolor{red}{सहायानामर्थः सहायार्थः}। \textcolor{red}{सोऽस्त्यस्मिन्निति सहायार्थः}। अर्शआद्यच्।\footnote{\textcolor{red}{अर्शआदिभ्योऽच्} पा॰सू॰~५.२.१२७) इत्यनेन।} तं \textcolor{red}{सहायार्थम्}। सहाय\-प्रयोजनवन्तमिति भावः। सेवकेभ्यो गौरवं दातुं भगवाञ्छ्रीरामो रावणेन पराजितानपि देवान् स्वकीय\-लीलोपकरणानि मत्वा तत्साहाय्यमभिलषति यथा श्रीमद्भागवते~–\end{sloppypar}
\centering\textcolor{red}{नेदं यशो रघुपतेः सुरयाच्ञयाऽऽत्तलीलातनोरधिकसाम्यविमुक्तधाम्नः।\nopagebreak\\
रक्षोवधो जलधिबन्धनमस्त्रपूगैः किं तस्य शत्रुहनने कपयः सहायाः॥}\nopagebreak\\
\raggedleft{–~भा॰पु॰~९.११.२०}\\
\begin{sloppypar}\hyphenrules{nohyphenation}\justifying\noindent यद्वा \textcolor{red}{सहायान् कपीश्वरानर्थयतेऽभि\-लषत्यन्वेष्टि वेति सहायार्थस्तं सहायार्थम्} इति विग्रहे णिजन्त\-\textcolor{red}{अर्थ्‌}\-धातोः (\textcolor{red}{अर्थ उपयाच्ञायाम्} धा॰पा॰~१९०६) कर्मण्यणि\footnote{\textcolor{red}{कर्मण्यण्} (पा॰सू॰~३.२.१) इत्यनेन।} दीर्घे विभक्ति\-कार्ये सिद्धमिदम्।\footnote{\textcolor{red}{अर्थ उपयाच्ञायाम्} (धा॰पा॰~१९०६)~\arrow \textcolor{red}{सत्याप\-पाश\-रूप\-वीणा\-तूल\-श्लोक\-सेना\-लोम\-त्वच\-वर्म\-वर्ण\-चूर्ण\-चुरादिभ्यो णिच्} (पा॰सू॰~३.१.२५)~\arrow अर्थ~णिच्~\arrow अर्थ~इ~\arrow \textcolor{red}{अतो लोपः} (पा॰सू॰~६.४.४८)~\arrow अर्थ्~इ~\arrow अर्थि~\arrow \textcolor{red}{सनाद्यन्ता धातवः} (पा॰सू॰~३.१.३२)~\arrow धातुसञ्ज्ञा। सहाय~शस्~अर्थि~\arrow \textcolor{red}{कर्मण्यण्} (पा॰सू॰~३.२.१)~\arrow सहाय~शस्~अर्थि~अण्~\arrow सहाय~शस्~अर्थि~अ~\arrow \textcolor{red}{णेरनिटि} (पा॰सू॰~६.४.५१)~\arrow सहाय~शस्~अर्थ्~अ~\arrow सहाय~शस्~अर्थ~\arrow \textcolor{red}{उपपदमतिङ्} (पा॰सू॰~२.२.१९)~\arrow \textcolor{red}{कृत्तद्धित\-समासाश्च} (पा॰सू॰~१.२.४६)~\arrow प्रातिपदिकसञ्ज्ञा~\arrow \textcolor{red}{सुपो धातु\-प्रातिपदिकयोः} (पा॰सू॰~२.४.७१)~\arrow सहाय~अर्थ~\arrow \textcolor{red}{अकः सवर्णे दीर्घः} (पा॰सू॰~६.१.१०१)~\arrow सहायार्थ~\arrow विभक्तिकार्यम्~\arrow सहायार्थ~अम्~\arrow \textcolor{red}{अमि पूर्वः} (पा॰सू॰~६.१.१०७)~\arrow सहायार्थम्।}\end{sloppypar}
\begin{sloppypar}\hyphenrules{nohyphenation}\justifying\noindent\hspace{10mm} \end{sloppypar}
\section[पुत्रीयम्]{पुत्रीयम्}
\centering\textcolor{blue}{गृहाण पायसं दिव्यं पुत्रीयं देवनिर्मितम्।\nopagebreak\\
लप्स्यसे परमात्मानं पुत्रत्वेन न संशयः॥}\nopagebreak\\
\raggedleft{–~अ॰रा॰~१.३.८}\\
\begin{sloppypar}\hyphenrules{nohyphenation}\justifying\noindent\hspace{10mm} अत्र \textcolor{red}{पुत्रे भवं पुत्रीयम्} इति भवार्थे \textcolor{red}{छ}प्रत्यये\footnote{\textcolor{red}{तत्र भवः} (पा॰सू॰~४.३.५३) इत्यनेन।} तस्य चेयादेशे\footnote{\textcolor{red}{आयनेयीनीयियः फढखच्छघां प्रत्ययादीनाम्‌} (पा॰सू॰~७.१.२) इत्यनेन।} विभक्ति\-कार्ये \textcolor{red}{पुत्रीयम्} इति। किन्तु भवार्थस्यानुपयोगादत्र \textcolor{red}{पुत्राय हितम्} इति विग्रहे \textcolor{red}{छ}\-प्रत्यये\footnote{\textcolor{red}{तस्मै हितम्} (पा॰सू॰~५.१.५) इत्यनेन।} शब्द\-सिद्धिः। यद्वा \textcolor{red}{आत्मनः पुत्रमिच्छति} इति विग्रहे \textcolor{red}{सुप आत्मनः क्यच्} (पा॰सू॰~३.१.८) इत्यनेन \textcolor{red}{क्यच्}प्रत्यये \textcolor{red}{क्यचि च} (पा॰सू॰~७.४.३३) इत्यनेन दीर्घ ईकारे धातु\-सञ्ज्ञायां\footnote{\textcolor{red}{सनाद्यन्ता धातवः} (पा॰सू॰~३.१.३२) इत्यनेन।} लटि तिपि शपि \textcolor{red}{पुत्रीयति}।\footnote{यथा भट्टिकाव्ये~– \textcolor{red}{पुत्रीयता तेन वराङ्गनाभिरानायि विद्वान् क्रतुषु क्रियावान्} (भ॰का॰~१.१०) इत्यत्र शत्रन्त\-प्रयोगे।} \textcolor{red}{पुत्रीयतीति पुत्रीयः} इति विग्रहे पचादित्वादच्।\footnote{\textcolor{red}{नन्दि\-ग्रहि\-पचादिभ्यो ल्युणिन्यचः} (पा॰सू॰~३.१.१३४) इत्यनेन।} तच्च भावे।
एवं तदस्त्यस्मिन्निति विग्रहेऽच्प्रत्यये\footnote{\textcolor{red}{अर्शआदिभ्योऽच्} (पा॰सू॰~५.२.१२७) इत्यनेन।} \textcolor{red}{पुत्रीयम्}। अथवा \textcolor{red}{पुत्री} इति पृथक्पदम्।\footnote{यथा रामरक्षास्तोत्रे~– \textcolor{red}{स चिरायुः सुखी पुत्री विजयी विनयी भवेत्} (रा॰र॰स्तो॰~१०)। पुत्र~\arrow \textcolor{red}{अत इनिठनौ} (पा॰सू॰~५.२.११५)~\arrow पुत्र~इनिँ~\arrow पुत्र~इन्~\arrow \textcolor{red}{यचि भम्} (पा॰सू॰~१.४.१८)~\arrow भसञ्ज्ञा~\arrow \textcolor{red}{यस्येति च} (पा॰सू॰~६.४.१४८)~\arrow पुत्र्~इन्~\arrow पुत्रिन्~\arrow विभक्ति\-कार्यम्~\arrow पुत्रिन्~सुँ~\arrow पुत्रिन्~स्~\arrow \textcolor{red}{सौ च} (पा॰सू॰~६.४.१३)~\arrow पुत्रीन्~स्~\arrow \textcolor{red}{हल्ङ्याब्भ्यो दीर्घात्सुतिस्यपृक्तं हल्} (पा॰सू॰~६.१.६८)~\arrow पुत्रीन्~\arrow \textcolor{red}{नलोपः प्रातिपदिकान्तस्य} (पा॰सू॰~८.२.७)~\arrow पुत्री।} अग्निर्यज्ञ\-स्थले प्रादुर्भूय महाराजं दशरथं कथयति यत् \textcolor{red}{त्वं पुत्री दिव्यं पायसं गृहाण यं गृहीत्वा पुत्रत्वोपलक्षितं परमात्मानं लप्स्यसे}।\end{sloppypar}
\section[इत्युक्ते]{इत्युक्ते}
\centering\textcolor{blue}{वृणीष्व वरमित्युक्ते त्वं मे पुत्रो भवामल।\nopagebreak\\
इति त्वया याचितोऽसौ भगवान्भूतभावनः॥}\nopagebreak\\
\raggedleft{–~अ॰रा॰~१.४.१६}\\
\begin{sloppypar}\hyphenrules{nohyphenation}\justifying\noindent\hspace{10mm} अत्राध्यात्म\-रामायणे बाल\-काण्डे चतुर्थ\-सर्गे विश्वामित्रो यज्ञ\-रक्षार्थं श्रीरामं दशरथमयाचत्। तत्र पुत्र\-वात्सल्य\-धिया मही\-पतिना महा\-मुनिः विश्वामित्रः प्रत्याख्यातः। पश्चाद्वसिष्ठो महा\-राजं प्रत्यबोधयद्यत्पुरा त्वया बहु तप्तं\footnote{\textcolor{red}{भवन्तौ तप उग्रं वै तेपाथे बहुवत्सरम्} (अ॰रा॰~१.४.१४)।} तदा \textcolor{red}{वृणीष्व वरम्} इति भगवत्युक्तवति त्वया भगवान् पुत्र\-रूपेण याचितः स एव परमात्मा श्रीराम\-रूपेण तव गृहेऽवातरत्। इह \textcolor{red}{उक्ते} इति भगवद्विशेषणम् \textcolor{red}{उक्तवति} इत्यभिप्रायकम्। प्रायो निष्ठा\-सञ्ज्ञकौ द्वौ प्रत्ययौ क्तः क्तवतुश्च।\footnote{\textcolor{red}{क्तक्तवतू निष्ठा} (पा॰सू॰~१.१.२६)।} क्त\-प्रत्ययः प्रायशः कर्मणि भावे च भवति।\footnote{\textcolor{red}{अदिकर्मणि क्तः कर्तरि च} (पा॰सू॰~३.४.७१) \textcolor{red}{गत्यर्थाकर्मक\-श्लिष\-शीङ्स्थास\-वस\-जन\-रुह\-जी\-र्यतिभ्यश्च} (पा॰सू॰~३.४.७२) इत्याभ्यां कर्तर्यपि भवति।} तथा च सूत्रम् \textcolor{red}{तयोरेव कृत्य\-क्त\-खलर्थाः} (पा॰सू॰~३.४.७०)। क्तवत्तु\-प्रत्ययः कर्तर्येव भवति।\footnote{\textcolor{red}{कर्तरि कृत्} (पा॰सू॰~३.४.६७) इत्यनेन।} किन्त्वत्र \textcolor{red}{उक्ते} इति कथमिति चेत्। \textcolor{red}{ब्रू}\-धातोः (\textcolor{red}{ब्रूञ् व्यक्तायां वाचि} धा॰पा॰~१०४४) भावे \textcolor{red}{क्त}\-प्रत्ययः।\footnote{\textcolor{red}{नपुंसके भावे क्तः} (पा॰सू॰~३.३.११४) इत्यनेन।} \textcolor{red}{ब्रुवो वचिः} (पा॰सू॰~२.४.५३) इत्यनेन \textcolor{red}{वच्} आदेशे \textcolor{red}{वचि\-स्वपि\-यजादीनां किति} (पा॰सू॰~६.१.१५) इत्यनेन सम्प्रसारणे \textcolor{red}{चोः कुः} (पा॰सू॰~८.२.३०) इत्यनेन ककारे विभक्ति\-कार्ये \textcolor{red}{उक्तम्}। तदस्त्यस्मिन्निति \textcolor{red}{उक्तम्}। अर्शआदित्वादच्।\footnote{\textcolor{red}{अर्शआदिभ्योऽच्} पा॰सू॰~५.२.१२७) इत्यनेन।} तस्मिन्नुक्ते। इत्थं मत्वर्थीयेनाच्प्रत्ययेन \textcolor{red}{उक्तवति} इत्यभिप्रायकः \textcolor{red}{उक्ते} इति शब्दोऽपि पाणिनीय एव। यद्वा \textcolor{red}{उक्तमाचष्टे करोति वा} इति विग्रहे \textcolor{red}{तत्करोति तदाचष्टे} (धा॰पा॰ ग॰सू॰~१८७) इति णिचि तिपि शपि \textcolor{red}{उक्तयति}।\footnote{उक्त~\arrow \textcolor{red}{तत्करोति तदाचष्टे} (धा॰पा॰ ग॰सू॰~१८७)~\arrow उक्त~णिच्~\arrow उक्त~इ~\arrow \textcolor{red}{णाविष्ठवत्प्राति\-पदिकस्य पुंवद्भाव\-रभाव\-टिलोप\-यणादि\-परार्थम्} (वा॰~६.४.४८)~\arrow उक्त्~इ~\arrow उक्ति~\arrow \textcolor{red}{सनाद्यन्ता धातवः} (पा॰सू॰~३.१.३२)~\arrow धातुसञ्ज्ञा~\arrow \textcolor{red}{शेषात्कर्तरि परस्मैपदम्} (पा॰सू॰~१.३.७८)~\arrow \textcolor{red}{वर्तमाने लट्} (पा॰सू॰~३.२.१२३)~\arrow उक्ति~लट्~\arrow उक्ति~तिप्~\arrow उक्ति~ति~\arrow \textcolor{red}{कर्तरि शप्‌} (पा॰सू॰~३.१.६८)~\arrow उक्ति~शप्~ति~\arrow उक्ति~अ~ति~\arrow \textcolor{red}{सार्वधातुकार्ध\-धातुकयोः} (पा॰सू॰~७.३.८४)~\arrow उक्ते~अ~ति~\arrow \textcolor{red}{एचोऽयवायावः} (पा॰सू॰~६.१.७८)~\arrow उक्तय्~अ~ति~\arrow उक्तयति।} \textcolor{red}{उक्तयतीत्युक्तः}। कर्तर्यच्।\footnote{उक्ति~\arrow पूर्ववद्धातु\-सञ्ज्ञा~\arrow \textcolor{red}{नन्दि\-ग्रहि\-पचादिभ्यो ल्युणिन्यचः} (पा॰सू॰~३.१.१३४)~\arrow उक्ति~अच्~\arrow अनुबन्धलोपः~\arrow उक्ति~अ~\arrow \textcolor{red}{णेरनिटि} (पा॰सू॰~६.४.५१)~\arrow णिलोपः~\arrow उक्त्~अ~\arrow उक्त~\arrow विभक्तिकार्यम्~\arrow उक्तः।} तस्मिन् \textcolor{red}{उक्ते}। यद्वा \textcolor{red}{उक्ते} इति कर्मण्येव प्रत्ययः। \textcolor{red}{वृणीष्व वरमिति भगवतोक्ते त्वयि} इति न दोषः।\footnote{\textcolor{red}{त्वयि} इत्यध्याहार्यमिति भावः।} अथवा \textcolor{red}{वाक्य उक्ते}।\footnote{\textcolor{red}{वाक्ये} इत्यध्याहार्यमिति भावः।}\end{sloppypar}
\section[प्रमुदितान्तरः]{प्रमुदितान्तरः}
\centering\textcolor{blue}{वसिष्ठेनैवमुक्तस्तु राजा दशरथस्तदा।\nopagebreak\\
कृतकृत्यमिवात्मानं मेने प्रमुदितान्तरः॥}\nopagebreak\\
\raggedleft{–~अ॰रा॰~१.४.२१}\\
\begin{sloppypar}\hyphenrules{nohyphenation}\justifying\noindent\hspace{10mm} श्रीरघुनाथस्याऽध्यात्मिक\-रहस्यं वसिष्ठ\-मुखान्निशम्य महा\-राजो मुमुदे। अत्र प्रयोगो वर्तते \textcolor{red}{प्रमुदितान्तरः}। \textcolor{red}{प्रमुदितमन्तरं यस्य स प्रमुदितान्तरः} इति।
\textcolor{red}{प्रामोद्यत} इति \textcolor{red}{प्रमोदितम्}। अयं भूत\-कालानुसारेण विग्रहः। णिचि \textcolor{red}{पुगन्त\-लघूपधस्य च} (पा॰सू॰~७.३.८६) इत्यनेन गुणे धातुसञ्ज्ञायां क्त\-प्रत्यये सति \textcolor{red}{आर्धधातुकस्येड्वलादेः} (पा॰सू॰~७.२.३५) इत्यनेनेटि कृते णिलोपे विभक्ति\-कार्ये \textcolor{red}{प्रमोदितम्} इत्येव पाणिनीयम्।\footnote{यथा \textcolor{red}{अनुमोदितम् आमोदितम्} इत्यादिषु। वाचस्पत्येऽपि~– \textcolor{red}{अनुमोदित। त्रि॰ अनु मुद-णिच्-कर्म्मणि क्तः}। \textcolor{red}{आमोदित। त्रि॰ आ मुद-णिच्-क्तः}। \textcolor{red}{मुदँ हर्षे} (धा॰पा॰~१६)~\arrow मुद्~\arrow \textcolor{red}{हेतुमति च} (पा॰सू॰~३.१.२६)~\arrow मुद्~णिच्~\arrow मुद्~इ~\arrow \textcolor{red}{पुगन्त\-लघूपधस्य च} (पा॰सू॰~७.३.८६)~\arrow मोद्~इ~\arrow मोदि~\arrow \textcolor{red}{सनाद्यन्ता धातवः} (पा॰सू॰~३.१.३२)~\arrow धातु\-सञ्ज्ञा। प्र~मोदि~\arrow \textcolor{red}{तयोरेव कृत्य\-क्तखलर्थाः} (पा॰सू॰~३.४.७०)~\arrow \textcolor{red}{निष्ठा} (पा॰सू॰~३.२.१०२)~\arrow प्र~मोदि~क्त~\arrow प्र~मोदि~त~\arrow \textcolor{red}{आर्धधातुकस्येड्वलादेः} (पा॰सू॰~७.२.३५)~\arrow \textcolor{red}{आद्यन्तौ टकितौ} (पा॰सू॰~१.१.४६)~\arrow प्र~मोदि~इट्~त~\arrow प्र~मोदि~इ~त~\arrow \textcolor{red}{निष्ठायां सेटि} (पा॰सू॰~६.४.५२)~\arrow प्र~मोद्~इ~त~\arrow प्रमोदित~\arrow विभक्तिकार्यम्~\arrow प्रमोदित~सुँप्~\arrow \textcolor{red}{अतोऽम्} (पा॰सू॰~७.१.२४)~\arrow प्रमोदित~अम्~\arrow \textcolor{red}{अमि पूर्वः} (पा॰सू॰~६.१.१०७)~\arrow प्रमोदितम्।}
न च \textcolor{red}{पुगन्तलघूपधस्य च} इति गुणोऽत्र न
प्रवर्त्स्यत्यतो नापाणिनीयता।\footnote{\textcolor{red}{प्रवर्त्स्यति} इत्यत्र \textcolor{red}{वृद्भ्यः स्यसनोः} (पा॰सू॰~१.३.९२) इत्यनेन परस्मैपदम्।} यतो हि \textcolor{red}{सार्वधातुकार्ध\-धातुकयोः} (पा॰सू॰~७.३.८४) इति सम्पूर्णस्य सूत्रस्यात्रानुवृत्तिस्तत्र च सप्तमी। एवं \textcolor{red}{मिदेर्गुणः} (पा॰सू॰~७.३.८२) इत्यतो \textcolor{red}{गुण}\-पदमनुवर्त्यते। तथा च \textcolor{red}{इको गुण\-वृद्धी} (पा॰सू॰~१.१.३) इति सूत्रेण षष्ठ्यन्तम् \textcolor{red}{इकः} इति पदमप्युपतिष्ठते। यतो हि षड्विध\-सूत्रेषु \textcolor{red}{इको गुण\-वृद्धी} इति सूत्रं परिभाषा\-सूत्रम्। \textcolor{red}{परिभाषा नामानियमे नियमकारिणी}। अर्थाद्गुणः कुत्र भवेदित्यनियमे सतीयं नियमयति यदिक एव गुणः स्यात्। तथा हि सूत्रार्थः \textcolor{red}{गुण\-वृद्धि\-शब्दाभ्यां यत्र गुण\-वृद्धी विधीयेते तत्र ‘इक’ इति षष्ठ्यन्तं पदमुपतिष्ठते} (वै॰सि॰कौ॰~३४)। \textcolor{red}{गुण\-वृद्धि\-शब्दाभ्याम्} इत्यत्र हि पञ्चमी। सा च \textcolor{red}{गुण\-वृद्धि\-शब्दावुच्चार्य} इत्यर्थे। उच्चार्य\-रूप\-ल्यबन्त\-कर्मणि \textcolor{red}{गुण\-वृद्धि\-शब्द}\-इत्यत्र \textcolor{red}{ल्यब्लोपे कर्मण्यधिकरणे च} (पा॰सू॰~२.३.२८) इति वार्त्तिकेन \textcolor{red}{श्वशुराज्जिह्रेति} (वै॰सि॰कौ॰~५९४) इतिवत्पञ्चमी। अर्थात् \textcolor{red}{गुण\-वृद्धी} उच्चार्य यत्र गुण\-वृद्धी विधीयेते तत्र \textcolor{red}{इकः} इति षष्ठ्यन्तमुपतिष्ठते। एवं गुण\-पद\-प्रयोज्य\-विधेयताश्रय\-भूते गुणे विधीयमाने तथा वृद्धि\-पद\-प्रयोज्य\-विधेयताश्रय\-भूतायां वृद्धौ विधीयमानायाम् \textcolor{red}{इकः} इति षष्ठ्यन्तं पदमुपतिष्ठत इति। अर्थात् \textcolor{red}{मिदेर्गुणः} (पा॰सू॰~७.३.८२) इति\-सूत्रादनुवृत्त\-लघूपध\-गुण\-विधायक\-सूत्रस्थ\-गुण\-पद\-प्रयोज्य\-विधेयताश्रयो गुणो विधीयतेऽतः षष्ठ्यन्तम् \textcolor{red}{इकः} इति पदमुपतिष्ठते। तस्य च \textcolor{red}{लघूपधस्य} इत्यनेनान्वये षष्ठी\-समभिव्याहारादवयवावयवि\-भाव\-सम्बन्धः। एवं \textcolor{red}{सार्वधातुकार्धधातुकाव्यवहित\-पूर्वत्व\-विशिष्टस्य पुगन्त\-लघूपधाङ्गावयवस्येकः स्थाने गुणः} अयं हि परिष्कृत\-सूत्रार्थः। तथा चात्र \textcolor{red}{णिच्} इत्यार्धधातुक\-सञ्ज्ञः। \textcolor{red}{आर्धधातुकं शेषः} (पा॰सू॰~३.४.११४) इति सूत्रविहिता सा सञ्ज्ञा। तथा च \textcolor{red}{प्रमुद्~इ~त}\footnote{\textcolor{red}{प्रमुद्~णिच्~क्त} इति भावः।} इत्यत्र \textcolor{red}{इक्} मकारानन्तरमुकारः। किन्तु तदार्धधातुकाव्यवहितं नास्ति दकारस्य व्यवधानात्। अत एव गुणाप्राप्तौ \textcolor{red}{प्रमुदित} इति कामं पाणिनीय इति चेत्। \textcolor{red}{येन नाव्यवधानं तद्व्यवहितेऽपि वचन\-प्रामाण्यात्} (भा॰पा॰सू॰~७.२.३, ७.३.४४, ७.३.५४, ७.४.१, ७.४.९३)। \textcolor{red}{येन} इति कर्तरि तृतीया। अर्थात् \textcolor{red}{यत्कर्तृकमव्यवधानं न तेन व्यवहितेऽपि वचन\-प्रामाण्याद्गुणो भवेत्}। यतो ह्ययं लघूपध\-गुणः।
\textcolor{red}{उपधात्वं नाम समुदाय\-विशिष्टवर्णत्वम्}।\footnote{\textcolor{red}{समुदाय\-विशिष्ट\-वर्णत्वमुपधात्वम्} (ल॰शे॰)।}
\textcolor{red}{वैशिष्ट्यञ्च स्वघटकत्व\-स्वघटकान्त्यालवधिक\-पूर्वत्व\-विशिष्टत्वम्} इति।\footnote{\textcolor{red}{वैशिष्ट्यञ्च स्वघटकत्व\-स्वघटकान्त्यालवधिक\-पूर्वत्वोभयसम्बन्धेन} (ल॰शे॰)।} एतदुभय\-सम्बन्धेन। तथा च सूत्रम् \textcolor{red}{अलोऽन्त्यात्पूर्व उपधा} (पा॰सू॰~१.१.६५)। \textcolor{red}{अन्त्यादलः पूर्वो वर्ण उपधा\-सञ्ज्ञः स्यात्} (वै॰सि॰कौ॰~२४९)। एवं \textcolor{red}{लघ्व्युपधा लघूपधा पुगन्तञ्च लघूपधा चेति} समाहार\-द्वन्द्वे तस्य च \textcolor{red}{अङ्गस्य} इत्यनेनावयवावयवि\-भाव\-सम्बन्धेनान्वय एवम् \textcolor{red}{अङ्गावयवं सार्वधातुकार्धधातुकाव्यवहित\-पूर्वत्व\-विशिष्टं यत्पुगन्तं लघूपधञ्च तस्येको गुणः} इति सुपरिष्कृतः सूत्रार्थः। तथा च लघूपध\-शब्द\-योगेन वर्ण\-मात्र\-व्यवधानं त्वनिवार्यमेव। तदभाव उपधा कुतः। यतो ह्यन्त्यादलः पूर्वस्थैव सा। अतः \textcolor{red}{येन नाव्यवधानं तेन व्यवहितेऽपि वचन\-प्रामाण्यात्} (वै॰सि॰कौ॰~२१८९) इति प्रवर्तते। एवं च \textcolor{red}{प्रमुद्~इ~त}\footnote{\textcolor{red}{प्रमुद्~णिच्~क्त} इति भावः।} इत्यत्र दकार\-कर्तृकं व्यवधानं त्वनिवार्यम्। अत एक\-वर्ण\-कर्तृक\-व्यवधानमत्र सोढव्यं \textcolor{red}{येन} इत्येक\-वचन\-प्रयोगात्। तेन \textcolor{red}{भिनत्ति} इत्यत्रानेक\-व्यवहित इकि न गुणः।\footnote{\textcolor{red}{अनेक\-व्यवहित इकि} इत्यत्र \textcolor{red}{यस्य च भावेन भावलक्षणम्‌} (पा॰सू॰~२.३.३७) इत्यनेन भावलक्षणा सप्तमी। अनेक\-व्यवहित इकि तस्येको न गुण इति भावः। \textcolor{red}{भिदिँर् विदारणे} (धा॰पा॰~१४३९)~\arrow भिद्~\arrow \textcolor{red}{शेषात्कर्तरि परस्मैपदम्} (पा॰सू॰~१.३.७८)~\arrow \textcolor{red}{वर्तमाने लट्} (पा॰सू॰~३.२.१२३)~\arrow भिद्~तिप्~\arrow भिद्~ति~\arrow \textcolor{red}{रुधादिभ्यः श्नम्} (पा॰सू॰~३.१.७८)~\arrow \textcolor{red}{मिदचोऽन्त्यात्परः} (पा॰सू॰~१.१.४७)~\arrow भि~श्नम्~द्~ति~\arrow भि~न~द्~ति~\arrow \textcolor{red}{खरि च} (पा॰सू॰~८.४.५५)~\arrow भि~न~त्~ति~\arrow भिनत्ति।} \textcolor{red}{भेत्ता} इत्यादौ गुणः।\footnote{\textcolor{red}{भिदिँर् विदारणे} (धा॰पा॰~१४३९)~\arrow भिद्~\arrow \textcolor{red}{ण्वुल्तृचौ} (पा॰सू॰~३.१.१३३)~\arrow भिद्~तृच्~\arrow भिद्~तृ~\arrow \textcolor{red}{पुगन्त\-लघूपधस्य च} (पा॰सू॰~७.३.८६)~\arrow भेद्~तृ~\arrow \textcolor{red}{खरि च} (पा॰सू॰~८.४.५५)~\arrow भेत्~तृ~\arrow भेत्तृ~\arrow विभक्तिकार्यम्~\arrow भेत्तृ~सुँ~\arrow भेत्तृ~स्~\arrow \textcolor{red}{ऋदुशनस्पुरुदंसोऽनेहसां च} (पा॰सू॰~७.१.९४)~\arrow भेत्त्~अनँङ्~स्~\arrow भेत्त्~अन्~स्~\arrow \textcolor{red}{अप्तृन्तृच्स्वसृ\-नप्तृ\-नेष्टृ\-त्वष्टृ\-क्षत्तृ\-होतृ\-पोतृ\-प्रशास्तॄणाम्} (पा॰सू॰~६.४.११)~\arrow भेत्त्~आन्~स्~\arrow \textcolor{red}{हल्ङ्याब्भ्यो दीर्घात्सुतिस्यपृक्तं हल्} (पा॰सू॰~६.१.६८)~\arrow भेत्त्~आन्~\arrow \textcolor{red}{नलोपः प्रातिपदिकान्तस्य} (पा॰सू॰~८.२.७)~\arrow भेत्त्~आ~\arrow भेत्ता।} इदं सर्वं सिद्धान्त\-कौमुद्यां भ्वादिप्रकरणे प्रपञ्चितम्।\footnote{\textcolor{red}{येन नाव्यवधानं तेन व्यवहितेऽपि। वचन\-प्रामाण्यात्। तेन भिनत्तीत्यादावनेक\-व्यवहितस्येको न गुणः} (वै॰सि॰कौ॰~२१८९)।} अतोऽत्र गुण\-प्राप्तिर्दुर्वारेति। अत्रोच्यते। वस्तुतोऽयं न णिजन्तोऽपि तु शुद्धात् \textcolor{red}{मुदँ हर्षे} (धा॰पा॰~१६) इति धातोः कर्तरि क्तान्तप्रयोगः।\footnote{\textcolor{red}{गत्यर्थाकर्मक\-श्लिष\-शीङ्स्थास\-वस\-जन\-रुह\-जीर्यतिभ्यश्च} (पा॰सू॰~३.४.७२) इत्यनेन कर्तरि क्तः। \textcolor{red}{हृष्टो मत्तस्तृप्तः प्रह्लन्नः प्रमुदितः प्रीतः} (अ॰को॰~३.१.१०३) इत्यमरः। तत्रैव सुधाटीकायां \textcolor{red}{प्रमुदितम्। प्रमोदते स्म} (अ॰को॰ व्या॰सु॰~३.१.१०३) इति भानुजि\-दीक्षिताः। कर्तरि \textcolor{red}{पुगन्त\-लघूपधस्य च} (पा॰सू॰~७.३.८६) इत्यनेन गुणे प्राप्ते \textcolor{red}{ग्क्ङिति च} (पा॰सू॰~१.१.५) इत्यनेन गुणनिषेधे \textcolor{red}{प्रमुदितम्} इत्येव। \textcolor{red}{उदुपधाद्भावादि\-कर्मणोरन्य\-तरस्याम्} (पा॰सू॰~१.२.२१) इत्यनेनाऽदिकर्मणि भावे च निष्ठाया वैकल्पिक\-कित्त्वात् \textcolor{red}{ग्क्ङिति च} (पा॰सू॰~१.१.५) इत्यनेन प्राप्तस्य गुणनिषेधस्यापि वैकल्पिकत्वात् \textcolor{red}{प्रमुदितम् प्रमोदितम्} इति रूपद्वयम्।} यद्वा नायं क्तान्तः प्रयोगोऽपि तु \textcolor{red}{प्रमुद्यत इति प्रमुत्}\footnote{यद्वा \textcolor{red}{मुत्प्रीतिः प्रमदो हर्षः प्रमोदामोदसम्मदाः} (अ॰को॰~१.४.२४) इत्यमरकोशानुशासनात् \textcolor{red}{प्रकृष्टा मुत् प्रमुत्}। यथा भागवते~– \textcolor{red}{तदङ्ग\-सङ्ग\-प्रमुदाकुलेन्द्रियाः} (भा॰पु॰~१०.३३.१८)। अत्रान्वितार्थ\-प्रकाशिका~–\textcolor{red}{तस्य भगवतः अङ्गसङ्गेन प्रकृष्टा या मुत् हर्षस्तया आकुलानि अवशानीन्द्रियाणी यासां ताः} (भा॰पु॰ अ॰प्र॰टी॰~१०.३३.१८)। एवमेव श्रीभार्गव\-राघवीये~– \textcolor{red}{वक्राणि मूर्ध्ना प्रमुदा वहन्ती} (भा॰रा॰~१३.९) \textcolor{red}{निशम्य पौराः प्रमुदा समाययुः} (भा॰रा॰~१७.११) इत्यादौ।} इति विग्रहे भावे \textcolor{red}{क्विप्‌}\-प्रत्यये\footnote{\textcolor{red}{सम्पदादिभ्‍यः क्विप्} (वा॰~३.३.१०८) इत्यनेन।} सर्वापहारि\-लोपे \textcolor{red}{सा सञ्जाताऽस्य} इति सञ्जातार्थ आकृति\-गणतया तारकादित्वात् \textcolor{red}{इतच्‌}\-प्रत्यये\footnote{\textcolor{red}{तदस्य सञ्जातं तारकादिभ्य इतच्} (पा॰सू॰~५.२.३६) इत्यनेन।} चकारानुबन्ध\-कार्ये विभक्ति\-कार्ये \textcolor{red}{प्रमुदितम्} इति। यद्वा \textcolor{red}{प्रमुदमितं प्रमुदितम्} इति विग्रहे \textcolor{red}{द्वितीया} इति योग\-विभागेन \textcolor{red}{द्वितीया श्रितातीत\-पतित\-गतात्यस्त\-प्राप्तापन्नैः} (पा॰सू॰~२.१.२४) इत्यनेन समासे \textcolor{red}{प्रमुदितम्} इति पाणिनीयमेव। \textcolor{red}{प्रमुदितमन्तरं यस्य प्रमुदितान्तरः} इति साधनिका\-प्रकारः।\end{sloppypar}
\section[क्षुत्क्षामादि]{क्षुत्क्षामादि}
\centering\textcolor{blue}{ददौ बलां चातिबलां विद्ये द्वे देवनिर्मिते।\nopagebreak\\
ययोर्ग्रहणमात्रेण क्षुत्क्षामादि न जायते॥}\nopagebreak\\
\raggedleft{–~अ॰रा॰~१.४.२५}\\
\begin{sloppypar}\hyphenrules{nohyphenation}\justifying\noindent\hspace{10mm} अत्र विश्वामित्रो यज्ञ\-रक्षार्थं सौमित्रि\-सहितं रण\-धीरं श्रीरघु\-वीरं सिद्धाश्रमं समानयन् सरयू\-तटे श्रीरामभद्राय बलामतिबलां चैव द्वे विद्ये प्रयच्छति स्म ययोर्ग्रहणमात्रेण क्षुत्पिपासे न लगतः।\footnote{\textcolor{red}{लगतः} इत्यत्र \textcolor{red}{लगेँ सङ्गे} (धा॰पा॰~७८६) इति धातुः।} तत्र \textcolor{red}{क्षुत्क्षामादि} इति प्रयोगो वर्तते। \textcolor{red}{क्षुच्च क्षामं चेति क्षुत्क्षामौ ते आदौ यस्य तत्} इति। एवं क्षाम\-शब्दो हि प्रायः कर्तरि\footnote{\textcolor{red}{क्षै क्षये} (धा॰पा॰~९१३) इत्यकर्मक\-धातोः \textcolor{red}{गत्यर्थाकर्मक\-श्लिष\-शीङ्स्थास\-वस\-जन\-रुह\-जीर्यतिभ्यश्च} (पा॰सू॰~३.४.७२) इत्यनेन कर्तरि \textcolor{red}{क्त}\-प्रत्यये \textcolor{red}{क्षाम}\-शब्दस्त्रिलिङ्गे। \textcolor{red}{क्षाम इति। ‘आदेचः’ इत्यात्वम्। ‘गत्यर्थाकर्मके’ति कर्तरि क्तः। क्षीण इत्यर्थः। अन्तर्भावितण्यर्थत्वे क्षपित इत्यर्थः} (बा॰म॰~३०३२) इति बालमनोरमा। यथा वाल्मीकीय\-रामायणे~– \textcolor{red}{तां क्षामां सुविभक्ताङ्गीं विनाभरणशोभिनीम्। प्रहर्षमतुलं लेभे मारुतिः प्रेक्ष्य मैथिलीम्॥} (वा॰रा॰~५.१५.३०) इति पाठे। अत्रत्या तिलक\-टीका~– \textcolor{red}{क्षामां कृशाम्} (वा॰रा॰ ति॰टी॰~५.१५.३०)। \textcolor{red}{क्षमाम्} इति गोविन्दराज\-सम्मतः पाठः। एवमेव भागवते \textcolor{red}{नातिक्षामं भगवतः स्निग्धापाङ्गावलोकनात्} (भा॰पु॰~३.२१.४६) \textcolor{red}{कालेन भूयसा क्षामां कर्शितां व्रतचर्यया} (भा॰पु॰~३.२३.५१) इत्यनयोः। तथा च मेघदूते \textcolor{red}{क्षामच्छायम्} (मे॰दू॰~२.१७) \textcolor{red}{मध्ये क्षामा} (मे॰दू॰~२.१९) \textcolor{red}{आधिक्षामाम्} (मे॰दू॰~२.२६) इत्यादिषु। प्रणेतॄणां भृङ्गदूताभिधे दूतकाव्येऽपि \textcolor{red}{श्यामां क्षामां क्षपितहृदिभूकुङ्कुमामश्रुधारा\-सारैर्नित्यं नमितवदनाम्भोरुहां रुग्णचित्ताम्। सूर्येन्दुभ्यामिव विरहितां कौहवीं सान्ध्यवेलां सीतां भीतामिव हरिणिकां द्रक्ष्यसि त्वं शुनीषु॥} (भृ॰दू॰~२.६१) \textcolor{red}{याऽयोध्यायां जननिसविधे वाग्भिरत्युज्ज्वलाभिर्नैच्छद्वस्तुं क्षणमपि तदा साधु सीताऽनुनीता। दूरीभूता मृगमृगयुतः साम्प्रतं सा मृगाक्षी क्षामा श्यामा श्वसिति किमहो कोटिकूटे त्रिकूटे॥} (भृ॰दू॰~२.१३७) इत्यनयोर्वृत्तयोः कर्तर्येव \textcolor{red}{क्षाम}\-शब्दः।} तर्हि कथं पिपासा\-वाचकः। अत्र \textcolor{red}{क्षै क्षये} (धा॰पा॰~९१३) इत्यस्माद्धातोर्भावे \textcolor{red}{क्त}\-प्रत्ययः।\footnote{\textcolor{red}{नपुंसके भावे क्तः} (पा॰सू॰~३.३.११४) इत्यनेन।} \textcolor{red}{आदेच उपदेशेऽशिति} (पा॰सू॰~६.१.४५) इत्यनेनाऽकारादेशे \textcolor{red}{क्षायो मः} (पा॰सू॰~८.२.५३) इत्यनेन मादेशे विभक्ति\-कार्ये \textcolor{red}{क्षामम्}।\end{sloppypar}
\section[क्रोधसम्मूर्च्छिता]{क्रोधसम्मूर्च्छिता}
\centering\textcolor{blue}{तच्छ्रुत्वाऽसहमाना सा ताटका घोररूपिणी।\nopagebreak\\
क्रोधसम्मूर्च्छिता राममभिदुद्राव मेघवत्॥}\nopagebreak\\
\raggedleft{–~अ॰रा॰~१.४.२९}\\
\begin{sloppypar}\hyphenrules{nohyphenation}\justifying\noindent\hspace{10mm} ताटकामवलोक्य श्रीरामभद्राय विश्वामित्र आदेशं ददौ। अग्नौ शलभ इव सा स्वयमेव क्रोध\-सम्मूर्च्छिता रघुनन्दनमभिदुद्राव। अत्र \textcolor{red}{क्रोधेन सम्मूर्च्छिता} इति प्रयोगोऽपाणिनीयो लगति। यतो हि \textcolor{red}{मुर्छाँ मोहस\-मुच्छ्राययोः} (धा॰पा॰~२१२) इति धातोः कर्तरि \textcolor{red}{क्त}\-प्रत्यये \textcolor{red}{राल्लोपः} (पा॰सू॰~६.४.२१) इत्यनेन रेफादङ्गस्य च्छस्य लोपे टापि \textcolor{red}{मूर्ता} इति पाणिनीयम्।\footnote{\textcolor{red}{मुर्छाँ मोह\-समुच्छ्राययोः} (धा॰पा॰~२१२)~\arrow मुर्छ्~\arrow \textcolor{red}{गत्यर्थाकर्मक\-श्लिष\-शीङ्स्थास\-वस\-जन\-रुह\-जी\-र्यतिभ्यश्च} (पा॰सू॰~३.४.७२)~\arrow मुर्छ्~क्त~\arrow मुर्छ्~त~\arrow \textcolor{red}{राल्लोपः} (पा॰सू॰~६.४.२१)~\arrow मुर्~त~\arrow \textcolor{red}{न ध्याख्यापॄमूर्छिमदाम्} (पा॰सू॰~८.२.५७)~\arrow नकारादेश\-निषेधः~\arrow \textcolor{red}{आदितश्च} (पा॰सू॰~७.२.१६)~\arrow इडागम\-निषेधः~\arrow \textcolor{red}{हलि च} (पा॰सू॰~८.२.७७)~\arrow मूर्~त~\arrow मूर्त~\arrow \textcolor{red}{अजाद्यतष्टाप्} (पा॰सू॰~४.१.४)~\arrow मूर्त~टाप्~\arrow मूर्त~आ~\arrow \textcolor{red}{अकः सवर्णे दीर्घः} (पा॰सू॰~६.१.१०१)~\arrow मूर्ता~\arrow विभक्तिकार्यम्~\arrow मूर्ता~सुँ~\arrow \textcolor{red}{हल्ङ्याब्भ्यो दीर्घात्सुतिस्यपृक्तं हल्} (पा॰सू॰~६.१.६८)~\arrow मूर्ता।} \textcolor{red}{मूर्च्छिता} इति कथम्। \textcolor{red}{मूर्च्छनं मूर्च्छा}।\footnote{\textcolor{red}{मुर्छाँ मोह\-समुच्छ्राययोः} (धा॰पा॰~२१२)~\arrow मुर्छ्~\arrow \textcolor{red}{षिद्भिदादिभ्योऽङ्} (पा॰सू॰~३.३.१०४)~\arrow मुर्छ्~अङ्~\arrow मुर्छ्~अ~\arrow \textcolor{red}{उपधायां च} (पा॰सू॰~८.२.७८)~\arrow मूर्छ्~अ~\arrow \textcolor{red}{अजाद्यतष्टाप्‌} (पा॰सू॰~४.१.४)~\arrow मूर्छ्~अ~आ~\arrow \textcolor{red}{अकः सवर्णे दीर्घः} (पा॰सू॰~६.१.१०१)~\arrow मूर्छ्~आ~\arrow मूर्छा~\arrow \textcolor{red}{अचो रहाभ्यां द्वे} (पा॰सू॰~८.४.४६)~\arrow मूर्~छ्~छा~\arrow \textcolor{red}{खरि च} (पा॰सू॰~८.४.५५)~\arrow मूर्~च्~छा~\arrow मूर्च्छा।} \textcolor{red}{मूर्च्छिता} इति तु \textcolor{red}{मूर्च्छाऽस्याः सञ्जाता सा मूर्च्छिता} इति विग्रहे \textcolor{red}{तदस्य सञ्जातं तारकादिभ्य इतच्} (पा॰सू॰~५.२.३६) इत्यनेन \textcolor{red}{इतच्} प्रत्यये भत्वाट्टिलोपे\footnote{\textcolor{red}{यचि भम्} (पा॰सू॰~१.४.१८) इत्यनेन भत्वे \textcolor{red}{यस्येति च} (पा॰सू॰~६.४.१४८) इत्यनेनाऽलोपः।} टापि \textcolor{red}{मूर्च्छिता}। यद्वा \textcolor{red}{मूर्च्छामिता मूर्च्छिता} इति विग्रहे द्वितीया\-तत्पुरुषे शकन्ध्वादित्वात्पर\-रूपे \textcolor{red}{मूर्च्छिता} इति।\footnote{यद्वाऽत्र \textcolor{red}{आदिकर्मणि क्तः कर्तरि च} (पा॰सू॰~३.४.७१) इत्यनेनाऽदिकर्मणि कर्तरि क्तः। ततः \textcolor{red}{विभाषा भावादिकर्मणोः} (पा॰सू॰~७.२.१७) इत्यनेन पाक्षिकेण्निषेध इट्पक्षे टापि विभक्तिकार्ये \textcolor{red}{मूर्च्छिता}। अपि च \textcolor{red}{घोररूपिणी} इत्यत्र \textcolor{red}{न कर्मधारयान्मत्वर्थीयो बहुव्रीहिश्चेत्तदर्थ\-प्रतिपत्ति\-करः} इत्यनेन न पाणिनीयतेति न भ्रमितव्यम्। \pageref{sec:ghorarupinah}तमे पृष्ठे \ref{sec:ghorarupinah} \nameref{sec:ghorarupinah} इति प्रयोगस्य विमर्शं पश्यन्तु।}\end{sloppypar}
\section[प्रस्थिता]{प्रस्थिताः}
\centering\textcolor{blue}{तत्र कामाश्रमे रम्ये कानने मुनिसङ्कुले।\nopagebreak\\
उषित्वा रजनीमेकां प्रभाते प्रस्थिताः शनैः॥}\nopagebreak\\
\raggedleft{–~अ॰रा॰~१.५.१}\\
\begin{sloppypar}\hyphenrules{nohyphenation}\justifying\noindent\hspace{10mm} कानने महर्षि\-विश्वामित्र\-महाभागैरेकां
रजनीमुषित्वा
प्रभाते प्रास्थायि। अत्र प्रोपसर्ग\-संयोजनेन गति\-निवृत्त्यर्थक\-\textcolor{red}{स्था}\-धातोः (\textcolor{red}{ष्ठा गतिनिवृत्तौ} धा॰पा॰~९२८) गत्यर्थकतयाऽकर्मकत्वाभावात्कर्तरि कथं \textcolor{red}{क्त}\-प्रत्ययः। एवं हि \textcolor{red}{प्रस्थिताः} इति स्था\-धातोः कर्तरि \textcolor{red}{क्त}\-प्रत्यये \textcolor{red}{घु\-मा\-स्था\-गा\-पा\-जहाति\-सां हलि} (पा॰सू॰~६.४.६६) इत्यनेनाऽकारस्येकारे विभक्तिकार्ये \textcolor{red}{प्रस्थिताः} इति चेत्। कर्मणोऽविवक्षायां \textcolor{red}{क्त}\-प्रत्ययः।\footnote{\textcolor{red}{धातोरर्थान्तरे वृत्तेर्धात्वर्थेनोपसङ्ग्रहात्। प्रसिद्धेरविवक्षातः कर्मणोऽकर्मिका क्रिया॥} (वा॰प॰~३.७.८८)। अकर्मकत्वात् \textcolor{red}{गत्यर्थाकर्मक\-श्लिष\-शीङ्स्थाऽऽस\-वस\-जन\-रुह\-जीर्यतिभ्यश्च} (पा॰सू॰~३.४.७२) इत्यनेन कर्तरि क्तः।} यद्वा गत्यर्थकतयाऽपि \textcolor{red}{क्त}\-प्रत्यय आपत्त्यभावः।\footnote{सोऽपि \textcolor{red}{गत्यर्थाकर्मक\-श्लिष\-शीङ्स्थाऽऽस\-वस\-जन\-रुह\-जीर्यतिभ्यश्च} (पा॰सू॰~३.४.७२) इत्यनेन।}\end{sloppypar}
\section[कुतः]{कुतः}
\centering\textcolor{blue}{दर्शयस्व महाभाग कुतस्तौ राक्षसाधमौ।\nopagebreak\\
तथेत्युक्त्वा मुनिर्यष्टुमारेभे मुनिभिः सह॥}\nopagebreak\\
\raggedleft{–~अ॰रा॰~१.५.४}\\
\begin{sloppypar}\hyphenrules{nohyphenation}\justifying\noindent\hspace{10mm} अत्र श्रीरामभद्रो राक्षसयोः सुबाहु\-मारीचयोः स्थितिं जिज्ञासते। \textcolor{red}{कुत्र तौ राक्षसाधमौ} इति प्रष्टव्ये \textcolor{red}{कुतः} इति पृच्छति। \textcolor{red}{कुतः} इत्यत्र \textcolor{red}{कस्मात्} इति विग्रहे \textcolor{red}{पञ्चम्यास्तसिल्} (पा॰सू॰~५.३.७) इति तसिलन्त\-प्रयोगः। सा च पञ्चमी विश्लेष\-मूलिका। अत्र विश्लेषाभावेऽपि कथं पञ्चमीति चेत्। अत्र \textcolor{red}{विवक्षाधीनानि कारकाणि भवन्ति}\footnote{मूलं मृग्यम्। यद्वा \textcolor{red}{कर्मादीनामविवक्षा शेषः} (भा॰पा॰सू॰~२.३.५०, २.३.५२, २.३.६७) इत्यस्य तात्पर्यमिदम्।} इति वचनेन पञ्चम्यन्त\-विवक्षया न दोषः। \textcolor{red}{विवक्षाधीनानि कारकाणि भवन्ति} इति वचने किं मानमिति चेत् \textcolor{red}{कारके} (पा॰सू॰~१.४.२३) इति हि सूत्रम्। अत्र प्रथमार्थे सप्तमीति भाष्य\-निर्देशात्\footnote{\textcolor{red}{किमिदं ‘कारके’ इति। सञ्ज्ञानिर्देशः} (भा॰पा॰सू॰~१.४.२३)। अत्र कैयटः~– \textcolor{red}{सञ्ज्ञानिर्देश इति। सुपां सुपो भवन्तीति प्रथमायाः स्थाने सप्तमी कृतेति भावः} (भा॰प्र॰ पा॰सू॰~१.४.२३)।} सूत्रकार\-प्रयोग एव मानम्। अथवाऽवध्यवधिमतोः\footnote{\textcolor{red}{उत्तरादिभ्य एनब्वा स्यादवध्यवधिमतोः सामीप्ये पञ्चमीं विना} (वै॰सि॰कौ॰~१९८४, ५.३.३५)।} \textcolor{red}{कुतः स्थानात्समीपं राक्षसाधमौ} इत्यर्थेऽन्तिक\-शब्द\-योगे पञ्चमी।\footnote{ \textcolor{red}{दूरान्तिकार्थैः षष्ठ्यन्यतरस्याम्} (पा॰सू॰~२.३.३४) इत्यनेन पञ्चमी।}
अथवा \textcolor{red}{कुं पृथ्वीं तनुतः कृशां कुरुत इति कुतः} इति विग्रहः। \textcolor{red}{कु}\-उपपदे \textcolor{red}{तन्‌}\-धातोः (\textcolor{red}{तनुँ विस्तारे} धा॰पा॰~१४६३) औणादिके \textcolor{red}{ड्विन्} प्रत्यये\footnote{\textcolor{red}{कार्याद्विद्यादनूबन्धम्} (भा॰पा॰सू॰~३.३.१) \textcolor{red}{केचिदविहिता अप्यूह्याः} (वै॰सि॰कौ॰~३१६९) इत्यनुसारमूह्योऽ\-त्राविहितो \textcolor{red}{ड्विन्‌}\-प्रत्ययः।} डित्त्व\-सामर्थ्यादभस्यापि टेर्लोपे\footnote{\textcolor{red}{डित्यभस्याप्यनु\-बन्धकरण\-सामर्थ्यात्} (वा॰~६.४.१४३)।} सर्वापहारि\-लोपे च \textcolor{red}{कुत्}। तस्य द्विवचनान्तं रूपं \textcolor{red}{कुतौ}। \textcolor{red}{व्यत्ययो बहुलम्} (पा॰सू॰~३.१.८५) इत्यनेन \textcolor{red}{औ}\-विभक्ति\-स्थाने \textcolor{red}{ङस्‌}\-विभक्तौ \textcolor{red}{कुतः} इति छान्दस\-रूपम्।\footnote{\textcolor{red}{सुप्तिङुपग्रह\-लिङ्गनराणां कालहलच्स्वर\-कर्तृयङां च। व्यत्ययमिच्छति शास्त्रकृदेषां सोऽपि च सिध्यति बाहुलकेन॥} (भा॰पा॰सू॰~३.१.८५)। \textcolor{red}{बहुलग्रहणं सर्वविधि\-व्यभिचारार्थम्} (का॰वृ॰~३.१.८५)।} अर्थात् \textcolor{red}{वसुमतीं कृशयतो राक्षसाधमौ दर्शयस्व}।
यद्वाऽत्र नास्ति \textcolor{red}{तसिल्} अपि तु \textcolor{red}{तसि\-प्रकरण आद्यादिभ्य उपसङ्ख्यानम्} (वा॰~५.४.४४) इत्यनेन सप्तमीतस्तसि प्रत्ययः।\footnote{तसेः सार्व\-विभक्तिकत्वं तदन्तानामाकृति\-गणत्वं च \pageref{fn:yatah}तमे पृष्ठे \ref{fn:yatah}तम्यां पादटिप्पण्यां स्पष्टीकृतम्।
}\end{sloppypar}
\section[त्वरितम्]{त्वरितम्}
\centering\textcolor{blue}{कदाचिन्मुनिवेषेण गौतमे निर्गते गृहात्।\nopagebreak\\
धर्षयित्वाऽथ निरगात्त्वरितं मुनिरप्यगात्॥}\nopagebreak\\
\raggedleft{–~अ॰रा॰~१.५.२२}\\
\begin{sloppypar}\hyphenrules{nohyphenation}\justifying\noindent\hspace{10mm} न चात्र \textcolor{red}{तूर्णम्} इति प्रयोक्तव्ये \textcolor{red}{त्वरितम्} इति प्रयुक्तम्।\footnote{पूर्वपक्षोऽयम्।}
अत्र \textcolor{red}{आदितश्च} (पा॰सू॰~७.२.१६) इत्यनेनेण्निषेधे प्राप्ते \textcolor{red}{रुष्यमत्वर\-सङ्घुषास्वनाम्} (पा॰सू॰~७.२.२८) इत्यनेन निषेधविकल्पः। तस्मात् \textcolor{red}{तूर्णम्} \textcolor{red}{त्वरितम्} इति रूपद्वयम्।\footnote{\textcolor{red}{ञित्वराँ सम्भ्रमे} (धा॰पा॰~७७५)~\arrow तवर्~\arrow \textcolor{red}{नपुंसके भावे क्तः} (पा॰सू॰~३.३.११४)~\arrow त्वर्~क्त~\arrow त्वर्~त~\arrow \textcolor{red}{आर्धधातुकस्येड्वलादेः} (पा॰सू॰~७.२.३५)~\arrow इट्प्राप्तिः~\arrow \textcolor{red}{आदितश्च} (पा॰सू॰~७.२.१६)~\arrow इण्निषेधः~\arrow \textcolor{red}{रुष्यमत्वर\-सङ्घुषास्वनाम्} (पा॰सू॰~७.२.२८)~\arrow वैकल्पिकेण्निषेधः। निषेधपक्षे~– त्वर्~त~\arrow \textcolor{red}{ज्वरत्वर\-श्रिव्यविम\-वामुपधायाश्च} (पा॰सू॰~६.४.२०)~\arrow त्~ऊठ्~र्~त~\arrow त्~ऊ~र्~त~\arrow \textcolor{red}{रदाभ्यां निष्ठातो नः पूर्वस्य च दः} (पा॰सू॰~८.२.४२)~\arrow त्~ऊ~र्~न~\arrow \textcolor{red}{रषाभ्यां नो णः समानपदे} (पा॰सू॰~८.४.१)~\arrow त्~ऊ~र्~ण~\arrow तूर्ण~\arrow विभक्तिकार्यम्~\arrow तूर्णम्। इट्पक्षे~– त्वर्~इट्~त~\arrow तवर्~इ~त~\arrow त्वरित~\arrow विभक्तिकार्यम्~\arrow त्वरितम्।} यद्वा
\textcolor{red}{त्वरामाचष्टे त्वरयति}\footnote{\textcolor{red}{घटादयः षितः} (धा॰पा॰ ग॰सू॰) इत्यनेन \textcolor{red}{त्वर्‌}\-धातोः षित्त्वम्। त्वर्~\arrow \textcolor{red}{षिद्भिदादिभ्योऽङ्} (पा॰सू॰~३.३.१०४)~\arrow त्वर्~अङ्~\arrow तवर्~अ~\arrow \textcolor{red}{अजाद्यतष्टाप्‌} (पा॰सू॰~४.१.४)~\arrow त्वर्~अ~आ~\arrow \textcolor{red}{अकः सवर्णे दीर्घः} (पा॰सू॰~६.१.१०१)~\arrow त्वर्~आ~\arrow त्वरा। त्वरा~\arrow \textcolor{red}{तत्करोति तदाचष्टे} (धा॰पा॰ ग॰सू॰)~\arrow त्वरा~णिच्~\arrow त्वरा~इ~\arrow \textcolor{red}{णाविष्ठवत्प्राति\-पदिकस्य पुंवद्भाव\-रभाव\-टिलोप\-यणादि\-परार्थम्} (वा॰~६.४.४८)~\arrow त्वर्~इ~\arrow त्वरि~\arrow \textcolor{red}{सनाद्यन्ता धातवः} (पा॰सू॰~३.१.३२)~\arrow धातु\-सञ्ज्ञा~\arrow \textcolor{red}{शेषात्कर्तरि परस्मैपदम्} (पा॰सू॰~१.३.७८)~\arrow \textcolor{red}{वर्तमाने लट्} (पा॰सू॰~३.२.१२३)~\arrow तवरि~तिप्~\arrow त्वरि~ति~\arrow \textcolor{red}{कर्तरि शप्‌} (पा॰सू॰~३.१.६८)~\arrow त्वरि~शप्~ति~\arrow त्वरि~अ~ति~\arrow \textcolor{red}{सार्वधातुकार्ध\-धातुकयोः} (पा॰सू॰~७.३.८४)~\arrow त्वरे~अ~ति~\arrow \textcolor{red}{एचोऽयवायावः} (पा॰सू॰~६.१.७८)~\arrow त्वरय्~अ~ति~\arrow त्वरयति।} इत्यस्मात् \textcolor{red}{त्वरितम्}।\footnote{त्वरि~\arrow धातु\-सञ्ज्ञा (पूर्ववत्)~\arrow \textcolor{red}{नपुंसके भावे क्तः} (पा॰सू॰~३.३.११४)~\arrow त्वरि~क्त~\arrow त्वरि~त~\arrow \textcolor{red}{आर्धधातुकस्येड्वलादेः} (पा॰सू॰~७.२.३५)~\arrow त्वरि~इट्~त~\arrow त्वरि~इ~त~\arrow \textcolor{red}{निष्ठायां सेटि} (पा॰सू॰~६.४.५२)~\arrow त्वर्~इ~त~\arrow त्वरित~\arrow विभक्तिकार्यम्~\arrow त्वरितम्।} भावे क्तः।\footnote{\textcolor{red}{नपुंसके भावे क्तः} (पा॰सू॰~३.३.११४) इत्यनेन।} क्रिया\-विशेषणत्वाद्द्वितीया। यद्वा \textcolor{red}{त्वरयेतं यथा स्यात्तथा} इति विग्रहे \textcolor{red}{तृतीया तत्कृतार्थेन गुण\-वचनेन} (पा॰सू॰~२.१.३०) इत्यत्र \textcolor{red}{तृतीया} इति योग\-विभागेन समासः।\footnote{\textcolor{red}{त्वरा~इत} इति स्थिते \textcolor{red}{शकन्ध्वादिषु पर\-रूपं वाच्यम्} (वा॰~६.१.९१) इत्यनेन शकन्ध्वादित्वात्पर\-रूपे \textcolor{red}{त्वरित} इति शेषः।
}\end{sloppypar}
\vspace{2mm}
\centering ॥ इति बालकाण्डीयप्रयोगाणां विमर्शः ॥\nopagebreak\\
\vspace{4mm}
\pdfbookmark[2]{अयोध्याकाण्डम्}{Chap2Part1Kanda2}
\phantomsection
\addtocontents{toc}{\protect\setcounter{tocdepth}{2}}
\addcontentsline{toc}{subsection}{अयोध्याकाण्डीयप्रयोगाणां विमर्शः}
\addtocontents{toc}{\protect\setcounter{tocdepth}{0}}
\centering ॥ अथायोध्याकाण्डीयप्रयोगाणां विमर्शः ॥\nopagebreak\\
\section[अतिविदूयता]{अतिविदूयता}
\centering\textcolor{blue}{हसन्ती मामुपायाति सा किं नैवाद्य दृश्यते।\nopagebreak\\
इत्यात्मन्येव सञ्चिन्त्य मनसाऽतिविदूयता॥}\nopagebreak\\
\raggedleft{–~अ॰रा॰~२.३.३}\\
\begin{sloppypar}\hyphenrules{nohyphenation}\justifying\noindent\hspace{10mm} अत्राध्यात्म\-रामायणेऽयोध्या\-काण्डे तृतीय\-सर्गे श्रीरामराज्याभिषेक\-समाचारं श्रावयितुमिच्छुश्चक्रवर्ती दशरथः
कैकेयि\-भवनं प्रविश्य तां दृष्ट्वा खिन्नेन मनसा पप्रच्छ। अत्र \textcolor{red}{अतिविदूयता} इति प्रयुक्तम्। अयं ह्यात्मने\-पदीयो धातुः। तथा च \textcolor{red}{दूङ् परितापे} (धा॰पा॰~११३३) इत्यस्मात्।\footnote{\textcolor{red}{अनुदात्तङित आत्मने\-पदम्} (पा॰सू॰~१.३.१२) इत्यनेन धातोर्ङित्त्वादात्मने\-पदमिति भावः।} यथा कालिदासोऽपि प्रयुङ्क्ते~–\end{sloppypar}
\centering\textcolor{red}{तया हीनं विधातर्मां कथं पश्यन्न दूयसे।\nopagebreak\\
सिक्तं स्वयमिव स्नेहाद्वन्ध्यमाश्रमवृक्षकम्॥}\nopagebreak\\
\raggedleft{–~र॰वं॰~१.७०}\\
\begin{sloppypar}\hyphenrules{nohyphenation}\justifying\noindent एवञ्च \textcolor{red}{दूयत इति दूयमानं तेन दूयमानेन}। आत्मनेपदीयत्वात् \textcolor{red}{शानच्}प्रत्यय एव सम्भवः। तथा च सूत्रम्~– \textcolor{red}{लटः शतृ\-शानचावप्रथमा\-समानाधिकरणे} (पा॰सू॰~३.२.१२४)। अनेन लड्लकारस्य शतृ\-शानचौ प्राप्तौ तदाऽग्रिम\-सूत्रेण \textcolor{red}{शानच्‌}\-प्रत्यय आत्मने\-पद एव विधीयते \textcolor{red}{तङानावात्मनेपदम्} (पा॰सू॰~१.४.१००) इत्यनेन। तर्हि \textcolor{red}{विदूयता} इति कथमिति चेत्।
\textcolor{red}{विदूयत इति विदूयः} इति विग्रहे पचाद्यच्।\footnote{\textcolor{red}{नन्दि\-ग्रहि\-पचादिभ्यो ल्युणिन्यचः} (पा॰सू॰~३.१.१३४) इत्यनेन।} \textcolor{red}{विदूयस्य भावो विदूयता} इति विग्रहे \textcolor{red}{तस्य भावस्त्वतलौ} (पा॰सू॰~५.१.११९) इत्यनेन \textcolor{red}{तल्‌}\-प्रत्यये टापि \textcolor{red}{विदूयता}। न च \textcolor{red}{मनसाऽति\-विदूयता} इत्यनेन न सामानाधिकरण्यमेकत्र तृतीयान्ताऽपरत्र प्रथमान्तेति। सामानाधिकरण्ये नैव राजाज्ञा। \textcolor{red}{मनसा} इति तृतीया तु \textcolor{red}{इत्थं\-भूत\-लक्षणे} (पा॰सू॰~२.३.२१) इत्यनेन। \textcolor{red}{मनसोपलक्षिताऽतिविदूयता}।\end{sloppypar}
\section[निवारयित्वा]{निवारयित्वा}
\centering\textcolor{blue}{निवारयित्वा तान्सर्वान्कैकेयी रोषमास्थिता।\nopagebreak\\
ततः प्रभातसमये मध्यकक्षमुपस्थिताः॥}\nopagebreak\\
\raggedleft{–~अ॰रा॰~२.३.३५}\\
\begin{sloppypar}\hyphenrules{nohyphenation}\justifying\noindent\hspace{10mm} अत्र श्रीराम\-वन\-वास\-षड्यन्त्रं कार्यान्वितं चिकीर्षुः कैकेयी सर्वानपि मङ्गलोत्सवान्निवारयति। अत्र \textcolor{red}{वारि}\-धातोः\footnote{\textcolor{red}{वृञ् आवरणे} (धा॰पा॰~१८१३)~\arrow वृ~\arrow \textcolor{red}{सत्याप\-पाश\-रूप\-वीणा\-तूल\-श्लोक\-सेना\-लोम\-त्वच\-वर्म\-वर्ण\-चूर्ण\-चुरादिभ्यो णिच्} (पा॰सू॰~३.१.२५)~\arrow वृ~णिच्~\arrow वृ~इ~\arrow \textcolor{red}{अचो ञ्णिति} (पा॰सू॰~७.२.११५)~\arrow \textcolor{red}{उरण् रपरः} (पा॰सू॰~१.१.५१)~\arrow वार्~इ~\arrow वारि~\arrow \textcolor{red}{सनाद्यन्ता धातवः} (पा॰सू॰~३.१.३२)~\arrow धातु\-सञ्ज्ञा।} \textcolor{red}{समान\-कर्तृकयोः पूर्व\-काले} (पा॰सू॰~३.४.२१) इत्यनेन \textcolor{red}{क्त्वा} प्रत्यये सतीटि\footnote{\textcolor{red}{आर्धधातुकस्येड्वलादेः} (पा॰सू॰~७.२.३५) इत्यनेनेडागमः।} \textcolor{red}{सार्वधातुकार्ध\-धातुकयोः} (पा॰सू॰~७.३.८४) इत्यनेन \textcolor{red}{वारि} इत्यस्य गुणेऽयादेशे \textcolor{red}{नि}\-उपसर्गे \textcolor{red}{निवारयित्वा} इत्याशङ्क्य \textcolor{red}{नि}\-पूर्वकात् \textcolor{red}{वारि}\-धातोर्निष्पन्नोऽयं शब्दः। तथा \textcolor{red}{समासेऽनञ्पूर्वे क्त्वो ल्यप्} (पा॰सू॰~७.१.३७) इत्यनेन \textcolor{red}{ल्यप्} प्रत्यये तथा च \textcolor{red}{निवार्य} इति पाणिनीयम्।\footnote{ल्यपो वलादित्वाभावादिण्न। नि~वारि~ल्यप्~\arrow नि~वारि~य~\arrow \textcolor{red}{णेरनिटि} (पा॰सू॰~६.४.५१)~\arrow नि~वार्~य~\arrow निवार्य।} \textcolor{red}{निवारयित्वा} इति कथम्।
उच्यते।
न चात्र समासो विकल्पो विवक्षाधीनो वाऽतः समासाभाव उक्त\-दोष एव नास्ति। \textcolor{red}{कु\-गति\-प्रादयः} (पा॰सू॰~२.२.१८) इत्यनेन हि नित्यसमासः। तर्हि 
\textcolor{red}{नयतीति नीः} इति विग्रहे क्विप्प्रत्यये\footnote{\textcolor{red}{क्विप् च} (पा॰सू॰~३.२.७६) इत्यनेन।} सर्वापहारि\-लोपेऽर्थात्
\textcolor{red}{सर्वान् दुखं नयन्ती वारयित्वा}।\footnote{\textcolor{red}{सुपां सुलुक्पूर्व\-सवर्णाच्छेयाडाड्यायाजालः} (पा॰सू॰~७.१.३९) इत्यनेन \textcolor{red}{नी}\-प्रातिपदिकात्सोश्छान्दसो लुक्। पृषोदरादित्वाद्ध्रस्वः। \textcolor{red}{नि} इति पृथक्पदमिति भावः।} यद्वा \textcolor{red}{निषिद्धं वारं निवारं करोतीति निवारयति}\footnote{निवार~\arrow \textcolor{red}{तत्करोति तदाचष्टे} (धा॰पा॰ ग॰सू॰)~\arrow निवार~णिच्~\arrow निवार~इ~\arrow \textcolor{red}{णाविष्ठवत्प्राति\-पदिकस्य पुंवद्भाव\-रभाव\-टिलोप\-यणादि\-परार्थम्} (वा॰~६.४.४८)~\arrow निवार्~इ~\arrow निवारि~\arrow \textcolor{red}{सनाद्यन्ता धातवः} (पा॰सू॰~३.१.३२)~\arrow धातु\-सञ्ज्ञा~\arrow \textcolor{red}{शेषात्कर्तरि परस्मैपदम्} (पा॰सू॰~१.३.७८)~\arrow \textcolor{red}{वर्तमाने लट्} (पा॰सू॰~३.२.१२३)~\arrow निवारि~तिप्~\arrow निवारि~ति~\arrow \textcolor{red}{कर्तरि शप्‌} (पा॰सू॰~३.१.६८)~\arrow निवारि~शप्~ति~\arrow निवारि~अ~ति~\arrow \textcolor{red}{सार्वधातुकार्ध\-धातुकयोः} (पा॰सू॰~७.३.८४)~\arrow निवारे~अ~ति~\arrow \textcolor{red}{एचोऽयवायावः} (पा॰सू॰~६.१.७८)~\arrow निवारय्~अ~ति~\arrow निवारयति।} इति विग्रह आचार\-णिजन्तात् \textcolor{red}{निवारि}\-धातोः \textcolor{red}{क्त्वा}\-प्रत्ययः।\footnote{निवारि~\arrow धातु\-सञ्ज्ञा (पूर्ववत्)~\arrow निवारि~\arrow \textcolor{red}{समान\-कर्तृकयोः पूर्व\-काले} (पा॰सू॰~३.४.२१)~\arrow निवारि~क्त्वा~\arrow निवारि~त्वा~\arrow \textcolor{red}{आर्धधातुकस्येड्वलादेः} (पा॰सू॰~७.२.३५)~\arrow निवारि~इट्~त्वा~\arrow निवारि~इ~त्वा~\arrow \textcolor{red}{सार्वधातुकार्ध\-धातुकयोः} (पा॰सू॰~७.३.८४)~\arrow नीवारे~इ~त्वा~\arrow \textcolor{red}{एचोऽयवायावः} (पा॰सू॰~६.१.७८)~\arrow निवारय्~इ~त्वा~\arrow निवारयित्वा।} एवं च कैकय्यभिन्नैक\-कर्तृक\-पूर्व\-कालावच्छिन्न\-निषिद्ध\-भूतक\-वारणानुकूलो
व्यापारः।\footnote{\textcolor{red}{निषिद्ध\-भूतक} इत्यत्र \textcolor{red}{यज्ञेभ्य इतीयत्युच्यमाने य एव/एते सञ्ज्ञीभूतका यज्ञास्तत उत्पत्तिः स्यात्} (भा॰पा॰सू॰~४.३.६८, ५.१.९५) इत्यत्र \textcolor{red}{सञ्ज्ञीभूतकाः} इतिवत्समासान्ते \textcolor{red}{कप्‌}\-प्रत्ययः।} इत्थं \textcolor{red}{निवारि}\-धातोः क्त्वा\-प्रत्यये नापाणिनीयता।\footnote{अन्यत्र च~– \textcolor{red}{निवारयित्वा सर्वान्स राजानमिदमब्रवीत्} (ग॰सं॰~१०.१६.५३) \textcolor{red}{निवारयित्वा कृतवाल्लोँक\-नाथो मरुद्गणान्} (वाम॰पु॰~७२.६९)। अयं प्रयोगः \textcolor{red}{प्रार्थयित्वा} इतिवत्। यथा~– \textcolor{red}{तं प्रार्थयित्वा विधिवत्प्रसाद्य च विशेषतः} (ब्रह्मा॰पु॰~२.५४.२८) \textcolor{red}{इति नीराजनं कृत्वा प्रार्थयित्वा निजेश्वरम्} (ना॰पु॰~६६.१४) \textcolor{red}{ततः कृष्णेन भीमेन प्रार्थयित्वा द्विजान्नृपान्} (ग॰सं॰~१०.५७.१) \textcolor{red}{प्रार्थयित्वा द्विजान् भोज्य भुक्ति\-मुक्तिमवाप्नुयात्} (अ॰पु॰~१८४.८) \textcolor{red}{उपेक्षितमशक्त्या वा प्रार्थयित्वा विरोधितम्} (अ॰शा॰~८.५.२८) इत्यादिषु।} अथवा \textcolor{red}{निवारयतीति निवारयित्वा} इति विग्रहे नि\-पूर्वक\-वारि\-धातोः \textcolor{red}{अन्येभ्योऽपि दृश्यन्ते} (पा॰सू॰~३.२.७५) इत्यनेन \textcolor{red}{वनिप्‌}\-प्रत्ययः। पकारस्येत्सञ्ज्ञायां लोप इटि गुणेऽयादेशे \textcolor{red}{ह्रस्वस्य पिति कृति तुक्} (पा॰सू॰~६.१.७१) इत्यनेन तुगागमे विभक्ति\-कार्ये \textcolor{red}{सर्वनामस्थाने चासम्बुद्धौ} (पा॰सू॰~६.४.८) इत्यनेन दीर्घे सुलोपनलोपयोः \textcolor{red}{निवारयित्वा} इति सम्यक्पाणिनीयम्।\footnote{स्त्रीत्वस्याविवक्षेति शेषः। अन्यथा \textcolor{red}{वनो र च} (पा॰सू॰~४.१.७) इत्यनेन \textcolor{red}{निवारयित्वरी} इति स्यात्। निवारि~\arrow धातु\-सञ्ज्ञा पूर्ववत्~\arrow \textcolor{red}{अन्येभ्योऽपि दृश्यन्ते} (पा॰सू॰~३.२.७५)~\arrow निवारि~वनिँप्~\arrow निवारि~वन्~\arrow \textcolor{red}{आर्धधातुकस्येड्वलादेः} (पा॰सू॰~७.२.३५)~\arrow निवारि~इट्~वन्~\arrow निवारि~इ~वन्~\arrow \textcolor{red}{सार्वधातुकार्ध\-धातुकयोः} (पा॰सू॰~७.३.८४)~\arrow निवारे~इ~वन्~\arrow \textcolor{red}{एचोऽयवायावः} (पा॰सू॰~६.१.७८)~\arrow निवारय्~इ~वन्~\arrow \textcolor{red}{ह्रस्वस्य पिति कृति तुक्} (पा॰सू॰~६.१.७१)~\arrow निवारय्~इ~तुँक्~वन्~\arrow निवारय्~इ~त्~वन्~\arrow निवारयित्वन्~\arrow विभक्ति\-कार्यम्~\arrow निवारयित्वन्~सुँ~\arrow निवारयित्वन्~स्~\arrow \textcolor{red}{सर्वनामस्थाने चासम्बुद्धौ} (पा॰सू॰~६.४.८)~\arrow निवारयित्वान्~स्~\arrow \textcolor{red}{हल्ङ्याब्भ्यो दीर्घात्सुतिस्यपृक्तं हल्} (पा॰सू॰~६.१.६८)~\arrow निवारयित्वान्~\arrow \textcolor{red}{नलोपः प्रातिपदिकान्तस्य} (पा॰सू॰~८.२.७)~\arrow निवारयित्वा।} \textcolor{red}{निवारण\-शीला कैकेयी सर्वान्प्रति रोषं प्रकटयन्ती मध्य\-कक्षमाश्रिता} इति ग्रन्थान्वय\-प्रकारः।\end{sloppypar}
\section[भुञ्जन्]{भुञ्जन्}
\centering\textcolor{blue}{बुद्ध्यादिभ्यो बहिः सर्वमनुवर्तस्व मा खिदः।\nopagebreak\\
भुञ्जन्प्रारब्धमखिलं सुखं वा दुःखमेव वा॥}\nopagebreak\\
\raggedleft{–~अ॰रा॰~२.४.४१}\\
\begin{sloppypar}\hyphenrules{nohyphenation}\justifying\noindent\hspace{10mm} अत्र श्रीरामो लक्ष्मणमुपदिशन् \textcolor{red}{भुञ्जन्} इति प्रयुङ्क्ते। \textcolor{red}{भुजँ पालनाभ्यवहारयोः} (धा॰पा॰~१४५४) इत्यस्माद्धातोः \textcolor{red}{भुङ्क्त इति भुञ्जानः} इत्येव रूपं सामान्यतः पाणिनीय\-मते धातोरस्याभ्यवहारार्थ
आत्मनेपदीयत्वात्।\footnote{\textcolor{red}{भुजोऽनवने} (पा॰सू॰~१.३.६६) इत्यनेन।} \textcolor{red}{भुङ्क्ते} इति विग्रहे लड्लकारे कर्तरि \textcolor{red}{लटः शतृ\-शानचावप्रथमा\-समानाधिकरणे} (पा॰सू॰~३.२.१२४) इत्यनेन शतृ\-शानचौ प्राप्तौ। \textcolor{red}{तङानावात्मनेपदम्} (पा॰सू॰~१.४.१००) इत्यनेन \textcolor{red}{शानच्‌}\-प्रत्यये कृते शकारस्यानुबन्ध\-कार्ये चकारानुबन्ध\-लोपे च \textcolor{red}{भुञ्जानः} इत्येव।\footnote{\textcolor{red}{भुजँ पालनाभ्यवहारयोः} (धा॰पा॰~१४५४)~\arrow भुज्~\arrow \textcolor{red}{भुजोऽनवने} (पा॰सू॰~१.३.६६)~\arrow \textcolor{red}{वर्तमाने लट्} (पा॰सू॰~३.२.१२३)~\arrow भुज्~लट्~\arrow \textcolor{red}{लटः शतृ\-शानचावप्रथमा\-समानाधिकरणे} (पा॰सू॰~३.२.१२४)~\arrow भुज्~शानच्~\arrow \textcolor{red}{सार्वधातुकमपित्} (पा॰सू॰~१.२.४)~\arrow शानचो ङित्त्वम्~\arrow भुज्~आन~\arrow \textcolor{red}{रुधादिभ्यः श्नम्}~\arrow \textcolor{red}{मिदचोऽन्त्यात्परः} (पा॰सू॰~१.१.४७)~\arrow भु~श्नम्~ज्~आन~\arrow भु~न~ज्~आन~\arrow \textcolor{red}{श्नसोरल्लोपः} (पा॰सू॰~६.४.१११)~\arrow भु~न्~ज्~आन~\arrow \textcolor{red}{नश्चापदान्तस्य झलि} (पा॰सू॰~८.३.२४)~\arrow भुं~ज्~आन~\arrow \textcolor{red}{अनुस्वारस्य ययि परसवर्णः} (पा॰सू॰~८.४.५८)~\arrow भुञ्~ज्~आन~\arrow भुञ्जान~\arrow \textcolor{red}{कृत्तद्धित\-समासाश्च} (पा॰सू॰~१.२.४६)~\arrow प्रातिपदिक\-सञ्ज्ञा~\arrow विभक्ति\-कार्यम्~\arrow भुञ्जानः।} \textcolor{red}{भुञ्जन्} इति कथमिति चेत् \textcolor{red}{भुङ्क्ते} इति विग्रहे। \textcolor{red}{भुञ्जानं वाऽचष्टे भुञ्जति}\footnote{अत्र \textcolor{red}{इव} इत्यर्थे \textcolor{red}{वा}। \textcolor{red}{उपमायां विकल्पे वा} (अ॰को॰~३.३.२४९) \textcolor{red}{व वा यथा तथैवेवं साम्ये} (अ॰को॰~३.४.९) इति कोषात्।
भुञ्जान~\arrow \textcolor{red}{तत्करोति तदाचष्टे } (धा॰पा॰ ग॰सू॰~१८७)~\arrow भुञ्जान~णिच्~\arrow भुञ्जान~इ~\arrow \textcolor{red}{णाविष्ठवत्प्राति\-पदिकस्य पुंवद्भाव\-रभाव\-टिलोप\-यणादि\-परार्थम् } (वा॰~६.४.४८)~\arrow भुञ्जान्~इ~\arrow भुञ्जानि~\arrow \textcolor{red}{सनाद्यन्ता धातवः} (पा॰सू॰~३.१.३२)~\arrow धातु\-सञ्ज्ञा~\arrow \textcolor{red}{क्विप् च } (पा॰सू॰~३.२.७६)~\arrow भुञ्जानि~क्विँप्~\arrow भुञ्जानि~व्~\arrow \textcolor{red}{वेरपृक्तस्य} (पा॰सू॰~६.१.६७)~\arrow भुञ्जानि~\arrow \textcolor{red}{णेरनिटि } (पा॰सू॰~६.४.५१)~\arrow भुञ्जान्~\arrow \textcolor{red}{सर्वप्राति\-पदिकेभ्य आचारे क्विब्वा वक्तव्यः } (वा॰~३.१.११)~\arrow भुञ्जान्~क्विँप्~\arrow भुञ्जान्~व्~\arrow \textcolor{red}{वेरपृक्तस्य} (पा॰सू॰~६.१.६७)~\arrow भुञ्जान्~\arrow \textcolor{red}{सनाद्यन्ता धातवः} (पा॰सू॰~३.१.३२)~\arrow धातुसञ्ज्ञा~\arrow \textcolor{red}{अन्येष्वपि दृश्यते } (पा॰सू॰~३.२.१०१)~\arrow भुञ्जान्~ड~\arrow भुञ्जान्~अ~\arrow \textcolor{red}{डित्यभस्याप्यनु\-बन्धकरण\-सामर्थ्यात्} (वा॰~६.४.१४३)~\arrow भुञ्ज्~अ~\arrow भुञ्ज~\arrow \textcolor{red}{सर्वप्राति\-पदिकेभ्य आचारे क्विब्वा वक्तव्यः } (वा॰~३.१.११) भुञ्ज~क्विँप्~\arrow भुञ्ज~व्~\arrow \textcolor{red}{वेरपृक्तस्य} (पा॰सू॰~६.१.६७)~\arrow भुञ्ज~\arrow \textcolor{red}{सनाद्यन्ता धातवः} (पा॰सू॰~३.१.३२)~\arrow धातु\-सञ्ज्ञा~\arrow \textcolor{red}{शेषात्कर्तरि परस्मैपदम्} (पा॰सू॰~१.३.७८)~\arrow \textcolor{red}{वर्तमाने लट्} (पा॰सू॰~३.२.१२३)~\arrow भुञ्ज~लट्~\arrow भुञ्ज~तिप्~\arrow भुञ्ज~ति~\arrow \textcolor{red}{कर्तरि शप्‌} (पा॰सू॰~३.१.६८)~\arrow भुञ्ज~शप्~ति~\arrow भुञ्ज~अ~ति~\arrow \textcolor{red}{अतो गुणे} (पा॰सू॰~६.१.९७)~\arrow भुञ्ज~ति~\arrow भुञ्जति।} इति विग्रहे पुनः शतरि \textcolor{red}{भुञ्जन्}।\footnote{भुञ्ज~\arrow पूर्ववद्धातु\-सञ्ज्ञा~\arrow \textcolor{red}{शेषात्कर्तरि परस्मैपदम्} (पा॰सू॰~१.३.७८)~\arrow \textcolor{red}{वर्तमाने लट्} (पा॰सू॰~३.२.१२३)~\arrow भुञ्ज~लट्~\arrow \textcolor{red}{लटः शतृ\-शानचावप्रथमा\-समानाधिकरणे} (पा॰सू॰~३.२.१२४)~\arrow भुञ्ज~शतृँ~\arrow भुञ्ज~अत्~\arrow \textcolor{red}{कर्तरि शप्‌} (पा॰सू॰~३.१.६८)~\arrow भुञ्ज~शप्~अत्~\arrow भुञ्ज~अ~अत्~\arrow \textcolor{red}{अतो गुणे} (पा॰सू॰~६.१.९७)~\arrow भुञ्ज~अत्~\arrow \textcolor{red}{अतो गुणे} (पा॰सू॰~६.१.९७)~\arrow भुञ्जत्~\arrow \textcolor{red}{कृत्तद्धित\-समासाश्च} (पा॰सू॰~१.२.४६)~\arrow प्रातिपदिक\-सञ्ज्ञा~\arrow विभक्ति\-कार्यम्~\arrow भुञ्जत्~सुँ~\arrow \textcolor{red}{उगिदचां सर्वनामस्थानेऽधातोः} (पा॰सू॰~७.१.७०)~\arrow \textcolor{red}{मिदचोऽन्त्यात्परः} (पा॰सू॰~१.१.४७)~\arrow भुञ्ज~नुँम्~त्~सुँ~\arrow भुञ्ज~न्~त्~सुँ~\arrow \textcolor{red}{हल्ङ्याब्भ्यो दीर्घात्सुतिस्यपृक्तं हल्} (पा॰सू॰~६.१.६८)~\arrow भुञ्जत्~न्~\arrow \textcolor{red}{संयोगान्तस्य लोपः} (पा॰सू॰~८.२.२३)~\arrow भुञ्जन्।} यद्वा \textcolor{red}{भुजोऽनवने} (पा॰सू॰~१.३.६६) इति सूत्रेणात्मनेपदं तच्चावन\-भिन्नेऽर्थे \textcolor{red}{अनवने} इति पर्युदासात्। ध्येयं यत्पाणिनि\-सूत्रेषु निषेधस्य द्वे क्रिये प्रसिद्धे पर्युदासः प्रसज्यश्च। पर्युदासस्तदा भवति प्रायो यदा निषेधः समास\-गर्भे यथा \textcolor{red}{स्थानिवदादेशोऽनल्विधौ} (पा॰सू॰~१.१.५६)। अत्र \textcolor{red}{अनल्विधौ} अर्थादल्विधि\-भिन्ने तत्सदृशे तथैवात्रापि। प्रसज्यस्तु यदा स्वतन्त्रो नञ्वाचको नकारो यथा \textcolor{red}{न विभक्तौ तुस्माः} (पा॰सू॰~१.३.४)। अत्र नकारः क्रियान्वयी। लक्षणमित्थं करणीयम्~– \textcolor{red}{पाणिन्युच्चरित\-निषेधार्थ\-बोधकत्वे सति कारकान्वयित्वे सति समास\-गर्भ\-नञ्धर्म\-वत्त्वं पर्युदासत्वम्}। प्रसज्यत्वं तु \textcolor{red}{पाणिन्युच्चरित\-निषेधार्थ\-बोधकत्वे सति क्रियान्वयित्वे सति नञ्धर्मतावच्छेदकतावत्त्वम्}। तथा चोच्यते~–\end{sloppypar}
\centering\textcolor{red}{द्वौ नञर्थौ समाख्यातौ पर्युदासप्रसज्यकौ।\nopagebreak\\
पर्युदासः सदृग्ग्राही प्रसज्यः प्रतिषेधकृत्॥}\nopagebreak\\
\raggedleft{–~इति गुरवः}\\
\begin{sloppypar}\hyphenrules{nohyphenation}\justifying\noindent अतोऽत्रापि \textcolor{red}{अनवने} इति पर्युदासः। अवन\-भिन्नेऽवन\-सदृशे। सादृश्यं चाऽत्र धातु\-वाच्यत्वेन। सदृशत्वं नाम \textcolor{red}{तद्भिन्नत्वे सति तद्गत\-भूयोधर्मवत्त्वम्}। यथा \textcolor{red}{चन्द्रमिव मुखं पश्यति} (का॰सू॰वृ॰~४.२.१३) इत्यत्र चन्द्र\-भिन्नत्वे सति चन्द्र\-गताह्लादकत्वमिति सदृशत्वम्। तथैव \textcolor{red}{अवन\-भिन्नत्वे सत्यवन\-गत\-धातु\-वाचकत्वम्}। एवमत्रावनार्थं प्रसृज्य भोजनार्थ आत्मनेपदम्। यतो हि \textcolor{red}{भुज्} धातोर्द्वयोरर्थयोः शक्तिर्भोजने पालने चैव। \textcolor{red}{भुजँ पालनाभ्यवहारयोः} (धा॰पा॰~१४५४) इति पठितत्वात्। अतोऽवन\-भिन्नेऽर्थ आत्मनेपदं भवत्यवने तु परस्मैपदम्। तत्रैव शतृ\-प्रत्ययः। अर्थात् \textcolor{red}{भुनक्तीति भुञ्जन्} इति विग्रहे \textcolor{red}{भुज्} धातोः वर्तमाने लड्लकारे \textcolor{red}{लटः शतृ\-शानचावप्रथमा\-समानाधिकरणे} (पा॰सू॰~३.२.१२४) इत्यनेन \textcolor{red}{शतृँ}\-प्रत्यये \textcolor{red}{लशक्वतद्धिते} (पा॰सू॰~१.३.८) इत्यनेन शकारस्येत्सञ्ज्ञायां \textcolor{red}{तस्य लोपः} (पा॰सू॰~१.३.९) इत्यनेन लोप ऋकारस्य च \textcolor{red}{उपदेशेऽजनुनासिक इत्} (पा॰सू॰~१.३.२) इत्यनेनेत्सञ्ज्ञायां तेनैव सूत्रेण लोपे प्रातिपदिक\-सञ्ज्ञायां विभक्ति\-कार्ये सौ \textcolor{red}{भुज्~अत्~सु} इति स्थिते \textcolor{red}{रुधादिभ्यः श्नम्} (पा॰सू॰~३.१.७८) इत्यनेन श्नमि प्रत्यये \textcolor{red}{मिदचोऽन्त्यात्परः} (पा॰सू॰~१.१.४७) इत्यनेन जकारात्पूर्वं कृते शकार\-मकारयोरनुबन्ध\-कार्ये \textcolor{red}{श्नसोरल्लोपः} (पा॰सू॰~६.४.१११) इत्यनेनालोपे \textcolor{red}{उगिदचां सर्वनामस्थानेऽधातोः} (पा॰सू॰~७.१.७०) इत्यनेन नुमि सोर्लोपे तकारलोपे च \textcolor{red}{भुञ्जन्}।\footnote{\textcolor{red}{भुजँ पालनाभ्यवहारयोः} (धा॰पा॰~१४५४)~\arrow भुज्~\arrow \textcolor{red}{शेषात्कर्तरि परस्मैपदम्} (पा॰सू॰~१.३.७८)~\arrow \textcolor{red}{वर्तमाने लट्} (पा॰सू॰~३.२.१२३)~\arrow भुज्~लट्~\arrow \textcolor{red}{लटः शतृ\-शानचावप्रथमा\-समानाधिकरणे} (पा॰सू॰~३.२.१२४)~\arrow भुज्~शतृँ~\arrow भुज्~अत्~\arrow \textcolor{red}{रुधादिभ्यः श्नम्}~\arrow \textcolor{red}{मिदचोऽन्त्यात्परः} (पा॰सू॰~१.१.४७)~\arrow भु~श्नम्~ज्~अत्~\arrow भु~न~ज्~अत्~\arrow \textcolor{red}{श्नसोरल्लोपः} (पा॰सू॰~६.४.१११)~\arrow भु~न्~ज्~अत्~\arrow \textcolor{red}{नश्चापदान्तस्य झलि} (पा॰सू॰~८.३.२४)~\arrow भुं~ज्~अत्~\arrow \textcolor{red}{अनुस्वारस्य ययि परसवर्णः} (पा॰सू॰~८.४.५८)~\arrow भुञ्~ज्~अत्~\arrow भुञ्जत्~\arrow \textcolor{red}{कृत्तद्धित\-समासाश्च} (पा॰सू॰~१.२.४६)~\arrow प्रातिपदिक\-सञ्ज्ञा~\arrow विभक्ति\-कार्यम्~\arrow भुञ्जत्~सुँ~\arrow \textcolor{red}{उगिदचां सर्वनामस्थानेऽधातोः} (पा॰सू॰~७.१.७०)~\arrow \textcolor{red}{मिदचोऽन्त्यात्परः} (पा॰सू॰~१.१.४७)~\arrow भुञ्ज~नुँम्~त्~सुँ~\arrow भुञ्ज~न्~त्~सुँ~\arrow \textcolor{red}{हल्ङ्याब्भ्यो दीर्घात्सुतिस्यपृक्तं हल्} (पा॰सू॰~६.१.६८)~\arrow भुञ्जत्~न्~\arrow \textcolor{red}{संयोगान्तस्य लोपः} (पा॰सू॰~८.२.२३)~\arrow भुञ्जन्।} भगवतोऽयमभिप्रायो यत्त्वं प्रारब्धं मा \textcolor{red}{भुङ्क्ष्व} अपि तु \textcolor{red}{भुङ्ग्धि}। कारणमिदं यत्त्वं योगेश्वरोऽसि। योगेश्वरो भोगं न बुभुक्षतेऽपि तु योगमेव युयुक्षते। अतस्त्वं \textcolor{red}{भुङ्ग्धि} मर्यादानुसारं रक्ष पालय। यतो हि लक्ष्मण ईश्वररूपः। ईश्वरश्चक्रवर्ति\-दशरथस्य गृह आत्मानं चतुर्धा कृत्वा प्रकटयाम्बभूवेति रामायण\-पुराणादौ प्रसिद्धम्। यथा वाल्मीकीय\-रामायणे~–\end{sloppypar}
\centering\textcolor{red}{आदिदेवो महाबाहुर्हरिर्नारायणः प्रभुः।\nopagebreak\\
साक्षाद्रामो रघुश्रेष्ठः शेषो लक्ष्मण उच्यते॥}\nopagebreak\\
\raggedleft{–~वा॰रा॰~६.१२८.१२०}\\
\begin{sloppypar}\hyphenrules{nohyphenation}\justifying\noindent एवं श्रीमद्भागवतेऽपि~–\end{sloppypar}
\centering\textcolor{red}{तस्यापि भगवानेष साक्षाद्ब्रह्ममयो हरिः।\nopagebreak\\
अंशांशेन चतुर्धाऽगात्पुत्रत्वं प्रार्थितः सुरैः।\nopagebreak\\
रामलक्ष्मणभरतशत्रुघ्ना इति सञ्ज्ञया॥}\nopagebreak\\
\raggedleft{–~भा॰पु॰~९.१०.२}\\
\begin{sloppypar}\hyphenrules{nohyphenation}\justifying\noindent एवमत्राप्यध्यात्म\-रामायणे \textcolor{red}{शेषस्तु लक्ष्मणो राजन्} (अ॰रा॰~१.४.१७) इति बहुत्र सङ्कीर्तनात्। ईश्वरस्य कर्म\-विपाको न भवति। सूत्रेऽपि \textcolor{red}{क्लेश\-कर्म\-विपाकाशयैरपरामृष्टः पुरुषविशेष ईश्वरः} (यो॰सू॰~१.२४) इति। अत एव~–\end{sloppypar}
\centering\textcolor{red}{न मां कर्माणि लिम्पन्ति न मे कर्मफले स्पृहा।\nopagebreak\\
इति मां योऽभिजानाति कर्मभिर्न स बध्यते॥}\nopagebreak\\
\raggedleft{–~भ॰गी॰~४.१४}\\
\begin{sloppypar}\hyphenrules{nohyphenation}\justifying\noindent इति भगवद्गीतोक्तमपि सङ्गच्छेत। अतो लक्ष्मण ईश्वरः। न वा तस्य प्रारब्धो न वा कर्मबन्धनं न वा बुभुक्षा\-मुमुक्षे। अत एवाऽचतुर्दशाब्दमरण्ये त्यक्त\-निद्रा\-नारी\-भोजनत्वमपि सङ्घटते। अतो भगवाञ्छ्रीरामः कथयति यत् \textcolor{red}{त्वं प्रारब्ध\-भोगाय न विवशोऽपि तु लोक\-सङ्ग्रहार्थमखिलं प्रारब्धं भुञ्जन्नवन् रक्षन्नित्यर्थस्तटस्थो भूत्वा वर्तस्व}। इत्थं लक्ष्मणाभिन्नैक\-कर्तृक\-प्रारब्ध\-कर्मक\-वर्तमान\-कालावच्छिन्न\-प्रारब्ध\-पालनानुकूल\-व्यापाराश्रयो लक्ष्मण इति शाब्द\-बोधः।\end{sloppypar}
\section[उद्वीक्षयन्]{उद्वीक्षयन्}
\label{sec:udviksayan}
\centering\textcolor{blue}{श्रीरामः सह सीतया नृपपथे गच्छन् शनैः सानुजः\nopagebreak\\
पौरान् जानपदान् कुतूहलदृशः सानन्दमुद्वीक्षयन्।\nopagebreak\\
श्यामः कामसहस्रसुन्दरवपुः कान्त्या दिशो भासयन्\nopagebreak\\
पादन्यासपवित्रिताखिलजगत्प्रापालयं तत्पितुः॥}\nopagebreak\\
\raggedleft{–~अ॰रा॰~२.४.८७}\\
\begin{sloppypar}\hyphenrules{nohyphenation}\justifying\noindent\hspace{10mm} अत्र \textcolor{red}{ईक्ष्‌}\-धातुर्दर्शने (\textcolor{red}{ईक्षँ दर्शने} धा॰पा॰~६१०) आत्मनेपदी। स च \textcolor{red}{उद्वि}\-इत्युपसर्ग\-द्वय\-पूर्वकः। तथा चाऽत्मनेपदित्वादत्र शानज्रूपम् \textcolor{red}{उद्वीक्षमाणः} इति पाणिनि\-तन्त्रानुरूपम्। \textcolor{red}{उद्वीक्षयन्} अपि तथैव। अत्र स्वार्थे णिच्। \textcolor{red}{उद्वीक्षते} इत्यर्थे \textcolor{red}{उद्वीक्षयति}।\footnote{स्वार्थे णिचि रामस्य सर्वद्रष्टृत्वात्क्रियाफलस्य परगामित्वात् \textcolor{red}{णिचश्च} (पा॰सू॰~१.३.७४) इत्यस्याप्रवृत्तौ \textcolor{red}{शेषात्कर्तरि परस्मैपदम्} (पा॰सू॰~१.३.७८) इत्यनेन परस्मैपदम्।} \textcolor{red}{उद्वीक्षयतीत्युद्वीक्षयन्}। यद्वा \textcolor{red}{उद्वीक्ष्यत इत्युद्वीक्षा} उद्विपूर्वक\-धातोर्भावे \textcolor{red}{अः} प्रत्ययः स्त्रियाम्।\footnote{\textcolor{red}{गुरोश्च हलः} (पा॰सू॰~३.३.१०३) इत्यनेन। उद्~वि~ईक्ष्~\arrow \textcolor{red}{गुरोश्च हलः} (पा॰सू॰~३.३.१०३)~\arrow उद्~वि~ईक्ष्~अ~\arrow \textcolor{red}{अकः सवर्णे दीर्घः} (पा॰सू॰~६.१.१०१)~\arrow उद्~वीक्ष्~अ~\arrow उद्वीक्ष~\arrow \textcolor{red}{अजाद्यतष्टाप्‌} (पा॰सू॰~४.१.४)~\arrow उद्वीक्ष~टाप्~\arrow उद्वीक्ष~आ~\arrow \textcolor{red}{अकः सवर्णे दीर्घः} (पा॰सू॰~६.१.१०१)~\arrow उद्वीक्षा।} \textcolor{red}{उद्वीक्षां करोत्युद्वीक्षयति} इति विग्रहे \textcolor{red}{तत्करोति तदाचष्टे} (धा॰पा॰ ग॰सू॰~१८७) इत्यनेन णिचि टिलोपादौ \textcolor{red}{उद्वीक्षयति}।\footnote{उद्वीक्षा~\arrow \textcolor{red}{तत्करोति तदाचष्टे } (धा॰पा॰ ग॰सू॰~१८७)~\arrow उद्वीक्षा~णिच्~\arrow उद्वीक्षा~इ~\arrow \textcolor{red}{णाविष्ठवत्प्राति\-पदिकस्य पुंवद्भाव\-रभाव\-टिलोप\-यणादि\-परार्थम् } (वा॰~६.४.४८)~\arrow उद्वीक्ष्~इ~\arrow उद्वीक्षि~\arrow \textcolor{red}{सनाद्यन्ता धातवः} (पा॰सू॰~३.१.३२)~\arrow धातु\-सञ्ज्ञा~\arrow \textcolor{red}{शेषात्कर्तरि परस्मैपदम्} (पा॰सू॰~१.३.७८)~\arrow \textcolor{red}{वर्तमाने लट्} (पा॰सू॰~३.२.१२३)~\arrow उद्वीक्षि~लट्~\arrow उद्वीक्षि~तिप्~\arrow उद्वीक्षि~ति~\arrow \textcolor{red}{कर्तरि शप्‌} (पा॰सू॰~३.१.६८)~\arrow उद्वीक्षि~शप्~ति~\arrow उद्वीक्षि~अ~ति~\arrow \textcolor{red}{सार्वधातुकार्ध\-धातुकयोः} (पा॰सू॰~७.३.८४)~\arrow उद्वीक्षे~अ~ति~\arrow \textcolor{red}{एचोऽयवायावः} (पा॰सू॰~६.१.७८)~\arrow उद्वीक्षय्~ति~\arrow उद्वीक्षयति।} ततः शतरि विभक्ति\-कार्ये \textcolor{red}{उद्वीक्षयन्}। यद्वा \textcolor{red}{पौराः श्रीराममुद्वीक्षन्ते रामस्तान् प्रेरयति} इत्यर्थे \textcolor{red}{रामः पौरानुद्वीक्षयति} इति विग्रहे \textcolor{red}{तत्प्रयोजको हेतुश्च} (पा॰सू॰~१.४.५५) इत्यनेन हेतु\-सञ्ज्ञायां \textcolor{red}{हेतुमति च} (पा॰सू॰~३.१.२६) इत्यनेन णिचि \textcolor{red}{सनाद्यन्ता धातवः} (पा॰सू॰~३.१.३२) इत्यनेन धातु\-सञ्ज्ञायां लटि तिपि शपि गुणेऽयादेशे \textcolor{red}{उद्वीक्षयति}। \textcolor{red}{उद्वीक्षयतीत्युद्वीक्षयन्} इति विग्रहे शतृ\-प्रत्यये पूर्वोक्त\-साधन\-प्रक्रियातः \textcolor{red}{उद्वीक्षयन्} इति पाणिनीयमेव। प्रथम\-पक्षे \textcolor{red}{पौर\-जन\-कर्मक\-दर्शनानुकूल\-व्यापाराश्रयो रामः}। द्वितीय\-कल्पे \textcolor{red}{वर्तमान\-कालावच्छिन्न\-पौर\-जन\-कर्तृक\-राम\-कर्मक\-वीक्षणानुकूल\-व्यापारानुकूल\-व्यापाराश्रयो रामः} इति शाब्द\-बोधः।\end{sloppypar}
\section[गच्छतीम्]{गच्छतीम्}
\label{sec:gacchatim}
\centering\textcolor{blue}{यत्र रामः सभार्यश्च सानुजो गन्तुमिच्छति।\nopagebreak\\
पश्यन्तु जानकीं सर्वे पादचारेण गच्छतीम्॥}\nopagebreak\\
\raggedleft{–~अ॰रा॰~२.५.५}\\
\begin{sloppypar}\hyphenrules{nohyphenation}\justifying\noindent\hspace{10mm} अत्र भगवतीं जनक\-नन्दिनीं सीतां वल्कल\-वस्त्र\-धारिणं पाद\-चारिणं सौमित्रि\-सुख\-कारिणं भक्त\-भय\-हारिणमाप्त\-कामं श्रीराममनुगच्छन्तीं विलोक्य सशोकाः पौर\-लोकाः समालोचमाना निगदन्ति \textcolor{red}{पश्यन्तु जानकीं सर्वे पाद\-चारेण गच्छतीम्}। अत्र \textcolor{red}{गच्छन्तीम्} इति हि पाणिनीयम्। यतो हि \textcolor{red}{गच्छतीति गच्छन्ती ताम्} इति विग्रहे \textcolor{red}{गम्‌}\-धातोः (\textcolor{red}{गमॢँ गतौ} धा॰पा॰~९८२) वर्तमान\-काले लड्लकारे शतृ\-प्रत्ययेऽनुबन्ध\-कार्ये \textcolor{red}{इषुगमियमां छः} (पा॰सू॰~७.३.७७) इत्यनेन छान्तादेशे \textcolor{red}{छे च} (पा॰सू॰~६.१.७३) इत्यनेन तुगागमे श्चुत्वे \textcolor{red}{उगितश्च} (पा॰सू॰~४.१.६) इत्यनेन ङीपि नुम्यमि \textcolor{red}{गच्छन्तीम्}।\footnote{\textcolor{red}{गमॢँ गतौ} (धा॰पा॰~९८२)~\arrow गम्~\arrow \textcolor{red}{शेषात्कर्तरि परस्मैपदम्} (पा॰सू॰~१.३.७८)~\arrow \textcolor{red}{वर्तमाने लट्} (पा॰सू॰~३.२.१२३)~\arrow गम्~लट्~\arrow \textcolor{red}{लटः शतृशानचावप्रथमा\-समानाधिकरणे} (पा॰सू॰~३.२.१२४)~\arrow गम्~शतृँ~\arrow गम्~अत्~\arrow \textcolor{red}{इषुगमियमां छः} (पा॰सू॰~७.३.७७)~\arrow गछ्~अत्~\arrow \textcolor{red}{छे च} (पा॰सू॰~६.१.७३)~\arrow \textcolor{red}{आद्यन्तौ टकितौ} (पा॰सू॰~१.१.४६)~\arrow ग~तुँक्~छ्~अत्~\arrow ग~त्~छ्~अत्~\arrow \textcolor{red}{स्तोः श्चुना श्चुः} (पा॰सू॰~८.४.४०)~\arrow ग~च्~छ्~अत्~\arrow गच्छ्~अत्~\arrow \textcolor{red}{कर्तरि शप्‌} (पा॰सू॰~३.१.६८)~\arrow गच्छ्~शप्~अत्~\arrow गच्छ्~अ~अत्~\arrow \textcolor{red}{अतो गुणे} (पा॰सू॰~६.१.९७)~\arrow गच्छ्~अत्~\arrow \textcolor{red}{उगितश्च} (पा॰सू॰~४.१.६)~\arrow गच्छ्~अत्~ङीप्~\arrow गच्छ्~अत्~ई~\arrow \textcolor{red}{शप्श्यनोर्नित्यम्} (पा॰सू॰~७.१.८१)~\arrow \textcolor{red}{आद्यन्तौ टकितौ} (पा॰सू॰~१.१.४६)~\arrow गच्छ्~अ~नुँम्~त्~ई~\arrow गच्छ्~अ~न्~त्~ई~\arrow गच्छन्ती~\arrow विभक्ति\-कार्यम्~\arrow गच्छन्ती~अम्~\arrow \textcolor{red}{अमि पूर्वः} (पा॰सू॰~६.१.१०७)~\arrow गच्छन्तीम्।} \textcolor{red}{गच्छतीम्} इति कथम्। अत्र हि पृषोदरादित्वान्नुमभावः। यद्वा \textcolor{red}{आगम\-शास्त्रमनित्यम्} (प॰शे॰~९३.३)। अतो नुमभावः कल्प्यताम्। यद्वाऽत्रौणादिकः \textcolor{red}{तृँच्‌}\-प्रत्ययः।\footnote{नायं \textcolor{red}{बहुलमन्यत्रापि} (प॰उ॰~२.९५) इति तृच्। स नोगित्। \textcolor{red}{कार्याद्विद्यादनूबन्धम्} (भा॰पा॰सू॰~३.३.१) \textcolor{red}{केचिदविहिता अप्यूह्याः} (वै॰सि॰कौ॰~३१६९) इत्यनुसारमूह्योऽ\-यमविहित उगित्प्रत्ययः। \textcolor{red}{तृँच्} प्रत्यये चात्र शबागमोऽप्यूह्यः। \textcolor{red}{नयतेः षुगागमः} (प॰उ॰ श्वे॰वृ॰~२.९६) इतिवत्।} ततश्च \textcolor{red}{उगितश्च} (पा॰सू॰~४.१.६) इत्यनेन ङीप्। इत्थम् \textcolor{red}{गच्छतीम्} इति पाणिनीयम्। उणादयः पाणिनि\-सम्मता न वेति चैत्तत्रैव पाणिनि\-सूत्रम् \textcolor{red}{उणादयो बहुलम्} (पा॰सू॰~३.३.१)। तत्रैव कारिका~–\end{sloppypar}
\centering\textcolor{red}{सञ्ज्ञासु धातुरूपाणि प्रत्ययाश्च ततः परे।\nopagebreak\\
कार्याद्विद्यादनूबन्धमेतच्छाशास्त्रमुणादिषु॥}\nopagebreak\\
\raggedleft{–~भा॰पा॰सू॰~३.३.१} \\
\section[विजानती]{विजानती}
\centering\textcolor{blue}{रामस्तु वस्त्राण्युत्सृज्य वन्यचीराणि पर्यधात्।\nopagebreak\\
लक्ष्मणोऽपि तथा चक्रे सीता तन्न विजानती॥}\nopagebreak\\
\raggedleft{–~अ॰रा॰~२.५.३६}\\
\begin{sloppypar}\hyphenrules{nohyphenation}\justifying\noindent\hspace{10mm} अत्रापि \textcolor{red}{विजानाति} इति विग्रहे \textcolor{red}{शतृ}\-प्रत्यये ङीपि नुमि \textcolor{red}{विजानन्ती} इति।\footnote{\setcounter{dummy}{\value{footnote}}\addtocounter{dummy}{-1}\refstepcounter{dummy}\label{fn:jananti}पूर्वपक्षोऽयम्। यथा \textcolor{red}{बलमात्मनि जानन्ती न मां शङ्कितुमर्हसि} (वा॰रा॰~२.१०.३५) \textcolor{red}{जानन्ती बत दिष्ट्या मां वैदेहि परिपृच्छसि} (वा॰रा॰~५.३५.६) \textcolor{red}{साऽहमेतद्विजानन्ती तोषयिष्ये द्विजोत्तमम्} (म॰भा॰~३.२८८.८) \textcolor{red}{एतत्सर्वं विजानन्ती सा क्षमामन्वपद्यत} (म॰भा॰~४.१९.६९) \textcolor{red}{साऽहं धर्मं विजानन्ती धर्मनित्ये त्वयि स्थिते} (म॰भा॰~१२.३४७.१२) \textcolor{red}{अजानन्त्या परं भावं तथाप्यस्त्वाभयाय मे} (भा॰पु॰~३.२३.५४) \textcolor{red}{कृतद्युतिरजानन्ती सपत्‍नीनामघं महत्} (भा॰पु॰~६.१४.४४) \textcolor{red}{द्वे ज्योतिषी अजानन्त्या निर्भिन्ने कण्टकेन वै} (भा॰पु॰~९.३.७) \textcolor{red}{वन इव पुरेऽपि विचरति पुरुषं त्वामेव जानन्ती} (आ॰स॰श॰~४६०) इत्यादिषु।} परमत्र नुमभावः। \textcolor{red}{श्नाभ्यस्तयोरातः} (पा॰सू॰~६.४.११२) इति लोपेन।\footnote{नित्यत्वादन्त\-रङ्गत्वाच्च \textcolor{red}{श्नाभ्यस्तयोरातः} (पा॰सू॰~६.४.११२) इति लोपविधिः \textcolor{red}{आच्छीनद्योर्नुम्} (पा॰सू॰~७.१.८०) इत्यागम\-विधेर्बलीयानिति भावः। वि~\textcolor{red}{ज्ञा अवबोधने} (धा॰पा॰~१५०७)~\arrow वि~ज्ञा~\arrow \textcolor{red}{शेषात्कर्तरि परस्मैपदम्} (पा॰सू॰~१.३.७८)~\arrow \textcolor{red}{वर्तमाने लट्} (पा॰सू॰~३.२.१२३)~\arrow वि~ज्ञा~लट्~\arrow \textcolor{red}{लटः शतृशानचावप्रथमा\-समानाधिकरणे} (पा॰सू॰~३.२.१२४)~\arrow वि~ज्ञा~शतृँ~\arrow वि~ज्ञा~अत्~\arrow \textcolor{red}{क्र्यादिभ्यः श्ना} (पा॰सू॰~३.१.८१)~\arrow वि~ज्ञा~श्ना~अत्~\arrow वि~ज्ञा~ना~अत्~\arrow \textcolor{red}{ज्ञाजनोर्जा} (पा॰सू॰~७.३.७९)~\arrow वि~जा~ना~अत्~\arrow \textcolor{red}{श्नाभ्यस्तयोरातः} (पा॰सू॰~६.४.११२)~\arrow वि~जा~न्~अत्~\arrow \textcolor{red}{उगितश्च} (पा॰सू॰~४.१.६)~\arrow वि~जा~न्~अत्~ङीप्‌~\arrow वि~जा~न्~अत्~ई~\arrow विजानती~\arrow विभक्तिकार्यम्~\arrow विजानती~सुँप्~\arrow \textcolor{red}{हल्ङ्याब्भ्यो दीर्घात्सुतिस्यपृक्तं हल्} (पा॰सू॰~६.१.६८)~\arrow विजानती। कथं तर्हि \ref{fn:jananti}तम्यां टिप्पण्यामुद्धृतेषूदाहरणेषु नुम्भावः। \textcolor{red}{आगम\-शास्त्रमनित्यम्} (प॰शे॰~९३.२)। यद्वा धात्वन्तरः कल्प्यताम्।}
\end{sloppypar}
\section[भाषतोः]{भाषतोः}
\centering\textcolor{blue}{गुहलक्ष्मणयोरेवं भाषतोर्विमलं नभः।\nopagebreak\\
बभूव रामः सलिलं स्पृष्ट्वा प्रातः समाहितः॥}\nopagebreak\\
\raggedleft{–~अ॰रा॰~२.६.१६}\\
\begin{sloppypar}\hyphenrules{nohyphenation}\justifying\noindent\hspace{10mm} अत्र गुह\-लक्ष्मणयोर्भाषतोर्भाष\-माणयो रात्रिर्व्यतीता। \textcolor{red}{भाषतोः} इत्यपाणिनीयमिव। यतो हि भाषणार्थको \textcolor{red}{भाष्‌}\-धातुः (\textcolor{red}{भाषँ व्यक्तायां वाचि} धा॰पा॰~६१२) आत्मनेपदी। ततश्च \textcolor{red}{भाषेते इति भाषमाणौ तयोर्भाषमाणयोः} इति पाणिनीयम्। किन्तु \textcolor{red}{भाषेते इति भाषौ} पचादित्वादच्।\footnote{\textcolor{red}{नन्दि\-ग्रहि\-पचादिभ्यो ल्युणिन्यचः} (पा॰सू॰~३.१.१३४) इत्यनेन।} पुनः \textcolor{red}{भाषाविवाऽचरतो भाषतः}।\footnote{भाष~\arrow \textcolor{red}{सर्वप्राति\-पदिकेभ्य आचारे क्विब्वा वक्तव्यः} (वा॰~३.१.११)~\arrow भाष~क्विँप्~\arrow भाष~व्~\arrow \textcolor{red}{वेरपृक्तस्य} (पा॰सू॰~६.१.६७)~\arrow भाष~\arrow \textcolor{red}{सनाद्यन्ता धातवः} (पा॰सू॰~३.१.३२)~\arrow धातुसञ्ज्ञा~\arrow \textcolor{red}{शेषात्कर्तरि परस्मैपदम्} (पा॰सू॰~१.३.७८)~\arrow \textcolor{red}{वर्तमाने लट्} (पा॰सू॰~३.२.१२३)~\arrow भाष~लट्~\arrow भाष~तस्~\arrow \textcolor{red}{कर्तरि शप्‌} (पा॰सू॰~३.१.६८)~\arrow भाष~शप्~तस्~\arrow भाष~अ~तस्~\arrow \textcolor{red}{अतो गुणे} (पा॰सू॰~६.१.९७)~\arrow भाष~तस्~\arrow \textcolor{red}{ससजुषो रुः} (पा॰सू॰~८.२.६६)~\arrow भाषतरुँ~\arrow \textcolor{red}{खरवसानयोर्विसर्जनीयः} (पा॰सू॰~८.३.१५)~\arrow भाषतः।} आचारार्थे क्विप्। ततश्च धातुत्वाल्लटि तसि शपि। ततः \textcolor{red}{भाषत इति भाषन्तौ}\footnote{भाष~\arrow धातुसञ्ज्ञा (पूर्ववत्)~\arrow \textcolor{red}{शेषात्कर्तरि परस्मैपदम्} (पा॰सू॰~१.३.७८)~\arrow \textcolor{red}{वर्तमाने लट्} (पा॰सू॰~३.२.१२३)~\arrow भाष~लट्~\arrow \textcolor{red}{लटः शतृशानचावप्रथमा\-समानाधिकरणे} (पा॰सू॰~३.२.१२४)~\arrow भाष~शतृँ~\arrow भाष~अत्~\arrow \textcolor{red}{अतो गुणे} (पा॰सू॰~६.१.९७)~\arrow भाषत्~\arrow \textcolor{red}{कृत्तद्धित\-समासाश्च} (पा॰सू॰~१.२.४६)~\arrow प्रातिपदिक\-सञ्ज्ञा~\arrow विभक्तिकार्यम्~\arrow भाषत्~औ~\arrow \textcolor{red}{उगिदचां सर्वनामस्थानेऽधातोः} (पा॰सू॰~७.१.७०)~\arrow \textcolor{red}{मिदचोऽन्त्यात्परः} (पा॰सू॰~१.१.४७)~\arrow भाष~नुँम्~त्~औ~\arrow भाष~न्~त्~औ~\arrow भाषन्तौ। } इति विग्रहे परस्मैपद\-धातोः \textcolor{red}{शतृ}\-प्रत्यये \textcolor{red}{भाषतोः} षष्ठ्यन्तं सप्तम्यन्तं वा।\footnote{भाषत्~\arrow प्रातिपदिक\-सञ्ज्ञा (पूर्ववत्)~\arrow विभक्तिकार्यम्~\arrow भाषत्~ओस्~\arrow भाषतोस्~\arrow \textcolor{red}{ससजुषो रुः} (पा॰सू॰~८.२.६६)~\arrow भाषतोरुँ~\arrow \textcolor{red}{खरवसानयोर्विसर्जनीयः} (पा॰सू॰~८.३.१५)~\arrow भाषतोः।} ज्ञात्वाऽपि श्रीराम\-तत्त्वं कथनोपकथन\-व्याजेन लोके श्रीराम\-तत्त्वाविश्चिकीर्षया भाषणमाचरतोरिवेति ग्रन्थ\-तात्पर्यं प्रतिभाति। इदं ह्यध्यात्म\-रामायणम्। अत्र प्रत्यक्षरं निगूढ\-दर्शन\-पीयूष\-निर्भरम्। अतो व्याकरण\-प्रयोगा अपि विशेषमेव निगूढ\-रहस्यात्मकं तत्त्वं वितन्वन्तो लक्ष्यन्ते। यतो हि लक्ष्मणः साक्षान्नारायणस्य भगवतः श्रीरामस्यांशः। दार्शनिक\-दृष्ट्या च चतस्रोऽवस्था जाग्रत्स्वप्नसुषुप्तितुरीयाः। चतसृणामपि चत्वारो विभवो विराड्ढिरण्यगर्भ\-सर्वज्ञ\-ब्रह्माख्याः। तत्राशेष\-विशेषातीतं
ज्ञान\-गीर्गोऽतीतं 
निर्गुणं ब्रह्म तुरीयं श्रीरामः। तदवस्था तुरीयावस्था श्रीसीता। तुरीयावस्था\-ब्रह्मणोरिव सीता\-रामयोरप्यभेदः। सुषुप्त्यवस्था माण्डवी। तच्चैतन्याधिष्ठान\-देवता सर्वज्ञ ईश्वरो भरतः। स्वप्नावस्था श्रुतकीर्तिः। तच्चैतन्याधिष्ठानं हिरण्यगर्भः शत्रुघ्नः। जागृतावस्थोर्मिला। तद्विभुर्विराट् श्रीलक्ष्मणः। अतस्तुलसीदासो गायति~–\end{sloppypar}
\centering\textcolor{red}{सुन्दरी सुन्दर बरनि सह सब एक मंडप राजहीं।\nopagebreak\\
जनु जीव उर चारिउ अवस्था बिभुन सहित बिराजहीं॥}\footnote{एतद्रूपान्तरम्–\textcolor{red}{सुरम्याः सुरम्यैर्वरैः सर्ववध्वो व्यराजन्नभिन्ने शुभे मण्डपे च। ध्रुवं जीवचित्तेऽब्धिसङ्ख्या अवस्थाः स्वकीयैरधीशैर्युताः संव्यराजन्॥} (मा॰भा॰~१.३२५.१४)।}\nopagebreak\\
\raggedleft{–~रा॰च॰मा॰~१.३२५.१४}\\
\begin{sloppypar}\hyphenrules{nohyphenation}\justifying\noindent एवमेवेमे चत्वारो राम\-भरत\-लक्ष्मण\-शत्रुघ्नाः क्रमशो मोक्ष\-काम\-धर्मार्थ\-रूपाः। \textcolor{red}{धर्मादिष्वनियमः} (वा॰~२.२.३४) इति वार्त्तिकेनात्राभि\-प्रायानुसारं क्रम\-व्यत्यासः। एवम् \textcolor{red}{ओम्} (ॐ) इत्यत्र हि चत्वारि पदानि~– अ उ म् अर्धमात्रा च। तत्रार्ध\-मात्रात्मको रामो मकारो भरतो वासुदेव उकारः शत्रुघ्नो विधि\-रूपोऽकारो लक्ष्मणः शिव\-रूपश्चेति कथ्यते\footnote{\textcolor{red}{अकाराक्षर\-सम्भूतः सौमित्रिर्विश्व\-भावनः। उकाराक्षर\-सम्भूतः शत्रुघ्नस्तैजसात्मकः॥ प्राज्ञात्मकस्तु भरतो मकाराक्षर\-सम्भवः। अर्ध\-मात्रात्मको रामो ब्रह्मानन्दैक\-विग्रहः॥} (रा॰उ॰ता॰उ॰~३.१.२)}~–\end{sloppypar}
\centering\textcolor{red}{बेद तत्त्व नृप तव सुत चारी॥}\footnote{एतद्रूपान्तरम्–\textcolor{red}{चत्वारोऽपि सुता भूप वेदतत्त्वानि सन्ति ते} (मा॰भा॰~१.१९८.१)। दशरथं प्रति वसिष्ठस्य वचनमिदम्।}\nopagebreak\\
\raggedleft{–~रा॰च॰मा॰~१.१९८.१}\\
\begin{sloppypar}\hyphenrules{nohyphenation}\justifying\noindent इदं सर्वं कण्ठ\-रवेण कथितं विस्तर\-भयान्न निरूप्यते। एवं लक्ष्मणः साक्षाद्भगवान्निषादश्च नित्यो भगवत्परिकरः किं तयोः किमप्यज्ञातम्।\footnote{\textcolor{red}{तयोः} इत्यत्र \textcolor{red}{क्तस्य च वर्तमाने} (पा॰सू॰~२.३.६७) इत्यनेन षष्ठी। \textcolor{red}{मति\-बुद्धि\-पूजार्थेभ्यश्च} (पा॰सू॰~३.२.१८८) इत्यनेन \textcolor{red}{ज्ञानम्} इत्यत्र बुद्ध्यर्थे वर्तमाने क्तः। \textcolor{red}{बुद्धिर्ज्ञानम्} (का॰वृ॰~३.२.१८८) इति काशिका।} केवलं लोक\-लीलार्थं प्रश्नं प्रश्नोत्तरं कुर्वन्ताविव लक्ष्येते। अतो भावतो विदित\-राम\-तत्त्वतयाऽभाष\-माणयोरेव भाषणमाचक्षाणयोरिवानयोर्बाह्य\-परिवेषः।\end{sloppypar}
\section[निरहङ्कारिणः]{निरहङ्कारिणः}
\centering\textcolor{blue}{निरहङ्कारिणः शान्ता ये रागद्वेषवर्जिताः।\nopagebreak\\
समलोष्टाश्मकनकास्तेषां ते हृदयं गृहम्॥}\nopagebreak\\
\raggedleft{–~अ॰रा॰~२.६.५७}\\
\begin{sloppypar}\hyphenrules{nohyphenation}\justifying\noindent\hspace{10mm} अत्र प्राचेतसाश्रमं गत्वा सुनिवास\-स्थानं श्रीरामचन्द्रेण पृष्टो हृष्टो वाल्मीकिः प्रणिगदति \textcolor{red}{निरहङ्कारिणः} इति। यद्यप्यत्र द्वेधा समासः कर्तुं पार्यते तत्पुरुषो बहुव्रीहिश्च तत्पुरुषो यथा \textcolor{red}{अहङ्कारान्निष्क्रान्ता निरहङ्काराः} इति \textcolor{red}{निरादयः क्रान्ताद्यर्थे पञ्चम्या} (वा॰~२.२.१८) इति वचन\-सहायेन \textcolor{red}{कु\-गति\-प्रादयः} (पा॰सू॰~२.२.१८) इति सूत्रेण प्रथमः समासो द्वितीयश्च \textcolor{red}{प्रादिभ्यो धातुजस्य वाच्यो वा चोत्तर\-पद\-लोपश्च} (वा॰~२.२.२२) इत्यनेन बहुव्रीहिरपि \textcolor{red}{निर्गतोऽहङ्कारो येषामिति निरहङ्काराः} इत्थं द्वयोः समासयोः कृतयोरपि \textcolor{red}{निरहङ्कारिणः} इति विधाय तत्पुरुषं पुनर्मत्वर्थीय \textcolor{red}{इनिः} इति। \textcolor{red}{न कर्मधारयान्मत्वर्थीयो बहुव्रीहिश्चेत्तदर्थ\-प्रतिपत्ति\-करः}\footnote{मूलं मृग्यम्।} इति पाणिनि\-सम्मत\-नियममुल्लङ्घ्य किमेतेन
द्रविड\-प्राणायामेन। द्रविड\-प्राणायाम\-प्रकार\-निदर्शनं हि \textcolor{red}{निर्गतोऽहङ्कारो निरहङ्कारः} इत्यत्र \textcolor{red}{कु\-गति\-प्रादयः} (पा॰सू॰~२.२.१८) इत्यनेन तत्पुरुष\-समासः पश्चात् \textcolor{red}{निरहङ्कार एषां निरहङ्कारिणः} इति चेत्।\footnote{\textcolor{red}{अत इनिठनौ} (पा॰सू॰~५.२.११५) इत्यनेनेनिः।} उच्यते। यदि बहुव्रीहावभीष्टार्थ\-लाभः स्यात्तदेदमनुधावनं न स्यात्। तदैव नियमोल्लङ्घन\-रूपाऽपाणिनीयता स्यात्। बहुव्रीहौ नित्य\-निरहङ्कार\-रूपेऽभीष्टेऽर्थे न प्राप्ते नित्य\-योग\-विवक्षायामिनिरत एव न दोषः।\footnote{\textcolor{red}{नित्यं निरहङ्कार एषां निरहङ्कारिणः} इति विग्रहप्रकारः।} अतो गीतायामपि \textcolor{red}{निरहङ्कारः} इति प्रयुक्तं यथा~–\end{sloppypar}
\centering\textcolor{red}{अद्वेष्टा सर्वभूतानां मैत्रः करुण एव च।\nopagebreak\\
निर्ममो निरहङ्कारः समदुःखसुखः क्षमी॥}\nopagebreak\\
\raggedleft{–~भ॰गी॰~१२.१३}\\
\begin{sloppypar}\hyphenrules{nohyphenation}\justifying\noindent अर्थान्निरहङ्कारस्तु मम प्रियः किन्तु निरहङ्कारिणो हृदये वसामीत्येवान्तरं बहुव्रीहि\-मत्वर्थीययोः। एवं \textcolor{red}{निरहङ्कारिणः} इत्यत्र नित्यमहङ्काराभाव इति गूढमर्थं ध्वनयितुं तत्पुरुष\-बहुव्रीहि\-वर्त्म विहाय विकटः पन्था आश्रितः। यद्वा अन्यमपि दर्शयामि। \textcolor{red}{निरहं कर्तुं शीलं येषां ते निरहङ्कारिणः} इति \textcolor{red}{सुप्यजातौ णिनिस्ताच्छील्ये} (पा॰सू॰~३.२.७८) इत्यनेन सूत्रेण णिनि\-विधाने \textcolor{red}{निरहङ्कारिणः}।\end{sloppypar}
\section[प्रकाशन्तः]{प्रकाशन्तः}
\centering\textcolor{blue}{साक्षान्मया प्रकाशन्तो ज्वलनार्कसमप्रभाः।\nopagebreak\\
तानन्वधावं लोभेन तेषां सर्वपरिच्छदान्॥}\nopagebreak\\
\raggedleft{–~अ॰रा॰~२.६.६८}\\
\begin{sloppypar}\hyphenrules{nohyphenation}\justifying\noindent\hspace{10mm} अत्र वाल्मीकिर्भगवन्तं श्रीरामं प्रति पूर्व\-वृत्तान्त\-वर्णन\-प्रसङ्गे सप्तर्षि\-वर्णनं प्रस्तौति। अत्र \textcolor{red}{प्रकाशन्तः} इति प्रयोगः। प्रपूर्वको \textcolor{red}{काश्‌}\-धातुर्दीप्त्यर्थः (\textcolor{red}{काशृँ दीप्तौ} धा॰पा॰~११६२) अकर्मक आत्मनेपदी च। अत्र शानचा भवितव्यं शत्रन्त\-प्रयोगो ह्यपाणिनीय इति चेत्।
\textcolor{red}{प्रकाशन्तः} इत्थमाचार\-क्विबन्ताद्रूप\-सिद्धिः।\footnote{प्रकाशं कुर्वन्ति प्रकाशन्ति। प्रकाश~\arrow \textcolor{red}{सर्वप्राति\-पदिकेभ्य आचारे क्विब्वा वक्तव्यः} (वा॰~३.१.११)~\arrow प्रकाश~क्विँप्~\arrow प्रकाश~व्~\arrow \textcolor{red}{वेरपृक्तस्य} (पा॰सू॰~६.१.६७)~\arrow प्रकाश~\arrow \textcolor{red}{सनाद्यन्ता धातवः} (पा॰सू॰~३.१.३२)~\arrow धातुसञ्ज्ञा~\arrow \textcolor{red}{शेषात्कर्तरि परस्मैपदम्} (पा॰सू॰~१.३.७८)~\arrow \textcolor{red}{वर्तमाने लट्} (पा॰सू॰~३.२.१२३)~\arrow प्रकाश~लट्~\arrow प्रकाश~झि~\arrow \textcolor{red}{कर्तरि शप्‌} (पा॰सू॰~३.१.६८)~\arrow प्रकाश~शप्~झि~\arrow प्रकाश~अ~झि~\arrow \textcolor{red}{अतो गुणे} (पा॰सू॰~६.१.९७)~\arrow प्रकाश~झि~\arrow \textcolor{red}{झोऽन्तः} (पा॰सू॰~७.१.३)~\arrow प्रकाश~अन्ति~\arrow \textcolor{red}{अतो गुणे} (पा॰सू॰~६.१.९७)~\arrow प्रकाशन्ति। प्रकाशन्तीति प्रकाशन्तः। प्रकाश~\arrow धातुसञ्ज्ञा (पूर्ववत्)~\arrow \textcolor{red}{शेषात्कर्तरि परस्मैपदम्} (पा॰सू॰~१.३.७८)~\arrow \textcolor{red}{वर्तमाने लट्} (पा॰सू॰~३.२.१२३)~\arrow प्रकाश~लट्~\arrow \textcolor{red}{लटः शतृशानचावप्रथमा\-समानाधिकरणे} (पा॰सू॰~३.२.१२४)~\arrow प्रकाश~शतृँ~\arrow प्रकाश~अत्~\arrow \textcolor{red}{अतो गुणे} (पा॰सू॰~६.१.९७)~\arrow प्रकाशत्~\arrow \textcolor{red}{कृत्तद्धित\-समासाश्च} (पा॰सू॰~१.२.४६)~\arrow प्रातिपादिक\-सञ्ज्ञा~\arrow विभक्ति\-कार्यम्~\arrow प्रकाशत्~जस्~\arrow प्रकाशत्~अस्~\arrow \textcolor{red}{उगिदचां सर्वनामस्थानेऽधातोः} (पा॰सू॰~७.१.७०)~\arrow \textcolor{red}{मिदचोऽन्त्यात्परः} (पा॰सू॰~१.१.४७)~\arrow प्रकाश~नुँम्~त्~अस्~\arrow प्रकाश~न्~त्~अस्~\arrow प्रकाशन्तस्~\arrow \textcolor{red}{ससजुषो रुः} (पा॰सू॰~८.२.६६)~\arrow प्रकाशन्तरुँ~\arrow \textcolor{red}{खरवसानयोर्विसर्जनीयः} (पा॰सू॰~८.३.१५)~\arrow प्रकाशन्तः।} यद्वा \textcolor{red}{अनुदात्तेत्त्व\-लक्षणमात्मने\-पदमनित्यम्} (प॰शे॰~९३.४) मत्वा परस्मैपदत्वाच्छतृ\-प्रत्ययः।\footnote{प्र~\textcolor{red}{काशृँ दीप्तौ} (धा॰पा॰~११६२)~\arrow प्र~काश्~\arrow \textcolor{red}{अनुदात्तेत्त्व\-लक्षणमात्मने\-पदमनित्यम्} (प॰शे॰~९३.४)~\arrow \textcolor{red}{शेषात्कर्तरि परस्मैपदम्} (पा॰सू॰~१.३.७८)~\arrow \textcolor{red}{वर्तमाने लट्} (पा॰सू॰~३.२.१२३)~\arrow प्र~काश्~लट्~\arrow \textcolor{red}{लटः शतृशानचावप्रथमा\-समानाधिकरणे} (पा॰सू॰~३.२.१२४)~\arrow प्र~काश्~शतृँ~\arrow प्र~काश्~अत्~\arrow प्रकाशत्~\arrow \textcolor{red}{कृत्तद्धित\-समासाश्च} (पा॰सू॰~१.२.४६)~\arrow प्रातिपादिक\-सञ्ज्ञा। शेषा प्रक्रिया पूर्ववत्।}\end{sloppypar}
\section[निश्चयः]{निश्चयः}
\centering\textcolor{blue}{वयं स्थास्यामहे तावदागमिष्यसि निश्चयः।\nopagebreak\\
तथेत्युक्त्वा गृहं गत्वा मुनिभिर्यदुदीरितम्॥}\nopagebreak\\
\raggedleft{–~अ॰रा॰~२.६.७३}\\
\begin{sloppypar}\hyphenrules{nohyphenation}\justifying\noindent\hspace{10mm} अत्र सप्तर्षि\-कथितमेवानुवदति वाल्मीकिर्यत् \textcolor{red}{तावद्वयमत्र स्थास्यामो यावत्त्वं कृत\-निश्चय आगमिष्यसि}। \textcolor{red}{आगमिष्यसि निश्चयः} इति वाक्य\-खण्डे \textcolor{red}{निश्चयः} इति हि युष्मत्पद\-वाच्य\-वाल्मीकि\-रूप\-कर्तृ\-विशेषणम्। \textcolor{red}{त्वं निश्चयः सन् यावदागमिष्यसि तावद्वयं स्थास्यामः} इति भावः। \textcolor{red}{निश्चय}\-शब्दः प्रायो भावेऽबन्तः।\footnote{\textcolor{red}{निस्‌}\-पूर्वकात् \textcolor{red}{चिञ् चयने} (धा॰पा॰~१२५१) इति धातोः \textcolor{red}{एरच्} (पा॰सू॰~३.३.५६) इत्यनेन प्राप्तं \textcolor{red}{अच्‌}\-प्रत्ययं बाधित्वा \textcolor{red}{ग्रहवृदृ\-निश्चिगमश्च} (पा॰सू॰~३.३.५८) इत्यनेन भावेऽकर्तरि च कारके सञ्ज्ञायाम् \textcolor{red}{अप्‌}\-प्रत्यये निश्चय\-शब्दो निष्पन्नः।
} भाव\-प्रत्ययान्त\-शब्दः कथं कर्तृ\-वाचकस्य विशेषणमसमानाधि\-करणात्। असामानाधि\-करण्ये कथं स्याद्विभक्तिः कथमन्वयश्च। सामानाधि\-करण्याभावे \textcolor{red}{आगमिष्यसि} इति प्रयोगोऽपि नैव सम्भविष्यति। तथा च सूत्रम् \textcolor{red}{युष्मद्युपपदे समानाधिकरणे स्थानिन्यपि मध्यमः} (पा॰सू॰~१.४.१०५)। अस्यार्थः \textcolor{red}{तिङ्वाच्य\-कारक\-वाचिनि युष्मद्युपपदे समानाधिकरणे प्रयुज्यमानेऽप्रयुज्यमाने वा मध्यमः}। न च क्वचिदसमानाधि\-करणमपि विशेषणम्। विशेषण\-विशेष्ययोः सामानाधिकरण्ये नैव राजाज्ञा। \textcolor{red}{भूतले घटः}, \textcolor{red}{राज्ञः पुरुषः} इत्यादौ यथा सप्तम्यन्त\-षष्ठ्यन्तयोरपि विशेषणत्व\-दर्शनात्। सत्यम्। किन्त्वत्र समानाधिकरणत्वं नाम समान\-विभक्तिकत्वम्। समान\-विभक्तिकत्वं प्राय एकार्थ\-वाचकयोर्विशेषण\-विशेष्ययोरेव भवति।
यथा \textcolor{red}{नीलमुत्पलम्} अत्रोत्पल\-शब्दोत्तर\-सुब्विभक्तिर्नील\-पदोत्तर\-सुब्विभक्तिश्चोभे अप्येकमेवोत्पल\-रूपमर्थं नीलत्व\-विशिष्टं कथयतः। यतो हि \textcolor{red}{नील}\-शब्दस्य \textcolor{red}{उत्पल}\-शब्देनाभेद\-सम्बन्धेनान्वयः। तथाऽपि व्युत्पत्तिवादे श्रीगदाधरभट्टा व्युत्पादयन्ति \textcolor{red}{यत्र विशेष्य\-वाचक\-पदोत्तर\-विभक्ति\-तात्पर्य\-विषय\-सङ्ख्या\-विरुद्ध\-सङ्ख्याया अविवक्षितत्वं तत्रैव विशेष्य\-विशेषण\-वाचक\-पदयोः समान\-वचनकत्व\-नियमः} (व्यु॰वा॰ का॰प्र॰)। यथा \textcolor{red}{सुन्दरो रामः} इत्यत्र विशेष्य\-वाचक\-पदं \textcolor{red}{रामः} इति तदुत्तर\-विभक्तिः \textcolor{red}{सु} इति तत्तात्पर्य\-विषय\-सङ्ख्यैकत्व\-रूपा तद्विरुद्ध\-सङ्ख्या द्वित्व\-त्रित्वादयस्तासां विशेष्य\-वाचक\-सुन्दर\-पदोत्तर\-\textcolor{red}{सु}\-विभक्त्या न विवक्षितत्वमतोऽत्र समान\-वचनकत्वम्। इति मीमांसा\-मात्रे \textcolor{red}{आगमिष्यसि} इत्यस्य समानाधिकरणः \textcolor{red}{त्वम्} इति। तस्य सामानाधिकरणं \textcolor{red}{निश्चयः} इति। अन्यथा कथं प्रथमा\-विभक्तिः क्रियेत। स च तदैव \textcolor{red}{त्वम्} इत्यस्य सामानाधिकरणो भविष्यति यदा तदर्थमेव भाषेत। भावे विहिताजन्त\-निश्चय\-शब्दः कथं कर्त्रर्थमनुवदिष्यति। अस्यैकार्थ\-वाचकत्वे कथमत्र समान\-वचनकत्वमिति महत्पङ्कमिति चेत्। अत्र \textcolor{red}{निश्चिनोतीति निश्चयः} इति विग्रहे पचादित्वादच्।\footnote{\textcolor{red}{नन्दि\-ग्रहि\-पचादिभ्यो ल्युणिन्यचः} (पा॰सू॰~३.१.१३४) इत्यनेन।} यद्वा \textcolor{red}{निश्चयः} इति भावेऽबन्त एव तथाऽपि \textcolor{red}{निश्चयोऽस्त्यस्य} इति विग्रहेऽर्श\-आदित्वान्मत्वर्थीयोऽच्।\footnote{\textcolor{red}{अर्शआदिभ्योऽच्} (पा॰सू॰~५.२.१२७) इत्यनेन।} \textcolor{red}{निश्चयः} इत्यस्य \textcolor{red}{निश्चयवान्} इत्यर्थः। अथवा पृथक्पदं वाक्यभेदश्च। यदाऽऽगमिष्यसि तदा निश्चयो भविष्यतीति व्यञ्जनया ध्वन्यते। काव्यस्यात्मा हि ध्वनिस्तथा च ध्वन्यालोकेऽभिनन्दयन्त्यानन्द\-वर्धनाचार्याः~–\end{sloppypar}
\centering\textcolor{red}{काव्यस्यात्मा ध्वनिरिति बुधैर्यः समाम्नातपूर्व-\nopagebreak\\
स्तस्याभावं जगदुरपरे भाक्तमाहुस्तमन्ये।\nopagebreak\\
केचिद्वाचां स्थितमविषये तत्त्वमूचुस्तदीयं\nopagebreak\\
तेन ब्रूमः सहृदयमनःप्रीतये तत्स्वरूपम्॥}\nopagebreak\\
\raggedleft{–~ध्व॰~१.१}\\
\begin{sloppypar}\hyphenrules{nohyphenation}\justifying\noindent न च साहित्यिक\-मतेन वैयाकरणानां किमिति चेत्। आनन्दवर्धनाचार्यैरपि \textcolor{red}{बुध}\-शब्दे वैयाकरणानां ध्वनि\-स्वीकृतत्व\-प्रतिपादनात्। यथा बुधैर्वैयाकरणैः स एव ध्वनिः स्फोट\-रूपेण पूर्ण\-समाम्नातोऽभ्यस्तः। \textcolor{red}{येनोच्चारितेन सास्ना\-लाङ्गूल\-ककुद\-खुर\-विषाणिनां सम्प्रत्ययो भवति स शब्दः} (भा॰प॰)। न च काव्यात्मभूत\-ध्वनिरत्र किमायातम्। अध्यात्म\-रामायणस्य सर्वत्र काव्य\-रूपेण चर्चितत्वात् यथा~–\end{sloppypar}
\centering\textcolor{blue}{रामायणं जनमनोहरमादिकाव्यं\nopagebreak\\
ब्रह्मादिभिः सुरवरैरपि संस्तुतं च।\nopagebreak\\
श्रद्धान्वितः पठति यः शृणुयात्तु नित्यं\nopagebreak\\
विष्णोः प्रयाति सदनं स विशुद्धदेहः॥}\nopagebreak\\
\raggedleft{–~अ॰रा॰~७.९.७३}\\
\begin{sloppypar}\hyphenrules{nohyphenation}\justifying\noindent इत्थं \textcolor{red}{यदा आगमिष्यसि तदा निश्चयो भविष्यति} इति न दोषः।\end{sloppypar}
\section[स्थाप्य]{स्थाप्य}
\label{sec:sthapya}
\centering\textcolor{blue}{बहिरेव रथं स्थाप्य राजानं द्रष्टुमाययौ।\nopagebreak\\
जय शब्देन राजानं स्तुत्वा तं प्रणनाम ह॥}\nopagebreak\\
\raggedleft{–~अ॰रा॰~२.७.२}\\
\begin{sloppypar}\hyphenrules{nohyphenation}\justifying\noindent\hspace{10mm} सुमन्त्रः ससौमित्रि\-सीतं श्रीरामभद्रं रथेन गङ्गातटं यावत्प्रापय्य\footnote{\textcolor{red}{प्रापय्य} इत्यत्र \textcolor{red}{विभाषाऽऽपः} (पा॰सू॰~६.४.५७) इत्यनेन णेरयादेशे वैकल्पिके द्वियं रूपं सिद्धम्। पक्षे \textcolor{red}{प्राप्य} इति।} तदाज्ञयाऽयोध्यां परावर्तमानो रथं बहिः स्थापयित्वा म्रियमाणं राजानं दशरथं ददर्श। अत्र \textcolor{red}{बहिरेव रथं स्थाप्य} इत्यपाणिनीयमिव। यतो हि \textcolor{red}{ष्ठा गति\-निवृत्तौ} (धा॰पा॰~९२८) इति धातोर्ण्यन्ते \textcolor{red}{अर्ति\-ह्री\-व्ली\-री\-क्नूयी\-क्ष्माय्यातां पुङ्णौ} (पा॰सू॰~७.३.३६) इत्यनेन पुगागमे \textcolor{red}{क्त्वा}\-प्रत्यय इटि गुणेऽयादेशे \textcolor{red}{स्थापयित्वा} इति हि वरम्।\footnote{ष्ठा~\arrow \textcolor{red}{धात्वादेः षः सः} (पा॰सू॰~६.१.६४)~\arrow \textcolor{red}{निमित्तापाये नैमित्तिकस्याप्यपायः}~\arrow स्था~\arrow \textcolor{red}{हेतुमति च} (पा॰सू॰~३.१.२६)~\arrow स्था~णिच्~\arrow स्था~इ~\arrow \textcolor{red}{अर्ति\-ह्री\-व्ली\-री\-क्नूयी\-क्ष्माय्यातां पुङ्णौ} (पा॰सू॰~७.३.३६)~\arrow \textcolor{red}{आद्यन्तौ टकितौ} (पा॰सू॰~१.१.४६)~\arrow स्था~पुँक्~इ~\arrow स्था~प्~इ~\arrow स्थापि~\arrow \textcolor{red}{सनाद्यन्ता धातवः} (पा॰सू॰~३.१.३२)~\arrow धातुसञ्ज्ञा~\arrow \textcolor{red}{समानकर्तृकयोः पूर्वकाले} (पा॰सू॰~३.४.२१)~\arrow स्थापि~क्त्वा~\arrow स्थापि~त्वा~\arrow \textcolor{red}{आर्धधातुकस्येड्वलादेः} (पा॰सू॰~७.२.३५)~\arrow \textcolor{red}{आद्यन्तौ टकितौ} (पा॰सू॰~१.१.४६)~\arrow स्थापि~इट्~त्वा~\arrow स्थापि~इ~त्वा~\arrow \textcolor{red}{सार्वधातुकार्ध\-धातुकयोः} (पा॰सू॰~७.३.८४)~\arrow स्थापे~इ~त्वा~\arrow \textcolor{red}{एचोऽयवायावः} (पा॰सू॰~६.१.७८)~\arrow स्थापय्~इ~त्वा~\arrow स्थापयित्वा।} \textcolor{red}{रथं स्थाप्य} इति कथम्। \textcolor{red}{ल्यप्}\-प्रत्ययस्तु समासमन्तरेण भवितुं शक्य एव नहि। तथा च सूत्रम् \textcolor{red}{समासेऽनञ्पूर्वे क्त्वो ल्यप्} (पा॰सू॰~७.१.३७)। इदं ह्यनञ्पूर्वके समासे सत्येव क्त्वा\-प्रत्ययस्य स्थाने ल्यपं विदधाति। अत्र समासोऽपि नास्ति। अतः \textcolor{red}{स्थाप्य} इति विमृश्यते। अत्र \textcolor{red}{संस्थाप्य} इति प्रयोग आसीत्। तदा सम्पूर्वक\-णिजन्त\-स्था\-धातोः क्त्वा\-प्रत्यये ल्यबादेशः। एवं \textcolor{red}{विनाऽपि प्रत्ययं पूर्वोत्तर\-पद\-लोपो वक्तव्यः} (वा॰~५.३.८३) इत्यनेन \textcolor{red}{सम्} उपसर्गस्य लोपः। न च समासे कृते कुतोऽत्र पदत्वम्। अन्तर्वर्तिनीं विभक्तिमाश्रित्यैवानेन वार्त्तिकेन लोप\-करणात्।\footnote{\textcolor{red}{प्राग्रीश्वरान्निपाताः} (पा॰सू॰~१.४.५६) इत्यस्याधिकारे पठितत्वात् \textcolor{red}{प्रादयः} (पा॰सू॰~१.४.५८) इति सूत्रेण प्रादीनां \textcolor{red}{निपात}\-सञ्ज्ञायां जातायाम् \textcolor{red}{उपसर्गाः क्रियायोगे} (पा॰सू॰~१.४.५९) इत्यनेन क्रियायोगे तेषामेव \textcolor{red}{उपसर्ग}\-सञ्ज्ञायामुप\-सर्गाणां निपातत्वे सिद्धे \textcolor{red}{स्वरादि\-निपातमव्ययम्} (पा॰सू॰~१.१.३७) इत्यनेन निपातानाम् \textcolor{red}{अव्यय}\-सञ्ज्ञायां विहितायामुप\-सर्गाणामव्यय\-सञ्ज्ञायां सिद्धायाम् \textcolor{red}{अव्ययादाप्सुपः} (पा॰सू॰~२.४.८२) इत्यनेनोपसर्गात्सुपां लुक्यपि \textcolor{red}{प्रत्ययलोपे प्रत्ययलक्षणम्} (पा॰सू॰~१.१.६२) इति सूत्रबलेन \textcolor{red}{सुप्तिङन्तं पदम्} (पा॰सू॰~१.४.१४) इत्युपसर्गाणां प्रत्यय\-लक्षणा पदसञ्ज्ञाऽक्षतैवेति भावः।} \textcolor{red}{सत्यभामा भामा} (भा॰प॰, भा॰पा॰सू॰~१.१.४५) इत्यादावपि समस्त\-पदस्यैव लोप\-दर्शनात्। इत्थं \textcolor{red}{सम्} इत्यस्य लोपे सति प्रथमं \textcolor{red}{स्थाप्य} इति प्रामाणिक एव प्रयोगः। न च \textcolor{red}{निमित्तापाये नैमित्तिकस्याप्यपायः} इति परिभाषया निमित्त\-भूते समुपसर्गे लुप्ते नैमित्तिकस्य \textcolor{red}{ल्यप्} इत्यस्याप्यपायस्ततः \textcolor{red}{स्थाप्य} इत्यनिष्टं रूपमिति वाच्यम्। अस्याः परिभाषाया भाष्येऽदर्शनात्।\footnote{तस्मादनित्या परिभाषेयमिति भावः।} \textcolor{red}{जात\-संस्कारो न निवर्तते} इति वचन\-बलेन लुप्तेऽपि समुपसर्गे तन्निमित्तक\-ल्यपो न निवर्तनम्। यद्वा \textcolor{red}{स्थापयितुं शक्यं योग्यं वेति स्थाप्यम्} इति विग्रहे णिजन्त\-पुगन्त\-स्था\-धातोः \textcolor{red}{ऋहलोर्ण्यत्} (पा॰सू॰~३.१.१२४) इत्यनेन \textcolor{red}{ण्यत्} प्रत्ययः। अर्थादशक्यत्वादेकं स्थापयितुं योग्यं राजानं ददर्श। साम्प्रतं हि राज्ञो गतिर्निवृत्ता बुद्धिरपि शान्ता प्रत्यवसानमप्यवरुद्धं शब्दोऽपि शिथिलोऽतः \textcolor{red}{स्थाप्य}\-शब्दः \textcolor{red}{राजन्} शब्दस्य विशेषणम्। तथा \textcolor{red}{स्थाप्यश्चासौ राजा चेति स्थाप्यराजा तं स्थाप्य\-राजानम्}। \textcolor{red}{देवराजानम्} (अ॰रा॰~१.५.२६) इतिवट्टजभावः।\footnote{\pageref{sec:devarajanam}तमे पृष्ठे \ref{sec:devarajanam} \nameref{sec:devarajanam} इति प्रयोगस्य विमर्शं पश्यन्तु।} यद्वा \textcolor{red}{स्थीयत इति स्था}।\footnote{\textcolor{red}{ष्ठा गतिनिवृत्तौ} (धा॰पा॰~९२८) इति धातोः \textcolor{red}{अन्येष्वपि दृश्यते} (पा॰सू॰~३.२.१०१) इत्यनेन डे \textcolor{red}{डित्यभस्याप्यनु\-बन्धकरण\-सामर्थ्यात्} (वा॰~६.४.१४३) इत्यनेन टिलोपे \textcolor{red}{अजाद्यतष्टाप्‌} (पा॰सू॰~४.१.४) इत्यनेन टापि। दृशिग्रहणादपि\-ग्रहणाच्चानुप\-पदेऽपि। \textcolor{red}{अपिशब्दः सर्वोपाधि\-व्यभिचारार्थः} (का॰वृ॰~३.२.१०१, वै॰सि॰कौ॰~३०११)। यद्वा \textcolor{red}{अन्येभ्योऽपि दृश्यते} (पा॰सू॰~३.२.१७८) इत्यनेन क्विपि। दृशिग्रहणाद्भावेऽपि। यद्वा भिदादि\-गणमाकृति\-गणं मत्वा \textcolor{red}{षिद्भिदादिभ्योऽङ्} (पा॰सू॰~३.३.१०४) इत्यनेनाङि \textcolor{red}{अजाद्यतष्टाप्‌} (पा॰सू॰~४.१.४) इत्यनेन टापि। कथं तर्हि \textcolor{red}{स्थितिः} इति क्तिन्। \textcolor{red}{अनर्थकास्तु प्रतिवर्णमर्थानुपलब्धेः} (भा॰शि॰) इति भाष्य\-प्रयोगात्षिद्भिदादिभ्यो बाहुलकात्क्तिन्नपि। अन्यथा \textcolor{red}{डुलभँष् प्राप्तौ} (धा॰पा॰~९७५) इत्यस्य षित्त्वात्क्तिन्बाधो दुर्वारः।
}
तथा च \textcolor{red}{स्थयाऽऽप्तुं योग्यमिति स्थाप्यम्} इत्थं भाव\-साधित\-स्था\-शब्देनाऽप्य\-शब्दस्य समासः। \textcolor{red}{आप्य}\-शब्दोऽपि ण्यत्प्रत्ययान्त इति। तथा गति\-निवृत्त्याऽव्याप्तमिति तात्पर्यम्। अथवा म्रियमाणत्वाच्चितायां स्थापयितुं योग्यं राजानं ददर्श। न च \textcolor{red}{स्थाप्य}\-शब्दस्य \textcolor{red}{राजन्} शब्देन सह तत्पुरुष\-समासे \textcolor{red}{राजाऽहस्सखिभ्यष्टच्} (पा॰सू॰~५.४.९१) इत्यनेन \textcolor{red}{टच्} प्रत्यये टिलोपे विभक्ति\-कार्येऽमि \textcolor{red}{स्थाप्य\-राजम्} स्यादिति वाच्यम् \textcolor{red}{महाराजम्} इतिवत्। समासान्त\-प्रत्यय\-प्रकरणं ह्यनित्यम्। प्रमाणं चात्र \textcolor{red}{यचि भम्} (पा॰सू॰~१.४.१८) इति सूत्रम्। अत्र \textcolor{red}{यश्चाच्च यच्} इति समाहार\-द्वन्द्वः। इह \textcolor{red}{द्वन्द्वाच्चु\-दषहान्तात्समाहारे} (पा॰सू॰~५.४.१०६) इत्यनेन चान्तत्वाट्टच्प्रत्ययः प्रयोक्तव्य आसीत्। तस्मिन् प्रयुक्ते \textcolor{red}{यचे भम्} इति स्यात्। यतो न प्रयुक्तोऽतः समासान्त\-प्रत्ययस्यानित्यता ज्ञायते।\footnote{अन्यत्राप्येतज्ज्ञापितं भगवता पाणिनिना। प्रथमसूत्र एव \textcolor{red}{वृद्धिरादैच्} (पा॰सू॰~१.१.१) इति प्रयुक्तम्। समासान्त\-प्रत्यय\-नित्यत्वे तु \textcolor{red}{द्वन्द्वाच्चु\-दषहान्तात्समाहारे} (पा॰सू॰~५.४.१०६) इत्यनेन चान्तत्वाट्टजन्तरूपेण \textcolor{red}{वृद्धिरादैचम्} इत्यनेन भवितव्यमासीत्। हलन्तप्रयोगोऽत्रापि समासान्त\-प्रत्ययानित्यत्व\-ज्ञापनार्थम्।} अतः \textcolor{red}{स्थाप्य\-राजानम्} अस्मिन्प्रयोगे नापाणिनीयता।\end{sloppypar}
\section[अवेक्षती]{अवेक्षती}
\label{sec:aveksati}
\centering\textcolor{blue}{सीता चाश्रुपरीताक्षी मामाह नृपसत्तम।\nopagebreak\\
दुःखगद्गदया वाचा रामं किञ्चिदवेक्षती॥}\nopagebreak\\
\raggedleft{–~अ॰रा॰~२.७.१२}\\
\begin{sloppypar}\hyphenrules{nohyphenation}\justifying\noindent\hspace{10mm} शोकाकुल\-दशरथं प्रति राम\-सन्देशं वर्णयित्वा सुमन्त्रः साम्प्रतं सीता\-मनोदशां प्रतिपादयति यदश्रु\-परीत\-लोचना सीता निरीक्ष्य राम\-भद्रं किमपि समादिशत्। यदिह \textcolor{red}{किञ्चिदवेक्षती} अयं प्रयोगः \textcolor{red}{अवेक्षमाणा} इति प्रयोक्तव्ये \textcolor{red}{अवेक्षती} इति
प्रयुक्तमत्रांश\-द्वयेऽपाणिनीयता\-भ्रान्तिः।
\textcolor{red}{अव}पूर्वको हि \textcolor{red}{ईक्ष्}धातुः (\textcolor{red}{ईक्षँ दर्शने} धा॰पा॰~६१०) आत्मनेपदी। एवं \textcolor{red}{तङानावात्मनेपदम्} (पा॰सू॰~१.४.१००) इत्यनेन विधीयमानः \textcolor{red}{शानच्} प्रत्ययोऽप्यात्मनेपद\-सञ्ज्ञकः। तथा च \textcolor{red}{आने मुक्} (पा॰सू॰~७.२.८२) इत्यनेन मुकि कृते \textcolor{red}{अवेक्षमाणा} इत्येव प्रयोगः प्रचलितः। एवं \textcolor{red}{अवेक्षती} इति विमृश्यते। प्रथमं त्वात्मनेपद\-प्रयुक्त\-धातोः परस्मैपदवद्व्यवहारः। सति परस्मैपदे प्रत्यय\-विचारस्तत्रैव नुमभाव\-निस्तारः। \textcolor{red}{अनुदात्तेत्त्व\-लक्षणमात्मने\-पदमनित्यम्} (प॰शे॰~९३.४) इति वचनेन सति परस्मैपदे \textcolor{red}{अवेक्षति} इति विग्रहे शतृ\-प्रत्ययान्त\-रूपम्। ननु \textcolor{red}{अनुदात्तेत्त्व\-लक्षणमात्मने\-पदमनित्यम्} (प॰शे॰~९३.४) अत्र न किमपि प्रमाणमिति चेन्न। \textcolor{red}{चक्षिँङ् व्यक्तायां वाचि} (धा॰पा॰~१०१७) इति हि धातुः। अयं चानुदात्तेन्ङिच्च। द्वयोरप्यनुबन्धयोरेकमेव फलमात्मनेपदम्। तत्र सूत्रम् \textcolor{red}{अनुदात्तङित आत्मनेपदम्} (पा॰सू॰~१.३.१२) एव। अस्यैव फलस्य कृते द्वयोरनुबन्ध\-करणत्वं किमर्थम्। अनुदात्तेत्त्व\-करणेनैवाऽत्मनेपद\-सिद्धिः। ङित्करणेन किं प्रयोजनम्। तदेव व्यर्थं सज्ज्ञापयति यदनुदातेत्त्व\-लक्षणमात्मनेपदं तत्कार्यं चानित्यम्। अतस्तस्यानित्यत्वात्परस्मैपद\-शतृ\-प्रयोगः। शतृ\-प्रयोगे सति कथं नुमभावः इति चेत्। आगम\-शास्त्रस्यानित्यत्वात्।\footnote{\textcolor{red}{आगम\-शास्त्रमनित्यम्} (प॰शे॰~९३.२)। \textcolor{red}{शप्श्यनोर्नित्यम्} (पा॰सू॰~७.१.८१) इत्यनेन शपि नुमागमस्य नित्यत्वादागम\-शास्त्रस्यानित्यत्वमाश्रितम्।} यद्वाऽत्र न शतृ\-प्रत्ययः। अत्र \textcolor{red}{तृँच्} प्रत्यय औणादिकः।\footnote{नायं \textcolor{red}{बहुलमन्यत्रापि} (प॰उ॰~२.९५) इति तृच्। स नोगित्। \textcolor{red}{कार्याद्विद्यादनूबन्धम्} (भा॰पा॰सू॰~३.३.१) \textcolor{red}{केचिदविहिता अप्यूह्याः} (वै॰सि॰कौ॰~३१६९) इत्यनुसारमूह्योऽ\-यमविहित उगित्प्रत्ययः। \textcolor{red}{तृँच्} प्रत्यये चात्र शबागमोऽप्यूह्यः। \textcolor{red}{नयतेः षुगागमः} (प॰उ॰ श्वे॰वृ॰~२.९६) इतिवत्।}
एवं नुमभावः। तथा च स्त्रीत्व\-विवक्षायां ङीप्प्रत्यये\footnote{\textcolor{red}{उगितश्च} (पा॰सू॰~४.१.६) इत्यनेन ङीप्।} \textcolor{red}{अवेक्षती} इति सिद्धम्।\end{sloppypar}
\section[रुदन्ती]{रुदन्ती}
\centering\textcolor{blue}{ततो दुःखेन महता पुनरेवाहमागतः।\nopagebreak\\
ततो रुदन्ती कौसल्या राजानमिदमब्रवीत्॥}\nopagebreak\\
\raggedleft{–~अ॰रा॰~२.७.१५}\\
\begin{sloppypar}\hyphenrules{nohyphenation}\justifying\noindent\hspace{10mm} भगवतः श्रीरामचन्द्रस्य वनवासे सत्ययोध्यामागते सुमन्त्रे निशम्य सकलं समाचारं विलपन्तं रघुनन्दं निरानन्दं म्रियमाणं राजानं विलोक्य रुदती कौशल्याऽब्रवीत्। अत्र \textcolor{red}{रुदन्ती} इति प्रयुक्तम्। तच्चासङ्गतमिव। यतो हि \textcolor{red}{रुद्‌}\-धातुः (\textcolor{red}{रुदिँर् अश्रुविमोचने} धा॰पा॰~१०६७) अदादि\-गण\-पठितः। तत्र \textcolor{red}{अदिप्रभृतिभ्यः शपः} (पा॰सू॰~२.४.७२) इत्यनेन शपो लुग्भवति। ततश्च शतृ\-प्रत्यये नुमभावः स्वत एव सिद्धः।\footnote{\textcolor{red}{रुदिँर् अश्रुविमोचने} धा॰पा॰~१०६७)~\arrow रुद्~\arrow \textcolor{red}{लक्षणहेत्वोः क्रियायाः} (पा॰सू॰~३.२.१२६)~\arrow रुद्~शतृँ~\arrow \textcolor{red}{सार्वधातुकमपित्} (पा॰सू॰~१.२.४)~\arrow शतुर्ङिद्वत्त्वम्~\arrow रुद्~अत्~\arrow \textcolor{red}{कर्तरि शप्‌} (पा॰सू॰~३.१.६८)~\arrow रुद्~शप्~अत्~\arrow \textcolor{red}{अदिप्रभृतिभ्यः शपः} (पा॰सू॰~२.४.७२)~\arrow रुद्~अत्~\arrow \textcolor{red}{ग्क्ङिति च} (पा॰सू॰~१.१.५)~\arrow लघूपध\-गुणनिषेधः~\arrow रुद्~अत्~\arrow रुदत्~\arrow \textcolor{red}{उगितश्च} (पा॰सू॰~४.१.६)~\arrow रुदत्~ङीप्~\arrow रुदत्~ई~\arrow रुदती~\arrow विभक्ति\-कार्यम्~\arrow रुदती~सुँ~\arrow \textcolor{red}{हल्ङ्याब्भ्यो दीर्घात्सुतिस्यपृक्तं हल्} (पा॰सू॰~६.१.६८)~\arrow सुलोपः~\arrow रुदती। \textcolor{red}{आच्छीनद्योर्नुम्} (पा॰सू॰~७.१.८०) इति सूत्रेणावर्णान्तादङ्गादुत्तरस्य शतुः शीनद्योः परतः पाक्षिकनुमागमो भवति। शपो लुक्यङ्गस्य हलन्तत्वादेतत्सूत्रं न प्रवर्तते।} अत्र नुमपाणिनीय इति कृत्वा विमृश्यते। \textcolor{red}{गण\-कार्यमनित्यम्} (प॰शे॰~९३.३)। अतः शपो न लुक्। तथा च \textcolor{red}{रुदतीति रुदन्ती} इति विग्रहे नुम्। यद्वा \textcolor{red}{रोदनं रुदः}। भावेऽच्प्रत्ययः।\footnote{\textcolor{red}{अजपि सर्वधातुभ्यः} (वा॰~३.१.१३४) इत्यनेन। बाहुलकाद्भावे। यथा \textcolor{red}{विरुदः} इत्यत्र। \textcolor{red}{विरुद}\-शब्दो भावेऽपीत्याप्टे\-कोशः।} \textcolor{red}{रुदमाचरतीति रुदति}। आचारार्थे क्विप्। ततो धातु\-सञ्ज्ञातो लटि तिपि शपि \textcolor{red}{रुदति}।\footnote{रुद~\arrow \textcolor{red}{सर्वप्राति\-पदिकेभ्य आचारे क्विब्वा वक्तव्यः} (वा॰~३.१.११)~\arrow रुद~क्विँप्~\arrow रुद~व्~\arrow \textcolor{red}{वेरपृक्तस्य} (पा॰सू॰~६.१.६७)~\arrow रुद~\arrow \textcolor{red}{सनाद्यन्ता धातवः} (पा॰सू॰~३.१.३२)~\arrow धातुसञ्ज्ञा~\arrow \textcolor{red}{शेषात्कर्तरि परस्मैपदम्} (पा॰सू॰~१.३.७८)~\arrow \textcolor{red}{वर्तमाने लट्} (पा॰सू॰~३.२.१२३)~\arrow रुद~लट्~\arrow रुद~तिप्~\arrow रुद~ति~\arrow \textcolor{red}{कर्तरि शप्‌} (पा॰सू॰~३.१.६८)~\arrow रुद~शप्~ति~\arrow रुद~अ~ति~\arrow \textcolor{red}{अतो गुणे} (पा॰सू॰~६.१.९७)~\arrow रुद~ति~\arrow रुदति।} अतः \textcolor{red}{रुदतीति रुदन्ती} इत्थं साधु।\footnote{रुद~\arrow धातुसञ्ज्ञा (पूर्ववत्)~\arrow \textcolor{red}{शेषात्कर्तरि परस्मैपदम्} (पा॰सू॰~१.३.७८)~\arrow \textcolor{red}{वर्तमाने लट्} (पा॰सू॰~३.२.१२३)~\arrow रुद~लट्~\arrow \textcolor{red}{लटः शतृशानचावप्रथमा\-समानाधिकरणे} (पा॰सू॰~३.२.१२४)~\arrow रुद~शतृँ~\arrow रुद~अत्~\arrow \textcolor{red}{अतो गुणे} (पा॰सू॰~६.१.९७)~\arrow रुदत्~\arrow \textcolor{red}{उगितश्च} (पा॰सू॰~४.१.६)~\arrow रुदत्~ङीप्~\arrow रुदत्~ई~\arrow \textcolor{red}{शप्श्यनोर्नित्यम्} (पा॰सू॰~७.१.८१)~\arrow \textcolor{red}{आद्यन्तौ टकितौ} (पा॰सू॰~१.१.४६)~\arrow रुद~नुँम्~त्~ई~\arrow रुद~न्~त्~ई~\arrow रुदन्ती~\arrow विभक्ति\-कार्यम्~\arrow रुदन्ती~सुँ~\arrow \textcolor{red}{हल्ङ्याब्भ्यो दीर्घात्सुतिस्यपृक्तं हल्} (पा॰सू॰~६.१.६८)~\arrow रुदन्ती। कर्तर्यच्यपि न दोषः। एवं तर्हि रुदतीति रुदः। रुद इवाऽचरतीति रुदति। रुदतीति रुदन्ती। प्रक्रिया पूर्ववत्।}\end{sloppypar}
\section[गृह्य]{गृह्य}
\centering\textcolor{blue}{अतिवृद्धावन्धदृशौ क्षुत्पिपासार्दितौ निशि।\nopagebreak\\
नायाति सलिलं गृह्य पुत्रः किं वात्र कारणम्॥}\nopagebreak\\
\raggedleft{–~अ॰रा॰~२.७.३१}\\
\begin{sloppypar}\hyphenrules{nohyphenation}\justifying\noindent\hspace{10mm} अत्र महाराजो दशरथः श्रमणकुमारात्सजलं कलशं गृहीत्वा वृद्ध\-दम्पती गतः। \textcolor{red}{सलिलं गृह्य पुत्रो नायाति अत्र किं वा कारणम्}~– इति तौ चिन्तितवन्तावास्ताम्। ल्यबन्त\-प्रयोगोऽयम् \textcolor{red}{गृह्य}। स च समस्त\-पदं विनाऽसम्भवः। इह कोऽपि समासो नहीति। मैवम्। \textcolor{red}{प्रगृह्य} इति पदम्। तस्य च प्रोपसर्गस्य \textcolor{red}{विनाऽपि प्रत्ययं पूर्वोत्तर\-पद\-लोपो वक्तव्यः} (वा॰~५.३.८३) इति वार्त्तिकेन लोपे तथा च \textcolor{red}{जात\-संस्कारो न निवर्तते} इति परिभाषा\-बलेन ल्यबपि न निवृत्त इत्थं नापाणिनीयता। यद्वा \textcolor{red}{गृह्य}\-शब्दो हि क्यप्प्रत्ययान्तः।\footnote{\textcolor{red}{पदास्वैरिबाह्यापक्ष्येषु च} (पा॰सू॰~३.१.११९) इत्यनेन।} कित्वात्सम्प्रसारणम्।\footnote{\textcolor{red}{ग्रहि\-ज्या\-वयि\-व्यधि\-वष्टि\-विचति\-वृश्चति\-पृच्छति\-भृज्जतीनां ङिति च} (पा॰सू॰~६.१.१६) इत्यनेन।}
एवं \textcolor{red}{गृह्यो ग्रहीतुं योग्यः पुत्रः सौम्यः} इति तात्पर्यम्।\footnote{अस्वैरीति भावः। \textcolor{red}{गृह्यः, त्रि॰~(गृह्यते स्वाम्यादिभिरिति। ग्रह + क्यप्।), अस्वैरी। अस्वतन्त्रः। पक्षः। इति विश्वमेदिन्यौ} इति शब्द\-कल्पद्रुमः।} तथा च \textcolor{red}{गृह्यश्चासौ पुत्रश्चेति गृह्यपुत्रः} इति कर्मधारयः। \textcolor{red}{आदाय} इति चाध्याहार्यम्।\footnote{\textcolor{red}{सलिलमादाय गृह्यपुत्रो नायाति अत्र किं वा कारणम्} इति योजना।}\end{sloppypar}
\section[क्रन्दमानौ]{क्रन्दमानौ}
\centering\textcolor{blue}{हाहेति क्रन्दमानौ तौ पुत्रपुत्रेत्यवोचताम्।\nopagebreak\\
जलं देहीति पुत्रेति किमर्थं न ददास्यलम्॥}\nopagebreak\\
\raggedleft{–~अ॰रा॰~२.७.४३}\\
\begin{sloppypar}\hyphenrules{nohyphenation}\justifying\noindent\hspace{10mm} अत्र मुमूर्षु\-दम्पत्योर्दशां कौशल्या\-समक्षं दशरथो वर्णयति \textcolor{red}{हाहेति क्रन्दमानौ तौ}। \textcolor{red}{क्रन्द्‌}\-धातुः (\textcolor{red}{क्रदिँ आह्वाने रोदने च} धा॰पा॰~७१) परस्मैपदी। \textcolor{red}{इदितो नुम् धातोः} (पा॰सू॰~७.१.५८) इत्यनेन \textcolor{red}{नुम्}। ततश्च \textcolor{red}{क्रन्दत इति क्रन्दन्तौ} अयमेव प्रयोगः सामान्यतः पाणिनीयः। \textcolor{red}{क्रन्दमानौ} इति कथमिति चेत्। \textcolor{red}{कर्तरि कर्म\-व्यतिहारे} (पा॰सू॰~१.३.१४) इत्यनेनात्रात्मनेपदम्। कर्म\-व्यतिहारो नाम कार्य\-परिवर्तनम्। \textcolor{red}{अन्यस्य योग्यं लवनमन्यः करोति व्यतिलुनीते} (वै॰सि॰कौ॰~२६८०)। तथैवाऽत्र साधारण\-जीवाचरितं क्रन्दनं तपःपूतावपीमौ कुरुतः। अत आत्मनेपदम्। \textcolor{red}{क्रन्देते इति क्रन्दमानौ}। शानच्प्रत्यये शप्यनुबन्ध\-कार्ये \textcolor{red}{आने मुक्} (पा॰सू॰~७.२.८२) इत्यनेन मुगागमे विभक्ति\-कार्ये \textcolor{red}{क्रन्दमानौ}। यद्वा \textcolor{red}{क्रन्दतीति क्रन्दम्}। क्रन्दनशीलम्। पचादित्वादच्।\footnote{\textcolor{red}{नन्दि\-ग्रहि\-पचादिभ्यो ल्युणिन्यचः} (पा॰सू॰~३.१.१३४) इत्यनेन।} \textcolor{red}{क्रन्दं मानं शरीर\-दशा ययोस्तौ क्रन्दमानौ} इति बहुव्रीहौ नापाणिनीयता\-शङ्का\-पङ्क\-कलङ्कावकाशः।\end{sloppypar}
\section[विलपितम्]{विलपितम्}
\centering\textcolor{blue}{असमर्प्यैव रामाय राज्ञे मां क्व गतोऽसि भोः।\nopagebreak\\
इति विलपितं पुत्रं पतितं मुक्तमूर्धजम्॥}\nopagebreak\\
\raggedleft{–~अ॰रा॰~२.७.६७}\\
\begin{sloppypar}\hyphenrules{nohyphenation}\justifying\noindent\hspace{10mm} दशरथ\-मरणानन्तरं वसिष्ठ\-निर्देशाद्धावकैरयोध्यां नीतो भरतः कैकयीतः सकल\-वृत्तान्तं विज्ञायोपरतं पितरमुद्दिश्य विलपति। \textcolor{red}{असमर्प्यैव} इत्यादि। अत्र \textcolor{red}{विलपितम्} इति भरतस्य विशेषणं \textcolor{red}{विलपितं पुत्रम्} इति। \textcolor{red}{क्त}\-प्रत्ययस्य कर्मणि भावे च विधानात् \textcolor{red}{क्तवतु}\-प्रत्ययस्य पारिशेष्यात्कर्तरि विधानात् \textcolor{red}{विलपितवन्तं पुत्रम्} इति प्रयोक्तव्यमासीत्। किन्त्वविवक्षातः कर्मणो धातुमिममकर्मकं मत्वा\footnote{\textcolor{red}{धातोरर्थान्तरे वृत्तेर्धात्वर्थेनोपसङ्ग्रहात्। प्रसिद्धेरविवक्षातः कर्मणोऽकर्मिका क्रिया॥} (वा॰प॰~३.७.८८)।} \textcolor{red}{गत्यर्थाकर्मक\-श्लिष\-शीङ्स्थास\-वस\-जन\-रुह\-जी\-र्यतिभ्यश्च} (पा॰सू॰~३.४.७२) इत्यनेन कर्तरि \textcolor{red}{क्त}\-प्रत्ययः। यद्वा भावे \textcolor{red}{क्त}\-प्रत्यये कृते\footnote{\textcolor{red}{नपुंसके भावे क्तः} (पा॰सू॰~३.३.११४) इत्यनेन।} \textcolor{red}{विलपितमस्त्यस्य} इत्यर्शआद्यजन्तं \textcolor{red}{विलपितम्}।\footnote{\textcolor{red}{अर्शआदिभ्योऽच्} (पा॰सू॰~५.२.१२७) इत्यनेन।}
अथवा \textcolor{red}{विलपनं विलपः}। 
पचाद्यच्।\footnote{\textcolor{red}{अजपि सर्वधातुभ्यः} (वा॰~३.१.१३४) इत्यनेन। बाहुलकाद्भावे। 
} \textcolor{red}{विलपमितो विलपितस्तं विलपितम्} इति द्वितीयान्त\-विलप\-शब्दस्य क्तान्त\-\textcolor{red}{इत}\-इत्यनेन समासः। शकन्ध्वादित्वात्पूर्वोक्त\-दिशा पर\-रूपे \textcolor{red}{विलपितम्}। यद्वा \textcolor{red}{विलप}\-शब्दं तारकादि\-गणे मत्वा 
\textcolor{red}{विलपोऽस्य सञ्जातः} 
इति विग्रहे \textcolor{red}{तदस्य सञ्जातं तारकादिभ्य इतच्} (पा॰सू॰~५.२.३६) इत्यनेनेतच्यनुबन्ध\-कार्ये
भ\-सञ्ज्ञालोपयोर्विभक्ति\-कार्ये
\textcolor{red}{विलपितम्}।\footnote{\textcolor{red}{यचि भम्} (पा॰सू॰~१.४.१८) इत्यनेन भसञ्ज्ञा। \textcolor{red}{यस्येति च} (पा॰सू॰~६.४.१४८) इत्यनेनालोपः। अमि विभक्तौ \textcolor{red}{अमि पूर्वः} (पा॰सू॰~६.१.१०७) इत्यनेन पूर्वरूपम्।} यद्वा \textcolor{red}{विलपमाचष्टे विलपं करोति वा विलपयति}।\footnote{विलप~\arrow \textcolor{red}{तत्करोति तदाचष्टे} (धा॰पा॰ ग॰सू॰)~\arrow विलप~णिच्~\arrow विलप~इ~\arrow \textcolor{red}{णाविष्ठवत्प्राति\-पदिकस्य पुंवद्भाव\-रभाव\-टिलोप\-यणादि\-परार्थम्} (वा॰~६.४.४८)~\arrow विलप्~इ~\arrow विलपि~\arrow \textcolor{red}{सनाद्यन्ता धातवः} (पा॰सू॰~३.१.३२)~\arrow धातु\-सञ्ज्ञा~\arrow \textcolor{red}{शेषात्कर्तरि परस्मैपदम्} (पा॰सू॰~१.३.७८)~\arrow \textcolor{red}{वर्तमाने लट्} (पा॰सू॰~३.२.१२३)~\arrow विलपि~तिप्~\arrow विलपि~ति~\arrow \textcolor{red}{कर्तरि शप्‌} (पा॰सू॰~३.१.६८)~\arrow विलपि~शप्~ति~\arrow विलपि~अ~ति~\arrow \textcolor{red}{सार्वधातुकार्ध\-धातुकयोः} (पा॰सू॰~७.३.८४)~\arrow विलपे~अ~ति~\arrow \textcolor{red}{एचोऽयवायावः} (पा॰सू॰~६.१.७८)~\arrow विलपय्~अ~ति~\arrow विलपयति।} \textcolor{red}{विलपयतीति विलपितस्तं विलपितम्}।\footnote{विलपि~\arrow पूर्ववद्धातुसञ्ज्ञा~\arrow \textcolor{red}{गत्यर्थाकर्मक\-श्लिष\-शीङ्स्थास\-वस\-जन\-रुह\-जी\-र्यतिभ्यश्च} (पा॰सू॰~३.४.७२)~\arrow विलपि~क्त~\arrow विलपि~त~\arrow \textcolor{red}{आर्धधातुकस्येड्वलादेः} (पा॰सू॰~७.२.३५)~\arrow विलपि~इट्~त~\arrow विलपि~इ~त~\arrow \textcolor{red}{निष्ठायां सेटि} (पा॰सू॰~६.४.५२)~\arrow विलप्~इ~त~\arrow विलपित~\arrow विभक्तिकार्यम्~\arrow विलपित~अम्~\arrow \textcolor{red}{अमि पूर्वः} (पा॰सू॰~६.१.१०७)~\arrow विलपितम्।} इत्थमाचक्षाण\-णिजन्तात्कर्तरि क्त\-प्रत्ययः।\footnote{\textcolor{red}{गत्यर्थाकर्मक\-श्लिष\-शीङ्स्थास\-वस\-जन\-रुह\-जीर्यतिभ्यश्च} (पा॰सू॰~३.४.७२) इत्यनेन कर्तरि क्तः।} श्रीराम एव स्वयं विलपित इव। वस्तुतस्तु तस्य हृदये भक्ति\-भावेन सदाऽपि श्रीरामः सन्निहित एव। केवलं भावुकेभ्यो भरतः प्रेम\-दिग्दर्शनं कारयति।\end{sloppypar}
\section[दूरे स्थाप्य]{दूरे स्थाप्य}
\centering\textcolor{blue}{तीर्त्वा गङ्गां ययौ शीघ्रं भरद्वाजाश्रमं प्रति।\nopagebreak\\
दूरे स्थाप्य महासैन्यं भरतः सानुजो ययौ॥}\nopagebreak\\
\raggedleft{–~अ॰रा॰~२.८.४१}\\
\begin{sloppypar}\hyphenrules{nohyphenation}\justifying\noindent\hspace{10mm} अत्र समस्त\-पदं विनाऽपि \textcolor{red}{स्थाप्य} इति ल्यबन्त\-प्रयोगः कथमिति चेत्। अत्र मयूर\-व्यंसकादित्वात्समासः।\footnote{\textcolor{red}{मयूरव्यंसकादयश्च} (पा॰सू॰~२.१.७२) इत्यनेन।} तथा च \textcolor{red}{हलदन्तात्सप्तम्या सञ्ज्ञायाम्} (पा॰सू॰~६.३.९) इत्यनेन सप्तम्या अलुक्ततो ल्यप्प्रत्यये\footnote{\textcolor{red}{समासेऽनञ्पूर्वे क्त्वो ल्यप्‌} (पा॰सू॰~७.१.३७) इत्यनेन क्त्वो ल्यबादेश इति भावः।} \textcolor{red}{दूरेस्थाप्य} इति। यद्वा \textcolor{red}{दूरे} इति सप्तमी\-प्रतिरूपकाव्ययं ततश्च सुविभक्तेरेव \textcolor{red}{अव्ययादाप्सुपः} (पा॰सू॰~२.४.८२) इत्यनेन लुक्। इत्थं \textcolor{red}{दूरेस्थाप्य} इति साधु।\footnote{यद्वाऽत्र \textcolor{red}{संस्थाप्य} इति प्रयोगः। \textcolor{red}{विनाऽपि प्रत्ययं पूर्वोत्तर\-पद\-लोपो वक्तव्यः} (वा॰~५.३.८३) इत्यनेन \textcolor{red}{सम्} उपसर्गस्य लोपः।}\end{sloppypar}
\vspace{2mm}
\centering ॥ इत्ययोध्याकाण्डीयप्रयोगाणां विमर्शः ॥\nopagebreak\\
\vspace{4mm}
\centering इत्यध्यात्म\-रामायणेऽपाणिनीय\-प्रयोगाणां\-विमर्श\-नामके शोध\-प्रबन्धे द्वितीयाध्याये प्रथम\-परिच्छेदः।\\
\pagebreak
\pdfbookmark[1]{द्वितीयः परिच्छेदः}{Chap2Part2}
\phantomsection
\addtocontents{toc}{\protect\setcounter{tocdepth}{1}}
\addcontentsline{toc}{section}{द्वितीयः परिच्छेदः}
\addtocontents{toc}{\protect\setcounter{tocdepth}{0}}
\centering ॥ अथ द्वितीयाध्याये द्वितीयः परिच्छेदः ॥\nopagebreak\\
\vspace{4mm}
\pdfbookmark[2]{अरण्यकाण्डम्}{Chap2Part2Kanda3}
\phantomsection
\addtocontents{toc}{\protect\setcounter{tocdepth}{2}}
\addcontentsline{toc}{subsection}{अरण्यकाण्डीयप्रयोगाणां विमर्शः}
\addtocontents{toc}{\protect\setcounter{tocdepth}{0}}
\centering ॥ अथारण्यकाण्डीयप्रयोगाणां विमर्शः ॥\nopagebreak\\
\section[यातः]{यातः}
\centering\textcolor{blue}{को वा दयालुः स्मृतकामधेनुरन्यो जगत्यां रघुनायकादहो।\nopagebreak\\
स्मृतो मया नित्यमनन्यभाजा ज्ञात्वा स्मृतिं मे स्वयमेव यातः॥}\nopagebreak\\
\raggedleft{–~अ॰रा॰~३.२.८}\\
\begin{sloppypar}\hyphenrules{nohyphenation}\justifying\noindent\hspace{10mm} अत्र \textcolor{red}{आयातः} इत्यर्थे \textcolor{red}{यातः} इति प्रयुक्तम्। यतो हि \textcolor{red}{आङ्‌}\-उपसर्गस्य लोपे सति\footnote{\textcolor{red}{विनाऽपि प्रत्ययं पूर्वोत्तर\-पद\-लोपो वक्तव्यः} (वा॰~५.३.८३) इत्यनेन।} \textcolor{red}{यः शिष्यते स लुप्यमानार्थाभिधायी}\footnote{मूलं मृग्यम्।} इत्यनेन तादृशार्थस्य बोधकत्वात्।\end{sloppypar}
\section[अरूपिणः]{अरूपिणः}
\centering\textcolor{blue}{पश्यामि राम तव रूपमरूपिणोऽपि\nopagebreak\\
मायाविडम्बनकृतं सुमनुष्यवेषम्।\nopagebreak\\
कन्दर्पकोटिसुभगं कमनीयचाप-\nopagebreak\\
बाणं दयार्द्रहृदयं स्मितचारुवक्त्रम्॥}\nopagebreak\\
\raggedleft{–~अ॰रा॰~३.२.३२}\\
\begin{sloppypar}\hyphenrules{nohyphenation}\justifying\noindent\hspace{10mm} अत्र \textcolor{red}{न रूपमरूपम्}।\footnote{\textcolor{red}{नञ्‌} (पा॰सू॰~२.२.६) इत्यनेन नञ्तत्पुरुष\-समासे \textcolor{red}{नलोपो नञः} (पा॰सू॰~६.३.७३) इत्यनेन नलोपे विभक्तिकार्ये सिद्धम्।} \textcolor{red}{अरूपमस्त्यस्येत्यरूपी तस्यारूपिणः}।\footnote{\textcolor{red}{अत इनिठनौ} (पा॰सू॰~५.२.११५) इत्यनेन \textcolor{red}{अरूप}\-प्रातिपदिकान्मत्वर्थीये \textcolor{red}{इनि}प्रत्यये विभक्तिकार्ये सिद्धम्।} अत्र \textcolor{red}{न कर्मधारयान्मत्वर्थीयो बहुव्रीहिश्चेत्तदर्थ\-प्रतिपत्ति\-करः}\footnote{मूलं मृग्यम्।} अयं नियमो जागरूकः। तस्मात् \textcolor{red}{नास्ति रूपं यस्य सोऽरूपस्तस्यारूपस्य} इति बहुव्रीहावेवार्थ\-बोधे \textcolor{red}{अरूपिणः} इति कथमिति चेत्। \textcolor{red}{अरूपिणः} इत्यत्र बहुव्रीहेरीप्सितोऽर्थो न लभ्यते। यतो ह्यत्र प्राशस्त्ये भूमत्वे नित्ययोगे चेनिः।\footnote{\textcolor{red}{भूम\-निन्दा\-प्रशंसासु नित्ययोगेऽति\-शायने। सम्बन्धेऽस्ति\-विवक्षायां भवन्ति मतुबादयः॥} (भा॰पा॰सू॰~५.२.९४)।} भगवतो रूपस्य भूयस्त्वं प्रशस्तत्वञ्च नित्य\-योगत्वमपि सिद्धमेव। तद्बहुव्रीहिणा कथमपि लब्धुमशक्यत्वात्।\footnote{न भूम\-प्रशस्त\-नित्ययुक्तं रूपमस्येत्यरूपी तस्यारूपिणोऽपि तद्वद्रूपं पश्यामीति भावः। अस्मिन्नेव पद्ये \textcolor{red}{सुमनुष्य\-वेषं कन्दर्प\-कोटि\-सुभगं कमनीय\-चाप\-बाणं स्मित\-चारुवक्त्रम्} इति प्राशस्त्यमुक्तम्। अन्यच्चास्मिन् ग्रन्थे \textcolor{red}{रामश्चापधरो नित्यं तूणीबाणान्वितः प्रभुः} (अ॰रा॰~१.३.६२) इत्यत्र चाप\-तूणी\-बाण\-सहित\-रूपस्य नित्ययोग उक्तः। अहल्या\-स्तुतौ च \textcolor{red}{कार्य\-कारण\-कर्तृत्व\-फल\-साधन\-भेदतः। एको विभासि राम त्वं मायया बहुरूपया॥} (अ॰रा॰~१.५.५४) इत्यत्र श्रीरामभद्रस्य रूपभूमत्वं चोक्तम्।} अथवा \textcolor{red}{अरूपयितुं तच्छील इत्यरूपी तस्यारूपिणः} इति ताच्छील्ये णिनिः। \textcolor{red}{सुप्यजातौ णिनिस्ताच्छील्ये} (पा॰सू॰~३.२.७८) इति सूत्रेण।\end{sloppypar}
\section[मदुपासनात्]{मदुपासनात्}
\centering\textcolor{blue}{इत्येवं स्तुवस्तस्य रामः सुस्मितमब्रवीत्।\nopagebreak\\
मुने जानामि ते चित्तं निर्मलं मदुपासनात्॥}\nopagebreak\\
\raggedleft{–~अ॰रा॰~३.२.३५}\\
\begin{sloppypar}\hyphenrules{nohyphenation}\justifying\noindent\hspace{10mm} अत्र \textcolor{red}{ण्यास\-श्रन्थो युच्} (पा॰सू॰~३.३.१०७) इत्यनेन युच्कथं न। \textcolor{red}{मदुपासनात्} इत्यपाणिनीय इव।\footnote{कर्तृभिन्न\-कारके युचि तु \textcolor{red}{उपासना} इति रूपम्। तस्मात् \textcolor{red}{मदुपासनायाः} इति पञ्चम्यन्त\-रूपम्। यथा योगसूत्रे भोज\-वृत्तौ \textcolor{red}{उपासनायाः फलमाह} (यो॰सू॰ भो॰वृ॰~१.२८) इति षष्ठ्यन्त\-रूपम्।} अत्र \textcolor{red}{ल्युट् च} (पा॰सू॰~३.३.११५) इत्यनेन भावे नपुंसके \textcolor{red}{ल्युट्}।\footnote{एतेन \textcolor{red}{उपासनम्} इत्यपि पाणिनीयम्। \textcolor{red}{वरिवस्या तु शुश्रूषा परिचर्याऽप्युपासनम्} (अ॰पु॰~३६५.७) इत्यग्नि\-पुराण\-प्रयोगादपि। अमरकोषे तु \textcolor{red}{वरिवस्या तु शुश्रूषा परिचर्याऽप्युपासना} (अ॰को॰~२.६.३५) इति पाठः।} लकार\-टकारयोरनुबन्ध\-कार्ये \textcolor{red}{युवोरनाकौ} (पा॰सू॰~७.१.१) इत्यनेनानादेशे पञ्चम्येक\-वचने \textcolor{red}{मदुपासनात्}।\footnote{यथा तत्रैव योगसूत्रे भोज\-वृत्तौ \textcolor{red}{उपासनमाह} (यो॰सू॰ भो॰वृ॰~१.२८) इति द्वितीयान्त\-रूपम्।}\end{sloppypar}
\section[ध्यायमानः काङ्क्षमाणः]{ध्यायमानः काङ्क्षमाणः}
\centering\textcolor{blue}{शीघ्रमानय भद्रं ते रामं मम हृदि स्थितम्।\nopagebreak\\
तमेव ध्यायमानोऽहं काङ्क्षमाणोऽत्र संस्थितः॥}\nopagebreak\\
\raggedleft{–~अ॰रा॰~३.३.१०}\\
\begin{sloppypar}\hyphenrules{nohyphenation}\justifying\noindent\hspace{10mm} प्रयोगाविमावध्यात्म\-रामायणेऽरण्य\-काण्डस्य तृतीय\-सर्गीयौ। श्रीरामभद्रः सीता\-लक्ष्मण\-समेतः पावित\-मुनि\-जन\-निकेतोऽगस्त्यस्याश्रममुप\-तिष्ठमानः\footnote{\textcolor{red}{उपतिष्ठमानः} इत्यत्र \textcolor{red}{उपाद्देवपूजा\-सङ्गतकरण\-मित्रकरण\-पथिष्विति} (वा॰~१.३.२४) इत्यनेन सङ्गतकरणे पथि वाऽऽत्मनेपदम्। यद्वा \textcolor{red}{वा लिप्सायाम्} (वा॰~१.३.२४) इत्यनेन।} सुतीक्ष्णेन सूचयति। रामागमन\-श्रवण\-सञ्जात\-हर्षोऽगस्त्यः प्रोवाच यत् \textcolor{red}{राममानयाहं तमेव ध्यायमानस्तिष्ठामि}। \textcolor{red}{ध्यै}\-धातुः (\textcolor{red}{ध्यै चिन्तायाम्} धा॰पा॰~९०८) परस्मैपदी। तथा सति लटि तिपि शपि गुणेऽयादेशे \textcolor{red}{ध्यायति}। शतरि च \textcolor{red}{ध्यायतीति ध्यायन्} इत्येव प्रयोगः। \textcolor{red}{ध्यायमानः} इति कथम्। अत्र \textcolor{red}{कर्तरि कर्म\-व्यतिहारे} (पा॰सू॰~१.३.१४) इत्यात्मनेपदम्। यतो हि ध्यानस्य बुद्धि\-विषयत्वात्तदेव ध्यानमहं\-पद\-वाच्य\-प्रत्यग्भिन्न\-चैतन्य आत्मन्यारोप्यतेऽतः कर्म\-व्यतिहारतयाऽऽत्मनेपदम्। ततश्च \textcolor{red}{ध्यायत इति ध्यायमानः} इति विग्रहे \textcolor{red}{ध्यै}\-धातोर्लटि\footnote{\textcolor{red}{वर्तमाने लट्} (पा॰सू॰~३.२.१२३) इत्यनेन।} तस्य शानजादेशे\footnote{\textcolor{red}{लटः शतृशानचावप्रथमा\-समानाधिकरणे} (पा॰सू॰~३.२.१२४) इत्यनेन।} शपि \textcolor{red}{आने मुक्} (पा॰सू॰~७.२.८२) इत्यनेन मुगागमे विभक्ति\-कार्ये \textcolor{red}{ध्यायमानः}। अथवा \textcolor{red}{ध्यायम् ध्यायम्} इत्याभीक्ष्ण्ये णमुल्।\footnote{\textcolor{red}{आभीक्ष्ण्ये णमुल् च} (पा॰सू॰~३.४.२२) इत्यनेन।} प्रथमस्य \textcolor{red}{ध्यायम्} इत्यस्य लोपे \textcolor{red}{आसमन्तादनितीत्यानः}।\footnote{\textcolor{red}{आङ्‌}\-उपसर्ग\-पूर्वकात् \textcolor{red}{अनँ प्राणने} (धा॰पा॰~१०७०) इति धातोः \textcolor{red}{नन्दि\-ग्रह\-पचादिभ्यो ल्युणिन्यचः} (पा॰सू॰~३.१.१३४) इत्यनेन पचाद्यचि।} ततो वर्णसम्मेलने \textcolor{red}{ध्यायमानः}। एवमेव \textcolor{red}{काङ्क्षमाणः} इत्यत्रापि \textcolor{red}{काङ्क्षति} इति परस्मैपदी। तत्रापि कर्म\-व्यतिहार आत्मने\-पदं ततः शानच्।\footnote{काङ्क्षा तु सांसारिकाणां जनानां कृते। अगस्त्यश्चर्षिः। दिव्यदृष्ट्या भगवद्रूप\-द्रष्टर्षिर्भूत्वाऽपि लौकिक\-नेत्र\-लाभार्थं भगवद्दर्शन\-काङ्क्षामाचरतीति कर्म\-व्यतिहार\-विवक्षायामात्मने\-पदम्। एवमेव \textcolor{red}{न काङ्क्षे विजयं कृष्ण न च राज्यं सुखानि च} (भ॰गी॰~१.३२) इत्यत्र। विजय\-राज्य\-सुखानामकाङ्क्षा न राजोचिता।}\end{sloppypar}
\section[प्रतीक्षन्]{प्रतीक्षन्}
\centering\textcolor{blue}{त्वदागमनमेवाहं प्रतीक्षन्समवस्थितः।\nopagebreak\\
यदा क्षीरसमुद्रान्ते ब्रह्मणा प्रार्थितः पुरा॥}\nopagebreak\\
\raggedleft{–~अ॰रा॰~३.३.१८}\\
\begin{sloppypar}\hyphenrules{nohyphenation}\justifying\noindent\hspace{10mm} सीता\-लक्ष्मण\-सेव्यमानो निर्मानो भगवान् श्रीरामोऽगस्त्य\-दर्शनार्थमाश्रम\-द्वार्युपस्थितः। तदागमन\-श्रवण\-सञ्जात\-कुतूहलो विह्वलोऽगस्त्यः प्रणिजगाद यत् \textcolor{red}{अहमपि भगवत आगमनं प्रतीक्षे}। अत्र \textcolor{red}{प्रति}\-पूर्वकः \textcolor{red}{ईक्ष्‌}\-धातुः (\textcolor{red}{ईक्षँ दर्शने} धा॰पा॰~६१०) आत्मनेपदी दर्शनार्थः। ततश्च \textcolor{red}{प्रतीक्षते} इति विग्रहे शानच्प्रत्यये कृते\footnote{\textcolor{red}{लटः शतृशानचावप्रथमा\-समानाधिकरणे} (पा॰सू॰~३.२.१२४) इत्यनेन।} मुगागमे\footnote{\textcolor{red}{आने मुक्} (पा॰सू॰~७.२.८२) इत्यनेन।} विभक्ति\-कार्ये \textcolor{red}{प्रतीक्षमाणः} इति पाणिनीयः। आत्मनेपदीयत्वाद्दुर्गमः \textcolor{red}{शतृ}\-प्रत्ययः। एवं \textcolor{red}{प्रतीक्षन्} इति कथमिति चेत्। उच्यते। \textcolor{red}{अनुदात्तेत्त्व\-लक्षणमात्मने\-पदमनित्यम्} (प॰शे॰~९३.४) इति नियमेन परस्मैपदीयत्वाच्छतृ\-प्रत्यये \textcolor{red}{प्रतीक्षन्}। यद्वा \textcolor{red}{प्रतीक्षणं प्रतीक्षा} इति विग्रहे भावे \textcolor{red}{आङ्} औणादिक\-प्रत्यये\footnote{\textcolor{red}{गुरोश्च हलः} (पा॰सू॰~३.३.१०३) इत्यनेन स्त्रियां भावे \textcolor{red}{अ}\-प्रत्यये \textcolor{red}{अजाद्यतष्टाप्‌} (पा॰सू॰~४.१.४) इत्यनेन टापि वा। \pageref{sec:aveksati}तमे पृष्ठे \ref{sec:aveksati} \nameref{sec:aveksati} इति प्रयोगस्य विमर्शं पश्यन्तु।} \textcolor{red}{प्रतीक्षां करोतीति प्रतीक्षयति}।\footnote{प्रतीक्षा~\arrow \textcolor{red}{तत्करोति तदाचष्टे} (धा॰पा॰ ग॰सू॰~१८७)~\arrow प्रतीक्षा~णिच्~\arrow प्रतीक्षा~इ~\arrow \textcolor{red}{णाविष्ठवत्प्राति\-पदिकस्य पुंवद्भाव\-रभाव\-टिलोप\-यणादि\-परार्थम्} (वा॰~६.४.४८)~\arrow प्रतीक्ष्~इ~\arrow प्रतीक्षि~\arrow \textcolor{red}{सनाद्यन्ता धातवः} (पा॰सू॰~३.१.३२)~\arrow धातुसञ्ज्ञा~\arrow \textcolor{red}{शेषात्कर्तरि परस्मैपदम्} (पा॰सू॰~१.३.७८)~\arrow \textcolor{red}{वर्तमाने लट्} (पा॰सू॰~३.२.१२३)~\arrow प्रतीक्षि~लट्~\arrow प्रतीक्षि~तिप्~\arrow \textcolor{red}{कर्तरि शप्‌} (पा॰सू॰~३.१.६८)~\arrow प्रतीक्षि~शप्~तिप्~\arrow प्रतीक्षि~अ~ति~\arrow \textcolor{red}{सार्वधातुकार्ध\-धातुकयोः} (पा॰सू॰~७.३.८४)~\arrow गुणः~\arrow प्रतीक्षे~अ~ति~\arrow \textcolor{red}{एचोऽयवायावः} (पा॰सू॰~६.१.७८)~\arrow प्रतीक्षय्~अ~ति~\arrow प्रतीक्षयति।} \textcolor{red}{प्रतीक्षयतीति प्रतीक्षः}।\footnote{प्रतीक्षि~\arrow पूर्ववद्धातु\-सञ्ज्ञा~\arrow \textcolor{red}{नन्दि\-ग्रहि\-पचादिभ्यो ल्युणिन्यचः} (पा॰सू॰~३.१.१३४)~\arrow प्रतीक्षि~अच्~\arrow प्रतीक्षि~अ~\arrow \textcolor{red}{णेरनिटि} (पा॰सू॰~६.४.५१)~\arrow प्रतीक्ष्~अ~\arrow प्रतीक्ष~\arrow विभक्तिकार्यम्~\arrow प्रतीक्षः।} \textcolor{red}{प्रतीक्ष इवाऽचरन्निति प्रतीक्षन्}।\footnote{प्रतीक्ष~\arrow \textcolor{red}{सर्वप्राति\-पदिकेभ्य आचारे क्विब्वा वक्तव्यः} (वा॰~३.१.११)~\arrow प्रतीक्ष~क्विँप्~\arrow प्रतीक्ष~व्~\arrow \textcolor{red}{वेरपृक्तस्य} (पा॰सू॰~६.१.६७)~\arrow प्रतीक्ष~\arrow \textcolor{red}{सनाद्यन्ता धातवः} (पा॰सू॰~३.१.३२)~\arrow धातुसञ्ज्ञा~\arrow \textcolor{red}{शेषात्कर्तरि परस्मैपदम्} (पा॰सू॰~१.३.७८)~\arrow \textcolor{red}{वर्तमाने लट्} (पा॰सू॰~३.२.१२३)~\arrow प्रतीक्ष~लट्~\arrow \textcolor{red}{लटः शतृ\-शानचावप्रथमा\-समानाधिकरणे} (पा॰सू॰~३.२.१२४)~\arrow प्रतीक्ष~शतृँ~\arrow प्रतीक्ष~अत्~\arrow \textcolor{red}{कर्तरि शप्} (पा॰सू॰~३.१.६८)~\arrow प्रतीक्ष~शप्~अत्~\arrow प्रतीक्ष~अ~अत्~\arrow \textcolor{red}{अतो गुणे} (पा॰सू॰~६.१.९७)~\arrow प्रतीक्ष~अत्~\arrow \textcolor{red}{अतो गुणे} (पा॰सू॰~६.१.९७)~\arrow प्रतीक्षत्~\arrow \textcolor{red}{कृत्तद्धित\-समासाश्च} (पा॰सू॰~१.२.४६)~\arrow प्रातिपदिक\-सञ्ज्ञा~\arrow विभक्ति\-कार्यम्~\arrow प्रतीक्षत्~सुँ~\arrow \textcolor{red}{उगिदचां सर्वनामस्थानेऽधातोः} (पा॰सू॰~७.१.७०)~\arrow \textcolor{red}{मिदचोऽन्त्यात्परः} (पा॰सू॰~१.१.४७)~\arrow प्रतीक्ष~नुँम्~त्~सुँ~\arrow प्रतीक्ष~न्~त्~सुँ~\arrow \textcolor{red}{हल्ङ्याब्भ्यो दीर्घात्सुतिस्यपृक्तं हल्} (पा॰सू॰~६.१.६८)~\arrow प्रतीक्ष~न्~त्~\arrow \textcolor{red}{संयोगान्तस्य लोपः} (पा॰सू॰~८.२.२३)~\arrow प्रतीक्षन्।} एवं आचक्षाण\-णिजन्तान्त\-कर्तर्यच्। तत आचारे क्विप्प्रत्यये लटि शतरि शपि पररूपे विभक्तिकार्ये \textcolor{red}{प्रतीक्षन्}।\end{sloppypar}
\section[समीपतः]{समीपतः}
\centering\textcolor{blue}{एकदा गौतमीतीरे पञ्चवट्यां समीपतः।\nopagebreak\\
पद्मवज्राङ्कुशाङ्गानि पदानि जगतीपतेः॥}\nopagebreak\\
\raggedleft{–~अ॰रा॰~३.५.२}\\
\begin{sloppypar}\hyphenrules{nohyphenation}\justifying\noindent\hspace{10mm} अत्र विश्लेषाभावात् \textcolor{red}{ध्रुवमपायेऽपादानम्} (पा॰सू॰~१.४.२४) इत्यनेनापादान\-सञ्ज्ञा तु नैव विचारसहा। तदभावे \textcolor{red}{पञ्चम्यास्तसिल्} (पा॰सू॰~५.३.७) इत्यनेन तसिल्प्रत्ययोऽपि नोपयुक्तः। तथा च सामीप्य\-वाचक\-शब्दात्सप्तम्यर्थ एव \textcolor{red}{तसि\-प्रकरण आद्यादिभ्य उपसङ्ख्यानम्} (वा॰~५.४.४४) इत्यनेन \textcolor{red}{तसि}\-प्रत्यये सिद्धमिदम्।\footnote{तसेः सार्व\-विभक्तिकत्वं तदन्तानामाकृति\-गणत्वं च \pageref{fn:yatah}तमे पृष्ठे \ref{fn:yatah}तम्यां पादटिप्पण्यां स्पष्टीकृतम्।
}
 \textcolor{red}{समीप एव समीपतः}। न चात्र समीप\-वाचक\-शब्दे कथं सप्तमीति चेत्। \textcolor{red}{सप्तम्यधिकरणे च} (पा॰सू॰~२.३.३६) इत्यत्र \textcolor{red}{च}\-कार\-ग्रहणेन दूरान्तिकार्थ\-वाचक\-शब्देभ्यः सप्तमी।\footnote{\textcolor{red}{चकाराद्दूरान्तिकार्थेभ्यश्च ... दूरान्तिकार्थेभ्यः खल्वपि। दूरे ग्रामस्य। अन्तिके ग्रामस्य। अभ्याशे ग्रामस्य} (का॰वृ॰~२.३.३६)। \textcolor{red}{चकाराद्दूरान्तिकार्थेभ्यः} (वै॰सि॰कौ॰~६३३)।} समीपं ह्यन्तिकार्थ\-वाचकं ततः सप्तमी।\end{sloppypar}
\section[कनीयान्]{कनीयान्}
\centering\textcolor{blue}{एषा मे सुन्दरी भार्या सीता जनकनन्दिनी।\nopagebreak\\
स तु भ्राता कनीयान्मे लक्ष्मणोऽतीवसुन्दरः॥}\nopagebreak\\
\raggedleft{–~अ॰रा॰~३.५.९}\\
\begin{sloppypar}\hyphenrules{nohyphenation}\justifying\noindent\hspace{10mm} अत्र श्रीरामो लक्ष्मणस्य परिचयं कारयति यत् \textcolor{red}{स मे कनीयान् भ्राता}। \textcolor{red}{ईयसुन्} प्रत्ययो हि यत्र द्वयोर्विभागः।\footnote{\textcolor{red}{द्विवचनविभज्योपपदे तरबीयसुनौ} (पा॰सू॰~५.३.५७) इत्यनेन।} अत्र रामापेक्षया कनीयान् भरतः। कथं लक्ष्मणं कनीयांसं कथयतीति चेत्। अत्र राम\-लक्ष्मणयोर्द्वयोरेव विवक्षितत्वात्। भरत\-शत्रुघ्नावयोध्यायामरण्ये राम\-लक्षणौ। अनयोर्द्वन्द्वः शाश्वतः शेष\-शेषि\-भावात्। लीलायामपि राम\-लक्ष्मणयोः संयोगः सर्वत्र प्रसिद्धः। अतः शब्दार्थ\-सन्देहे विशेष\-स्मृति\-हेतूनां परिगणनावसरे साहचर्य उदाहरणं \textcolor{red}{राम\-लक्ष्मणौ}।\footnote{\textcolor{red}{संसर्गो विप्रयोगश्च साहचर्यं विरोधिता। अर्थः प्रकरणं लिङ्गं शब्दस्यान्यस्य सन्निधिः॥ सामर्थ्यमौचिती देशः कालो व्यक्तिः स्वरादयः। शब्दार्थस्यानवच्छेदे विशेषस्मृतिहेतवः॥} (वा॰प॰~२.३१५-३१६)।} यद्यपि \textcolor{red}{लक्ष्मण}\-शब्दस्य द्वावर्थौ लक्ष्मणो दाशरथि\-लक्ष्मणो दुर्योधन\-पुत्रश्च किन्तु राम\-साहचर्यादत्र दाशरथिर्लक्ष्मण एवार्थ\-बोध\-विषयो भवति। एवं \textcolor{red}{राम}\-शब्दस्य त्रयोऽर्था रामो जामदग्न्यो रामो दाशरथी रामो वासुदेवश्चेति। अत्रोच्चरिते राम\-शब्द उपस्थितेषु त्रिष्वर्थेषु को रामो गृह्यतामिति चेत् \textcolor{red}{राम\-लक्ष्मणौ} इति कथनेन दशरथ\-पुत्रस्य ग्रहणं भवति। अध्यात्म\-रामायणेऽपि~– \end{sloppypar}
\centering\textcolor{blue}{लक्ष्मणो रामचन्द्रेण शत्रुघ्नो भरतेन च।\nopagebreak\\
द्वन्द्वीभूय चरन्तौ तौ पायसांशानुसारतः॥}\nopagebreak\\
\raggedleft{–~अ॰रा॰~१.३.४२}\\
\begin{sloppypar}\hyphenrules{nohyphenation}\justifying\noindent\hspace{10mm} यद्वा \textcolor{red}{मे} इत्युच्चारण आत्मानं भ्रातृ\-समुदायात्पृथक्करोति। ज्येष्ठस्य भ्रातुः पितृवद्दायित्वं लोक\-प्रसिद्धम्।\footnote{\textcolor{red}{पितेव पालयेत्पुत्रान् ज्येष्ठो भ्रातॄन् यवीयसः। पुत्रवच्चापि वर्तेरन् ज्येष्ठे भ्रातरि धर्मतः॥} (म॰स्मृ॰~९.१०८)।} अतस्ते सन्ति त्रयो भ्रातरः। षष्ठी पालक\-पाल्य\-भाव\-सम्बन्धे। अहं त्रयाणां भ्रातॄणां पालको ज्येष्ठत्वात्। उपरते पितरि मे मय्येव सर्वेषामुत्तर\-दायित्वम्। वाल्मीकीयेऽपि विभीषणं शरणागतं स्वीकुर्वञ्छ्रीरामभद्रो रावण\-हननाय त्रयाणां भ्रातॄणां शपथं करोति। यथा~–\end{sloppypar}
\centering\textcolor{red}{अहत्वा रावणं सङ्ख्ये सपुत्रबलवाहनम्।\nopagebreak\\
अयोध्यां न प्रवेक्ष्यामि भ्रातृभिश्च त्रिभिः शपे॥}\nopagebreak\\
\raggedleft{–~वा॰रा॰~६.४१.७}\\
\begin{sloppypar}\hyphenrules{nohyphenation}\justifying\noindent इत्थं मम त्रिषु भ्रातृषु ज्येष्ठो भ्राता भरतोऽयोध्यामधितिष्ठति ततो लक्ष्मणः कनीयानिति श्रीरामचन्द्रस्य तात्पर्यं प्रतिभाति। अस्मादेव कारणात् \textcolor{red}{स तु भ्राता कनीयान्मे} इत्यत्र षष्ठी। अन्यथा \textcolor{red}{पञ्चमी विभक्ते} (पा॰सू॰~२.३.४२) इति सूत्रेण विभज्यमाने पञ्चमी स्यात्। तथा च \textcolor{red}{द्विवचनविभज्योपपदे तरबीयसुनौ} (पा॰सू॰~५.३.५७) इत्यनेन द्विवचन\-विभज्यमान\-वाचक उपपदे \textcolor{red}{तरप्} \textcolor{red}{ईयसुन्} च प्रत्ययो भवति। तत्र च पञ्चम्यनिवार्या। यथा \textcolor{red}{रामाच्छ्यामो लघुतरः}। \textcolor{red}{रामाद्भरतः कनीयान्}। अतः षष्ठीं दृष्ट्वाऽत्र रामस्य समुदायात्पृथग्भूतत्वं प्रतीयते। यद्वा
\textcolor{red}{भ्रियन्ते पुष्यन्त इति भ्रातरः}।\footnote{\textcolor{red}{भ्रातृ}\-शब्दो \textcolor{red}{भ्रातृपुत्रौ स्वसृ\-दुहितृभ्याम्} (पा॰सू॰~१.२.६८) इति सूत्रकार\-प्रयोगादेव सिद्धः। \textcolor{red}{नप्तृ\-नेष्टृ\-त्वष्टृ\-होतृ\-पोतृ\-भ्रातृ\-जामातृ\-मातृ\-पितृ\-दुहितृ} (प॰उ॰~२.९५) इत्युणादि\-सूत्र\-पाठभेदेन तृनन्तस्तृजन्तो वा निपात्यते। अन्यत्र \textcolor{red}{नप्तृ\-नेष्टृ\-त्वष्टृ\-क्षत्तृ\-होतृ\-पोतृ\- जामातृ\-पितृ\-दुहितृ} (प॰उ॰~२.९५) इति \textcolor{red}{भ्रातृ}\-शब्द\-रहितः पाठः। कोशेषूणादि\-सूत्र\-टीकासु विविधा व्युत्पत्तयः। \textcolor{red}{भ्राज-तृच् पृषो॰} इति वाचस्पत्यम्। \textcolor{red}{भ्राजते इति भ्राज + नप्तृ\-नेष्टृ\-त्वष्टृ\-होत्रिति प॰उ॰~२.९६ इति तृन्। निपात्यते च} इति शब्द\-कल्प\-द्रुमः। \textcolor{red}{भ्राजते। भ्राजृँ दीप्तौ (भ्वा॰आ॰से॰)। ‘नप्तृ\-नेष्टृ-’ इति साधु} (अ॰को॰ व्या॰सु॰~२.६.३६) इति व्याख्या\-सुधा। \textcolor{red}{भ्राजते दीप्यतेऽसौ भ्राता सोदर्यो वा} (उ॰को॰~२.९५) इति दयानन्द\-सरस्वती। \textcolor{red}{भ्राजृँ दीप्तौ} (पा॰सू॰~१८१) इत्यस्मात् \textcolor{red}{टुभ्राजृँ दीप्तौ} (धा॰पा॰~८३३) इत्यस्माद्वा तृनि तृचि वा पृषोदरादित्वाज्जकार\-लोप इत्येषां भावः। \textcolor{red}{‘भज सेवायाम्’ इति भ्राता} (प॰उ॰~श्वे॰वृ॰~२.९५) इति श्वेत\-वनवासि\-वृत्ति\-पाठभेदः। \textcolor{red}{भजँ सेवायाम्} (धा॰पा॰~९९८) इत्यस्मात्तृनि तृचि वा पृषोदरादित्वाज्जकार\-लोपो भ्रादेश्चेति भावः। अत्र प्रणेतारो \textcolor{red}{भ्रियन्ते पुष्यन्त इति भ्रातरः} इति। \textcolor{red}{भृञ् भरणे} (धा॰पा॰~८९८) इत्यस्माद्धातोः कर्मणि तृनि तृचि वा गुणे रपरत्वे पृषोदरादित्वात् \textcolor{red}{भर्} इत्यस्य \textcolor{red}{भ्रा} इत्यादेशे \textcolor{red}{भ्रातृ} इति भावः। न च कथं कर्मणि तृन्तृजौ \textcolor{red}{कर्तरि कृत्‌} (पा॰सू॰~३.४.६७) इत्यनेन कृतां कर्तर्येव विधानात्। \textcolor{red}{ताभ्यामन्यत्रोणादयः} (पा॰सू॰~३.४.७५) इत्यनेन कर्मण्यपीति दिक्। यद्वा \textcolor{red}{उणादयो बहुलम्} (धा॰पा॰~८९८) इत्यत्र \textcolor{red}{बहुल}\-ग्रहणादुणादयः क्वचित्कर्मण्यपि।}
तथा च रामः खलु निर्गुणो महा\-विष्णू राम एव पर\-ब्रह्मेत्युत्तर\-तापनीय\-श्रुतेः।\footnote{\textcolor{red}{ॐ यो वै श्रीरामचन्द्रः स भगवान् य ॐ नमो भगवते वासुदेवाय महाविष्णुर्भूर्भुवःस्वस्तस्मै वै नमो नमः} (रा॰उ॰ता॰उ॰~५.४४)।} एवं तस्य महा\-विष्णोर्भगवतः परात्परब्रह्मणः श्रीरामचन्द्रस्य त्रयोंऽशा ब्रह्म\-विष्णु\-महेशाख्याः।
ब्रह्मा खलु शत्रुघ्नो विष्णुर्भरतो लक्ष्मणः शिवः। कर्पूर\-गौरत्वाच्छिवस्य लक्ष्मणो गौरो विष्णोश्च श्यामतया भरतः श्यामो विष्णुत्वेन जनकत्वाद्ब्रह्मावतारः शत्रुघ्नो भरतं जनकमिवान्वञ्चति। अतस्तेषु पोष्यमाणेषु भ्रातृ\-रूपेषु त्रिष्वंशेषु लक्ष्मणः कनीयान्। यद्वा~–\end{sloppypar}
\centering\textcolor{red}{स्थूलं चाष्टभुजं प्रोक्तं सूक्ष्मं चैव चतुर्भुजम्।\nopagebreak\\
परं च द्विभुजं रूपं तस्मादेतत्त्रयं यजेत्॥}\nopagebreak\\
\raggedleft{–~आ॰सं॰}\footnote{मूलं मृग्यम्। शिवसहाय\-कृताया रामायण\-शिरोमण्याख्यायाष्टीकाया मङ्गलाचरणे एष श्लोक उद्धृतः। तत्र \textcolor{red}{इत्यानन्द\-संहिता\-वचनं च} इत्युक्तम्।}\\
\begin{sloppypar}\hyphenrules{nohyphenation}\justifying\noindent
द्वाभ्यां भुजाभ्यां भक्तस्य योगं क्षेमं च वहत्यथवा द्वाभ्यां भुजाभ्यां ज्ञान\-प्रधानांश्च भागवतान् भुनक्ति। त्रिपुर\-सुन्दरी\-तन्त्रे तस्य भगवतो महाविष्णोः श्रीरामभद्रस्येमे त्रयो विष्णव एवांशाः। तत्र क्षीरशायी भरतो वैकुण्ठाधीशो लक्ष्मणः।\footnote{मूलं वैष्णवागम\-ग्रन्थेषु मृग्यम्।} वैकुण्ठाधीशस्य शुक्ल\-वर्णता पुराणे प्रसिद्धा यथा~–\end{sloppypar}
\centering\textcolor{red}{केनोपयान चैतेषां दुःखनाशो भवेद्ध्रुवम्।\nopagebreak\\
इति सञ्चिन्त्य मनसा विष्णुलोकं गतस्तदा॥\nopagebreak\\
तत्र नारायणं देवं शुक्लवर्णं चतुर्भुजम्।\nopagebreak\\
शङ्खचक्रगदापद्मवनमालाविभूषितम्॥}\nopagebreak\\
\raggedleft{–~स्क॰पु॰~रे॰ख॰~२३३.५,६}\\
\begin{sloppypar}\hyphenrules{nohyphenation}\justifying\noindent एवमेव लक्ष्मणस्याऽपि गौराङ्गता स्पष्टा। क्षीराब्धि\-स्वामी श्यामलो यथा~–\end{sloppypar}
\centering\textcolor{red}{शान्ताकारं भुजगशयनं पद्मनाभं सुरेशं\nopagebreak\\
विश्वाधारं गगनसदृशं मेघवर्णं शुभाङ्गम्।\nopagebreak\\
लक्ष्मीकान्तं कमलनयनं योगिभिर्ध्यानगम्यं\nopagebreak\\
वन्दे विष्णुं भवभयहरं सर्वलोकैकनाथम्॥}\nopagebreak\\
\begin{sloppypar}\hyphenrules{nohyphenation}\justifying\noindent तथा क्षीराब्धि\-स्वामी भरतो वैकुण्ठ\-विहारी विष्णुर्लक्ष्मणो भूमा शत्रुघ्नो रामः सनातनं ब्रह्म। तथा चोक्तं बृहद्ब्रह्मसंहितायां यत्~–\end{sloppypar}
\centering\textcolor{red}{क्षीराब्धीशस्तु भरतो वैकुण्ठेशस्तु लक्ष्मणः।\nopagebreak\\
भूमा तु शत्रुघ्नो ज्ञेयो रामो ब्रह्म सनातनम्॥}\nopagebreak\\
\raggedleft{–~बृ॰ब्र॰सं॰}\\
\begin{sloppypar}\hyphenrules{nohyphenation}\justifying\noindent अतो मे त्रयो भ्रातरोंऽशाः। तत्र क्षीर\-शायि\-भरतापेक्षया वैकुण्ठेशो लक्ष्मणः कनीयानित्येव हार्दं हरेः।\end{sloppypar}
\section[क्रन्दमाना]{क्रन्दमाना}
\centering\textcolor{blue}{क्रन्दमाना पपाताग्रे खरस्य परुषाक्षरा।\nopagebreak\\
किमेतदिति तामाह खरः खरतराक्षरः॥}\nopagebreak\\
\raggedleft{–~अ॰रा॰~३.५.२१}\\
\begin{sloppypar}\hyphenrules{nohyphenation}\justifying\noindent\hspace{10mm} अत्र शूर्पणखा\-परिस्थितिं वर्णयति। \textcolor{red}{क्रन्दमाना} इति। आह्वाने रोदने च \textcolor{red}{क्रन्द्‌}\-धातुः (\textcolor{red}{क्रदिँ आह्वाने रोदने च} धा॰पा॰~७१) परस्मैपदी। ततश्चात्र शत्रा भवितव्यम्। एवं \textcolor{red}{क्रन्दतीति क्रन्दन्ती} इत्येव पाणिनीयम्।\footnote{यथा भागवते~– \textcolor{red}{गौर्भूत्वाऽश्रुमुखी खिन्ना क्रन्दन्ती करुणं विभोः} (भा॰पु॰~१०.१.१८)। अत्र \textcolor{red}{रुदन्ती} इत्यपि पाठः।} अत्र तु \textcolor{red}{क्रन्दमाना} इति।\footnote{एवमेव भारते शल्यपर्वणि~– \textcolor{red}{शोचन्त्यस्तत्र रुरुदुः क्रन्दमाना विशाम्पते} (म॰भा॰~९.२९.७०)।} 
\textcolor{red}{क्रन्दते} इति कर्म\-व्यतिहारादात्मनेपदम्।\footnote{\textcolor{red}{कर्तरि कर्म\-व्यतिहारे} (पा॰सू॰~१.३.१४) इत्यनेन।} \textcolor{red}{क्रन्दत इति क्रन्दमाना} इति शानचि शपि मुकि टापि च कृते सिद्धम्।\footnote{\textcolor{red}{क्रदिँ आह्वाने रोदने च} (धा॰पा॰~७१)~\arrow क्रद्~\arrow \textcolor{red}{इदितो नुम् धातोः} (पा॰सू॰~७.१.५८)~\arrow \textcolor{red}{मिदचोऽन्त्यात्परः} (पा॰सू॰~१.१.४७)~\arrow क्र~नुँम्~द्~\arrow क्र~न्~द्~\arrow \textcolor{red}{नश्चापदान्तस्य झलि} (पा॰सू॰~८.३.२४)~\arrow क्रंद्~\arrow \textcolor{red}{अनुस्वारस्य ययि परसवर्णः} (पा॰सू॰~८.४.५८)~\arrow क्रन्द्~\arrow \textcolor{red}{कर्तरि कर्म\-व्यतिहारे} (पा॰सू॰~१.३.१४)~\arrow \textcolor{red}{वर्तमाने लट्} (पा॰सू॰~३.२.१२३)~\arrow क्रन्द्~लट्~\arrow \textcolor{red}{लटः शतृशानचावप्रथमा\-समानाधिकरणे} (पा॰सू॰~३.२.१२४)~\arrow क्रन्द्~शानच्~\arrow क्रन्द्~आन~\arrow \textcolor{red}{कर्तरि शप्‌} (पा॰सू॰~३.१.६८)~\arrow क्रन्द्~शप्~आन~\arrow क्रन्द्~अ~आन~\arrow \textcolor{red}{आने मुक्} (पा॰सू॰~७.२.८२)~\arrow \textcolor{red}{आद्यन्तौ टकितौ} (पा॰सू॰~१.१.४६)~\arrow क्रन्द्~अ~मुँक्~आन~\arrow क्रन्द्~अ~म्~आन~\arrow क्रन्दमान~\arrow \textcolor{red}{अजाद्यतष्टाप्‌} (पा॰सू॰~४.१.४)~\arrow क्रन्दमान~टाप्~\arrow क्रन्दमान~आ~\arrow \textcolor{red}{अकः सवर्णे दीर्घः} (पा॰सू॰~६.१.१०१)~\arrow क्रन्दमाना।} यद्वाऽत्र वैक्लव्ये \textcolor{red}{क्रन्द्‌}\-धातुः (\textcolor{red}{क्रदिँ वैकल्ये} धा॰पा॰~७७३) आत्मनेपदी। ततः शानचि टापि \textcolor{red}{क्रन्दमाना}। यद्वेममाकृति\-गणत्वात्स्वरितेतं पठित्वा \textcolor{red}{स्वरितञितः कर्त्रभिप्राये क्रिया\-फले} (पा॰सू॰~१.३.७२) इत्यनेनाऽत्मनेपदत्वाच्छानचि शपि मुकि टापि \textcolor{red}{क्रन्दमाना}।\footnote{\textcolor{red}{बहुलमेतन्निदर्शनम्} (धा॰पा॰ ग॰सू॰~१९३८) \textcolor{red}{आकृतिगणोऽयम्} (धा॰पा॰ ग॰सू॰~१९९२) \textcolor{red}{भूवादिष्वेतदन्तेषु दशगणीषु धातूनां पाठो निदर्शनाय तेन स्तम्भुप्रभृतयः सौत्राश्चुलुम्पादयो वाक्यकारीयाः प्रयोगसिद्धा विक्लवत्यादयश्च} (मा॰धा॰वृ॰~१०.३२८) इत्यनुसारमाकृति\-गणत्वाद्भ्वादि\-गण ऊह्योऽयं स्वरितेद्धातुः। तेन \textcolor{red}{क्रन्दति} \textcolor{red}{क्रन्दते} इति सिध्यतः। \textcolor{red}{क्रन्दति} यथा वामनपुराणे~– \textcolor{red}{सिंहाभिपन्नो विपिने यथैव मत्तो गजः क्रन्दति वेदनार्तः} (वाम॰पु॰~१०.४७) गरुडपुराणे च~– \textcolor{red}{गच्छन्वनानि रौद्राणि दृष्ट्वा क्रन्दति तत्र सः} (ग॰पु॰~२.१६.१३)। \textcolor{red}{क्रन्दते} यथा ब्रह्मपुराणे~– \textcolor{red}{यदैव क्रन्दते जन्तुर्दुःखार्तः पतितः क्वचित्} (ब्र॰पु॰~२१४.१००) गरुडपुराणे च~– \textcolor{red}{हाहेति क्रन्दते नित्यं कीदृशं तु मया कृतम्} (ग॰पु॰~२.१५.८५)। एतेन पूर्वोक्त\-भारत\-प्रयोगोऽपि व्याख्यातः।}\end{sloppypar}
\section[घोर\-रूपिणः]{घोर\-रूपिणः}
\label{sec:ghorarupinah}
\centering\textcolor{blue}{सीतां नीत्वा गुहां गत्वा तत्र तिष्ठ महाबल।\nopagebreak\\
हन्तुमिच्छाम्यहं सर्वान् राक्षसान् घोररूपिणः॥}\nopagebreak\\
\raggedleft{–~अ॰रा॰~३.५.३०}\\
\begin{sloppypar}\hyphenrules{nohyphenation}\justifying\noindent\hspace{10mm} खर\-दूषणौ दृष्ट्वा निर्वासित\-खर\-दूषणो निरस्त\-दूषणो रघु\-कुल\-भूषणः श्रीरामो लक्ष्मणं सतर्कयति यत्त्वं \textcolor{red}{सीतां नीत्वा गुहां प्रविश तावदहमद्यैव घोर\-रूपिणो राक्षसान् हन्तुमिच्छामि}। अत्र \textcolor{red}{घोर\-रूपिणः} इति प्रयोगः कथं \textcolor{red}{न कर्मधारयान्मत्वर्थीयो बहुव्रीहिश्चेत्तदर्थ\-प्रतिपत्तिकरः} इति नियमस्य जागरूकत्वे \textcolor{red}{घोरं च तद्रूपं चेति घोर\-रूपं तदस्ति येषां ते घोर\-रूपिणस्तान् घोर\-रूपिणः} इति चेत्। बहुव्रीहौ सतीप्सितार्थस्यानव\-गतावेष पन्था अनुगतः। \textcolor{red}{घोरं रूपं येषां ते} इत्यर्थे सति न किमप्यर्थ\-वैलक्षण्यम्। मत्वर्थीय इनिर्निन्दायाम्।\footnote{\textcolor{red}{अत इनिठनौ} (पा॰सू॰~५.२.११५) इत्यनेन। \textcolor{red}{भूम\-निन्दा\-प्रशंसासु नित्ययोगेऽति\-शायने। सम्बन्धेऽस्ति\-विवक्षायां भवन्ति मतुबादयः॥} (भा॰पा॰सू॰~५.२.९४)। भाष्यकार\-मते निन्दायामिनिरेव। यथा \textcolor{red}{निन्दायाम्। ककुदावर्ती। सङ्खादकी} (भा॰पा॰सू॰~५.२.९४)।} \textcolor{red}{इन्‌}\-प्रत्यय\-विधानात् \textcolor{red}{निन्दित\-घोर\-रूप\-युक्ताः} इत्यर्थो ध्वन्यते। अथवा \textcolor{red}{घोरं यथा स्यात्तथा रूपयितुं तच्छीलाः} इति विग्रहे \textcolor{red}{सुप्यजातौ णिनिस्ताच्छील्ये} (पा॰सू॰~३.२.७८) इत्यनेन णिनि\-प्रत्यये समाधानम्।\end{sloppypar}
\section[शापितः]{शापितः}
\centering\textcolor{blue}{अत्र किञ्चिन्न वक्तव्यं शापितोऽसि ममोपरि।\nopagebreak\\
तथेति सीतामादाय लक्ष्मणो गह्वरं ययौ॥}\nopagebreak\\
\raggedleft{–~अ॰रा॰~३.५.३१}\\
\begin{sloppypar}\hyphenrules{nohyphenation}\justifying\noindent\hspace{10mm} \textcolor{red}{शप्‌}\-धातोः (\textcolor{red}{शपँ आक्रोशे} धा॰पा॰~१०००, ११६८) कर्मणि \textcolor{red}{क्त}\-प्रत्ययेऽनिट्कत्वात्\footnote{\textcolor{red}{एकाच उपदेशेऽनुदात्तात्‌} (पा॰सू॰~७.२.१०) इत्यनेनानिट्कत्वम्।} \textcolor{red}{शप्तः} इत्येव।\footnote{यथाऽस्मिन्नेव काण्डे \textcolor{red}{दुर्वाससाऽकारणकोपमूर्तिना शप्तः पुरा सोऽद्य विमोचितस्त्वया} (अ॰रा॰~३.१३८)।} \textcolor{red}{आगम\-शास्त्रमनित्यम्} (प॰शे॰~९३.२) इति कृत्वेडागमेऽपि \textcolor{red}{शपितः}।\footnote{यथा भारते दाक्षिणात्य\-पाठे शकुन्तलां प्रति कण्वः~– \textcolor{red}{प्रतिवाक्यं न दद्यास्त्वं शपिता मम पादयोः} (म॰भा॰~१.९६.९)। एवमेव नान्दी\-पुराणे~– \textcolor{red}{मत्कृते येऽत्र शपिताः सावित्र्या ब्राह्मणाः सुराः} (नान्दी॰पु॰~२७.७)।} \textcolor{red}{शापितः} इति कथम्। अत्र स्वार्थे णिचि प्रत्यये ततः क्तान्तेन समाधानम्।\footnote{\textcolor{red}{शपँ आक्रोशे} (धा॰पा॰~१०००, ११६८)~\arrow शप्~\arrow स्वार्थे णिच्~\arrow शप्~णिच्~\arrow शप्~इ~\arrow \textcolor{red}{अत उपधायाः} (पा॰सू॰~७.२.११६)~\arrow शाप्~इ~\arrow शापि~\arrow \textcolor{red}{सनाद्यन्ता धातवः} (पा॰सू॰~३.१.३२)~\arrow धातु\-सञ्ज्ञा~\arrow \textcolor{red}{तयोरेव कृत्य\-क्तखलर्थाः} (पा॰सू॰~३.४.७०)~\arrow शापि~क्त~\arrow शापि~त~\arrow \textcolor{red}{आर्धधातुकस्येड्वलादेः} (पा॰सू॰~७.२.३५)~\arrow \textcolor{red}{आद्यन्तौ टकितौ} (पा॰सू॰~१.१.४६)~\arrow शापि~इट्~त~\arrow \textcolor{red}{णेरनिटि} (पा॰सू॰~६.४.५१)~\arrow शापि~त~\arrow शापित~\arrow विभक्तिकार्यम्~\arrow शापितः।}\end{sloppypar}
\section[समाधिविरमे]{समाधिविरमे}
\centering\textcolor{blue}{ध्यायन् हृदि परात्मानं निर्गुणं गुणभासकम्।\nopagebreak\\
समाधिविरमेऽपश्यद्रावणं गृहमागतम्॥}\nopagebreak\\
\raggedleft{–~अ॰रा॰~३.६.३}\\
\begin{sloppypar}\hyphenrules{nohyphenation}\justifying\noindent\hspace{10mm} सङ्ग्रामे श्रीरामेण स्वधामनीतेषु खर\-दूषण\-त्रिशीर्षेषु प्रतिशोधं चिकीर्षन् रावणः श्रीरामं ध्यायन्तं मारीचं प्रत्यगात्। तत्र समाधि\-विरामे मारीचो रावणमपश्यत्। तत्र \textcolor{red}{समाधि\-विरमे} इति प्रयोगस्तु विचारकोटावागच्छति। न च \textcolor{red}{समाधेर्विरामः समाधि\-विरामस्तस्मिन् समाधि\-विरामे}। \textcolor{red}{विरमणं विरामः}। \textcolor{red}{भावे} (पा॰सू॰~३.३.१८) इति घञ्। \textcolor{red}{विरामोऽवसानम्} (पा॰सू॰~१.४.११०) इति सूत्रं प्रमाणम्। यद्वा \textcolor{red}{विरमन्तेऽस्मिन्निति विरामः}। \textcolor{red}{हलश्च} (पा॰सू॰~३.३.१२१) इति घञ्।\footnote{स चाधिकरणे। \textcolor{red}{हलश्च} (पा॰सू॰~३.३.१२१) इत्यत्र \textcolor{red}{करणाधिकरणयोश्च} (पा॰सू॰~३.३.११७) इत्यस्यानुवृत्तेः।} \textcolor{red}{अच्‌}\-प्रत्ययः कर्तरि। करणाधि\-करणयोः सञ्ज्ञायाञ्चान्य\-प्रत्ययानां \textcolor{red}{हलश्च} (पा॰सू॰~३.३.१२१) इत्यनेन बाधः। अत्रोच्यते। \textcolor{red}{विगता रमा यस्मात्तद्विरमम्}। समाधौ हि भक्ति\-रूपिणी रमा। ब्रह्मानुभवत्वात्। तदभावे रमाया निर्गमः स्वभाविक एवातः। यद्वा \textcolor{red}{रमाऽस्त्यस्मिन्निति रमम्}। अर्शआदित्वादच्।\footnote{\textcolor{red}{अर्शआदिभ्योऽच्} (पा॰सू॰~५.२.१२७) इत्यनेन।} \textcolor{red}{विगतं रममिति विरमम्}।\footnote{\textcolor{red}{कु\-गति\-प्रादयः} (पा॰सू॰~२.२.१८) इत्यनेन प्रादि\-समासः।} \textcolor{red}{समाधेर्विरमं समाधि\-विरमं तस्मिन् समाधि\-विरमे}।\footnote{यद्वा \textcolor{red}{विरमणं विरमः} इति विग्रहे \textcolor{red}{वि}\-पूर्वकात् \textcolor{red}{रम्‌}\-धातोः (\textcolor{red}{रमुँ क्रीडायाम्} धा॰पा॰~८५३) \textcolor{red}{अज्विधौ भयादीनामुपसंख्यानं नपुंसके क्तादिनिवृत्त्यर्थम्} (वा॰~३.३.५६) इत्यनेन \textcolor{red}{अच्‌}\-प्रत्यये विरमम्। अबन्तो \textcolor{red}{विरमः} (वि~रम्~अप्) शब्द इत्याप्टे\-कोशः। तस्मिन् विरमे। यथा \textcolor{red}{सोऽहं नृणां क्षुल्लसुखाय दुःखं महद्गतानां विरमाय तस्य} (भा॰पु॰~३.८.२) इति भागवते \textcolor{red}{विरमाय} इत्यत्र। अत्र \textcolor{red}{विरमाय नाशाय} (भा॰पु॰ वी॰रा॰व्या॰~३.८.२) \textcolor{red}{विरमाय निवृत्तये} (भा॰पु॰ सि॰प्र॰~३.८.२, भा॰पु॰ बा॰प्र॰~३.८.२) इति टीकाकाराः। एवमेव \textcolor{red}{अथ धूमस्य विरमे द्वितीयं रूपदर्शनम्} (म॰भा॰~१२.२४२.१८) इति भारत\-प्रयोगेऽपि।}\end{sloppypar}
\section[सहायं मे]{सहायं मे}
\label{sec:sahayam_me}
\centering\textcolor{blue}{त्वं तु तावत्सहायं मे कृत्वा स्थास्यसि पूर्ववत्।\nopagebreak\\
इत्येवं भाषमाणं तं रावणं वीक्ष्य विस्मितः॥}\nopagebreak\\
\raggedleft{–~अ॰रा॰~३.६.१४}\\
\begin{sloppypar}\hyphenrules{nohyphenation}\justifying\noindent\hspace{10mm} अत्र भावे \textcolor{red}{ष्यञ्} प्रत्यये \textcolor{red}{साहाय्यम्} इति तु पाणिनीयमेव।\footnote{\textcolor{red}{सह अयते एति वेति सहायः।} \textcolor{red}{अयँ गतौ} (धा॰पा॰~४७४) इत्यस्मात् \textcolor{red}{इण् गतौ} (धा॰पा॰~१०४५) इत्यस्माद्वा \textcolor{red}{नन्दि\-ग्रहि\-पचादिभ्यो ल्युणिन्यचः} (पा॰सू॰~३.१.१३४) इत्यनेन कर्तरि पचाद्यच्। ततः \textcolor{red}{सह सुपा} (पा॰सू॰~२.१.४) इत्यनेन सुप्सुपा\-समासः। \textcolor{red}{अनुप्लवः सहायश्चानुचरोऽभिसरः समाः} (अ॰को॰~२.८.७१) इत्यमरः। \textcolor{red}{सहाय}\-शब्दात् \textcolor{red}{गुण\-वचन\-ब्राह्मणादिभ्यः कर्मणि च} (पा॰सू॰~५.१.१२४) इत्यनेन भावे कर्मणि वा \textcolor{red}{ष्यञ्‌}\-प्रत्ययेऽनुबन्ध\-लोपे \textcolor{red}{तद्धितेष्वचामादेः} (पा॰सू॰~७.२.११७) इत्यनेनादि\-वृद्धौ \textcolor{red}{यचि भम्} (पा॰सू॰~१.४.१८) इत्यनेन भसञ्ज्ञायाम् \textcolor{red}{यस्येति च} (पा॰सू॰~६.४.१४८) इत्यनेनाकार\-लोपे विभक्तिकार्ये \textcolor{red}{साहाय्यम्} इति रूपम्। पक्षे \textcolor{red}{सहायाद्वेति वक्तव्यम्} (वा॰~५.१.१३१) इत्यनेन भावे कर्मणि वा \textcolor{red}{वुञ्‌}\-प्रत्ययेऽनुबन्ध\-लोपे \textcolor{red}{तद्धितेष्वचामादेः} (पा॰सू॰~७.२.११७) इत्यनेनादि\-वृद्धौ \textcolor{red}{युवोरनाकौ} (पा॰सू॰~७.१.१) इत्यनेनाकादेशे विभक्तिकार्ये \textcolor{red}{साहायकम्} इति रूपम्।} किन्तु \textcolor{red}{अयनम् अयः} इति भावे \textcolor{red}{एरच्} (पा॰सू॰~३.३.५६) इत्यनेन \textcolor{red}{इ}\-धातोः (\textcolor{red}{इण् गतौ} धा॰पा॰~१०४५) अचि गुणेऽयादेशे विभक्ति\-कार्ये \textcolor{red}{अयः}। \textcolor{red}{सह अयः सहायस्तं सहायम्} इति पूर्वोक्त\-दिशा साधनीयम्।\footnote{\pageref{sec:sahayam}तमे पृष्ठे \ref{sec:sahayam} \nameref{sec:sahayam} इति प्रयोगस्य विमर्शं पश्यन्तु। \textcolor{red}{सह अयः सहायः} इत्यत्र \textcolor{red}{सह सुपा} (पा॰सू॰~२.१.४) इत्यनेन सुप्सुपा\-समासः।}\end{sloppypar}
\section[स्थाप्य]{स्थाप्य}
\centering\textcolor{blue}{श्रुत्वा रामोदितं वाक्यं सापि तत्र तथाकरोत्।\nopagebreak\\
मायासीतां बहिः स्थाप्य स्वयमन्तर्दधेऽनले॥}\nopagebreak\\
\raggedleft{–~अ॰रा॰~३.७.४}\\
\begin{sloppypar}\hyphenrules{nohyphenation}\justifying\noindent\hspace{10mm} अत्र गति\-निवृत्त्यर्थक\-णिजन्त\-\textcolor{red}{स्था}\-धातोः (\textcolor{red}{ष्ठा गतिनिवृत्तौ} धा॰पा॰~९२८) पुकि\footnote{\textcolor{red}{अर्ति\-ह्रीव्ली\-रीक्नूयी\-क्ष्माय्यातां पुङ्णौ} (पा॰सू॰~७.३.३६) इत्यनेन।} ल्यबन्तमेवम्।\footnote{ल्यपि च \textcolor{red}{णेरनिटि} (पा॰सू॰~६.४.५१) इत्यनेन णिलोपः।} किन्त्विदं समासं विना कथं सम्भवमिति चेत्।
अत्र साक्षात्प्रभृति\-गणे \textcolor{red}{बहिस्} इत्यपि पठित्वा \textcolor{red}{साक्षात्प्रभृतीनि च} (पा॰सू॰~१.४.७४) इत्यनेन गतिसञ्ज्ञायां समासः।\footnote{\textcolor{red}{साक्षात्प्रभृतीनि च} (पा॰सू॰~१.४.७४) इत्यत्र \textcolor{red}{क्वचिदेक\-देशोऽप्यनुवर्तते} (प॰शे॰~१८) इति परिभाषया \textcolor{red}{कृञि} (पा॰सू॰~१.४.७२) इत्यस्य मण्डूक\-प्लुत्या निवृत्तौ \textcolor{red}{विभाषा} (पा॰सू॰~१.४.७२) इत्यस्यानुवृत्तौ कृञभावेऽपि साक्षात्प्रभृतीनां गति\-सञ्ज्ञा क्वाचित्का। यद्वा चकार\-ग्रहणात्कृञभावेऽपि गति\-सञ्ज्ञा क्वाचित्का। सत्यां गतिसञ्ज्ञायां \textcolor{red}{कुगतिप्रादयः} (पा॰सू॰~२.२.१८) इत्यनेन समास इति भावः।} ततो \textcolor{red}{समासेऽनञ्पूर्वे क्त्वो ल्यप्} (पा॰सू॰~७.१.३७) इत्यनेन क्त्वा\-स्थाने ल्यबादेशे \textcolor{red}{बहिःस्थाप्य} इति।\end{sloppypar}
\section[कानकम्]{कानकम्}
\centering\textcolor{blue}{पश्य राम मृगं चित्रं कानकं रत्नभूषितम्।\nopagebreak\\
विचित्रबिन्दुभिर्युक्तं चरन्तमकुतोभयम्।\nopagebreak\\
बद्ध्वा देहि मम क्रीडामृगो भवतु सुन्दरः॥}\nopagebreak\\
\raggedleft{–~अ॰रा॰~३.७.६}\\
\begin{sloppypar}\hyphenrules{nohyphenation}\justifying\noindent\hspace{10mm} कनक\-मृगं दृष्ट्वा माया\-सीता माया\-मनुष्यं रामं माया\-मृगं हन्तुं चोदयति यत् \textcolor{red}{कानकं मृगं पश्य}। अत्र विकारार्थे मयटि कृते\footnote{\textcolor{red}{मयड्वैतयोर्भाषायामभक्ष्याच्छादनयोः} (पा॰सू॰~४.३.१४३) इत्यनेन।} \textcolor{red}{कनक\-मयम्}। परन्तु माया\-मृगत्वात्कनक\-विकाराभावे \textcolor{red}{कनकस्यायं कानकस्तं कानकम्}। \textcolor{red}{तस्येदम्} (पा॰सू॰~४.३.१२०) इत्यनेन \textcolor{red}{अण्}। वृद्धौ भत्वादलोपे विभक्तिकार्ये \textcolor{red}{कानकम्}।\footnote{कनक~\arrow \textcolor{red}{तस्येदम्} (पा॰सू॰~४.३.१२०)~\arrow कनक~अण्~\arrow कनक~अ~\arrow \textcolor{red}{तद्धितेष्वचामादेः} (पा॰सू॰~७.२.११७)~\arrow कानक~अ~\arrow \textcolor{red}{यचि भम्} (पा॰सू॰~१.४.१८)~\arrow \textcolor{red}{यस्येति च} (पा॰सू॰~६.४.१४८)~\arrow कानक्~अ~\arrow कानक~\arrow \textcolor{red}{कृत्तद्धित\-समासाश्च} (पा॰सू॰~१.२.४६)~\arrow प्रातिपदिक\-सञ्ज्ञा~\arrow विभक्तिकार्यम्~\arrow कानक~सुँ~\arrow \textcolor{red}{अतोऽम्} (पा॰सू॰~७.१.२४)~\arrow कानक~अम्~\arrow \textcolor{red}{अमि पूर्वः} (पा॰सू॰~६.१.१०७)~\arrow कानकम्।} यद्वा \textcolor{red}{कनकानां समूहः कानकम्}।\footnote{\textcolor{red}{भिक्षादिभ्योऽण्} (पा॰सू॰~४.२.३८) इत्यनेन \textcolor{red}{अण्}। प्रक्रिया पूर्ववत्।} \textcolor{red}{तदस्त्यस्मिन्निति कानकस्तं कानकम्} इति समूहार्थाणन्तान्मत्वर्थीयेऽचि\footnote{\textcolor{red}{अर्शआदिभ्योऽच्} (पा॰सू॰~५.२.१२७) इत्यनेन \textcolor{red}{अच्}। भत्वादलोपः पूर्ववत्।} विभक्ति\-कार्ये।
न च तद्धित\-प्रत्ययान्तादत्र कथं तद्धित\-प्रत्ययः। तथा च कारिका~–\end{sloppypar}
\centering\textcolor{red}{शैषिकान्मतुबर्थीयाच्छैषिको मतुबर्थकः।\nopagebreak\\
सरूपः प्रत्ययो नेष्टः सन्नन्तान्न सनिष्यते॥}\nopagebreak\\
\raggedleft{–~भा॰पा॰सू॰~३.१.७}\\
\begin{sloppypar}\hyphenrules{nohyphenation}\justifying\noindent इतिवचनाच्छैषिकाच्छैषिकः सरूप\-प्रत्ययो न मत्वर्थीयान्मत्वर्थीयो न। सरूपत्वं नाम समान\-देशत्वे सति समानार्थ\-बोधकत्वम्। अतो न दोषः।\footnote{यतो ह्यत्र चातुरर्थिक\-तद्धित\-प्रत्ययान्मत्वर्थीय\-तद्धित\-प्रत्ययः। तौ न सरूपौ। भाष्येऽपि \textcolor{red}{शाकल्यस्य च्छात्राः} इत्यर्थे \textcolor{red}{शाकलाः} (भा॰पा॰सू॰~४.१.१८) इत्युदाहरणेऽपत्यार्थक\-तद्धित\-प्रत्ययाच्छैषिक\-तद्धित\-प्रत्ययः।} यद्वा \textcolor{red}{कनकमेव कानकस्तं कानकम्} इति स्वार्थेऽण्।\footnote{\textcolor{red}{प्रज्ञादिभ्यश्च} (पा॰सू॰~५.४.३८) इत्यनेन। प्रक्रिया पूर्ववत्।}  यद्वा \textcolor{red}{कनके भवः कानकस्तं कानकम्}।\footnote{\textcolor{red}{तत्र भवः} (पा॰सू॰~४.३.५३) इत्यनेन \textcolor{red}{अण्‌}। प्रक्रिया पूर्ववत्।} यद्वा \textcolor{red}{कनके जातः कानकस्तं कानकम्}।\footnote{\textcolor{red}{तत्र जातः} (पा॰सू॰~४.३.२५) इत्यनेन \textcolor{red}{अण्‌}। प्रक्रिया पूर्ववत्।} \textcolor{red}{कानकं मृगं बद्ध्वाऽऽनय} इति भगवत्यादिशति।\end{sloppypar}
\section[वध्यमाना]{वध्यमाना}
\centering\textcolor{blue}{इत्युक्त्वा वध्यमाना सा स्वबाहुभ्यां रुरोद ह।\nopagebreak\\
तच्छ्रुत्वा लक्ष्मणः कर्णौ पिधायातीव दुःखितः॥}\nopagebreak\\
\raggedleft{–~अ॰रा॰~३.७.३५}\\
\begin{sloppypar}\hyphenrules{nohyphenation}\justifying\noindent\hspace{10mm} माया\-मृगस्य श्रीरामानुकारि\-स्वरं श्रुत्वा लक्ष्मणं गमयितुमिच्छन्ती यातुमनिच्छन्तं तं भर्त्सयित्वा बाहुभ्यां हृदयं ताडयन्ती रुरोद माया\-सीता। तत्र \textcolor{red}{वध्यमाना} इति प्रयुक्तम्। \textcolor{red}{वध} आदेशो हि \textcolor{red}{हन्‌}\-धातोः (\textcolor{red}{हनँ हिंसागत्योः} धा॰पा॰~१०१२) लिङि लुङि चार्ध\-धातुक\-प्रत्यये भवति। तथा च सूत्र\-द्वयम्~– \textcolor{red}{हनो वध लिङि} (पा॰सू॰~२.४.४२) \textcolor{red}{लुङि च} (पा॰सू॰~२.४.४३)। अतः \textcolor{red}{वध्यमाना} अपाणिनीयमेव \textcolor{red}{बाहुभ्यां हृदये हन्यते} इति विग्रहे \textcolor{red}{हन्यमाना} इत्येव पाणिनीयमिति चेत्। हिंसार्थे \textcolor{red}{वध्‌}\-धातुरपि।\footnote{स च \textcolor{red}{जनिवध्योश्च} (पा॰सू॰~७.३.३५) इति सूत्रेण ज्ञापितः। \textcolor{red}{‘जनिवध्योश्च’। जनकः। ‘वधँ हिंसायाम्’। वधकः} (वै॰सि॰कौ॰~२८९५)। \textcolor{red}{‘वधँ हिंसायामिति’। धात्वन्तरं भौवादिकम्। भ्वादेराकृतिगणत्वात्} (बा॰म॰~२८९५)।} तस्य \textcolor{red}{वध्यते} इति कर्म\-वाच्ये \textcolor{red}{सार्वधातुके यक्} (पा॰सू॰~३.१.६७) इत्यनेन यकि ततश्च शानच्प्रत्यये मुगागमे टापि \textcolor{red}{वध्यमाना}।\footnote{वध्~\arrow \textcolor{red}{भावकर्मणोः} (पा॰सू॰~१.३.१३)~\arrow \textcolor{red}{वर्तमाने लट्} (पा॰सू॰~३.२.१२३)~\arrow \textcolor{red}{लटः शतृशानचावप्रथमा\-समानाधिकरणे} (पा॰सू॰~३.२.१२४)~\arrow वध्~शानच्~\arrow वध्~आन~\arrow \textcolor{red}{सार्वधातुके यक्} (पा॰सू॰~३.१.६७)~\arrow \textcolor{red}{आद्यन्तौ टकितौ} (पा॰सू॰~१.१.४६)~\arrow वध्~यक्~आन~\arrow वध्~य~आन~\arrow \textcolor{red}{आने मुक्} (पा॰सू॰~७.२.८२)~\arrow \textcolor{red}{आद्यन्तौ टकितौ} (पा॰सू॰~१.१.४६)~\arrow वध्~य~मुँक्~आन~\arrow वध्~य~म्~आन~\arrow वध्यमान~\arrow \textcolor{red}{अजाद्यतष्टाप्‌} (पा॰सू॰~४.१.४)~\arrow वध्यमान~टाप्~\arrow वध्यमान~आ~\arrow \textcolor{red}{अकः सवर्णे दीर्घः} (पा॰सू॰~६.१.१०१)~\arrow वध्यमाना~\arrow विभक्तिकार्यम्~\arrow वध्यमाना।} यद्वा \textcolor{red}{हननं वधः}।\footnote{\textcolor{red}{हनँ हिंसागत्योः} (धा॰पा॰~१०१२)~\arrow हन्~\arrow \textcolor{red}{हनश्च वधः} (पा॰सू॰~३.३.७६)~\arrow वध्~अप्~\arrow वध्~अ~\arrow वध~\arrow विभक्तिकार्यम्~\arrow वधः।} \textcolor{red}{वधमाचरतीति वधति}।\footnote{वध~\arrow \textcolor{red}{सर्वप्राति\-पदिकेभ्य आचारे क्विब्वा वक्तव्यः} (वा॰~३.१.११)~\arrow वध~क्विँप्~\arrow वध~व्~\arrow \textcolor{red}{वेरपृक्तस्य} (पा॰सू॰~६.१.६७)~\arrow वध~\arrow \textcolor{red}{सनाद्यन्ता धातवः} (पा॰सू॰~३.१.३२)~\arrow धातुसञ्ज्ञा~\arrow \textcolor{red}{शेषात्कर्तरि परस्मैपदम्} (पा॰सू॰~१.३.७८)~\arrow \textcolor{red}{वर्तमाने लट्} (पा॰सू॰~३.२.१२३)~\arrow वध~लट्~\arrow वध~तिप्~\arrow वध~ति~\arrow \textcolor{red}{कर्तरि शप्‌} (पा॰सू॰~३.१.६८)~\arrow वध~शप्~ति~\arrow वध~अ~ति~\arrow \textcolor{red}{अतो गुणे} (पा॰सू॰~६.१.९७)~\arrow वध~ति~\arrow वधति।} \textcolor{red}{बाहू हृदये वधमाचरत इति वधतः}\footnote{वध~\arrow \textcolor{red}{धातुसञ्ज्ञा} (पूर्ववत्)~\arrow \textcolor{red}{शेषात्कर्तरि परस्मैपदम्} (पा॰सू॰~१.३.७८)~\arrow \textcolor{red}{वर्तमाने लट्} (पा॰सू॰~३.२.१२३)~\arrow वध~लट्~\arrow वध~तस्~\arrow \textcolor{red}{कर्तरि शप्‌} (पा॰सू॰~३.१.६८)~\arrow वध~शप्~तस्~\arrow वध~अ~तस्~\arrow \textcolor{red}{अतो गुणे} (पा॰सू॰~६.१.९७)~\arrow वध~तस्~\arrow \textcolor{red}{ससजुषो रुः} (पा॰सू॰~८.२.६६)~\arrow वधतरुँ~\arrow \textcolor{red}{खरवसानयोर्विसर्जनीयः} (पा॰सू॰~८.३.१५)~\arrow वधतः।} विग्रहेऽस्मिन् \textcolor{red}{वध}\-शब्दादाचार\-क्विबन्तात्कर्मणि लकारे \textcolor{red}{बाहुभ्यां हृदये वध्यते} इति विग्रहे \textcolor{red}{वध्यमाना}।\footnote{वध~\arrow धातु\-सञ्ज्ञा (पूर्ववत्)~\arrow \textcolor{red}{भावकर्मणोः} (पा॰सू॰~१.३.१३)~\arrow \textcolor{red}{वर्तमाने लट्} (पा॰सू॰~३.२.१२३)~\arrow \textcolor{red}{लटः शतृशानचावप्रथमा\-समानाधिकरणे} (पा॰सू॰~३.२.१२४)~\arrow वध~शानच्~\arrow वध~आन~\arrow \textcolor{red}{सार्वधातुके यक्} (पा॰सू॰~३.१.६७)~\arrow \textcolor{red}{आद्यन्तौ टकितौ} (पा॰सू॰~१.१.४६)~\arrow वध~यक्~आन~\arrow वध~य~आन~\arrow \textcolor{red}{अतो लोपः} (पा॰सू॰~६.४.४८)~\arrow वध्~य~आन~\arrow \textcolor{red}{आने मुक्} (पा॰सू॰~७.२.८२)~\arrow \textcolor{red}{आद्यन्तौ टकितौ} (पा॰सू॰~१.१.४६)~\arrow वध्~य~मुँक्~आन~\arrow वध्~य~म्~आन~\arrow वध्यमान~\arrow \textcolor{red}{अजाद्यतष्टाप्‌} (पा॰सू॰~४.१.४)~\arrow वध्यमान~टाप्~\arrow वध्यमान~आ~\arrow \textcolor{red}{अकः सवर्णे दीर्घः} (पा॰सू॰~६.१.१०१)~\arrow वध्यमाना~\arrow विभक्तिकार्यम्~\arrow वध्यमाना।} अत आचार\-क्विबन्त\-\textcolor{red}{वध}\-धातोः कर्म\-वाच्ये शानचि कृते सिद्धं रूपमिदम्।\end{sloppypar}
\section[क्रोशमानाम्]{क्रोशमानाम्}
\centering\textcolor{blue}{वाक्शरेण हतस्त्वं मे क्षन्तुमर्हसि देवर।\nopagebreak\\
इत्येवं क्रोशमानां तां रामागमनशङ्कया॥}\nopagebreak\\
\raggedleft{–~अ॰रा॰~३.७.६१}\\
\begin{sloppypar}\hyphenrules{nohyphenation}\justifying\noindent\hspace{10mm} \textcolor{red}{क्रुश्‌}\-धातुः (\textcolor{red}{क्रुशँ आह्वाने रोदने च} धा॰पा॰~८५६) परस्मैपदी। ततः \textcolor{red}{क्रोशति} इति विग्रहे शतृ\-प्रत्यये ङीपि नुमि \textcolor{red}{क्रोशन्ती} इति पाणिनीय\-प्रयोगः।\footnote{यथा \textcolor{red}{क्रोशन्ती राम रामेति लक्ष्मणेति च विस्वरम्} (वा॰रा॰~४.६.९) इति वाल्मीकि\-प्रयोगे। अस्मिन्नेव ग्रन्थे च~– \textcolor{red}{लङ्कां गत्वा सभामध्ये क्रोशन्ती पादसन्निधौ} (अ॰रा॰~३.५.३८) \textcolor{red}{क्रोशन्ती करुणं दीना जगाद दशकन्धरम्} (अ॰रा॰~६.१०.२९)।} \textcolor{red}{क्रोशमानाम्} इत्यपि तथा हि \textcolor{red}{कर्तरि कर्म\-व्यतिहारे} (पा॰सू॰~१.३.१४) 
इत्यत्राऽत्मनेपदे। यतो हि सत्यपि वेदवती जानन्त्यपि राम\-पराक्रमं प्रवृत्ति\-योग्यं क्रोशनं रावण\-वधार्थं स्वयं करोति तत्राऽत्मनेपदम्। \textcolor{red}{क्रोशत इति क्रोशमाना ताम्} इति विग्रह आत्मने\-पदीयत्वाच्छानचि प्रत्यये मुगागमे टापि विभक्ति\-कार्ये सुलोपे \textcolor{red}{क्रोशमानाम्} शब्दोऽपि पाणिनीयः।\end{sloppypar}
\section[विलप्यमाना]{विलप्यमाना}
\centering\textcolor{blue}{कृशातिदीना परिकर्मवर्जिता दुःखेन शुष्यद्वदनातिविह्वला।\nopagebreak\\
हा राम रामेति विलप्यमाना सीता स्थिता राक्षसवृन्दमध्ये॥}\nopagebreak\\
\raggedleft{–~अ॰रा॰~३.७.६६}\\
\begin{sloppypar}\hyphenrules{nohyphenation}\justifying\noindent\hspace{10mm} अत्र रावण\-नीतां सीतां वर्णयति ग्रन्थकृद्यत् \textcolor{red}{विलप्यमाना} इति। \textcolor{red}{वि}\-पूर्वको \textcolor{red}{लप्‌}\-धातुः (\textcolor{red}{लपँ व्यक्तायां वाचि} धा॰पा॰~४०२) परस्मैपदी। ततः \textcolor{red}{शतृ}\-प्रत्यये \textcolor{red}{विलपन्ती} इति।\footnote{यथा \textcolor{red}{आदीप्य चानुमरणे विलपन्ती मनो दधे} (भा॰पु॰~४.२८.५०) इति भागवत\-प्रयोगे।} \textcolor{red}{विलप्यमाना} इति कथम्। अत्र\footnote{\textcolor{red}{लप्‌}\-धातोः कर्तरि शतरि प्रत्यये कृते।} यक्प्रत्ययाभाव आत्मनेपदाभावश्च। न च \textcolor{red}{कर्तरि कर्म\-व्यतिहारे} (पा॰सू॰~१.३.१४) इत्यनेनाऽत्मनेपदं क्रियतां सीता साधारण\-विलापं करोतीत्यर्थ\-स्वीकारे ग्रन्थ\-गौरव\-पुरः\-सरं भाव\-माधुर्यमपीति चेत्। आत्मनेपदे सत्यपि \textcolor{red}{यक्} कुतः सीताया विलपन\-कर्तृत्वादिति चेत्।\footnote{यतो हि \textcolor{red}{सार्वधातुके यक्} (पा॰सू॰~३.१.६७) इति हि सूत्रं भावकर्मणोरेव प्रवर्तते। \textcolor{red}{चिण् भावकर्मणोः} (पा॰सू॰~३.१.६६) इत्यतो \textcolor{red}{भावकर्मणोः} इत्यस्यानुवृत्तेः।} \textcolor{red}{विलप्यत इति विलप्यमानम्} भावे शानचि \textcolor{red}{भाव\-कर्मणोः} (पा॰सू॰~१.३.१३) इत्यात्मने\-पदत्वात् \textcolor{red}{विलप्यमानमस्ति नित्यमस्यामिति विलप्यमाना}।\footnote{\textcolor{red}{अर्शआदिभ्योऽच्} (पा॰सू॰~५.२.१२७) इत्यनेनाचि \textcolor{red}{यचि भम्} (पा॰सू॰~१.४.१८) इत्यनेन भसञ्ज्ञायां \textcolor{red}{यस्येति च} (पा॰सू॰~६.४.१४८) इत्यनेनालोपे \textcolor{red}{अजाद्यतष्टाप्‌} (पा॰सू॰~४.१.४) इत्यनेन टापि विभक्ति\-कार्ये। विग्रहे \textcolor{red}{नित्यम्} इति तु \textcolor{red}{भूम\-निन्दा\-प्रशंसासु नित्ययोगेऽति\-शायने। सम्बन्धेऽस्ति\-विवक्षायां भवन्ति मतुबादयः॥} (भा॰पा॰सू॰~५.२.९४) इत्यनुसारम्।} अथवेमं दिवादिगणे मत्वा\footnote{\textcolor{red}{बहुलमेतन्निदर्शनम्} (धा॰पा॰ ग॰सू॰~१९३८) \textcolor{red}{आकृतिगणोऽयम्} (धा॰पा॰ ग॰सू॰~१९९२) \textcolor{red}{भूवादिष्वेतदन्तेषु दशगणीषु धातूनां पाठो निदर्शनाय तेन स्तम्भुप्रभृतयः सौत्राश्चुलुम्पादयो वाक्यकारीयाः प्रयोगसिद्धा विक्लवत्यादयश्च} (मा॰धा॰वृ॰~१०.३२८) इत्यनुसारमाकृति\-गणत्वाद्दिवादि\-गण ऊह्योऽयमात्मने\-पदी धातुरिति भावः।} दिवादित्वाच्छ्यन्यात्मनेपदे \textcolor{red}{विलप्यमाना}।\end{sloppypar}
\section[घातितः]{घातितः}
\centering\textcolor{blue}{रावणं तत्र युद्धं मे बभूवारिविमर्दन।\nopagebreak\\
तस्य वाहान् रथं चापं छित्त्वाहं तेन घातितः॥}\nopagebreak\\
\raggedleft{–~अ॰रा॰~३.८.२८}\\
\begin{sloppypar}\hyphenrules{nohyphenation}\justifying\noindent\hspace{10mm} अत्र जटायु\-समीपं गत्वा श्रीरामभद्रस्तद्दशां विलोक्य तत्पराभव\-कारणमपृच्छत्। अत्र जटायुषा निगद्यते यत् \textcolor{red}{तेनाहं घातितः}। तत्र \textcolor{red}{हन्‌}\-धातोः (\textcolor{red}{हनँ हिंसागत्योः} धा॰पा॰~१०१२) कर्मणि \textcolor{red}{क्त}\-प्रत्यये कृते \textcolor{red}{हतः} इत्यनेन भवितव्यम्। यतो \textcolor{red}{रावणो मामहन्} पुनः कर्म\-वाच्ये \textcolor{red}{रावणेनाहमहन्ये} इत्यर्थे \textcolor{red}{रावणेनाहं हतः}। किन्तु \textcolor{red}{घातितः} अयं प्रयोगो हि ण्यन्त\-क्तान्तस्येत्येवापाणिनीयः प्रतिभाति। यतो हि ण्यन्त\-प्रयोगास्तु प्रायः प्रेरक\-कर्तृके भवन्ति। यथा \textcolor{red}{रामो रावणं हन्ति विभीषणस्तं प्रेरयति} इत्यर्थे \textcolor{red}{विभीषणो रामेण रावणं घातयति}। अत्र तु कश्चन प्रेरको नासीत्। अतः प्रयोजकाभावे \textcolor{red}{हेतुमति च} (पा॰सू॰~३.१.२६) इत्यनेन कथं णिजिति चेत्सत्यम्।
\textcolor{red}{रावणो खड्गेन जटायुषमघातयत्}।\footnote{\textcolor{red}{हनँ हिंसागत्योः} (धा॰पा॰~१०१२)~\arrow हन्~\arrow \textcolor{red}{हेतुमति च} (पा॰सू॰~३.१.२६)~\arrow हन्~णिच्~\arrow हन्~इ~\arrow \textcolor{red}{हनस्तोऽचिण्णलोः} (पा॰सू॰~७.३.३२)~\arrow हत्~इ~\arrow \textcolor{red}{हो हन्तेर्ञ्णिन्नेषु} (पा॰सू॰~७.३.५४)~\arrow घत्~इ~\arrow \textcolor{red}{अत उपधायाः} (पा॰सू॰~७.२.११६)~\arrow घात्~इ~\arrow घाति~\arrow \textcolor{red}{सनाद्यन्ता धातवः} (पा॰सू॰~३.१.३२)~\arrow धातुसञ्ज्ञा~\arrow \textcolor{red}{शेषात्कर्तरि परस्मैपदम्} (पा॰सू॰~१.३.७८)~\arrow \textcolor{red}{अनद्यतने लङ्} (पा॰सू॰~३.२.१११)~\arrow घाति~लङ्~\arrow घाति~तिप्~\arrow घाति~ति~\arrow \textcolor{red}{लुङ्लङ्लृङ्क्ष्वडुदात्तः} (पा॰सू॰~६.४.७१)~\arrow अट्~घाति~ति~\arrow अ~घाति~ति~\arrow \textcolor{red}{कर्तरि शप्‌} (पा॰सू॰~३.१.६८)~\arrow अ~घाति~शप्~ति~\arrow अ~घाति~अ~ति~\arrow \textcolor{red}{सार्वधातुकार्ध\-धातुकयोः} (पा॰सू॰~७.३.८४)~\arrow अ~घाते~अ~ति~\arrow \textcolor{red}{एचोऽयवायावः} (पा॰सू॰~६.१.७८)~\arrow अ~घातय्~अ~ति~\arrow \textcolor{red}{इतश्च} (पा॰सू॰~३.४.१००)~\arrow अ~घातय्~अ~त्~\arrow अघातयत्।} पुनः कर्म\-वाच्ये \textcolor{red}{रावणेन खड्गेन जटायुरघात्यत}\footnote{घाति~\arrow धातुसञ्ज्ञा (पूर्ववत्)~\arrow \textcolor{red}{भावकर्मणोः} (पा॰सू॰~१.३.१३)~\arrow \textcolor{red}{अनद्यतने लङ्} (पा॰सू॰~३.२.१११)~\arrow घाति~लङ्~\arrow घाति~त~\arrow \textcolor{red}{लुङ्लङ्लृङ्क्ष्वडुदात्तः} (पा॰सू॰~६.४.७१)~\arrow अट्~घाति~त~\arrow अ~घाति~त~\arrow \textcolor{red}{सार्वधातुके यक्} (पा॰सू॰~३.१.६७)~\arrow \textcolor{red}{आद्यन्तौ टकितौ} (पा॰सू॰~१.१.४६)~\arrow अ~घाति~यक्~ति~\arrow अ~घाति~य~त~\arrow \textcolor{red}{णेरनिटि} (पा॰सू॰~६.४.५१)~\arrow अ~घात्~य~त्~\arrow अघात्यत।} इत्यर्थे हिंसार्थक\-\textcolor{red}{हन्‌}\-धातोर्णिचि \textcolor{red}{हनस्तोऽचिण्णलोः} (पा॰सू॰~७.३.३२) इत्यनेन तान्तादेशे \textcolor{red}{हो हन्तेर्ञ्णिन्नेषु} (पा॰सू॰~७.३.५४) इत्यनेन घकारे \textcolor{red}{अत उपधायाः} (पा॰सू॰~७.२.११६) इत्यनेन वृद्धौ ततः \textcolor{red}{समान\-कर्तृकयोः पूर्व\-काले} (पा॰सू॰~३.४.२१) इत्यनेन कर्मणि \textcolor{red}{क्त}\-प्रत्यये विभक्ति\-कार्ये \textcolor{red}{घातितः} इति।\footnote{घाति~\arrow धातुसञ्ज्ञा पूर्ववत्~\arrow \textcolor{red}{समान\-कर्तृकयोः पूर्व\-काले} (पा॰सू॰~३.४.२१)~\arrow घाति~क्त~\arrow घाति~त~\arrow \textcolor{red}{आर्धधातुकस्येड्वलादेः} (पा॰सू॰~७.२.३५)~\arrow \textcolor{red}{आद्यन्तौ टकितौ} (पा॰सू॰~१.१.४६)~\arrow घाति~इट्~त~\arrow घाति~इ~त~\arrow \textcolor{red}{निष्ठायां सेटि} (पा॰सू॰~६.४.५२)~\arrow घात्~इ~त~\arrow घातित~\arrow विभक्तिकार्यम्~\arrow घातितः।} यद्वाऽत्र स्वार्थे णिचि \textcolor{red}{रावणेन जटायुरहन्यत} इत्यर्थ एव \textcolor{red}{रावणेन जटायुरघात्यत}।\footnote{प्रक्रिया पूर्ववत्।} पुनरस्मिन्स्वार्थे \textcolor{red}{क्त}\-प्रत्यये \textcolor{red}{घातितः}।\footnote{प्रक्रिया पूर्ववत्।} यद्वा \textcolor{red}{हननमेव घातः}।\footnote{\textcolor{red}{हनँ हिंसागत्योः} (धा॰पा॰~१०१२)~\arrow हन्~\arrow \textcolor{red}{भावे} (पा॰सू॰~३.३.१८)~\arrow हन्~घञ्~\arrow हन्~अ~\arrow \textcolor{red}{हनस्तोऽचिण्णलोः} (पा॰सू॰~७.३.३२)~\arrow हत्~अ~\arrow \textcolor{red}{हो हन्तेर्ञ्णिन्नेषु} (पा॰सू॰~७.३.५४)~\arrow घत्~अ~\arrow \textcolor{red}{अत उपधायाः} (पा॰सू॰~७.२.११६)~\arrow घात्~अ~\arrow घात~\arrow विभक्ति\-कार्यम्~\arrow घातः। यथा \textcolor{red}{चरेद्व्रतमहत्वाऽपि घातार्थं चेत्समागतः} (या॰स्मृ॰~३.२५२) \textcolor{red}{वियोगो मुग्धाक्ष्याः स खलु रिपुघातावधिरभूत्} (उ॰रा॰च॰~३.४४) \textcolor{red}{यो हन्यात्तस्य पापं स्याच्छत\-ब्राह्मण\-घातजम्} (प॰त॰~१.३१२) \textcolor{red}{सदय\-हृदय\-दर्शित\-पशुघातम्} (गी॰गो॰~१.९) इत्यादि\-प्रयोगेषु।} \textcolor{red}{घातमितो घातितः}।\footnote{\textcolor{red}{द्वितीया श्रितातीत\-पतित\-गतात्यस्त\-प्राप्तापन्नैः} (पा॰सू॰~२.१.२४) इत्यत्र \textcolor{red}{द्वितीया} इति योग\-विभागेन समासे \textcolor{red}{घात~इत} इति स्थिते \textcolor{red}{कृत्तद्धित\-समासाश्च} (पा॰सू॰~१.२.४६) इत्यनेन प्रातिपदिक\-सञ्ज्ञायां \textcolor{red}{सुपो धातु\-प्रातिपदिकयोः} (पा॰सू॰~२.४.७१) इत्यनेन विभक्ति\-लुकि \textcolor{red}{शकन्ध्वादिषु पर\-रूपं वाच्यम्} (वा॰~६.१.९४) इत्यनेन पररूपे विभक्तिकार्ये सिद्धम्।}\end{sloppypar}
\section[स्मयन्]{स्मयन्}
\centering\textcolor{blue}{तथेति रामः पस्पर्श तदङ्गं पाणिना स्मयन्।\nopagebreak\\
ततः प्राणान्परित्यज्य जटायुः पतितो भुवि॥}\nopagebreak\\
\raggedleft{–~अ॰रा॰~३.८.३६}\\
\begin{sloppypar}\hyphenrules{nohyphenation}\justifying\noindent\hspace{10mm} जटायुषो वृत्तान्तं समाकर्ण्य राजीव\-लोचनो रामः स्मयमानस्तदङ्गं पस्पर्श। अत्र \textcolor{red}{स्मयन्} प्रयोगो ह्यपाणिनीय इव। यतो हीषद्धासार्थकः \textcolor{red}{स्मि}\-धातुः (\textcolor{red}{ष्मिङ् ईषद्धसने} धा॰पा॰~९४८) आत्मनेपदी। ततश्च \textcolor{red}{स्मयत इति स्मयमानः} इत्येव पाणिनीयः। किन्तु \textcolor{red}{स्मयन्} इत्यपि। तथा च \textcolor{red}{स्मयत इति स्मयः}। पचादित्वादच्।\footnote{\textcolor{red}{नन्दि\-ग्रहि\-पचादिभ्यो ल्युणिन्यचः} (पा॰सू॰~३.१.१३४) इत्यनेन।} \textcolor{red}{स्मय इवाऽचरति} इत्यर्थे क्विपि सर्वापहारि\-लोपे सनाद्यन्तत्वाद्धातु\-सञ्ज्ञायां लटि तिपि शपि पररूपे \textcolor{red}{स्मयति}।\footnote{स्मय~\arrow \textcolor{red}{सर्वप्राति\-पदिकेभ्य आचारे क्विब्वा वक्तव्यः} (वा॰~३.१.११)~\arrow स्मय~क्विँप्~\arrow स्मय~व्~\arrow \textcolor{red}{वेरपृक्तस्य} (पा॰सू॰~६.१.६७)~\arrow स्मय~\arrow \textcolor{red}{सनाद्यन्ता धातवः} (पा॰सू॰~३.१.३२)~\arrow धातुसञ्ज्ञा~\arrow \textcolor{red}{शेषात्कर्तरि परस्मैपदम्} (पा॰सू॰~१.३.७८)~\arrow \textcolor{red}{वर्तमाने लट्} (पा॰सू॰~३.२.१२३)~\arrow स्मय~लट्~\arrow स्मय~तिप्~\arrow स्मय~ति~\arrow \textcolor{red}{कर्तरि शप्‌} (पा॰सू॰~३.१.६८)~\arrow स्मय~शप्~ति~\arrow स्मय~अ~ति~\arrow \textcolor{red}{अतो गुणे} (पा॰सू॰~६.१.९७)~\arrow स्मय~ति~\arrow स्मयति।} \textcolor{red}{स्मयतीति स्मयन्} इत्यर्थे शतृ\-प्रयोगे न दोषः।\footnote{स्मय~\arrow धातुसञ्ज्ञा (पूर्ववत्)~\arrow \textcolor{red}{शेषात्कर्तरि परस्मैपदम्} (पा॰सू॰~१.३.७८)~\arrow \textcolor{red}{वर्तमाने लट्} (पा॰सू॰~३.२.१२३)~\arrow स्मय~लट्~\arrow \textcolor{red}{लटः शतृशानचावप्रथमा\-समानाधिकरणे} (पा॰सू॰~३.२.१२४)~\arrow स्मय~शतृँ~\arrow स्मय~अत्~\arrow \textcolor{red}{अतो गुणे} (पा॰सू॰~६.१.९७)~\arrow स्मयत्~\arrow \textcolor{red}{कृत्तद्धित\-समासाश्च} (पा॰सू॰~१.२.४६)~\arrow प्रातिपादिक\-सञ्ज्ञा~\arrow विभक्ति\-कार्यम्~\arrow स्मयत्~सुँ~\arrow स्मयत्~स्~\arrow \textcolor{red}{उगिदचां सर्वनामस्थानेऽधातोः} (पा॰सू॰~७.१.७०)~\arrow \textcolor{red}{मिदचोऽन्त्यात्परः} (पा॰सू॰~१.१.४७)~\arrow स्मय~नुँम्~त्~स्~\arrow स्मय~न्~त्~स्~\arrow \textcolor{red}{हल्ङ्याब्भ्यो दीर्घात्सुतिस्यपृक्तं हल्} (पा॰सू॰~६.१.६८)~\arrow स्मय~न्~त्~\arrow \textcolor{red}{संयोगान्तस्य लोपः} (पा॰सू॰~८.२.२३)~\arrow स्मय~न्~त्~स्~\arrow स्मयन्।} \textcolor{red}{श्रीराम ईषद्धासानुकूल\-व्यापार\-सदृशाचरणानुकूल\-व्यापाराश्रयः} इति शाब्द\-बोधः। यतो हि श्रीरामचन्द्रस्तु जटायुषं पितरं मन्यमानस्तन्म्रियमाण\-दशां दृष्ट्वा शोकातुर आसीत्। किन्तु मरण\-काले जटायुषो मनसि व्यथा मा भूदिति कृत्वा \textcolor{red}{स्मयमान इव प्रतीयते स्म}। अतः \textcolor{red}{स्मयन्} इति सम्यक्पाणिनीयः। जटायुषं प्रति राघवेन्द्रस्य व्यथामत्रैव ग्रन्थकृत्स्पष्टयति यथा~–\end{sloppypar}
\centering\textcolor{blue}{रामस्तमनुशोचित्वा बन्धुवत्साश्रुलोचनः।\nopagebreak\\
लक्ष्मणेन समानाय्य काष्ठानि प्रददाह तम्॥}\nopagebreak\\
\raggedleft{–~अ॰रा॰~३.८.३७}\\
\begin{sloppypar}\hyphenrules{nohyphenation}\justifying\noindent अतः \textcolor{red}{स्मयन्} इत्याचार\-क्विबन्ताच्छतृ\-प्रत्ययः कमपि निगूढं भावं व्यञ्जयति।\end{sloppypar}
\section[अनुशोचित्वा]{अनुशोचित्वा}
\centering\textcolor{blue}{रामस्तमनुशोचित्वा बन्धुवत्साश्रुलोचनः।\nopagebreak\\
लक्ष्मणेन समानाय्य काष्ठानि प्रददाह तम्॥}\nopagebreak\\
\raggedleft{–~अ॰रा॰~३.८.३७}\\
\begin{sloppypar}\hyphenrules{nohyphenation}\justifying\noindent\hspace{10mm} जटायुषो मृत्युं दृष्ट्वा भक्त\-वत्सलः श्रीरामस्तमनुशोच्य तद्दाह\-संस्कारं चक्रे। अत्रानूपसर्ग\-योगेन \textcolor{red}{समासेऽनञ्पूर्वे क्त्वो ल्यप्} (पा॰सू॰~७.१.३७) इत्यनेन ल्यपि \textcolor{red}{अनुशोच्य} इत्येव पाणिनि\-सम्मतम्। \textcolor{red}{अनुशोचित्वा} अपि शोकार्थकात् \textcolor{red}{शुच्} धातोः (\textcolor{red}{शुचँ शोके} धा॰पा॰~१८३)
\textcolor{red}{क्त्वा} प्रत्यय इति कृते गुणे। न च कित्वाल्लघूपध\-गुण\-निषेधः शङ्क्यः।\footnote{\textcolor{red}{ग्क्ङिति च} (पा॰सू॰~१.१.५) इत्यनेन कित्त्वे गुणनिषेधः प्राप्तः।} \textcolor{red}{न क्त्वा सेट्} (पा॰सू॰~१.२.१८) इत्यनेन कित्व\-निषेधे गुणे \textcolor{red}{अनुशोचित्वा}। न च समासे सति \textcolor{red}{अनुशोच्य} इति भविष्यति कथम् \textcolor{red}{अनुशोचित्वा} इति। \textcolor{red}{अनु}\-शब्दस्य \textcolor{red}{शोचित्वा} इत्यनेन सह न योगोऽपि तु \textcolor{red}{तम्} इत्यनेन सह। तथा च \textcolor{red}{अनुर्लक्षणे} (पा॰सू॰~१.४.८४) इत्यनेन कर्म\-प्रवचनीय\-सञ्ज्ञायां \textcolor{red}{कर्म\-प्रवचनीय\-युक्ते द्वितीया} (पा॰सू॰~२.३.८) इत्यनेन द्वितीया\-विभक्तिः। अथ चानूपयोगाभावे \textcolor{red}{क्त्वा}\-प्रत्ययः निर्विवादः। यद्वा \textcolor{red}{लक्षणेत्थम्भूताख्यान\-भाग\-वीप्सासु प्रतिपर्यनवः} (पा॰सू॰~१.४.९०) इत्यनेन \textcolor{red}{अनु}\-शब्दस्य कर्म\-प्रवचनीय\-सञ्ज्ञायां द्वितीयायाञ्च सत्यां \textcolor{red}{तमनु शोचित्वा} इति साधु।\end{sloppypar}
\section[प्राणरिरक्षया]{प्राणरिरक्षया}
\centering\textcolor{blue}{बाहुभ्यां वेष्टितावत्र तव प्राणरिरक्षया।\nopagebreak\\
छिन्नौ तव भुजौ त्वं च को वा विकटरूपधृक्॥}\nopagebreak\\
\raggedleft{–~अ॰रा॰~३.९.१४}\\
\begin{sloppypar}\hyphenrules{nohyphenation}\justifying\noindent\hspace{10mm} अत्र कबन्धः \textcolor{red}{तव प्राण\-रिरक्षया} इति प्रयुङ्क्ते। \textcolor{red}{प्राणस्य रक्षितुमिच्छा} इति \textcolor{red}{प्राण\-रिरक्षा}। किन्त्वत्र \textcolor{red}{रिरक्षा} इत्यपाणिनीयः। तथा हि \textcolor{red}{रक्षितुमिच्छा} इति विग्रहे \textcolor{red}{रक्ष्‌}\-धातोः (\textcolor{red}{रक्षँ पालने} धा॰पा॰~६५८) \textcolor{red}{धातोः कर्मणः समान\-कर्तृकादिच्छायां वा} (पा॰सू॰~३.१.७) इत्यनेन \textcolor{red}{सन्} प्रत्यय इटि षत्वे \textcolor{red}{सन्यङोः} (पा॰सू॰~६.१.९) इत्यनेन द्वित्वे \textcolor{red}{पूर्वोऽभ्यासः} (पा॰सू॰~६.१.४) इत्यनेनाभ्यास\-सञ्ज्ञायामभ्यास\-कार्ये \textcolor{red}{हलादिः शेषः} (पा॰सू॰~७.४.६०) इत्यनेन रकार\-भावे शिष्टे \textcolor{red}{सन्यतः} (पा॰सू॰~७.४.७९) इत्यनेनेकारे \textcolor{red}{अ प्रत्ययात्} (पा॰सू॰~३.३.१०२) इत्यनेन \textcolor{red}{अ}\-प्रत्यये टापि \textcolor{red}{रिरक्षिषा} इत्येव।\footnote{यथा \textcolor{red}{जगद्रिरक्षिषया} (भा॰पु॰~५.१५.६) इत्यत्र। \textcolor{red}{रक्षँ पालने} (धा॰पा॰~६५८)~\arrow रक्ष्~\arrow \textcolor{red}{धातोः कर्मणः समान\-कर्तृकादिच्छायां वा} (पा॰सू॰~३.१.७)~\arrow रक्ष्~सन्~\arrow रक्ष्~स~\arrow \textcolor{red}{आर्धधातुकस्येड्वलादेः} (पा॰सू॰~७.२.३५)~\arrow रक्ष्~इट्~स~\arrow रक्ष्~इ~स~\arrow \textcolor{red}{आदेशप्रत्यययोः} (पा॰सू॰~८.३.५९)~\arrow रक्ष्~इ~ष~\arrow रक्षिष~\arrow \textcolor{red}{सन्यङोः} (पा॰सू॰~६.१.९)~\arrow रक्ष्~रक्षिष~\arrow \textcolor{red}{हलादिः शेषः} (पा॰सू॰~७.४.६०)~\arrow र~रक्षिष~\arrow \textcolor{red}{सन्यतः} (पा॰सू॰~७.४.७९)~\arrow रि~रक्षिष~\arrow रिरक्षिष~\arrow \textcolor{red}{सनाद्यन्ता धातवः} (पा॰सू॰~३.१.३२)~\arrow धातु\-सञ्ज्ञा~\arrow \textcolor{red}{अ प्रत्ययात्} (पा॰सू॰~३.३.१०२)~\arrow रिरक्षिष~अ~\arrow \textcolor{red}{अतो लोपः} (पा॰सू॰~६.४.४८)~\arrow रिरक्षिष्~अ~\arrow रिरक्षिष~\arrow \textcolor{red}{अजाद्यतष्टाप्‌} (पा॰सू॰~४.१.४)~\arrow रिरक्षिष~टाप्~\arrow रिरक्षिष~आ~\arrow \textcolor{red}{अकः सवर्णे दीर्घः} (पा॰सू॰~६.१.१०१)~\arrow रिरक्षिषा।} अत्र \textcolor{red}{अ प्रत्ययात्} (पा॰सू॰~३.३.१०२) इत्यनेनाकारे पृषोदरादि\-त्वात्सन्प्रत्ययस्य लोपे\footnote{\textcolor{red}{पृषोदरादीनि यथोपदिष्टम्} (पा॰सू॰~६.३.१०९)। पृषोदरादित्वादिटोऽपि लोपो बोध्यः। विकल्पत्वात्पक्षे \textcolor{red}{रिरक्षिषा} इत्यपि यत्र न लोपकार्यौ।} टापि समासे तृतीयैक\-वचने \textcolor{red}{प्राणरिरक्षया}।\footnote{एवमेव \textcolor{red}{एष साक्षाद्धरेरंशो जातो लोकरिरक्षया} (भा॰पु॰~४.१५.६) इत्यत्रापि।} यद्वा \textcolor{red}{प्राणस्यारयः प्राणारयः क्षुत्पिपासादयस्तेभ्यो रक्षेति प्राणरिरक्षा तया} इति विग्रहे \textcolor{red}{प्राण}\-शब्दस्य \textcolor{red}{अरि}\-शब्देन समासे शकन्ध्वादित्वात्पर\-रूपे पुनः \textcolor{red}{प्राणरि}\-शब्दस्य \textcolor{red}{रक्षा}\-शब्देन सह समासे तृतीयैक\-वचने \textcolor{red}{प्राणरिरक्षया}।\footnote{एवमेव \textcolor{red}{पुरोऽवतस्थे कृष्णस्य पुत्रप्राणरिरक्षया} (भा॰पु॰~१०.६३.२०) इत्यत्रापि।}\end{sloppypar}
\section[आदृता]{आदृता}
\centering\textcolor{blue}{रामलक्ष्मणयोः सम्यक्पादौ प्रक्षाल्य भक्तितः।\nopagebreak\\
तज्जलेनाभिषिच्याङ्गमथार्घ्यादिभिरादृता॥}\nopagebreak\\
\raggedleft{–~अ॰रा॰~३.१०.७}\\
\begin{sloppypar}\hyphenrules{nohyphenation}\justifying\noindent\hspace{10mm} भक्तवत्सलः श्रीरामः शबरीमुद्दिधीर्षन् तया बहुमानितः। सा भगवतः श्रीचरणारविन्दं प्रक्षाल्यार्घ्यादिभिरादृता। \textcolor{red}{आदृतवती} इति प्रयोक्तव्ये \textcolor{red}{आदृता} इति प्रयुक्तम्। यद्यपि \textcolor{red}{आ}\-पूर्वकात् \textcolor{red}{दृ}\-धातोः (\textcolor{red}{दृङ् आदरे} धा॰पा॰~१४११) सकर्मकतया \textcolor{red}{तयोरेव कृत्य\-क्त\-खलर्थाः} (पा॰सू॰~३.४.७०) इति सूत्रेण \textcolor{red}{क्त}\-प्रत्यय\-विधानं कर्मण्येव पाणिन्यनुकूलं तथाऽपि \textcolor{red}{गत्यर्थाकर्मक\-श्लिष\-शीङ्स्थास\-वस\-जन\-रुह\-जी\-र्यतिभ्यश्च} (पा॰सू॰~३.४.७२) इत्यत्र \textcolor{red}{च}कारात्क्वचित्सकर्मकादपि। तेनात्र सकर्मक\-धातोः कर्तरि \textcolor{red}{आदृता}। यद्वा \textcolor{red}{आदर एव आदृतम्}। भावे \textcolor{red}{क्त}\-प्रत्ययः।\footnote{\textcolor{red}{नपुंसके भावे क्तः} (पा॰सू॰~३.३.११४) इत्यनेन।} \textcolor{red}{आदृतमस्त्यस्या इत्यादृता}। अर्शआद्यजन्तः प्रयोगः।\footnote{\textcolor{red}{अर्शआदिभ्योऽच्} (पा॰सू॰~५.२.१२७) इत्यनेन।} यद्वा कर्मणोऽविवक्षायामकर्मकत्वात्\footnote{\textcolor{red}{धातोरर्थान्तरे वृत्तेर्धात्वर्थेनोपसङ्ग्रहात्। प्रसिद्धेरविवक्षातः कर्मणोऽकर्मिका क्रिया॥} (वा॰प॰~३.७.८८)} \textcolor{red}{गत्यर्थाकर्मक\-श्लिष\-शीङ्स्थास\-वस\-जन\-रुह\-जीर्यतिभ्यश्च} (पा॰सू॰~३.४.७२) इत्यनेन कर्तरि \textcolor{red}{क्त}\-प्रत्यये टापि \textcolor{red}{आदृता} इति सम्यक्।\footnote{यथाऽऽह मनुः~– \textcolor{red}{सर्वेष्वेव व्रतेष्वेवं प्रायश्चित्तार्थमादृतः} (म॰स्मृ॰~११.२२५)। अत्र टीकाकाराः~– \textcolor{red}{आदृतो यत्नवान्} (म॰स्मृ॰ मे॰टी॰~११.२२५) \textcolor{red}{यत्नवान्} (म॰स्मृ॰ कु॰टी॰~११.२२५) \textcolor{red}{आदृतः श्रद्धालुः} (म॰स्मृ॰ राघ॰टी॰~११.२२५)। एवमेव भागवते~– \textcolor{red}{यावद्‍ब्रह्म विजानीयान्मामेव गुरुमादृतः} (भा॰पु॰~११.१८.३९)। अत्र वीरराघव\-टीका~– \textcolor{red}{तत्राऽदृत आदरयुक्तः} (भा॰पु॰ वी॰रा॰व्या॰~११.१८.३९)। रघुवंशेऽपि कालिदासः~– \textcolor{red}{इत्यादृतेन कथितौ रघुनन्दनेन व्युत्क्रम्य लक्ष्मणमुभौ भरतो ववन्दे} (र॰वं॰~१३.७२)। अत्र सञ्जीविन्यां मल्लिनाथः~– \textcolor{red}{इत्यादृतेन आदरवता। कर्तरि क्तः} (र॰वं॰ स॰व्या॰~१३.७२)। दर्पण\-टीकाकारो हेमाद्रिस्तु \textcolor{red}{इत्यादरेण कथितौ रघुनन्दनेन} (र॰वं॰ द॰टी॰~१३.७२) इति पाठं स्वीचक्रे। पञ्चतन्त्रे विष्णुशर्मा च~– \textcolor{red}{आत्मानमादृतो रक्षेत्प्रमादाद्धि विनश्यति} (प॰त॰~३.२२९)। अत्राभिनव\-राज\-लक्ष्मी टीका~– \textcolor{red}{आदृतः सावधानः सन्} (प॰त॰ अ॰टी॰~३.२२९)।}\end{sloppypar}
\section[विरागित्वम्]{विरागित्वम्}
\centering\textcolor{blue}{मद्भक्तेष्वधिका पूजा सर्वभूतेषु मन्मतिः।\nopagebreak\\
बाह्यार्थेषु विरागित्वं शमादिसहितं तथा॥}\nopagebreak\\
\raggedleft{–~अ॰रा॰~३.१०.२६}\\
\begin{sloppypar}\hyphenrules{nohyphenation}\justifying\noindent\hspace{10mm} \textcolor{red}{विगतो रागो यस्य स विरागस्तस्य भावो विरागत्वम्} इति बहुव्रीहि\-जन्य\-\textcolor{red}{विराग}\-शब्दात् \textcolor{red}{त्व}\-प्रत्यये\footnote{\textcolor{red}{तस्य भावस्त्वतलौ} (पा॰सू॰~५.१.११९) इत्यनेन।} यद्यप्यर्थ\-सिद्धिः किं \textcolor{red}{न कर्मधारयान्मत्वर्थीयो बहुव्रीहिश्चेत्तदर्थ\-प्रतिपत्तिकरः}\footnote{मूलं मृग्यम्।} इत्यस्य नियमस्योल्लङ्घनेन तथाऽपि \textcolor{red}{विगतो रागो विरागः} इत्यत्र \textcolor{red}{प्रगत आचार्यः प्राचार्यः} (भा॰पा॰सू॰~२.२.१८)\footnote{\textcolor{red}{प्रगत आचार्यः प्राचार्यः} इत्यत्र \textcolor{red}{प्रादयः क्तार्थे} (वा॰~२.२.१८) इत्यनेन \textcolor{red}{प्रादयो गताद्यर्थे प्रथमया} (वा॰~२.२.१८) इत्यनेन च समासो भाष्ये दर्शितः। उभे वार्तिके \textcolor{red}{कु\-गति\-प्रादयः} (पा॰सू॰~२.२.१८) इति सूत्रे पठिते।} इतिवत् \textcolor{red}{कु\-गति\-प्रादयः} (पा॰सू॰~२.२.१८) इत्यनेन समासे \textcolor{red}{विरागः प्रशस्तो नित्यो वाऽस्त्यस्मिन्} इति बहुव्रीह्यलब्ध\-विशिष्टार्थं बोधयितुं कर्मधारयादिनिः।\footnote{\textcolor{red}{भूम\-निन्दा\-प्रशंसासु नित्ययोगेऽति\-शायने। सम्बन्धेऽस्ति\-विवक्षायां भवन्ति मतुबादयः॥} (भा॰पा॰सू॰~५.२.९४)।} ततश्च \textcolor{red}{तस्य भावस्त्वतलौ} (पा॰सू॰~५.१.११९) इत्यनेन \textcolor{red}{त्व}\-प्रत्यये विभक्ति\-कार्ये \textcolor{red}{विरागित्वम्}।\end{sloppypar}
\vspace{2mm}
\centering ॥ इत्यरण्यकाण्डीयप्रयोगाणां विमर्शः ॥\nopagebreak\\
\vspace{4mm}
\pdfbookmark[2]{किष्किन्धाकाण्डम्}{Chap2Part2Kanda4}
\phantomsection
\addtocontents{toc}{\protect\setcounter{tocdepth}{2}}\addtocontents{toc}{\protect\setcounter{tocdepth}{2}}
\addcontentsline{toc}{subsection}{किष्किन्धाकाण्डीयप्रयोगाणां विमर्शः}
\addtocontents{toc}{\protect\setcounter{tocdepth}{0}}
\centering ॥ अथ किष्किन्धाकाण्डीयप्रयोगाणां विमर्शः ॥\nopagebreak\\
\section[दाशरथो रामः]{दाशरथो रामः}
\centering\textcolor{blue}{अहं दाशरथो रामस्त्वयं मे लक्ष्मणोऽनुजः।\nopagebreak\\
सीतया भार्यया सार्धं पितुर्वचनगौरवात्॥}\nopagebreak\\
\raggedleft{–~अ॰रा॰~४.१.१९}\\
\begin{sloppypar}\hyphenrules{nohyphenation}\justifying\noindent\hspace{10mm} सीतामन्वीक्षमाणौ धनुर्बाणधरौ श्रीराम\-लक्ष्मणौ विलोक्य
तदाशङ्कया
सुग्रीवेण प्रेषितं वटु\-वेष\-धारिणं मारुतिं दृष्ट्वा तेन नामादि\-परिचयं पृष्टः श्रीरामभद्रः सङ्क्षिप्य परिचयं प्रस्तौति \textcolor{red}{दाशरथो रामः} इति। अत्र \textcolor{red}{दशरथस्यापत्यं पुमान् दाशरथिः} इति विग्रहे \textcolor{red}{तस्यापत्यम्} (पा॰सू॰~४.१.९२) इति सूत्रार्थानुसारमपत्यार्थे षष्ठ्यन्त\-दशरथ\-प्रातिपदिकात् \textcolor{red}{अत इञ्} (पा॰सू॰~४.१.९५) इत्यनेन \textcolor{red}{इञ्} प्रत्यये \textcolor{red}{दाशरथिः} इत्येव प्रसिद्ध\-प्रयोगः \textcolor{red}{दाशरथः} इति कथमकारान्तात् \textcolor{red}{इञ्} प्रत्ययस्य दुर्निवारत्वात्। श्रीरामो वस्तुतो दशरथस्य नापत्यं तत्क्षेत्र\-जन्य\-व्यवहाराद्दाशरथिरित्युपचर्यते। अतो हनुमतः समक्षमपत्य\-रूपमर्थं न कथयन्नाह \textcolor{red}{दाशरथः}। \textcolor{red}{दशरथस्यायं दाशरथः} इति विग्रहे \textcolor{red}{तस्येदम्} (पा॰सू॰~४.३.१२०) इत्यनेन \textcolor{red}{अण्‌}\-प्रत्यये भत्वादकार\-लोपे\footnote{\textcolor{red}{यचि भम्} (पा॰सू॰~१.४.१८) इत्यनेन भत्वम्। \textcolor{red}{यस्येति च} (पा॰सू॰~६.४.१४८) इत्यनेनाकार\-लोपः।} \textcolor{red}{दाशरथः}। दशरथस्य स्वेन सह केवलं पाल्य\-पालक\-भाव\-रूप\-सम्बन्धस्यैव विवक्षा भक्त\-प्रवर\-हनुमतः सम्मुखे राघवेन्द्रस्य। यद्वा \textcolor{red}{दशरथादागतो दाशरथः} इति विग्रहे पञ्चम्यन्त\-दशरथ\-शब्दात् \textcolor{red}{तत आगतः} (पा॰सू॰~४.३.७४) इति \textcolor{red}{अण्‌}\-प्रत्ययः। लोपादि\-कार्ये \textcolor{red}{तद्धितेष्वचामादेः} (पा॰सू॰~७.२.११७) इत्यनेन वृद्धौ \textcolor{red}{दाशरथः}। अर्थादण् पार्थक्ये। यतो हि दशरथस्य सकाशादहमागतः। अथवाऽपि \textcolor{red}{विभाषा गुणेऽस्त्रियाम्} (पा॰सू॰~२.३.२५) इत्यनेन पञ्चमी। ततोऽण्। अनेन \textcolor{red}{प्रदीयतां दाशरथाय मैथिली} (वा॰रा॰~६.१४.३) इति वाल्मीकीय\-रामायण\-प्रयोगोऽपि व्याख्यातः।\end{sloppypar}
\section[अभिषेचनम्]{अभिषेचनम्}
\centering\textcolor{blue}{तच्छ्रुत्वा दुःखिताः सर्वे मामनिच्छन्तमप्युत।\nopagebreak\\
राज्येऽभिषेचनं चक्रुः सर्वे वानरमन्त्रिणः॥}\nopagebreak\\
\raggedleft{–~अ॰रा॰~४.१.५३}\\
\begin{sloppypar}\hyphenrules{nohyphenation}\justifying\noindent\hspace{10mm} अत्र \textcolor{red}{मामभिषेचनं चक्रुः} इति सामानाधिकरण्य\-दर्शनात् \textcolor{red}{अभिषेचनम्} इत्यत्र प्रत्यय\-सन्देह\-परं भवति \textcolor{red}{माम्}। तथा च भावे \textcolor{red}{ल्युट्}\footnote{\textcolor{red}{ल्युट् च} (पा॰सू॰~३.३.११५) इत्यनेन।} चेत्कृद्योगे \textcolor{red}{कर्तृ\-कर्मणोः कृति} (पा॰सू॰~२.३.६५) इति सूत्रेण षष्ठ्यां \textcolor{red}{ममाभिषेचनम्} इति स्यात्। द्वितीयायां प्रत्यय\-जिज्ञासा तदवस्थेति चेत्।
\textcolor{red}{अभिषिच्यत इत्यभिषेचनः} इति कर्मणि \textcolor{red}{कृत्य\-ल्युटो बहुलम्} (पा॰सू॰~३.३.११३) इत्यनेन ल्युट्। यद्वा \textcolor{red}{सेचनम्} इति भाव\-ल्युडन्तम्। \textcolor{red}{अभितः सेचनं यस्य सोऽभिषेचनस्तमभिषेचनम्} इति \textcolor{red}{प्रादिभ्यो धातुजस्य वाच्यो वा चोत्तरपद\-लोपश्च} (वा॰~२.२.२२) इत्यनेन समासे \textcolor{red}{अभिषेचनम्} इति सिद्धम्।\end{sloppypar}
\section[बलवतां बली]{बलवतां बली}
\centering\textcolor{blue}{सुग्रीवोऽप्याह राजेन्द्र वाली बलवतां बली।\nopagebreak\\
कथं हनिष्यति भवान्देवैरपि दुरासदम्॥}\nopagebreak\\
\raggedleft{–~अ॰रा॰~४.१.६०}\\
\begin{sloppypar}\hyphenrules{nohyphenation}\justifying\noindent\hspace{10mm} अत्र बहूनां निर्धारणतया \textcolor{red}{तमप्} प्रत्ययः \textcolor{red}{इष्ठन्} प्रत्ययो वा प्राप्तः\footnote{\textcolor{red}{अतिशायने तमबिष्ठनौ} (पा॰सू॰~५.३.५५) इत्यनेन।} किन्त्वविवक्षणतया न।\end{sloppypar}
\section[चेतनम्]{चेतनम्}
\centering\textcolor{blue}{तदा मुहूर्त्तं निःसंज्ञो भूत्वा चेतनमाप सः।\nopagebreak\\
ततो वाली ददर्शाग्रे रामं राजीवलोचनम्।\nopagebreak\\
धनुरालम्ब्य वामेन हस्तेनान्येन सायकम्॥}\nopagebreak\\
\raggedleft{–~अ॰रा॰~४.२.४८}\\
\begin{sloppypar}\hyphenrules{nohyphenation}\justifying\noindent\hspace{10mm} श्रीराम\-बाण\-भिन्न\-शरीरो भूमौ पतितो वाली चेतनां प्राप्तत्वान्। अत्र \textcolor{red}{चेतनम्} इति प्रयुक्तम्। \textcolor{red}{चितीँ सञ्ज्ञाने} (धा॰पा॰~३९) इति धातोः स्वार्थे णिचि\footnote{\textcolor{red}{चितँ सञ्चेतने} (धा॰पा॰~१६७३) इत्यस्मात्स्वार्थे णिचि वा।} ततश्च \textcolor{red}{चेत्यत इति चेतना}\footnote{चेति~यक्~त इति स्थिते \textcolor{red}{णेरनिटि} (पा॰सू॰~६.४.५१) इत्यनेन णिलोपे चेत्यते।} इति विग्रहे \textcolor{red}{ण्यास\-श्रन्थो युच्} (पा॰सू॰~३.३.१०७) इत्यनेन भावे युचि प्रत्ययेऽनादेशे स्त्रीत्वाट्टाप्प्रत्यये \textcolor{red}{चेतना}।\footnote{यद्वा \textcolor{red}{चितँ सञ्चेतने} (धा॰पा॰~१६७३) इत्यतः \textcolor{red}{सत्याप\-पाश\-रूप\-वीणा\-तूल\-श्लोक\-सेना\-लोम\-त्वच\-वर्म\-वर्ण\-चूर्ण\-चुरादिभ्यो णिच्} (पा॰सू॰~३.१.२५) इत्यनेन स्वार्थे णिचि \textcolor{red}{ण्यास\-श्रन्थो युच्} (पा॰सू॰~३.३.१०७) इत्यनेन भावे युच्यनादेशे टापि \textcolor{red}{चेतना}।} \textcolor{red}{चेतनम्} इत्यत्र हि शुद्धात् \textcolor{red}{चेतति} इत्यस्माद्भावे ल्युट्।\footnote{\textcolor{red}{ल्युट् च} (पा॰सू॰~३.३.११५) इत्यनेन।} \textcolor{red}{चेतनम्}। \end{sloppypar}
\section[भ्राजद्वनमालाविभूषितम्]{भ्राजद्वनमालाविभूषितम्}
\centering\textcolor{blue}{बिभ्राणं चीरवसनं जटामुकुटधारिणम्।\nopagebreak\\
विशालवक्षसं भ्राजद्वनमालाविभूषितम्॥}\nopagebreak\\
\raggedleft{–~अ॰रा॰~४.२.४९}\\
\begin{sloppypar}\hyphenrules{nohyphenation}\justifying\noindent\hspace{10mm} भूमिपतितो वाली समर\-धीर\-रघु\-वीरस्य भुवन\-मोहन\-सौन्दर्यं लोचनातिथी\-करोति यच्छ्रीरामो वल्कल\-धरो विविध\-भूषण\-भूषितः। तत्र \textcolor{red}{भ्राजद्वनमाला\-विभूषितम्} इति शब्द\-घटिते \textcolor{red}{भ्राजत्} इत्यत्र \textcolor{red}{शतृ}\-प्रयोगोऽनुचितः।\footnote{\textcolor{red}{भ्राजृँ दीप्तौ} (पा॰सू॰~१८१) \textcolor{red}{टुभ्राजृँ दीप्तौ} (धा॰पा॰~८३३) इत्यनयोरात्मने\-पदीयत्वाच्छानचि \textcolor{red}{भ्राजमान\-वनमाला\-विभूषितम्} इति वक्तव्यमिति भावः।} तथाऽप्यस्याऽत्मने\-पदीयत्वं त्वौप\-चारिकमेव। अनुदात्तेत्त्व\-लक्षणस्याऽत्मने\-पदस्यानित्यत्वात्।\footnote{\textcolor{red}{अनुदात्तेत्त्व\-लक्षणमात्मने\-पदमनित्यम्} (प॰शे॰~९३.४)।} \textcolor{red}{भ्राजन्ती चासौ वनमाला चेति भ्राजद्वनमाला तया भूषितम्} इति।\end{sloppypar}
\section[तिरोभूत्वा]{तिरोभूत्वा}
\centering\textcolor{blue}{राजधर्ममविज्ञाय गर्हितं कर्म ते कृतम्।\nopagebreak\\
वृक्षखण्डे तिरोभूत्वा त्यजता मयि सायकम्॥}\nopagebreak\\
\raggedleft{–~अ॰रा॰~४.२.५२}\\
\begin{sloppypar}\hyphenrules{nohyphenation}\justifying\noindent\hspace{10mm} वाली श्रीरामं भर्त्सयन्नाह यन्मां गुप्तवेषो हतवान्। अत्र \textcolor{red}{तिरोभूत्वा} इति प्रयुक्तम्। \textcolor{red}{तिरस्‌}\-शब्दस्य हि \textcolor{red}{भू}\-शब्देन समासे क्त्वो ल्यपि\footnote{\textcolor{red}{समासेऽनञ्पूर्वे क्त्वो ल्यप्‌} (पा॰सू॰~७.१.३७) इत्यनेन।} \textcolor{red}{तिरोभूय} इत्येव। असति समासे \textcolor{red}{तिरो} इति पृथक्पदम्। कथं न \textcolor{red}{तिरस्} तर्हि। संहिताया विवक्षणात्। अतः समासाभावे \textcolor{red}{तिरो भूत्वा} इति न दोषः।\end{sloppypar}
\section[वानरम्]{वानरम्}
\centering\textcolor{blue}{ वानरं व्याधवद्धत्वा धर्मं कं लप्स्यसे वद।\nopagebreak\\
अभक्ष्यं वानरं मांसं हत्वा मां किं करिष्यसि॥}\nopagebreak\\
\raggedleft{–~अ॰रा॰~४.२.५८}\\
\begin{sloppypar}\hyphenrules{nohyphenation}\justifying\noindent\hspace{10mm} वाली कथयति यत् \textcolor{red}{वानरं मांसं} विगर्हितम्। अत्र \textcolor{red}{वानरे भवमिति वानरीयम्}। \textcolor{red}{वृद्धाच्छः} (पा॰सू॰~४.२.११४) इत्यनेन \textcolor{red}{छ}\-प्रत्यये \textcolor{red}{वानरीयम्} इति पाणिनीयम्। \textcolor{red}{वानरम्} इति कथम्। \textcolor{red}{नीलो घटः} इतिवत् \textcolor{red}{वानरं मांसम्}। \textcolor{red}{शाब्द\-बोधे चैक\-पदार्थेऽपर\-पदार्थस्य संसर्गः संसर्ग\-मर्यादया भासते} (व्यु॰वा॰ का॰प्र॰) इति व्युत्पत्ति\-वाद\-प्रयोगात् \textcolor{red}{वानराभिन्नं मांसम्}। यद्वा \textcolor{red}{वानरस्येदं वानरम्} इति विग्रहे \textcolor{red}{तस्येदम्} (पा॰सू॰~४.३.१२०) इति \textcolor{red}{अण्‌}\-प्रत्ययः।\end{sloppypar}
\section[बहु भाषन्तम्]{बहु भाषन्तम्}
\centering\textcolor{blue}{इत्येवं बहु भाषन्तं वालिनं राघवोऽब्रवीत्।\nopagebreak\\
धर्मस्य गोप्ता लोकेऽस्मिंश्चरामि सशरासनः॥}\nopagebreak\\
\raggedleft{–~अ॰रा॰~४.२.५९}\\
\begin{sloppypar}\hyphenrules{nohyphenation}\justifying\noindent\hspace{10mm} पतितं वालिनं बहु भाषमाणं श्रीरामभद्रः प्रतिवक्ति। अत्र \textcolor{red}{भाषमाणम्} इत्यर्थे \textcolor{red}{भाषन्तम्} इति प्रयुक्तम्। यतो हि \textcolor{red}{भाष्} धातुः (\textcolor{red}{भाषँ व्यक्तायां वाचि} धा॰पा॰~६१२) आत्मनेपदीयस्तथा च \textcolor{red}{भाषत इति भाषमाणस्तं भाषमाणम्} इति शानचा भवितव्यमासीत्। किन्तु \textcolor{red}{भाषत इति भाषः} पचादित्वादच्।\footnote{\textcolor{red}{नन्दि\-ग्रहि\-पचादिभ्यो ल्युणिन्यचः} (पा॰सू॰~३.१.१३४) इत्यनेन।} \textcolor{red}{भाष इवाऽचरति}
इति क्विपि लटि तिपि शपि पररूपे \textcolor{red}{भाषति}।\footnote{भाष~\arrow \textcolor{red}{सर्वप्राति\-पदिकेभ्य आचारे क्विब्वा वक्तव्यः} (वा॰~३.१.११)~\arrow भाष~क्विँप्~\arrow भाष~व्~\arrow \textcolor{red}{वेरपृक्तस्य} (पा॰सू॰~६.१.६७)~\arrow भाष~\arrow \textcolor{red}{सनाद्यन्ता धातवः} (पा॰सू॰~३.१.३२)~\arrow धातुसञ्ज्ञा~\arrow \textcolor{red}{शेषात्कर्तरि परस्मैपदम्} (पा॰सू॰~१.३.७८)~\arrow \textcolor{red}{वर्तमाने लट्} (पा॰सू॰~३.२.१२३)~\arrow भाष~लट्~\arrow भाष~तिप्~\arrow भाष~ति~\arrow \textcolor{red}{कर्तरि शप्‌} (पा॰सू॰~३.१.६८)~\arrow भाष~शप्~ति~\arrow भाष~अ~ति~\arrow \textcolor{red}{अतो गुणे} (पा॰सू॰~६.१.९७)~\arrow भाष~ति~\arrow भाषति।} \textcolor{red}{भाषतीति भाषन् तं भाषन्तम्} इत्याचार\-क्विबन्तात् \textcolor{red}{शतृ}\-प्रत्यये न दोषः।\footnote{भाष~\arrow धातुसञ्ज्ञा (पूर्ववत्)~\arrow \textcolor{red}{शेषात्कर्तरि परस्मैपदम्} (पा॰सू॰~१.३.७८)~\arrow \textcolor{red}{वर्तमाने लट्} (पा॰सू॰~३.२.१२३)~\arrow भाष~लट्~\arrow \textcolor{red}{लटः शतृशानचावप्रथमा\-समानाधिकरणे} (पा॰सू॰~३.२.१२४)~\arrow भाष~शतृँ~\arrow भाष~अत्~\arrow \textcolor{red}{अतो गुणे} (पा॰सू॰~६.१.९७)~\arrow भाषत्~\arrow \textcolor{red}{कृत्तद्धित\-समासाश्च} (पा॰सू॰~१.२.४६)~\arrow प्रातिपादिक\-सञ्ज्ञा~\arrow विभक्ति\-कार्यम्~\arrow भाषत्~सुँ~\arrow भाषत्~स्~\arrow \textcolor{red}{उगिदचां सर्वनामस्थानेऽधातोः} (पा॰सू॰~७.१.७०)~\arrow \textcolor{red}{मिदचोऽन्त्यात्परः} (पा॰सू॰~१.१.४७)~\arrow भाष~नुँम्~त्~स्~\arrow भाष~न्~त्~स्~\arrow \textcolor{red}{हल्ङ्याब्भ्यो दीर्घात्सुतिस्यपृक्तं हल्} (पा॰सू॰~६.१.६८)~\arrow भाष~न्~त्~\arrow \textcolor{red}{संयोगान्तस्य लोपः} (पा॰सू॰~८.२.२३)~\arrow भाष~न्~त्~स्~\arrow भाषन्। भाषत्~\arrow प्रातिपदिक\-सञ्ज्ञा (पूर्ववत्)~\arrow विभक्तिकार्यम्~\arrow भाषत्~अम्~\arrow \textcolor{red}{उगिदचां सर्वनामस्थानेऽधातोः} (पा॰सू॰~७.१.७०)~\arrow \textcolor{red}{मिदचोऽन्त्यात्परः} (पा॰सू॰~१.१.४७)~\arrow भाष~नुँम्~त्~अम्~\arrow भाष~न्~त्~अम्~\arrow भाषन्तम्।} \textcolor{red}{वालिर्भाषते नह्यपि तु भाष इवाऽचरति} इति परस्मैपदस्य तात्पर्यम्।\end{sloppypar}
\section[दापितम्]{दापितम्}
\centering\textcolor{blue}{सुग्रीवं त्वं सुखं राज्यं दापितं वालिघातिना।\nopagebreak\\
रामेण रुमया सार्धं भुङ्क्ष्व सापत्नवर्जितम्॥}\nopagebreak\\
\raggedleft{–~अ॰रा॰~४.३.११}\\
\begin{sloppypar}\hyphenrules{nohyphenation}\justifying\noindent\hspace{10mm} अत्र \textcolor{red}{दत्तम्} इति न कथयित्वा \textcolor{red}{दापितम्} इति प्रयुक्तम्। स्वार्थे णिचि \textcolor{red}{अर्ति\-ह्री\-व्ली\-री\-क्नूयी\-क्ष्माय्यातां पुङ्णौ} (पा॰सू॰~७.३.३६) इत्यनेन पुकि \textcolor{red}{क्त}\-प्रत्यये \textcolor{red}{दापितम्}। अर्थात् \textcolor{red}{रामेण राज्यमदाप्यत}। यद्वा \textcolor{red}{वाल्यददाद्रामः प्रेरयत्} इत्यर्थे \textcolor{red}{वालिना राज्यमदाप्यत} कर्म\-वाच्ये \textcolor{red}{वालि\-घातिना वालिना राज्यमदाप्यत} इत्यस्मिन्नर्थे \textcolor{red}{क्त}\-प्रत्ययः। अत्र \textcolor{red}{हेतुमति च} (पा॰सू॰~३.१.२६) इत्यनेन णिच्।\end{sloppypar}
\section[कुर्वन्ती]{कुर्वन्ती}
\centering\textcolor{blue}{ध्यात्वा मद्रूपमनिशमालोचय मयोदितम्।\nopagebreak\\
प्रवाहपतितं कार्यं कुर्वन्त्यपि न लिप्यसे॥}\nopagebreak\\
\raggedleft{–~अ॰रा॰~४.३.३५}\\
\begin{sloppypar}\hyphenrules{nohyphenation}\justifying\noindent\hspace{10mm} श्रीरामस्तारां प्रति कथयति \textcolor{red}{कार्यं कुर्वत्यपि मत्कृपया न लिप्ता भविष्यसि}। अत्र \textcolor{red}{नुम्} अपाणिनीय इव। \textcolor{red}{करोतीति कुर्वती} इति तनादित्वाच्छबभावे\footnote{शाभावे श्यनभावे चेति बोध्यम्।} \textcolor{red}{नुम्} कथमिति चेत्। \textcolor{red}{गण\-कार्यमनित्यम्} (प॰शे॰~९३.३) इत्यनेन शपि \textcolor{red}{नुम्} सङ्गत एव। यद्वा सौत्र\-धातव इवात्राप्याकृति\-गणतया क्रियार्थः \textcolor{red}{कुर्वँ धातुः} भ्वादिगणे पठ्यतां\footnote{\textcolor{red}{बहुलमेतन्निदर्शनम्} (धा॰पा॰ ग॰सू॰~१९३८) \textcolor{red}{आकृतिगणोऽयम्} (धा॰पा॰ ग॰सू॰~१९९२) \textcolor{red}{भूवादिष्वेतदन्तेषु दशगणीषु धातूनां पाठो निदर्शनाय तेन स्तम्भुप्रभृतयः सौत्राश्चुलुम्पादयो वाक्यकारीयाः प्रयोगसिद्धा विक्लवत्यादयश्च} (मा॰धा॰वृ॰~१०.३२८) इत्यनुसारमाकृति\-गणत्वाद्भ्वादि\-गण ऊह्योऽयं धातुः।} तथा च \textcolor{red}{कुर्वतीति कुर्वन्ती} इति \textcolor{red}{शतृ}\-प्रत्यये \textcolor{red}{नुम्} साधुः।\end{sloppypar}
\section[असहन्]{असहन्}
\centering\textcolor{blue}{रामस्तु पर्वतस्याग्रे मणिसानौ निशामुखे।\nopagebreak\\
सीताविरहजं शोकमसहन्निदमब्रवीत्॥}\nopagebreak\\
\raggedleft{–~अ॰रा॰~४.५.१}\\
\begin{sloppypar}\hyphenrules{nohyphenation}\justifying\noindent\hspace{10mm} \textcolor{red}{षहँ मर्षणे} (धा॰पा॰~८५२, १८०९) इत्यात्मनेपदीय\-धातुः। तत्र शानचि \textcolor{red}{सहमानः} इति पाणिनीयः। \textcolor{red}{असहन्} इति तु \textcolor{red}{सहत इति सहो न सह इत्यसहः} पचादित्वादच्\footnote{\textcolor{red}{नन्दि\-ग्रहि\-पचादिभ्यो ल्युणिन्यचः} (पा॰सू॰~३.१.१३४) इत्यनेन।} नञ्समासश्च।\footnote{\textcolor{red}{नलोपो नञः} (पा॰सू॰~६.३.७३) इत्यनेन नलोपः।} \textcolor{red}{असह इवाऽचरतीत्यसहति}।\footnote{असह~\arrow \textcolor{red}{सर्वप्राति\-पदिकेभ्य आचारे क्विब्वा वक्तव्यः} (वा॰~३.१.११)~\arrow असह~क्विँप्~\arrow असह~व्~\arrow \textcolor{red}{वेरपृक्तस्य} (पा॰सू॰~६.१.६७)~\arrow असह~\arrow \textcolor{red}{सनाद्यन्ता धातवः} (पा॰सू॰~३.१.३२)~\arrow धातुसञ्ज्ञा~\arrow \textcolor{red}{शेषात्कर्तरि परस्मैपदम्} (पा॰सू॰~१.३.७८)~\arrow \textcolor{red}{वर्तमाने लट्} (पा॰सू॰~३.२.१२३)~\arrow असह~लट्~\arrow असह~तिप्~\arrow असह~ति~\arrow \textcolor{red}{कर्तरि शप्‌} (पा॰सू॰~३.१.६८)~\arrow असह~शप्~ति~\arrow असह~अ~ति~\arrow \textcolor{red}{अतो गुणे} (पा॰सू॰~६.१.९७)~\arrow असह~ति~\arrow असहति।} \textcolor{red}{असहतीत्यसहन्}। आचार\-क्विबन्ताच्छतृ\-प्रत्ययः।\footnote{असह~\arrow धातुसञ्ज्ञा (पूर्ववत्)~\arrow \textcolor{red}{शेषात्कर्तरि परस्मैपदम्} (पा॰सू॰~१.३.७८)~\arrow \textcolor{red}{वर्तमाने लट्} (पा॰सू॰~३.२.१२३)~\arrow असह~लट्~\arrow \textcolor{red}{लटः शतृशानचावप्रथमा\-समानाधिकरणे} (पा॰सू॰~३.२.१२४)~\arrow असह~शतृँ~\arrow असह~अत्~\arrow \textcolor{red}{अतो गुणे} (पा॰सू॰~६.१.९७)~\arrow असहत्~\arrow \textcolor{red}{कृत्तद्धित\-समासाश्च} (पा॰सू॰~१.२.४६)~\arrow प्रातिपादिक\-सञ्ज्ञा~\arrow विभक्ति\-कार्यम्~\arrow असहत्~सुँ~\arrow असहत्~स्~\arrow \textcolor{red}{उगिदचां सर्वनामस्थानेऽधातोः} (पा॰सू॰~७.१.७०)~\arrow \textcolor{red}{मिदचोऽन्त्यात्परः} (पा॰सू॰~१.१.४७)~\arrow असह~नुँम्~त्~स्~\arrow असह~न्~त्~स्~\arrow \textcolor{red}{हल्ङ्याब्भ्यो दीर्घात्सुतिस्यपृक्तं हल्} (पा॰सू॰~६.१.६८)~\arrow असह~न्~त्~\arrow \textcolor{red}{संयोगान्तस्य लोपः} (पा॰सू॰~८.२.२३)~\arrow असह~न्~त्~स्~\arrow असहन्।
} \textcolor{red}{असहन\-शील\-समानमाचरणं करोति}। वस्तुतस्तु तस्य क्व वियोग इत्येवाऽचार\-क्विबन्ताच्छतृ\-प्रत्ययस्याऽध्यात्मिकं तात्पर्यं प्रतिभाति।\end{sloppypar}
\section[विस्मृतः]{विस्मृतः}
\centering\textcolor{blue}{कृतघ्नवत्त्वया नूनं विस्मृतः प्रतिभाति मे।\nopagebreak\\
त्वत्कृते निहितो वाली वीरस्त्रैलोक्यसम्मतः॥}\nopagebreak\\
\raggedleft{–~अ॰रा॰~४.४.४५}\\
\centering\textcolor{blue}{रामकार्यार्थमनिशं जागर्ति न तु विस्मृतः।\nopagebreak\\
आगताः परितः पश्य वानराः कोटिशः प्रभो॥}\nopagebreak\\
\raggedleft{–~अ॰रा॰~४.५.५५}\\
\begin{sloppypar}\hyphenrules{nohyphenation}\justifying\noindent\hspace{10mm} सकर्मक\-स्मृ\-धातोः (\textcolor{red}{स्मृ आध्याने} धा॰पा॰~८०७) \textcolor{red}{वि}\-उपसर्ग\-पूर्वकात्कर्तरि \textcolor{red}{क्तवतु}\-प्रत्यये \textcolor{red}{विस्मृतवान्} इति पाणिनीयः। किन्तु \textcolor{red}{विस्मरणं विस्मृतम्} इति भाव\-क्तान्त\-\textcolor{red}{विस्मृत}\-शब्दात्\footnote{\textcolor{red}{नपुंसके भावे क्तः} (पा॰सू॰~३.३.११४) इत्यनेन भाव\-क्तान्त\-शब्दः।} तदस्त्यस्येत्यर्शआद्यजन्तात्\footnote{\textcolor{red}{अर्शआदिभ्योऽच्} पा॰सू॰~५.२.१२७) इत्यनेन।} \textcolor{red}{विस्मृतः} अपि पाणिनीयं कर्तृ\-विशेषणं सदपि।\footnote{\textcolor{red}{विस्मृतमस्त्यस्येति विस्मृतः} इति भावः। यद्वा कर्मणोऽविवक्षायामकर्मकत्वात् \textcolor{red}{गत्यर्थाकर्मक\-श्लिष\-शीङ्स्थास\-वस\-जन\-रुह\-जीर्यतिभ्यश्च} (पा॰सू॰~३.४.७२) इत्यनेन कर्तरि क्तः।}\end{sloppypar}
\section[गृह्य]{गृह्य}
\centering\textcolor{blue}{इत्युक्त्वा लक्ष्मणं भक्त्या करे गृह्य स मारुतिः।\nopagebreak\\
आनयामास नगरमध्याद्राजगृहं प्रति॥}\nopagebreak\\
\raggedleft{–~अ॰रा॰~४.५.३९}\\
\begin{sloppypar}\hyphenrules{nohyphenation}\justifying\noindent\hspace{10mm} अत्रोपसर्गं विना कथं ल्यबिति चेत्।
\textcolor{red}{करे} इत्यस्य साक्षाद्गणे पाठात् \textcolor{red}{साक्षात्प्रभृतीनि च} (पा॰सू॰~१.४.७४) इत्यनेन गतिसञ्ज्ञायां समासे ल्यबादेशे न दोषः।\footnote{\textcolor{red}{साक्षात्प्रभृतीनि च} (पा॰सू॰~१.४.७४) इत्यत्र \textcolor{red}{क्वचिदेक\-देशोऽप्यनुवर्तते} (प॰शे॰~१८) इति परिभाषया \textcolor{red}{कृञि} (पा॰सू॰~१.४.७२) इत्यस्य मण्डूक\-प्लुत्या निवृत्तौ \textcolor{red}{विभाषा} (पा॰सू॰~१.४.७२) इत्यस्यानुवृत्तौ कृञभावेऽपि साक्षात्प्रभृतीनां गति\-सञ्ज्ञा क्वाचित्का। यद्वा चकार\-ग्रहणात्कृञभावेऽपि गति\-सञ्ज्ञा क्वाचित्का। सत्यां गतिसञ्ज्ञायां \textcolor{red}{कुगतिप्रादयः} (पा॰सू॰~२.२.१८) इत्यनेन समास इति भावः।}
यद्वा \textcolor{red}{प्र}\-उपसर्ग आसीत्तस्य \textcolor{red}{विनाऽपि प्रत्ययं पूर्वोत्तर\-पद\-लोपो वक्तव्यः} (वा॰~५.३.८३) इति वार्त्तिकेन लोपः।\end{sloppypar}
\section[मारुतिः]{मारुतिः}
\centering\textcolor{blue}{इत्युक्त्वा लक्ष्मणं भक्त्या करे गृह्य स मारुतिः।\nopagebreak\\
आनयामास नगरमध्याद्राजगृहं प्रति॥}\nopagebreak\\
\raggedleft{–~अ॰रा॰~४.५.३९}\\
\begin{sloppypar}\hyphenrules{nohyphenation}\justifying\noindent\hspace{10mm} यद्यपि \textcolor{red}{मरुतोऽयम्} इति विग्रहे तु \textcolor{red}{मारुतः}।\footnote{\textcolor{red}{तस्येदम्} (पा॰सू॰~४.३.१२०) इत्यनेन।} अकारान्ताभावात् \textcolor{red}{इञ्} प्रत्ययस्याऽप्यभावः। किन्तु \textcolor{red}{मरुदेव मारुतः}।\footnote{\textcolor{red}{समीरमारुतमरुज्जगत्प्राणसमीरणाः} (अ॰को॰~१.१.६२)} प्रज्ञादित्वात्स्वार्थे \textcolor{red}{अण्}।\footnote{\textcolor{red}{मरुत्‌}\-प्रातिपदिकात् \textcolor{red}{प्रज्ञादिभ्यश्च} (पा॰सू॰~५.४.३८) इत्यनेन \textcolor{red}{अण्‌}\-प्रत्यये \textcolor{red}{तद्धितेष्वचामादेः} (पा॰सू॰~७.२.११७) इत्यनेनाऽदिवृद्धौ विभक्तिकार्ये।} \textcolor{red}{मारुतस्यापत्यं पुमान् मारुतिः} इति \textcolor{red}{अत इञ्} (पा॰सू॰~४.१.९५) इत्यनेनेञ्प्रत्यये विभक्ति\-कार्ये \textcolor{red}{मारुतिः}।\footnote{\textcolor{red}{यचि भम्} (पा॰सू॰~१.४.१८) इत्यनेन भत्वम्। \textcolor{red}{यस्येति च} (पा॰सू॰~६.४.१४८) इत्यनेनाकार\-लोपः।}\end{sloppypar}
\section[दशसाहस्राः]{दशसाहस्राः}
\centering\textcolor{blue}{प्रेषिता दशसाहस्रा हरयो रघुसत्तम।\nopagebreak\\
आनेतुं वानरान् दिग्भ्यो महापर्वतसन्निभान्॥}\nopagebreak\\
\raggedleft{–~अ॰रा॰~४.५.४६}\\
\begin{sloppypar}\hyphenrules{nohyphenation}\justifying\noindent\hspace{10mm} \textcolor{red}{दशसाहस्रमस्त्येषामिति दशसाहस्राः} इत्यर्शआद्यजन्तम्।\footnote{\textcolor{red}{अर्शआदिभ्योऽच्} पा॰सू॰~५.२.१२७) इत्यनेन। \textcolor{red}{यचि भम्} (पा॰सू॰~१.४.१८) इत्यनेन भत्वम्। \textcolor{red}{यस्येति च} (पा॰सू॰~६.४.१४८) इत्यनेनाकार\-लोपः।}\end{sloppypar}
\vspace{2mm}
\centering ॥ इति किष्किन्धाकाण्डीयप्रयोगाणां विमर्शः ॥\nopagebreak\\
\vspace{4mm}
\pdfbookmark[2]{सुन्दरकाण्डम्}{Chap2Part2Kanda5}
\phantomsection
\addtocontents{toc}{\protect\setcounter{tocdepth}{2}}
\addcontentsline{toc}{subsection}{सुन्दरकाण्डीयप्रयोगाणां विमर्शः}
\addtocontents{toc}{\protect\setcounter{tocdepth}{0}}
\centering ॥ अथ सुन्दरकाण्डीयप्रयोगाणां विमर्शः ॥\nopagebreak\\
\section[उद्वमती]{उद्वमती}
\centering\textcolor{blue}{हनूमानपि तां वाममुष्टिनाऽवज्ञयाऽहनत्।\nopagebreak\\
तदैव पतिता भूमौ रक्तमुद्वमती भृशम्॥}\nopagebreak\\
\raggedleft{–~अ॰रा॰~५.१.४६}\\
\begin{sloppypar}\hyphenrules{nohyphenation}\justifying\noindent\hspace{10mm} अत्र हनुमन्मुष्टि\-प्रहारेण रक्तमुद्वमन्ती लङ्किनी पपात। \textcolor{red}{वम्} धातोः (\textcolor{red}{टुवमँ उद्गिरणे} धा॰पा॰~९८४) भ्वादित्वात् \textcolor{red}{नुम्} प्रयोक्तव्यः।\footnote{\textcolor{red}{शप्श्यनोर्नित्यम्} (पा॰सू॰~७.१.८१) इत्यनेन \textcolor{red}{नुम्} नित्यमिति भावः।} किन्त्वत्र \textcolor{red}{उद्वमति} इति विग्रह उणादिः \textcolor{red}{तृँच्} प्रत्ययः।\footnote{नायं \textcolor{red}{बहुलमन्यत्रापि} (प॰उ॰~२.९५) इति तृच्। स नोगित्। \textcolor{red}{कार्याद्विद्यादनूबन्धम्} (भा॰पा॰सू॰~३.३.१) \textcolor{red}{केचिदविहिता अप्यूह्याः} (वै॰सि॰कौ॰~३१६९) इत्यनुसारमूह्योऽ\-यमविहित उगित्प्रत्ययः। \textcolor{red}{तृँच्‌}\-प्रत्यये चात्र शबागमोऽप्यूह्यः। \textcolor{red}{नयतेः षुगागमः} (प॰उ॰ श्वे॰वृ॰~२.९६) इतिवत्।} ततो ङीपि\footnote{\textcolor{red}{उगितश्च} (पा॰सू॰~४.१.६) इत्यनेन।} \textcolor{red}{उद्वमती}। यद्वा \textcolor{red}{गण\-कार्यमनित्यम्} (प॰शे॰~९३.३) इत्यनेन शबभावे नुमभावः।\footnote{शतर्येव शम्नुमभावे \textcolor{red}{उद्वमती} इति भावः। यद्वा \textcolor{red}{आगम\-शास्त्रमनित्यम्} (प॰शे॰~९३.२) इत्यनेन नुमभावः। उत् \textcolor{red}{टुवमँ उद्गिरणे} (धा॰पा॰~९८४)~\arrow उत्~वम्~\arrow \textcolor{red}{शेषात्कर्तरि परस्मैपदम्} (पा॰सू॰~१.३.७८)~\arrow \textcolor{red}{वर्तमाने लट्} (पा॰सू॰~३.२.१२३)~\arrow उत्~वम्~लट्~\arrow \textcolor{red}{लटः शतृशानचावप्रथमा\-समानाधिकरणे} (पा॰सू॰~३.२.१२४)~\arrow उत्~वम्~शतृँ~\arrow उत्~वम्~अत्~\arrow \textcolor{red}{गण\-कार्यमनित्यम्} (प॰शे॰~९३.३)~\arrow शबभावः~\arrow उत्~वमत्~\arrow \textcolor{red}{झलां जशोऽन्ते} (पा॰सू॰~८.२.३९)~\arrow उद्~वमत्~\arrow उद्वमत्~\arrow \textcolor{red}{उगितश्च} (पा॰सू॰~४.१.६)~\arrow उद्वमत्~ङीप्‌~\arrow उद्वमत्~ई~\arrow उद्वमती~\arrow \textcolor{red}{कृत्तद्धित\-समासाश्च} (पा॰सू॰~१.२.४६)~\arrow प्रातिपदिक\-सञ्ज्ञा~\arrow उद्वमती~सुँ~\arrow \textcolor{red}{हल्ङ्याब्भ्यो दीर्घात्सुतिस्यपृक्तं हल्} (पा॰सू॰~६.१.६८)~\arrow उद्वमती। \textcolor{red}{यद्वा उत्~वम्~अत्} (पूर्ववत्)~\arrow \textcolor{red}{कर्तरि शप्‌} (पा॰सू॰~३.१.६८)~\arrow उत्~वम्~शप्~अत्~\arrow उत्~वम्~अ~अत्~\arrow \textcolor{red}{अतो गुणे} (पा॰सू॰~६.१.९७)~\arrow उत्~वमत्~\arrow \textcolor{red}{झलां जशोऽन्ते} (पा॰सू॰~८.२.३९)~\arrow उद्~वमत्~\arrow उद्वमत्~\arrow \textcolor{red}{उगितश्च} (पा॰सू॰~४.१.६)~\arrow उद्वमत्~ङीप्‌~\arrow उद्वमत्~ई~\arrow \textcolor{red}{शप्श्यनोर्नित्यम्} (पा॰सू॰~७.१.८१)~\arrow नित्यनुम्प्राप्तिः~\arrow \textcolor{red}{आगम\-शास्त्रमनित्यम्} (प॰शे॰~९३.२)~\arrow नुमभावः~\arrow उद्वमती~\arrow शेषं पूर्ववत्।}\end{sloppypar}
\section[ऐन्द्रः]{ऐन्द्रः}
\centering\textcolor{blue}{ऐन्द्रः काकस्तदागत्य नखैस्तुण्डेन चासकृत्।\nopagebreak\\
मत्पादाङ्गुष्ठमारक्तं विददाराऽमिषाशया॥}\nopagebreak\\
\raggedleft{–~अ॰रा॰~५.३.५४}\\
\begin{sloppypar}\hyphenrules{nohyphenation}\justifying\noindent\hspace{10mm} \textcolor{red}{इन्द्रस्यापत्यं पुमान्} इति विग्रहे \textcolor{red}{अत इञ्} (पा॰सू॰~४.१.९५) इत्यनेन \textcolor{red}{इञ्} प्रत्यये \textcolor{red}{ऐन्द्रिः}। किन्तु \textcolor{red}{इन्द्रस्यायम्} इति विग्रहे \textcolor{red}{तस्येदम्} (पा॰सू॰~४.३.१२०) इत्येन \textcolor{red}{अण्} प्रत्यये भत्वादकार\-लोपे\footnote{\textcolor{red}{यचि भम्} (पा॰सू॰~१.४.१८) इत्यनेन भत्वम्। \textcolor{red}{यस्येति च} (पा॰सू॰~६.४.१४८) इत्यनेनाकार\-लोपः।} \textcolor{red}{ऐन्द्रः}। सीतापहार\-रूप\-गर्हित\-कर्मत्वादपत्यत्व\-कलङ्क\-धियाऽत्र तदपत्यत्वं न विवक्षितम्।\end{sloppypar}
\section[ब्रह्मपाशतः]{ब्रह्मपाशतः}
\centering\textcolor{blue}{बद्ध्वाऽऽनेष्ये द्रुतं तात वानरं ब्रह्मपाशतः।\nopagebreak\\
इत्युक्त्वा रथमारुह्य राक्षसैर्बहुभिर्वृतः॥}\nopagebreak\\
\raggedleft{–~अ॰रा॰~५.३.९२}\\
\begin{sloppypar}\hyphenrules{nohyphenation}\justifying\noindent\hspace{10mm} अत्र \textcolor{red}{ब्रह्म\-पाशतः} इति \textcolor{red}{ब्रह्मपाशेन} विग्रहेऽस्मिन्तृतीयार्थे सार्वविभक्तिकस्तसिः।\footnote{तसेः सार्व\-विभक्तिकत्वं तदन्तानामाकृति\-गणत्वं च \pageref{fn:yatah}तमे पृष्ठे \ref{fn:yatah}तम्यां पादटिप्पण्यां स्पष्टीकृतम्।
\textcolor{red}{ब्रह्मपाशतः} इत्यत्र तृतीयायां तसिरिति भावः।}\end{sloppypar}
\section[भेदयित्वा]{भेदयित्वा}
\centering\textcolor{blue}{भेदयित्वा ततो घोरं सिंहनादमथाकरोत्।\nopagebreak\\
ततोऽतिहर्षाद्धनुमान स्तम्भमुद्यस्य वीर्यवान्॥}\nopagebreak\\
\raggedleft{–~अ॰रा॰~५.३.९६}\\
\begin{sloppypar}\hyphenrules{nohyphenation}\justifying\noindent\hspace{10mm} \textcolor{red}{भिदिँर विदारणे} (धा॰पा॰~१४३९) इत्यत्र स्वार्थिको णिच्।\footnote{\textcolor{red}{निवृत्तप्रेषणाद्धातोः प्राकृतेऽर्थे णिजुच्यते} (वा॰प॰~३.७.६०)।} ततो \textcolor{red}{क्त्वा}\-प्रत्यय इटि गुणेऽयादेशे \textcolor{red}{भेदयित्वा}।\footnote{\textcolor{red}{भिदिँर विदारणे} (धा॰पा॰~१४३९)~\arrow भिद्~\arrow स्वार्थे णिच्~\arrow भिद्~णिच्~\arrow भिद्~इ~\arrow \textcolor{red}{पुगन्त\-लघूपधस्य च} (पा॰सू॰~७.३.८६)~\arrow भेद्~इ~\arrow भेदि~\arrow \textcolor{red}{सनाद्यन्ता धातवः} (पा॰सू॰~३.१.३२)~\arrow धातु\-सञ्ज्ञा~\arrow भेदि~क्त्वा~\arrow \textcolor{red}{समान\-कर्तृकयोः पूर्वकाले} (पा॰सू॰~३.४.२१)~\arrow भेदि~क्त्वा~\arrow भेदि~त्वा~\arrow \textcolor{red}{आर्धधातुकस्येड्वलादेः} (पा॰सू॰~७.२.३५)~\arrow भेदि~इट्~त्वा~\arrow भेदि~इ~त्वा~\arrow \textcolor{red}{सार्वधातुकार्ध\-धातुकयोः} (पा॰सू॰~७.३.८४)~\arrow भेदे~इ~त्वा~\arrow \textcolor{red}{एचोऽयवायावः} (पा॰सू॰~६.१.७८)~\arrow भेदय्~इ~त्वा~\arrow भेदयित्वा~\arrow \textcolor{red}{तद्धितश्चासर्व\-विभक्तिः} (पा॰सू॰~१.१.३८)~\arrow अव्यय\-सञ्ज्ञा~\arrow \textcolor{red}{कृत्तद्धित\-समासाश्च} (पा॰सू॰~१.२.४६)~\arrow प्रातिपदिक\-सञ्ज्ञा~\arrow विभक्तिकार्यम्~\arrow भेदयित्वा~सुँ~\arrow \textcolor{red}{अव्ययादाप्सुपः} (पा॰सू॰~२.४.८२)~\arrow भेदयित्वा। यद्वा \textcolor{red}{भेदनं भेदः}। \textcolor{red}{भावे} (पा॰सू॰~३.३.१८) इत्यनेन घञि। \textcolor{red}{तत्करोति तदाचष्टे} (धा॰पा॰ ग॰सू॰) इत्यनेन णिचि लटि शपि तिपि \textcolor{red}{भेदं करोति} इत्यर्थे  \textcolor{red}{भेदयति}। ततः \textcolor{red}{भेदं कृत्वा} इत्यर्थे \textcolor{red}{भेदयित्वा}। प्रकृत\-धातोरनिट्कत्वाण्णिजभावे \textcolor{red}{भित्त्वा} इति रूपम्।}\end{sloppypar}
\vspace{2mm}
\centering ॥ इति सुन्दरकाण्डीयप्रयोगाणां विमर्शः ॥\nopagebreak\\
\vspace{4mm}
\pdfbookmark[2]{युद्धकाण्डम्}{Chap2Part2Kanda6}
\phantomsection
\addtocontents{toc}{\protect\setcounter{tocdepth}{2}}
\addcontentsline{toc}{subsection}{युद्धकाण्डीयप्रयोगाणां विमर्शः}
\addtocontents{toc}{\protect\setcounter{tocdepth}{0}}
\centering ॥ अथ युद्धकाण्डीयप्रयोगाणां विमर्शः ॥\nopagebreak\\
\section[आरोहयन्तः]{आरोहयन्तः}
\centering\textcolor{blue}{शैलानारोहयन्तश्च जग्मुर्मारुतवेगतः।\nopagebreak\\
असङ्ख्याताश्च सर्वत्र वानराः परिपूरिताः॥}\nopagebreak\\
\raggedleft{–~अ॰रा॰~६.१.३९}\\
\begin{sloppypar}\hyphenrules{nohyphenation}\justifying\noindent\hspace{10mm} \textcolor{red}{आ}\-पूर्वकं \textcolor{red}{रुह्‌}\-धातुं (\textcolor{red}{रुहँ बीजजन्मनि प्रादुर्भावे च} धा॰पा॰~८५९) स्वार्थे णिजन्तं कृत्वा ततः शतरि शपि गुणेऽयादेशे प्रथमा\-बहुवचने \textcolor{red}{आरोहयन्तः}।\footnote{\textcolor{red}{न्यग्भावना न्यग्भवनं रुहौ शुद्धे प्रतीयते। न्यग्भावना न्यग्भवनं ण्यन्तेऽपि प्रतिपद्यते॥ अवस्थां पञ्चमीमाहुर्ण्यन्ते तां कर्मकर्तरि। निवृत्तप्रेषणाद्धातोः प्राकृतेऽर्थे णिजुच्यते॥} (वा॰प॰~३.७.५९–६०)। \textcolor{red}{रुहँ बीजजन्मनि प्रादुर्भावे च} (धा॰पा॰~८५९)~\arrow रुह्~\arrow स्वार्थे णिच्~\arrow रुह्~णिच्~\arrow रुह्~इ~\arrow \textcolor{red}{पुगन्त\-लघूपधस्य च} (पा॰सू॰~७.३.८६)~\arrow रोह्~इ~\arrow रोहि~\arrow \textcolor{red}{सनाद्यन्ता धातवः} (पा॰सू॰~३.१.३२)~\arrow धातु\-सञ्ज्ञा। आङ्~रोहि~\arrow आ~रोहि~\arrow \textcolor{red}{शेषात्कर्तरि परस्मैपदम्} (पा॰सू॰~१.३.७८)~\arrow \textcolor{red}{वर्तमाने लट्} (पा॰सू॰~३.२.१२३)~\arrow आ~रोहि~लट्~\arrow \textcolor{red}{लटः शतृशानचावप्रथमा\-समानाधिकरणे} (पा॰सू॰~३.२.१२४)~\arrow आ~रोहि~शतृँ~\arrow आ~रोहि~अत्~\arrow \textcolor{red}{सार्वधातुकार्ध\-धातुकयोः} (पा॰सू॰~७.३.८४)~\arrow आ~रोहे~अत्~\arrow \textcolor{red}{एचोऽयवायावः} (पा॰सू॰~६.१.७८)~\arrow आ~रोहय्~अत्~\arrow आरोहयत्~\arrow \textcolor{red}{कृत्तद्धित\-समासाश्च} (पा॰सू॰~१.२.४६)~\arrow प्रातिपादिक\-सञ्ज्ञा~\arrow विभक्ति\-कार्यम्~\arrow आरोहयत्~जस्~\arrow आरोहयत्~अस्~\arrow \textcolor{red}{ उगिदचां सर्वनामस्थानेऽधातोः} (पा॰सू॰~७.१.७०)~\arrow \textcolor{red}{मिदचोऽन्त्यात्परः} (पा॰सू॰~१.१.४७)~\arrow आरोहय~नुँम्~त्~अस्~\arrow आरोहय~न्~त्~अस्~\arrow आरोहयन्तस्~\arrow \textcolor{red}{ससजुषो रुः} (पा॰सू॰~८.२.६६)~\arrow आरोहयन्तरुँ~\arrow \textcolor{red}{खरवसानयोर्विसर्जनीयः} (पा॰सू॰~८.३.१५)~\arrow आरोहयन्तः।}\end{sloppypar}
\section[याचते]{याचते}
\centering\textcolor{blue}{सकृदेव प्रपन्नाय तवास्मीति च याचते।\nopagebreak\\
अभयं सर्वभूतेभ्यो ददाम्येतद्व्रतं मम॥}\nopagebreak\\
\raggedleft{–~अ॰रा॰~६.३.१२}\footnote{वाल्मीकीय\-रामायणे च~– \textcolor{red}{सकृदेव प्रपन्नाय तवास्मीति च याचते। अभयं सर्वभूतेभ्यो ददाम्येतद्व्रतं मम॥} (वा॰रा॰~६.१८.३३)।}\\
\begin{sloppypar}\hyphenrules{nohyphenation}\justifying\noindent\hspace{10mm} अत्र शरणागतं विभीषणं प्रति स्व\-स्वभावं प्रकटयन् रामभद्रः प्राह यत् \textcolor{red}{प्रपन्नाय जनाय तवास्मीति याचमानायाहं सर्व\-प्राणिभ्योऽभयं ददामि}। आत्मनेपदीयः \textcolor{red}{याच्} धातुः (\textcolor{red}{टुयाचृँ याञ्चायाम्} धा॰पा॰~८६३)।\footnote{पूर्वपक्षोऽयम्।} तस्मात् \textcolor{red}{याचते} इति हि प्रयोगः।\footnote{यथा \textcolor{red}{बलिं याचते वसुधाम्} (वै॰सि॰कौ॰~५३९, ल॰सि॰कौ॰~८९५) इति प्रसिद्धोदाहरणे।} अस्माच्छानचि चतुर्थ्यैक\-वचने \textcolor{red}{याचमानाय} इति हि पाणिनीयम्। किन्तु \textcolor{red}{याचत इति याचः}।\footnote{\textcolor{red}{नन्दि\-ग्रहि\-पचादिभ्यो ल्युणिन्यचः} (पा॰सू॰~३.१.१३४) इत्यनेन कर्तरि पचाद्यच्।} \textcolor{red}{याच इवाऽचरतीति याचति}।\footnote{याच~\arrow \textcolor{red}{सर्वप्राति\-पदिकेभ्य आचारे क्विब्वा वक्तव्यः} (वा॰~३.१.११)~\arrow याच~क्विँप्~\arrow याच~व्~\arrow \textcolor{red}{वेरपृक्तस्य} (पा॰सू॰~६.१.६७)~\arrow याच~\arrow \textcolor{red}{सनाद्यन्ता धातवः} (पा॰सू॰~३.१.३२)~\arrow धातुसञ्ज्ञा~\arrow \textcolor{red}{शेषात्कर्तरि परस्मैपदम्} (पा॰सू॰~१.३.७८)~\arrow \textcolor{red}{वर्तमाने लट्} (पा॰सू॰~३.२.१२३)~\arrow याच~लट्~\arrow याच~तिप्~\arrow याच~ति~\arrow \textcolor{red}{कर्तरि शप्‌} (पा॰सू॰~३.१.६८)~\arrow याच~शप्~ति~\arrow याच~अ~ति~\arrow \textcolor{red}{अतो गुणे} (पा॰सू॰~६.१.९७)~\arrow याच~ति~\arrow याचति।} \textcolor{red}{याचतीति याचन्}।\footnote{याच~\arrow धातुसञ्ज्ञा (पूर्ववत्)~\arrow \textcolor{red}{शेषात्कर्तरि परस्मैपदम्} (पा॰सू॰~१.३.७८)~\arrow \textcolor{red}{वर्तमाने लट्} (पा॰सू॰~३.२.१२३)~\arrow याच~लट्~\arrow \textcolor{red}{लटः शतृशानचावप्रथमा\-समानाधिकरणे} (पा॰सू॰~३.२.१२४)~\arrow याच~शतृँ~\arrow याच~अत्~\arrow \textcolor{red}{अतो गुणे} (पा॰सू॰~६.१.९७)~\arrow याचत्~\arrow \textcolor{red}{कृत्तद्धित\-समासाश्च} (पा॰सू॰~१.२.४६)~\arrow प्रातिपादिक\-सञ्ज्ञा~\arrow विभक्ति\-कार्यम्~\arrow याचत्~सुँ~\arrow याचत्~स्~\arrow \textcolor{red}{उगिदचां सर्वनामस्थानेऽधातोः} (पा॰सू॰~७.१.७०)~\arrow \textcolor{red}{मिदचोऽन्त्यात्परः} (पा॰सू॰~१.१.४७)~\arrow याच~नुँम्~त्~स्~\arrow याच~न्~त्~स्~\arrow \textcolor{red}{हल्ङ्याब्भ्यो दीर्घात्सुतिस्यपृक्तं हल्} (पा॰सू॰~६.१.६८)~\arrow याच~न्~त्~\arrow \textcolor{red}{संयोगान्तस्य लोपः} (पा॰सू॰~८.२.२३)~\arrow याच~न्~त्~स्~\arrow याचन्।} \textcolor{red}{तस्मै याचते} इत्याचार\-क्विबन्ताच्छतृ\-प्रत्यये \textcolor{red}{याचते}।\footnote{याचत्~\arrow प्रातिपादिक\-सञ्ज्ञा (पूर्ववत्)~\arrow विभक्ति\-कार्यम्~\arrow याचत्~ङे~\arrow याचत्~ए~\arrow याचते।} याचकवदाचरण\-कारिणेऽप्यभयं ददामि तदा याचमानाय किं दद्यामिति स्वकारुण्यात्स्वयं ध्वनयितुं स्वयं निखिल\-निगमार्णवो राघवो \textcolor{red}{याचते} इति प्रायुङ्क्त। अथवोभयपदी धातुरयम्। तथा च भगवान्पाणिनिः \textcolor{red}{टुयाचृँ याच्ञायाम्} (धा॰पा॰~८६३)। स्वरितेदयम्। एष कर्त्रभिप्राये क्रियाफल आत्मनेपदी।\footnote{\textcolor{red}{स्वरितञितः कर्त्रभिप्राये क्रियाफले} (पा॰सू॰~१.३.७२) इत्यनेन।} अन्याभिप्राये क्रियाफले परस्मैपदी।\footnote{\textcolor{red}{शेषात्कर्तरि परस्मैपदम्} (पा॰सू॰~१.३.७८) इत्यनेन।} भगवतोऽभिप्रायो यत्~– यः स्वार्थं त्यक्त्वा प्रेमभक्तये मां प्रपन्नः \textcolor{red}{तवास्मि} इति वदन् मामभयं ब्रह्मैव याचति तस्मा अहं सर्वभूतेभ्योऽभयं ददामि। अतोऽत्र परस्मैपदं ततः शतृप्रत्ययः। तस्य चतुर्थ्येकवचनरूपं \textcolor{red}{याचते}। \textcolor{red}{याचतीति याचन् तस्मै याचते}।\end{sloppypar}
\section[अभिषेकम्]{अभिषेकम्}
\centering\textcolor{blue}{लङ्काराज्याधिपत्यार्थमभिषेकं रमापतिः।\nopagebreak\\
कारयामास सचिवैर्लक्ष्मणेन विशेषतः॥}\nopagebreak\\
\raggedleft{–~अ॰रा॰~६.३.४५}\\
\begin{sloppypar}\hyphenrules{nohyphenation}\justifying\noindent\hspace{10mm} अत्र नायं भाव\-साधनोऽपि तु \textcolor{red}{पुंसि सञ्ज्ञायां घः प्रायेण} (पा॰सू॰~३.३.११८) इत्यनेन \textcolor{red}{घ}\-प्रत्ययः सञ्ज्ञायामित्यस्य प्रायिकत्वात्।\end{sloppypar}
\section[शासिता]{शासिता}
\centering\textcolor{blue}{अनुजीव्य सुदुर्बुद्धे गुरुवद्भाषसे कथम्।\nopagebreak\\
शासिताऽहं त्रिजगतां त्वं मां शिक्षन्न लज्जसे॥}\nopagebreak\\
\raggedleft{–~अ॰रा॰~६.५.२}\\
\begin{sloppypar}\hyphenrules{nohyphenation}\justifying\noindent\hspace{10mm} अत्र \textcolor{red}{शास्तीति शास्ता} इत्यदादेर्धातोः (\textcolor{red}{शासुँ अनुशिष्टौ} धा॰पा॰~१०७५) तु नेट्।\footnote{\textcolor{red}{तृंस्तृचौ शंसिक्षदादिभ्यः संज्ञायां चानिटौ} (प॰उ॰~२.९४, द॰उ॰~२.१)। \textcolor{red}{शास्ता वरुणः। प्रशास्ता । शास्ता गुरुः} (प॰उ॰श्वे॰वृ॰~२.९४)। \textcolor{red}{प्रशास्तीति प्रशास्ता। राजा गुरुर्वा} (द॰उ॰वृ॰~२.१)।} किन्तु \textcolor{red}{आगम\-शास्त्रमनित्यम्} (प॰शे॰~९३.२) इति कृत्वा सेट्।\footnote{यथा मनुस्मृतौ~– \textcolor{red}{प्रशासितारं सर्वेषामणीयांसमणोरपि} (म॰स्मृ॰~१२.१२२)। शकुन्तला\-नाटकेऽपि~– \textcolor{red}{कः पौरवे वसुमतीं शासति शासितरि दुर्विनीतानाम्} (अ॰शा॰~१.२४)। यद्वाऽसञ्ज्ञायां \textcolor{red}{तृंस्तृचौ शंसिक्षदादिभ्यः संज्ञायां चानिटौ} (प॰उ॰~२.९४, द॰उ॰~२.१) इत्यस्याप्रवृत्तौ \textcolor{red}{ण्वुल्तृचौ} (पा॰सू॰~३.१.१३३) इत्यनेन तृचि \textcolor{red}{तृन्} (पा॰सू॰~३.२.१३५) इत्यनेन तृनि वा \textcolor{red}{आर्धधातुकस्येड्वलादेः} (पा॰सू॰~७.२.३५) इत्यनेनेडागमे विभक्ति\-कार्ये \textcolor{red}{शासिता}।}\end{sloppypar}
\section[शिक्षन्]{शिक्षन्}
\centering\textcolor{blue}{अनुजीव्य सुदुर्बुद्धे गुरुवद्भाषसे कथम्।\nopagebreak\\
शासिताऽहं त्रिजगतां त्वं मां शिक्षन्न लज्जसे॥}\nopagebreak\\
\raggedleft{–~अ॰रा॰~६.५.२}\\
\begin{sloppypar}\hyphenrules{nohyphenation}\justifying\noindent\hspace{10mm} \textcolor{red}{शिक्ष्} धातुः (\textcolor{red}{शिक्षँ विद्योपादाने} धा॰पा॰~६०५) आत्मनेपदी। स च शिक्षा\-ग्रहणार्थको न तु शिक्षा\-दानार्थकः।\footnote{यथा शकुन्तला\-नाटकस्य मैथिलबङ्गीयपाठयोः~– \textcolor{red}{विवर्तित\-भ्रूरियमद्य शिक्षते भयादकामाऽपि हि दृष्टि\-विभ्रमम्} (अ॰शा॰~१.२४)। रघुवंशे च~– \textcolor{red}{अशिक्षतास्रं पितुरेव मन्त्रवत्} (र॰वं॰~३.३१)।} अत्र \textcolor{red}{शिक्षयन्} इति पाणिनीयम्। किन्त्वन्तर्भावित\-ण्यर्थत्वादनित्यमात्मनेपदम्। तस्मात् \textcolor{red}{शिक्षन्} प्रयोगोऽयं पाणिनीयोऽस्ति सर्वतः।\end{sloppypar}
\section[आप्लवन्तः]{आप्लवन्तः}
\centering\textcolor{blue}{कोटिशतयुताश्चान्ये रुरुधुर्नगरं भृशम्।\nopagebreak\\
आप्लवन्तः प्लवन्तश्च गर्जन्तश्च प्लवङ्गमाः॥}\nopagebreak\\
\raggedleft{–~अ॰रा॰~६.५.५२}\\
\begin{sloppypar}\hyphenrules{nohyphenation}\justifying\noindent\hspace{10mm} \textcolor{red}{प्लु}\-धातुः (\textcolor{red}{प्लुङ् गतौ} धा॰पा॰~९५८) आत्मनेपदी। अत्रापि \textcolor{red}{शतृ}\-प्रत्यय आत्मनेपदस्यानित्यता\-स्वीकारेण।\footnote{\textcolor{red}{व्यत्ययो बहुलम्} (पा॰सू॰~३.१.८५) इत्यनेनात्मनेपदस्य छान्दसानित्यताया परस्मैपदमिति भावः। यद्वा \textcolor{red}{अनुदात्तेत्त्व\-लक्षणमात्मने\-पदमनित्यम्} (प॰शे॰~९३.४) इतिवत् ङित्त्व\-लक्षणमात्मने\-पदमप्यनित्यम्।}\end{sloppypar}
\section[ग्रसन्ती]{ग्रसन्ती}
\centering\textcolor{blue}{ततो ददर्श हनुमान् ग्रसन्तीं मकरीं रुषा।\nopagebreak\\
दारयामास हस्ताभ्यां वदनं सा ममार हे॥}\nopagebreak\\
\raggedleft{–~अ॰रा॰~६.७.२३}\\
\begin{sloppypar}\hyphenrules{nohyphenation}\justifying\noindent\hspace{10mm} \textcolor{red}{ग्रस्‌}\-धातुः (\textcolor{red}{ग्रसुँ अदने} धा॰पा॰~६३०) आत्मनेपदी। \textcolor{red}{ग्रसमाना} इति वक्तव्ये \textcolor{red}{ग्रसन्ती} इति \textcolor{red}{शतृ}\-प्रत्ययान्तम् \textcolor{red}{अनुदात्तेत्त्व\-लक्षणमात्मने\-पदमनित्यम्} (प॰शे॰~९३.४) स्वीकृत्योक्तम्।\end{sloppypar}
\section[शासयन्तम्]{शासयन्तम्}
\centering\textcolor{blue}{पादुके ते पुरस्कृत्य शासयन्तं वसुन्धराम्।\nopagebreak\\
मन्त्रिभिः पौरमुख्यैश्च काषायाम्बरधारिभिः॥}\nopagebreak\\
\raggedleft{–~अ॰रा॰~६.१४.५३}\\
\begin{sloppypar}\hyphenrules{nohyphenation}\justifying\noindent\hspace{10mm} अत्र \textcolor{red}{शास्‌}\-धातुं (\textcolor{red}{शासुँ अनुशिष्टौ} धा॰पा॰~१०७५) स्वार्थे णिजन्तं मत्वा शतरि शपि गुणेऽयादेशे नुमि विभक्तिकार्ये च \textcolor{red}{शासयन्तम्}।\footnote{\textcolor{red}{शासुँ अनुशिष्टौ} (धा॰पा॰~१०७५)~\arrow शास्~\arrow स्वार्थे णिच्~\arrow शास्~णिच्~\arrow शास्~इ~\arrow शासि~\arrow \textcolor{red}{सनाद्यन्ता धातवः} (पा॰सू॰~३.१.३२)~\arrow धातु\-सञ्ज्ञा~\arrow \textcolor{red}{शेषात्कर्तरि परस्मैपदम्} (पा॰सू॰~१.३.७८)~\arrow \textcolor{red}{वर्तमाने लट्} (पा॰सू॰~३.२.१२३)~\arrow शासि~लट्~\arrow \textcolor{red}{लटः शतृशानचावप्रथमा\-समानाधिकरणे} (पा॰सू॰~३.२.१२४)~\arrow शासि~शतृँ~\arrow शासि~अत्~\arrow \textcolor{red}{सार्वधातुकार्ध\-धातुकयोः} (पा॰सू॰~७.३.८४)~\arrow शासे~अत्~\arrow \textcolor{red}{एचोऽयवायावः} (पा॰सू॰~६.१.७८)~\arrow शासय्~अत्~\arrow शासयत्~\arrow \textcolor{red}{कृत्तद्धित\-समासाश्च} (पा॰सू॰~१.२.४६)~\arrow प्रातिपादिक\-सञ्ज्ञा~\arrow विभक्ति\-कार्यम्~\arrow शासयत्~अम्~\arrow \textcolor{red}{ उगिदचां सर्वनामस्थानेऽधातोः} (पा॰सू॰~७.१.७०)~\arrow \textcolor{red}{मिदचोऽन्त्यात्परः} (पा॰सू॰~१.१.४७)~\arrow शासय~नुँम्~त्~अम्~\arrow शासय~न्~त्~अम्~\arrow शासयन्तम्।}\end{sloppypar}
\section[गायमानाः]{गायमानाः}
\centering\textcolor{blue}{पश्चाद्दुरात्मना राम रावणेनाभिविद्रुताः।\nopagebreak\\
तमेव गायमानाश्च तदाराधनतत्पराः॥}\nopagebreak\\
\raggedleft{–~अ॰रा॰~६.१५.६९}\\
\begin{sloppypar}\hyphenrules{nohyphenation}\justifying\noindent\hspace{10mm} \textcolor{red}{गै}\-धातुः (\textcolor{red}{गै शब्दे} धा॰पा॰~९१८) परस्मैपदी। \textcolor{red}{गायन्तीति गायन्तः} इति पाणिनीयम्। किन्तु \textcolor{red}{कर्तरि कर्म\-व्यतिहारे} (पा॰सू॰~१.३.१४) इत्यात्मनेपदे \textcolor{red}{गायन्त इति गायमानाः}। न च कर्म\-व्यतिहार\-द्योतकः शब्दो नास्तीति वाच्यम्। \textcolor{red}{व्यतिगायन्ते} इति प्रयोगः। \textcolor{red}{विनाऽपि प्रत्ययं पूर्वोत्तर\-पद\-लोपो वक्तव्यः} (वा॰~५.३.८३) इत्यनेन \textcolor{red}{व्यति} इत्यस्य लोपः। ततः शानचि \textcolor{red}{गायमानाः}। \textcolor{red}{अयोग्यं गायं कुर्वाणाः} इति भावः।\end{sloppypar}
\vspace{2mm}
\centering ॥ इति युद्धकाण्डीयप्रयोगाणां विमर्शः ॥\nopagebreak\\
\vspace{4mm}
\pdfbookmark[2]{उत्तरकाण्डम्}{Chap2Part2Kanda7}
\phantomsection
\addtocontents{toc}{\protect\setcounter{tocdepth}{2}}
\addcontentsline{toc}{subsection}{उत्तरकाण्डीयप्रयोगाणां विमर्शः}
\addtocontents{toc}{\protect\setcounter{tocdepth}{0}}
\centering ॥ अथोत्तरकाण्डीयप्रयोगाणां विमर्शः ॥\nopagebreak\\
\section[पौत्रान्]{पौत्रान्}
\centering\textcolor{blue}{सुमाली वरलब्धांस्ताञ्ज्ञात्वा पौत्रान् निशाचरान्।\nopagebreak\\
पातालान्निर्भयः प्रायात्प्रहस्तादिभिरन्वितः॥}\nopagebreak\\
\raggedleft{–~अ॰रा॰~७.२.२४}\\
\begin{sloppypar}\hyphenrules{nohyphenation}\justifying\noindent\hspace{10mm} \textcolor{red}{पुत्र्या अपत्यानि पुमांसः} इति विग्रहे \textcolor{red}{स्त्रीभ्यो ढक्} (पा॰सू॰~४.१.१२०) इत्यनेन \textcolor{red}{ढक्} प्रत्यये \textcolor{red}{आयनेयीनीयियः
फढखच्छघां प्रत्ययादीनाम्} (पा॰सू॰~७.१.२) इत्यनेन \textcolor{red}{एय्} आदेशे \textcolor{red}{किति च} (पा॰सू॰~७.२.११८) इत्यनेन वृद्धौ विभक्ति\-कार्ये \textcolor{red}{पौत्रेयान्} इति पाणिनीयम्। किन्तु \textcolor{red}{पुत्र्या इमे} इति विग्रहे \textcolor{red}{तस्येदम्} (पा॰सू॰~४.३.१२०) इत्यनेन \textcolor{red}{अण्}। \textcolor{red}{यस्येति च} (पा॰सू॰~६.४.१४८) इत्यनेन भत्वादीकार\-लोपे विभक्तिकार्ये \textcolor{red}{पौत्रान्}।\end{sloppypar}
\section[विकल्पोज्झितः]{विकल्पोज्झितः}
\centering\textcolor{blue}{राम त्वं परमेश्वरोऽसि सकलं जानासि विज्ञानदृग्\nopagebreak\\
भूतं भव्यमिदं त्रिकालकलनासाक्षी विकल्पोज्झितः।\nopagebreak\\
भक्तानामनुवर्तनाय सकलां कुर्वन् क्रियासंहतिं\nopagebreak\\
त्वं शृण्वन्मनुजाकृतिर्मुनिवचो भासीश लोकार्चितः॥}\nopagebreak\\
\raggedleft{–~अ॰रा॰~७.४.१२}\\
\begin{sloppypar}\hyphenrules{nohyphenation}\justifying\noindent\hspace{10mm} \textcolor{red}{उज्झितो विकल्पो येन} इति विकल्पे \textcolor{red}{सप्तमीविशेषणे बहुव्रीहौ} (पा॰सू॰~२.२.३५) इत्यनेन विशेषणस्य पूर्वं प्रयोक्तव्ये विकल्प\-शब्दस्य प्रयोगो नापाणिनीयः। पूर्व\-निपात\-प्रकरणस्यानित्यत्वात्। \textcolor{red}{समुद्राभ्राद्घः} (पा॰सू॰~४.४.११८) इत्यत्र समुद्र\-शब्दस्य पूर्व\-प्रयोगात्।\footnote{\textcolor{red}{समुद्राभ्रात्} इत्यत्र द्वन्द्व\-समासे \textcolor{red}{अल्पाच्तरम्} (पा॰सू॰~२.२.३४) इत्यनेन \textcolor{red}{अभ्र}\-शब्दस्य पूर्व\-निपाते \textcolor{red}{अभ्रसमुद्रात्} इत्यनेन भवितव्यमासीत्। \textcolor{red}{लक्षण\-हेत्वोः क्रियायाः} (पा॰सू॰~३.२.१२६) इतिवदिदं सूत्रमपि पूर्व\-निपात\-प्रकरणस्यानित्यत्वं ज्ञापयति। \textcolor{red}{लक्षण\-हेत्वोरिति निर्देशः पूर्वनिपात\-व्यभिचार\-लिङ्गम्} (का॰वृ॰~३.२.१२६)।}\end{sloppypar}
\section[पूज्य]{पूज्य}
\centering\textcolor{blue}{तासां भावानुगं राम प्रसादं कर्तुमर्हसि।\nopagebreak\\
श्रुत्वा वसिष्ठवचनं ताः समुत्थाप्य पूज्य च॥}\nopagebreak\\
\raggedleft{–~अ॰रा॰~७.९.१०}\\
\begin{sloppypar}\hyphenrules{nohyphenation}\justifying\noindent\hspace{10mm} अत्र साकेत\-गमनाय कृत\-सङ्कल्पानां प्रजानां विषये वसिष्ठस्य प्रार्थनं\footnote{\textcolor{red}{प्रार्थना प्रार्थनम्} इति द्वावपि शब्दौ भावे। प्र~\textcolor{red}{अर्थँ उपयाच्ञायाम्} (धा॰पा॰~१९०५)~\arrow प्र~अर्थ्~\arrow \textcolor{red}{सत्याप\-पाश\-रूप\-वीणा\-तूल\-श्लोक\-सेना\-लोम\-त्वच\-वर्म\-वर्ण\-चूर्ण\-चुरादिभ्यो णिच्} (पा॰सू॰~३.१.२५)~\arrow प्र~अर्थ्~णिच्~\arrow प्र~अर्थ्~इ~\arrow प्र~अर्थि~\arrow~\arrow \textcolor{red}{सनाद्यन्ता धातवः} (पा॰सू॰~३.१.३२)~\arrow धातु\-सञ्ज्ञा~\arrow \textcolor{red}{ण्यासश्रन्थो युच्} (पा॰सू॰~३.३.१०७)~\arrow प्र~अर्थि~युच्~\arrow प्र~अर्थि~यु~\arrow \textcolor{red}{णेरनिटि} (पा॰सू॰~६.४.५१)~\arrow प्र~अर्थ्~यु~\arrow \textcolor{red}{युवोरनाकौ} (पा॰सू॰~७.१.१)~\arrow प्र~अर्थ्~अन~\arrow \textcolor{red}{अजाद्यतष्टाप्‌} (पा॰सू॰~४.१.४)~\arrow प्र~अर्थ्~अन~टाप्~\arrow प्र~अर्थ्~अन~आ~\arrow \textcolor{red}{अकः सवर्णे दीर्घः} (पा॰सू॰~६.१.१०१)~\arrow प्र~अर्थना~\arrow \textcolor{red}{अकः सवर्णे दीर्घः} (पा॰सू॰~६.१.१०१)~\arrow प्रार्थना~\arrow विभक्तिकार्यम्~\arrow प्रार्थना। प्र~अर्थि (पूर्ववत्)~\arrow \textcolor{red}{ल्युट् च} (पा॰सू॰~३.३.११५)~\arrow प्र~अर्थि~ल्युट्~\arrow प्र~अर्थि~यु~\arrow \textcolor{red}{णेरनिटि} (पा॰सू॰~६.४.५१)~\arrow प्र~अर्थ्~यु~\arrow \textcolor{red}{युवोरनाकौ} (पा॰सू॰~७.१.१)~\arrow प्र~अर्थ्~अन~\arrow \textcolor{red}{अकः सवर्णे दीर्घः} (पा॰सू॰~६.१.१०१)~\arrow प्रार्थन~\arrow विभक्तिकार्यम्~\arrow प्रार्थनम्।} श्रुत्वा करुणा\-वरुणालयो भगवाञ्छ्रीरामोऽनुगन्तुं ता आज्ञप्तवान्। अत्र \textcolor{red}{पूज्य} इति प्रयुक्तम्। ल्यप्प्रत्ययः समासं विना सम्भवो नहि अतः \textcolor{red}{पूज्य} इति कथं पाणिनीयमिति चेत्। \textcolor{red}{सम्पूज्य} इति प्रयोगः। अस्य च \textcolor{red}{विनाऽपि प्रत्ययं पूर्वोत्तर\-पद\-लोपो वक्तव्यः} (वा॰~५.३.८३) इति वार्त्तिकेन लोपे \textcolor{red}{जात\-संस्कारो न निवर्तते} इति परिभाषया ल्यम्निवृत्त्यभावे \textcolor{red}{पूज्य} इति पाणिनीयमेव।\end{sloppypar}
\begin{sloppypar}\hyphenrules{nohyphenation}\justifying\noindent\hspace{10mm} \end{sloppypar}
\vspace{2mm}
\centering ॥ इत्युत्तरकाण्डीयप्रयोगाणां विमर्शः ॥\nopagebreak\\
\vspace{4mm}
\centering इत्यध्यात्म\-रामायणेऽपाणिनीय\-प्रयोगाणां\-विमर्श\-नामके शोध\-प्रबन्धे द्वितीयाध्याये द्वितीय\-परिच्छेदः।\nopagebreak\\
\vspace{4mm}
\centering इत्यध्यात्म\-रामायणेऽपाणिनीय\-प्रयोगाणां\-विमर्श\-नामके शोध\-प्रबन्धे द्वितीयोऽध्यायः।
