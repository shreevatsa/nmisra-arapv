% Nityanand Misra: LaTeX code to typeset a book in Sanskrit
% Copyright (C) 2016 Nityanand Misra
%
% This program is free software: you can redistribute it and/or modify it under
% the terms of the GNU General Public License as published by the Free Software
% Foundation, either version 3 of the License, or (at your option) any later
% version.
%
% This program is distributed in the hope that it will be useful, but WITHOUT
% ANY WARRANTY; without even the implied warranty of  MERCHANTABILITY or FITNESS
% FOR A PARTICULAR PURPOSE. See the GNU General Public License for more details.
%
% You should have received a copy of the GNU General Public License along with
% this program.  If not, see <http://www.gnu.org/licenses/>.

\setcounter{footnote}{0}
\renewcommand{\thefootnote}{\footnotesize{\devanagarinumeral{footnote}}}
\renewcommand\chaptername{}
\chapter[ग्रन्थकर्तृजीवनवृत्तम्]{ग्रन्थकर्तृजीवनवृत्तम्}
\markboth{ग्रन्थकर्तृजीवनवृत्तम्}{}
\fontsize{14}{21}\selectfont
\vspace{-4mm}
\raggedleft{लेखकौ –~वाचस्पतिमिश्रः, तुलसीदासपरौहा\nopagebreak\\
(सम्पादकः~– नित्यानन्दमिश्रः)}\nopagebreak\\
\vspace{4mm}
\begin{sloppypar}\hyphenrules{nohyphenation}\justifying\noindent\hspace{10mm} संस्कृत\-साहित्य\-जगति प्रथित\-कीर्तयो विलक्षण\-प्रतिभा\-वन्तः पदवाक्य\-प्रमाण\-पारावारीणाः पद्मविभूषण\-साहित्याकादमी\-राष्ट्रिय\-पुरस्कारादि\-पुरस्कृताः कविकुलरत्नानि मध्यप्रदेशस्य सतना\-मण्डल\-वर्तिनि मन्दाकिनी\-सलिल\-विमल\-सलिलासक्ते चित्रकूटे कृतनिवासा अहोरात्रं काव्य\-रचना\-व्याजेन सुरभारती\-समर्चने समर्पित\-जीवनास्तथा च 
विकलाङ्ग\-जनतोपासनायां सततं संलग्नाः शतावधानि\-कवयः स्वामि\-रामभद्राचार्याः केन न ज्ञायन्ते।\end{sloppypar}
\vspace{-2mm}
\begin{center}
वदनं प्रसादसदनं सदयं हृदयं सुधामुचो वाचः।\nopagebreak\\
करणं परोपकरणं येषां केषां न ते वन्द्याः॥\\
\end{center}
\vspace{-2mm}
\begin{sloppypar}\hyphenrules{nohyphenation}\justifying\noindent\hspace{10mm} इति यद्भर्तृहरिणा 
साकूतं समान्वभाणि तदेतदशेषं समञ्जसं प्रस्तुतग्रन्थ\-रत्नस्य प्रणेतृषु धर्मचक्रवर्तिषु महामहोपाध्यायेषु वाचस्पतिषु महाकविषु सर्वतन्त्र\-स्वतन्त्रेषु प्रस्थानत्रयी\-भाष्यकारेषु सर्वाम्नाय\-श्रीचित्रकूट\-तुलसी\-पीठाधीश्वरेषु श्रीमज्जगद्गुरु\-श्रीरामानन्दाचार्यास्पद\-स्वामि\-श्रीरामभद्राचार्यचरणेषु।\end{sloppypar}
\begin{sloppypar}\hyphenrules{nohyphenation}\justifying\noindent\hspace{10mm} \textcolor{red}{जन्म}~– स्वामि\-रामभद्राचार्याणां जनिरुत्तरप्रदेशे जौनपुर\-जनपदान्तर्वर्तिनि शाण्डिखुर्द\-नामके ग्रामे द्विसहस्राधिक\-षष्ठे वैक्रमेऽब्दे माघकृष्णैकादश्यां तदनुसारं पञ्चाशदुत्तरैकोन\-विंशतिशत ईसवीयाब्दे मकर\-सङ्क्रान्तौ (१४ जनवरी १९५०) वैवस्वत\-वासरे निशि निशीथाभिमुख्यां 
मकरराशिगते सवितरि वसिष्ठ\-गोत्रीय\-सरयूपारीण\-विप्रामले कुले विश्व\-विश्रुत\-ब्राह्मण\-कुलालङ्कारस्य श्रीराजदेव\-मिश्रस्य चतुर्थापत्य\-रूपेणाखण्ड\-सौभाग्य\-वत्याः श्रीमत्याः शची\-देव्या दक्षिण\-कुक्षितः समभवत्। एतेषां नामकरणं \textcolor{red}{गिरिधरलाल} इति कृत्वा पितरौ मोदमावहन्तौ मासद्वयावधिमेव यापितवन्तौ हा हन्त तदानीमेव 
दैवदुर्विपाकात् \textcolor{red}{रोहुआ}\-रुजा\-प्रकोपाद्बालकस्य बाह्यदृष्टिर्गता। वस्तुतस्त्वसारमेतं संसारं नावलोकयितु\-कामेनानेन नवजात\-शिशुनैव द्विमासेन भौतिके चक्षुषी निमीलिते इत्युत्प्रेक्षामहे। एतान् दृष्टि\-बाधितान् विज्ञाय महद्दुःखमनुभवन्तः पितामहाः श्रीसूर्यबलि\-मिश्र\-महोदयास्तान् श्रीमद्भगवद्गीतां वारं वारं श्रावितवन्तः। तेन पञ्चवर्षस्यावस्थायामेव जन्मान्तरीय\-प्रतिभा\-प्रागल्भ्य\-भगवद्भजन\-साधन\-धना विलक्षण\-प्रतिभाशीलाः प्रज्ञाचक्षुष्मन्तो गिरिधरनामानोऽशेषां गीतां कण्ठस्थीकृतवन्तः।\end{sloppypar}
\begin{sloppypar}\hyphenrules{nohyphenation}\justifying\noindent\hspace{10mm} \textcolor{red}{शिक्षार्जनम्}~– बाल्ये गिरिधर\-लालाभिधेयानां जगद्गुरु\-रामभद्राचार्याणां प्राथमिकी शिक्षा पितामह\-सन्निधावेव सम्पन्ना। एते जानकी\-जीवन\-लीला\-लालित्य\-लुब्ध\-धियो पितामह\-चरणेभ्यो गोस्वामि\-तुलसीदास\-प्रणीतं श्रीमद्रामचरितमानसं साङ्गोपाङ्गं सानुपूर्वीकं पङ्क्तिसङ्ख्यासहितं द्वाभ्यामेव मासाभ्यां कण्ठस्थीकृत्याष्टम एव वर्षे रामनवम्यामश्रावयन्। एतेषां विशिष्टां प्रज्ञां लोचं लोचं पितामहाः सूर्यबलि\-महोदयाः पितरौ च संस्कृतस्य पारम्परिक\-शिक्षां दापयितुं निकटवर्तिनं सुजानपुरस्थं श्रीगौरीशङ्कर\-संस्कृत\-महाविद्यालयमेतान् प्रैषयन्। कुशाग्र\-बुद्धि\-सम्पन्ना एकश्रुता\footnote{एकपाठिन एकसन्धिग्राहिणो वा।} गिरिधरमिश्रास्तत्र प्रारम्भिक\-व्याकरण\-शिक्षां तत्रत्य\-व्याकरण\-विभागाध्यक्षेभ्यः श्रीशीतला\-प्रसाद\-मिश्रेभ्यः सम्प्राप्तवन्तः श्लोक\-संस्कार\-रचनां च समधिगतवन्तः तत्रत्य\-साहित्य\-विभागाध्यक्षेभ्यः श्रीराम\-मनोरथ\-त्रिपाठि\-महाभागेभ्यः। श्रीगौरीशङ्कर\-संस्कृत\-महाविद्यालये नव्यव्याकरणं मुख्यविषयत्वेन स्वीकृत्य प्रथमातो मध्यमाश्रेणी\-पर्यन्तं सर्वाः परीक्षाः सविषेशाङ्कं सर्वप्राथम्येन प्रथम\-श्रेण्या समुत्तीर्योच्च\-शिक्षार्थं वाराणस्यां सम्पूर्णानन्द\-संस्कृत\-विश्वविद्यालयं प्रविष्टवन्तः। तत्कालीन\-व्याकरण\-विभागाध्यक्षाणां श्रीभूपेन्द्र\-पति\-त्रिपाठिनां शीतल\-स्नेह\-च्छायायां \textcolor{red}{वैयाकरण\-भूषण\-सारम्} एवं शाब्दिक\-शिरोमणि\-श्रीपण्डित\-कालिका\-प्रसाद\-शुक्ल\-महाभागानां श्रीचरण\-कमल\-सन्निधावन्यांश्च व्याकरण\-टीका\-ग्रन्थान् समधीत्य पुनश्चातिरिक्तसमये प्रतिदिनं सायं षड्ढोरावधिं\footnote{\textcolor{red}{कालाध्वनोरत्यन्त\-संयोगे} (पा॰सू॰~२.३.५) इत्यनेन द्वितीया।} समुपविश्य शब्द\-सागर\-मन्दर\-मति\-व्याकरण\-विभागाध्यक्ष\-चराणानामभिनव\-पाणिनीनां डॉ॰\-रामप्रसाद\-त्रिपाठिनां सन्निधौ भाष्यान्त\-व्याकरण\-ग्रन्थानामन्येषां नव्यव्याकरणस्य पद\-वाक्य\-प्रमाण\-ग्रन्थानां सपरिष्कारं निगूढ\-तत्त्व\-बोध\-पुरःसरं विशेषमध्ययनं विधिवत्कुर्वाणाः शास्त्रिकक्षामीसवीयाब्दे चतुःसप्तत्युत्तरैकोन\-विंशतिशते विश्वविद्यालये सर्वोत्तमाङ्कैः सह समुत्तीर्य प्रथमं स्थानं लब्धवन्तः। आचार्यकक्षायामधीयानाः सन्तो भारत\-सर्वकार\-द्वारा प्रायोजितास्वखिल\-भारतीय\-प्रतियोगितासु सर्वानपि प्रतियोगिनः प्रतिभया परिभाव्य वाद\-विवादान्त्याक्षरी\-समस्यापूर्ति\-व्याकरण\-साङ्ख्येति\-पञ्च\-प्रतियोगितासु प्रथमं स्थानं लब्धवन्तः। तत्र पुरस्कर्तुं स्वयं समागता तत्कालीना प्रधान\-मन्त्रिणी श्रीमतीन्दिरागान्धी पञ्चपुरस्कारैः सहोत्तरप्रदेशस्य कृते विशाल\-रजत\-पट्टिका\-रूपं चलवैजयन्ती\-पुरस्कारमपि ससाधुवादमदात्। ईसवीयाब्दे षट्सप्तत्युत्तरैकोन\-विंशतिशत आचार्य\-कक्षां विश्वविद्यालयस्य सर्वेष्वपि विभागेषु सर्वाधिकाङ्कैः सह समुत्तीर्य विशिष्टां कीर्तिमाभजद्भ्यः स्वामि\-वर्येभ्यो गिरिधर\-लालाभिधेयेभ्यः सप्तस्वर्ण\-पदकानि कुलाधिपति\-स्वर्ण\-पदकं चामिलन्। सम्पूर्णानन्द\-संस्कृत\-विश्वविद्यालयस्यैव पी॰एच्॰डी॰ (विद्यावारिधिः) इत्युपाधये \textcolor{red}{अध्यात्म\-रामायणेऽपाणिनीय\-प्रयोगाणां विमर्शः} इति विषये शोध\-प्रबन्धमपि लिखित\-वन्तोऽनुसन्धान\-विधया केवलैस्त्रयोदशभिरेव दिवसैः। तथा च \textcolor{red}{पाणिनीयाष्टाध्याय्याः प्रतिसूत्रं शाब्दबोधसमीक्षा} इति विषये द्विसहस्रपृष्ठात्मकं शोधप्रबन्धं प्रणीय तस्मादेव विश्वविद्यालयतो डी॰लिट्॰ (वाचस्पतिः) इत्युपाधिमप्यलभन्त।\end{sloppypar}
\begin{sloppypar}\hyphenrules{nohyphenation}\justifying\noindent\hspace{10mm} \textcolor{red}{विरक्तदीक्षोत्तरजीवनम्}~– १९८३ ईसवीयाब्दे कार्त्तिक\-शुक्ल\-पौर्णमास्यां श्रीरामानन्द\-वैष्णव\-सम्प्रदाये विरक्तवेषो गृहीतो गिरिधर\-लाल\-मिश्र\-चरणैः श्रीमदलर्क\-पुरी\-प्रयाग\-निवासिभ्यः फलाहारि\-सञ्ज्ञया प्रसिद्धेभ्यः श्रीश्रीरामचरण\-दास\-महाभागेभ्यः। यद्यपि मिश्रचरणैः श्रीराममन्त्रो विरक्त\-वैष्णव\-सम्प्रदाय\-दीक्षा च गृहीते जीवनस्याष्टम एव वर्षे श्रीमद्राम\-वल्लभा\-शरण\-महाराज\-चरणारविन्द\-प्रमुख\-कृपा\-पात्रेभ्यः श्रीमदीश्वर\-दास\-महाभागेभ्यो विरक्तवेष\-समकालमेव पञ्चसंस्कार\-विधिनाऽमीषां पूर्वाश्रमौप\-चारिकताऽपीतिहासपृष्ठं समधिशिश्रिये। अतोऽच्युत\-गोत्राणामेषां महात्मनां \textcolor{red}{रामभद्रदासः} इति नाम ख्यातिमगमत्। रामभद्रदासाः समष्टि\-मङ्गलाय १९८७ ईसवीये तुलसीजयन्त्यां मन्दाकिनी\-विमल\-सलिलासक्त आमोदवने श्रीतुलसीपीठस्य प्रामाणिकीं प्रतिष्ठां विधाय श्रीचित्रकूट\-तुलसी\-पीठाधीश्वर\-पदालङ्कृता जाताः। नारदभक्तिसूत्र ईशाद्येकादशोपनिषत्सु श्रीमद्भगवद्गीतायां ब्रह्मसूत्रे च श्रीराघवकृपाभाष्यं विलिख्य प्रस्थानत्रयी\-भाष्य\-कारत्वेन प्रथितयशसः सारस्वत्या प्रतिभया योग्यतया स्वामिवर्यान् श्रीरामानन्द\-सम्प्रदायस्य जगद्गुरुपदे काशीविद्वत्परिषत्सम्मत्याऽखाड़ा\-परिषन्महान्तो मनीषिणो गण्यमान्या विद्वांसो जगद्गुरवश्च समवेतघोषणेन ससम्मानं प्रतिष्ठापितवन्तोऽभि\-षेचितवन्तश्च। एवं हि स्वामिवर्याः श्रीचित्रकूट\-तुलसी\-पीठाधीश्वर\-प्रस्थानत्रयी\-भाष्यकार\-जगद्गुरु\-रामानन्दाचार्य\-स्वामि\-रामभद्राचार्या इत्याख्येन भुवि लब्धयशोराशयः सततमेव विकलाङ्ग\-जनतोपासनायां समष्टि\-हिताय राष्ट्रसेवायै च समर्पितजीवना राजमाना जाताः।\end{sloppypar}
\begin{sloppypar}\hyphenrules{nohyphenation}\justifying\noindent\hspace{10mm} \textcolor{red}{विशेषः}~– निगूढानां शास्त्रविषयाणामवगमनं नव\-नवोन्मेष\-शालिनी प्रतिभा\-कल्पना\-कमनीयता तर्ककर्कश\-मस्तिष्कं प्रौढ\-पाण्डित्यं गुरुजने विनम्रता नैष्ठिकं ब्रह्मचर्यं श्रीरामोपासना\-रुचिश्चामीषां देशिकवर्याणां निसर्गसिद्धा विशेषता। काव्य\-रचना\-पाटवं त्वमीषां सार्धत्रिवर्षावस्थायामेव सुस्पष्टमुद्भूतम्। राष्ट्रभाषा\-संस्कृत\-भाषयोरप्याशु\-कवित्वं सहजसिद्धम्। एभिः श्रीचित्रकूटे नव पयोव्रतान्यापि विशालानि षण्मासपर्यन्तानि विहितानि। शरीरस्य वसिष्ठ\-गोत्रीयत्वादिमे शिशु\-रूप\-राघवमेव समुपासते लालयन्ति च वात्सल्य\-भावनया। देवतुल्यसमाराध्या अस्माकं सद्गुरवः जगद्गुरु\-स्वामि\-रामभद्राचार्य\-महाराजाः शाश्वतं भासन्तामिह लोके यावच्चन्द्र\-दिवाकराविति मङ्गलं कामयमानास्तेषां चरणकमलयोरर्पयन्तश्च प्रणतीः कथयन्त इदं विरमामः~–\end{sloppypar}
\vspace{-2mm}
\begin{center}
रामभद्रो हि जानाति रामभद्रसरस्वतीम्।\nopagebreak\\
रामभद्रो हि जानाति रामभद्रसरस्वतीम्॥\nopagebreak\\
\end{center}
\begin{sloppypar}\hyphenrules{nohyphenation}\justifying\noindent\hspace{10mm} \textcolor{red}{ग्रन्थसूची}~– जगद्गुरु\-रामभद्राचार्य\-प्रणीताः प्रमुख\-ग्रन्था अधोलिखिताः सन्ति।\end{sloppypar}
\begin{itemize}
\item पद्यकृतयः
	\begin{itemize}
	\item महाकाव्यानि
	\begin{itemize}
		\item अरुन्धती (१९९४)। हिन्द्याम्।
		\item श्रीभार्गवराघवीयम् (२००२)। संस्कृते। हिन्द्यनुवादसहितम्।
		\item अष्टावक्र (२०१०)। हिन्द्याम्।
		\item गीतरामायणम् (२०११)। संस्कृते। हिन्द्यनुवादसहितम्।
	\end{itemize}
	\item खण्डकाव्यानि
	\begin{itemize}
		\item काका विदुर (१९८०)। हिन्द्याम्।
		\item माँ शबरी (१९८२)। हिन्द्याम्।
		\item आजादचन्द्रशेखरचरितम् (१९९६)। संस्कृते। हिन्द्यनुवादसहितम्।
		\item लघुरघुवरम् (२००१)। संस्कृते। हिन्दीपद्यगद्यानुवादसहितम्।
		\item श्रीसरयूलहरी (२००१)। संस्कृते। हिन्द्यनुवादसहिता।
		\item भृङ्गदूतम् (२००४)। संस्कृते। हिन्द्यनुवादसहितम्।
		\item अवध कै अँजोरिया (२०११)। अवध्याम्। 
		\item श्रीसीतासुधानिधिः (२०११)। संस्कृते। हिन्द्यनुवादसहितः।
	\end{itemize}
	\item पत्रकाव्यम्
	\begin{itemize}
		\item कुब्जापत्रम् (२००३)। संस्कृते। हिन्द्यनुवादसहितम्।
	\end{itemize}
	\item गीतकाव्यानि
	\begin{itemize}
		\item राघवगीतगुञ्जन (१९९१)। हिन्द्याम्।
		\item भक्तिगीतसुधा (१९९३)। हिन्द्याम्।
	\end{itemize}
	\item रीतिकाव्यम्
	\begin{itemize}
		\item श्रीसीतारामकेलिकौमुदी (२००८)। ब्रजभाषायाम्। हिन्द्यनुवादसहिता।
	\end{itemize}
	\item शतककाव्यानि
	\begin{itemize}
		\item आर्याशतकम् (१९९६)। संस्कृते। हिन्द्यनुवादसहितम्।
		\item श्रीगणपतिशतकम् (१९९६)। संस्कृते। हिन्द्यनुवादसहितम्।
		\item श्रीचण्डीशतकम् (१९९६)। संस्कृते। हिन्द्यनुवादसहितम्।
		\item श्रीराघवेन्द्रशतकम् (१९९६)। संस्कृते। हिन्द्यनुवादसहितम्।
		\item श्रीरामभक्तिसर्वस्वम् (१९९७)। संस्कृते। हिन्द्यनुवादसहितम्।
		\item श्रीराघवचरणचिह्नशतकम् (२००१)। संस्कृते। हिन्द्यनुवादसहितम्।
		\item श्रीजानकीचरणचिह्नशतकम् (२००१)। संस्कृते। हिन्द्यनुवादसहितम्।
		\item मन्मथारिशतकम् (२००७)। संस्कृते। हिन्द्यनुवादसहितम्।
		\item श्रीरघुनाथशतकम् (२०११)। संस्कृते। हिन्द्यनुवादसहितम्।
	\end{itemize}
	\item स्तोत्रकाव्यानि
	\begin{itemize}
		\item श्रीहनुमच्चत्वारिंशिका (१९८३)। संस्कृते। हिन्द्यनुवादसहिता।
		\item श्लोकमौक्तिकम् (१९८३)। संस्कृते। हिन्द्यनुवादसहितम्।
		\item मुकुन्दस्मरणम् (१९९६)। संस्कृते। हिन्द्यनुवादसहितम्।
		\item श्रीजानकीकृपाकटाक्षस्तोत्रम् (१९९६)। संस्कृते।हिन्द्यनुवादसहितम्।
		\item भक्तिसारसर्वस्वम् (१९९७)। संस्कृते। हिन्द्यनुवादसहितम्।
		\item श्रीगङ्गामहिम्नःस्तोत्रम् (१९९८)। संस्कृते। हिन्द्यनुवादसहितम्।
		\item नमोराघवायाष्टकम् (२००१)। संस्कृते। हिन्द्यनुवादसहितम्।
		\item श्रीचित्रकूटविहार्यष्टकम् (२००१)। संस्कृते। हिन्द्यनुवादसहितम्।
		\item श्रीरामवल्लभास्तोत्रम् (२००१)। संस्कृते। हिन्द्यनुवादसहितम्।
		\item श्रीराघवभावदर्शनम् (२००२)। संस्कृते। हिन्द्यनुवादसहितम्।
		\item चरणपीडाहराष्टकम् (२००८)। संस्कृते। हिन्द्यनुवादसहितम्।
		\item सर्वरोगहराष्टकम् (२०१०)। संस्कृते। हिन्द्यनुवादसहितम्।
	\end{itemize}
	\item सुप्रभातकाव्यम्
	\begin{itemize}
		\item श्रीसीतारामसुप्रभातम् (२००९)। संस्कृते। हिन्द्यनुवादसहितम्।
	\end{itemize}
	\item वृत्तिकाव्यम्
	\begin{itemize}
		\item अष्टाध्याय्याः प्रतिसूत्रं शाब्दबोधसमीक्षा (१९९७)। संस्कृते।
	\end{itemize}
	\end{itemize}
\item नाटके
	\begin{itemize}
	\item उत्साह (१९९६)। हिन्द्याम्।
	\item श्रीराघवाभ्युदयम् (१९९६)। संस्कृते। हिन्द्यनुवादसहितम्।
	\end{itemize}
\item गद्यकृतयः
	\begin{itemize}
	\item प्रस्थानत्रय्यां श्रीराघवकृपाभाष्याणि
		\begin{itemize}
		\item श्रीब्रह्मसूत्रेषु श्रीराघवकृपाभाष्यम् (१९९८)। संस्कृते हिन्द्यां च।
		\item श्रीमद्भगवद्गीतासु श्रीराघवकृपाभाष्यम् (१९९८)। संस्कृते हिन्द्यां च।
		\item कठोपनिषदि श्रीराघवकृपाभाष्यम् (१९९८)। संस्कृते हिन्द्यां च।
		\item केनोपनिषदि श्रीराघवकृपाभाष्यम् (१९९८)। संस्कृते हिन्द्यां च।
		\item माण्डूक्योपनिषदि श्रीराघवकृपाभाष्यम् (१९९८)। संस्कृते हिन्द्यां च।
		\item ईशावास्योपनिषदि श्रीराघवकृपाभाष्यम् (१९९८)। संस्कृते हिन्द्यां च।
		\item प्रश्नोपनिषदि श्रीराघवकृपाभाष्यम् (१९९८)। संस्कृते हिन्द्यां च।
		\item तैत्तिरीयोपनिषदि श्रीराघवकृपाभाष्यम् (१९९८)। संस्कृते हिन्द्यां च।
		\item ऐतरेयोपनिषदि श्रीराघवकृपाभाष्यम् (१९९८)। संस्कृते हिन्द्यां च।
		\item श्वेताश्वतरोपनिषदि श्रीराघवकृपाभाष्यम् (१९९८)। संस्कृते हिन्द्यां च।
		\item छान्दोग्योपनिषदि श्रीराघवकृपाभाष्यम् (१९९८)। संस्कृते हिन्द्यां च।
		\item बृहदारण्यकोपनिषदि श्रीराघवकृपाभाष्यम् (१९९८)। संस्कृते हिन्द्यां च।
		\item मुण्डकोपनिषदि श्रीराघवकृपाभाष्यम् (१९९८)। संस्कृते हिन्द्यां च।
		\end{itemize}
	\item अन्ये श्रीराघवकृपाभाष्ये
		\begin{itemize}
		\item श्रीनारदभक्तिसूत्रेषु श्रीराघवकृपाभाष्यम् (१९९१)। संस्कृते हिन्द्यां च।
		\item श्रीरामस्तवराजस्तोत्रे श्रीराघवकृपाभाष्यम् (२०००)। संस्कृते हिन्द्यां च।
		\end{itemize}
	\item हिन्दीभाष्याणि
		\begin{itemize}
		\item श्रीहनुमानचालीसायां महावीरी टीका (१९८३)। हिन्द्याम्।
		\item श्रीरामचरितमानसे भावार्थबोधिनी टीका (२००५)। हिन्द्याम्।
		\item श्रीभक्तमाले मूलार्थबोधिनी टीका (२०१४)। हिन्द्याम्।
		\end{itemize}
	\item विमर्शौ
		\begin{itemize}
		\item अध्यात्मरामायणेऽपाणिनीयप्रयोगानां विमर्शः (१९८१)। संस्कृते।
		\item श्रीरासपञ्चाध्यायीविमर्शः (२००७)। हिन्द्याम्।
		\end{itemize}
	\item प्रवचनसङ्ग्रहाः
		\begin{itemize}
		\item मानस में तापस प्रसंग (१९८२)। हिन्द्याम्।
		\item सुग्रीव का अघ और विभीषण की करतूति (१९८५)। हिन्द्याम्।
		\item श्रीगीतातात्पर्य (१९८५)। हिन्द्याम्।
		\item सनातनधर्म की विग्रहस्वरूप गोमाता (१९८८)। हिन्द्याम्।
		\item श्रीतुलसीसाहित्य में कृष्णकथा (१९८८)। हिन्द्याम्।
		\item मानस में सुमित्रा (१९८९)। हिन्द्याम्।
		\item सीता निर्वासन नहीं (१९९०)। हिन्द्याम्।
		\item प्रभु करि कृपा पाँवरी दीन्ही (१९९२)। हिन्द्याम्।
		\item परम बड़भागी जटायु (१९९३)। हिन्द्याम्।
		\item श्रीसीताराम विवाह दर्शन (२००१)। हिन्द्याम्।
		\item तुम पावक मँह करहु निवासा (२००४)। हिन्द्याम्।
		\item अहल्योद्धार (२००६)। हिन्द्याम्।
		\item हर ते भे हनुमान (२००८)। हिन्द्याम्।
		\item सत्य रामप्रेमी श्रीदशरथ (२००९)। हिन्द्याम्।
		\item वेणुगीत (२०११)। हिन्द्याम्।
		\end{itemize}
	\end{itemize}
\end{itemize}
