% Nityanand Misra: LaTeX code to typeset a book in Sanskrit
% Copyright (C) 2016 Nityanand Misra
%
% This program is free software: you can redistribute it and/or modify it under
% the terms of the GNU General Public License as published by the Free Software
% Foundation, either version 3 of the License, or (at your option) any later
% version.
%
% This program is distributed in the hope that it will be useful, but WITHOUT
% ANY WARRANTY; without even the implied warranty of  MERCHANTABILITY or FITNESS
% FOR A PARTICULAR PURPOSE. See the GNU General Public License for more details.
%
% You should have received a copy of the GNU General Public License along with
% this program.  If not, see <http://www.gnu.org/licenses/>.

\setcounter{footnote}{0}
\renewcommand\chaptername{}
\chapter[परिचयः]{परिचयः}
\markboth{परिचयः}{}
\fontsize{14}{21}\selectfont
\begin{sloppypar}\hyphenrules{nohyphenation}\justifying\noindent\hspace{10mm} {\engtextfont \textbf{Adhyātmarāmāyaṇe’pāṇinīyaprayogāṇāṃ Vimarśaḥ} (English: \textit{Deliberation on non-Pāṇinian usages in the Adhyātma Rāmāyaṇa}) is a comprehensive Sanskrit essay (\textit{nibandha}) in around 50,000 words, authored in 1981 by my Gurudeva, Ācārya Giridharalāla Miśra Prajñācakṣu (known in his current Āśrama as Jagadguru Rāmanandācārya Svāmī Rāmabhadrācārya). The work was spontaneously dictated by Gurudeva over only thirteen days in 1981. A disciple of Gurudeva, Dayāśaṅkara Pāṇdeya, was the scribe who took dictation from Gurudeva. The entire work was authored by Gurudeva without any book or references, with all the text of \textit{Adhyātma Rāmāyaṇa}, \textit{Aṣṭādhyāyī}, \textit{Mahābhāṣya} and many other works in the memory (\textit{kaṇṭhastha}) and all the Pāṇinian \textit{prakriyā}‑s in the mind (\textit{buddhistha}). At the end of the work, Gurudeva says—}\end{sloppypar}
\vspace{-2mm}
\begin{center}
बुद्ध्या श्रीगुरुपादपद्मरजसा संशुद्धया सादरं\nopagebreak\\
कृत्वा लेखकमाप्तशीलयशसं शिष्यं शिशुं राघवम्।\nopagebreak\\
बालो नष्टविलोचनो गिरिधरः शब्दान् विभाव्याऽत्मना\nopagebreak\\
बध्नाति स्म निबन्धमेतममलं तोषाय सीतापतेः॥
\end{center}
\vspace{-2mm}
\begin{sloppypar}\hyphenrules{nohyphenation}\justifying\noindent\hspace{10mm} {\engtextfont \lqtwo Making the infant Rāma, his disciple endowed with moral conduct and fame, as the scribe, the child Giridhara, devoid of physical vision, after respectfully examining words with his intellect which was made especially pure by the pollen from the lotus-feet of the revered Guru, composed this work for the pleasure of the lord of Sītā.\rqtwo}\end{sloppypar}
\begin{sloppypar}\hyphenrules{nohyphenation}\justifying\noindent\hspace{10mm} {\engtextfont The work was then typed and presented as a doctoral thesis (\textit{śodha\-prabandha}) at the Sampurnanand Sanskrit University, for which the degree of \textit{Vidyāvāridhi} (Ph.D.) was conferred upon Gurudeva in 1981. The thesis was reviewed by the grammarian (\textit{vaiyākaraṇa}) and epic-poet (\textit{mahākavi}) Kālikāprasāda Śukla, who wrote the following verse in the \textit{Vasantatilakā} metre to describe it—}\end{sloppypar}
\vspace{-2mm}
\begin{center}
शोधप्रबन्धपरिशीलनतः समन्तात्सञ्जायते मतमिदं मम युक्तियुक्तम्।\nopagebreak\\
शोधप्रबन्धमकरन्दमधुव्रतोऽयं विद्वद्विमृग्यविरुदं लभतामिदानीम्॥
\end{center}
\vspace{-2mm}
\begin{sloppypar}\hyphenrules{nohyphenation}\justifying\noindent\hspace{10mm} {\engtextfont \lqtwo My logical conclusion, that arises from having thoroughly examined the dissertation, is that he (Giridhara Miśra) is the bumblebee for the nectar of literary compositions for purification. May he now [effortlessly] obtain the praise and fame which is especially sought after by the learned.\rqtwo}\end{sloppypar}
\begin{sloppypar}\hyphenrules{nohyphenation}\justifying\noindent\hspace{10mm} {\engtextfont The work is divided into four parts—\textit{Prastāvanā}, \textit{Sandhi\-kāraka\-samāsa\-prakaraṇam}, \textit{Kṛttaddhita\-prakaraṇam}, and \textit{Dhātu\-prakaraṇam}.}\end{sloppypar}
\begin{sloppypar}\hyphenrules{nohyphenation}\justifying\noindent\hspace{10mm} {\engtextfont \textbf{Prastāvanā}: The introduction begins with tracing the tradition of \textit{Vyākaraṇa}. A wide range of topics are first covered like nature of \textit{Veda}, origin of \textit{Veda}, \textit{apauruṣeyatā} of \textit{Veda}, \textit{vedatrayī} and \textit{vedacatuṣṭayī}, \textit{śruti} and \textit{Veda}, \textit{lakṣaṇa} of \textit{vidhi} and \textit{niṣedha}, relation between \textit{śruti} and \textit{smṛti}, five \textit{sampradāya}‑s based on \textit{smṛti}‑s, five types of \textit{Vaiṣṇava upāsanā}‑s, \textit{Vedānta}, \textit{Purāṇa}, \textit{Itihāsa}, fourteen \textit{vidyā}‑s including six \textit{darśana}‑s, three types of \textit{āgama}‑s and two types of \textit{mārga}‑s, and the six \textit{vedāṅga}‑s—the chief amongst which is \textit{Vyākaraṇa}. Next, the importance of \textit{Vyākaraṇa} is stressed. The nine \textit{vyākaraṇa}‑s are mentioned, following which some unique aspects of Pāṇini’s grammar are discussed. The terms \textit{āpta} and \textit{śiṣṭa} are defined, along with generic definitions of \textit{śiṣṭa\-prayoga} and \textit{sādhutva}, which are to be followed by \textit{Vyākaraṇa} and not the other way round. \textit{Sādhutva} is then redefined in the context of the \textit{prakriyā} of Pāṇini’s grammar. Deeper insights going beyond the realm of grammar are arrived at from some \textit{sūtra}‑s of Pāṇini. The essay then delves on \textit{sūtratva} and \textit{lakṣaṇa} of six types of \textit{sūtra}‑s and their interplay is explained with the comprehensive example of \textit{iko yaṇaci} (PS 6.1.77). This is followed by the position of works of Kātyāyana and Patañjali in the \textit{Vyākaraṇa} tradition. \textit{Mahābhāṣya} and its position is described in detail with examples. Following this, \textit{prakriyā} and \textit{darśana}—the two eyes of the \textit{Vyākaraṇa} tradition—are described with the works of both, and the philosophy of \textit{Vyākaraṇa} tradition is compared with that of other \textit{darśana}‑s. \textit{Śābdbodha} and \textit{śakti} are then explained as per \textit{Vyākaraṇa}. The relevance of Pāṇini’s grammar is discussed in the context of the \textit{Rāmāyaṇa} tradition, including the \textit{Adhyātma Rāmāyaṇa}. Some insights into the \textit{Rāmāyaṇa} tradition from Pāṇini’s grammar are discussed, and all the fourteen \textit{Śiva Sūtra}‑s are then interpreted in the context of \textit{Rāmāyaṇa}. The question of seemingly non-Pāṇinian usages in \textit{śiṣṭaprayoga}‑s, like those in \textit{Adhyātma Rāmāyaṇa}, is raised. Gurudeva says that it is very much possible to explain all \textit{śiṣṭaprayoga}‑s by Pāṇini’s grammar. After explaining the the relation between \textit{Vyākaraṇa} and \textit{Adhyātma Rāmāyaṇa}, the latter work is described in some detail. \textit{Adhyātma Rāmāyaṇa}’s origin, poetic features, \textit{rasa}‑s, \textit{nāyaka} and its relevance and usefulness in the context of the \textit{Rāmāyaṇa} tradition is discussed. The author stresses that since both Pāṇini’s grammar and \textit{Adhyātma Rāmāyaṇa} come from Lord Śiva, there must be consistency (\textit{ekavākyatā}) between the two. Gurudeva says that in the \textit{Adhyātma Rāmāyaṇa}, there are 700-odd usages that “appear to be non-Pāṇinian.” Around half of these usages are examined and explained using the Pāṇinian framework (others being similar to those explained).}\end{sloppypar}
\begin{sloppypar}\hyphenrules{nohyphenation}\justifying\noindent\hspace{10mm} {\engtextfont \textbf{I. Sandhi\-kāraka\-samāsa\-prakaraṇam}: The first chapter is split into two parts and examines 140 sequential usages in the \textit{Adhyātma Rāmāyaṇa} pertaining to \textit{sandhi}, \textit{kāraka} and \textit{samāsa}. It begins with an insightful grammatical explanation of the compound \textit{Adhyātma Rāmāyaṇa}. Quite often two or three solutions, and sometimes upto six or seven solutions in the Pāṇinian framework are given for the seemingly non-Pāṇinian forms.}\end{sloppypar}
\begin{sloppypar}\hyphenrules{nohyphenation}\justifying\noindent\hspace{10mm} {\engtextfont \textbf{II. Kṛttaddhita\-prakaraṇam}: The second chapter is also split into two parts and examines 90 sequential usages in the \textit{Adhyātma Rāmāyaṇa} pertaining to \textit{kṛt} and \textit{taddhita} affixes. Again, multiple Pāṇinian explanations are offered for many usages.}\end{sloppypar}
\begin{sloppypar}\hyphenrules{nohyphenation}\justifying\noindent\hspace{10mm} {\engtextfont \textbf{IIII. Dhātu\-prakaraṇam}: The third chapter explains 135 sequential \textit{tiṅanta} usages in the \textit{Adhyātma Rāmāyaṇa} that are seemingly non-Pāṇinian. Mostly one and sometimes two or three explanations are offered in the Pāṇinian framework for these usages.}\end{sloppypar}
\vspace{4mm}
\begin{sloppypar}\hyphenrules{nohyphenation}\justifying\noindent {\engtextfont \textbf{Features}}\end{sloppypar}
\begin{sloppypar}\hyphenrules{nohyphenation}\justifying\noindent\hspace{10mm} {\engtextfont Some unique features of the work include—}\end{sloppypar}
\begin{enumerate}[itemsep=0mm,label={{\engtextfont \arabic*.}}]
\item \begin{sloppypar}\hyphenrules{nohyphenation}\justifying\noindent {\engtextfont Pāṇinian explanations of 365 seemingly non-Pāṇinian usages using various Pāṇinian \textit{sūtra}‑s, \textit{vārttika}‑s, \textit{kārikā}‑s, \textit{niyama}‑s, \textit{pari\-bhāṣā}‑s, and \textit{jñāpaka}‑s, in accordance with traditional commentaries.}\end{sloppypar}
\item \begin{sloppypar}\hyphenrules{nohyphenation}\justifying\noindent {\engtextfont More than 1500 citations from works of diverse fields, and many more from oral traditions. On including the editor’s footnotes and derivations in the third edition, there are more than 6500 citations from around 175 works.}\end{sloppypar}
\item \begin{sloppypar}\hyphenrules{nohyphenation}\justifying\noindent {\engtextfont \textit{Śiva Sūtra}‑s explained in the context of \textit{Rāmāyaṇa}, with a \textit{Rāmāyaṇa}-centric interpretation of each \textit{sūtra}. This section of the work compares with the \textit{Nandikeśvara Kāśikā}.}\end{sloppypar}
\item \begin{sloppypar}\hyphenrules{nohyphenation}\justifying\noindent {\engtextfont Hundreds of Pāṇinian \textit{prakriyā}‑s, some abridged and some detailed. In the second addition, hundreds of complete Pāṇinian derivations are given in the editor’s footnotes, corresponding to the abridged and detailed derivations by the author.}\end{sloppypar}
\item \begin{sloppypar}\hyphenrules{nohyphenation}\justifying\noindent {\engtextfont Lucid explanations of complex grammatical concepts.}\end{sloppypar}
\item \begin{sloppypar}\hyphenrules{nohyphenation}\justifying\noindent {\engtextfont Vivid descriptions in poetic style, with some \textit{daṇḍaka}-style \textit{samāsa}‑s formed from several hundreds of words compounded together.}\end{sloppypar}
\item \begin{sloppypar}\hyphenrules{nohyphenation}\justifying\noindent {\engtextfont Didactic approach with many counter-questions and doubts raised by \textit{nanu}, \textit{na ca}, et cetera, and all of them resolved in favour of the proposed solutions.}\end{sloppypar}
\item \begin{sloppypar}\hyphenrules{nohyphenation}\justifying\noindent {\engtextfont \textit{Nyāya}-styled \textit{lakṣaṇa}‑s of many concepts, both grammatical and non-grammatical.}\end{sloppypar}
\item \begin{sloppypar}\hyphenrules{nohyphenation}\justifying\noindent {\engtextfont \textit{Vaiyākaraṇa śābdabodha}‑s of common terms (\textit{śruti}, \textit{veda}, \textit{anuśāsana}, \textit{vyākaraṇa}, et cetera) and involved grammatical usages (\textit{vāraṇārtha}, \textit{karmamūlaka\-sambandha}, \textit{samāsa}, \textit{śaiṣika\-ṣaṣṭhī}, \textit{tatkaroti} usage, \textit{tadiva ācarati} usage, \textit{tiṅanta} usage from a \textit{pacādyajanta\-kvibanta nāmadhātu}, \textit{ṇijanta} usage, \textit{svārtha\-ṇijanta} usage, \textit{vartamāna\-sāmīpya} usage, et cetera).}\end{sloppypar}
\item \begin{sloppypar}\hyphenrules{nohyphenation}\justifying\noindent {\engtextfont Critical insights into many original verses of \textit{Adhyātma Rāmāyaṇa} using traditional methods of interpretation. The work may be thought of as a mini-commentary on the \textit{Adhyātma Rāmāyaṇa}.}\end{sloppypar}
\end{enumerate}
\begin{sloppypar}\hyphenrules{nohyphenation}\justifying\noindent\hspace{10mm} {\engtextfont Besides satisfying scholars of \textit{Vyākaraṇa} and the connoisseurs of \textit{Adhyātma Rāmāyaṇa}, the work is very useful for students learning Pāṇini’s \textit{Vyākaraṇa}. Having acquired all my limited learning in Sanskrit grammar exclusively from self-study (\textit{svādhyāya}), I can personally attest to the extraordinary benefits the work offers for grammar students, especially those studying by \textit{svādhyāya}. Just like a picture is worth a thousand words, similarly a \textit{prakriyā} is worth the understanding of tens of \textit{sūtra}‑s, a grammatical insight is worth tens of such \textit{prakriyā}‑s, and a \textit{śābdabodha} is worth tens of such grammatical insights. The work—replete with \textit{prakriyā}‑s, insights and \textit{śābdabodha}‑s—offers a rare source of learning for students and scholars of Pāṇinian \textit{Vyākaraṇa}.}\end{sloppypar}
\begin{sloppypar}\hyphenrules{nohyphenation}\justifying\noindent\hspace{10mm} {\engtextfont I am a quantitative analyst by profession, and Applied Statistics is one of my areas of work. From the viewpoint of Statistics, I see the Pāṇinian grammar as a parsinomious statistical model which Pāṇini formulated to explain the variation in the infinitely many \textit{śiṣṭa prayoga}‑s which formed his dataset. No statistical model with finite predictors can explain all of the variation in an infinitely large dataset. Pāṇini’s model was the best model ever formulated for this purpose, and could account for a very large degree of variation. Vararuci’s \textit{vārttika}‑s and Patañjali’s \textit{bhāṣya} extended the model by introducing additional model complexity, and explaining further variation in the data of \textit{śiṣṭa prayoga}‑s. All of these great grammarians were mathematicians, and I see them as statisticians since like Statistics, Sanskrit grammar also combines mathematics and philosophy—the two eyes of Sanskrit grammar being \textit{prakriyā} (mathematical derivations) and \textit{darśana} (philosophy). This work is also largely mathematical in nature, interpreting or extending Pāṇini’s model to explain the variation of \textit{śiṣṭa prayoga}‑s seen in the \textit{Adhyātma Rāmāyaṇa}. With this conclusion, I believe students and scholars of computational liguistics also stand to gain from the study of this work.}\end{sloppypar}
\begin{sloppypar}\hyphenrules{nohyphenation}\justifying\noindent\hspace{10mm} {\engtextfont The original work and the editor’s footnotes in the third edition (except for some citations) are entirely in Sanskrit. For the benefit of students of Sanskrit and linguistics, I plan to translate the work into English and Hindi some day. The next edition of the work will hopefully come with an English or Hindi translation.}\end{sloppypar}
\vspace{5mm}
\raggedleft{{\engtextfont Nityānanda Miśra}}\\
\raggedleft{{\engtextfont Mumbai, August 22 2015}}\\
