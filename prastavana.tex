% Nityanand Misra: LaTeX code to typeset a book in Sanskrit
% Copyright (C) 2016 Nityanand Misra
%
% This program is free software: you can redistribute it and/or modify it under
% the terms of the GNU General Public License as published by the Free Software
% Foundation, either version 3 of the License, or (at your option) any later
% version.
%
% This program is distributed in the hope that it will be useful, but WITHOUT
% ANY WARRANTY; without even the implied warranty of  MERCHANTABILITY or FITNESS
% FOR A PARTICULAR PURPOSE. See the GNU General Public License for more details.
%
% You should have received a copy of the GNU General Public License along with
% this program.  If not, see <http://www.gnu.org/licenses/>.

\renewcommand\chaptername{अथ प्रस्तावना}
\chapter[प्रस्तावना]{प्रस्तावना}
\fontsize{14}{21}\selectfont
\begin{sloppypar}\hyphenrules{nohyphenation}\justifying\noindent\hspace{10mm} वेदो वै विगलित\-विभेदो विखात\-खेदो विच्छिन्न\-दोषच्छेदो निरस्ताशेष\-भ्रम\-प्रमाद\-विप्रलिप्सा\-करणापाटवादि\-पुंदोष\-शङ्का\-पङ्क\-कलङ्कावकाशो विलसित\-भगवल्लीला\-विलासो विलसति लीला\-गृहीत\-दिव्य\-विग्रहस्य विहित\-खल\-गण\-निग्रहस्य निखिल\-विभूति\-सङ्ग्रहस्य सीता\-परिग्रहस्य नारायणादि\-सहस्र\-पर\-नाम\-लक्ष्यतावच्छेदकतया सम्प्रतीतस्य दाशरथेः सन्निहित\-सच्चिदानन्द\-शर्मणो धर्म\-वर्मणो व्यापक\-ब्रह्मणः पर\-ब्रह्मणो निश्श्वास\-भूतो लोकाभिरामस्य श्रीरामस्येति निखिल\-तार्किक\-विपश्चिदपश्चिम\-मनस्वि\-मानिता मान्यता। अतो \textcolor{red}{यस्य निश्श्वसितं वेदाः} (ऋ॰वे॰सं॰ सा॰भा॰ उ॰प्र॰~२) इत्यभियुक्तोक्तिः। भाषा\-विद्वांसोऽपि तुलसीदास\-प्रभृति\-मनीषिणः कवयो मान्यतामिमां नतमस्तका मानयन्तो महीयन्ते। यथा रामचरितमानसे श्रीतुलसीदासः कथयति~–\end{sloppypar}
\centering\textcolor{red}{जाकी सहज श्वास श्रुति चारी। सो हरि पढ़ यह कौतुक भारी॥}\footnote{एतद्रूपान्तरम्–\textcolor{red}{यस्य स्वाभाविकः श्वासश्चतस्रः श्रुतयो मताः। स हरिः पठतीत्येतत्परमं कौतुकं स्थितम्॥} (मा॰भा॰~१.२०४.५)।}\nopagebreak\\
\raggedleft{–~रा॰च॰मा॰~१.२०४.५}\\
\begin{sloppypar}\hyphenrules{nohyphenation}\justifying\noindent\hspace{10mm} अस्यापौरुषेयताऽपि निर्विवादा। यद्यपि वैयाकरण\-धौरेयाः शब्दं ब्रह्मेत्यामनन्ति तथा च पातञ्जल\-महाभाष्य\-पस्पशाह्निके प्राञ्जलिः पतञ्जलिः शब्द\-नित्यत्वं प्रतिपादयन् वार्त्तिकमाह \textcolor{red}{सिद्धे शब्दार्थ\-सम्बन्धे} (भा॰प॰) इति तथाऽपि शब्द\-योजना\-क्रमोल्लङ्घन\-धिया वेदातिरिक्त\-वाङ्मयस्य पौरुषेयता। अर्थाद्भगवतो वेदस्य शब्द\-क्रमोऽपि सनातनः। यथा पूर्वं मन्त्राणां पाठस्तथाऽस्मिन् कल्पे तद्वत्पश्चाद्भाविनि कल्पेऽपि। कल्पादौ जीवैः सह स्वस्मिन्नेवोपसंहृत\-पूर्वो वेदोऽपि यथाऽऽनुपूर्व्या भगवतो ब्रह्मणो हृदये समुद्भासितो भगवता नारायणेन। तथा चास्य राद्धान्तस्य पुष्टौ \textcolor{red}{गति\-बुद्धि\-प्रत्यवसानार्थ\-शब्द\-कर्माकर्मकाणामणि कर्ता स णौ} (पा॰सू॰~१.४.५२) इति सूत्रस्य प्रयोग\-दिग्दर्शनावसरे कारक\-प्रकरणे वैयाकरण\-सिद्धान्त\-कौमुद्यां श्रीभट्टोजि\-दीक्षितोऽपि लिखति यत्~–\end{sloppypar}
\centering\textcolor{red}{शत्रूनगमयत्स्वर्गं वेदार्थं स्वानवेदयत्। \nopagebreak\\
आशयच्चामृतं देवान्वेदमध्यापयद्विधिम्॥ \nopagebreak\\
आसयत्सलिले पृथ्वीं यः स मे श्रीहरिर्गतिः।}\nopagebreak\\
\raggedleft{–~वै॰सि॰कौ॰~५४०}\\
\begin{sloppypar}\hyphenrules{nohyphenation}\justifying\noindent\hspace{10mm} इति। एवं च पुराणेष्वपि समाख्यायत इयमाख्यायिका। कल्प\-क्षये सलिल\-सम्प्लव\-परिप्लाविते सागरीभूते निखिलेऽवनि\-तले चराचरे च निमग्ने लग्ने च निद्रायां भगवति विष्णौ शेष\-शायिनि दैत्य एको ब्रह्म\-मुख\-च्युतं वेदराशिमहार्षीत्। ततो निरीक्ष्य वेदापहरणं मत्स्य\-रूपेण समवतीर्य हत्वा दिति\-सुतं पृथ्वीं शृङ्ग\-बद्ध\-नौकामारोप्य समुपदिश्य ब्रह्मविद्यां वैवस्वतमनवे प्रत्यर्पयामास भूयो वेदं विधात्रे भगवान् वासुदेवः। तथा च भागवते~–\end{sloppypar}
\centering\textcolor{red}{प्रलयपयसि धातुः सुप्तशक्तेर्मुखेभ्यः श्रुतिगणमपनीतं प्रत्युपादत्त हत्वा।\nopagebreak\\
दितिजमकथयद्यो ब्रह्म सत्यव्रतानां तमहमखिलहेतुं जिह्ममीनं नतोऽस्मि॥}\nopagebreak\\
\raggedleft{–~भा॰पु॰~८.२४.६१}\\
\begin{sloppypar}\hyphenrules{nohyphenation}\justifying\noindent\hspace{10mm} अनुमानरीत्याऽपि वैदिक्यपौरुषेयताऽवगन्तुं शक्यते। यथा वेदोऽपौरुषेयोऽविच्छिन्न\-गुरुपरम्परा\-पाठक्रमत्वात्। यन्नैवं तन्नैवमेष व्यतिरेकि\-व्याप्ति\-मूलकानुमान\-प्रकारः। अस्य माहात्म्यं सकल\-शाब्दिक\-शिरोमणि\-र्दार्शनिक\-कुञ्जर\-हरिः श्रीभर्तृहरिः पाणिनीय\-व्याकरण\-दार्शनिक\-संस्करण\-भूते स्वकीय\-वाक्यपदीय\-नामके ग्रन्थे ब्रह्मकाण्डे शब्दब्रह्म\-प्राप्ति\-मुख्य\-माध्यमतया श्लोक\-षट्केन मुक्तकण्ठं सञ्जगौ। यथा~–\end{sloppypar}
\centering\textcolor{red}{प्राप्त्युपायोऽनुकारश्च तस्य वेदो महर्षिभिः। \nopagebreak\\
एकोऽप्यनेकवर्त्मेव समाम्नातः पृथक्पृथक्॥\\
भेदानां बहुमार्गत्वं कर्मण्येकत्र चाङ्गता।\nopagebreak\\
शब्दानां यतशक्तित्वं तस्य शाखासु दृश्यते॥\\
स्मृतयो बहुरूपाश्च दृष्टादृष्टप्रयोजनाः।\nopagebreak\\
तमेवाश्रित्य लिङ्गेभ्यो वेदविद्भिः प्रकाशिताः॥\\
तस्यार्थवादरूपाणि निश्रिताः स्वविकल्पजाः।\nopagebreak\\
एकत्विनां द्वैतिनां च प्रवादा बहुधा मताः॥\\
सत्या विशुद्धिस्तत्रोक्ता विद्यैवैकपदागमा।\nopagebreak\\
युक्ता प्रणवरूपेण सर्ववादाविरोधिना॥\\
विधातुस्तस्य लोकानामङ्गोपाङ्गनिबन्धनाः।\nopagebreak\\
विद्याभेदाः प्रतायन्ते ज्ञानसंस्कारहेतवः॥}\nopagebreak\\
\raggedleft{–~वा॰प॰~१.५–१०}\\
\begin{sloppypar}\hyphenrules{nohyphenation}\justifying\noindent\hspace{10mm} एवं निखिल\-ज्ञान\-प्रादुर्भूति\-स्थानमखिल\-दर्शन\-निधानं स्वर्गापवर्ग\-सोपानं चतुर्वर्ग\-साधनं भागवत\-परम\-धनं वर्णाश्रम\-धर्म\-सदनं सकल\-कलि\-कलुष\-कदनं वदनं कमल\-वदनस्य वेदमिमं चतुर्धा व्याचक्रिरे वेद\-व्यास\-वर्या ऋग्यजुः\-सामाथर्व\-भेदेन। यथा भागवते~–\end{sloppypar}
\centering\textcolor{red}{ऋग्यजुःसामाथर्वाख्या वेदाश्चत्वार उद्धृताः।\nopagebreak\\
इतिहासपुराणं च पञ्चमो वेद उच्यते॥\\
तत्रर्ग्वेदधरः पैलः सामगो जैमिनिः कविः।\nopagebreak\\
वैशम्पायन एवैको निष्णातो यजुषामभूत्॥\\
अथर्वाङ्गिरसामासीत्सुमन्तुर्दारुणो मुनिः।\nopagebreak\\
इतिहासपुराणानां पिता मे रोमहर्षणः॥\\
त एत ऋषयो वेदं स्वं स्वं व्यस्यन्ननेकधा। \nopagebreak\\
शिष्यैः प्रशिष्यैस्तच्छिष्यैर्वेदास्ते शाखिनोऽभवन्॥\\
तथैव वेदा दुर्मेधैर्धार्यन्ते पुरुषैर्यथा।\nopagebreak\\
एवं चकार भगवान् व्यासः कृपणवत्सलः॥}\nopagebreak\\
\raggedleft{–~भा॰पु॰~१.४.२०–२४}\\
\begin{sloppypar}\hyphenrules{nohyphenation}\justifying\noindent\hspace{10mm} तत्रर्ग्वेद ऋचां सङ्ग्रह एवं च यजुर्वेदे याज्ञिक\-प्रयोगाणां समुपग्रह एवं सामवेदे ललित\-स्तोत्र\-द्वारा भगवद्यशो\-गान\-परिग्रहस्तथाऽथर्व\-वेदे सामरिक\-शास्त्र\-सम्प्रतिग्रहः। यद्वा मुख्या वेद\-त्रयी। अत एव शिवमहिम्नःस्तोत्रादौ त्रयीत्वेन चर्चा। तद्यथा~–\end{sloppypar}
\centering\textcolor{red}{त्रयी साङ्ख्यं योगः पशुपतिमतं वैष्णवमिति\nopagebreak\\
प्रभिन्ने प्रस्थाने परमिदमदः पथ्यमिति च।\nopagebreak\\
रुचीनां वैचित्र्यादृजुकुटिलनानापथजुषां\nopagebreak\\
नृणामेको गम्यस्त्वमसि पयसामर्णव इव॥}\nopagebreak\\
\raggedleft{–~शि॰म॰स्तो॰~७}\\
\begin{sloppypar}\hyphenrules{nohyphenation}\justifying\noindent\hspace{10mm} एवं च महर्षि\-वाल्मीकि\-प्रणीत आदि\-काव्ये श्रीमद्वाल्मीकीय\-रामायणे किष्किन्धा\-काण्डस्य तृतीय\-सर्गे भगवद्दर्शनेन शङ्कमानेन सुग्रीवेण प्रेषितं परिचय\-मनीषया पृष्टवन्तं धीमन्तं हनुमन्तं प्रशंसता लक्ष्मणं प्रति वदमानेन\footnote{\textcolor{red}{वदमानेन} इत्यत्र \textcolor{red}{भासनोप\-सम्भाषा\-ज्ञान\-यत्न\-विमत्युपमन्त्रणेषु वदः} (पा॰सू॰~१.३.४७) इत्यनेन भासने (भासमानेन वदता) यत्ने (उत्साहं प्रकटयता वदता) उपमन्त्रणे (लक्ष्मणं प्रति रहसि वदता) वाऽऽत्मनेपदम्।} विगत\-मानेन भगवता राघवेन्द्रेण श्रीरामचन्द्रेण त्रयाणामेव वेदानां 
चर्चा समचर्चि।\footnote{\textcolor{red}{चर्चा समचर्चि} इति कर्मणि प्रयोगः। \textcolor{red}{षिद्भिदादिभ्योऽङ्} (पा॰सू॰~३.३.१०४) इत्यनेन \textcolor{red}{चिन्ति\-पूजि\-कथि\-कुम्बि\-चर्चश्च} (पा॰सू॰~३.३.१०५) इत्यनेन वा निष्पन्नस्य \textcolor{red}{चर्चा} शब्दस्य कर्मत्वं \textcolor{red}{यतोऽनुजीविना पराधिकारचर्चा सर्वथा न कर्तव्या} (हि॰~२.३१) \textcolor{red}{मम नियोगस्य चर्चा त्वया न कर्तव्या} (हि॰~२.३२) \textcolor{red}{स्वनियोगचर्चा क्रियताम्} (हि॰~२.३५) इतिवत्।} यथा~–\end{sloppypar}
\centering\textcolor{red}{नानृग्वेदविनीतस्य नायजुर्वेदधारिणः।\nopagebreak\\
नासामवेदविदुषः शक्यमेवं प्रभाषितुम्॥}\nopagebreak\\
\raggedleft{–~वा॰रा॰~४.३.२८}\\
\begin{sloppypar}\hyphenrules{nohyphenation}\justifying\noindent\hspace{10mm} मार्कण्डेय\-पुराणान्तर्भूतायाः स्तोत्र\-रत्न\-दुर्गा\-सप्तशत्याश्चतुर्थाध्याये शक्रोऽपि त्रयीमयीमेव भगवतीं स्तौति यथा~–\end{sloppypar}
\centering\textcolor{red}{शब्दात्मिका सुविमलर्ग्यजुषां निधानमुद्गीथरम्यपदपाठवतां च साम्नाम्। \nopagebreak\\
देवी त्रयी भगवती भवभावनाय वार्ता च सर्वजगतां परमार्तिहन्त्री॥}\nopagebreak\\
\raggedleft{–~दु॰स॰श॰~४.१०}\\
\begin{sloppypar}\hyphenrules{nohyphenation}\justifying\noindent\hspace{10mm} श्रीमद्भागवतेऽपि श्रीकृष्ण\-बाल\-लीला\-प्रसङ्गे मृद्भक्षण\-लीलायां भगवन्मुखे निखिल\-ब्रह्माण्डमवलोक्य स्तब्धा यशोदा पुनस्तत्कृपया प्राप्त\-भगवदैश्वर्य\-विस्मृतिर्वात्सल्य\-धिया श्रीकृष्ण\-मुख\-कमलं लालयन्ती साश्चर्यं शुकाचार्येणाऽक्षिप्ता। अत्रापि च त्रयीशब्देनैव चर्चां चकार भगवान् बादरायणिः। यथा~–\end{sloppypar}
\centering\textcolor{red}{त्रय्या चोपनिषद्भिश्च साङ्ख्ययोगैश्च सात्वतैः। \nopagebreak\\
उपगीयमानमाहात्म्यं हरिं साऽमन्यतात्मजम्॥}\nopagebreak\\
\raggedleft{–~भा॰पु॰~१०.८.४५}\\
\begin{sloppypar}\hyphenrules{nohyphenation}\justifying\noindent\hspace{10mm} एवमेव श्रीमद्भगवद्गीतायामपि श्रीकृष्ण\-चन्द्रस्त्रयीमिमां सानुरागं समगायद्यथा~–\end{sloppypar}
\centering\textcolor{red}{त्रैविद्या मां सोमपाः पूतपापा यज्ञैरिष्ट्वा स्वर्गतिं प्रार्थयन्ते। \nopagebreak\\
ते पुण्यमासाद्य सुरेन्द्रलोकमश्नन्ति दिव्यान्दिवि देवभोगान्॥}\nopagebreak\\
\raggedleft{–~भ॰गी॰~९.२०}\\
\begin{sloppypar}\hyphenrules{nohyphenation}\justifying\noindent\hspace{10mm} गद्य\-काव्य\-निष्णातो महाकविर्बाणोऽपि कादम्बर्या मङ्गलाचरणे प्रथम\-श्लोके भगवन्तं त्रयीमयमेव तुष्टाव। यथा~–\end{sloppypar}
\centering\textcolor{red}{रजोजुषे जन्मनि सत्त्ववृत्तये स्थितौ प्रजानां प्रलये तमःस्पृशे।\nopagebreak\\
अजाय सर्गस्थितिनाशहेतवे त्रयीमयाय त्रिगुणात्मने नमः॥}\nopagebreak\\
\raggedleft{–~का॰~१.१}\\
\begin{sloppypar}\hyphenrules{nohyphenation}\justifying\noindent\hspace{10mm} उपबृंहणेनैतेन वेदानां त्रैविध्यं सुस्पष्टम्। तथा च ज्ञान\-सिद्धान्त\-वर्णन\-प्रधान ऋग्वेदो यज्ञनिर्देशकतया कर्म\-काण्ड\-प्राधान्य\-प्रतिपादन\-परो यजुर्वेदः स्तवन\-गान\-परायणतयोपासना\-सिद्धान्त\-सङ्कीर्तन\-तत्परः सामवेदः। अथर्व\-वेदस्त्रिष्वत्रान्तर्भवति यद्यप्यस्मिन् युद्ध\-विद्यायाः प्राधान्येन वर्णनमवलोक्यते। यतो ह्यथर्वशब्दः शब्दद्वयस्य संहितरूपः। \textcolor{red}{अर्व}\-शब्दो घोटकपरः। \textcolor{red}{अर्वणस्त्रसावनञः} (पा॰सू॰~६.४.१२७) इति सूत्रमपि प्रमाणम्। \textcolor{red}{अथ}\-शब्दोऽधिकार\-वाची यथा पातञ्जल\-महाभाष्ये \textcolor{red}{अथेत्ययं शब्दोऽधिकारार्थः प्रयुज्यते। शब्दानुशासनं नाम शास्त्रमधिकृतं वेदितव्यम्} (भा॰प॰) इति। इत्थमधिकार\-वाचकाथ\-शब्देन सहार्व\-शब्दस्य सन्धिः। तत्राकृति\-गणत्वादथर्व\-शब्दः शकन्ध्वादिगण आकृत्या गण्यते मनीषादिवत्। एवं \textcolor{red}{शकन्ध्वादिषु पर\-रूपं वाच्यम्} (वा॰~६.१.९१) इति वार्त्तिक\-सह\-कारेण \textcolor{red}{अचोऽन्त्यादि टि} (पा॰सू॰~१.१.६४) इति टिसञ्ज्ञया\footnote{\textcolor{red}{‘शकन्ध्वादिषु पररूपं वाच्यम्।’ तच्च टेः} (वै॰सि॰कौ॰~७९)। शकन्ध्वादित्वात् \textcolor{red}{अथर्व}\-शब्द\-सिद्धौ टिसञ्ज्ञाया उपयोगिता नास्ति परन्तु \textcolor{red}{मनीषा पतञ्जलि} इत्यादिषु टिसञ्ज्ञया विना शब्दसिद्धिर्न।} पर\-रूपम्। इत्थमथर्वशब्दस्य सिद्धिः। अन्यथाऽथ\-शब्द\-घटकाकारेण सहार्व\-शब्द\-घटकाकारस्य सन्धौ दीर्घः स्यात्। अतोऽधिकृतोऽर्ववेद इति सरलार्थः। सङ्क्षेपत ऋग्वेदे मन्त्र\-देवता\-ध्यानं यजुर्वेदे यज्ञोपयोगि\-देवतानामाह्वानं सामवेदे भगवद्विभूति\-भूत\-देवानां भगवतश्च यशो\-गानमथर्व\-वेदे रण\-प्रयाणं वर्णितं पर्यवसीयते। इमे वेदा एव निखिल\-ज्ञान\-विज्ञान\-कला\-कौशल\-सकल\-सृष्टि\-समुद्भवस्थानानि। भगवानपि सर्व\-तन्त्र\-स्वतन्त्रः सर्वद्रष्टा सर्वान्तर्यामी सत्यकामः सत्यसङ्कल्पः सत्यसन्धः सर्वाधिष्ठान\-स्वरूपः कर्तुमकर्तुमन्यथा\-कर्तुं समर्थोऽपि कदाऽपि स्वेच्छया जगदिदन्तनं सिसृक्षति। सोऽपि वेद\-प्रतिपादन\-प्रकारमेवानुकरोति। किं बहुना परमात्मनः प्रामाण्यमपि वेदमेवाधिशेते। अत एव वेदमान्यता\-बहिर्भूतमीश्वरोक्तं वयमास्तिका नाद्रियामहे यथा बुद्धोक्तम्। ईश्वरोऽपि कदाचिद्धर्मसमुद्दिधीर्षया भक्त\-हृदय\-विजिहीर्षयाऽसुर\-दर्पाप\-जिहीर्षया चिकीर्षया च सुर\-कार्यस्य समवतीर्णोऽवनितले विडम्बयन्मनुज\-लोकं मानवोचित\-च्छल\-प्रपञ्चादिकमङ्गीकरोति। अस्यां युक्तौ जालन्धर\-वध\-प्रसङ्गः प्रमाणम्। किन्तु वेदः कदाचिदपि भ्रम\-प्रमाद\-विप्रलिप्सा\-करणापाटवादि\-पुंदोषान्दूरतो न परामृशति। अत एव वेदानां स्वतः प्रामाण्यमिति मीमांसकसिद्धान्तः सन्नपि सकलपण्डितसम्मतः। वेदाः स्वतः प्रमाणम्। ततः पूर्वं कस्यापि ग्रन्थस्यानुपलब्धत्वाद्वेदस्य स्वतः प्रामाण्यमपर\-शास्त्राणां च परतः प्रामाण्यम्। इमे वेदाः परम्परयेतर\-देवताः प्रतिपादयन्तोऽपि परमात्मानमेव प्रतिपादयन्ति यथा काचित्कुलाङ्गना परिवार\-जनाञ्छ्वश्रु\-प्रभृतीञ्छुश्रूष\-माणाऽपि साक्षात्पतिमेव परिचरति तथैव वेदापरनामधेयाः श्रुतयोऽपि परिवार\-जनानिवेतर\-देवान् गायन्त्योऽपि साक्षान्महा\-तात्पर्यतया स्वकीयं पतिं परमेश्वरमेव प्रतिपद्यन्ते। इयं भगवती श्रुतिरेव समेषां माता परमात्मा च परा\-पराणां पिता। यथा च पुत्रस्य कृते पितुः सत्तायां माता प्रमाणं तथैव परमेश्वर\-सत्तायामस्मत्कृते श्रुतिः प्रमाणम्। अतोऽभियुक्ता आमनन्ति \textcolor{red}{सर्वे वेदा यत्पदमामनन्ति} (क॰उ॰~१.२.१५)। भगवती श्रुतिरपि श्रावयति \textcolor{red}{अ॒मृत॑स्य पु॒त्राः} (शु॰य॰वा॰मा॰~११.५, श्वे॰उ॰~२.५) तमेव \textcolor{red}{विजिज्ञासस्व} (तै॰उ॰~३.१.१) तमेव \textcolor{red}{ब्राह्मणाः व्रतेन तपसाऽनाशकेन च विविदिषन्ति} (बृ॰उ॰~४.४.२२) \textcolor{red}{सत्यं ज्ञानमनन्तं ब्रह्म} (तै॰उ॰~२.१.१) \textcolor{red}{इ॒दं विष्णु॒र्वि च॑क्रमे} (ऋ॰वे॰सं॰~१.२२.१७, शु॰य॰वा॰मा॰~५.१५) \textcolor{red}{खं ब्रह्म} (शु॰य॰वा॰मा॰~४०.१७, बृ॰उ॰~५.१.१) इत्यादि\-मन्त्रैः। गीताऽपि परमात्मन एव वेद\-वेद्यत्वं तस्यैव च वेद\-वेत्तृत्वं गायति। यथा~–\end{sloppypar}
\centering\textcolor{red}{सर्वस्य चाहं हृदि सन्निविष्टो मत्तः स्मृतिर्ज्ञानमपोहनं च।\nopagebreak\\
वेदैश्च सर्वैरहमेव वेद्यो वेदान्तकृद्वेदविदेव चाहम्॥}\nopagebreak\\
\raggedleft{–~भ॰गी॰~१५.१५}\\
\begin{sloppypar}\hyphenrules{nohyphenation}\justifying\noindent\hspace{10mm} \textcolor{red}{श्रुति}\-शब्दः श्रवणार्थक\-\textcolor{red}{श्रु}\-धातोर्निष्पद्यते (\textcolor{red}{श्रु श्रवणे} धा॰पा॰~९४२)। एवं च \textcolor{red}{श्रूयत आनुपूर्व्याऽनादि\-कालाद्गुरु\-परम्परया या सा श्रुतिः} इति व्युत्पत्तौ कर्मणि क्तिन्प्रत्ययेऽनुबन्ध\-लोपे कृत्प्रत्ययान्तत्वाद्विभक्ति\-कार्ये प्रथमैकवचने श्रुतिः। इत्थं हि \textcolor{red}{अस्मदभिन्न\-कर्तृक\-वर्तमान\-कालिक\-श्रवणानुकूल\-व्यापार\-विशिष्ट\-निःसीम\-शब्द\-कर्मक\-श्रवण\-कर्म\-रूपं फलम्} इति फलमुख्य\-विशेष्यकः शाब्द\-बोधः। यद्वा \textcolor{red}{श्रूयते परमात्मा यया सा श्रुतिः} इति व्युत्पत्तौ \textcolor{red}{स्त्रियां क्तिन्} (पा॰सू॰~३.३.९४) इति सूत्रेण करणे क्तिन्। एवं \textcolor{red}{वर्तमान\-कालिक\-श्रवणानुकूल\-व्यापारावच्छिन्न\-श्रवण\-कर्मत्वाभिन्न\-फलत्व\-निष्ठ\-प्रकारता\-निरूपित\-फल\-निष्ठ\-विशेष्यता\-
शालि\-फल\-करणम्} एव श्रुतिः। अथवा \textcolor{red}{श्रावयति परमात्मानं पितरं पुत्रैः परिचाययति या सा श्रुतिः} इति व्युत्पत्तावन्तर्भावित\-ण्यर्थता\-वच्छेदक\-धातुत्वावच्छिन्न\-\textcolor{red}{श्रु}\-धातोर्बाहुलक\-\textcolor{red}{क्तिन्‌}प्रत्ययः। एवं च \textcolor{red}{वर्तमान\-कालावच्छिन्न\-ब्रह्म\-विषयक\-श्रवणानुकूल\-व्या\-पारानुकूल\-व्यापाराश्रयः} श्रुतिरिति दिक्। एवमेव \textcolor{red}{वेद}\-शब्दोऽपि चतुर्भिर्धातुभिर्निष्पादयितुं
शक्यते। तत्र \textcolor{red}{विदँ सत्तायाम्} (धा॰पा॰~११७१) इत्यस्मात्सत्तार्थक\-धातोः \textcolor{red}{विद्यते निरन्तरं वर्तते यद्ब्रह्म\-विषयक\-ज्ञानं तदेव वेद} इति कर्तरि घञ्‌प्रत्ययेऽचि वा गुणे कृते सति वेद\-शब्दः सिध्यति। एवं \textcolor{red}{वर्तमान\-कालावच्छिन्न\-भवन\-फलानुकूल\-व्यापाराश्रयः} इति शाब्द\-बोधः। द्वितीयस्मिन् कल्पे \textcolor{red}{विदँ विचारणे} (धा॰पा॰~१४५०) इत्यस्माद्धातोः \textcolor{red}{विन्ते ब्रह्मतत्त्वं विचारयति जनो यस्मिन् स वेदः} इति विग्रहे \textcolor{red}{हलश्च} (पा॰सू॰~३.३.१२१) इत्यनेनाधिकरणे घञ्‌प्रत्ययः। तथा चानुबन्ध\-कार्ये \textcolor{red}{पुगन्तलघूपधस्य च} (पा॰सू॰~७.३.८६) इत्यनेन गुणे निष्पन्नो वेद\-शब्दः। तथा च \textcolor{red}{विचारकाभिन्न\-कर्तृक\-वर्तमान\-कालावच्छिन्न\-मनो\-बुद्धि\-संयोगानुकूल\-व्यापाराधिकरणं वेदः} इति निर्गलितोऽर्थः। तृतीयस्मिन् कल्पे \textcolor{red}{विदँ ज्ञाने} (धा॰पा॰~१०६४) इत्यस्माद्धातोः \textcolor{red}{वेत्ति जानाति सनातनं परमं ब्रह्म निरञ्जनं निर्गुणं निराकारं निष्प्रकारं निर्लेपं निर्मानं निर्मोहं सच्चिदानन्द\-सन्दोहं निर्विकल्पं तथा चानन्द\-कन्द\-मुकुन्द\-सतत\-मुनिजन\-परिपीत\-चरणारविन्दामन्द\-मकरन्द\-सच्चिदानन्द\-चिन्मय\-समलङ्कृत\-प्रणयि\-हृदय\-कोसलेन्द्र\-सरयू\-तरल\-तरङ्ग\-भङ्गि\-विक्षालित\-चरण\-सरोज\-मनोज\-वैरि\-वन्दित\-पाद\-पयोज\-कोशल\-वीचि\-विघूर्ण\-धूलि\-धूसर\-सरस\-शिरोज\-कपोलावलम्बि\-रोलम्ब\-निन्दक\-कुटिल\-कुन्तल\-संस्मारित\-पाटल\-कोश\-विहारि\-भृङ्ग\-सौभग\-कौसल्या\-क्रोड\-वर्त्ति\-दशरथ\-नयन\-वर्ति\-वर्तुल\-मुख\-मृगाङ्क\-संव्रीडित\-शारद\-शशाङ्क\-परम\-करुण\-नील\-तरुण\-तामरस\-शरीर\-परम\-रणधीर\-रघुवीर\-पूर्णचन्द्र\-निभानन\-निहत\-दशानन\-श्याम\-सुन्दर\-पर\-ब्रह्म\-सगुण\-विग्रह\-साकार\-भगवन्तं श्रीरामचन्द्रं वा येन स वेदः}। अथ \textcolor{red}{साधकाभिन्न\-कर्तृक\-वर्तमान\-कालावच्छिन्न\-ज्ञानानुकूल\-व्यापार\-विशिष्ट\-निर्गुण\-सगुण\-ब्रह्म\-विषयक\-ज्ञान\-रूप\-फल\-मुख्य\-व्यापाराख्यं करणं वेदः} इति फलितार्थः। चतुर्थकल्पे \textcolor{red}{विदॢँ लाभे} (धा॰पा॰~१४३२) इति लाभार्थक\-धातोः \textcolor{red}{विन्दति लभते सगुण\-ब्रह्म श्रीरामचन्द्रं येन स वेदः} इति व्युत्पत्तौ करणे घञ्यनुबन्ध\-कार्ये गुणे च सति वेदशब्दस्य सिद्धिः। एवं \textcolor{red}{वर्तमान\-कालावच्छिन्न\-लाभानुकूल\-व्यापारविशिष्ट\-फलकरणं वेदः}। निर्गुणं सगुणं वा ब्रह्म वेद\-प्रतिपाद्यैरुपायैरेवानु\-भवितुं द्रष्टुं वा शक्यते। तत्र वेदे त्रय उपायास्त्रिभिः काण्डैः प्रतिपादिताः। कर्म उपासना ज्ञानं चेति। कर्मकाण्डं कर्म\-प्रति\-पादकमशीति\-सहस्रमन्त्रात्मकम्। उपासना\-काण्डमुपासनां बोधयति। तच्च षोडश\-सहस्र\-मन्त्रात्मकम्। ज्ञान\-काण्डं ज्ञान\-निरूपण\-परम्। तच्च चतुस्सहस्र\-मन्त्रात्मकम्। तत्र \textcolor{red}{वेद\-प्रणिहितो धर्मः} (भा॰पु॰~६.१.४०) इति वचनेन वेदप्रतिपाद्यमेव कर्म। तच्च द्विविधं नित्यं नैमित्तिकञ्च। \textcolor{red}{नित्यत्वं नाम कृते सत्यदृष्ट\-फलाजनकत्वेऽपि न कृते पापजनकत्वम्}।
\textcolor{red}{नैमित्तिकत्वं नाम निमित्ते विधीयमानत्वे सति पुण्य\-जनकतावच्छेदकत्वम्}। नित्यं कर्म सन्ध्यादिकं श्रौतं स्मार्तं च। तस्यावश्यं करणीयता यथा गीतायामशेष\-विशेषातीतोऽतसी\-कुसुमोपमेय\-कान्तिः पार्थ\-सारथिर्निखिल\-गुण\-गण\-निलय उत्तम\-श्लोको भक्तानुग्रह\-कातरः श्रीकृष्ण\-चन्द्रः साटोपं पार्थं प्रति प्रत्य\-पादयद्यत्~–\end{sloppypar}
\centering\textcolor{red}{नियतं कुरु कर्म त्वं कर्म ज्यायो ह्यकर्मणः। \nopagebreak\\
शरीरयात्राऽपि च ते न प्रसिद्ध्येदकर्मणः॥}\nopagebreak\\
\raggedleft{–~भ॰गी॰~३.८}\\
\centering\textcolor{red}{एवं ज्ञात्वा कृतं कर्म पूर्वैरपि मुमुक्षुभिः।\nopagebreak\\
कुरु कर्मैव तस्मात्त्वं पूर्वैः पूर्वतरं कृतम्॥}\nopagebreak\\
\raggedleft{–~भ॰गी॰~४.१५}\\
\centering\textcolor{red}{कर्मण्येवाधिकारस्ते मा फलेषु कदाचन।\nopagebreak\\
मा कर्मफलहेतुर्भूर्मा ते सङ्गोऽस्त्वकर्मणि॥}\nopagebreak\\
\raggedleft{–~भ॰गी॰~२.४७}\\
\begin{sloppypar}\hyphenrules{nohyphenation}\justifying\noindent\hspace{10mm} किं बहुना कार्याकार्य\-व्यवस्थायां शास्त्रमेव प्रमाणं गीता\-कारः प्रमिमीते यथा~–\end{sloppypar}
\centering\textcolor{red}{यः शास्त्रविधिमुत्सृज्य वर्तते कामकारतः। \nopagebreak\\
न स सिद्धिमवाप्नोति न सुखं न परां गतिम्॥\nopagebreak\\
तस्माच्छास्त्रं प्रमाणं ते कार्याकार्यव्यवस्थितौ।\\
ज्ञात्वा शास्त्रविधानोक्तं कर्म कर्तुमिहार्हसि॥}\nopagebreak\\
\raggedleft{–~भ॰गी॰~१६.२३–२४}\\
\begin{sloppypar}\hyphenrules{nohyphenation}\justifying\noindent\hspace{10mm} इत्थं कर्मणोऽस्य द्वौ भागौ विधिर्निषेधश्चेति। तत्रान्यकर्मणोऽत्यन्ताप्राप्तौ विधीयमान\-कर्मतावच्छेदकत्वं विधित्वम्। तद्यथा \textcolor{red}{अहरहः सन्ध्यामुपासीत}\footnote{मूलं मृग्यम्।} इति विधिः। अत्रैव निषेधस्यैव क्रोडे द्वावपरावप्यंशौ नियमः परिसङ्ख्या च। \textcolor{red}{नियमः पाक्षिके सति} (त॰वा॰~१.२.४२)। यथा \textcolor{red}{व्रीहीनवहन्ति}। अत्रावहननमावश्यकं नखैर्वाऽन्यैः साधनैर्वा। परिसङ्ख्या। तत्र विधेयांशेऽन्यत्राविधेयांशे चाप्राप्तौ परिसङ्ख्या। यथा विद्वांसो निगदन्ति~–\end{sloppypar}
\centering\textcolor{red}{विधिरत्यन्तमप्राप्तौ नियमः पाक्षिके सति। \nopagebreak\\
तत्र चान्यत्र चाप्राप्तौ परिसङ्ख्येति कीर्त्यते॥}\nopagebreak\\
\raggedleft{–~त॰वा॰~१.२.४२}\\
\begin{sloppypar}\hyphenrules{nohyphenation}\justifying\noindent\hspace{10mm} निषेधो नाम कृते सति पापजनकतावच्छेदकः शास्त्र\-समभिव्याहृत\-नञर्थावच्छेदक\-वदभित्याज्यः कर्म\-विशेषः। यथा \textcolor{red}{स्वाध्याय\-प्रवचनाभ्यां न प्रमदितव्यम्} (तै॰उ॰~१.११.१) \textcolor{red}{मा हिंस्याः सर्वभूतानि},\footnote{मूलं मृग्यम्।} \textcolor{red}{मा गृधः कस्य\-स्विद्धनम्} (ई॰उ॰~१) इत्यादि। एवमुपासना\-काण्ड\-प्रतिपादनं भगवन्तं वेदमाश्रित्यैवोपास्यमानः परमात्मा साधकानां कामदो भवति। शास्त्रमतिरिच्य किमपि 
कृतं नैव फलदम्। अतो गीतायाः पूर्व\-निर्दिष्टे \textcolor{red}{यः शास्त्रविधिम्} (भ॰गी॰~१६.२३) इत्यादि\-श्लोके शास्त्र\-विधेरेव सफलत्वं सकलं प्रत्यपादि भगवता श्रीकृष्णेन। उपासनायां श्रुति\-सम्मतत्वं नितरामपेक्षितम्। तत्र वेदार्थ\-प्रतिपादक\-स्मृत्यनुसारं पञ्च सम्प्रदायाः। स्मृतेर्हि वेद\-मूलकत्वेन प्रामाण्यम्। वेदस्य निगूढार्थान् स्मृतिः स्मरति। स्मृति\-कर्तार ऋषयो मन्वादयः। स्मृतेः प्रामाण्यं पूर्वमेव वेदः प्रशंसति यथा \textcolor{red}{यद्वै किं च॒ मनु॒रव॑द॒त्तद्भे॑ष॒जम्} (कृ॰य॰ तै॰सं॰~२.२.१०.२)। महाकवि\-कालिदासोऽपि स्मृतेर्वेदानु\-गन्तृत्वमुपमया स्तौति। यथा रघुवंश\-महाकाव्ये~–\end{sloppypar}
\centering\textcolor{red}{तस्याः खुरन्यासपवित्रपांसुमपांसुलानां धुरि कीर्तनीया। \nopagebreak\\
मार्गं मनुष्येश्वरधर्मपत्नी श्रुतेरिवार्थं स्मृतिरन्वगच्छत्॥}\nopagebreak\\
\raggedleft{–~र॰वं॰~२.२}\\
\begin{sloppypar}\hyphenrules{nohyphenation}\justifying\noindent\hspace{10mm} तस्माच्छ्रुत्यनुमोदितानेवार्थान् स्मृतिः स्मरति। तस्यां गाणपत्य\-सौर\-शाक्त\-शैव\-वैष्णव\-नामधेयाः पञ्च सम्प्रदायाः। गाण\-पत्य\-सम्प्रदायो गणेशमेव मुख्यमाद्रियतेऽन्यान् गौणत्वेनैवोपास्ते। तस्य बाहुल्यमद्यापि महाराष्ट्रे विलोक्यते। सौर\-सम्प्रदायः सूर्यमेव प्रधानत्वेन समर्थयतीतरान् तदङ्गत्वेनैवाञ्चति। अस्य प्राधान्यं गुर्जर\-प्रान्ते। शाक्त\-सम्प्रदाये शक्तेः प्राधान्यम्। अयं सम्प्रदायश्च कश्मीर\-बङ्गयोः प्रचुरः प्रसरति। शैव\-सम्प्रदाये प्रामुख्येन शशाङ्क\-शेखरो भूत\-भावनोऽभयङ्करः प्रलयङ्करः 
काशीश्वरः प्रचण्ड\-ताण्डवोपसंहृत\-भूत\-वरूथ\-भसित\-भासित\-भाल\-पट्टिकः 
कराल\-भाल\-क्रतु\-वेदिका\-जाज्वल्यमान\-समिन्धन\-घन\-धर्षित\-निखिल\-जन्तु\-जलारि\-धनञ्जय\-समुच्छलत्स्फुलिङ्ग\-समाहुतीकृत\-निखिल\-जन\-मनः\-क्षोभ\-कर\-मकर\-केतनः कैलास\-निकेतनो विरसित\-वैराग्य\-रस\-निमग्न\-जगती\-तल\-भीषण\-भोग\-लालसा\-मनो\-निकेतन\-भवानी\-समलङ्कृत\-वाम\-विग्रहः कृत\-जालन्धर\-गजासुर\-त्रिपुर\-प्रभृति\-प्रचण्ड\-दानव\-निग्रहः शान्त\-वासनाग्रहो भगवाञ्छिवः प्राधान्येन परिगण्यते। एतत्प्राचुर्यं केरल\-कर्णाटकादौ। वैष्णव\-सम्प्रदाये प्रमुखतया निखिल\-कोटि\-ब्रह्माण्डाधीशो जगदीशः पुरन्दर\-त्रिपुर\-हर\-वसु\-विरिञ्चि\-पावक\-पवमान\-हिमभानु\-चित्रभानु\-कृशानु\-निखिल\-नक्षत्र\-गण\-सकल\-सुरासुर\-यक्ष\-गन्धर्व\-किन्नर\-चारण\-सिद्ध\-साध्य\-नर\-नाग\-लोक\-पाल\-नाक\-पाल\-काल\-कराल\-व्याल\-माल\-मौलि\-सम्पूजित\-पादारविन्द\-लेखाधीश\-मुकुटमणि\-संलालित\-चरण\-सरोज\-रजो\-मकरन्द\-त्रैलोक्य\-लक्ष्मी\-नवनलिन\-ललित\-लोचन\-लसित\-कमन\-कटाक्ष\-संवीक्षित\-शुभेक्षित\-भव्य\-सौन्दर्य\-सार\-सर्वस्व\-सकल\-सुषमा\-सार\-भूत\-मनोरम\-सार्वभौम\-शर्वरी\-प्राण\-वल्लभ\-संव्रीडक\-मदन\-मान\-पीडक\-वदन\-सरसीरुह\-निसर्ग\-निहित\-सुधित\-सुधा\-माधुरीक\-श्रीवत्स\-लाञ्छन\-शरच्छशाङ्कानन\-मधुर\-मन्द\-स्मित\-तुहिनांशु\-दीधिति\-विहित\-प्रपन्न\-हृदय\-सरसी\-कैरव\-विकास\-ललित\-लीला\-विलास\-महा\-लक्ष्मी\-निवास\-क्षीर\-सागर\-सम्मन्थन\-सञ्जात\-कण\-संस्पर्श\-भग्न\-भक्त\-भूरि\-भव\-भय\-व्रण\-पाणि\-पल्लव\-जगती\-तल\-पाथोधि\-प्लव\-नाशित\-प्रणिपात\-क्लव\-विरचित\-गणिका\-गजाजामिल\-पङ्किल\-पङ्क\-कलङ्क\-भव\-विहित\-भव\-संस्तव\-करुणार्णव\-सकल\-शरण्य\-वरेण्य\-तरुणारुण\-सरसिज\-चरण\-तरुण\-तमाल\-नील\-सरसीरुह\-मरकत\-मणि\-कालिन्दी\-कीलाल\-विनिन्दक\-भुवनाभि\-राम\-श्याम\-शरीर\-दामिनी\-द्युति\-विनिन्दक\-संवीत\-पीत\-परिधान\-म्लान\-मदमत्त\-मनोभव\-धीर\-मधु\-कैटभ\-नरक\-मुर\-प्रचुर\-महासुर\-भूरि\-मद\-गर्वित\-गजेन्द्र\-गण्डस्थल\-विनिर्गत\-शोणित\-करकमल\-कलित\-कौमोदकी\-धर\-शङ्ख\-चक्र\-धर\-कमल\-कलित\-कमल\-कर\-केयूर\-कुण्डल\-कटक\-वनमाला\-नूपुरादि\-भूषण\-मण्डित\-सकल\-कला\-कलाप\-पण्डित\-कमलामल\-मुख\-चन्द्र\-चकोर\-वैकुण्ठ\-विहरण\-परायण\-नारायण\-प्रभविष्णु\-विष्णुर्हृदय\-मन्दिरे महीयते। अस्य प्रायशो विश्वस्मिन् विश्वे प्रचारः प्रसारश्च। अस्य 
प्रवर्तका रामानुज\-रामानन्द\-वल्लभ\-निम्बार्क\-मध्व\-सूर\-तुलसीदास\-प्रमुखाः। उपासना\-प्रतिपादक\-शास्त्रेषु वेद\-मूलकेषु नारद\-पाञ्चरात्र\-शाण्डिल्य\-सूत्र\-नारद\-भक्ति\-सूत्र\-भक्ति\-रसायन\-भक्ति\-रसामृत\-सिन्धु\-भक्तिसुधा\-प्रभृति\-नामानि। तत्रोपासनायां भावानुसारिणः पञ्च प्रकाराः। ते सन्ति शान्त\-वात्सल्य\-दास्य\-सख्य\-मधुराख्याः। शान्ते शान्त\-परम\-प्रकाश\-रूपं परम\-ज्योतिर्मयमखण्डमनन्तं भगवन्तं हृदय\-सहस्रदल\-कमले समुपासते योगिनः शिवादयः। वात्सल्ये पुत्रं शिष्यं वा मत्वा शिशु\-रूपं निर्गुणमपि सगुणं निरञ्जनमपि साञ्जनं व्यापकमपि व्याप्यं पितरमपि पुत्रं पुराणमपि नूतनं प्राचीनमप्यार्वाचीनं निराकारमपि साकारमकलमपि सकलं निरामयमपि सामयं दुर्लभमपि सुलभमचलमपि चलं निर्लेपमपि धूलिलेपं निरावरणमपि साभरणावरणमवर्णमप्यनुगत\-पितृवर्णमव्यक्तमपि व्यक्त\-कौतुकमविषादमपि स्तन्यपानार्थ\-विलसित\-विषादं निराधारमपि स्वीकृतकौसल्यकाधारं श्रीराममरूपमपि समाश्रित\-नीलनीरधरश्यामं भुवनाभिराम\-सकल\-विश्राम\footnote{\textcolor{red}{विश्रामो विश्रमश्चापि} इति द्विरूपकोशे श्रीहर्षः।}\-धाम\-निखिल\-लावण्य\-ललाम\-लोक\-लोचनाभिराम\-बालरूपं परमेश्वरं सम्भजन्ते दशरथ\-कौसल्या\-विश्वामित्र\-वसिष्ठादयः। दास्ये जगदात्मानं परमात्मानमैश्वर्य\-बुद्ध्या सकल\-निर्मातारं रचयितारं त्रिभुवनस्य निखिल\-प्रपञ्च\-भूतं त्रिलोकपितरं कर्तुमकुर्तुमन्यथाकर्तुं समर्थं सत्य\-सङ्कल्प\-सत्य\-काम\-सौन्दर्येश्वर\-माधुर्य\-सौशील्य\-सौजन्य\-सौलभ्य\-सौरभ्य\-सारल्य\-तारल्य\-कारुण्य\-तारुण्य\-वात्सल्यादि\-गुण\-गण\-रत्नाकरं करुणाकरं श्री\-राघवेन्द्रं स्वामिनं स्व\-सर्वस्वं मत्वा सर्व\-भावेन भावयन्ति हनुमत्प्रमुखाः सन्तः। सख्ये कूटस्थमपि भूयिष्ठं हृदयेशमपि विडम्बित\-नरेश\-वेषमदेहमपि लीलागृहीत\-देहं सर्व\-गेह\-मपि स्वीकृत\-पर्णकुटीर\-गेहं सच्चिदानन्द\-सन्दोहं कोशलेन्द्रं प्रणत\-सहायं सखायं मत्वा सञ्चक्षते सुग्रीवादयः। माधुर्येण निखिल\-रसामृत\-मूर्तिं सौन्दर्य\-क्षीरनिधि\-रोहिणीशं भुवन\-सुन्दरं सीता\-पाणिग्रहण\-लालसं धृत\-वर\-वेषं धनुर्धरमपि कामारि\-कार्मुक\-खण्डन\-परं परमात्मानं मधुरं तत्तत्सम्बन्ध\-समाश्रयं मत्वा भजन्ते भावुका मिथिला\-निवासिनः। एषु सर्वेषु भावेषु मौलिकतया दास्यमेव। भावनावैषम्यदृष्ट्या प्रकाराः प्रादर्शिषत। अतो विपश्चितो ब्रुवन्ति \textcolor{red}{दासभूतास्स्वतः सर्वे ह्यात्मानः परमात्मनः} (अहि॰सं॰~११) इति। उपासनाऽपि श्रुतिमूलिका। अर्वाचीन\-कल्पितायां भक्तौ नैव फलजनकतावच्छेदकता। अतो मान्यतैषा मान्यानां महात्मनाम्~–\end{sloppypar}
\centering\textcolor{red}{श्रुतिस्मृतिपुराणादिपञ्चरात्रविधिं विना। \nopagebreak\\
ऐकान्तिकी हरेर्भक्तिरुत्पातायैव कल्पते॥}\footnote{मूलं मृग्यम्।}\\
\begin{sloppypar}\hyphenrules{nohyphenation}\justifying\noindent\hspace{10mm} तुलसीदासोऽपि वेदानुमोदितामेव भक्तिं वरीयसीं वरयाम्बभूव। यथा~–\end{sloppypar}
\centering\textcolor{red}{श्रुति सम्मत हरि भक्ति पथ संजुत बिरति बिबेक।\nopagebreak\\
तेहिं न चलहिं नर मोहबश कल्पहिं पंथ अनेक॥}\footnote{एतद्रूपान्तरम्–\textcolor{red}{यो मार्ग आस्ते श्रुतिसम्मतः श्रीनाथस्य भक्तेः सविरक्तिबोधः। तस्मिन्नरा यान्ति न मोहनिघ्ना मार्गाननेकानपि कल्पयन्ति॥} (मा॰भा॰~७.१००ख)।}\nopagebreak\\
\raggedleft{–~रा॰च॰मा॰~७.१००ख}\\
\begin{sloppypar}\hyphenrules{nohyphenation}\justifying\noindent\hspace{10mm} ज्ञान\-काण्डं ज्ञान\-प्रतिपादकं चतुः\-सहस्त्र\-मन्त्रात्मकम्। इदमेव वेदान्त\-दर्शनमिति कथ्यते। तत्र \textcolor{red}{वेदस्यान्तः सिद्धान्तः परम\-तात्पर्यं वा यस्मिन् सः} इति विग्रहे \textcolor{red}{अतिँ बन्धने} (धा॰पा॰~६१) इत्यस्माद्धातोरन्तति बध्नाति व्यवस्थापयति वेदानां महा\-तात्पर्यं यः स वेदान्तः। \textcolor{red}{अन्तः} इत्यत्र पचादित्वात् \textcolor{red}{अच्‌}\-प्रत्ययः।\footnote{\textcolor{red}{नन्दि\-ग्रहि\-पचादिभ्यो ल्युणिन्यचः} (पा॰सू॰~३.१.१३४) इत्यनेन कर्तरि।} तस्य \textcolor{red}{वेद}\-शब्देन सह व्यधिकरण\-बहुव्रीहिः। यद्यपि \textcolor{red}{अनेकमन्यपदार्थे} (पा॰सू॰~२.२.२४) इति सूत्रं समानाधिकरण\-बहुव्रीहि\-परम्। तस्य ह्यर्थोऽन्य\-पदार्थे वर्तमानमनेक\-सुबन्तं समस्यते स च बहुव्रीहिः। \textcolor{red}{अन्य\-पदार्थे} इत्यत्र सप्तमी। सा चाधिकरण\-सञ्ज्ञा\-परा। तत्र च विषयता\-रूपाऽधिकरणता। सा च वर्तमानत्व\-रूपा। तस्मादन्य\-पदार्थ\-निष्ठ\-वर्तमानत्व\-विषयता\-परमनेक\-सुबन्तं तदेव परस्परं समस्यत इत्यर्थानुरोधेन प्रथमान्तानामेव समासः। तत्र समानाधिकरणत्वं नाम शब्द\-विशिष्टत्वम्। वैशिष्ट्यं च स्व\-समभि\-व्याहृतत्व\-स्व\-समान\-विभक्तिकत्वमित्येतदुभय\-सम्बन्धेन। तथाऽपि \textcolor{red}{सप्तमी\-विशेषणे बहुव्रीहौ} (पा॰सू॰~२.२.३५) इति सूत्रे सप्तमी\-शब्दस्य विशेषणात्पृथगुपादानेन व्यधिकरण\-बहुव्रीहेः कल्पना। अन्यथा विशेषण\-शब्देनैव समानाधिकरण\-बहुव्रीहेः सङ्ग्रहः किमेतेन सप्तमीपदविन्यासेन। अत्र सप्तमीशब्दस्य पृथक्श्रवणाद्विशेषणं विशेष्य\-समान\-विभक्तिकमेव तत्र प्रथमान्तमेव। तेन \textcolor{red}{पीतमम्बरं यस्य स पीताम्बरः} इत्यादि\-स्थलानि सुसङ्गतानि। पीत\-शब्दो ह्यम्बर\-शब्दस्य विशेषणत्वावच्छेदकतया पूर्वं प्रयुक्तः। यद्यपि \textcolor{red}{भूतले घटः} इत्यादौ सप्तम्यन्तस्यापि विशेषणता। अत एव गदाधर\-भट्टाचार्या व्युत्पत्ति\-वादे \textcolor{red}{वेदाः प्रमाणम्} इत्यस्य स्थलस्य\footnote{\textcolor{red}{स्थल}\-शब्दः प्रसङ्गे विषये भागे चेत्याप्टे\-कोशः।} व्याख्यावसरे विशेषण\-विशेष्य\-पदयोः समान\-विभक्तिकत्व\-नियमं सार्वकालिकं न स्वीचक्रुः। अतस्ते व्युत्पत्तिमिमां प्रणिन्युर्यत् \textcolor{red}{यत्र विशेष्य\-वाचक\-पदोत्तर\-विभक्ति\-तात्पर्य\-विषय\-सङ्ख्या\-विरुद्ध\-सङ्ख्याया अविवक्षितत्वं तत्र विशेष्य\-विशेषण\-वाचक\-पदयोः समान\-वचनकत्व\-नियमः} (व्यु॰वा॰ का॰प्र॰)। तद्यथा \textcolor{red}{सुन्दरो रामः} इत्यत्र विशेष्य\-वाचक\-पदं रामस्तदुत्तर\-विभक्तिः सु\-विभक्तिस्तत्तात्पर्य\-विषय\-सङ्ख्यैकत्व\-सङ्ख्या तद्विरुद्ध\-सङ्ख्यायार्द्वित्वादेः सुन्दर\-पदोत्तर\-सु\-विभक्त्या विवक्षा नास्ति। अतोऽत्र समान\-वचनकत्वम्। किन्तु \textcolor{red}{वेदाः प्रमाणम्} इत्यत्र विशेष्य\-वाचक\-पदं वेदास्तदुत्तर\-विभक्तिर्जस्विभक्तिस्तत्तात्पर्य\-विषय\-सङ्ख्या बहुत्व\-सङ्ख्या तद्विरुद्ध\-सङ्ख्याया एकत्व\-नाम\-सङ्ख्यायाः प्रमाण\-पदोत्तर\-सु\-विभक्त्या विवक्षितत्वमेव। अतोऽत्र विषम\-वचनता। विगत\-सप्तमी\-पदस्य पृथगुपादानादस्मिन्नियमेऽत्र सङ्कोचं चकार भगवान् पाणिनिः। अतः सिद्धान्ते व्यधिकरण\-बहुव्रीहौ \textcolor{red}{कण्ठेकालः} (कण्ठे कालो यस्य सः)\footnote{\textcolor{red}{कण्ठेकालः} इत्यत्र \textcolor{red}{अमूर्ध\-मस्तकात्स्वाङ्गादकामे} (पा॰सू॰~६.३.१२) इति सूत्रेण काशिका\-कौमुदी\-शब्दकल्पद्रुम\-वाचस्पत्यादिषु समासः प्रादर्शि। \textcolor{red}{कण्ठे कालोऽस्य कण्ठेकालः} (का॰वृ॰~६.३.१२) इति काशिका। \textcolor{red}{कण्ठेकालः} (वै॰सि॰कौ॰~९७०) इति कौमुदी। \textcolor{red}{कण्ठे कालः क्षीरोद\-मन्थनोद्भव\-विषपान\-निदर्शन\-रूप\-नील\-वर्णो यस्य। सप्तम्या अलुक्} इति शब्दकल्पद्रुमः। \textcolor{red}{कण्टे कालः कण्ठे कालोऽस्य वा सप्तम्या अलुक्} इति वाचस्पत्यम्। भाष्ये च \textcolor{red}{समानाधिकरण\-समासाद्बहुव्रीहिः} (वा॰~२.१.६९) इति वार्त्तिके \textcolor{red}{बहुव्रीहेरवकाशः~–कण्ठेकालः} इति बहुव्रीहि\-समासमुदाहृत्य \textcolor{red}{सप्तम्युपमान\-पूर्वपदस्योत्तर\-पदलोपश्च} (वा॰~२.२.२४) इति वार्त्तिके \textcolor{red}{कण्ठेस्थः कालोऽस्य कण्ठेकालः} इत्युक्तम्। इत्युभयथा विग्रहः।} \textcolor{red}{पाणौमहा\-सायक\-चारु\-चापम्} (पाणौ महान् सायकश्चारु चापं च यस्य तम्) इत्यादि\-स्थलमिवात्रापि समाधातव्यम्। वेदान्तेऽस्मिन्माया\-ब्रह्म\-जीवानां निरूपणम्। \textcolor{red}{वेदान्तो नामोपनिषत्प्रमाणं तदुपकारीणि शारीरकसूत्रादीनि च} (वे॰सा॰~३) इति वेदान्तसारे। वेदान्तेऽप्यद्वैत\-वादो विशिष्टाद्वैत\-वादः शुद्धाद्वैत\-वादो द्वैताद्वैत\-वादोऽचिन्त्याद्वैत\-वादो द्वैत\-वाद इति मतानि प्रस्तुतान्याचार्यैः। इदमेव दर्शनमुत्तर\-मीमांसा\-शब्देन व्यवह्रियते। \end{sloppypar}
\begin{sloppypar}\hyphenrules{nohyphenation}\justifying\noindent\hspace{10mm} इममेव वेदमधिकृत्याष्टादश पुराणान्युप\-बृंहितानि वेदव्यासेन। तत्रैषां सङ्ग्रह\-श्लोकः~–\end{sloppypar}
\centering\textcolor{red}{मद्वयं भद्वयं चैव ब्रत्रयं वचतुष्टयम्। \nopagebreak\\
अनापकूष्कलिङ्गानि पुराणानि प्रचक्षते॥}\nopagebreak\\
\raggedleft{–~सङ्ग्रहश्लोकः}\footnote{\textcolor{red}{मद्वयं भद्वयं चैव ब्रत्रयं वचतुष्टयम्। अनापलिङ्गकूस्कानि पुराणानि पृथक्पृथक्॥} (दे॰भा॰पु॰~१.३.२)।}\\
\begin{sloppypar}\hyphenrules{nohyphenation}\justifying\noindent इति। आचार्या इति शेषः। यथा मत्स्य\-पुराणं मार्कण्डेय\-पुराणं भविष्योत्तर\-पुराणं भागवत\-पुराणं ब्रह्म\-वैवर्त\-पुराणं ब्रह्माण्ड\-पुराणं ब्रह्म\-पुराणं वाराह\-पुराणं वामन\-पुराणं विष्णु\-पुराणं वायव्य\-पुराणमग्नि\-पुराणं नारद\-पुराणं पद्म\-पुराणं कूर्म\-पुराणं स्कन्द\-पुराणं लिङ्ग\-पुराणं गरुड\-पुराणमित्यष्टादश पुराणानि। तत्र देवी\-भागवतमेषु। श्रीमद्भागवत\-महा\-पुराणं त्वेभ्यः पृथक्। महा\-पुराणत्वात्। इदमेवोन\-विंशं वेद\-व्यास\-देवस्य चरमा रचनेति विपश्चितां मनीषा।\footnote{श्रीमद्भागवतमष्टादश\-पुराणेभ्यो भिन्नं महापुराणं व्यासदेवस्यान्तिमा रचनेति विपश्चितां मतमिति भावः। उपपुराणानि तु नात्र परामृष्टानि।} इतिहासश्च द्वेधा रामायणं महाभारतं चेति। रामायणस्य प्रवर्तक आदि\-कवि\-वाल्मीकिर्यः खलु पुरा लुण्ठकोऽपि सप्तर्षि\-प्रसाद\-संलब्ध\-लालित्य\-पूर्ण\-प्रतिमा\-लवित्र\-सञ्छिन्न\-मोह\-महा\-महीरुहो हृदय\-सरसी\-समुल्लसित\-सीता\-निवास\-चरण\-सरोरुहो मुनि\-व्रत\-स्वीकार\-संवर्धित\-ललाट\-तपस्तपन\-दीधिति\-समिद्ध\-तीव्र\-तर\-ताप\-सहिष्णु\-तपोऽनुष्ठानोप\-योगि\-योगि\-जन\-दुर्लभ\-जटी\-भूत\-शिरोरुहो 
विमल\-वैराग्य\-विहित\-तृणायमान\-कुटुम्ब\-त्याग\-परिपीत\-रामचन्द्र\-पद\-पाथोज\-पराग\-रस\-सरस\-मानस\-कूजित\-मङ्गलमय\-काव्य\-गान\-संस्मृत\-मुकुन्द\-करुणा\-वत्सलत्व\-सौशील्य\-सौलभ्यादि\-मनुजेतर\-गुण\-गण\-विहित\-भूति\-वल्मीक\-वरूथ\-शरीर\-व्रण\-रघुपति\-चिन्तन\-
विचक्षण\-कविकुल\-शरण\-सम्मण्डितारण्य\-वरेण्य\-वागीश्वर\-प्रसाद\-बृंहित\-वाचस्पति\-शेमुषी\-दुर्लभ\-मनीषोत्कर्ष\-चरण\-पङ्कजभावित\-भारत\-वर्ष\-प्रस्तुतोत्कटादर्श\-रघुनाथ\-लीला\-तरलतर\-सुर\-तरङ्गिणी\-निस्यन्दमान\-प्रेम\-सुधापान\-सञ्जन्यमान\-हर्षोत्कर्ष\-पुण्य\-पुलकित\-तनूरुहः श्रीराम\-विपरीत\-नाम\-जप\-प्रभाव\-संलब्ध\-ब्रह्म\-समान\-वैभवोऽपि\footnote{विपरीत\-नाम\-जपोऽध्यात्म\-रामायणे वर्णितः~– \textcolor{red}{इत्युक्त्वा राम ते नाम व्यत्यस्ताक्षरपूर्वकम्। एकाग्रमनसाऽत्रैव मरेति जप सर्वदा॥} (अ॰रा॰~२.६.८०)।} सीता\-पति\-पद\-पद्म\-भक्ति\-मदिरया मदिरेक्षणो नरपिशाचित\-गृहदारादि\-जगतीतल\-भोग\-सम्भारं समुल्लसित\-षड्विकारमनुभूत\-पत्नी\-पुत्र\-भ्रातृ\-कठोर\-तिरस्कारं स्वार्थ\-शृङ्गारं पतितं परिवारं तृणमिव तिरश्चकार। यस्य खलु 
धारित\-तपोव्रतस्य कवि\-कुल\-गुरोः प्राचेतसस्य भगवतो वाल्मीकेः केकोपमा निरुपमा कोविद\-कुल\-मनोरमा गति\-विभ्रमा विलसित\-त्रिविक्रम\-विक्रमा परिष्कृत\-पण्डित\-परिश्रमा भग्नभव\-भयभ्रमा समुल्लसित\-काव्यकला\-कलाप\-सर्वोत्तम\-संस्करण\-सम्भ्रमा श्री\-रामलीला\-सुरधुनी\-विमल\-वीचि\-कीलाल\-क्रीडा\-गीत\-कवि\-कुल\-व्यतिक्रमा मङ्गल\-मयी लोक\-हितैषिणी जगदुप\-कारिणी कविता\-कानन\-स्वच्छन्द\-विहारिणी धूर्जटि\-जटा\-चारिणीव त्रिपथ\-विहारिणी मुनि\-मनोहारिणी पतित\-तारिणी तरणि\-किरण\-मालिकेव निर्मला मनीषा क्रौञ्च\-द्वन्द्व\-निधन\-निरीक्षण\-सञ्जात\-तरुण\-करुण\-शोक\-समालोकित\-जगदीशा प्रथमं काव्य\-सर्गं व्यधत्त करतल\-चतुर्वर्गाप\-वर्ग\-सम्मोदित\-भर्ग\-सुलभीकृत\-स्वर्गं श्लोकमेव विरचित\-सर्गमिव समुत्ससर्ज च। स च श्लोकोऽनुष्टुप्छन्दसा समगीयत~–\end{sloppypar}
\centering\textcolor{red}{मा निषाद प्रतिष्ठां त्वमगमः शाश्वतीः समाः। \nopagebreak\\
यत्क्रौञ्चमिथुनादेकमवधीः काममोहितम्॥}\nopagebreak\\
\raggedleft{–~वा॰रा॰~१.२.१५}\\
\begin{sloppypar}\hyphenrules{nohyphenation}\justifying\noindent\hspace{10mm} अयमेव निखिल\-काव्य\-जगति प्रथमो रचना\-विशेषः। शापात्मकोऽस्यार्थः~– हे \textcolor{red}{निषाद} नीचैः सीदति तिष्ठतीति निषादस्तत्सम्बुद्धौ निषाद। निम्न\-गामिन्निति भावः। \textcolor{red}{त्वं शाश्वतीः समाः} अनन्ताः समा वर्षाणि यावत् \textcolor{red}{प्रतिष्ठां} सम्मानं \textcolor{red}{मा अगमः} मा प्राप्नुहि \textcolor{red}{यत्} यतो हि \textcolor{red}{क्रौञ्च\-मिथुनात्} क्रौञ्च\-युग्मतः \textcolor{red}{काम\-मोहितमेकं} पुरुषमिति भावस्त्वम् \textcolor{red}{अवधीः}। पुरुषं हत्वा शृङ्गार\-रसमपि करुण\-रसे परिवर्तितवानिति भावः। विचारे कृतेऽयं श्लोकः सर्व\-काव्य\-प्राथम्य\-भाक्तयाऽमङ्गलशापोक्तिमत्तया कथं परिणमितः। अतोऽपरा व्याख्या~– मा सीता। \textcolor{red}{इन्दिरा लोक\-माता मा} (अ॰को॰~१.१.२७क) इति कोशात्। सैव \textcolor{red}{मा सीता नितरां सीदति तिष्ठति यस्मिन् स मानिषादः}। अधिकरणे घञ्।\footnote{\textcolor{red}{नितरां सीदति यस्मिन्स निषादः}। निपूर्वकात् \textcolor{red}{सद्‌}\-धातोः (\textcolor{red}{षद्ऌँ विशरण\-गत्यवसादनेषु} धा॰पा॰~८५४, १४२७) \textcolor{red}{हलश्च} (पा॰सू॰~३.३.१२१) इत्यनेनाधिकरणे घञि \textcolor{red}{धात्वादेः षः सः} (पा॰सू॰~६.१.६४) इत्यनेन सत्वे \textcolor{red}{अत उपधायाः} (पा॰सू॰~७.२.११६) इत्यनेन वृद्धौ \textcolor{red}{सदिरप्रतेः} (पा॰सू॰~८.३.६६) इत्यनेन षत्वे विभक्तिकार्ये \textcolor{red}{निषादः}। \textcolor{red}{माया निषाद इति मानिषादस्तत्सम्बुद्धौ मानिषाद}।}
अथवा \textcolor{red}{मायां सीतायां नितरां सीदति तिष्ठति यः स मानिषादः}। कर्तरि घञ्।\footnote{\textcolor{red}{नितरां सीदति निषीदतीति निषादः}। सञ्ज्ञायां बाहुलकात्कर्तरि घञ्। प्रक्रिया पूर्ववत्। \textcolor{red}{मायां निषाद इति मानिषादस्तत्सम्बुद्धौ मानिषाद}।} श्रीरामचन्द्र इति भावः। अर्थात्~– हे \textcolor{red}{मानिषाद} सीतानिवास श्रीराम \textcolor{red}{त्वं} भवान् \textcolor{red}{शाश्वतीः समाः} अनन्त\-वर्षाणि यावत् \textcolor{red}{प्रतिष्ठां} सम्मानं पूजाम् \textcolor{red}{अगमः} प्राप्तवान् \textcolor{red}{यत्} यतो हि \textcolor{red}{क्रौञ्च\-मिथुनात्} क्रौञ्चयोः पक्षिवद्रावण\-मन्दोदर्यो\-र्मिथुनाद्युग्मात् \textcolor{red}{काम\-मोहितं} कामिनम् \textcolor{red}{एकं} रावणम् \textcolor{red}{अवधीः} हतवान्। रावणं हत्वा भवता महती प्रतिष्ठा समर्जितेति श्लोकस्य भावः। अयमेव सर्व\-प्रथमः श्लोकोऽनन्तरं नारदोपदिष्ट\-सङ्क्षिप्त\-सीताराम\-कथां विगतव्यथां भगवान् वाल्मीकिरादि\-काव्यरूपां पञ्चशतसर्गात्मिकां षट्काण्डां रामायण\-नामधेयां चतुर्विंशति\-साहस्री\-संहितामिमां\footnote{\textcolor{red}{चतुर्विंशत्सहस्राणि श्लोकानामुक्तवानृषिः। तथा सर्गशतान्पञ्च षट् काण्डानि तथोत्तरम्॥} (वा॰रा॰~१.४.२)।} काव्यवसन्त\-कोकिलः कोकिल\-काकलीमिव चुकूज। इदमितिहास\-भूतं वेद\-सम्मतम्। तत्रेत्थं ज्ञातव्यम्। यथा विज्ञाने सम्प्रति \textcolor{red}{साईन्स्‌}\-नामधेये योग\-प्रयोगौ \textcolor{red}{थिअरी\-प्रैक्टिकल्}\-शब्दाभ्यां व्यवह्रियेते। तथैव वेदः \textcolor{red}{थिअरी} इति। अर्थात्सिद्धान्त\-सङ्ग्रहः। रामायणं \textcolor{red}{प्रैक्टिकल्} इति। अर्थात्प्रयोग\-सङ्ग्रहः। अत्र वेदस्य सम्पूर्णाः सिद्धान्ता अनेक\-कथा\-व्याजेन सविस्तरं प्रतिपादिताः। अयमेवेतिहासः। इतिहास\-शब्दो हि त्रिभिः शब्दैर्निष्पद्यते \textcolor{red}{इति ह आस}। \textcolor{red}{इति}\-शब्दः पूर्वकाल\-घटना\-क्रम\-सूचको \textcolor{red}{ह}\-शब्दो निश्चयवाचक \textcolor{red}{आस}\-शब्दो भूतकाल\-क्रिया\-सूचकः। एवमिदं निश्चितमासीदित्येवमितिहास\-शब्दार्थः। अस्य सम्पूर्णतो वेदमूलतां स्वयं वाल्मीकिरकथयत्~–\end{sloppypar}
\centering\textcolor{red}{इदं पवित्रं पापघ्नं पुण्यं वेदैश्च सम्मितम्।\nopagebreak\\
यः पठेद्रामचरितं सर्वपापैः प्रमुच्यते॥}\nopagebreak\\
\raggedleft{–~वा॰रा॰~१.१.९८}\\
\begin{sloppypar}\hyphenrules{nohyphenation}\justifying\noindent\hspace{10mm} द्वितीय इतिहास\-ग्रन्थो महाभारत\-नामधेयः। अयमष्टादश\-पर्वयुक्तो लक्षश्लोकात्मकः। अस्य रचयिता श्रीमन्नारायण\-कलावतारः साक्षाद्भगवान् वेदव्यासः। यथा रामायणे महर्षिणा वाल्मीकिना परब्रह्मणः सनातस्य मर्यादा\-पुरुषोत्तमस्य चारु\-चरित्रं सजीव\-चित्रमिव चित्रितं तस्य रामो महाविष्णुः परं ब्रह्म सुरकार्य\-चिकीर्षया दशरथ\-पुत्रतां स्वीकृत्य कौसल्यायां प्रकटयाम्बभूव यथा~–\end{sloppypar}
\centering\textcolor{red}{स हि देवैरुदीर्णस्य रावणस्य वधार्थिभिः। 	\nopagebreak\\
अर्थितो मानुषे लोके जज्ञे विष्णुः सनातनः॥}\nopagebreak\\
\raggedleft{–~वा॰रा॰~२.१.७}\\
\begin{sloppypar}\hyphenrules{nohyphenation}\justifying\noindent\hspace{10mm} तथैव महाभारतेऽपि परम\-भागवतस्य पाण्डु\-वंशस्य वर्णनच्छलेनातसी\-कुसुमोपमेय\-कान्ते रुक्मिणी\-रमणस्य भगवतो गृहीत\-नराकारस्य पार्थ\-सूतस्य वासुदेवस्य श्री\-कृष्ण\-चन्द्रस्य भक्त\-वश्यतोप\-बृंहिता वेदव्यासेन। अत्र भगवान् कृपण\-वत्सलो व्यासः स्त्री\-शूद्र\-द्विज\-बन्धूनां कृतेऽपि वेदार्थो यथा सरलः स्यादिति हेतोः
समस्त\-वेदे वर्णित\-विषयाणां कथोदाहरण\-प्रसङ्गेषु निबन्धनमकार्षीत्। अतः श्रूयते \textcolor{red}{यन्न भारते तन्न भारते} इति। श्रीमद्भागवतेऽपि महाभारत\-रचना\-कारणमिदमेव प्रादर्शि	~–\end{sloppypar}
\centering\textcolor{red}{स्त्रीशूद्रद्विजबन्धूनां त्रयी न श्रुतिगोचरा। \nopagebreak\\
कर्मश्रेयसि मूढानां श्रेय एवं भवेदिह। \nopagebreak\\
इति भारतमाख्यानं कृपया मुनिना कृतम्॥}\nopagebreak\\
\raggedleft{–~भा॰पु॰~१.४.२५}\\
\begin{sloppypar}\hyphenrules{nohyphenation}\justifying\noindent\hspace{10mm} एवं निगूढा वेदार्था इतिहास\-पुराणाभ्यामुप\-बृंह्यन्ते। स्वयमेव वेदव्यासः कथयति~–\end{sloppypar}
\centering\textcolor{red}{इतिहासपुराणाभ्यां वेदार्थमुपबृंहयेत्। \nopagebreak\\
बिभेत्यल्पश्रुताद्वेदो मामसौ प्रहरेदिति॥}\nopagebreak\\
\raggedleft{–~म॰भा॰~१.१.२६७, १.१.२९३, १.१.२९४}\\
\begin{sloppypar}\hyphenrules{nohyphenation}\justifying\noindent\hspace{10mm} इममेव वेदमाधारी\-कृत्य प्रवर्तिताश्चतुर्दश विद्याः। तत्र षड्दर्शनान्यान्वीक्षिकी त्रयी वार्ता दण्डनीतिः पुराणं मीमांसा धर्मशास्त्रं सङ्गीतं चेति। तत्र षड्दर्शनेषु साङ्ख्यं योगो वैशेषिकं न्यायः पूर्व\-मीमांसोत्तर\-मीमांसा चेति। \textcolor{red}{दर्शनं नाम दृश्यते ज्ञायते परमात्म\-तत्त्वं येन तत्} इति व्युत्पत्तौ \textcolor{red}{दृश्‌}\-धातोः (\textcolor{red}{दृशिँर् प्रेक्षणे} धा॰पा॰~९८८) \textcolor{red}{करणाधि\-करणयोश्च} (पा॰सू॰~३.३.११७) इति सूत्रेण ल्युट्। टकारस्येत्सञ्ज्ञा \textcolor{red}{हलन्त्यम्} (पा॰सू॰~१.३.३) इति सूत्रेण \textcolor{red}{तस्य लोपः} (पा॰सू॰~१.३.९) इत्यनेन लोपश्च। लकारस्य \textcolor{red}{लशक्वतद्धिते} (पा॰सू॰~१.३.८) इति सूत्रेणेत्सञ्ज्ञा पुनर्लोपः।\footnote{\textcolor{red}{दृशिँर्} इत्यत्र \textcolor{red}{इर उपसङ्ख्यानम्} (वा॰~१.३.७) इत्यनेनेर इत्सञ्ज्ञा \textcolor{red}{तस्य लोपः} (पा॰सू॰~१.३.९) इत्यनेन लोपश्च बोध्यः।} उकारस्य विधान\-सामर्थ्यादुप\-देशानु\-नासिकत्वेऽपि नेत्त्वम्। लकारस्य लित्स्वरार्थः प्रयोगः।\footnote{\textcolor{red}{लिति} (पा॰सू॰~६.१.१९३) इत्यनेन लित्स्वरे प्रत्ययात्पूर्वः स्वर उदात्तः। दर्शनम्~\arrow \textcolor{red}{लिति} (पा॰सू॰~६.१.१९३)~\arrow \textcolor{red}{अनुदात्तौ सुप्पितौ} (पा॰सू॰~३.१.४)~\arrow \textcolor{red}{अनुदात्तं पदमेकवर्जम्‌} (पा॰सू॰~६.१.१५८)~\arrow दर्श॒न॒म्~\arrow \textcolor{red}{उदात्तादनुदात्तस्य स्वरितः} (पा॰सू॰~८.४.६६)~\arrow दर्श॑न॒म्~\arrow \textcolor{red}{स्वरितात्संहितायामनु\-दात्तानाम्} (पा॰सू॰~१.२.३९)~\arrow दर्श॑नम्। ऋग्वेद\-संहितायां च \textcolor{red}{दर्श॑नाय} (ऋ॰वे॰सं॰~१.११६.२३) इति। न च \textcolor{red}{उदात्तादनुदात्तस्य स्वरितः} (पा॰सू॰~८.४.६६) इत्यस्य त्रिपादीस्थत्वात् \textcolor{red}{स्वरितात्संहितायामनु\-दात्तानाम्} (पा॰सू॰~१.२.३९) इत्यस्य सपाद\-सप्ताध्यायी\-सूत्रस्यासिद्धे स्वरितकार्ये कथं प्रवृत्तिरिति चेत्। विधान\-सामर्थ्यात्त्रिपादी\-सूत्रकार्यं सिद्धम्। यद्वा \textcolor{red}{स्वरितस्यार्ध\-ह्रस्वोदात्तादोदात्त\-स्वरित\-परस्य सन्नतरादूर्ध्वमुदात्तादनुदात्तस्य स्वरितात्कार्यं स्वरितादिति सिद्ध्यर्थम्} (पा॰सू॰~१.२.३२) इति वार्त्तिक\-वचनात्सिद्धम्। यद्वा \textcolor{red}{न मु ने} (पा॰सू॰~८.२.३) इत्यत्र \textcolor{red}{न} इति योगविभागात् \textcolor{red}{पूर्वत्रासिद्धम्} (पा॰सू॰~८.२.१) इति सूत्रस्याप्रवृत्तिः क्वाचित्का। अन्तिम\-पक्षस्य विस्तराय \pageref{sec:jaayeti_siiteti}तमे पृष्ठे \ref{sec:jaayeti_siiteti} \nameref{sec:jaayeti_siiteti} इति प्रयोगस्य विमर्शं पश्यन्तु।} तस्मिंश्चादृष्ट\-जनकत्वरूपं फलम्। ततो \textcolor{red}{युवोरनाकौ} (पा॰सू॰~७.१.१) इत्यनेनानादेशे गुणे\footnote{\textcolor{red}{पुगन्त\-लघूपधस्य च} (पा॰सू॰~७.३.८६) इत्यनेन।} रपरत्वे\footnote{\textcolor{red}{उरण् रपरः} (पा॰सू॰~१.१.५१) इत्यनेन।} कृते \textcolor{red}{दर्शनम्} इति सिद्धम्। तत्रास्तिक\-दर्शनानि षट्। \textcolor{red}{आस्तिको नाम वेद\-प्रतिपादक\-वेदप्रतिपाद्य\-मान्यतास्वास्थावान्}। \textcolor{red}{अस्ति\-नास्ति\-दिष्टं मतिः} (पा॰सू॰~४.४.६०) इति सूत्रेण \textcolor{red}{अस्ति}\-शब्दात् \textcolor{red}{ठक्‌}\-प्रत्ययः। विधान\-सामर्थ्यात् \textcolor{red}{चूटू} (पा॰सू॰~१.३.७) इत्यनेन ठकारस्य नेत्त्वं कस्येत्सञ्ज्ञा\-लोपयोः ठस्येकादेशे \textcolor{red}{ठस्येकः} (पा॰सू॰~७.३.५०) इति सूत्रेण \textcolor{red}{यचि भम्} (पा॰सू॰~१.४.१८) इत्यनेन भसञ्ज्ञायां \textcolor{red}{यस्येति च} (पा॰सू॰~६.४.१४८) इत्यनेनास्तिघटकेकारलोपे \textcolor{red}{तद्धितेष्वचामादेः} (पा॰सू॰~७.२.११७) इत्यनेन वृद्धौ विभक्ति\-कार्ये \textcolor{red}{आस्तिकः} इति सिद्धम्। अर्थाद्वेद\-शास्त्रेतिहास\-पुराण\-स्मृति\-धर्मशास्त्रे सिद्धान्ते श्रद्धधानत्वे 
सतीश्वर\-पूजकत्वे सति गो\-विप्र\-प्रतिमा\-पूजकत्वमास्तिकत्वम्। अत एव व्यास आस्तिकानुत्साहयति~–\end{sloppypar}
\centering\textcolor{red}{सन्दिग्धेऽपि परे लोके कर्तव्यो धर्मसङ्ग्रहः। \nopagebreak\\
नास्ति चेदस्ति का हानिरस्ति चेन्नास्तिको हतः॥}\footnote{मूलं भारते शान्ति\-पर्वणि मृग्यम्।}\nopagebreak\\
\begin{sloppypar}\hyphenrules{nohyphenation}\justifying\noindent इति महाभारते शान्तिपर्वणि। तत्र साङ्ख्य\-दर्शन\-प्रवर्तको भगवदंशावतारः साक्षाद्भगवान् कपिलः। अत्र त्रीणि तत्त्वानि व्यक्तमव्यक्तं ज्ञ इति। व्यक्तं 
त्रयोविंशति\-तत्त्व\-निकरः। महदहङ्कारो दशेन्द्रियाणि चक्षुः\-श्रोत्र\-रसना\-घ्राण\-त्वक्पाणि\-पाद\-पायूपस्थ\-वागाख्यानि मनः शब्द\-स्पर्श\-रूप\-रस\-गन्धाः क्षिति\-जल\-पावक\-गगन\-समीरा इति त्रयोविंशति\-तत्त्वानि व्यक्तानि। अव्यक्तं प्रकृतिः पुरुषश्च। ज्ञः षड्विंशो हि पुरुषः। इदं तावत्साङ्ख्ये \textcolor{red}{सम्यक्ख्यायन्ते गण्यन्ते योद्धारो यस्मिन् तादृशे दुःख\-त्रयाभिघात\-रूपे युद्धे सङ्ख्ये प्रवर्तितमिति साङ्ख्यम्}। सङ्ख्यं बहुशो युद्धादौ प्रयुक्तं यथा \textcolor{red}{एवमुक्त्वाऽर्जुनः सङ्ख्ये} (भ॰गी॰~१.४७) इति।\footnote{\textcolor{red}{युद्धमायोधनं जन्यं प्रधनं प्रविदारणम्॥ मृधमास्कन्दनं सङ्ख्यं समीकं साम्परायिकम्।} (अ॰को॰~२.८.१०३-१०४) इत्यमरः।} आधिभौतिकाधिदैविकाध्यात्मिक\-नामधेयानां त्रयाणां दुःखानां प्रतिकूलतयाऽऽत्मनि व्याघातात्तदभिघातक\-हेतु\-विषयक\-जिज्ञासायां व्यक्ताव्यक्त\-ज्ञ\-ज्ञान\-दीपकरूपमिदं प्राचीनतमम्। तद्यथा~–\end{sloppypar}
\centering\textcolor{red}{दुःखत्रयाभिघाताज्जिज्ञासा तदपघातके हेतौ। \nopagebreak\\
दृष्टे साऽपार्था चेन्नैकान्तात्यन्ततोऽभावात्॥}\nopagebreak\\
\raggedleft{–~सा॰का॰~१}\\
\begin{sloppypar}\hyphenrules{nohyphenation}\justifying\noindent\hspace{10mm} अत्र प्रकृति\-पुरुष\-संयोगेन सृष्टिः। तयोः संयोगस्तत्र पङ्ग्वन्धवत्संयोगः। अत्र धाराद्वयी। केचिन्निरीश्वरवादं केचिच्च सेश्वरवादं व्यवस्थापयन्ति। इदं दर्शनं द्वैतवाद\-परम्। अत्र पञ्चविंशतितत्त्वानां सङ्ग्रहः। प्रकृतिः कर्त्री पुरुषः पुष्कर\-पलाशवन्निर्लेपः। पुरुषेण संयुक्तायां प्रकृतौ पुरुष\-निष्ठ\-चेतनत्वमारोप्यते। पुरुषे च प्रकृति\-गत\-कर्तृत्वमध्यस्यते। तथा चोक्तम्~–\end{sloppypar}
\centering\textcolor{red}{मूलप्रकृतिरविकृतिर्महदाद्याः प्रकृतिविकृतयः सप्त। \nopagebreak\\
षोडशकस्तु विकारो न प्रकृतिर्न विकृतिः पुरुषः॥}\nopagebreak\\
\raggedleft{–~सा॰का॰~३}\\
\begin{sloppypar}\hyphenrules{nohyphenation}\justifying\noindent\hspace{10mm} अस्मिन्दर्शने त्रीण्येव प्रमाणानि स्वीचक्रिरे दृष्टमनु\-मानमाप्त\-वचनं च।\footnote{\textcolor{red}{दृष्टमनु\-मानमाप्त\-वचनं च सर्व\-प्रमाण\-सिद्धत्वात्। त्रिविधं प्रमाणमिष्टं प्रमेय\-सिद्धिः प्रमाणाद्धि॥} (सा॰का॰~४)।} अनुमानमपि त्रिविधं पूर्ववच्छेष\-वत्सामान्यतो\-दृष्ट\-भेदेन।\footnote{\textcolor{red}{त्रिविधमनु\-मानमाख्यातम्} (सा॰का॰~५)। \textcolor{red}{त्रिविधमनु\-मानमाख्यातम्। पूर्ववच्छेषवत्सामान्यतो\-दृष्टं चेति} (सा॰का॰ गौ॰भा॰~५)। न्यायशास्त्रेऽपि~– \textcolor{red}{अथ तत्पूर्वकं त्रिविधमनुमानं पूर्ववच्छेषवत्सामान्यतो दृष्टं च} (न्या॰सू॰~१.१.५)।} यत्र हेतुं दृष्ट्वा व्याप्तिस्मरण\-पुरःसरं भूतपूर्वं कार्यमनुमीयते तत्र पूर्ववद्यथा जलसङ्कुलं जलाशयं दृष्ट्वा पूर्ववर्षाऽनुमिता।\footnote{वर्षर्त्वर्थे वर्षा\-शब्दो बहुवचनान्तो वृष्ट्यर्थे चैकवचनान्त इत्याप्टे\-कोशः।} यत्र शेषं दृष्ट्वा कार्यमनुमीयते तत्र शेषवद्यथा बिन्दुमात्रं समुद्रजलं निपीय शेषं क्षारमनुमीयते। यत्र सामान्य\-परिस्थित्या विशेषस्यानुमानं तत्र सामान्यतो\-दृष्टम्। यथा कुत्रचिदाम्रफलानां दर्शनेनेतरत्र रसालफलत्वानुमानम्। शास्त्रेऽस्मिन्नात्मैव पुरुष\-शब्देन व्यवह्रियते। तथा च \textcolor{red}{पुरि शरीरे शेते साक्षित्वेन तिष्ठति यः स पुरुष आत्मा}।\footnote{पृषोदरादित्वात्साधुः। \textcolor{red}{पुरि देहे शेते शी–ड पृषो॰} इति वाचस्पत्यकारः। भागवते च~– \textcolor{red}{पुराण्येन सृष्टानि नृतिर्यगृषिदेवताः। शेते जीवेन रूपेण पुरेषु पुरुषो ह्यसौ॥} (भा॰पु॰~७.१४.३७)।} साङ्ख्याः पुरुषबहुत्वमपि साधयन्ति।\footnote{\textcolor{red}{जनन\-मरण\-करणानां प्रति\-नियमादयुगपत्प्रवृत्तेश्च। पुरुष\-बहुत्वं सिद्धं त्रैगुण्य\-विपर्ययाच्चैव॥} (सा॰का॰~१८)।} यतो ह्येकस्मिन् सति पुरुषे सर्वेषां सहैव युगपज्जन्ममरणे स्याताम्।\footnote{\textcolor{red}{यद्येक एवाऽत्मा स्यात्तत एकस्य जन्मनि सर्व एव जायेरन्नेकस्य मरणे सर्वेऽपि म्रियेरन्} (सा॰का॰ गौ॰भा॰~१८)।} यथैका विद्युद्यदा गच्छति तदा सहैव सर्वेषु कक्षेष्वन्धकारो भवति तदागमने युगपदेव सकल\-धामसु सुप्रकाशः सहैव विद्युद्व्यजन\-वीजनम्। तस्मात्पुरुषाणां बहुत्वम्। एकत्वे सति जनानां सुख\-दुःख\-प्रभृतीनां स्वभावस्य च वैषम्यं कथमपि न सङ्गच्छेत। राद्धान्तोऽयं समीचीनो युक्तियुक्तो वैष्णव\-सम्मतश्च। अन्यथा यदि सर्वेष्वेक एवाऽत्मा तर्हि कथमेकस्मिन्नेव क्षणे कोऽपि म्रियते कोऽपि जायते कोऽपि तिष्ठति कोऽपि वर्धते कश्चन विपरिणमति कश्चिद्ध्रसति। कथं वा कोऽपि विद्वान् कश्चिन्मूर्खः केचन खलाः केचित्साधवः। अस्तु। इदं शास्त्रं चापि वेदमूलकमेव। पुरुषबहुत्वे श्रुतिरपि प्रमाणं यथा \textcolor{red}{इन्द्रो॑ मा॒याभि॑ पुरु॒रूप॑ ईयते} (ऋ॰वे॰सं॰~६.४७.१८)।\end{sloppypar}
\begin{sloppypar}\hyphenrules{nohyphenation}\justifying\noindent\hspace{10mm} योगश्च चित्त\-वृत्ति\-निरोधपरः। यथा योगसूत्रे \textcolor{red}{योगश्चित्तवृत्तिनिरोधः} (यो॰सू॰~१.२)। चित्त\-वृत्ति\-निरोधायाष्टाङ्गयोगस्य चर्चा। यम\-नियमासन\-प्राणायाम\-प्रत्याहार\-धारणा\-ध्यान\-समाधयोऽष्टाङ्गानि। एतैश्चित्तवृत्तिर्निरुध्यते। अयं च हठयोग इति कथ्यते। राजयोगस्त्वीश्वर\-प्रणिधानात्मकः। यथा योग\-सूत्रे \textcolor{red}{ईश्वरप्रणिधानाद्वा} (यो॰सू॰~१.२३)। हठ\-योगे प्राण\-जयेन मनो जयन्ति योगिनः। राजयोगे च मनोजय\-द्वारेण प्राणमतिशेरते संयमिनः। अस्याऽचार्यो भगवान् पतञ्जलिः। स एव पाणिनि\-व्याकरण\-महाभाष्यकारः। स एव चाऽयुर्वेदे चरक\-संहिता\-रचयितेति श्रूयते गुरुभ्यः। तथा च सङ्कीर्तयन्ति गुरुचरणाः~–\end{sloppypar}
\centering\textcolor{red}{योगेन चित्तस्य पदेन वाचां मलं शरीरस्य च वैद्यकेन।\nopagebreak\\
योऽपाकरोत्तं प्रवरं मुनीनां पतञ्जलिं प्राञ्जलिरानतोऽस्मि॥}\footnote{मूलं मृग्यम्।}\\
\begin{sloppypar}\hyphenrules{nohyphenation}\justifying\noindent अर्थाद्योगसूत्रं 
निर्माय चित्तमलममलितवन्तं व्याकरण\-शास्त्रे पद\-पदार्थ\-प्रतिपादक\-परम\-प्रमाणभूत\-महाभाष्य\-सागर\-संरचनया वाणीं निर्दोषयन्तं चरकसंहिता\-प्रतिपादनेन शरीरं भूषयन्तं प्राञ्जलिः पतञ्जलिं प्रणिपतामीति तात्पर्यम्। \label{text:patanjali} शब्दस्यास्य व्युत्पत्ति\-प्रकारश्च \textcolor{red}{अञ्जलौ पतन्} इति विग्रहे सप्तम्यन्ताञ्जलिशब्दस्य शतृ\-प्रत्ययान्त\-प्रथमान्त\-\textcolor{red}{पतन्‌}\-शब्देन मयूर\-व्यंसकादित्वात्समासः\footnote{\textcolor{red}{मयूर\-व्यंसकादयश्च} (पा॰सू॰~२.१.७२) इत्यनेन।} \textcolor{red}{पतन्‌}\-शब्दस्य च पूर्वनिपातः।\footnote{\textcolor{red}{पतन्तः अञ्जलयोऽस्मिन् नमस्कार्यत्वादिति पतञ्जलिः} (त॰बो॰~७९)। \textcolor{red}{पतञ्जलिरिति। पतन् अञ्जलिर्यस्मिन् नमस्कार्यत्वादिति विग्रहः। अत्र ‘अत्’ इति टेरकारस्य च स्थाने पररूपमकारः। केचित्तु गोर्नदाख्य\-देशे कस्यचिदृषेस्सन्ध्योपासन\-समयेऽञ्जलेर्निर्गत इत्यैतिह्यात् ‘अञ्जलेः पतन्’ इति विगृह्णन्ति। मयूर\-व्यंसकादित्वात्समासः} (बा॰म॰~७९)।} एवं शकन्ध्वादि\-गणमाकृति\-गणं मत्वा \textcolor{red}{शकन्ध्वादिषु पररूपं वाच्यम्} (वा॰~६.१.९४) इति वार्त्तिकेन पररूपे विभक्तिकार्ये \textcolor{red}{पतञ्जलिः}। इत्थमाख्यायिका श्रुता गुरुमुखेभ्यो यदीसा\-पूर्वान्तिम\-शताब्द्यां पाणिनि\-सूत्रार्थ\-परिश्चिकीर्षयाऽशेष\-विशेषातीतो भगवाञ्छेष एकस्यचित्तपो\-निष्ठ\-ब्राह्मणस्य सन्ध्यार्थं गतस्य सरोवराभ्यासमञ्जलिं प्रसार्य सहस्र\-रश्मि\-मालिनं परम\-प्रकाश\-शालिनं कमलिनी\-कुल\-वल्लभं भगवत्साकाररूपं दिव्य\-रोचिषा विभासित\-भुवन\-मण्डलं दीप्तिमय\-मण्डलं जगदाखण्डलं भास्वन्तं विवस्वन्तं निखिल\-जगदुपादान\-भूतं शास्त्र\-स्व\-स्यन्दनं कश्यप\-नन्दनमदिति\-कीर्ति\-केतुं वैदिक\-धर्म\-सेतुमलौकिक\-तेजसं प्रशान्तं भगवन्तं भास्करं समुप\-तिष्ठमानस्य समपतत्सर्पशावकी\-भूयाञ्जलि\-पुटे। आकस्मिक\-करपुट\-भोगि\-तोक\-पतन\-सञ्जात\-प्रबल\-कुतूहलतया द्विजन्मना \textcolor{red}{कोर्भवान्} इति पृष्टः सन् \textcolor{red}{सप्पोऽहम्} इति समुदतीतरत्परिकलित\-सकल\-विद्याविशेषो भगवाञ्छेषः। सर्पघटकरेफः कुत्रेति प्रतिपृष्टः सन् \textcolor{red}{को भवान् कोर्भवान्} इति निरर्थकं मध्येरेफं
व्यवहरता भवता पूर्वमेवोक्त इति वयङ्ग्य\-कटाक्षं समार्पिपत्। स एव सर्प\-शावकः शाब्दिक\-सम्प्रदाय\-नभोमण्डल\-समागत\-पाणिनि\-सूत्रार्थ\-ज्ञान\-तरुण\-तल\-तिमिर\-पटल\-पाटन\-प्रबल\-प्रभञ्जनोपमो निखिल\-पण्डित\-मनोरमा\-परिणत\-परिश्रमो विगत\-भ्रमः प्रस्तुताष्टाध्यायी\-मौलिक\-पाठक्रमः पण्डित\-जगदुद्धार\-चिकीर्षया विप्रपुत्रतामङ्गीचकार। स एव भगवान् पतञ्जलिः पश्चाद्योगिजन\-सेवितं दुराराध्यं सकल\-काम\-कल्याण\-कल्पद्रुमं स्वर्गापवर्ग\-चतुर्वर्ग\-सोपानं साधना\-जाटिल्य\-कटु\-कण्टक\-विषमं सुदुर्गमं योगि\-वर्त्म योग\-दर्शन\-दीपकेन परिश्चकार। इदं दर्शनमपि द्वैतवादपरमीश्वरवाद\-प्रतिपादकमास्तिकं वेदमूलकञ्च।\end{sloppypar}
\begin{sloppypar}\hyphenrules{nohyphenation}\justifying\noindent\hspace{10mm} वैशेषिकं दर्शनं तावत्सप्त\-पदार्थात्मकम्। अत्र द्रव्य\-गुण\-कर्म\-सामान्य\-विशेष\-समवायाभाव\-नामधेय\-सप्त\-पदार्थानां ज्ञानान्मोक्षः। एतस्याऽचार्यः काणादः। इदं सकल\-शास्त्रोपकारकम्। तथा च गुरु\-चरणाः प्राहुर्यत् \textcolor{red}{काणादं पाणिनीयञ्च सर्वशास्त्रोपकारकम्} इति।\footnote{मूलं मृग्यम्।} इदमपि वेद\-मूलकमास्तिक\-दर्शनमीश्वर\-वाद\-परम्। न्याय\-दर्शनं तावत्प्रमाण\-प्रमेयादि\-षोडश\-वस्तु\-मीमांसापरम्। अत्र षोडश\-तत्त्व\-ज्ञानादेव मोक्षः। न्यायदर्शनस्याऽचार्यो भगवान् गौतमः। इदमपीश्वरं कर्तृत्वेन स्वीकरोति। तद्यथा यद्यत्कार्यं तत्तत्सकर्तृकमिति व्याप्तिं स्मरन्तो नैयायिका इत्थमनुमान्ति यत्क्षित्यङ्कुरादिकं कर्तृजन्यं कार्यत्वाद्घटादिवत्। इदं तु परमास्तिक\-दर्शनम्। एतस्य प्राचीनाचार्या मिथिला\-भुवः शेखरीभूताः श्रीमदुदयनाचार्या ईश्वर\-सिद्धावेव न्याय\-सिद्धान्त\-कुसुमाञ्जलि\-नामक\-ग्रन्थं प्रतुष्टुवुः। न्याय\-शास्त्रं परम\-नास्तिक\-विद्या\-मद\-मत्त\-गजेन्द्र\-गण्ड\-स्थल\-भेदन\-शीलं तरल\-तर्क\-तीक्ष्ण\-नख\-युक्त\-सिंहोपमम्। बौद्धादीनां खण्डनायेदमेव धारित\-व्यसनम्। आकर्ण्यते प्राचीनेभ्यो गुरुचरणेभ्यो यत्कदाचित्तीर्थ\-यात्रा\-व्यपदेशेन चरण\-सरोज\-रजः\-सनाथितोत्कल\-प्रदेशाः स्वीकृत\-शास्त्रार्थ\-व्यसन\-रत\-पण्डितेन्द्र\-वेशाः श्रीमदुदयनाचार्य\-चरणाः शरीर\-सन्निहित\-नास्तिक\-कृतेश्वर\-खण्डन\-सन्ताप\-गम्भीर\-व्रणाः निम्नगा\-वल्लभ\-तरस्वि\-तरल\-तरङ्ग\-सङ्क्षालित\-पाद\-कमलाममलां जगन्नाथपुरीं समलञ्चक्रुः। तत्र 
श्रीमज्जगन्नाथ\-भुवन\-पावन\-पादारविन्द\-निस्स्यन्द\-प्रेम\-मकरन्द\-रोलम्बायमान\-मानसतया
भगवन्मुख\-मृगाङ्क\-सौन्दर्य\-सुधा\-माधुरीं पिपासवो जिज्ञासवश्च तत्पिहित\-द्वारोद्घाटन\-कालमर्चकाद्विलम्ब\-श्रवण\-सञ्जात\-रोष\-ज्वाला\-जालं नाल्पं धैर्यं धरन्तो व्याहार्षुर्व्यङ्ग्य\-मिश्रं श्लोकमिमम्। यत्~–\end{sloppypar}
\centering\textcolor{red}{ऐश्वर्यमदमत्तोऽसि मामनादृत्य तिष्ठसि।\nopagebreak\\
उपस्थितेषु बौद्धेषु मदधीना तव स्थितिः॥}\nopagebreak\\
\raggedleft{–~मुक्तकम्}\\
\begin{sloppypar}\hyphenrules{nohyphenation}\justifying\noindent\hspace{10mm} आकर्ण्य तत्प्रेम\-विह्वलां वाणीं पण्डित\-मचर्चिकाया भक्त\-वत्सलो भगवानकालमुद्घाट्य द्वारमुदयनमुदय\-गिरि\-रविमिवाऽत्मानं दर्शयामास। अस्मिन्न्याये सम्प्रदाय\-द्वयं प्रावर्तत। प्राचीन\-न्यायस्य नव्य\-न्यायस्य च। प्राचीन\-न्याय\-प्रवर्तका उदयनाचार्य\-प्रमुख\-मैथिलाः। नव्य\-न्यायाचार्या गङ्गेशोपाध्याय\-गोकुलचन्द्र\-रघुनाथशिरोमणि\-विश्वनाथ\-तर्कपञ्चानन\-गदाधर\-भट्टाचार्य\-प्रभृतयो मैथिलबङ्गीयाः।\end{sloppypar}
\begin{sloppypar}\hyphenrules{nohyphenation}\justifying\noindent\hspace{10mm} तत्र पूर्व\-मीमांसा\-दर्शनं कर्म\-काण्ड\-प्रतिपादकम्। इदं वेदस्य प्रयोग\-पद्धति\-परिष्कारं मीमांसते। एतस्याऽचार्या जैमिनि\-महाभागाः। मीमांसा\-सूत्रं द्वादशाध्याय\-परम्। अस्मिन् धर्म\-जिज्ञासैव मीमांसिता। यथा \textcolor{red}{अथातो धर्मजिज्ञासा} (मी॰सू॰~१.१.१)। इदं तु सर्वभावेन वेदमाश्रयति। अत्रापि द्वौ सम्प्रदायौ भाट्टः प्राभाकरश्च। एतन्मतेऽपौरुषेयवेद एव निखिल\-लोक\-नियन्ता। तत्र कर्मैव ब्रह्मत्वेन प्रतिपादितम्।\footnote{\textcolor{red}{यं शैवाः समुपासते शिव इति ब्रह्मेति वेदान्तिनो बौद्धा बुद्ध इति प्रमाणपटवः कर्तेति नैयायिकाः। अर्हन्नित्यथ जैनशासनरताः कर्मेति मीमांसकाः सोऽयं वो विदधातु वाञ्छितफलं त्रैलोक्यनाथो हरिः॥} (ह॰ना॰~१.३)।}\end{sloppypar}
\begin{sloppypar}\hyphenrules{nohyphenation}\justifying\noindent\hspace{10mm} वेदान्त\-दर्शनं षष्ठम्। इदं वेदस्य ज्ञानकाण्डं सुस्पष्टयति। एतस्य प्रवर्तका नारायण\-कलावतारा हृदय\-सन्निहित\-सकल\-श्रुति\-सारा वेद\-सिद्धान्तानुशीलन\-भागवत\-धर्म\-परिशीलन\-परिष्कृत\-विचारा विहित\-परम\-धर्म\-रस\-तरङ्गिणी\-प्रचारा निखिल\-कोविद\-कवि\-कुलालङ्कारा निरहङ्काराः सन्दर्शित\-समाजोचित\-मुनि\-मनो\-दुर्लभ\-दुरासद\-दुस्सह\-दुरन्त\-दुर्गम\-दुष्प्रेक्ष्य\-दुष्प्राप्य\-सामग्री\-सम्भारा निखिल\-योगीन्द्र\-मुनीन्द्र\-यतीन्द्र\-सुरेन्द्रासुरेन्द्र\-नरेन्द्र\-साधक\-वृन्दवन्दित\-चरण\-कमल\-कल्हाराः परमोदारा बादरायणापर\-नामधेया महर्षि\-वेद\-व्यास\-वर्याः। इदं ब्रह्मसूत्रमध्याय\-चतुष्टयात्मकम्। एतदवलम्ब्यैव विभिन्न\-वेदान्त\-मत\-प्रवर्तकाः स्व\-स्व\-सम्प्रदायानुसारं भाष्याणि बभाषिरे। अत्र ब्रह्मैव निरूपितम्। \textcolor{red}{अथातो ब्रह्म\-जिज्ञासा} (ब्र॰सू॰~१.१.१) इति। इदं वेदस्य शिरोभागव्याख्यानभूतम्। अत्र ब्रह्मणो निर्गुण\-सगुण\-पक्षौ निरूपितौ। \end{sloppypar}
\begin{sloppypar}\hyphenrules{nohyphenation}\justifying\noindent\hspace{10mm} एवमेव वेदखण्डन\-पराणि त्रीणि नास्तिक\-दर्शनानि। बोद्धं जैनं चार्वाकमिति। प्रकारान्तरेण सर्वस्यापि वाङ्मयस्य वेद एवाऽश्रयः। सर्वत्र च भगवान् स्तुत्या क्वचिन्निन्दया च प्रस्तुतः। यथा बुद्धो महावीर\-स्वामी च वेदं शिष्टतया निन्दतः किं च चार्वाकोऽशिष्टतया निन्दति।\end{sloppypar}
\begin{sloppypar}\hyphenrules{nohyphenation}\justifying\noindent\hspace{10mm} वेदमाश्रित्य त्रय आगमाः। आगमो नाम प्राचीन\-विचार\-सङ्ग्रहः। स च त्रिविधः। शैवो वैष्णवः शाक्तश्च। तत्राऽगम\-विषयक एकः पारम्परिकः श्लोकः श्रुतो गुरुभ्यः~–\end{sloppypar}
\centering\textcolor{red}{आगतं शिववक्त्रेभ्यो गतं च गिरिजाश्रुतौ।\nopagebreak\\
मतञ्च वासुदेवस्य तस्मादागम उच्यते॥}\nopagebreak\\
\raggedleft{–~इति साम्प्रदायिकाः}\\
\begin{sloppypar}\hyphenrules{nohyphenation}\justifying इदमेव तन्त्र\-शास्त्रमिति कथ्यते। अत्र द्वौ मार्गौ दक्षिणो वामश्च। दक्षिण\-मार्ग आत्म\-शुद्धि\-पूर्वक\-सिद्धिः स्वीकृता। वाममार्गे चाऽडम्बर\-पूर्णा चमत्कार\-मयी सिद्धिः साध्यते। मार्गोऽयं पञ्च\-मकार\-सेवन\-परतया हेयप्रायः। \end{sloppypar}
\begin{sloppypar}\hyphenrules{nohyphenation}\justifying\noindent\hspace{10mm} इममेव वेदमधिगन्तुं षडङ्गानि शिक्षा कल्पो निरुक्तं ज्यौतिषं छन्दो व्याकरणं च। षडङ्ग\-वेदाध्ययनं पुराऽस्माकं विधेयमासीत्। वाल्मीकिरपि सुन्दरकाण्डे षडङ्गानां प्रशंसां सोत्साहमकार्षीत्। यथा~–\end{sloppypar}
\centering\textcolor{red}{षडङ्गवेदविदुषां क्रतुप्रवरयाजिनाम्।\nopagebreak\\
शुश्राव ब्रह्मघोषान् स विरात्रे ब्रह्मरक्षसाम्॥}\nopagebreak\\
\raggedleft{–~वा॰रा॰~५.१८.२}\\
\begin{sloppypar}\hyphenrules{nohyphenation}\justifying\noindent\hspace{10mm} शिक्षायां वेद\-स्वर\-पाठ\-प्रक्रिया। बहव आचार्याः पृथक्पृथक्शिक्षा\-ग्रन्थान् विलिलिखुः। पाणिनीय\-शिक्षाऽपि महत्त्वपूर्णा। कल्पेषु वैदिक\-मन्त्र\-प्रयोगाणां नियमाः। निरुक्ते वेदस्य गहन\-शब्दानां वैदुष्य\-पूर्णा व्युत्पत्तिः। एतदाचार्यो भगवान् यास्कः। छन्दः\-शास्त्रं वेद\-च्छन्दसां व्यवस्था निर्विशति।
गायत्र्युष्णिगनुष्टुब्बृहती पङ्क्तिस्त्रिष्टुब्जगतीति सप्त च्छन्दांसि। शेषेषु यद्यपि वैदिकी रचना मिलति तथाऽपि प्रधानतया लोके तेषामेव प्रयोगः। इदमेव पिङ्गल\-शास्त्रं कथ्यते। एतत्प्रवर्तकः पिङ्गल\-नामधेयः सर्पः। श्रूयते महात्ममुखेभ्यो यत्कदाचिद्भगवाञ्छ्रीलक्ष्मी\-रमण\-रमणीय\-चरण\-कमल\-समलङ्कृत\-पृष्ठभाग\-भोगि\-वरूथ\-शत्रु\-महा\-पराक्रम\-समूह\-त्रिविक्रम\-हिरण्मय\-पक्ष\-व्याघात\-विमदीकृत\-दैत्य\-दानवाराति\-पक्ष\-लक्षो वैनतेयो भगवान् गरुडो बुभुक्षा\-परः सर्पमाजिर्हीषुरभ्यधावत्। स च दुद्राव। अनिवर्तमान\-हरियानमालोक्य प्रार्थयाञ्चक्रे सर्पो यद्देव मा जहि विद्यामेकां गोपित\-पूर्वां मत्तः प्राप्नुहीति निगद्य भगवान् सर्पभूतः पिङ्गलः पलायमानो गरुडं विद्यामुवाच। तदेव पिङ्गल\-शास्त्रं कथ्यते। अन्तिमं छन्दो भुजङ्गप्रयातं शिक्षयन् शीघ्रमेव दुरापं स्वकीयं विलमाविवेश। तदेव पिङ्गल\-शास्त्रम्। \end{sloppypar}
\begin{sloppypar}\hyphenrules{nohyphenation}\justifying\noindent\hspace{10mm} व्याकरणमेव वेदस्य मुख्याङ्गम्। यद्यपि षडङ्गवेदोऽध्येय इति स्मृति\-श्रुतिर्यथा पतञ्जलिः प्रमाणयन् प्रोवाच \textcolor{red}{ब्राह्मणेन निष्कारणो धर्मः षडङ्गो वेदोऽध्येयो ज्ञेयः} (भा॰प॰)। \textcolor{red}{निष्कारणः} इत्यत्र निर्गतं कारणमर्थ\-धर्म\-काम\-मोक्ष\-प्रभृति\-प्रवृत्ति\-प्रयोजकं यस्मात्तथा\-भूतः। इत्थं सर्वतो\-भावेन वेद\-ज्ञानाय षडङ्गानां सत्यामुपयोगितायामपि सर्व\-प्राथम्येन व्याकरणाध्ययनं नितान्तोपयोगि। विद्वद्भिर्व्याकरणं भगवतो वेदस्य मुखमित्यभ्य\-धायि। \textcolor{red}{मुखं व्याकरणं प्रोक्तम्} इति प्राचीनोक्तेः।
\textcolor{red}{प्रथमं छन्दसामङ्गं प्राहुर्व्याकरणं बुधाः} (वा॰प॰~१.११) इति भर्तृहरि\-सूक्तेश्च। व्याकरणमन्तरा व्यवस्थित\-भाषा\-ज्ञानमसम्भवम्। ऋते भाषा\-ज्ञानं वेदार्थ\-ज्ञानमपि सर्वतो\-भावेनासम्भवम्। व्याकरणं भाषाया अलङ्कारः। यथा प्राचीनैरुक्तम्~–\end{sloppypar}
\centering\textcolor{red}{यद्यपि बहु नाधीषे तथाऽपि पठ पुत्र व्याकरणम्। \nopagebreak\\
स्वजनः श्वजनो माऽभूत्सकलः शकलः सकृच्छकृत्॥}\nopagebreak\\
\raggedleft{–~प्राचीनोक्तिः}\\
\begin{sloppypar}\hyphenrules{nohyphenation}\justifying\noindent अर्थाद्बहु\-शास्त्राध्ययनेऽलभ्य\-रुचेऽपि पुत्र व्याकरणमधिगच्छ। व्याकरणं विना सन्धि\-समास\-ज्ञान\-कोष\-ज्ञान\-शून्यतया सकार\-शकारयोरुच्चारण\-भ्रमेण त्वं \textcolor{red}{स्व\-जनम्} स्व\-कुटुम्बं \textcolor{red}{श्वजनम्} इति शुनः कुक्कुरस्य जनं
मा मंस्थाः। दन्त्य\-सकार\-घटितः \textcolor{red}{स्व}\-शब्द आत्मीय\-वाची तालव्य\-शकार\-घटितः \textcolor{red}{श्व}\-शब्दः कुक्कुरवाची। व्याकरण\-ज्ञानं विनाऽयं विवेकः कथं सम्भवेत्। इत्थमेव दन्त्य\-सकार\-घटित\-\textcolor{red}{सकल}\-शब्दस्य समुदायोऽर्थस्तालव्य\-शकार\-घटित\-\textcolor{red}{शकल}\-शब्दः खण्ड\-वाचक इति कथं निर्णीयेत। तथैकवार\-वाचि\-दन्त्य\-सकार\-घटित\-\textcolor{red}{सकृत्‌}\-शब्दस्य\footnote{\textcolor{red}{एकस्य सकृच्च} (पा॰सू॰~५.४.१९)।} पुरीष\-वाचि\-तालव्य\-शकार\-घटित\-\textcolor{red}{शकृत्‌}\-शब्दात्\footnote{\textcolor{red}{शकेरृतिन्} (प॰उ॰~४.५८)। \textcolor{red}{उच्चारावस्करौ शमलं शकृत्} (अ॰को॰~२.६.६७)।} का भिन्नतेति ज्ञानं कथं स्यात्। व्याकरणं विना विधवा\-ललाट\-चर्चित\-सिन्दूरमिव मुखारविन्द\-निहित\-शास्त्रमपि नैव शोभामाटीकते। तथा वृद्धाः प्राहुः~–\end{sloppypar}
\centering\textcolor{red}{अङ्गीकृतं कोटिमितं च शास्त्रं नाङ्गीकृतं व्याकरणं च येन। \nopagebreak\\
न शोभते तस्य मुखारविन्दे सिन्दूरबिन्दुर्विधवाललाटे॥}\nopagebreak\\
\raggedleft{–~वृद्धोक्तिः}\\
\begin{sloppypar}\hyphenrules{nohyphenation}\justifying\noindent किं बहुना सत्यपि पुण्य\-जनकतावच्छेदकतावती निखिल\-निगमागम\-शास्त्र\-दर्शन\-पुराणेतिहास\-नाटक\-चम्पू\-गद्य\-पद्य\-प्रहेलिकादि\-चित्र\-बन्ध\-मणि\-बन्ध\-बहु\-विध\-स्वच्छन्द\-च्छन्दः\-प्रवाह\-कमन\-काव्य\-ज्योतिष्मती निखिल\-संस्कार\-शोधन\-शीला भगवती देव\-भारती व्याकरण\-ज्ञानमन्तरेणापुण्य\-कला कल्प\-लतिकेव भाति। यतो हि प्रयोग\-विधि\-ज्ञान\-पूर्वक\-शब्द एव ददाति समीप्सितं फलम्। यथा श्रुतिः \textcolor{red}{एकः शब्दः सम्यग्ज्ञातः शास्त्रान्वितः सुप्रयुक्तः स्वर्गे लोके कामधुग्भवति} (भा॰पा॰सू॰~६.१.८४)। तस्माज्ज्ञान\-पूर्वक\-प्रयोग एव पुण्य\-जनकतावच्छेदकः। तस्मादितर\-भाषेव व्याकरण\-ज्ञान\-शून्य\-जनोच्चरित\-संस्कृत\-भाषाऽपि वन्ध्या गौरिव निष्प्रयोजना। तथा च~–\end{sloppypar}
\centering\textcolor{red}{यथा शशाङ्केन विना विभावरी जलं विना भाति न निम्नगा यथा।\nopagebreak\\
सुबोधभाषा सरला रसान्विता न शोभते व्याकरणं विना तथा॥}\nopagebreak\\
\raggedleft{–~इति मम}\\
\begin{sloppypar}\hyphenrules{nohyphenation}\justifying\noindent संस्कृता वाण्येव पुरुषमलङ्करोति। उक्तं नीतिशतके~–\end{sloppypar}
\centering\textcolor{red}{केयूराणि न भूषयन्ति पुरुषं हारा न चन्द्रोज्ज्वला\nopagebreak\\
न स्नानं न विलेपनं न कुसुमं नालङ्कृता मूर्धजाः।\nopagebreak\\
वाण्येका समलङ्करोति पुरुषं या संस्कृता धार्यते\nopagebreak\\
क्षीयन्ते खलु भूषणानि सततं वाग्भूषणं भूषणम्॥}\nopagebreak\\
\raggedleft{–~भ॰नी॰~१९}\\
\begin{sloppypar}\hyphenrules{nohyphenation}\justifying\noindent इत्थं निरस्त\-दूषणं भूषण\-भूषणं वाग्भूषणं भूषणत्वसम्पत्तये संस्कारमपेक्षते। स च संस्कारो व्याकरणमन्तरेणाऽकाश\-पुष्पमिव काल्पनिकः। तस्मात्पुण्य\-जनकता\-पुरःसर\-यथेष्ट\-भाषा\-ज्ञान\-सम्पत्त्यै व्याकरणं रत्नदीप\-मालिकेव। ममायं प्रत्ययः साधु\-शब्दानां प्रयोगेणैव धर्मः। \end{sloppypar}
\begin{sloppypar}\hyphenrules{nohyphenation}\justifying\noindent\hspace{10mm} व्याकरणाध्ययने मुख्यतया पञ्च प्रयोजनानि प्रतिपादितान्यशेष\-विद्या\-विशेषेण भगवता शेषेण प्राञ्जलिना पतञ्जलिना। यथा तद्वार्तिकं \textcolor{red}{रक्षोहागमलघ्वसन्देहाः प्रयोजनम्} (भा॰प॰)। इमानि पञ्च प्रयोजनानि। अत्रोद्देश्य\-दलानुरोधेन विधेय\-दले कथं न बहु\-वचनतेति चेदेक\-शेष\-महिम्नेत्यवधेयम्। तत्र रक्षा\-शब्दः स्त्रीलिङ्ग ऊहागमासन्देहा उभये लघु नपुंसकलिङ्गे। एवं च रक्षा प्रयोजन्यूहः प्रयोजन आगमः प्रयोजनो लघु प्रयोजनमसन्देहः प्रयोजन इति प्रयोजनी च प्रयोजनश्च प्रयोजनश्च प्रयोजनं च प्रयोजनश्चेति विग्रहे \textcolor{red}{नपुंसकमनपुंसकेनैकवच्चास्यान्यतरस्याम्} (पा॰सू॰~१.२.६९) इति सूत्रेणानपुंसकानां चतुर्णामपि लोपे सति वैकल्पिकमेकवचनम्। यद्वा \textcolor{red}{वेदाः प्रमाणम्} इतिवद्विशेष्य\-वाचक\-पदोत्तर\-सु\-विभक्ति\-तात्पर्य\-सङ्ख्याया बहुत्वाख्यायाः प्रयोजन\-पदोत्तर\-सु\-विभक्ति\-तात्पर्य\-ग्राहकैकत्व\-सङ्ख्यया सत्यपि विरोधे तस्या एकस्यैव प्रयोजनत्वस्य सकलेष्वन्वयाद्विवक्षितत्वमेव।\end{sloppypar}
\begin{sloppypar}\hyphenrules{nohyphenation}\justifying\noindent\hspace{10mm} वेदानां रक्षार्थं व्याकरणमध्येयम्। व्याकरण\-ज्ञानं विनाऽर्थाभावे कथं वेदेषु श्रद्धा कथं वाऽवगमनम्। सर्वेऽपि मन्त्रा नैव विभक्तिषु निर्दिष्टाः। व्याकरण\-ज्ञानं विना यथायथं परिणमयितुं कथं शक्ष्यत्यवैयाकरणः। यथा \textcolor{red}{अग्नये स्वाहा} इति निर्दिष्टमिन्द्र\-प्रसङ्गे चतुर्थी\-ज्ञानं विनाऽवैयाकरण \textcolor{red}{इन्द्राय} कथं कथयिष्यति।\end{sloppypar}
\begin{sloppypar}\hyphenrules{nohyphenation}\justifying\noindent\hspace{10mm} जीवनावधिरल्पः शास्त्रञ्च विशालम्। श्रूयते यत् \textcolor{red}{बृहस्पतिरिन्द्राय दिव्यं वर्ष\-सहस्रं प्रतिपदोक्तानां शब्दानां शब्द\-पारायणं प्रोवाच नान्तं जगाम। बृहस्पतिश्च प्रवक्ता इन्द्रश्चाध्येता दिव्यं वर्षसहस्रमध्ययन\-कालो न चान्तं जगाम। किं पुनरद्यत्वे। यः सर्वथा चिरं जीवति सोऽपि शतं जीवति। चतुर्भिश्च प्रकारैर्विद्योपयुक्ता भवति। आगम\-कालेन स्वाध्याय\-कालेन प्रवचन\-कालेन व्यवहार\-कालेनेति} (भा॰प॰)। एकस्मिन्नेव काले यदि समस्तस्याऽयुषः क्षयस्तर्ह्यन्य\-प्रकाराणां का गतिः। एवमेव \textcolor{red}{बृहस्पतिश्च प्रवक्ता इन्द्रश्चाध्येता दिव्यं वर्ष\-सहस्रमध्ययन\-कालो न चान्तं जगाम} (भा॰प॰) तर्ह्यस्माकं का कथा। अतोऽल्पेन कालेन विपुलस्य शब्द\-सागरस्य यथा पारं व्रजेम लघुनोपायेनेत्यध्येयं व्याकरणम्। व्याकरणे च वृत्तयो व्यक्तं सामासिक\-सिद्धान्त\-पर्यवसायिन्य इति विपश्चितां मनीषितम्।\end{sloppypar}
\begin{sloppypar}\hyphenrules{nohyphenation}\justifying\noindent\hspace{10mm} समस्त\-वाङ्मयस्य यथार्थावगमनं कथं स्यादित्यध्येयं व्याकरणम्। \textcolor{red}{असन्देहार्थं चाध्येयं व्याकरणम्} (भा॰प॰)। यतो हि वेदे स्वर\-विचारः प्रधानः। श्रुतिरस्ति~–\end{sloppypar}
\centering\textcolor{red}{दुष्टः शब्दः स्वरतो वर्णतो वा मिथ्याप्रयुक्तो न तमर्थमाह। \nopagebreak\\
स वाग्वज्रो यजमानं हिनस्ति यथेन्द्रशत्रुः स्वरतोऽपराधात्॥}\nopagebreak\\
\raggedleft{–~भा॰प॰}\\
\begin{sloppypar}\hyphenrules{nohyphenation}\justifying\noindent\hspace{10mm} अर्थात्स्वरतो वर्णतश्चाशुद्धो मन्त्र इष्टमर्थं न प्रयच्छत्यपि तु वज्रमिव यजमानं हन्ति। वेद आख्यानमिदं प्रसिद्धम्। कूट्यात्स्वार्थ\-साधन\-निपुणेन पुरन्दरेण हते पुरोधसि विश्वरूपे स्वसुत\-निधन\-दीप्त\-क्रोधानलस्त्वष्टेन्द्र\-नाशन\-परं यज्ञं समीजे। तत्र \textcolor{red}{इन्द्रशत्रो विवर्धस्व}\footnote{\textcolor{red}{हतपुत्रस्ततस्त्वष्टा जुहावेन्द्राय शत्रवे। इन्द्रशत्रो विवर्धस्व मा चिरं जहि विद्विषम्॥} (भा॰पु॰~६.९.११)। \textcolor{red}{इन्द्रस्याभिचारो वृत्रेणारब्धस्तत्र ‘इन्द्रशत्रुर्वर्धस्व’ इति मन्त्र ऊहितः} (भा॰प्र॰) इति पस्पशायां कैयटः।} अस्मिन्मन्त्र\-खण्डे स्वर\-व्यतिक्रमादिन्द्रस्य विजयो वृत्रस्य च पराजयो जातः। व्याकरणं विना कथं समास\-सन्देह\-निवारणं स्यात्। वैयाकरणस्तु प्रकृति\-स्वरं दृष्ट्वा बहुव्रीहिं समासान्तोदात्तं दृष्ट्वा तत्पुरुषं सारल्येनावगन्तुं पारयिष्यति। अतोऽसन्देहार्थं व्याकरणं पठितव्यम्। \end{sloppypar}
\begin{sloppypar}\hyphenrules{nohyphenation}\justifying\noindent\hspace{10mm} इत्थं समस्त\-शास्त्रोपकारक\-तया पण्डित\-मचर्चिका\-चर्चिते समभ्यर्चिते च विद्वत्तल्लजैर्विबुध\-भारती\-संस्करणे शब्द\-ब्रह्म\-विहरणे सकल\-विद्यालङ्करणे व्याकरणेऽस्माभिः श्रद्धेयता निरापदं धारणीया विचारणीया च तस्य शब्द\-सागर\-समुन्मथन\-मनोरम\-मञ्जुतर\-म्रदिम\-गाम्भीर्य\-सरणिः। इदमेव सकल\-विद्यानां सोपान\-भूतं परम\-पूतं पाटवमयं निरामयं रसमयं ध्वस्त\-संसारामयं विगत\-भव\-भयं दत्ताभयं परम\-प्रकाशरूपं शब्द\-ब्रह्म\-प्रापकं मोक्ष\-द्वारं सकल\-सौष्ठवमयम्। एवं राद्धान्तिते व्याकरण\-महिमनि सकल\-विद्वन्मनोरम\-तया प्रति\-पादिते परम\-रमणीय\-शास्त्र\-विलोचनेऽस्मिञ्छब्द\-शास्त्रे समभवन् बहवः प्रवर्तकाः। \end{sloppypar}
\begin{sloppypar}\hyphenrules{nohyphenation}\justifying\noindent\hspace{10mm} तत्र पुरा नव व्याकरणानि प्रसिद्धान्यासन्। पुरा यथा वाल्मीकीय\-रामायणस्योत्तरकाण्डे भगवन्तमेव सम्बोधयन्नगस्त्यो हनुमद्विषये प्राह यत्~–\end{sloppypar}
\centering\textcolor{red}{सोऽयं नवव्याकरणार्थवेत्ता ब्रह्मा भविष्यत्यपि ते प्रसादात्॥}\nopagebreak\\
\raggedleft{–~वा॰रा॰~७.३६.४७}\\
\begin{sloppypar}\hyphenrules{nohyphenation}\justifying\noindent अथ काल\-क्रमेण लुप्त\-प्रायेषु नवसु व्याकरणेष्वनवस्था जाता। अलोक\-लोक\-लोचन\-निभ\-शास्त्र\-विषयेषु जटिलतापन्नेषु च शास्त्रीय\-विचारेषु शिथिली\-भूतेषु च धर्म\-सिद्धान्त\-प्रचारेषु सकल\-कला\-कलाप\-कलनो निखिल\-जगदुद्धरण\-शील\-ताण्डव\-प्रथित\-लीलो लीलावती\-हैमवती\-ललित\-लोचन\-विलास\-लालित\-मकर\-केतन\-वदन\-तामरस\-रस\-लुब्ध\-शैल\-सुता\-मनोमिलिन्दो मुररिपु\-पाद\-पयोज\-नख\-मणि\-चन्द्र\-कान्त\-द्रवीभूत\-परम\-पूत\-निखिल\-निगम\-सार\-सर्वस्व\-वस्तु\-भिक्षण\-समादृत\-योगीन्द्र\-मुनीन्द्र\-वृन्द\-वेद\-पुराण\-पुरस्कृत\-भगीरथ\-जटिल\-साधन\-धृत\-शरीर\-फल\-भुवन\-पावन\-नीर\-नीरज\-नयन\-नयन\-वल्लभा\-नीराजित\-जित\-मुनि\-मनो\-मलय\-समीरण\-समीरित\-भक्त\-भय\-सङ्ग\-भङ्ग\-निरङ्ग\-रिपु\-जटा\-जूट\-जटिल\-जाम्बूनद\-प्रख्य\-तरलतम\-तरङ्ग\-भग्नान्तरङ्गामङ्गल\-स्फटिक\-निर्मल\-मृदित\-कश्मल\-शमित\-संसारानल\-क्षपित\-सगर\-सूनु\-शाप\-ताप\-पाप\-प्रचण्ड\-दावानल\-दमित\-षड्विकार\-गरल\-धारा\-पङ्क्ति\-विजित\-सरल\-परम\-विमल\-परम\-तरल\-मौक्तिक\-महोज्ज्वल\-सुधा\-सम्मित\-जल\-जीर्ण\-जगज्जरा\-जन्म\-जाल\-मालती\-मालोपम\-दिव्य\-नव्य\-भव्य\-धृत\-भूरिमान\-भागीरथी\-भावित\-सुविशाल\-भालदेशो वलयीकृत\-शेषः शशाङ्क\-मौलिर्व्याकरण\-मन्तरा जगतः कल्याणमसम्भवमिति कृत्वा परम\-सन्तं प्रतिभया विलसन्तं पाणिनिमेव प्रादुर्भावयामास।\end{sloppypar}
\begin{sloppypar}\hyphenrules{nohyphenation}\justifying\noindent\hspace{10mm} स च शिव\-लब्ध\-वेद\-प्रसादो विगत\-विषादो दर्भ\-पवित्र\-पाणिरुदङ्मुखः सकल\-व्याकरण\-समन्वय\-भूतं लघु\-कायं सकल\-शास्त्राध्ययन\-सहायं निहिताष्टाध्यायं विशाल\-विशद\-सरलीकृत\-सङ्क्षिप्त\-शब्द\-शास्त्र\-स्वाध्यायमष्टाध्यायी\-सूत्र\-पाठं सुपठं पपाठ। इदं शिव\-चतुर्दश\-सूत्राणि समाश्रित्य प्रावर्तत। तानि च (१)~अइउण् (२)~ऋऌक् (३)~एओङ् (४)~ऐऔच् (५)~हयवरट् (६)~लण् (७)~ञमङणनम् (८)~झभञ् (९)~घढधष् (१०)~जबगडदश् (११)~खफछठथचटतव् (१२)~कपय् (१३)~शषसर् (१४)~हल्। \textcolor{red}{इति माहेश्वराणि सूत्राण्यणादिसञ्ज्ञार्थानि। एषामन्त्या इतः। लण्सूत्रेऽकारश्च। हकारादिष्वकार उच्चारणार्थः} (वै॰सि॰कौ॰ सञ्ज्ञाप्रकरणे)।\end{sloppypar}
\begin{sloppypar}\hyphenrules{nohyphenation}\justifying\noindent\hspace{10mm} तत्र \textcolor{red}{माहेश्वराणि} इत्यत्र \textcolor{red}{महेश्वरादागतानि}।
शङ्कर\-वर\-प्रसादात्पाणिनिना लब्धानीति भावः। पञ्चम्यन्त\-महेश्वर\-शब्दात् \textcolor{red}{तत आगतः} (पा॰सू॰~४.३.७४) इति सूत्रेण \textcolor{red}{अण्‌}\-प्रत्ययः। भत्वान्महेश्वर\-घटकाकार\-लोपो वृद्धिर्विभक्ति\-कार्यञ्च।\footnote{महेश्वर~\arrow \textcolor{red}{तत आगतः} (पा॰सू॰~४.३.७४)~\arrow महेश्वर~अण्~\arrow महेश्वर~अ~\arrow \textcolor{red}{अचो ञ्णिति} (पा॰सू॰~७.२.११५)~\arrow माहेश्वर~अ~\arrow \textcolor{red}{यचि भम्} (पा॰सू॰~१.४.१८)~\arrow भसञ्ज्ञा~\arrow \textcolor{red}{यस्येति च} (पा॰सू॰~६.४.१४८)~\arrow माहेश्वर्~अ~\arrow माहेश्वर~\arrow विभक्ति\-कार्यम्~\arrow माहेश्वर~जस्~\arrow \textcolor{red}{जश्शसोः शिः} (पा॰सू॰~७.१.२०)~\arrow माहेश्वर~शि~\arrow माहेश्वर~इ~\arrow शि सर्वनामस्थानम् (पा॰सू॰~१.१.४२)~\arrow सर्वनामस्थान\-सञ्ज्ञा~\arrow \textcolor{red}{नपुंसकस्य झलचः} (पा॰सू॰~७.१.७२)~\arrow \textcolor{red}{मिदचोऽन्त्यात्परः} (पा॰सू॰~१.१.४७)~\arrow माहेश्वर~नुँम्~इ~\arrow माहेश्वर~न्~इ~\arrow \textcolor{red}{सर्वनामस्थाने चासम्बुद्धौ} (पा॰सू॰~६.४.८)~\arrow माहेश्वरा~न्~इ~\arrow \textcolor{red}{अट्कुप्वाङ्नुम्व्यवायेऽपि} (पा॰सू॰~८.४.२)~\arrow माहेश्वरा~ण्~इ~\arrow माहेश्वराणि।} इत्थं श्रूयते भगवान् पाणिनिः शालातुरीयः कदाचिद्गुरुकुलेऽधीयानः प्रकृति\-मन्द\-बुद्धिरासीत्। \textcolor{red}{होनहार बिरवान के होत चीकने पात} इति ग्रामीण\-सूक्त्यनुसारं भविष्णु\-जनानां जीवनमपि प्रायशो विषमतामयं नूनमेव। महा\-पुरुषो भीषण\-परिस्थिति\-प्रचण्ड\-झञ्झा\-वातेन सहैवावतरति। आघातं विना व्यक्तित्व\-परिष्कारो न भवति। को जानीयाद्गुरुकुलस्थः पाणिनिर्विद्यार्थि\-जीवने मूर्ख\-चक्र\-चूडामणिः पश्चान्निखिल\-विद्वज्जन\-समर्चित\-चरण\-कमलो भवितेति। तदा कस्य हृदीत्थमनुमानं भूतं स्यादधुना यः परीक्षार्थि\-विद्यार्थिनां परिहास\-भाजनतामुपेतः स एव पाणिनिः कदाचिद्वाचस्पतेरपि सम्मान\-पात्रतां पात्रयिष्यतीति। अतो नीतिश्लोकः पठ्यते~–\end{sloppypar}
\centering\textcolor{red}{नृपस्य चित्तं कृपणस्य वित्तं मनोरथं दुर्जनमानवानाम्।	\nopagebreak\\
स्त्रीणां चरित्रं पुरुषस्य भाग्यं देवो न जानाति कुतो मनुष्यः॥}\nopagebreak\\
\raggedleft{–~नीतिश्लोकः}\\
\begin{sloppypar}\hyphenrules{nohyphenation}\justifying\noindent सम्भवतोऽनयैव धारणयैतस्य नाम पाणिनिरिति। \textcolor{red}{पाणिभ्यां गृहीत्वा नीयते गुरु\-सन्निधिं यः स पाणिनिः}। अर्थात्स्वयमध्ययन\-दुर्बलतया गुरु\-सन्निधिं न गच्छति स्म तदा गुरु\-प्रेषितैर्विद्यार्थिभिः पाणिभ्यां गृहीत्वा केशेषु सावज्ञोऽयं नीयते स्म। साम्प्रतं तु \textcolor{red}{पाणिं गृहीत्वाऽन्यानप्यूर्ध्वं नयति स्व\-रचित\-व्याकरण\-ज्ञान\-द्वारा मोक्षं प्रापयति यः स पाणिनिः}। पूर्व\-व्युत्पत्तौ तृतीयान्त\-पाणि\-शब्दोपपद\-पूर्वकात् \textcolor{red}{नी}\-धातोः (\textcolor{red}{णीञ् प्रापणे} धा॰पा॰~९०१) कर्मणि क्विप्।
ततः सर्वापहारि\-लोपे विभक्ति\-कार्ये पाणिनिः।\footnote{\textcolor{red}{पृषोदरादीनि यथोपदिष्टम्} (पा॰सू॰~६.३.१०९) इत्यनेन पृषोदरादित्वाद्ध्रस्व इति शेषः।} द्वितीयस्मिन् कल्पे द्वितीयान्त\-पाणि\-शब्दोपपदात् \textcolor{red}{नी}\-धातोः कर्तरि क्विप्।
सर्वापहारि\-लोपे सुप्कार्ये पाणिनिः।\footnote{\textcolor{red}{पृषोदरादीनि यथोपदिष्टम्} (पा॰सू॰~६.३.१०९) इत्यनेन पृषोदरादित्वाद्ध्रस्व इति शेषः।} स एव कदाचिच्छास्त्रार्थे सतीर्थ्यैः पराजितो ग्लानि\-झञ्झावात\-विलुलित\-कोमल\-हृदयतः स्वीकृत\-मुनि\-व्रतः सुर\-धुनी\-तरल\-तरङ्ग\-सङ्क्षालित\-शृङ्गमालं गौरी\-तपश्चीर\-संश्रय\-पवित्रीकृत\-विटप\-जालं शशाङ्क\-मौलि\-मञ्जुल\-विलास\-समुल्लास\-भग्न\-भक्त\-भव\-भय\-ज्वालं सकल\-शैल\-शिरोमणिं हिमाचलं समाश्रित्य निखिल\-विद्या\-निकेतं गिरिजा\-समेतं रोष\-रुक्ष\-निटिलाक्ष\-समुद्भूत\-भीषण\-पावक\-स्फुलिङ्ग\-भस्मीकृत\-मीन\-केतं शिशु\-शशि\-मधुर\-मयूख\-सुधा\-यूष\-सम्पोषित\-पद\-पाथोज\-प्रपन्न\-वर्गं स्व\-सङ्कल्प\-सृष्ट\-सकल\-सर्गमुमावन्तं भगवन्तं परिमहित\-महसा प्रबलतर\-तपसा सन्तोषयामास शिवमाशु\-तोषम्। स वै चन्द्रावतंसः श्रुति\-विहित\-प्रशंसो ढक्का\-निनाद\-च्छलेन चतुर्दश सूत्र\-रत्नानि समुन्मथ्य स्वकीय\-व्याकरण\-महा\-सागरात्समर्प्य पाणिनये सकल\-विश्वोपकारकं स्वकरुणा\-विग्रहं परिष्कर्तुं व्याकरण\-शास्त्रं प्रणेतुं चाष्टाध्यायीं सम्प्रेरयाम्बभूव। अतः श्लोको गीयते~–\end{sloppypar}
\centering\textcolor{red}{नृत्तावसाने नटराजराजो ननाद ढक्कां नव पञ्च वारम्।\nopagebreak\\
उद्धर्तुकामः सनकादिसिद्धानेतद्विमर्शे शिवसूत्रजालम्॥}\nopagebreak\\
\raggedleft{–~न॰का॰~१}\\
\begin{sloppypar}\hyphenrules{nohyphenation}\justifying\noindent\hspace{10mm} स एव पाणिनिः शिव\-वर\-प्रभाव\-परिष्कृत\-विमल\-मनीषो मनीषीशश्चतुः\-सहस्र\-सूत्रात्मकमभूतपूर्वं ग्रन्थ\-रत्नं लौकिक\-वैदिक\-सकल\-शब्द\-साधुत्व\-परं परायणञ्च विदुषां पारावारमिव प्रणिनाय।\footnote{\textcolor{red}{अथ कालेन बहवः शिष्या वर्षमुपाययुः। एकोऽपि पाणिनिर्नाम जडबुद्धिरुपाययौ॥ शिष्यान्तरोपहासेन सावमानः स पाणिनिः। शुश्रूषाक्लेशतो यातः कदाचित्तुहिनाचलम्॥ आराध्य तपसा तत्र विद्याकामः स शङ्करम्। प्राप व्याकरणं दिव्यं स च विद्यामुखं शुभम्॥} (ह॰च॰चि॰~२७.७२-७४)।} यद्यप्येतस्मात्पूर्वमप्यैन्द्र\-चान्द्रादीनि व्याकरणान्यासन्नेवञ्च काश्यप\-शाकटायन\-स्फोटायनापिशलि\-भारद्वाज\-प्रभृतीनामाचार्याणां नामानि चर्चितानि यथा \textcolor{red}{त्रिप्रभृतिषु शाकटायनस्य} (पा॰सू॰~८.४.५०) \textcolor{red}{अवङ् स्फोटायनस्य} (पा॰सू॰~६.१.१२३) \textcolor{red}{सर्वत्र शाकल्यस्य} (पा॰सू॰~८.४.५१) \textcolor{red}{ऋतो भारद्वाजस्य} (पा॰सू॰~७.२.६३) \textcolor{red}{दीर्घादाचार्याणाम्} (पा॰सू॰~८.४.५२) \textcolor{red}{वा सुप्यापिशलेः} (पा॰सू॰~६.१.९२) \textcolor{red}{हलि सर्वेषाम्} (पा॰सू॰~८.३.२२) इत्यादिषु तथाऽप्येषां विशालत्वादसमन्वयाच्च लौकिक\-वैदिक\-शब्दानां भगवान् पाणिनिः परम\-सरलं परम\-लघु\-व्याकरणं व्याचकार। यथोदाहरणमेकं द्रष्टव्यम्। इकारोकार\-ऋकार\-ऌकाराणां\footnote{अत्र \textcolor{red}{ऋत्यकः} (पा॰सू॰~६.१.१२८) इत्यनेन प्रकृतिभावः।} स्थाने स्वरे परतः क्रमाद्यकार\-वकार\-रकार\-लकार\-व्यवस्थापनार्थमाचार्याश्चत्वारि सूत्राणि पेठुः। तानि यथा~– (१)~\textcolor{red}{इ यं स्वरे} (२)~\textcolor{red}{उ वं स्वरे} (३)~\textcolor{red}{ऋ रं स्वरे} (४)~\textcolor{red}{ऌ लं स्वरे} इति।\footnote{मूलं मृगयम्।} किन्तु भगवान् पाणिनिश्चतुर्णां सूत्राणां स्थाने प्रत्याहार\-सरण्या लाघवार्थमेकमेव सूत्रमसूत्रयत्कार्यमपि सम्पूर्णं चकार। यथा \textcolor{red}{इको यणचि} (पा॰सू॰~६.१.७७)। \textcolor{red}{इक्‌}\-शब्देन चतुर्णां स्थानिनां चर्चा \textcolor{red}{यण्‌}\-शब्देन च चतुर्णामादेशानाम्। तत्रापि सवर्णयोरिगचोर्यण्निवृत्तये बाध्य\-बाधको भावः प्रास्तावि। यथा \textcolor{red}{मुनीशः} अत्रेश\-घटक ईकारेऽचि परतः मुनि\-घटकस्येकारस्य स्थाने यण् प्राप्तः स च \textcolor{red}{अकः सवर्णे दीर्घः} (पा॰सू॰~६.१.१०१) इत्यनेन बाधितः। अर्थादसवर्णयोरेवेगचोर्यण्संहिता।\end{sloppypar}
\begin{sloppypar}\hyphenrules{nohyphenation}\justifying\noindent\hspace{10mm} विस्तार\-भीत्याऽनुवृत्तिरपि पाणिनेः पटुतायाः पण्डित\-विस्मापनं प्रमाणम्। यथा \textcolor{red}{अतो भिस ऐस्} (पा॰सू॰~७.१.९) इति सूत्रम्। अदन्तादङ्गात्परस्य भिस ऐसिति सूत्रार्थः। किन्तु \textcolor{red}{तपरस्तत्कालस्य} (पा॰सू॰~१.१.७०) इति सूत्रेण तपरस्तकारात्परस्तकार\-पूर्व\-वर्ती वा सम\-कालस्य बोधकः समानोच्चारण\-सदृशोच्चारण\-कालस्य प्रत्यायक इत्यर्थ इति निर्दिश्यते। अत्र \textcolor{red}{अतो भिस ऐस्} (पा॰सू॰~७.१.९) इति सूत्रेऽप्यकारस्तकार\-पूर्व\-वर्त्यपि स्व\-समान\-ह्रस्वोच्चारणस्येकारस्यापि बोधकः स्यात्। तथा च हरिभिरित्यत्र भिस ऐस्स्यात्। अस्मिन्नसामञ्जस्ये \textcolor{red}{अणुदित्सवर्णस्य चाप्रत्ययः} (पा॰सू॰~१.१.६९) इति पूर्व\-वर्ति\-सूत्रात् \textcolor{red}{सवर्णस्य} इतिपदमनुवृत्तम्। अर्थात्तपरः सवर्णस्य सम\-कालस्य सञ्ज्ञेत्यर्थे जात इकाराकारयोः सावर्ण्याभावाद्बोधकत्वावच्छिन्न\-प्रतियोगिकत्वाभावेन नैव दोषः। अतो लघु\-सिद्धान्त\-कौमुद्यां वरदराजाचार्याः प्राहुः \textcolor{red}{सूत्रेष्वदृष्टं पदं सूत्रान्तरादनुवर्तनीयं सर्वत्र} (ल॰सि॰कौ॰~१)।\end{sloppypar}
\begin{sloppypar}\hyphenrules{nohyphenation}\justifying\noindent\hspace{10mm} एवमेव बहुत्र। एतद्दृश्यते सङ्क्षेप\-प्रक्रिया\-व्यवस्थार्थमेव महा\-मुनिना सपाद\-सप्ताध्याय्यां त्रिपाद्यामप्युत्तरोत्तर\-सम्बन्ध\-व्यवस्था कृता। अतो लक्ष्यानुरोधेन पृथक्सूत्र\-निर्माणमन्तराऽपि \textcolor{red}{पूर्वत्रासिद्धम्} (पा॰सू॰~८.२.१) इति सूत्र\-बलेन \textcolor{red}{सपाद\-सप्ताध्यायीं प्रति त्रिपाद्यसिद्धा त्रिपाद्यामपि पूर्वं प्रति परं शास्त्रमसिद्धम्} (ल॰सि॰कौ॰~३१)। \textcolor{red}{पूर्वत्र} इत्यत्र सप्तम्यन्त\-\textcolor{red}{त्रल्‌}\-प्रत्ययः। तथा \textcolor{red}{पूर्वस्मिन्} इति विग्रहे सप्तम्यन्त\-\textcolor{red}{पूर्व}\-शब्दात् \textcolor{red}{सप्तम्यास्त्रल्} (पा॰सू॰~५.३.१०) इति सूत्रेण \textcolor{red}{त्रल्‌}\-प्रत्यये विभक्ति\-कार्ये \textcolor{red}{तद्धितश्चासर्वविभक्तिः} (पा॰सू॰~१.१.३८) इत्यनेनाव्यय\-सञ्ज्ञायाम् \textcolor{red}{अव्ययादाप्सुपः} (पा॰सू॰~२.४.८२) इत्यनेन विभक्ति\-लोपे \textcolor{red}{पूर्वत्र} इति सिद्धम्। तथा च पूर्वत्र परमसिद्धमिति फलितार्थः।
पूर्वस्मिन् परमसिद्धमित्यर्थे कृतेऽत्र सप्तमी विषयता\-रूपा। विषयता च कर्तव्यता\-रूपा। एवं पूर्वस्मिन् कर्तव्ये शास्त्रे परमसिद्धमिति जातम्। कस्मात्परमिति जिज्ञासायां प्रकरणमवलोक्य निश्चीयते। सूत्रमिदमष्टाध्याय्या अष्टमाध्यायस्य द्वितीयस्य पादस्य प्रथमम्। पूर्व\-पर\-शब्दयोस्तस्मात्पूर्वस्मिन्निति फलितमेतस्माच्च परम्। इत्थं निर्दिष्ट\-सूत्रात्पूर्वा सपाद\-सप्ताध्यायी परा च त्रिपादी। अतः सपादसप्ताध्याय्यां कर्तव्यायां पूर्व\-शब्द\-सङ्केत्यायां परा त्रिपाद्यसिद्धा। अस्मिन्नंशे सूत्रस्यास्य विधित्वं प्रतिभाति। त्रिपाद्यामपि पूर्वं प्रति परशास्त्रमसिद्धम्। अयमर्थोऽधिकारगम्यः। यथा कश्चिन्निर्देशकः प्रतिजन\-समुदायं गच्छन् दण्डेन निर्दिशति यत्त्वं स्व\-पूर्वं दृष्ट्वाऽसिद्ध\-कार्यो भव तथैव सूत्रमिदं त्रिपाद्यां प्रतिसूत्रं गत्वा निर्दिशति त्वं सपादसप्ताध्यायीं प्रति स्वपूर्वत्रिपादीं प्रति चासिद्धं भवेति। त्रिपाद्यंश एतस्य अधिकार\-सूत्रता। इयं मे मनीषा। यद्यप्यन्य आचार्या भट्टोजि\-दीक्षित\-हरि\-दीक्षित\-नागेश\-भट्टपाद\-प्रमुखा इदं सामग्र्येणाधिकारमेव स्वीकुर्वन्ति। यथा प्रयत्न\-निर्देश\-प्रसङ्गे श्री\-भट्टोजि\-दीक्षितश्चतुर एव प्रयत्नानाभ्यन्तरान् स्वीकुर्वन् प्रतिपादयति~– \textcolor{red}{ह्रस्वस्यावर्णस्य प्रयोगे संवृतम्। प्रक्रिया\-दशायां तु विवृतमेव} (वै॰सि॰कौ॰~१०)। तथा च सूत्रं \textcolor{red}{अ अ} (पा॰सू॰~८.४.६८) इत्यनुसारेण \textcolor{red}{विवृतमनूद्य संवृतोऽनेन विधीयते। अस्य चाष्टाध्यायीं सम्पूर्णां प्रत्यसिद्धात्वाच्छास्त्र\-दृष्ट्या विवृतत्वमस्त्येव} (वै॰सि॰कौ॰~११)। \textcolor{red}{तथा हि सूत्रं “पूर्वत्रासिद्धम्” (पा॰सू॰~८.२.१)। अधिकारोऽयम्} (वै॰सि॰कौ॰~१२) इति। अर्थादकारं वर्जयित्वा सर्वेषामपि स्वराणां विवृत आभ्यन्तरः प्रयत्नः। केवलमकारस्य संवृत इति। स च \textcolor{red}{अ अ} (पा॰सू॰~८.४.६८) इति सूत्रेण विधीयते। तत्र प्रथमः अ इति लुप्त\-षष्ठीकः। अस्य अ इति तात्पर्यं सामान्यम्। विवृतस्याकारस्य स्थाने संवृतो भवत्वकार इति भावः। तदा \textcolor{red}{रामागमनम्} इत्यत्र राम\-घटकस्याकारस्य संवृतत्वादागमन\-घटकाऽऽकारस्य विवृतत्वाच्चोभयो\-र्विषमाभ्यन्तर\-प्रयत्नतया सावर्ण्य\-प्रतियोगिकाभावेन \textcolor{red}{अकः सवर्णे दीर्घः} (पा॰सू॰~६.१.१०१) इत्यनेन दीर्घानुपपत्तौ \textcolor{red}{पूर्वत्रासिद्धम्} (पा॰सू॰~८.२.१) इत्यनेनाष्टाध्याय्या अन्तिम\-सूत्रत्वात्सम्पूर्णं प्रत्येतस्यासिद्धौ सिद्धायां दीर्घ\-विधायक\-सूत्र\-कर्तव्यतायामेतस्य चासिद्धौ द्वयोर्विवृत\-प्रयत्नतयाऽकाराकारयोः सावर्ण्ये सति दीर्घे \textcolor{red}{रामागमनम्} इत्यपि प्रयोगः सुसिद्धः। इयं तत्रत्या परिस्थितिः। किन्त्वत्र पूर्व\-शब्दस्यावधि\-सापेक्षत्वेनोक्त\-सूत्रस्याष्टमाध्याय\-द्वितीय\-पाद\-प्रथमत्वात्ततः पूर्वस्मिन् कर्तव्य इतः परमसिद्धमित्यर्थः सुस्पष्ट एव। ततः सपाद\-सप्ताध्यायीं प्रति विधित्वेनैव कार्य\-सिद्धौ तदंशेऽलमधिकार\-कल्पनया। अधिकारत्वं नाम स्व\-देशे वाक्यार्थ\-ज्ञान\-शून्यत्वे सति पर\-देशे वाक्यार्थ\-ज्ञान\-बोधकत्वमुत्तरोत्तर\-सम्बन्धकत्वं वा। त्रिपाद्यामधिकारत्वं भवतूत्तरोत्तर\-सम्बन्धत्वात्सपाद\-सप्ताध्याय्यामन्योऽन्यं प्रत्यन्यो\-ऽन्यमसिद्धमित्यस्यापेक्षैव नास्ति। अतस्तत्रोत्तरोत्तर\-सम्बन्धत्व\-मूलकाधिकारत्व\-सूचकासिद्धि\-विधायकस्य \textcolor{red}{पूर्वत्रासिद्धम्} (पा॰सू॰~८.२.१) इत्यस्य प्रसर\-परिसर एव नास्ति। तत्र हि \textcolor{red}{विप्रतिषेधे परं कार्यम्} (पा॰सू॰~१.४.२) इति सूत्रस्य जागरूकत्वात्। विप्रतिषेधो नामान्यत्रान्यत्र\-लब्धावकाशयोरेकत्र युगपत्प्रवृत्तिः। यथा \textcolor{red}{इको यणचि} (पा॰सू॰~६.१.७७) इति \textcolor{red}{सुध्युपास्यः} इत्यादौ चरितार्थम्। \textcolor{red}{अकः सवर्णे दीर्घः} (पा॰सू॰~६.१.१०१) इति च \textcolor{red}{दैत्यारिः} इत्यादौ चरितार्थम्। उभयोरपि \textcolor{red}{श्री ईशः} इत्यादौ युगपत्प्राप्तिः। परत्वाद्दीर्घः। तस्मात्सपादसप्ताध्याय्यां पूर्वत्रासिद्धमित्यस्य प्रसरणाभावेन तत्राधिकारस्यानुपयोगात्सपाद\-सप्ताध्यायीं प्रति त्रिपाद्यसिद्धत्वस्य शब्द\-महिम्नैव सिद्धतया केवलं त्रिपाद्यामेवैतस्याधिकारता। सोपयोगा प्रतिभाति मे बाल\-मनीषया। स्वं प्रत्यप्येतस्याधिकारता नास्ति। इदमस्त्यस्यासिद्धि\-सूचकम्। स्वयं नासिद्धिमाटीकते। यथा कश्चित्प्राण\-दण्ड\-दाता नैव स्वयं दण्ड्यो भवति। एकस्मिन्नेव दण्ड\-दातृत्व\-दण्ड्यत्व\-धर्मयोरसम्भवमिव। एकस्मिन् सूत्रेऽसिद्धि\-सूचकत्वासिद्धत्व\-धर्मौ न घटेताम्। नह्यत्र चतुरोऽपि नटः स्वेनैव स्वस्य स्कन्धमारोढुं शक्नोति। यद्यप्येतस्याधिकार\-सूत्रत्व\-चर्चा सिद्धान्त\-कौमुदी\-प्रौढ\-मनोरमा\-शब्दरत्न\-शब्देन्दुशेखरेषु साटोपं प्रतिपादिता। इत्यलमतिपल्लवितेन।\end{sloppypar}
\begin{sloppypar}\hyphenrules{nohyphenation}\justifying\noindent\hspace{10mm} विविध\-लक्ष्याणामल्पैरेव सूत्रैर्यादृश्या चातुर्य\-पूर्ण\-तूर्ण\-प्रतिभया व्यवस्था दत्ता भगवता पाणिनिना सा नूनं स्वस्यामलौकिक्यनुपमा मनीषि\-मनोरमा च। अन्येभ्यो व्याकरणेभ्योऽमुष्मिन् पाणिनीय\-व्याकरण एकं वैलक्षण्यं यत्केषुचिदपि वैदिक\-शब्दानामनुशासनं न वर्तते। किन्त्वस्मिन् लौकिक\-वैदिकानामुभयेषामप्यनुशासनं वैदिक\-शब्द\-प्रक्रिया\-सङ्कलनम्। सिद्धान्त\-कौमुद्या वैदिक\-प्रकरणं स्वर\-प्रकरणं च सर्वेषामज्ञानावरणं निस्स्पृणोति। सूत्रेषु त्रिशत\-प्राय\-सूत्राणि लौकिक\-शब्दतो\-विलक्षण\-वैदिक\-शब्द\-साधुत्व\-प्रक्रिया\-प्रकार\-प्रतिपादकानि।\footnote{\textcolor{red}{लौकिक\-शब्दतो\-विलक्षण} इत्यत्र सुप्सुपा\-समासः।} यथा प्रथमाध्यायस्य द्वितीयपादे~–\end{sloppypar}
\centering\textcolor{red}{फल्गुनीप्रोष्ठपदानां च नक्षत्रे।\nopagebreak\\
छन्दसि पुनर्वस्वोरेकवचनम्।\\
विशाखयोश्च।\nopagebreak\\
तिष्यपुनर्वस्वोर्नक्षत्रद्वन्द्वे बहुवचनस्य द्विवचनं नित्यम्।}\nopagebreak\\
\raggedleft{–~पा॰सू॰~१.२.६०–१.२.६३}\\
\begin{sloppypar}\hyphenrules{nohyphenation}\justifying\noindent\hspace{10mm} पातञ्जल\-महाभाष्येऽपि \textcolor{red}{अथ शब्दानुशासनम्। अथेत्ययं शब्दोऽधिकारार्थः प्रयुज्यते। शब्दानुशासनं नाम शास्त्रमधिकृतं वेदितव्यम्। केषां शब्दानाम्। लौकिकानां वैदिकानां च इति} (भा॰प॰)। \textcolor{red}{शासुँ अनुशिष्टौ} (धा॰पा॰~१०७५) इत्यस्माद्धातोः \textcolor{red}{ल्युट् च} (पा॰सू॰~३.३.११५) इति सूत्रेण नपुंसके भावे ल्युट्। अनुबन्ध\-कार्येऽनादेशे\footnote{\textcolor{red}{युवोरनाकौ} (पा॰सू॰~७.१.१) इत्यनेन।} विभक्ति\-कार्ये \textcolor{red}{अनु}\-उपसर्ग\-जुष्टतया \textcolor{red}{अनुशासनम्} इति सिद्धम्। यद्यपि शासन\-शब्द एवाऽनुपूर्वी\-पूर्ण\-शास्त्ररूपोऽर्थो निर्गलति तदाऽनूपसर्गस्योप\-योगिताऽऽनुकूल्य\-प्रतिपादनार्था। एवं च \textcolor{red}{पाणिनिः शब्दाननुशास्ति} इत्यत्र \textcolor{red}{पाणिन्यभिन्नैक\-कर्तृक\-वर्तमान\-कालतावच्छेदकतावच्छिन्न\-शब्द\-कर्मानुकूल्य\-पुरःसरानुपूर्वी\-समन्वित\-शब्दानुशिष्ट्यनुकूलो व्यापार} इति शाब्द\-बोध\-प्रकारः। धन्यो भगवान् पाणिनिर्यः खलु निखिल\-पण्डित\-भयावहं भग्न\-विपश्चिद्धैर्य\-राशिमगाधं सागरमिव साक्षात्पर\-ब्रह्म दुरासदं दुर्धर्षं दुर्ज्ञेयं दुरवगमं सर्वतन्त्र\-स्वतन्त्रमनन्तं शब्द\-कण्ठीरवमपि सर्वमनुशशास। आनुकूल्यञ्च साधुत्वम्। तच्च शिष्ट\-प्रयुक्तत्वे सति पुण्य\-जनकतावच्छेदकत्वम्। लक्षणेनानेन तुलसी\-कृतादौ नाव्याप्तिः। तुलसी\-कृत\-मानसमपि शिष्ट\-प्रयुक्तं पुण्य\-जनकतावच्छेदकमपि। यद्यपि सामान्य\-भाषा\-भणितत्वादिदं न पुण्यजनकं ये स्वीकुर्वन्ति विभ्रान्तास्ते। अत्राद्याप्येतत्पाठेन लौकिक\-पारलौकिक\-सिद्धयो दृश्यन्ते। \end{sloppypar}
\begin{sloppypar}\hyphenrules{nohyphenation}\justifying\noindent\hspace{10mm} समानाधिकरणयोरेवावच्छेद्यावच्छेदक\-भाव\-नियमः। यथा गोत्वावच्छेदकं सास्नादिमत्त्वमेकस्मिन्नेव गवि गोत्व\-सास्नादिमत्त्व\-धर्मौ तिष्ठतः। अतो \textcolor{red}{गोत्वावच्छिन्नो गौः} इति प्रयोगो गोत्व\-युक्त\-परः। किन्त्वयं नियमोऽपि प्रायिक एव। अवच्छेद्यावच्छेदक\-भाव\-व्यवस्थाया विशेषं दर्शनमवच्छेदकत्व\-निरुक्तौ निरूपितम्। शिष्टत्वं नामाप्तत्वम्। तच्च सकल\-दुर्गुण\-शून्यत्वे सति दिव्य\-विज्ञान\-विध्वस्त\-कल्मषत्वे सति त्रिकाल\-दर्शित्वे सति यथार्थ\-वक्तृत्वम्। अत एव न्याय\-प्रवर्तकाः श्रीगौतम\-पादा आप्त\-वाक्यमेव शब्द\-प्रमाणं मन्यन्ते। यथा \textcolor{red}{आप्तोपदेशः शब्दः} (न्या॰सू॰~१.१.७)। \textcolor{red}{आपॢँ व्याप्तौ} (धा॰पा॰~१२६०) इत्यस्माद्धातोः \textcolor{red}{आप्नोति त्रिकालज्ञतया सर्वमपि चराचरं व्याप्नोति} इति विग्रहे कर्तरि क्तः।\footnote{\textcolor{red}{गत्यर्थाकर्मक\-श्लिष\-शीङ्स्थास\-वस\-जन\-रुह\-जीर्यतिभ्यश्च} (पा॰सू॰~३.४.७२) इत्यनेन। कर्मणोऽविवक्षाया अकर्मकत्वम्। \textcolor{red}{धातोरर्थान्तरे वृत्तेर्धात्वर्थेनोपसङ्ग्रहात्। प्रसिद्धेरविवक्षातः कर्मणोऽकर्मिका क्रिया॥} (वा॰प॰~३.७.८८)।} अयमेवाऽप्तशब्दो भाषायामप\-भ्रंशतया \textcolor{red}{आप} इति कथ्यते। आप्तो नाम यथार्थ\-वक्ता। \textcolor{red}{रागादि\-वशादपि नानन्यथा\-वादी यः स इति चरके पतञ्जलिः} (ल॰म॰, प॰ल॰म॰) इति वैयाकरण\-सिद्धान्त\-लघु\-मञ्जूषायां वैयाकरण\-सिद्धान्त\-परम\-लघु\-मञ्जूषायां च नागेश\-भट्टपादाः। अपि च~–\end{sloppypar}
\centering\textcolor{red}{रजस्तमोभ्यां निर्मुक्ता नित्यज्ञानबलेन ये।\nopagebreak\\
येषां त्रिकालममलं ज्ञानमव्याहतं सदा॥\nopagebreak\\
आप्ताः शिष्टा विबुधास्ते तेषां वाक्यमसंशयम्।\nopagebreak\\
सत्यं वक्ष्यन्ति ते कस्मादसत्यं नीरजस्तमाः॥}\nopagebreak\\
\raggedleft{–~च॰सं॰ सू॰स्था॰~११.१८,१९}\\
\begin{sloppypar}\hyphenrules{nohyphenation}\justifying\noindent इति। भ्रम\-प्रमाद\-विप्रलिप्सा\-करणापाटवादि\-दोष\-दूषितान्तः\-करण\-स्वार्थ\-नाशित\-चक्षुषो\-ऽप्रमाणं वक्तुं पारयन्ति। किन्तु ये दिव्य\-विज्ञान\-सम्पन्नाः प्रपन्नाश्च हरेः पदमुपपन्नास्ते कथं ब्रूयुर्विपन्ना इव दूषितम्। तुलसीदासोऽपि मानस आप्त\-वाक्यस्यैव प्रामाण्यं बाभाष्यते~–\end{sloppypar}
\centering\textcolor{red}{ते श्रोता बक्ता सम शीला। समदरशी जानहिं हरि लीला॥\nopagebreak\\
जानहिं तीनि काल निज ग्याना। करतल गत आमलक समाना॥}\footnote{एतद्रूपान्तरम्–\textcolor{red}{तावुभौ श्रोतृवक्तारौ समशीलौ समेक्षणौ। जानीतश्च हरेर्लीलां सकलां सर्वभावतः॥ निजज्ञानप्रभावेण तौ कालत्रितयं पुनः। करामलकवत्साक्षात्सर्वदैवावगच्छतः॥} (मा॰भा॰~१.३०.६,७)।}\nopagebreak\\
\raggedleft{–~रा॰च॰मा॰~१.३०.६,७}\\
\begin{sloppypar}\hyphenrules{nohyphenation}\justifying\noindent ऋषय एवाऽप्ताः। ते खल्वितर\-राज\-कवय इव राज्ञो न किमपि गृह्णन्ति स्म। रामायण इदमाख्यानं प्रसिद्धं यत्~–\end{sloppypar}
\centering\textcolor{red}{निर्माय रामायणमादिकाव्यं श्रीमैथिलीरामरसायनञ्च।\nopagebreak\\
अध्याप्य सैतेयकुशीलवौ तौ वाल्मीकिवर्यः किल निर्दिदेश॥\\
गत्वा रामायणं काव्यं गायतां रामसन्निधौ।\nopagebreak\\
युवाभ्यां कर्हिचिद्राज्ञो ग्रहीतव्यं न किञ्चन॥\\
गृहीते द्रविणे राज्ञो बुद्धिमालिन्यकारणात्।\nopagebreak\\
व्याहन्येताप्तता वत्सौ निर्लोभं गायतामतः॥}\footnote{मूलं रामायणेषु मृग्यम्।}\nopagebreak\\
\begin{sloppypar}\hyphenrules{nohyphenation}\justifying\noindent एतादृश\-वीत\-राग\-माहात्म्य\-समानाधिकरणमेवाप्तत्वम्। एषामेव वाक्यं प्रामाणिकम्। अत एव~–\end{sloppypar}
\centering\textcolor{red}{इक्ष्वाकुवंशप्रभवो रामो नाम जनैः श्रुतः। \nopagebreak\\
नियतात्मा महावीर्यो द्युतिमान् धृतिमान् वशी॥\nopagebreak\\
बुद्धिमान्नीतिमान् वाग्मी श्रीमाञ्छत्रुनिबर्हणः। \nopagebreak\\
विपुलांसो महाबाहुः कम्बुग्रीवो महाहनुः॥}\nopagebreak\\
\raggedleft{–~वा॰रा॰~१.१.८–९}\\
\begin{sloppypar}\hyphenrules{nohyphenation}\justifying\noindent इत्यादि वाल्मीकीयं प्रमाणम्। अन्यथा रामो राजा बभूवेति को मन्येत। वेदव्यासोऽपि निर्लोभस्यैवाऽप्ततां समामनति~–\end{sloppypar}
\centering\textcolor{red}{चीराणि किं पथि न सन्ति दिशन्ति भिक्षां\nopagebreak\\
नैवाङ्घ्रिपाः परभृतः सरितोऽप्यशुष्यन्।\nopagebreak\\
रुद्धा गुहाः किमजितोऽवति नोपसन्नान्\nopagebreak\\
कस्माद्भजन्ति कवयो धनदुर्मदान्धान्॥}\nopagebreak\\
\raggedleft{–~भा॰पु॰~२.२.५}\\
\begin{sloppypar}\hyphenrules{nohyphenation}\justifying\noindent ईदृशाप्तोच्चरितत्वमेव साधुत्वावच्छेदकम्। साधुत्वञ्च जातिः। \textcolor{red}{जातित्वं नामैकत्वे सति नित्यत्वे सत्यनेक\-समवेतत्वम्}। यथा घटत्वम्। तच्चैकं नित्यमनेक\-घट\-समवेतञ्च। मञ्जूषायामेतस्य चर्चा।\footnote{\textcolor{red}{साधुत्वं च व्याकरण\-व्यङ्ग्योऽर्थ\-विशिष्ट\-शब्द\-निष्ठ\-पुण्य\-जनकतावच्छेदको जातिविशेषः} (ल॰म॰)।} साधुष्वसाधुषु च वाचकत्वाविशेषः पुण्य\-पापयोरेव तत्र नियमः।\footnote{\textcolor{red}{वाचकत्वाविशेषे वा नियमः पुण्यपापयोः} (वा॰प॰~३.३.३०)।} साधूनां शब्दानामुच्चारणे पुण्यं जायते। यथा भाष्यकारोऽपि \textcolor{red}{भ॒द्रैषां॑ ल॒क्ष्मीर्निहि॒ताधि॑ वा॒चि} (ऋ॰वे॰सं॰~१०.७१.२, भा॰प॰) इति कथयति। नैयायिकानां नये साधु\-शब्देषु वर्तमानमर्थमसाधु\-शब्देषु स्मृत्वैवार्थं प्रतिपद्यते।\footnote{\textcolor{red}{असाधुरनुमानेन वाचकः कैश्चिदिष्यते} (वा॰प॰~३.३.३०)।} यथा कश्चित् \textcolor{red}{गगरी}\-शब्दमुच्चारयति। तत्र घटे वर्तमानं कम्बु\-ग्रीवादिमानित्यर्थं स्मृत्वा भावं प्रतिपद्यते। किन्तु वैयाकरणानां नय इयं मान्यता नहि। यतो हि वर्ष\-कल्पो बालः शुद्धं घट\-शब्दं न जानन्नपि \textcolor{red}{गगली} इत्युच्चारणेनैव तमर्थं प्रतिपद्यते। अतो वाचकत्वमुभयत्र किन्तु पुण्य\-जनकता साधुष्वेव। साधूनामन्वाख्यानं व्याकरणं ह्यसाधून् साधु\-शब्देभ्यो व्याकरोति। एवं \textcolor{red}{व्याकरणाभिन्नैक\-कर्तृक\-वर्तमान\-कालावच्छिन्नासाधु\-शब्द\-कर्मक\-साधु\-शब्दावधिक\-पृथक्करणानुकूल\-व्यापारः}। \textcolor{red}{व्याक्रियन्ते साधु\-शब्दा येन तद्व्याकरणम्} इति व्युत्पत्तौ \textcolor{red}{वि आङ्} पूर्वक \textcolor{red}{डुकृञ् करणे} (धा॰पा॰~१४७२) इति धातोः \textcolor{red}{करणाधिकरणयोश्च} (पा॰सू॰~३.३.११७) इत्यनेन ल्युटि लटोश्चेतोर्लोपे योरनादेशे\footnote{\textcolor{red}{युवोरनाकौ} (पा॰सू॰~७.१.१) इत्यनेन।} णत्वे\footnote{\textcolor{red}{अट्कुप्वाङ्नुम्व्यवायेऽपि} (पा॰सू॰~८.४.२) इत्यनेन।} यणि\footnote{\textcolor{red}{इको यणचि} (पा॰सू॰~६.१.७७) इत्यनेन।} विभक्ति\-कार्ये च \textcolor{red}{व्याकरणम्} इति सिद्धम्। व्युत्पत्तिर्हि शब्दार्थ\-फल\-प्रयोजन\-ज्ञानम्। तत्र कृधातुना साकं \textcolor{red}{वि आङ्} इत्युपसर्ग\-द्वयस्य समभिव्याहारोऽपूर्वमर्थं व्यनक्ति। तत्र \textcolor{red}{वि} इत्यस्यार्थो विवेचनम्। \textcolor{red}{आ} इत्यस्याऽसमन्तात्। अर्थाद्व्याकरणेन विविच्याऽसमन्तात्क्रियते साधु\-शब्दः प्रकट्यत\footnote{\textcolor{red}{सम्प्रोदश्च कटच्} (पा॰सू॰~५.२.२९) इत्यनेन निष्पन्नात् \textcolor{red}{प्रकट}\-शब्दात् \textcolor{red}{तत्करोति तदाचष्टे} (धा॰पा॰ ग॰सू॰~१८७) इत्यनेन णिचि धातुसञ्ज्ञायां कर्मणि लटि यकि \textcolor{red}{णेरनिटि} (पा॰सू॰~६.४.५१) इति णिलोपे तिपि \textcolor{red}{टित आत्मनेपदानां टेरे} (पा॰सू॰~३.४.७९) इत्यनेनैत्वे \textcolor{red}{प्रकट्यते} इति सिद्धम्। प्रकटीक्रियते इत्यर्थः।} इति तात्पर्यम्। तदेव लक्ष्य\-लक्षणे व्याकरणमिति कथयति। \textcolor{red}{व्याकरणत्वं नाम पाणिनि\-प्रभृति\-त्रिमुन्युच्चरितत्वे सत्यसाधु\-शब्द\-पृथक्कर्तृत्वे सति साधु\-शब्दानुख्यातृत्वम्} इति मे मतम्।\end{sloppypar}
\begin{sloppypar}\hyphenrules{nohyphenation}\justifying\noindent\hspace{10mm} एवं \textcolor{red}{शिष्टोच्चरितत्वे सति पुण्य\-जनकतावच्छेदकत्वरूपं साधुत्वं} सार्वभौमम्। अत्र व्याकरण\-सम्मतत्वे नाग्रहः। तेन वाल्मीकि\-रामायणे पुराणेषु त बहुत्र सत्यपि पाणिनि\-सिद्धान्त\-विरुद्ध\-प्रयोगे नैव साधुत्वोच्छित्तिः। शिष्ट\-प्रयुक्तत्वात्। यथा \textcolor{red}{सीतायाः पतये नमः} (रा॰र॰स्तो॰~२७) अत्रासमस्त\-पति\-शब्दाद्घि\-सञ्ज्ञा\-फल\-रूपो \textcolor{red}{घेर्ङिति} (पा॰सू॰~७.३.१११) इति गुण अपाणिनीय एव। पाणिनिस्तु समास एव यत्र पति\-शब्दं घिसञ्ज्ञं वाञ्छति। यथा तत्सूत्रं \textcolor{red}{पतिः समास एव} (पा॰सू॰~१.४.८)। किन्त्वपाणिनीयत्वेऽपि शिष्ट\-प्रयुक्तत्वादत्र साधुत्वम्। एवमेव \textcolor{red}{साधवो हृदयं मह्यम्} (भा॰पु॰~९.४.६८)। अत्र व्यासो ममेत्यस्य स्थाने मह्यमिति लिखति। किन्तु शिष्ट\-प्रयुक्तत्वेनात्र साधुता। ईदृशेषु स्थलेष्वार्षत्वादित्येव समाधानं समादधति सुधियः। अत्र दृष्टादौ ह्यृषयः परमात्म\-चिन्तका नैवमुपगच्छन्ति व्याकरणम्। अपि तु स्वसफलतार्थं व्याकरणमेव ताननुगच्छेत्। अतः शिष्ट\-प्रयुक्तानेव शब्दान् साधुत्व\-प्रतिपादनरूप\-सुमनोभिः सम्पूज्य कृतकृत्यतां व्रजति व्याकरणम्। अत इदानीं पाणिनीय\-प्रक्रियोपयोगि\-साधुत्वं मीमांसामहे। पूर्वोक्त\-साधुत्वं तु सार्वत्रिकम्। इदानीं पाणिनीय\-प्रक्रियायां साधुत्वमनुसन्दधे। \textcolor{red}{साधुत्वं नामाप्रवृत्त\-नित्य\-विध्युद्देश्यतावच्छेदकतानाक्रान्तत्वम्}।
यथा \textcolor{red}{सुध्युपास्यः} इत्यत्राप्रवृत्तो यो नित्य\-विधिर्दीर्घ\-गुणादिस्तदुद्देश्यतावच्छेदकता याऽक्त्वाच्त्व\-रूपा तदनाक्रान्तत्वं \textcolor{red}{सुध्युपास्यः} इत्यत्र साधुत्वमस्त्येव। अप्रवृत्तेति पदं \textcolor{red}{बाभवति} इत्याद्यसाधुत्व\-निरासार्थम्।
अन्यथा प्रवृत्तस्य विघात\-रूपस्य विधेरिक्त्व\-रूपोद्देश्यतावच्छेदकतया भवतीति पदमाक्रान्तमेव। अतोऽप्रवृत्तेति। नित्य\-विधि\-शब्दोपादानं विकल्प\-स्थलेऽपि साधुत्व\-प्रतिपादनार्थं यथा \textcolor{red}{चक्री अत्र} इति स्थले \textcolor{red}{इकोऽसवर्णे शाकल्यस्य ह्रस्वश्च} (पा॰सू॰~६.१.१२७) इत्यनेनासवर्णाजुपश्लिष्टाः पदान्ता इकः शाकल्यमते ह्रस्व\-समुचितं प्रकृति\-भावं भजन्त इत्यर्थानुसारमत्र घटकेऽकारेऽसवर्णेऽच्परे पदान्तश्चक्रीघटक ईकारः सह्रस्वं प्रकृति\-भावमभजद्विकल्पेन \textcolor{red}{चक्रि अत्र} इति पक्षे। \textcolor{red}{चक्री अत्र} इत्यवस्थायां यणि \textcolor{red}{चक्र्यत्र}। अत्रापि विकल्प\-विध्युद्देश्यतावच्छेदकताक्रान्तत्वेऽपि साधुत्वे न क्षतिः। इदमेव साधुत्वं पाणिनीय\-प्रक्रियार्थमुपयोगि। अतः शब्द\-नित्यत्व\-पक्षेऽयमर्थः क्रियते \textcolor{red}{इको यणचि} (पा॰सू॰~६.१.७७) इत्यादेः। अजुपश्लिष्टेग्घटितस्य स्थाने यण्घटितः प्रयोक्तव्यः स च साधुः। अतो भाष्यकारः कथयति~–\end{sloppypar}
\centering\textcolor{red}{सर्वे सर्वपदादेशा दाक्षीपुत्रस्य पाणिनेः।\nopagebreak\\
एकदेशविकारे हि नित्यत्वं नोपपद्यते॥}\nopagebreak\\
\raggedleft{–~भा॰पा॰सू॰~१.१.२०, ७.१.२७}\\
\begin{sloppypar}\hyphenrules{nohyphenation}\justifying\noindent अतः प्रक्रियार्थं शब्देषु काल्पनिको विकारः। प्रकृति\-प्रत्यय\-कल्पना तत्तदर्थ\-विकार\-कल्पना सर्वाऽप्यौपचारिकी। यथाऽऽह श्रीहरिः~–\end{sloppypar}
\centering\textcolor{red}{उपायाः शिक्षमाणानां बालानामुपलालनाः।\nopagebreak\\
असत्ये वर्त्मनि स्थित्वा ततः सत्यं समीहते॥}\nopagebreak\\
\raggedleft{–~वा॰प॰~२.२३८}\\
\begin{sloppypar}\hyphenrules{nohyphenation}\justifying\noindent अतः सत्यस्य शब्द\-ब्रह्मणः परिचयार्थमसत्याऽपि व्याकरण\-प्रक्रिया नितरामुपयोगिनी। यथा सोपानमन्तरा कोऽपि प्रासादमारोढुं न शक्नोति तथैव शब्द\-ब्रह्म\-ज्ञानमन्तरेण कश्चनापि पर\-ब्रह्म न साक्षात्कर्तुमीष्टे। अतो गुरवः पठन्ति~–\end{sloppypar}
\centering\textcolor{red}{शब्दब्रह्मणि निष्णातः परं ब्रह्माधिगच्छति॥}\nopagebreak\\
\raggedleft{–~ब्र॰उ॰~१७}\\
\begin{sloppypar}\hyphenrules{nohyphenation}\justifying\noindent\hspace{10mm} वस्तुतस्तु स्फोट एव मुख्यः। स चाष्ट\-विधः।\footnote{\textcolor{red}{वर्णस्फोटः पदस्फोटो वाक्य\-स्फोटोऽखण्ड\-पदवाक्य\-स्फोटौ वर्णपदवाक्य\-भेदेन त्रयो जातिस्फोटा इत्यष्टौ पक्षाः सिद्धान्त\-सिद्धाः} (वै॰भू॰सा॰~१४.६१)।} यथा स्वच्छं स्फटिकं जपाकुसुम\-संयोगे तद्गत\-रक्तिम्ना रक्ततामुपैति तथा निर्मलं चैतन्यमात्मा
कत्व\-गत्वादि\-ध्वनि\-रूप\-रूषितान्तः\-करणावच्छिन्नः सन् स्फोट\-सञ्ज्ञां लभते। अयमेव मुख्यः। अत्र व्युत्पत्ति\-द्वयं \textcolor{red}{स्फुटँ विकसने} (धा॰पा॰~२६०, १३७३) इत्यस्माद्धातोः \textcolor{red}{स्फुट्यते प्रकाश्यते} इति कर्म\-व्युत्पत्तौ कर्मणि घञ्।\footnote{\textcolor{red}{अकर्तरि च कारके सञ्ज्ञायाम्} (पा॰सू॰~३.३.१९) इत्यनेन।} अनुबन्धकार्ये गुणे\footnote{\textcolor{red}{पुगन्त\-लघूपधस्य च} (पा॰सू॰~७.३.८६) इत्यनेन।} विभक्तिकार्ये च स्फोटः। द्वितीये च \textcolor{red}{स्फुटत्यर्थो येन} इति विग्रहे बाहुलकाल्ल्युटं प्रबाध्य पुनः करणे घञ्।\footnote{सोऽपि \textcolor{red}{अकर्तरि च कारके सञ्ज्ञायाम्} (पा॰सू॰~३.३.१९) इत्यनेन।} इत्थं \textcolor{red}{ध्वनि\-व्यङ्ग्यत्वे सत्यर्थ\-विषयक\-बोध\-जनकतावच्छेदकत्वं स्फोटत्वम्} इति लक्षणम्। इमे द्वे व्युत्पत्ती भाष्य\-प्रदीप\-सम्मते। अथ \textcolor{red}{गौरित्यत्र कः शब्दः} इति जिज्ञासायां \textcolor{red}{येनोच्चारितेन सास्ना\-लाङ्गूल\-ककुद\-खुर\-विषाणिनां सम्प्रत्ययो भवति} (भा॰प॰) इति सिद्धान्तितं भगवता भाष्यकृता। अत्रोच्चारितेनेत्यस्य व्याख्यानं व्याचक्षते कैयटोपाध्याया~– \textcolor{red}{उच्चारितेन प्रकाशितेनेत्यर्थः}।\footnote{अत्र नागेशभट्ट\-पादाश्च~– \textcolor{red}{प्रकाशितेनेत्यभि\-व्यञ्जकैरीति शेषः}।} यथा तत्रत्य\-भाष्य\-प्रदीपौ। एतस्य वाक्यपदीय\-वैयाकरण\-भूषण\-सार\-वैयाकरण\-सिद्धान्त\-मञ्जूषा\-वैयाकरण\-सिद्धान्त\-परम\-लघु\-मञ्जूषादौ सविस्तरं चर्चा।\end{sloppypar}
\begin{sloppypar}\hyphenrules{nohyphenation}\justifying\noindent\hspace{10mm} अस्याञ्चाष्टाध्याय्यां प्रत्यध्यायं चत्वारः पादाः। प्रायश्चतुःसहस्रशो विचित्राणि सूत्राणि। अहो आश्चर्यमेतत्। कोटि\-कोटि\-प्रयोगाणां किं बहुना निखिलस्यापि वाङ्मय\-वारिधेश्चतुःसहस्र\-सूत्रैरेवानुशासनमित्येवास्य व्याकरणस्य मुख्यं वैशिष्ट्यम्। अत एव सर्व\-मान्यम्। एकैक\-सूत्रे विविध\-विषयाणां समन्वयः। स च शास्त्र\-विशेषः शब्दानुशासनं कुर्वन्नपि मानव\-मनो\-वृत्तिमपि व्याचष्टे। यथा सूत्रं स्पष्टं \textcolor{red}{श्वयुव\-मघोनामतद्धिते} (पा॰सू॰~६.४.१३३)। तद्धितं विहाय यजादि\-प्रत्यये परे\footnote{भसञ्ज्ञायामिति भावः। \textcolor{red}{यचि भम्} (पा॰सू॰~१.४.१८) इत्यनेन यजादि\-प्रत्यये परे पूर्वस्य भसञ्ज्ञा भवति।} \textcolor{red}{श्वन् युवन् मघवन्} शब्दाः सम्प्रसारणं लभन्ते। यद्यप्यनेन \textcolor{red}{शुनः यूनः मघोनः} इत्यादयः प्रयोगाः सिध्यन्ति तथाऽपि सहोक्त्या त्रयाणां प्रवृत्ति\-साम्यमपि प्रतीयते। तद्यथा श्वा कुक्कुरो युवा युवको मघवेन्द्र इमे त्रयः समानमेव विषय\-लोलुपाः कामुकाः स्वार्थान्धाश्च। अतस्तुलसी\-दासो रामचरितमानसेऽपि~–\end{sloppypar}
\centering\textcolor{red}{लखि हिय हँसि कह कृपानिधानू। सरिस श्वान मघवान जुबानू॥}\footnote{एतद्रूपान्तरम्–\textcolor{red}{दशां वीक्ष्य विहस्यापि हृद्यवोचत्कृपानिधिः। मघवा श्वा युवा चैव वर्तन्ते समतान्विताः॥} (मा॰भा॰~२.३०२.८)।}
\nopagebreak\\
\raggedleft{–~रा॰च॰मा॰~२.३०२.८}\\
\begin{sloppypar}\hyphenrules{nohyphenation}\justifying\noindent इममेवार्थं प्रतिपादयन् कविरेकोऽकथयद्यत्~–\end{sloppypar}
\centering\textcolor{red}{काचं मणिं काञ्चनमेकसूत्रे ग्रथ्नासि बाले किमु चित्रमेतत्।\nopagebreak\\
अशेषवित्पाणिनिरेकसूत्रे श्वानं युवानं मघवानमाह॥}\nopagebreak\\
\raggedleft{–~मौक्तिकम्}\\
\begin{sloppypar}\hyphenrules{nohyphenation}\justifying\noindent एवमेव बहुत्र व्यवहार\-पक्षस्याप्रत्यक्ष\-रूपेण चर्चा कृता वर्तते। यथा सावर्ण्यं व्याचक्षाणः पाणिनिः \textcolor{red}{तुल्यास्य\-प्रयत्नं सवर्णम्} (पा॰सू॰~१.१.९)। सत्यपि सवर्ण\-सञ्ज्ञा\-विधायकेऽस्मिन्नन्योऽप्यर्थो निर्गलति यत्तयोरेव वर्ण\-साम्यं ययोरास्य\-प्रयत्नावर्थादाकार\-प्रकारौ सदृशौ भवेताम्। \end{sloppypar}
\begin{sloppypar}\hyphenrules{nohyphenation}\justifying\noindent\hspace{10mm} इत्थमेव प्रातिपदिक\-सञ्ज्ञा\-विधायक\-सूत्र\-विषयेऽप्येका किंवदन्ती प्रहेलिका यत्~–\end{sloppypar}
\centering\textcolor{red}{धीरः कीदृग्वचो ब्रूते को रोगी कश्च नास्तिकः। \nopagebreak\\
कीदृक्चन्द्रं न पश्यन्ति तत्सूत्रं पाणिनेर्वद॥}\nopagebreak\\
\raggedleft{–~मौक्तिकम्}\\
\begin{sloppypar}\hyphenrules{nohyphenation}\justifying\noindent इति प्रश्ने। अर्थाद्धीरोऽर्थवद्वचो ब्रूते। अधातू रोगी भवति। अप्रत्ययो नास्तिकः कथ्यते। प्रातिपदिकं चन्द्रं न पश्यन्ति। सम्पूर्णं सूत्रं चतुर्णामपि प्रश्नानामुत्तर\-रूपं यत् \textcolor{red}{अर्थवदधातुरप्रत्ययः प्रातिपदिकम्} (पा॰सू॰~१.२.४५)। \end{sloppypar}
\begin{sloppypar}\hyphenrules{nohyphenation}\justifying\noindent\hspace{10mm} सङ्केतेनाव्युत्पन्नस्य चापि व्याख्या कृता। अव्युत्पन्नः प्रायशोऽर्थवान् धातु\-रहित ईश्वरे प्रत्यय\-रहितो भवति। शास्त्रे यद्यपि डित्थ\-डवित्थ\-साम्प्रतिक\-नाम\-शब्दानामेवाव्युत्पन्नत्वम्। तदर्थमेव सूत्रमिदम्। किन्तु व्यवहारेऽप्यनेनैवाव्युत्पन्न\-लक्षणं सङ्गमयितुं शक्यते। व्युत्पन्न\-प्रातिपदिक\-सञ्ज्ञा\-विधायकं सूत्रं \textcolor{red}{कृत्तद्धितसमासाश्च} (पा॰सू॰~१.२.४६)। अनेन \textcolor{red}{कर्ता वाराणसेयः रामदासः} इत्यादौ कृत्तद्धित\-समासानां प्रातिपदिक\-सञ्ज्ञा। तत्रापि व्यवहार\-व्युत्पन्न\-लक्षणं द्रष्टव्यम्। कृदर्थाद्यः सक्रियः। तद्धितोऽर्थात्तस्मै हितः परोपकारी। समासा अर्थात्समन्वय\-वादिनः। त एव व्युत्पन्नाः। यद्यपि शाकटायन\-मते सर्वमपि प्रातिपदिकं व्युत्पन्नमर्थाद्धातुजं प्रकृति\-प्रत्यय\-विभाग\-पूर्वकम्। यथोक्तम्~–\end{sloppypar}
\centering\textcolor{red}{नाम च धातुजमाह निरुक्ते व्याकरणे शकटस्य च तोकम्।\nopagebreak\\
यन्न विशेषपदार्थसमुत्थं प्रत्ययतः प्रकृतेश्च तदूह्यम्॥}\nopagebreak\\
\raggedleft{–~भा॰पा॰सू॰~३.३.१}\\
\begin{sloppypar}\hyphenrules{nohyphenation}\justifying\noindent किन्तु भगवान् पाणिनिर्महा\-सञ्ज्ञामन्वर्थां मन्यते। अतोऽर्थमविचार्यार्वाचीन\-विहित\-नामसु भाषान्तरीय\-शब्देषु च व्युत्पत्तिमस्वीकुर्वन् तेषां साधुत्वार्थ\-विभक्ति\-प्रतिपत्तयेऽव्युत्पन्न\-प्रातिपदिकं स्वीकृत्योक्त्वा चान्वर्थं नाम रामादीनां साधुत्वाय व्युत्पन्न\-प्रातिपदिक\-सञ्ज्ञार्थं कृत्तद्धितेति सूत्रं सूत्रयामास। \textcolor{red}{अव्युत्पन्न\-प्रातिपदिकत्वं नाम धातु\-प्रत्यय\-प्रत्ययान्त\-रहितत्वे सति लोकेऽर्थ\-बोधकत्वे सति प्रातिपदिक\-सञ्ज्ञावत्त्वम्}। यथा डित्थादौ। \textcolor{red}{व्युत्पन्न\-प्रातिपदिकत्वं नाम प्रकृति\-प्रत्यय\-जन्य\-लौकिकार्थ\-बोधकत्वे सति कृत्तद्धित\-समासान्यतमत्वे सति प्रातिपदिक\-सञ्ज्ञा\-भाक्त्वम्}। यथा कस्यचित्कुरूपस्य नाम मदन\-मोहन इति। व्युत्पत्तिः क्रियते मदनं कामं मोहयति। तर्हि कुरूप इयं व्युत्पत्तिर्घटिष्यते। अत एतादृशेषु स्थलेषु सम्भवायां व्युत्पत्तौ व्युत्पत्त्यनुसारमर्थाभावे प्रकृति\-प्रत्ययार्थ\-कल्पना\-त्याग एव स्वीकार्य इति। रूढानां पूर्वं कथित\-स्थलानां कृते चाव्युत्पन्न\-प्रातिपदिकतैवेति पाणिनि\-मनीषितं मे प्रतिभाति। व्युत्पन्न\-प्रातिपदिकता च रामादीनामन्वर्थानां कृते। यथा \textcolor{red}{रामः}। \textcolor{red}{रमन्ते योगिनो यस्मिन् स रामः} अथवा \textcolor{red}{रमयति सर्वाणि भूतानि यः स रामः} अथवा \textcolor{red}{रमते सर्वेषु भूतेषु यः स रामः} इत्यादयः सहस्रशोऽपि व्युत्पत्तयः सार्था अर्थापयितुं शक्यन्ते। राम\-शब्दस्य वाच्यतावच्छेदकत्वं लक्ष्यतावच्छेदकत्वञ्च परमात्मनि सच्चिदानन्द\-घने पर\-ब्रह्मणि दशरथात्मजे। अतोऽस्य कृते व्युत्पन्न\-प्रातिपदिकता। यथा कस्यचिद्भोजनार्थं म्रियमाणस्य परम\-दरिद्रस्य पुत्रस्य नाम \textcolor{red}{राज\-कुमारः} इति। सत्यपि सुलभतया तत्पुरुष\-समास\-सम्भवे \textcolor{red}{राज्ञः कुमारः} इति दरिद्र\-कुमारे जन्य\-जनक\-भाव\-सम्बन्धावच्छिन्न\-राज\-प्रतियोगि\-कुमारता नान्वर्था। अत एतत्कृते \textcolor{red}{अर्थवत्} (पा॰सू॰~१.२.४५) इति सूत्रमेव। रामस्तु धातूनां धातुः प्रत्ययानाञ्च प्रत्ययस्तस्य कृतेऽधातु\-घटित\-सूत्रं सञ्ज्ञार्थं न मे रोचते।\end{sloppypar}
\begin{sloppypar}\hyphenrules{nohyphenation}\justifying\noindent\hspace{10mm} अर्थवदिति सूत्रं चतुष्पदम्। एतत्सूत्रेण चतुष्पदेन चतुष्पद\-तुल्यानामव्युत्पन्नानामेव सञ्ज्ञा करणीया। तथा च कृत्तद्धितेऽति सूत्रञ्च श्रूयते चतुष्पदम्। पूर्व\-सूत्रतोऽर्थवत्प्रातिपदिकञ्चेति द्वे पदे अनुवर्त्येते। एवं श्रुतानुवृत्त\-सम्मेलनेन षट्पदम्~–\end{sloppypar}
\begin{sloppypar}\hyphenrules{nohyphenation}\justifying\noindent\hspace{10mm} (१) कृत्\end{sloppypar}
\begin{sloppypar}\hyphenrules{nohyphenation}\justifying\noindent\hspace{10mm} (२) तद्धित\end{sloppypar}
\begin{sloppypar}\hyphenrules{nohyphenation}\justifying\noindent\hspace{10mm} (३) समासाः\end{sloppypar}
\begin{sloppypar}\hyphenrules{nohyphenation}\justifying\noindent\hspace{10mm} (४) च\end{sloppypar}
\begin{sloppypar}\hyphenrules{nohyphenation}\justifying\noindent\hspace{10mm} (५) अर्थवत्\end{sloppypar}
\begin{sloppypar}\hyphenrules{nohyphenation}\justifying\noindent\hspace{10mm} (६) प्रातिपदिकम् \end{sloppypar}
\begin{sloppypar}\hyphenrules{nohyphenation}\justifying\noindent षडैश्वर्यञ्च~–\end{sloppypar}
\centering\textcolor{red}{ऐश्वर्यस्य समग्रस्य धर्मस्य यशसः श्रियः। \nopagebreak\\
ज्ञानवैराग्ययोश्चैव षण्णां भग इतीरणा॥}\nopagebreak\\
\raggedleft{–~वि॰पु॰~६.५.७४}\\
\begin{sloppypar}\hyphenrules{nohyphenation}\justifying\noindent अतः षडैश्वर्य\-सम्पन्नस्य रामस्य भगवतो वाचकस्य \textcolor{red}{रामः} इति शब्दस्य द्वितीय\-सूत्रेण प्रातिपदिक\-सञ्ज्ञा करणीया। यद्वा समास\-महिम्ना \textcolor{red}{कृत्तद्धित\-समासाः} इत्येकं पदं \textcolor{red}{च} इति द्वितीयम्। अतो द्विपद\-सूत्रेण द्विपदं मनुष्यमनुकुर्वतो रामभद्रस्य वाचकस्य \textcolor{red}{रामः} इति शब्दस्य द्वितीय\-सूत्रेणैव प्रातिपदिक\-सञ्ज्ञा मेऽतिरुचि\-करा लगति। द्विपद\-सूत्रेण सीता\-समेत\-रामचन्द्र\-वाचकस्य \textcolor{red}{रामः} इति शब्दस्य द्वितीय\-सूत्रेण प्रातिपदिक\-सञ्ज्ञा युक्ति\-युक्ता भक्ति\-सहकृता हृदय\-रमणीया च। तस्मात्प्रथम\-सूत्रेणार्वाचीन\-नाम्नां भाषान्तरीय\-शब्दानां डित्थादीनां रूढतावच्छेदकवतां प्रातिपदिक\-सञ्ज्ञा राम\-मुख्यानां च द्वितीय\-सूत्रेण। एवं लघु\-सिद्धान्त\-कौमुदी\-वैयाकरण\-सिद्धान्त\-कौमुद्यादावर्थवत्सूत्रोदाहरणं राम\-कृष्ण\-मुकुन्दादि किमपि चेतस्तुदति। कदाचिदिमान्यर्वाचीन\-जन\-साधारण\-वाचकानि तदा सुष्ठु। किं बहुना। दशरथापत्य\-ब्रह्म\-वाचक\-राम\-शब्दमव्युत्पन्न\-प्रातिपदिकमित्यङ्गीकर्तुमहमाशिरसि च्छेदमपि नोत्सह इति विद्वांसो बाल\-चापलं क्षमन्ताम्। एवं\-विधानि बहूनि स्थलानि सन्ति येषु पदे पदे व्यवहारिकता सामाजिकता वैज्ञानिकताऽऽध्यात्मिकता च। किं बहुना। अष्टाध्याय्या राम\-कथया समन्वयः। यथा राम\-कथाया वृद्धौ तात्पर्यं रामायणस्यान्ते कवि\-कोकिलो भगवान् वाल्मीकिः कथयति~–\end{sloppypar}
\centering\textcolor{red}{बलं विष्णोः प्रवर्धताम्॥}\nopagebreak\\
\raggedleft{–~वा॰रा॰~६.१२८.१२१}\\
\begin{sloppypar}\hyphenrules{nohyphenation}\justifying\noindent एवमेव भगवान् पाणिनिरप्यष्टाध्याय्याः प्रथमं सूत्रं लिखन् वृद्धि\-शब्दं रामायणस्य सार\-रूपं महा\-मन्त्रं स्मरति। \textcolor{red}{वृद्धिरादैच्} (पा॰सू॰~१.१.१) इति। \textcolor{red}{भूवादयो धातवः} (पा॰सू॰~१.३.१) इत्यत्र भाष्यं भाषमाणाः पतञ्जलयः प्राहुर्यत् \textcolor{red}{मङ्गलादीनि मङ्गलमध्यानि मङ्गलान्तानि शास्त्राणि प्रथन्ते वीरपुरुषाणि भवन्त्यायुष्मत्पुरुषाणि चाध्येतारश्च सिद्धार्था यथा स्युरिति} (भा॰पा॰सू॰~१.३.१)। माङ्गलिक आचार्यो मङ्गलार्थं वृद्धिशब्दं प्रयुङ्क्ते। तत्रेत्थं विचारश्चलितो यद्वृद्धिः सञ्ज्ञाऽऽदैच्च सञ्ज्ञी। नियमोऽयमुद्देश्यं पूर्वमतः कथयन्त्याचार्याः \textcolor{red}{उद्देश्य\-शब्दः पूर्वं विधेयश्च ततः परम्}। किन्त्वत्र कथं विधेयं पूर्वमुद्देश्यं परमिति। तदेत्थं समाधानं कर्तुं शक्यते यदुद्देश्यस्य पूर्वं प्रयोगः प्रायिकः। यथा \textcolor{red}{अदेङ्गुणः} (पा॰सू॰~१.१.२) अत्रोद्देश्यः पूर्वं विधेयश्च परम्।\end{sloppypar}
\begin{sloppypar}\hyphenrules{nohyphenation}\justifying\noindent\hspace{10mm} पुनः विप्रतिपत्तिर्यथा \textcolor{red}{घु टि भ घि} इत्यादयो लघवः सञ्ज्ञाः। तथैव कथं नात्र लघु\-सञ्ज्ञा किं महा\-सञ्ज्ञया। तदाऽऽचार्यो मङ्गलार्थमिति समुच्चारयामास। तच्च वृद्धिरूपं मङ्गलं रामकथायाः सिद्धान्त\-भूतम्। अष्टाध्यायी राम\-चरित\-मानसञ्चोभावपि ग्रन्थौ वकारतः प्रारब्धौ।\footnote{\textcolor{red}{वृद्धिरादैच्} (पा॰सू॰~१.१.१) इत्यनेन सूत्रेणाष्टाध्यायी प्रारब्धा। \textcolor{red}{वर्णानामर्थ\-सङ्घानां रसानां छन्दसामपि} (रा॰च॰मा॰~१/म॰श्लो॰१) इत्यनेनानुष्टुपा रामचरितमानसं प्रारब्धम्। अयं ग्रन्थोऽपि वकारतः प्रारब्धः।} वकारञ्चामृत\-बीजम्।\footnote{यथा शिवपुराणे~– \textcolor{red}{शं नित्यं सुखमानन्दमिकारः पुरुषः स्मृतः॥ वकारः शक्तिरमृतं मेलनं शिव उच्यते। तस्मादेवं स्वमात्मानं शिवं कृत्वाऽर्चयेच्छिवम्॥} (शि॰पु॰~१८.७६–७७)। तन्त्रशास्त्रेऽपि~– \textcolor{red}{“ब्रह्मरन्ध्रे यवादूर्ध्वं कुलपद्मं महेश्वरि। श्वेतं सुकेसरोपेतं सहस्रारमधोमुखम्॥” इत्यादिना “व्यापिनी केवला शक्तिरमृतौघप्रवर्षिणी।” इत्यन्तेन स्वच्छन्द\-सङ्ग्रहोक्त\-रीत्याऽमृतमयो वकारः} (यो॰हृ॰ दी॰टी॰~३.१३७)।} सम्पूर्णेऽस्मिन् पाणिनीये शब्द\-ब्रह्मामृतस्य चर्चा। \textcolor{red}{न म्रियत इत्यमृतम्}। \textcolor{red}{मृङ् प्राण\-त्यागे} (धा॰पा॰~१४०३) इत्यस्माद्धातोः कर्तरि क्तो नञ्समासश्च। शब्द\-ब्रह्म जन्म\-मरण\-रहितं यथा~–\end{sloppypar}
\centering\textcolor{red}{अनादिनिधनं ब्रह्म शब्दतत्त्वं यदक्षरम्।\nopagebreak\\
विवर्ततेऽर्थभावेन प्रक्रिया जगतो यतः॥}\nopagebreak\\
\raggedleft{–~वा॰प॰~१.१}\\
\begin{sloppypar}\hyphenrules{nohyphenation}\justifying\noindent एवमादौ वृद्धिरूपं मङ्गलाचरणं कृत्वा अन्ते \textcolor{red}{अ अ} (पा॰सू॰~८.४.६८) इति विलिख्य \textcolor{red}{अकारो वासुदेवः} इति श्रुत्यनुसारं पुनरकार\-वाच्यं वासुदेवं श्रीरामचन्द्रं एव स्मरन् विरमति रामे। अस्यां प्रत्यध्यायं चत्वारः पादाः समग्रस्य वाङ्मयस्य सङ्क्षिप्त\-परिचयो भौगोलिक\-परिस्थितेः परिशीलनमार्ष\-चक्षुषा शब्दानां परिलोकनम्। यथा \textcolor{red}{उदक्च विपाशः} (पा॰सू॰~४.२.७४) इत्यादि।\end{sloppypar}
\begin{sloppypar}\hyphenrules{nohyphenation}\justifying\noindent\hspace{10mm} किं नाम सूत्रत्वमित्यपेक्षायाम् \textcolor{red}{अल्पाक्षरत्वे सति बह्वर्थ\-बोधकत्वम्} इति।\footnote{पूर्वपक्षोऽयम्।} अस्मिल्लँक्षणे स्वीकृते \textcolor{red}{हरि}\-शब्देऽति\-व्याप्तिः। तत्राऽप्यल्पाक्षर\-त्वादिन्द्र\-सूर्य\-सर्प\-सिंह\-विष्णु\-प्रभृति\-बह्वर्थ\-बोधकत्वाच्च।\footnote{\textcolor{red}{यमानिलेन्द्र\-चन्द्रार्क\-विष्णु\-सिंहांशु\-वाजिषु॥ शुकाहि\-कपि\-भेकेषु हरिर्ना कपिले त्रिषु।} (अ॰को॰~३.३.१७४-१७५) इत्यमरः। अपि च~– \textcolor{red}{हरिर्विष्णावहाविन्द्रे भेके सिंहे हये रवौ। चन्द्रे कोले प्लवङ्गे च यमे वाते च कीर्तितः। वारि वारिदके वाऽपि नवपञ्चार्थकः स्मृतः॥} (मूलं मृग्यम्)।} हरि\-शब्दस्यानेकार्थ\-तामाशङ्क्य कवि\-कुल\-गुरुः कविता\-कामिनी\-विलासो महा\-कवि\-कालिदासः स्वकीय\-रघुवंश\-महा\-काव्यस्य त्रयोदशे सर्गे प्रथमे श्लोके पुष्पकारूढ\-रामं सीतायायाशंसन्तं समुद्रं हरि\-शब्दस्य \textcolor{red}{रामाभिधानः} इति विशेषणं प्रयुञ्जानस्तं दशरथापत्य\-श्रीराम\-रूप\-ब्रह्म\-वाचकं व्यवस्थापयन्नितरार्थेभ्यो व्यावर्तयति~–\end{sloppypar}
\centering\textcolor{red}{अथात्मनः शब्दगुणं गुणज्ञः पदं विमानेन विगाहमानः।\nopagebreak\\
रत्नाकरं वीक्ष्य मिथः स जायां रामाभिधानो हरिरित्युवाच॥}\nopagebreak\\
\raggedleft{–~र॰वं॰~१३.१}\\
\begin{sloppypar}\hyphenrules{nohyphenation}\justifying\noindent \textcolor{red}{रामाभिधानः} इति विशेषणमेव हरि\-शब्दस्यानेकार्थत्वं प्रमाणयति। तस्मादुक्तं लक्षणमत्रातिव्याप्तम्। \textcolor{red}{तदेव हि लक्षणं यदव्याप्त्यति\-व्याप्त्यसम्भवरूप\-दोष\-त्रय\-शून्यम्}। तथा च  \textcolor{red}{अव्याप्त्यति\-व्याप्त्यसम्भव\-रूप\-दोष\-त्रय\-शून्यत्वे सत्यसाधारण\-धर्मत्वं लक्षणत्वम्}। अस्यैक\-देशावृत्तित्वमव्याप्तित्वम्। यथा कपिलत्वं गोत्वम्। कपिलत्वं हि लक्ष्यस्य गोः सकल\-देशे न वर्तत इत्यव्याप्तिः। लक्ष्य\-वृत्तित्वे सत्यलक्ष्य\-वृत्तित्वमतिव्याप्तित्वम्। यथा शृङ्गित्वं गोत्वम्। शृङ्गित्वं हि गवि च लक्ष्येऽलक्ष्य\-भूते गवेतरे महिषादौ चातितिष्ठतीत्यतिव्याप्तम्। असम्भवत्वं लक्ष्यमात्रावृत्तित्वम्। यथा पुष्पवत्त्वमाकाशत्वम्। लक्ष्य\-भूत आकाशे पुष्पाभावादसम्भवमत्र। इत्थमल्पाक्षरत्वे सति बह्वर्थ\-बोधकत्वं सूत्रत्वमिति सूत्र\-लक्षणस्य हरि\-शब्दोऽतिव्याप्तत्वे स्वरूपं लक्षयामः~–\end{sloppypar}
\centering\textcolor{red}{अल्पाक्षरमसन्दिग्धं सारवद्विश्वतोमुखम्। \nopagebreak\\
अस्तोभमनवद्यञ्च सूत्रं सूत्रविदो विदुः॥}\nopagebreak\\
\raggedleft{–~परा॰उ॰~१८.१३,१४}\\
\begin{sloppypar}\hyphenrules{nohyphenation}\justifying\noindent\hspace{10mm} इत्थम् \textcolor{red}{असन्दिग्ध\-बह्वर्थ\-बोधकाल्पाक्षरत्वे सति सिद्धान्त\-प्रतिपादकत्वं सूत्रत्वम्} इति मे शिशु\-मतिः। इमानि सूत्राणि षड्विधानि। सञ्ज्ञा\-सूत्रं परिभाषा\-सूत्रं विधि\-सूत्रं नियम\-सूत्रमतिदेश\-सूत्रमधिकार\-सूत्रञ्च। \textcolor{red}{सञ्ज्ञा\-सूत्रत्वं नाम साक्षाच्छक्ति\-ग्राहकत्वे सति पाणिन्युच्चरितत्वम्}। यथा \textcolor{red}{वृद्धिरादैच्} (पा॰सू॰~१.१.१) \textcolor{red}{अदेङ्गुणः} (पा॰सू॰~१.१.२) इत्यादि। अत्रादैचि वृद्धिरूपा साक्षाच्छक्तिर्ग्राहिता। आदैज्वृद्धि\-निष्ठ\-शक्तिमान् भवत्विति। \textcolor{red}{परिभाषा\-सूत्रत्वं नामानियमे नियमकारित्वम्}। यथा \textcolor{red}{सुधी उपास्य} इति स्थिते \textcolor{red}{इको यणचि} (पा॰सू॰~६.१.७७) इत्यनेन यणि विधीयमानेऽनियमः। \textcolor{red}{अचि इकः यण् स्यात्} इत्येव सूत्रार्थः। अत्र षष्ठ्याः कोऽर्थः कोऽनुयोगी कः प्रतियोगी यतो हीक्शब्दस्य व्यवहार\-जडस्य केन सम्बन्ध इत्यनियमे \textcolor{red}{षष्ठी स्थानेयोगा} (पा॰सू॰~१.१.४९) इति परिभाषा\-सूत्रमागतम्। \textcolor{red}{स्थानेयोगा} इत्यत्र स्थानेन प्रसङ्गेन योगो यस्याः सा। एकार आर्षः। अथवा \textcolor{red}{कण्ठेकालः} इतिवत् \textcolor{red}{स्थाने योगो यस्याः सा} इति सप्तम्या अलुक्। अनिर्धारित\-सम्बन्ध\-विशेषा षष्ठी स्थानानुयोगिक\-सम्बन्धार्थवती भवेदिति तात्पर्यम्। सम्बन्धश्च विषय\-विषयि\-भावः। इत्थं \textcolor{red}{स्थाने} इति पदेन \textcolor{red}{इकः स्थाने यण्} इत्यर्थः। \textcolor{red}{अचि} इत्यत्र पुनरनियमः। \textcolor{red}{अचि} इत्यत्र सप्तमी। सा च \textcolor{red}{सप्तम्यधिकरणे च} (पा॰सू॰~२.३.३६) इति सूत्रेणाधिकरणे। अधिकरणं नाम \textcolor{red}{आधारोऽधिकरणम्} (पा॰सू॰~१.४.४५) इति सूत्रेणाधारनामकम्। आधारश्च यथाऽभिव्यापक औपश्लेषिको वैषयिकश्च। यत्र सम्पूर्णमाधेयमाधारो व्याप्नोति तत्रैवाभिव्यापकः। यथा दध्नि घृतम्। सर्वस्मिन् आत्मा। वैषयिको यदाऽऽधार आधेयं विषयतया गृह्णाति। यथा मोक्ष इच्छा। रामे प्रेम। गुरौ श्रद्धा। चरित्रे निष्ठा। सिद्धान्ते दृढता। भक्तौ हठः। संसारे नीरसतेत्यादि। औपश्लेषिको यदाऽऽधार आधेयमुपश्लिष्यति तेन सह सम्बद्धो भवति। उपश्लेषश्च संयोगेन समवायेन सामीप्येन। संयोगेन यथा \textcolor{red}{कटे शेते}। समवायेन यथा \textcolor{red}{शरीरे चेष्टा}। सामीप्येन यथा \textcolor{red}{गुरौ वसति}।\footnote{गुरोः समीपे वसतीत्यर्थः।} अतः \textcolor{red}{अचि} इत्यत्रौपश्लेषिकी। सा च संयोगात्मिका। इत्थम् \textcolor{red}{अजुपश्लिष्टस्येकः स्थाने यण् स्यात्} इत्यर्थः। पुनरनियमः सम्बन्धस्तु संयोगात्मकः पूर्वेण परेण च भवति तदा कुत्र यण् यथा \textcolor{red}{सुधी उपास्य} इत्यत्र सुघटक उकार इत्यच्पुनर्धी\-घटक ईकारोऽच्पुनर्धी\-घटक ईकार इगुपास्य\-घटक उकारोऽच्पुनरुपास्य\-घटक उकार इक्पाकार\-घटक आकारोऽजित्थं व्यवहितेऽव्यवहिते पूर्वत्र परत्र च यणि प्रसक्ते नियमः कृतः परिभाषया। अच्यव्यवहित उच्चरिते पूर्वस्याव्यवहितस्यैव। यथा सूत्रं \textcolor{red}{तस्मिन्निति निर्दिष्टे पूर्वस्य} (पा॰सू॰~१.१.६६)। अस्यार्थः सप्तमी\-निर्देशेन विधीयमानं कार्यं वर्णान्तरेणाव्यवहितस्य पूर्वस्य बोध्यम्। पुनरनियमश्चतुर्षु यण्सु को भवेत्तदा परिभाषा\-सूत्रं न्ययमयत्। \textcolor{red}{स्थानेऽन्तरतमः} (पा॰सू॰~१.१.५०)। स्थानञ्च प्रसङ्गः। तस्मिन् सत्यन्तरतम आदेशः स्यात्। यद्यप्यान्तरतम्यं साधयितुं शक्यम्। तत्र हि स्थान\-कृतमान्तर्यं यथा \textcolor{red}{कृष्णैकत्वम्}। अत्राकारैकारयोः स्थानिनोः सदृशतमः कण्ठतालु\-स्थानी \textcolor{red}{ऐ} इति। द्वितीयमर्थ\-कृतमान्तर्यम्।
क्रोष्टु\-शब्दोऽर्थद्वये प्रसिद्धो राजर्षौ\footnote{\textcolor{red}{क्रोष्टोः शृणुत राजर्षेर्वंशमुत्तमपूरुषम्॥ यस्यान्ववाये संभूतो वृष्णिर्वृष्णिकुलोद्वहः।} (ब्रह्मा॰पु॰~३.७०.१४-१५)। } वृके\footnote{\textcolor{red}{स्त्रियां शिवा भूरिमायगोमायुमृगधूर्तकाः। शृगाल\-वञ्चक\-क्रोष्टु\-फेरुफेरव\-जम्बुकाः॥} (अ॰को॰~२.५.५)।} च। 
तत्र \textcolor{red}{तृज्वत्क्रोष्टुः} (पा॰सू॰~७.१.९५) इत्यत्रार्थ\-कृतान्तर्यानुरोधेन शृगाल\-वाचक\-क्रोष्टु\-शब्द एवाऽदेशत्वेन जातः। गुण\-कृतमान्तर्यं यथा \textcolor{red}{वाग्घरिः} इत्यत्र \textcolor{red}{वाग् हरिः} इत्यवस्थायां \textcolor{red}{झयो होऽन्यतरस्याम्} (पा॰सू॰~८.४.६२) इत्यनेन हकारस्य पूर्व\-सवर्णे प्राप्ते पूर्वत्र गकारस्तस्य सन्ति चत्वारः सवर्णाः स्वयं च मिलित्वा पञ्च हकारस्य स्थाने पञ्चसु को भवेत्तदा गुण\-कृतान्तर्यानुरोधेन नाद\-घोष\-संवार\-महा\-प्राणवतो हकारस्य स्थाने पूर्वत्र पञ्चसु तादृङ्नाद\-घोष\-संवार\-महा\-प्राणवाञ्चतुर्थो घकारः। प्रमाण\-कृतमान्तर्यं यथा \textcolor{red}{अदसोऽसेर्दादु दो मः} (पा॰सू॰~८.२.८०)। अत्र क्रमशो ह्रस्वस्य स्थाने ह्रस्व उकारो दीर्घस्य स्थाने दीर्घः \textcolor{red}{अमू} इति सिद्धमिदं प्रमाण\-कृतमान्तर्यम्। एषु दर्शितेषु चतुर्विधेष्वान्तर्येषु \textcolor{red}{सुधी उपास्य} इत्यत्र किं स्यादित्यपेक्षायां \textcolor{red}{यत्रानेक\-विधमान्तर्यं तत्र स्थानत आन्तर्यं बलीयः} (वै॰सि॰कौ॰~३९) इति परिभाषयेतरेषु व्यावर्तितेषु स्थान\-कृतमान्तर्यमालम्ब्य चतुर्षु यण्स्विकार\-सदृश\-तालु\-स्थानवान् यकारः। इत्थं परिभाषा\-त्रय\-योग\-दानेन \textcolor{red}{इको यणचि} (पा॰सू॰~६.१.७७) इत्यस्य निष्कृष्टोऽर्थः \textcolor{red}{अजुपश्लिष्ट\-पूर्वत्व\-विशिष्टस्येकः स्थानेऽन्तरतमो यण् स्यात्स च प्रयोक्तव्यः स्यात्स च साधु स्यात्}। एवमिग्घटित\-स्थाने यण्घटितः स्यात्स च प्रयोक्तव्यः स च साधु इति शब्द\-नित्यत्व\-पक्षीयोऽर्थः। \textcolor{red}{विधि\-सूत्रं नाम मुख्य\-लक्ष्य\-संस्कारकमर्थाल्लक्ष्य\-संस्कारे मुख्यतया सहायकतावच्छेदकम्}। यथा \textcolor{red}{इको यणचि} (पा॰सू॰~६.१.७७) \textcolor{red}{आद्गुणः} (पा॰सू॰~६.१.८७) \textcolor{red}{वृद्धिरेचि} (पा॰सू॰~६.१.८८) \textcolor{red}{अकः सवर्णे दीर्घः} (पा॰सू॰~६.१.१०१) \textcolor{red}{मोऽनुस्वारः} (पा॰सू॰~८.३.२३) इति। एवमेव \textcolor{red}{नियमो विधौ संशोधन\-रूपः}। यथा \textcolor{red}{कृत्तद्धित\-समासाश्च} (पा॰सू॰~१.२.४६) \textcolor{red}{धातोस्तन्निमित्तस्यैव} (पा॰सू॰~६.१.८०) इत्यादि। अयं प्रायशो लक्ष्यं नियमयति। यथा \textcolor{red}{एचोऽयवायावः} (पा॰सू॰~६.१.७८) इत्यनेन सर्वत्रैचोऽच्ययादौ प्राप्ते नियमो जातो धातु\-घटकैचो यद्यवावादेशस्तर्हि धातु\-निमित्तस्यैव। तेन लव्यमित्यत्र \textcolor{red}{लूञ् छेदने} (धा॰पा॰~१४८३) इति धातोः \textcolor{red}{अचो यत्} (पा॰सू॰~३.१.९७) इत्यनेन यति \textcolor{red}{सार्वधातुकार्धधातुकयोः} (पा॰सू॰~७.३.८४) इत्यनेन गुणे धातु\-निमित्तकस्यौकारस्यैवावादेश एवमन्यत्रापि। \textcolor{red}{अतिदेशत्वं नाम वति\-घटितत्वम्}। अर्थात् \textcolor{red}{वति\-घटितत्वे सति समारोपित\-धर्म\-प्रतिपादकतावच्छेदकतावत्त्वम्}। यथा \textcolor{red}{स्थानिवदादेशोऽनल्विधौ} (पा॰सू॰~१.१.५६) \textcolor{red}{अचः परस्मिन् पूर्वविधौ} (पा॰सू॰~१.१.५७) \textcolor{red}{आद्यन्तवदेकस्मिन्} (पा॰सू॰~१.१.२१) इत्यादि।
\textcolor{red}{स्थानिवदादेशोऽनल्विधौ} (पा॰सू॰~१.१.५६) इदं सूत्रमादेशे स्थानि\-धर्ममारोपयति। \textcolor{red}{अधिकारत्वं नाम स्व\-देशे वाक्यार्थ\-जनकता\-शून्यत्वे सति पर\-देशे बोधकत्वमुत्तरोत्तर\-सम्बन्धत्वं वा}। यथा \textcolor{red}{संहितायाम्} (पा॰सू॰~६.१.७२, ६.३.११४) \textcolor{red}{धातोः} (पा॰सू॰~३.१.९१) \textcolor{red}{प्रत्ययः} (पा॰सू॰~३.१.१) \textcolor{red}{परश्च} (पा॰सू॰~३.१.२) इत्यादि। सङ्ग्रहश्चामीषां सूत्र\-प्रकाराणामित्थम्~–\end{sloppypar}
\centering\textcolor{red}{सञ्ज्ञा च परिभाषा च विधिर्नियम एव च। \nopagebreak\\
अतिदेशोऽधिकारश्च षड्विधं सूत्रलक्षणम्॥}\footnote{मूलं मृग्यम्। श्लोकमिमुद्धृत्य \textcolor{red}{मुग्धबोधटीकायां दुर्गादासः} इति वाचस्पत्य\-काराः।}\\
\begin{sloppypar}\hyphenrules{nohyphenation}\justifying\noindent\hspace{10mm} इमामेवाष्टाध्यायीमाश्रित्य पाणिनीयं व्याकरणं प्रावर्तत। लोकेऽस्मिन्नपराणि गौरव\-प्रधानानीदं च लाघव\-प्रधानम्। अतः परिभाषामिमां भाषन्ते भाष्यकाराः~– \textcolor{red}{अर्धमात्रालाघवेन पुत्रोत्सवं मन्यन्ते वैयाकरणाः}।\footnote{\textcolor{red}{‘एओङ्-ऐऔच्’सूत्रयोर्ध्वनितैषा भाष्ये} (प॰शे॰~१३३)।} अतो यावत्सम्भवमासीत्तावल्लघुता वर्तिता पाणिनिना। अल्पान्येव सन्ति सूत्राणि। कानिचिद्विशालानि यथा \textcolor{red}{न पदान्त\-द्विर्वचन\-वरेय\-लोप\-स्वर\-सवर्णानुस्वार\-दीर्घ\-जश्चर्विधिषु} (पा॰सू॰~१.१.५८)। बहूनि च सूत्राणि पञ्चाक्षराणि चतुरक्षराणि त्र्यक्षराणि द्व्यक्षराण्येकाक्षराणि च सन्ति। यथा \textcolor{red}{इको यणचि} (पा॰सू॰~६.१.७७) \textcolor{red}{अनचि च} (पा॰सू॰~८.४.४७) \textcolor{red}{नाज्झलौ} (पा॰सू॰~१.१.१०) \textcolor{red}{चोः कुः} (पा॰सू॰~८.२.३०) \textcolor{red}{अचः} (पा॰सू॰~६.४.१३८) \textcolor{red}{हलः} (पा॰सू॰~६.४.२) \textcolor{red}{चौ} (पा॰सू॰~६.१.२२२, ६.३.१३८) \textcolor{red}{टेः} (पा॰सू॰~६.४.१४३, ६.४.१४५) इत्यादीनि। इत्थमेव महता प्रयासेन लघु\-तममिदं व्याकरणं निर्माय लोकमिममुप\-चकार शालातुरीयः। भाष्यकाराणां प्रमाणेन महापुरुषस्यास्य जन्म \textcolor{red}{शलातुर}\-नामके स्थाने सूच्यते।\footnote{मूलं भाष्य\-संस्करणेषु मृग्यम्।} सूत्रेऽपि \textcolor{red}{शलातुर}\-शब्दस्य चर्चा कृता (\textcolor{red}{तूदी\-शलातुर\-वर्मती\-कूच\-वाराड्ढक्छण्ढञ्यकः} पा॰सू॰~४.३.९४)।\end{sloppypar}
\begin{sloppypar}\hyphenrules{nohyphenation}\justifying\noindent\hspace{10mm} एतस्य पितुर्नाम \textcolor{red}{पणिनः}। अत एव \textcolor{red}{पणिनस्यापत्यं पुमान् पाणिनिः} इति कथ्यते। एतस्य मातुर्नामासीत् \textcolor{red}{दाक्षी} इति भाष्य\-वचनादवगम्यते। भाष्य\-कारो भगवन्तं पाणिनिं \textcolor{red}{दाक्षीपुत्र} इति सम्बोधयति यथा~–\end{sloppypar}
\centering\textcolor{red}{सर्वे सर्वपदादेशा दाक्षीपुत्रस्य पाणिनेः। \nopagebreak\\
एकदेशविकारे हि नित्यत्वं नोपपद्यते॥}\nopagebreak\\
\raggedleft{–~भा॰पा॰सू॰~१.१.२०, ७.१.२७}\\
\begin{sloppypar}\hyphenrules{nohyphenation}\justifying\noindent अयं च दाक्षी\-पुत्रो भगवान् पाणिनिः पाटलिपुत्रेऽधीतवानिति कथा\-सरित्सागरे लिखितं\footnote{क॰स॰सा॰~१.२.४५–४६, १.४.२०–२५।} किन्तु दृढतमं प्रमाणं किमपि न लभ्यते। वररुचिरेतस्य मित्रं सहाध्यायी च तत्परिस्पर्धी। अनेनाष्टाध्यायी लिखिता वररुचिना कात्यायनापर\-नामधेयेन सूत्रेषु त्रुटीरन्विष्य वार्त्तिकानि लिखितानि। यथा सूत्रं \textcolor{red}{न पदान्ताट्टोरनाम्} (पा॰सू॰~८.४.४२) तदुपरि वार्त्तिकम् \textcolor{red}{अनाम्नवति\-नगरीणामिति वाच्यम्}। सूत्रम् \textcolor{red}{आलजाटचौ बहुभाषिणि} (पा॰सू॰~५.२.१२५) वार्त्तिकं \textcolor{red}{कुत्सित\-ग्रहणं कर्तव्यम्}। एवमन्यत्रापि। यद्यपि पर\-वर्ती भाष्यकारः सूत्रमेव प्रमाणं मत्वा वार्त्तिकानि प्रायो निरर्थकानि स्वीचक्रे। इत्थमपि श्रूयते यद्द्वयोरप्याचार्ययोरीर्ष्या पराकाष्ठामाञ्चत्। बहुत्र सूत्राणि निराधारमाक्षेप्य वार्त्तिक\-रचनेन पाणिनिं प्रति कात्यायनो द्वेषं प्रमाणयति। अन्ते चोभौ मृत्यवे परस्परं शप्तवन्तौ। प्रातः कात्यायनो दिवं गतः पश्चात्पाणिनिः स्वर्गमाससाद त्रयोदश्याम्। अतोऽस्मत्सम्प्रदाये त्रयोदश्यां व्याकरणं न पाठ्यते। अतः पठन्ति गुरवः~–\end{sloppypar}
\centering\textcolor{red}{काणादं पाणिनीयं च त्रयोदश्यां न पाठयेत्।}\nopagebreak\\
\raggedleft{–~इत्यस्मद्गुरवः}\\
\begin{sloppypar}\hyphenrules{nohyphenation}\justifying\noindent कुत्रचिदित्थं मिलति यद्वने वसन्तं तमारण्यको व्याघ्रोऽकालयत्। विशेषेणासमन्ताज्जिघ्रतीति व्युत्पत्त्यनुसारं व्याघ्रस्य हिंसा\-कर्म न प्रामाणिकं किन्त्वन्ते तदेव जातम्। अतः पठ्यते~– \textcolor{red}{व्याघ्रो व्याकरणस्य शीघ्रमहरत्प्राणान् प्रियान् पाणिनेः}।\footnote{मूलं मृग्यम्। \textcolor{red}{सिंहो व्याकरणस्य कर्तुरहरत्प्राणान् प्रियान् पाणिनेः} (प॰त॰~२.३६) इति पञ्चतन्त्रकाराः।} इदमपि स्पष्ट\-प्रमाणाभावे निःसारमेव प्रतिभाति मे। अस्तु महा\-पुरुषाणां जन्म यत्र कुत्रापि स्यात्किन्तु तेषां कलित\-धर्म\-वर्म\-संवर्धित\-नम्र\-मनीषि\-नर्म\-समुद्घाटित\-वेदान्त\-निगूढ\-मर्म\-सकल\-लोकालङ्कार\-परमोदार\-लोकोत्तर\-कर्म शर्मणे कल्पते निखिल\-भुवनानाम्। इदमेव व्याकरणं समभवद्विबुध\-भारती\-वल्लभालङ्करणं तरुणतर\-मनीषा\-समुल्लसित\-विकसित\-कञ्ज\-माला\-लसल्ललाम\-भुवनाभिराम\-नव\-नवोन्मेष\-शालि\-प्रतिभा\-सम्भासुर\-सकल\-सद्गुण\-प्रचुर\-निखिल\-कला\-पुर\-विद्या\-नव\-वनिता\-नूपुर\-मुखर\-मधुर\-भाव\-विलसदुरःस्थल\-ललित\-कम्र\-कल्पना\-कलेवर\-वरेण्य\-वन्दित\-पाणि\-विलास\-समुल्लसच्चित्तानां विदुषां विहरणम्।\end{sloppypar}
\begin{sloppypar}\hyphenrules{nohyphenation}\justifying\noindent\hspace{10mm} पाणिनेरनन्तरं कात्यायन\-नामधेय आचार्यः पाणिनीय\-व्याकरण\-श्रियं समर्चयामास स्वकीय\-बुद्धि\-कौशल\-कुसुमावलीभिः। अनेन द्वि\-सहस्र\-प्राय\-वार्त्तिकानि समावर्तितानि। बहुशः प्रयोगाश्च वार्त्तिक\-द्वारा साधिताः। यथा \textcolor{red}{अभिवादि\-दृशोरात्मनेपदे वेति वाच्यम्} (वा॰~१.४.५३) \textcolor{red}{संपुंकानां सो वक्तव्यः} (वा॰~८.३.५) इत्यादि। पश्चात्पाणिनीय\-व्याकरणस्यान्तिम आचार्याः शेषावताराः पाणिनीय\-व्याकरणालङ्काराः पाणिनि\-चरण\-कमल\-बद्धाञ्जलयः शब्द\-सागर\-सम्पूर्ण\-सलिल\-सलील\-मण्डिताञ्जलयः पतञ्जलयः प्रादुर्बभूवुर्येषां जीवन\-कृतं योग\-दर्शनाचार्य\-प्रसङ्गे सङ्क्षेपतो वर्णितम्। यः पाणिनीय\-व्याकरण\-परिश्चिकीर्षया पितृ\-चरणस्य सन्ध्या\-तर्पणार्थं बद्धाञ्जलेरञ्जलौ पतन् पतञ्जलिरिति विश्रुतः। \textcolor{red}{पतन्तो नमस्कार्यत्वेन जनानामञ्जलयो यस्मिन् विषये स पतञ्जलिः}\footnote{\textcolor{red}{“पतञ्जलिरिति” पतन्नञ्जलिर्यस्मिन् नमस्कार्यत्वादिति विग्रहः} (बा॰म॰~७९)। \textcolor{red}{पतन्तोऽञ्जलयोऽस्मिन् नमस्कार्यत्वादिति पतञ्जलिः} (त॰बो॰~७९)। \textcolor{red}{नमस्कार्यत्वेन नमस्कार्यत्वात्} इत्यनयोः \textcolor{red}{विभाषा गुणेऽस्त्रियाम्‌} (पा॰सू॰~२.३.२५) इत्यनेन तृतीया\-पञ्चम्यौ। पूर्वत्रापि \pageref{text:patanjali}तमे पृष्ठे व्युत्पत्तिर्विमृष्टा।} इति वा बहुव्रीहौ शकन्ध्वादि\-गणस्याकृति\-गणत्वात्पर\-रूपम्। इमे शेषावतारा आसन्नित्यास्तिक\-वैयाकरणानां हृदयम्। यथा~–\end{sloppypar}
\centering\textcolor{red}{अशेषफलदातारं भवाब्धितरणे तरिम्।\nopagebreak\\
शेषाशेषार्थलाभार्थं प्रार्थये शेषभूषणम्॥}\nopagebreak\\
\raggedleft{–~वै॰भू॰सा॰ मङ्गलाचरणे २}\\
\begin{sloppypar}\hyphenrules{nohyphenation}\justifying\noindent इति वैयाकरण\-भूषण\-सारे कौण्डभट्टः। भट्टोजिदीक्षितश्च स्वकीय\-वैयाकरण\-सिद्धान्त\-कारिकावल्यामिमं \textcolor{red}{फणि}\-शब्देन सम्बोधयति। यथा~–\end{sloppypar}
\centering\textcolor{red}{फणिभाषितभाष्याब्धेः शब्दकौस्तुभ उद्धृतः। \nopagebreak\\
तत्र निर्णीत एवार्थः सङ्क्षेपेणेह कथ्यते॥}\nopagebreak\\
\raggedleft{–~वै॰सि॰का॰~१}\\
\begin{sloppypar}\hyphenrules{nohyphenation}\justifying\noindent एवमेव हरि\-दीक्षित एनं शब्दरत्ने शेषमेव स्वीकरोति~–\end{sloppypar}
\centering\textcolor{red}{शेषविभूषणमीडे शेषाशेषार्थलाभाय।\nopagebreak\\
दातुं सकलमभीष्टं फलमीष्टे यत्कृपादृष्टिः॥}\nopagebreak\\
\raggedleft{–~श॰र॰ मङ्गलाचरणे १}\\
\begin{sloppypar}\hyphenrules{nohyphenation}\justifying\noindent नागोजिभट्टस्त्वात्मनो नाम \textcolor{red}{नागेश} इति लिखति परिभाषेन्दुशेखरे शब्देन्दुशेखरे च। यथा परिभाषेन्दुशेखरे मङ्गलाचरणं कुर्वन्नागोजिभट्टो लिखति~–\end{sloppypar}
\centering\textcolor{red}{नत्वा साम्बं शिवं ब्रह्म नागेशः कुरुते सुधीः।\nopagebreak\\
बालानां सुखबोधाय परिभाषेन्दुशेखरम्॥}\nopagebreak\\
\raggedleft{–~प॰शे॰~मङ्गलाचरणम्}\\
\begin{sloppypar}\hyphenrules{nohyphenation}\justifying\noindent तत्र \textcolor{red}{नागेश}\-शब्दे बहुव्रीहि\-समासः। \textcolor{red}{नागो नागावतारः पतञ्जलिरीशो यस्य स नागेशः} इति। नागोजिभट्ट आत्मन ईश्वरं पतञ्जलिमेव मन्यते। एवमेव लघु\-शब्देन्दु\-शेखरस्य मङ्गलाचरणे पतञ्जलिं \textcolor{red}{फणीश}\-शब्देन स्तौति। यथा~–\end{sloppypar}
\centering\textcolor{red}{नत्वा फणीशं नागेशस्तनुतेऽर्थप्रकाशकम्। \nopagebreak\\
मनोरमोमार्द्धदेहं लघुशब्देन्दुशेखरम्॥}\nopagebreak\\
\raggedleft{–~ल॰शे॰ म॰~३}\\
\begin{sloppypar}\hyphenrules{nohyphenation}\justifying\noindent तस्य वाल्मीकि\-रामयणे टीकाऽपि नागेश्वरी नाम्ना प्रसिद्धिं गता। लघुत्रयी\-बृहत्त्रय्योः प्रबुद्ध\-टीकाकारो मल्लिनाथोऽपि पातञ्जल\-महाभाष्यं \textcolor{red}{पन्नगगवी}\-शब्देन तुष्टाव।\footnote{कि॰ घ॰व्या॰म॰~४, र॰वं॰ स॰व्या॰म॰~४, शि॰व॰ स॰व्या॰म॰~४।} एवमेव कवि\-तार्किक\-चूडामणिः कल्पना\-कानन\-कण्ठीरवः शब्द\-रचना\-विन्यास\-प्रगल्भः सर्व\-तन्त्र\-स्वतन्त्रः कविता\-कामिनी\-हर्षः श्रीहर्षो भगवन्तं भाष्यकारं शेषमेव मन्यते। कथयति \textcolor{red}{फणि\-भाषित\-भाष्य\-फक्किका\-विषमा कुण्डल\-नामवापिता} (नै॰च॰~२.९५) इति। यः खलु भगवान् श्रीहर्ष एतावदात्मनो वैदुष्यं प्रकटयन् सगर्वं कथयति यत्~–\end{sloppypar}
\centering\textcolor{red}{ग्रन्थग्रन्थिरिह क्वचित्क्वचिदपि न्यासि प्रयत्नान्मया \nopagebreak\\
प्राज्ञंमन्यमना हठेन पठिती माऽस्मिन् खलः खेलतु।\\
श्रद्धाराद्धगुरुश्लथीकृतदृढग्रन्थिः समासादय-\nopagebreak\\
त्वेतत्काव्यरसोर्मिमज्जनसुखव्यासज्जनं सज्जनः॥}\nopagebreak\\
\raggedleft{–~नै॰च॰~२२.१५४}\\
\begin{sloppypar}\hyphenrules{nohyphenation}\justifying\noindent एतादृग्वैदुष्य\-सम्पन्नोऽपि श्रीहर्षो भाष्यकारं शेषं मत्वा प्रपद्यत एतदधिकं किं ब्रूमहे। अधुनाऽपि वाराणस्यां प्रसिद्धो नागकूपो यस्य परिसरे भगवान् भाष्यकारः स्थित्वा सम्पूर्ण\-महा\-भाष्यं बभाषे। साम्प्रतं भाषायां स \textcolor{red}{नागकुआँ} इति कथ्यते। प्रतिवर्षं सर्वेऽपि वाराणसेया विद्वांसो मादृशा विद्यार्थिनश्च श्रावण\-शुक्ल\-पञ्चम्यामाशीर्वाद\-लिप्सया शास्त्रार्थ\-चिकीर्षया च सोत्साहं गुरुभिः सह गच्छन्ति। तत्र च सानन्दं शास्त्रार्थ\-माध्यमेन शास्त्रं गुरुभ्यः शिक्षन्ते शिक्षयन्ति च शिष्यान्। श्रूयते गुरुभ्यो यदस्य नाग\-कूपस्य परिसरे भगवान् पतञ्जलिः प्रायशो भाष्य\-पारायणं प्रोवाच। स च यवनिका\-पट\-खण्डेन मुखमाच्छाद्य पाठयति स्म। महाभाष्यकाराः प्रतिदिवसं यावद्विषयं स्पष्टयन्ति स्म तदेव \textcolor{red}{आह्निकम्} इति कथ्यते। \textcolor{red}{अह्ना निर्वृत्तं जातमाह्निकम्}।\footnote{\textcolor{red}{तेन निर्वृत्तम्} (पा॰सू॰~५.१.७९) इत्यनेन \textcolor{red}{अहन्‌}\-प्रातिपदिकात् \textcolor{red}{ठञ्‌}\-प्रत्ययः। \textcolor{red}{ठस्येकः} (पा॰सू॰~७.३.५०) इत्यनेनेकादेशे \textcolor{red}{अल्लोपोऽनः} (पा॰सू॰~६.४.१३४) इत्यनेनालोपे \textcolor{red}{तद्धितेष्वचामादेः} (पा॰सू॰~७.२.११७) इत्यनेनादि\-वृद्धौ विभक्ति\-कार्ये सिद्धम्।} एवं पञ्चाशीति दिवसान् यावत्पाणिनि\-सूत्राणामुपरि भाष्य\-प्रवचनं चक्रुः। तानि पञ्चाशीत्याह्निकानि जातानि। तानि च भूर्जपत्रेषु कदल्यादिपत्रेषु च लिख्यन्ते स्म।
कुत्रत्यानि\-चित्स्थलानि पत्रेषु लिखितानि विद्यार्थिभिस्तान्यजाभिर्भक्षितानि। अद्याप्यजा\-भक्षित\-स्थलानि चर्चन्तेऽस्मद्गुरु\-परम्परायाम्। गुरवः श्रावितवन्तो यत्कदाचिदेको विद्यार्थी महाभाष्य\-काराणां भाष्य\-प्रवचन\-वेलायां किञ्चित्कार्य\-गौरवात्तानपृष्ट्वा पाठं त्यक्त्वा बहिर्गतः। आगते च तस्मिन् क्रोध\-वशेन भगवता वास्तवं रूपं सहस्र\-शिरो\-युक्तं सहस्र\-फणा\-समुल्लसित\-दिव्य\-कोटि\-कोटि\-सहस्र\-मरीचि\-मालि\-निन्दक\-सहस्र\-मणि\-फण\-प्रकाश\-निरस्त\-निखिल\-भुवन\-तिमिर\-पटलं फूत्कार\-समुच्छलित\-सहस्र\-वदन\-ज्वाला\-जाल\-मात्र\-तिरस्कृत\-प्रलय\-कालानलमनलं भयानकं नारायणांश\-शिरो\-लसित\-रजः\-कणायित\-शैल\-सरित्समुद्र\-कानन\-वृक्ष\-जड\-जङ्गम\-सचराचर\-धर\-धरणि\-गौरवं शेषाख्यं प्रकटयाम्बभूवे। विद्यार्थी च सर्वैश्छात्त्रैः सहानुनिनाय। अनन्तरमुपसंहृत्य रूपमलौकिकं विससर्ज तान् भगवान् विरराम च पाठात्। भगवतः पतञ्जलेर्भाषा सुवासा योषेव कलित\-भाषा विचक्षण\-पोषा विहित\-विपश्चित्सत्तोषा निरस्त\-निखिल\-दोषा दोषाकर\-कौमुदीव मोदयति सज्जन\-चकोर\-निकुरम्बकम्। सरल\-भाषायां गम्भीरतम\-विषयाणां तात्पर्यं यादृक्चतुरतया स्पष्टीकृतं भगवता नूनं तदभूतपूर्वम्। विविध\-दृष्टान्त\-दार्ष्टान्तिक\-व्याज\-भङ्गिम्ना प्रतिपादनं काञ्चने मणिरिव सम्यक्स्थिरतां याति। भाषा\-भाव\-शैली\-प्राञ्जल्यं दृष्ट्वा लगति यद्भगवाननायासं शब्द\-समुद्रमवगाहमानोऽमन्द\-मन्दरतामुपेतः। प्रश्न\-भाष्यमाक्षेप\-भाष्यं सिद्धान्त\-भाष्यं विषये स्वर्ण\-सौरभ\-योगमापादयन्ति। कुत्रचिदैश्वर्य\-मुद्रायां निभ्रान्तं पाणिनिं प्रशंसतो भगवतो वाक्यं विदुषां मनो हरति। यथा \textcolor{red}{सामर्थ्य\-योगान्नहि किञ्चिदत्र पश्यामि शास्त्रे यदनर्थकं स्यात्} (भा॰पा॰सू॰~६.१.७७)। एवमेव \textcolor{red}{वृद्धिरादैच्} (पा॰सू॰~१.१.१) सूत्रे भगवान् साटोपं घोषयति यत्~– \textcolor{red}{प्रमाणभूत आचार्यो दर्भपवित्रपाणिः शुचाववकाशे प्राङ्मुख उपविश्य महता यत्नेन सूत्रं प्रणयति स्म। तत्राशक्यं वर्णेनाप्यनर्थकेन भवितुम्। किं पुनरियता सूत्रेण} (भा॰पा॰सू॰~१.१.१) इति। स्वकीयया प्रतिभया बहुत्र सूत्राणि प्रत्याख्यातानि भगवता। तत्र भगवत इत्थं तात्पर्यं फलं द्विविधं दृष्टमदृष्टञ्च। सूत्राणां लक्ष्य\-सिद्धौ सहायकत्वं दृष्टं फलम्। पुण्य\-जनकता चादृष्टं फलम्। अप्रत्याख्यात\-सूत्राणां द्विविधम्। प्रत्याख्यात\-सूत्राणामदृष्टमात्रं फलम्। अतो यानि प्रत्याख्यातानि तेषां दृष्टं फलं नहि। तदभावेऽपि लक्ष्याणां सिद्धेः। यथा \textcolor{red}{नाज्झलौ} (पा॰सू॰~१.१.१०) इति सूत्रमचां हलाञ्च मिथः सावर्ण्यं निषेधयति। \textcolor{red}{विश्वपाभिः} इत्यत्राऽकारेण हकारस्य सावर्ण्यात्प्राप्त\-ढत्व\-निषेधाय\footnote{ \textcolor{red}{हो ढः} (पा॰सू॰~८.२.३१) इत्यनेन हकारस्य झलि ढत्वं प्राप्तम्।}
\textcolor{red}{हे यियासो} इत्यत्र प्लुताकारेण च हकार\-सावर्ण्य\-वारणाय। दीर्घाकार\-प्लुताकारयोः प्रश्लेषं विधाय दीर्घाकार\-प्लुताकार\-सहितानामचां हलां च न मिथः सावर्ण्यमिति व्यवस्थायामेव \textcolor{red}{दधि हरति दधि शीतलम् दधि षष्ठम् दधि सान्द्रम्} इत्यादौ सावर्ण्याद्यण्दीर्घो न। अन्यथोष्म\-सञ्ज्ञक\-हकार\-शकार\-षकार\-सकाराणां स्वर\-सञ्ज्ञकेकारेण समान\-विवृत\-प्रयत्नत्वात्स्थान\-साम्याच्च सावर्ण्येन दोषो दुर्वार एव। तथा च दीक्षितः \textcolor{red}{तेन दधीत्यस्य हरति शीतलं षष्ठं सान्द्रमित्येतेषु परेषु यणादिकं न। अन्यथा दीर्घादीनामिव हकारादीनामपि ग्रहणक\-शास्त्र\-बलादच्त्वं स्यात्} (वै॰सि॰कौ॰~१३)। दीक्षितादयः प्राचीनाश्चतुर एवाभ्यन्तर\-प्रयत्नान् प्रतियन्ति। अत ऊष्मणां स्वराणाञ्च विवृतमेव प्रयत्नं मानयन्ति। यथा \textcolor{red}{आद्यश्चतुर्धा। स्पृष्टेषत्स्पृष्ट\-विवृत\-संवृत\-भेदात्। तत्र स्पृष्टं प्रयत्नं स्पर्शानाम्। ईषत्स्पृष्टमन्तःस्थानाम्। विवृतमूष्मणां स्वराणां च। ह्रस्वस्यावर्णस्य प्रयोगे संवृतम्। प्रक्रिया\-दशायां तु विवृतमेव} (वै॰सि॰कौ॰~१०) इति। एवं यदा चत्वारः प्रयत्नाः स्वीक्रियन्ते तदा \textcolor{red}{नाज्झलौ} (पा॰सू॰~१.१.१०) इत्यस्याऽवश्यकता। यदा चेषद्विवृतमूष्मणां विवृतञ्च स्वराणामिति प्रयत्न\-भेदः क्रियते तदा हकारादीनामिकारादिभिः प्रयत्न\-वैषम्यात्सावर्ण्याभावेन दीर्घादीनामप्रसक्त्या तन्निषेधार्थं नाज्झलावित्यलब्ध\-लौकिक\-फलतया प्रत्याख्यायि भगवता भाष्यकृता।
अदृष्टं तु फलमस्त्येवेति कृत्वा प्रत्याख्यातं किन्तु निष्कासितं न सूत्रपाठात्। अत एव कस्यचिदपि वर्णस्य व्यर्थता नहि स्वीकृता भगवता। इममेव पन्थानमनुसरद्भिः श्री\-वरद\-राजाचार्यैर्लघुसिद्धान्त\-कौमुद्यां पञ्च\-प्रयत्नता स्वीकृता। यथा \textcolor{red}{आद्यः पञ्चधा स्पृष्टेषत्स्पृष्टेषद्विवृत\-विवृत\-संवृत\-भेदात्। ईषद्विवृतमूष्मणाम्। विवृतं स्वराणाम्} (ल॰सि॰कौ॰~१०) इति। \end{sloppypar}
\begin{sloppypar}\hyphenrules{nohyphenation}\justifying\noindent\hspace{10mm} अस्मिन् प्रक्रिया\-लाघवं कार्य\-सिद्धिश्च बहुत्रान्य\-प्रकारैरपि दर्शयन्तः साधनोपायानपाणिनीय\-भीत्या श्रद्धावनत\-मस्तकाः पुनः प्राञ्चः प्राञ्जलयः प्रार्थयन्ते पतञ्जलयो यच्छक्यमेवमपि वक्तुं किन्तु \textcolor{red}{अपाणिनीयं तु भवति} (भा॰प॰, भा॰पा॰सू॰~१.१.३)। वीक्ष्यतां कीदृशी श्रद्धा भगवति पाणिनौ भगवतां भाष्यकाराणाम्। कात्यायनस्य बहुत्र त्रुटि\-संशोधनात्मक\-वार्त्तिक\-प्रयासो द्वेष\-ग्रस्त\-धियैवेति भाष्यकाराणां मनीषितम्। पस्पशाह्निके यथा लौकिकवैदिकेषु इति वार्त्तिकं भाषयन्तो भाषन्ते भाष्यकृतो यत् \textcolor{red}{प्रिय\-तद्धिता दाक्षिणात्याः। “यथा लोके वेदे च” इति प्रयोक्तव्ये “यथा लौकिक\-वैदिकेषु” इति प्रयुञ्जते} (भा॰प॰) इति। \textcolor{red}{संयोगान्तस्य लोपः} (पा॰सू॰~८.२.२३) इति सूत्रेण \textcolor{red}{सुद्ध्य् उपास्य} इत्यादौ संयोगान्तस्य यणो लोपो मा भूदिति \textcolor{red}{यणः प्रतिषेधो वाच्यः} वचनमिदमुपन्यस्तं वार्त्तिक\-कृता। एतस्य भाष्यकारः प्रत्याख्यानं करोति यत् \textcolor{red}{झलो झलि} (पा॰सू॰~८.२.२६) इति सूत्रात् \textcolor{red}{झलः} इति पञ्चम्यन्तं पदं षष्ठ्या विपरिणम्यात्राप\-कर्षणीयम्।\footnote{\textcolor{red}{न वा वक्तव्यम्। किं कारणम्। झलो लोपात्। झलो लोपः संयोगान्त\-लोपो वक्तव्यः} (भा॰पा॰सू॰~८.२.२३)। \textcolor{red}{न वेति। “झलो झली”त्यतः सिंहाव\-लोकित\-न्यायेन झल्ग्रहणमिहानु\-वर्तते। तत्षष्ठ्या विपरिणम्यत इति यणो लोपाभावः} (भा॰प्र॰ पा॰सू॰~८.२.२३)।} एवं च झलः संयोगान्तस्य लोप इत्यर्थे यणो लोपः सुतरामसम्भवस्तदर्थं वचनारम्भो व्यर्थः। स्वकीय\-दृष्टान्त\-प्रवाह\-प्रसङ्गे मनन\-शीलेनानेन महा\-मुनिना समग्रा भारतीय\-संस्कृति\-वाङ्मयी दर्शिता। धर्मशास्त्र\-राजशास्त्र\-नीतिशास्त्रार्थशास्त्र\-वेदान्त\-लोक\-व्यवहारादि\-मानव\-जीवनोपयोगि\-निगूढ\-विषयाणां मञ्जुलं चित्रणं दृश्यते कृतम्। दृष्टान्तेषु शास्त्र\-प्रतिपादन\-व्याजेन शिक्षाऽप्यतिचतुरतया प्रत्ता। यथा पस्पशाह्निके \textcolor{red}{समानश्च खेद\-विगमो गम्यायां चागम्यायां च। तत्र नियमः क्रियते। इयं गम्येयमगम्येति} (भा॰प॰)। एवं यथा \textcolor{red}{यो ह्यजानन् वै ब्राह्मणं हन्यात्सुरां वा पिबेत्सोऽपि मन्ये पतितः स्यात्} (भा॰प॰) इति।\end{sloppypar}
\begin{sloppypar}\hyphenrules{nohyphenation}\justifying\noindent\hspace{10mm} भगवतां भाष्यकृतां भगवती भागीरथीव भास्वती भासुरा भाषा विषयाणां सुस्पष्ट\-प्रतिपादनं भाव\-व्यक्तीकरण\-सामर्थ्यं निसर्ग\-सिद्ध\-प्रवाहो वाचस्पति\-मति\-रञ्जने गहनतमा विचार\-सरणिः कुशाग्र\-तीव्र\-प्रतिभैकैकस्य प्रश्नस्यानेकान्युत्तराणि स्वस्थो भाव\-बोध\-प्रकारः सिंहवद्विक्रान्ति\-युक्ति\-प्रदर्शनमिदं सर्वमलौकिकमेव।\end{sloppypar}
\centering\textcolor{red}{अन्यानि सन्ति भाष्याणि आचार्यैर्विहितानि वै।\nopagebreak\\
महाभाष्यमिदं प्रोक्तं दिव्यं शेषेण धीमता॥}\nopagebreak\\
\raggedleft{–~इति मम}\\
\begin{sloppypar}\hyphenrules{nohyphenation}\justifying\noindent अन्यान्य\-भाष्याणामपेक्षयाऽस्य वैशिष्ट्यं यदन्यानि सूत्र\-व्याख्यानानि भूतानीदं परिष्कृत\-सूत्र\-व्याख्यानं सदपीष्ट्यादिना महत्त्व\-पूर्णम्। क्वचित्क्वचिद्भगवता स्वतन्त्राऽपीष्टिर्दत्ता। यथा \textcolor{red}{त्यदादीनामः} (पा॰सू॰~७.२.१०२) इदं सूत्रं विभक्तौ परतस्त्यदादीनामकारान्तादेश\-विधायकमिति पाणिनि\-मतं त्यदमारभ्य। किमन्ताः सन्ति त्यदादयः किं पर्यन्तं चेदकारान्तादेशस्तदा \textcolor{red}{युष्मत् अस्मत् भवत्} इत्यादावकारान्तादेशे कृते \textcolor{red}{युवाम् आवाम् भवन्तौ} इत्यादिरूपाणि न सिद्धानि स्युः। अतो भगवतेष्टिर्दत्ता~– \textcolor{red}{द्विपर्यन्तानामेवेष्टिः} इति।\footnote{\textcolor{red}{तस्माद्द्वि\-पर्यन्तानामत्त्वं वक्तव्यम्} (भा॰पा॰सू॰~७.२.१०२)। भैमीकारा अपि~– \textcolor{red}{इसकी अवधि भाष्यकार ने ‘द्वि’ शब्द पर्यन्त नियत की है} (ल॰सि॰कौ॰ भै॰टी॰~१९३)। केषुचित्संस्करणेषु \textcolor{red}{द्विपर्यन्तानामेवेष्टिः} इति वार्त्तिकम्।} बहुत्र कल्पना\-बलेन संसाध्य सूत्र\-प्रयोग आर्ष\-दृष्ट्या तेषां लोकेऽनभिधानं पश्यन् स्वत एव व्यरंसीन्निरर्थक\-बुद्धि\-विलासात्। अत एव महत्त्वञ्चेष्ट्यादिना भाष्य\-प्रदीप\-टीकायां कैयटेन लिखितम्। कैयटः स्वयं कथयति~–\end{sloppypar}
\centering\textcolor{red}{भाष्याब्धिः क्वातिगम्भीरः क्वाहं मन्दमतिस्ततः।\nopagebreak\\
छात्त्राणामुपहास्यत्वं यास्यामि पिशुनात्मनाम्॥}\nopagebreak\\
\raggedleft{–~भा॰प्र॰ मङ्गलाचरणे~६}\\
\begin{sloppypar}\hyphenrules{nohyphenation}\justifying\noindent काशिकायां पूर्व\-वर्तिनां पाणिनि\-कात्यायान\-पतञ्जलीनां कृते मुनि\-सञ्ज्ञा कथिता लोकान्धकार\-निनाशयिषया।\footnote{का॰वृ॰~२.१.१९।} \textcolor{red}{यथोत्तरं मुनीनां प्रामाण्यम्}। तस्मात् \textcolor{red}{त्रिमुनि} व्याकरणं कथ्यते। अन्तिमं प्रामाण्यं महाभाष्यकृतामेव। इत्थं शिवप्रेरणतया पाणिनि\-भाषितस्य कात्यायन\-पतञ्जलि\-परिष्कृतस्यैतस्य लौकिक\-वैदिक\-शब्द\-साधुत्व\-परायणस्य पाणिनीय\-व्याकरणस्य द्वे नेत्रे प्रक्रिया दर्शनञ्च। प्रक्रियायां त्रयाणां मुनीनामनन्तरं मुख्या आचार्याः काशिकाकाराः कैयटोपाध्याया वृत्तिकारा भट्टोजिदीक्षित\-महाभागा वरदराजाचार्या नागोजिभट्ट\-महा\-भागाश्च येषु भट्टोजिदीक्षित\-प्रक्रिया\-प्रकारः सरलः सुष्ठुर्लोके चलितश्च। अनेन प्रक्रिया\-प्रकरणमनुसृत्य सूत्राणि सङ्कलितानि। यथा प्रथमं सन्धिः शब्दानां पश्चात् षड्लिङ्ग\-विभक्ति\-रूपाणि स्त्री\-प्रत्यया एतत्पर्यन्तं शब्द\-विवेचनम्। पुनर्वाक्यार्थं कारक\-निर्देशः समास\-वर्णनं तद्धितीय\-प्रयोग\-दिग्दर्शनं धातु\-प्रक्रिया\-निर्देशः कृदन्त\-निर्देशश्चेति। प्रक्रियायां टीका\-ग्रन्थ एतस्य प्रौढ\-मनोरमा। कुत्रचिन्नागेश एतन्मतं विरुणद्धि। अनुबन्ध\-विषयेऽयमित्सञ्ज्ञकत्वमनुबन्धत्वं मन्यते नागेशश्चेत्सञ्ज्ञा\-योगत्वमनुबन्धत्वं स्वीकरोति।
अन्तिम आचार्यो भगवान्नागेशः। प्रधानतया परिभाषेन्दुशेखर\-लघुशब्देन्दुशेखरौ प्रक्रिया\-प्रकार\-परिष्कारकौ। श्रूयते यन्नागेशभट्टः पुरा कुब्ज आसीत्कदाचिद्बाल\-स्वभावतयोच्चासनमधिश्रितश्चरण\-ताडन\-पुरःसरं तिरस्कृतः। ग्लानि\-खिन्न\-चेतास्त्रिरात्रेण वागीश्वरीमाराध्य वर्ष\-त्रयेण सकल\-शास्त्रं समधिगम्य पण्डित\-चक्र\-चूडामणिर्जातो नागेशः। विवाहे सति शास्त्र\-चिन्तन\-तन्मयतया भोग\-वासनातो विरतः सन्तानार्थं समभ्यर्थितो भार्यया सरल\-भावेनोत्तरयन्नाह नागेशो यत् \textcolor{red}{पुत्र्यश्चैता हि मञ्जूषाः पुत्रौ चैतौ हि शेखरौ}।\footnote{\textcolor{red}{शब्देन्दुशेखरः पुत्रो मञ्जूषा चैव कन्यका। स्वमतौ सम्यगुत्पाद्य शिवयोरर्पितौ मया॥} इत्यपि तेनोक्तं शब्देन्दु\-शेखरे।} गुरु\-चरणा वदन्ति यत्कुड्ये गर्तं कृत्वा तस्मिन्निवेश्य कुब्ज\-भागं लिखन्ति स्म ग्रन्थानाचार्या नागेशा अहो। स्वशरीर\-चर्म\-निर्मित\-पदत्राण\-समर्पणेनाऽपि वयं किममीषां महामहिम\-त्यागशीलानां प्रत्युपकारं कर्तुं क्षमेमहि। कदाचिद्राम\-सिंहेन शृङ्गपुराधीशेन पृष्टो न्यूनतार्थं बहुशो ग्रन्थ\-समागता अनुपपत्तीरेव दर्शयति। ईदृशं शास्त्र\-व्यसनम्। तेभ्यः परं परिष्कार\-प्रक्रिया\-शास्त्रार्थे व्याकरणस्य न्याय\-वासनया वैशिष्ट्यादि\-धारा\-मयी भ्रामक\-जटिल\-विशाल\-शब्द\-जाल\-युक्ता पण्डित\-मनोरमा परम्परा प्रावर्तत यस्यां गण्यन्ते श्री\-शिवकुमार\-शास्त्रि\-दामोदर\-शास्त्रि\-बाल\-शास्त्रि\-तात्य\-शास्त्रि\-जयदेव\-मिश्रास्मद्गुरु\-चरण\-प्रभृतयः।\end{sloppypar}
\centering\textcolor{red}{पाणिन्यादि मुनीन्नत्वा आचार्यान् स्वगुरूंस्तथा।\nopagebreak\\
साञ्जलिर्याचते दिव्यां मतिं गिरिधरः शिशुः॥}\nopagebreak\\
\raggedleft{–~इति मम}\\
\begin{sloppypar}\hyphenrules{nohyphenation}\justifying\noindent\hspace{10mm} अथ व्याकरणस्य दर्शन\-रूप\-नेत्र\-विषये सङ्क्षेपतश्च चर्चयामः। समस्त\-वैयाकरण\-सिद्धान्तानां मूल\-भूतं तु पाणिनि\-व्याकरणमेव। अष्टाध्याय्येको महान् रत्नाकरो यस्मिन्ननेकान्युपयोगीनि रत्नान्यनायासं समुपलब्धुं शक्यन्तेऽहो। विधीयमाने विचारेऽष्टाध्यायी\-रूपो महा\-सागरः क्षीर\-सागरमप्यतिशेते। क्षीर\-सागरे चतुर्दश रत्नान्यत्रापि माहेश्वर\-सूत्र\-रूप\-चतुर्दश\-रत्नानि। तत्र सुन्दराण्यमृतादीन्यभद्राणि विष\-वारुणी\-प्रभृतीनि रत्नानि राजन्ते। अत्र तु सर्वाण्यपि प्रकृति\-रमणीयानि। अत्र महा\-सागरे जल\-स्थानीयं शब्द\-ब्रह्म विचकास्ति। गम्भीरतमानामस्ति दर्शन\-सिद्धान्तानां समुपबृंहणम्। सम्पूर्णान्यपि दर्शनानि रत्नानीव विराजमानानि दिव्यां सुषमामञ्चन्ति। वेदान्तिनां यथाऽद्वैत\-वादस्तथैव वैयाकरणानां शब्द\-विशिष्टाद्वैत\-वादः। वैयाकरणाः शब्दमेव ब्रह्म मन्यन्ते। तत्र शब्देऽद्वैतम्। द्वाभ्यां भेद\-प्रतिपादकाभ्यामितं प्राप्तं ज्ञानं द्वीतं तस्य भावो द्वैतम्। न द्वैतमित्यद्वैतम्। शब्द\-विशिष्टमद्वैतं शब्द\-विशिष्टाद्वैतम्। शब्द\-विशिष्टाद्वैतं ब्रह्म वैयाकरणानां प्रबला मान्यता। यद्यपि नैयायिकाः शब्दमनित्यं मन्यन्ते किन्तु वैयाकरणा नित्यमेव शब्दमुद्घोषयन्ति। किं बहुना शब्दतो निखिल\-प्रपञ्चस्योद्भवं मन्यन्ते यथा~–\end{sloppypar}
\centering\textcolor{red}{अनादिनिधनं ब्रह्म शब्दतत्त्वं यदक्षरम्।\nopagebreak\\
विवर्ततेऽर्थभावेन प्रक्रिया जगतो यतः॥}\nopagebreak\\
\raggedleft{–~वा॰प॰~१.१}\\
\begin{sloppypar}\hyphenrules{nohyphenation}\justifying\noindent आदि\-निधन\-रहितं यदक्षरं शब्द\-ब्रह्म यतो जगतः प्रक्रिया तदेवार्थ\-भावेन विवर्तते। \textcolor{red}{अनादि\-निधनम्} इति कथयित्वाऽपि \textcolor{red}{अक्षरम्} इति कथयन् ब्रह्मणोऽक्षर\-शीलतां प्रतिपादयन् ब्रह्मणो व्यापकतां संस्तौति। \textcolor{red}{न क्षरतीति अक्षरम्}।\footnote{क्षरतीति क्षरम्। \textcolor{red}{क्षरँ सञ्चलने} (धा॰पा॰~८८१)~\arrow क्षर्~\arrow\textcolor{red}{ नन्दि\-ग्रहि\-पचादिभ्यो ल्युणिन्यचः} (पा॰सू॰~३.१.१३४)~\arrow क्षर् अच्~\arrow क्षर् अ~\arrow क्षर~\arrow विभक्तिकार्यम्~\arrow क्षरम्। न क्षरमित्यक्षरम्। \textcolor{red}{नञ्‌} (पा॰सू॰~२.२.६) इत्यनेन तत्पुरुष\-समासे \textcolor{red}{नलोपो नञः} (पा॰सू॰~६.३.७३) इत्यनेन नलोपे \textcolor{red}{अक्षर} इति प्रातिपदिके जाते विभक्तिकार्ये सौ \textcolor{red}{अतोऽम्} (पा॰सू॰~७.१.२४) इत्यनेनामि \textcolor{red}{अमि पूर्वः} (पा॰सू॰~६.१.१०७) इत्यनेन पूर्वरूपे \textcolor{red}{अक्षरम्} इति सिद्धम्।} अथवा \textcolor{red}{अश्नुते सम्पूर्णं जगदिदं व्याप्नोति} इत्यर्थे \textcolor{red}{अशेः सरन्} (प॰उ॰~३.७०) इत्युणादि\-सूत्रेण \textcolor{red}{अश्‌}\-धातोः (\textcolor{red}{अशँ भोजने} धा॰पा॰~१५२३) \textcolor{red}{सरन्‌}\-प्रत्यये षत्वे कत्वे षत्वे चाक्षरमिति।\footnote{\textcolor{red}{अशँ भोजने} (धा॰पा॰~१५२३)~\arrow अश्~\arrow \textcolor{red}{उणादयो बहुलम्} (पा॰सू॰~३.३.१)~\arrow \textcolor{red}{अशेः सरन्} (प॰उ॰~३.७०)~\arrow अश् सरन्~\arrow अश् सर~\arrow \textcolor{red}{व्रश्च\-भ्रस्ज\-सृज\-मृज\-यज\-राज\-भ्राजच्छशां षः} (पा॰सू॰~८.२.३६)~\arrow अष् सर~\arrow \textcolor{red}{षढोः कः सि} (पा॰सू॰~८.२.४१)~\arrow अक् सर~\arrow \textcolor{red}{आदेश\-प्रत्यययोः} (पा॰सू॰~८.३.५९)~\arrow अक् षर~\arrow अक्षर~\arrow विभक्तिकार्यम्~\arrow अक्षरम्।} \textcolor{red}{प्रक्रिया} इति प्रकृष्टा क्रिया। प्रकृष्टत्वञ्च सर्ग\-स्थिति\-विनाश\-रूपम्। \textcolor{red}{यतः} इत्यत्र सार्वविभक्तिकस्तसिः।\footnote{\setcounter{dummy}{\value{footnote}}\addtocounter{dummy}{-1}\refstepcounter{dummy}\label{fn:yatah}\textcolor{red}{इतराभ्योऽपि दृश्यन्ते} (पा॰सू॰~५.३.१४) इत्यनेन। यद्वा \textcolor{red}{तसि\-प्रकरण आद्यादिभ्य उपसङ्ख्यानम्} (वा॰~५.४.४४) इत्यनेन। तसिप्रत्ययान्तानामाकृति\-गणत्वम्। यथा \textcolor{red}{आकृतिगणश्चायम्} (का॰वृ॰~५.४.४४) \textcolor{red}{आकृतिगणोऽयम्} (वै॰सि॰कौ॰~२११२)। न चास्मिन् वार्तिके \textcolor{red}{प्रतियोगे पञ्चम्यास्तसिः} (का॰वृ॰~५.४.४४) इत्यतः \textcolor{red}{पञ्चम्याः} इत्यनुवृत्तम्। \textcolor{red}{अयं सार्वविभक्तिकस्तसिः} (बा॰म॰~२११२) इति बाल\-मनोरमायां वासुदेव\-दीक्षिताः। यथा \textcolor{red}{आदौ आदितः। मध्यतः। पार्श्वतः। पृष्ठतः} (का॰वृ॰~५.४.४४) इत्यादौ सप्तम्यां तसिः। \textcolor{red}{स्वरेण स्वरतः। वर्णतः} (वै॰सि॰कौ॰~२११२) इत्यादौ तृतीयायां तसिः।} अर्थात् \textcolor{red}{यतः} इति पञ्चमी\-तृतीया\-सप्तमी\-विभक्तिषु व्याख्यातुं शक्यते। यतो जगतो जन्म यस्मादिति तात्पर्यं यस्मादुत्पत्तिर्येन पालनं यस्मिल्लँयः। \textcolor{red}{यतो वा इमानि भूतानि जायन्ते येन जातानि जीवन्ति यत्प्रयन्त्यभिसंविशन्ति} (तै॰उ॰~३.१.१) यस्मिल्लीँयन्ते वेति श्रुतेः। तदेव जन्म\-मरण\-रहितमतिशय\-वर्धन\-शीलं सर्व\-व्यापकं शब्द\-ब्रह्म \textcolor{red}{अर्थ\-भावेन} अर्थानां पदार्थानां शक्त्या घट\-पटादि\-रूपेण \textcolor{red}{विवर्तते} विवृतं भवति परिणमतीति तात्पर्यम्। श्लोकेऽस्मिन् गुरु\-चरणा विवर्तत इति पदानुसारं व्याकरण\-सिद्धान्ते विवर्त\-वादं निश्चिन्वन्ति। ब्रह्मणो जगद्रूप\-परिवर्तने दार्शनिक\-जगति धारा\-द्वयी। नैयायिका जगद्ब्रह्म\-परिणाम\-भूतं स्वीकुर्वन्ति। अद्वैत\-वादिनो वेदान्तिनश्च ब्रह्मणो विवर्तं जगन्मन्यन्ते। परिणाम\-वादः प्रकृतेर्विकृति\-रूपेण सतात्त्विक\-परिवर्तनं यथा दुग्धस्य परिणामो दधि। विवर्त\-वादो वस्तुनोऽतात्त्विक\-परिवर्तनं यथा रज्जौ सर्पः। अद्वैत\-वादिनां मते जगदसदिति हेतोस्ते ब्रह्मणोऽन्यथा\-भावरूपं जगत्स्वीकुर्वन्ति। ते कथयन्ति ब्रह्म सत्यं जगन्मिथ्या। द्वैत\-वाद\-प्रतिपादक\-श्रुतीस्तेऽर्थवाद\-रूपेण स्वीकुर्वन्ति। तर्कतः सिद्धान्तितोऽपि सिद्धान्तोऽयं हृदय\-वीणा\-तन्त्रीं न स्पृशति। प्रत्यक्ष\-जगतोऽसत्त्वेन लापनं कर्तुं न शक्यते। वेदे तस्थिवच्छब्देन जगद्व्यवह्रियते। यथा \textcolor{red}{सूर्य॑ आ॒त्मा जग॑तस्त॒स्थुष॑श्च} (शु॰य॰वा॰मा॰~७.४२)।
एवमेव जीव\-ब्रह्मणोरैक्यं प्रतिपादयन्ति तदप्यसमीचीनं लगति। ब्रह्मविद्ब्रह्मैव भवतीति यदि तेषां मनीषा तर्हि कथं घटज्ञो न घटो भवति। यदि चेदात्मा परमात्मैव तर्हि कथं नित्य\-चेतन\-घने सर्व\-तन्त्र\-स्वतन्त्रेऽपरिच्छिन्ने परम\-प्रकाशे विशुद्ध\-ज्योतिषि निर्विकल्पे ब्रह्मण्यसत्य\-माया\-प्रसरः। कथं तस्मिन्नज्ञानं किमनेक\-योजन\-मण्डलं नक्षत्राखण्डलं भुवन\-भास्करं तिमिरता वृणुयात्। आत्मैव परमात्मा चेत्स्वयमेव गुरूणां गुरुस्तर्हि कः शिष्यः कं गुरुं गच्छेत्पठितुम्। कथं वा कोऽपि कामी लोभी क्रोधी किं निष्कलुषं विमल\-बोधमखण्डं ब्रह्म कामादयो मलिनयितुं क्षमन्ते। व्यावहारिकं तदिति चेदलं निरर्थक\-व्यवहार\-प्रपञ्चेन। \textcolor{red}{नित्यो नित्यानां चेतनश्चेतनानामेको बहूनां यो विदधाति कामान्} (श्वे॰उ॰~६.१३) इति श्रुतेः \textcolor{red}{ममैवांशो जीव\-लोके जीव\-भूतः सनातनः} (भ॰गी॰~१५.७) इति स्मृतेश्च का गतिः। तस्माद्विषयान्तरीय\-चर्चां समाप्य प्रकृत इदमेव कथनं पर्याप्तं यज्जीवो ब्रह्मणोंऽश\-भूतः स च सनातनः। अत एव गीतायाम्~–\end{sloppypar}
\centering\textcolor{red}{ममैवांशो जीवलोके जीवभूतः सनातनः।\nopagebreak\\
मनःषष्ठानीन्द्रियाणि प्रकृतिस्थानि कर्षति॥}\nopagebreak\\
\raggedleft{–~भ॰गी॰~१५.७}\\
\begin{sloppypar}\hyphenrules{nohyphenation}\justifying\noindent एकता च तयोः सम्बन्ध\-निबन्धनात्मिका यथा वाल्मीकीये \textcolor{red}{राम\-सुग्रीवयोरैक्यं देव्येवं समजायत} (वा॰रा॰~५.३५.५२)। उभावपि नित्यौ। इदमेव व्याकरणमतमपि प्रतिभाति मे। परिणाम\-वाद\-विवर्त\-वाद\-परिभाषानुसारं विवर्तत इति नास्ति। विवर्तत इत्यस्यार्थो निर्विशेषमपि नाम\-रूपात्मकतया विशिष्टं सद्वर्तते पर\-व्यूह\-विभवान्तर्याम्यर्चा\-रूपेणेति तात्पर्यम्। परिभाषा च परिणाम\-विवर्त\-वादयोरित्थम्~–\end{sloppypar}
\centering\textcolor{red}{सतत्त्वतोऽन्यथाप्रथा विकार इत्युदीरितः। \nopagebreak\\
अतत्त्वतोऽन्यथाप्रथा विवर्त इत्युदीरितः॥}\nopagebreak\\
\raggedleft{–~वे॰सा॰~१३८}\\
\begin{sloppypar}\hyphenrules{nohyphenation}\justifying\noindent तेषामद्वैत\-वादिनां ब्रह्म तु निर्विशेषः। किन्त्वस्मद्ब्रह्म शब्दत्व\-विशिष्टम्। अतस्तेषां ब्रह्म\-तत्त्वमस्माकं शब्द\-तत्त्वम्। तद्ब्रह्मणि निर्धर्मिताऽस्मद्ब्रह्मणि च शब्द\-धर्मिता। तत्रैकमेव नित्यमत्र शब्दार्थ\-सम्बन्धास्त्रयोऽपि नित्याः। अतो वाक्यपदीये \textcolor{red}{नित्याः शब्दार्थसम्बन्धाः} (वा॰प॰~१.२३) इति। जगदपि सम्बन्ध\-कल्पना\-दृष्ट्याऽसत्यं तेषां क्षण\-भङ्गुरत्वादर्वाचीन\-स्वीकृतत्वाच्च। किन्त्वस्माकं जगदीश्वर\-सम्बन्ध\-दृष्ट्या सत्यम्। अतः \textcolor{red}{ममैवांशो जीव\-लोके} (भ॰गी॰~१५.७) इत्यत्रांश\-शब्दो न विभाग\-परः। अन्यथाऽखण्डे ब्रह्मणि सखण्डतापत्तिः। महाकाशे घटाकाशमिवोपाध्यवच्छिन्नतयांऽशः कल्पित इति चेन्न। \textcolor{red}{जीव\-भूतः सनातनः} (भ॰गी॰~१५.७) इति सनातन\-शब्दस्य जीव\-नित्यता\-प्रतिपादकत्वात्कल्पने मानाभावाच्च। \textcolor{red}{नित्यो नित्यानां} (श्वे॰उ॰~६.१३) इति श्रुतेश्च स्वारस्याज्जीवस्य नित्यताऽनेकता च निर्विवादा। तस्मादंश\-शब्दोऽत्र पुत्र\-परः। \textcolor{red}{अमृतस्य पुत्राः} (श्वे॰उ॰~२.५) इति श्रुतेः। एवं शब्दोऽर्थः सम्बन्धश्चास्माकं मते नित्यः।
अतो नित्य\-ब्रह्मणो परिणामोऽपि नित्यमेव तस्माज्जगत्। शब्द\-ब्रह्मणः परिणाम\-भूतं जगदिदं वाक्यपदीय\-कारा अपि समर्थयन्ति~–\end{sloppypar}
\centering\textcolor{red}{शब्दस्य परिणामोऽयमित्याम्नायविदो विदुः। \nopagebreak\\
छन्दोभ्य एव प्रथममेतद्विश्वं प्रवर्तते ॥}\nopagebreak\\
\raggedleft{–~वा॰प॰~१.१२४}\\
\begin{sloppypar}\hyphenrules{nohyphenation}\justifying\noindent शब्दो नित्यः। \textcolor{red}{सिद्धे शब्दार्थ\-सम्बन्धे} (भा॰प॰) इति वार्त्तिकेऽपि सिद्ध\-शब्दो नित्य\-पर्यायः। नित्य\-शब्दञ्च \textcolor{red}{त्यब्नेर्ध्रुव इति वक्तव्यम्} (वा॰~४.२.१०४) इति वार्त्तिकमपि\footnote{सिद्धान्त\-कौमुदी\-संस्करणेषु लघु\-सिद्धान्त\-कौमुदी\-संस्करणेषु वार्त्तिकमिदम्। केषुचिन्महाभाष्य\-संस्करणेषु \textcolor{red}{अमेहक्वतसित्रेभ्यः} इति परिगणनानन्तरं भाष्य\-वचनमिदम्।} \textcolor{red}{ध्रुव} इत्यर्थे
\textcolor{red}{नि}\-अव्ययात् \textcolor{red}{त्यप्}\-प्रत्ययतया ध्रुवार्थं साधयति। शब्दो निरन्तरं ध्रुवो वेदोद्भवत्वात्। शब्द\-नित्यतायां त्रयाणामपि मुनीनां सम्मतिः। नित्यत्वञ्च \textcolor{red}{ध्वंस\-भिन्नत्वे सति ध्वंसाप्रतियोगित्वम्} (त॰स॰ प॰व्या॰~१०)। यथा ध्वंसस्य प्रतियोगिनो घटादयोऽनित्या अप्रतियोगि ब्रह्म नित्यम्। नैयायिकानां मते शब्दो गुणः।\footnote{\textcolor{red}{श्रोत्रग्राह्यो गुणः शब्दः} (त॰स॰~३३)।}
अस्मन्मतेऽगुणोऽद्रव्यं शब्दः।\footnote{\textcolor{red}{अथ गौरित्यत्र कः शब्दः। किं यत्तत्सास्ना\-लाङ्गूल\-ककुद\-खुर\-विषाण्यर्थरूपं स शब्दः। नेत्याह। द्रव्यं नाम तत्। ... यत्तर्हि तच्छुक्लो नीलः कपिलः कपोत इति स शब्दः। नेत्याह। गुणो नाम सः} (भा॰प॰)।} द्रव्यञ्च \textcolor{red}{सत्त्वम्} इति कथ्यते।\footnote{\textcolor{red}{द्रव्यासुव्यवसायेषु सत्त्वम्} (अ॰को॰~३.३.२१३)।} यथा \textcolor{red}{चादयोऽसत्त्वे} (पा॰सू॰~१.४.५७)।\footnote{\textcolor{red}{सत्त्वमिति द्रव्यमुच्यते} (का॰वृ॰~१.४.५७)। \textcolor{red}{अद्रव्यार्थाश्चादयो निपातसञ्ज्ञाः स्युः} (वै॰सि॰कौ॰~२०)। \textcolor{red}{‘सत्त्व’शब्देन द्रव्यमुच्यते} (बा॰म॰~२०)। \textcolor{red}{अद्रव्यार्थाश्चादयो निपाताः स्युः} (ल॰सि॰कौ॰~५३)।} \textcolor{red}{सीदतस्तिष्ठतो लिङ्ग\-सङ्ख्ये यस्मिन् तल्लिङ्ग\-सङ्ख्यान्वयि द्रव्यम्}।\footnote{\textcolor{red}{लिङ्गसङ्ख्याकारकान्वितं द्रव्यम्} (बा॰म॰~२०)। \textcolor{red}{लिङ्गसङ्ख्यान्वितं द्रव्यम्} (त॰बो॰~२०)।} शब्द\-नित्यत्व\-पक्षे धातु\-प्रातिपदिक\-प्रकृति\-प्रत्यय\-विभाग\-तत्तदर्थ\-विभाग\-कल्पनाऽपि सर्वा निर्मूला बाल\-बोधनाय कल्पिता। परमार्थतस्तु वाक्य\-स्फोट एव। तस्मात् \textcolor{red}{वाक्य\-स्फोटोऽति\-निष्कर्षे तिष्ठतीति व्यवस्थितिः} (वै॰सि॰का॰~५९)।\footnote{\textcolor{red}{व्यवस्थितिः} इति \textcolor{red}{मतस्थितिः} इत्यस्य पाठभेदः। वैयाकरण\-सिद्धान्त\-कारिकाः (ख्रिस्ताब्दः १९०१), आनन्दाश्रम\-मुद्रणालयः, पुण्याख्य\-पत्तनम्, ५६तमे पृष्ठे।} अत एव भगवान् भाष्यकार इकः स्थाने यणित्याद्येकदेश\-विकारेषु 
सत्सु शब्द\-नित्यतानुपपत्तिमाशङ्क्य स्थानिनि सर्वपदादेशं प्रतिजानीते। लिखति च~–\end{sloppypar}
\centering\textcolor{red}{सर्वे सर्वपदादेशा दाक्षीपुत्रस्य पाणिनेः।\nopagebreak\\
एकदेशविकारे हि नित्यत्वं नोपपद्यते॥}\nopagebreak\\
\raggedleft{–~भा॰पा॰सू॰~१.१.२०, ७.१.२७}\\
\begin{sloppypar}\hyphenrules{nohyphenation}\justifying\noindent अत इग्घटित\-स्थाने यण्घटितो बोध्यः स च प्रयोक्तव्यः स च साधुरित्येव तात्पर्यं निर्दिशन्ति। शब्दं परम\-प्रमाणतया वयं मन्यामहे निराकारमपि ब्रह्म शब्दाकारतया वयं समर्थयामहे।\end{sloppypar}
\centering\textcolor{red}{अद्वैतास्तु निराकारं नराकारञ्च द्वैतिनः।\nopagebreak\\
वैयाकरणा वयं ब्रह्म शब्दाकारमुपास्महे॥}\nopagebreak\\
\raggedleft{–~इति मम}\\
\begin{sloppypar}\hyphenrules{nohyphenation}\justifying\noindent निराकार\-वादिनां ब्रह्म मनसा दृश्यते साकार\-वादिनां नयन\-चरतामाटीकते शब्दाकार\-वादिनामस्माकं वैयाकरणानां ब्रह्म तु श्रवण\-गोचरतामापन्नं द्वाभ्यां श्रवणाभ्यां शब्दतनु तनोति हृदयम्। लिखितं श्रीराम\-चरित\-मानसे यथा त्रयोविंशति\-सहस्र\-वर्षाणि यावद्भगवद्दर्शनार्थमुग्रं तपस्तप्य\-मानयोर्मनु\-शतरूपयोः श्रीरामभद्रस्य भगवतः प्राकट्यं प्रथमं शब्द\-पुरःसरमेव समभवत्। यथा~–\end{sloppypar}
\centering\textcolor{red}{प्रभु सर्वग्य दास निज जानी। गति अनन्य तापस नृप रानी॥\nopagebreak\\
माँगु माँगु बर भइ नभ बानी। परम गँभीर कृपामृत सानी॥}\footnote{एतद्रूपान्तरम्–\textcolor{red}{अजानान्निजदासौ तौ सर्वज्ञः परमेश्वरः। अनन्यगतिकौ चापि नृपं राज्ञीञ्च तापसौ॥ याचतं याचतं चेति नभोवागभवत्ततः। अत्यन्तमेव गम्भीरा कृपामृतसुमेलिता॥} (मा॰भा॰~१.१४५.५,६)।}
\nopagebreak\\
\raggedleft{–~रा॰च॰मा॰~१.१४५.५,६}\\
\begin{sloppypar}\hyphenrules{nohyphenation}\justifying\noindent स एव ब्रह्मभूतः शब्दः पुनरर्थभावेन सीता\-राम\-रूपेण प्रकटयाम्बभूव। \textcolor{red}{वाग्वै ब्रह्म} (श॰ब्रा॰~२.१.४.१०, बृ॰उ॰~१.३.२१) इति श्रुतिरपि शब्दमेव ब्रह्मतया श्रावयति। अस्यैव शब्द\-ब्रह्मणो भगवतो देवस्य वृषभ\-रूपता श्रुतौ प्रतिपादिता।\footnote{\textcolor{red}{च॒त्वारि॒ शृङ्गा॒ त्रयो॑ अस्य॒ पादा॒ द्वे शी॒र्षे स॒प्त हस्ता॑सो अस्य। त्रिधा॑ ब॒द्धो वृ॑ष॒भो रो॑रवीति म॒हो दे॒वो मर्त्याँ॒ आ वि॑वेश॥} (ऋ॰वे॰सं॰~४.५८.३)। अत्र पतञ्जलयः~– \textcolor{red}{‘चत्वारि शृङ्गाणि’ पदजातानि नामाख्यातोप\-सर्गनिपाताश्च। ‘त्रयो अस्य पादाः’ त्रयः काला भूतभविष्यद्वर्तमानाः। ‘द्वे शीर्षे द्वौ’ शब्दात्मानौ नित्यः कार्यश्च। ‘सप्त हस्तासो अस्य’ सप्त विभक्तयः। ‘त्रिधा बद्धः’ त्रिषु स्थानेषु बद्ध उरसि कण्ठे शिरसीति। ‘वृषभः’ वर्षणात्। ‘रोरवीति’ शब्दं करोति। कुत एतत्। रौतिः शब्दकर्मा। ‘महो देवो मर्त्याँ आविवेश’ इति। महान् देवः शब्दः। मर्त्या मरणधर्माणो मनुष्याः। तानाविवेश। महता देवेन नः साम्यं यथा स्यादित्यध्येयं व्याकरणम्} (भा॰प॰)।} इदमेव शब्दं ब्रह्म ज्ञातुं व्याकरणं प्रवृत्तं पाणिनीयम्। त्रिमुनि\-व्याख्यानेषु बहुत्र
दार्शनिक\-विषय\-चर्चा। तत्र शब्द\-ब्रह्मणः प्रतिपादनम्। तस्यैव शृङ्ग\-भूतानां नामाख्यातोपसर्ग\-निपातानां पाद\-भूतानाञ्च भूत\-भविष्यद्वर्तमान\-कालानां हस्त\-भूतानां सप्त\-विभक्तीनां शिरसोर्व्युत्पन्नाव्युत्पन्न\-प्रातिपदिकयोः\footnote{यद्वा नित्यकार्य\-शब्दात्मनोर्व्यङ्ग्य\-व्यञ्जक\-ध्वन्योः।} बन्धन\-भूतानां कण्ठ\-तालु\-शिरसां\footnote{उरःकण्ठशिरसां वा।} विशद\-वर्णनेनैव समृद्धमिदं शब्द\-शास्त्रम्। स्वीकृते शब्द\-विशिष्टाद्वैत\-राद्धान्ते ब्रह्मणो निराकार\-साकार\-रूपे मन्यमाना वैयाकरणाः साधुत्वमेव शब्दानां ब्रह्म\-प्राप्ति\-करं निर्णयन्ति। तत्र तत्तद्धातूनां कोऽर्थो लकारार्थ\-निर्णये समासे च स्वतन्त्र\-परतन्त्र\-शक्ति\-पर्यालोचनं नञर्थ\-विचारो निपातानां द्योतकत्वं वाचकत्वं वा शक्तेः स्वरूप\-विमर्शः स्फोटस्य व्यवस्थापनमित्यादयः सन्ति परम\-गभीरा दार्शनिका विषयाः।\end{sloppypar}
\begin{sloppypar}\hyphenrules{nohyphenation}\justifying\noindent\hspace{10mm} त्रिमुनि\-व्याख्यानानन्तरं व्याकरण\-दर्शनस्य परम\-प्राचीना आचार्याः श्रीभर्तृहरयः। तेषां मुख्यो ग्रन्थो \textcolor{red}{वाक्यपदीयम्} इति। तत्र \textcolor{red}{वाक्यञ्च पदञ्चेति वाक्यपदे}। लघ्वक्षरत्वाद्यद्यपि \textcolor{red}{पद}\-शब्दस्य पूर्वं प्रयोगः कर्तव्य आसीत्\footnote{\textcolor{red}{लघ्वक्षरम्} (वा॰~२.२.३४)। \textcolor{red}{लघ्वक्षरं पूर्वं निपततीति वक्तव्यम्। कुशकाशम्। शरशीर्यम्} (भा॰पा॰सू॰~२.२.३४)।} तथाऽपि वाक्यस्याभ्यर्हितत्वात्पूर्वं प्रयोगः।\footnote{\textcolor{red}{अभ्यर्हितम्} (वा॰~२.२.३४)। \textcolor{red}{अभ्यर्हितं च पूर्वं निपततीति वक्तव्यम्। मातापितरौ। श्रद्धामेधे ... अपर आह – सर्वत्र एवाभ्यर्हितं पूर्वं निपततीति वक्तव्यम्। लघ्वक्षरादपीति। दीक्षातपसी। श्रद्धातपसी} (भा॰पा॰सू॰~२.२.३४)।} इत्थं \textcolor{red}{वाक्यञ्च पदञ्चेति वाक्य\-पदे ते अधिकृत्य कृतमिति वाक्यपदीयम्} इति विग्रहे द्वन्द्व\-\textcolor{red}{वाक्यपद}\-शब्दात् \textcolor{red}{शिशु\-क्रन्द\-यम\-सभ\-द्वन्द्वेन्द्र\-जननादिभ्यश्छः} (पा॰सू॰~४.३.८८) इत्यनेन \textcolor{red}{छ}\-प्रत्यय ईयादेशे\footnote{\textcolor{red}{आयनेयीनीयियः फढखच्छघां प्रत्ययादीनाम्‌} (पा॰सू॰~७.१.२) इत्यनेन।} विभक्तिकार्ये च सिद्धम्। अत्र शब्द\-ब्रह्मणः प्रतिपादनमस्यैव जगतश्च कारणता सूचिता। अस्य शब्द\-ब्रह्माण्ड\-निखिल\-कलावतीं कारण\-शक्तिमुपाश्रित्य जन्मादयो विकारा भाव\-भेदं भावयन्तीति प्रतिपादितम्। यथा~–\end{sloppypar}
\centering\textcolor{red}{अध्याहितकला यस्य कालशक्तिमुपाश्रिताः।\nopagebreak\\
जन्मादयो विकाराः षड्भावभेदस्य योनयः॥}\nopagebreak\\
\raggedleft{–~वा॰प॰~१.३}\\
\begin{sloppypar}\hyphenrules{nohyphenation}\justifying\noindent\hspace{10mm} सर्व\-बीज\-रूपमिदमेव शब्द\-ब्रह्म भोक्तृ\-भोक्तव्य\-रूपेण भोग\-रूपेण च विपरिणमति। एवमनेके दार्शनिका विषयाः प्रतिपादिताः। वाक्यपदीये काण्ड\-त्रयं ब्रह्म\-काण्डं वाक्य\-काण्डं पद\-काण्डं चेति।\end{sloppypar}
\begin{sloppypar}\hyphenrules{nohyphenation}\justifying\noindent\hspace{10mm} एतदेवोपजीव्य पुनः श्रीकौण्डभट्टो वैयाकरण\-भूषणसारं लिखित्वा विचारैर्व्याकरणं समभूषयत्। तेन धात्वर्थ\-निर्णयेऽति\-चातुरी प्रदर्शिता। व्यापार\-मुख्य\-विशेष्यक\-शाब्दबोधस्तेषामतीव पाण्डित्य\-पूर्णोऽन्वेषण\-विशेषः। यथा कारिका~–\end{sloppypar}
\centering\textcolor{red}{फलव्यापारयोर्धातुराश्रये तु तिङः स्मृताः।\nopagebreak\\
फले प्रधानं व्यापारस्तिङर्थस्तु विशेषणम्॥}\nopagebreak\\
\raggedleft{–~वै॰सि॰का॰~२}\\
\begin{sloppypar}\hyphenrules{nohyphenation}\justifying\noindent धातुः फल\-व्यापार\-वाचक आश्रय\-भूत\-कर्म\-कर्तृ\-वाचकस्तिङ्। फलापेक्षया व्यापारः प्रधानम्। तिङर्थाः कर्तृ\-कर्म\-संख्या\-कारका विशेषणमिति। \textcolor{red}{रामो हरिं भजति} इत्यत्र \textcolor{red}{रामाभिन्नैक\-कर्तृक\-हरि\-कर्मक\-वर्तमान\-कालावच्छिन्नो भजनानुकूल\-व्यापार} इति शाब्दबोधः। नैयायिक\-सम्मत\-प्रथमान्त\-मुख्य\-विशेष्यक\-शाब्द\-बोधस्य \textcolor{red}{पश्य मृगो धावति} इत्यत्र भाष्य\-सम्मत\-शाब्दबोध\-महास्त्रेण खण्डनं मीमांसक\-निर्णीत\-शाब्दबोध\-खण्डनञ्चामीषां पाण्डित्य\-परिचायकम्।\footnote{\textcolor{red}{अपि चाख्यातार्थ\-प्राधान्ये तस्य देवदत्तादिभिः सममभेदान्वयात्प्रथमान्तार्थस्य प्राधान्यापत्तिः। तथा च “पश्य मृगो धावति” इत्यत्र भाष्य\-सिद्धैक\-वाक्यता न स्यात्। प्रथमान्त\-मृगस्य धावन\-क्रिया\-विशेष्यस्य दृशिक्रियायां कर्मत्वापत्तौ द्वितीयापत्तेः। न चैवमप्रथमा\-सामानाधि\-करण्याच्छतृप्रसङ्गः। एवमपि द्वितीयाया दुर्वारत्वेन “पश्य मृगः” इत्यादि\-वाक्यस्यैवाऽसम्भवापत्तेः} (वै॰भू॰सा॰~१.२)। भाष्ये च \textcolor{red}{क्रियाऽपि क्रिययेप्सिततमा भवति। कया क्रियया। सन्दर्शनक्रियया वा प्रार्थयतिक्रियया वाऽध्यवस्यतिक्रियया वा। इह य एष मनुष्यः प्रेक्षापूर्वकारी भवति स बुद्ध्या तावत्कञ्चिदर्थं सम्पश्यति सन्दृष्टे प्रार्थना प्रार्थनायामध्यवसायोऽध्यवसाय आरम्भ आरम्भे निर्वृत्तिर्निर्वृत्तौ फलावाप्तिः। एवं क्रियाऽपि कृत्रिमं कर्म} (भा॰पा॰सू॰~१.४.३२)।} अकर्मक\-सकर्मक\-धातु\-लक्षणे निर्णयोऽपि महत्त्वपूर्णः। तत्र \textcolor{red}{स्वार्थ\-व्यापार\-समानाधिकरण\-फल\-वाचकत्वमकर्मकत्वं स्वार्थ\-व्यापार\-व्यधिकरण\-फल\-वाचकत्वञ्च सकर्मकत्वम्} इति।\end{sloppypar}
\begin{sloppypar}\hyphenrules{nohyphenation}\justifying\noindent\hspace{10mm} समासे विशिष्ट\-शक्ति\-निर्णयो यथा \textcolor{red}{रामस्य} इत्यस्य \textcolor{red}{पुत्रः} इत्यनेन च व्यस्त\-दशायां कोऽपि सम्बन्धो नास्ति। रामस्येति षष्ठ्यन्तस्य पृथगर्थः पुत्र इति प्रथमान्तस्य पृथक्। समासे सति जात एकार्थी\-भावेऽन्योऽर्थो द्वयोः स्वतन्त्रः कोऽप्यर्थो न पङ्कज\-शब्दवत्। \textcolor{red}{एकार्थीभावो नाम पृथगर्थानामेकोपस्थित्योप\-स्थापनम्}। अतः \textcolor{red}{समासे खलु भिन्नैव शक्तिः पङ्कज\-शब्दवत्} (वै॰सि॰का॰~३१)। अयं च समासोऽव्ययीभाव\-तत्पुरुष\-द्विगु\-कर्मधारय\-बहुव्रीहि\-द्वन्द्व\-भेदैः षड्विधः। अयं च सुपां सुपा तिङा प्रातिपदिकेन कदाचिद्धातुना च तिङन्तस्य तिङन्तेन सुबन्तेन च षड्विधो भवति। तथा च सिद्धान्तकौमुद्याम्~–\end{sloppypar}
\centering\textcolor{red}{सुपां सुपा तिङा नाम्ना धातुनाऽथ तिङां तिङा।\nopagebreak\\
सुबन्तेनेति विज्ञेयः समासः षड्विधो बुधैः॥}\nopagebreak\\
\raggedleft{–~वै॰सि॰कौ॰ सर्वसमासशेष\-प्रकरणे}\\
\begin{sloppypar}\hyphenrules{nohyphenation}\justifying\noindent एवं शक्ति\-निर्णये कौण्डभट्टो बोध\-जनकता\-रूपां शक्तिं मन्यते।\footnote{\textcolor{red}{इन्द्रियाणां चक्षुरादीनां स्वविषयेषु चाक्षुषेषु घटादिषु यथाऽनादिर्योग्यता तदीय\-चाक्षुषादि\-कारणता तथा शब्दानामप्यर्थैः सह तद्बोध\-कारणतैव योग्यता सैव शक्तिरित्यर्थः} (वै॰भू॰सा॰~६.३७)।} बोध\-जनकता च नैयायिक\-परम्परातः किमपि प्रभाविता प्रतीयते।\end{sloppypar}
\begin{sloppypar}\hyphenrules{nohyphenation}\justifying\noindent\hspace{10mm} एतस्मादनन्तरं व्याकरण\-दर्शन\-विचारे क्रान्ति\-पूर्ण\-परिवर्तन\-कर्तारः श्रीनागेशभट्टाः। अमी लघुशब्देन्दौ परिभाषेन्दौ च बहुत्र दर्शन\-विचारान् कृत्वाऽपि स्वतन्त्रान् वैयाकरण\-सिद्धान्त\-मञ्जूषा\-वैयाकरण\-लघु\-सिद्धान्त\-मञ्जूषा\-वैयाकरण\-परम\-लघु\-सिद्धान्त\-मञ्जूषा\-नामकांस्त्रीन्दर्शन\-ग्रन्थान् व्यरचयन्। तत्र शब्दार्थयोः कार्य\-कारण\-भावं भावयन् नागेशो लिखति \textcolor{red}{तद्धर्मावच्छिन्न\-विषयक\-शाब्द\-बुद्धित्वावच्छिन्नं प्रति तद्धर्मावच्छिन्न\-निरूपित\-वृत्ति\-विशिष्ट\-ज्ञानं हेतुः} (प॰ल॰म॰~६) इति। पुरातनैः स्वीकृतां खण्डयन् तस्या एकस्थतया सम्बन्धत्वाभावात्सम्बन्धस्य द्विष्ठत्वाद्वाच्य\-वाचक\-भाव\-रूपां शक्तिं मन्यते।\footnote{\textcolor{red}{तस्मात्पद\-पदार्थयोः सम्बन्धान्तरमेव शक्तिर्वाच्य\-वाचक\-भावापर\-पर्याया} (प॰ल॰म॰~१०)।} इयं नवीना गवेषणा। धात्वर्थ\-निर्णयेऽपि प्राचीनाः\footnote{भट्टोजिदीक्षित\-कौण्ड\-भट्टादयः।} फले व्यापारे च पृथक्शक्तिं कल्पयन्ति किन्त्विमे फल\-विशिष्ट\-व्यापार\footnote{व्यापार\-विशिष्ट\-फले च।} एकामेव शक्तिं मन्यन्ते।\footnote{\textcolor{red}{धातोरर्थद्वये शक्तिद्वयकल्पनं ... चातिगौरवम्। तस्मात्फलावच्छिन्ने व्यापारे व्यापारावच्छिन्ने फले च धातूनां शक्तिः कर्तृ\-कर्मार्थक\-तत्तत्प्रत्यय\-समभि\-व्याहारश्च ततद्बोधे नियामक इत्याहुः} (प॰ल॰म॰~४७)।} यद्यपि \textcolor{red}{पदार्थः पदार्थेनैवान्वेति न तु तदेक\-देशेन}\footnote{मूलं मृग्यम्। उद्द्योते तु \textcolor{red}{सविशेषणानां वृत्तिर्न वृत्तस्य वा विशेषणयोगो न। अगुरुपुत्रादीनाम्} (वा॰~२.१.१) इति वार्त्तिके समाधान\-भाष्ये स्थिते \textcolor{red}{महत्कष्टश्रितः} (भा॰पा॰सू॰~२.१.१) इत्युदाहरणे प्रदीपे स्थितं \textcolor{red}{न तु कष्टविशेषणम्} (भा॰प्र॰ पा॰सू॰~२.१.१) इत्युक्तिं व्याचक्षाणा नागेशभट्टपादाः~– \textcolor{red}{इदमेव “पदार्थः पदार्थेने”ति व्युत्पत्तेर्मूलमिति दिक्} (भा॰उ॰ पा॰सू॰~२.१.१)।} इति व्युत्पत्त्या फलस्य स्वतन्त्रं पदार्थत्वं नास्ति तर्हि कथं व्यापारेऽन्वयः। अत्रैकदेशान्वयस्यापि कल्पना क्रियतेऽत उभयोरपि दोषः। तस्मादस्मद्गुरु\-चरणा एक\-वृन्तावलम्बि\-फलद्वयवद्द्वयोरेव शक्तिं स्वीकुर्वन्ति।\footnote{\textcolor{red}{मम त्वेक\-वृन्ताव\-लम्बि\-फल\-द्वय\-वदुभयांश एका खण्डशश्शक्तिरिति न शक्ति\-द्वय\-कल्पनं न वा बोध\-जनकत्व\-सम्बन्ध\-द्वय\-कल्पनम्} (प॰ल॰म॰ ज्यो॰टी॰~४७) इति प्रणेतॄणां गुरुचरणाः कालिका\-प्रसाद\-शुक्ल\-वर्याः परम\-लघु\-मञ्जूषाया ज्योत्स्ना\-टीकायाम्।} शाब्द\-बोध\-विषयेऽपि प्राचीनाः सर्वत्र व्यापार\-मुख्य\-विशेष्यकं शाब्द\-बोधं स्वीकुर्वन्ति परन्तु नागेशास्ततः पृथग्विचारयन्ति। कर्तृ\-वाच्य\-स्थले भाव\-वाच्ये चेमे व्यापार\-मुख्य\-विशेष्यक\-शाब्द\-बोधं मन्यन्ते कर्मवाच्ये च फल\-मुख्य\-विशेष्यकं शाब्दबोधमङ्गीकुर्वन्ति।\footnote{\textcolor{red}{“कर्तृ\-कर्मार्थेति”~– कर्तृ\-भाव\-प्रत्यय\-समभि\-व्याहारः फल\-विशिष्ट\-व्यापार\-बोधे नियामकः। कर्म\-प्रत्यय\-समभि\-व्याहारो व्यापार\-विशिष्ट\-फल\-बोध\-नियामकः} (प॰ल॰म॰ ज्यो॰टी॰~४७)।} एवं बहुत्रामी भाष्यं पर्यालोच्य नवीन\-दृष्ट्यैव विचारयन्तो विलोक्यन्ते। लिखति च स्वयं \textcolor{red}{पातञ्जले महाभाष्ये कृत\-भूरि\-परिश्रमः} (ल॰शे॰ म॰~१) इति। तादात्म्य\-विषयेऽपि व्याकरण\-दर्शने नागेश\-सम्मतो द्वैतवादः। अद्वैत\-वेदान्तिनस्तादात्म्यस्य लक्षणं कुर्वन्तः \textcolor{red}{तदभिन्नत्वे सति तद्भेदेन प्रतीयमानत्वम्} इति भेदं पारमार्थिकं न मन्यन्त इमे चाभेदमेव व्यवहारिकतया स्वीकुर्वन्ति~– \textcolor{red}{तादात्म्यञ्च तद्भिन्नत्वे सति तदभेदेन प्रतीयमानत्वम्} (ल॰म॰, प॰ल॰म॰~१६)। एवमेवान्यैः शक्ति\-लक्षणा\-व्यञ्जनासु स्वीकृतास्विमे शक्ति\-व्यञ्जने एव स्वीकृत्य लक्षणां शक्यतावच्छेदकारोप\-रूपामेव मत्वा लक्षणां शक्तावन्तर्भावयन्ति। यथा \textcolor{red}{गङ्गायां घोषः} इत्यत्र भगीरथ\-रथ\-खातावच्छिन्न\-जल\-प्रवाहे शक्ये घोषस्यासम्भवात्तात्पर्यानुपपत्त्या शक्य\-सम्बन्धतया गङ्गा\-पदस्य गङ्गा\-तीरे लक्षणेत्यन्य\-दार्शनिकानां धीः। किन्तु नागेशो विप्रतिपद्यते यत् \textcolor{red}{गङ्गायां मकर\-घोषौ} इत्यत्र शक्यार्थे गङ्गा\-जले मकरस्यान्वयः सम्भवति तत्तीरे घोषस्य तर्हि गङ्गा\-पदस्य धर्म\-द्वयावच्छिन्नत्वादेक\-धर्मावच्छिन्नत्वाभाव एक\-धर्मावच्छिन्न एकत्रावच्छिन्नस्यैकस्यार्थ\-भावावच्छिन्न\-संसर्गेण साहित्ये सह विवक्षा तदा च द्वन्द्व इति कथं द्वन्द्व\-समासः मकर\-घोष\-पदयोर्विरुद्ध\-धर्मावच्छिन्नेऽन्वयात्। अतः शक्यतावच्छेदकस्य गङ्गा\-रूपार्थस्य गङ्गा\-तीर आरोप एवं द्वयोरप्येक\-धर्मावच्छिन्ने गङ्गा\-प्रवाह एवार्थेऽन्वयः। इयं नवीना क्रान्तिः। इत्थमेव बौद्धार्थः स्वीकृतः नागेशेन। असम्भवार्थ\-शब्दानामपि वन्ध्या\-सुतादीनां साधुत्वम्।
एवमिमे व्याकरण\-दर्शनं चरम\-शिखरे प्रतिष्ठापयामासुरिति।\end{sloppypar}
\centering\textcolor{red}{व्याकरणस्य द्वे नेत्रे प्रक्रिया दर्शनं तथा।\nopagebreak\\
सङ्क्षेपतो निर्दिष्टोऽत्रोभयोः परिचयो मया॥}\\ 
\begin{sloppypar}\hyphenrules{nohyphenation}\justifying\noindent\hspace{10mm} एवं भुवन\-विदित\-महिम्नो शब्द\-लघिम्नो विद्वद्गणित\-गरिम्नो पाणिनीय\-व्याकरणस्य रामकथया सह पूर्ण\-सम्बन्धो वर्तते। यतो हि पाणिनीयं व्याकरणं श्रीराम\-कथा चेति द्वे अपि वेद\-मूलके। पाणिनीय\-व्याकरणस्य वेद\-मूलकत्वमुपपादितम्। इदानीं रामायणस्य वेद\-मूलकत्वं मीमांस्यते। जगदुद्धार\-चिकीर्षया भुवन\-विदित\-लीलो दिव्य\-शीलो जगदात्मा स्वयं परमात्मा भक्त\-रक्षण\-परायणो नारायणः सकल\-मङ्गलायने दशरथायने राजीव\-नयनो नयन\-गोचरतां समागमद्भुवो भार\-जिहीर्षया। वेद\-वेद्यं परं तत्त्वं लोकाभिरामः श्रीरामः। तदा स्व\-प्रतिपाद्यं परं ब्रह्म सगुणं साकारं कोशलेन्द्र\-कुमारं तनु\-विजित\-कोटि\-मारं परमोदारं श्रीरामं दृष्ट्वा प्राचेतसं वाल्मीकिं माध्यमं कृत्वा रामायण\-रूपेण स्वयमेव वेदः प्रादुर्बभूव। अतोऽभियुक्ता वर्णयन्ति~–\end{sloppypar}
\centering\textcolor{red}{वेदवेद्ये परे पुंसि जाते दशरथात्मजे। \nopagebreak\\
वेदः प्राचेतसादासीत्साक्षाद्रामायणात्मना॥\nopagebreak\\
तस्माद्रामायणं देवि वेद एव न संशयः॥}\nopagebreak\\
\raggedleft{–~अग॰सं॰}\\
\begin{sloppypar}\hyphenrules{nohyphenation}\justifying\noindent इति। यदि चेद्वेदस्य रक्षार्थं व्याकरणमध्येयं तदा वेदावतारस्य रामायणस्यापि रक्षार्थं व्याकरणमध्येयम्। किं बहुना वेदस्य षट्स्वङ्गेषु व्याकरणं मुख्यतयोपादेयम्। तथैव वेद\-सम्मितस्य\footnote{\textcolor{red}{इदं पवित्रं पापघ्नं पुण्यं वेदैश्च सम्मितम्। यः पठेद्रामचरितं सर्वपापैः प्रमुच्यते॥} (वा॰रा॰~१.१.९८)।} रामायणस्यापि सम्यग्ज्ञानार्थं पाणिनीयं व्याकरणमुपादेयम्। व्याकरणं विना गूढा भावाः कथं ज्ञातुं शक्यन्ते। यथा वाल्मीकीये हनुमान् श्रीरामं पृच्छति यत्~–\end{sloppypar}
\centering\textcolor{red}{आयताश्च सुवृत्ताश्च बाहवः परिघोपमाः॥\nopagebreak\\
सर्वभूषणभूषार्हाः किमर्थम् न विभूषिताः।}\nopagebreak\\
\raggedleft{–~वा॰रा॰~४.३.१५–१६}\\
\begin{sloppypar}\hyphenrules{nohyphenation}\justifying\noindent अत्र \textcolor{red}{सर्व\-भूषणानि भूषयितुमर्हन्ति}\footnote{\textcolor{red}{“सर्वभूषणभूषार्हाः” आभरणस्याभरणमित्युक्त\-रीत्या भूषणान्यपि भूषयितुमर्हाः किमर्थं न विभूषिताः। इमान् भूषणैरलङ्कृत्याऽभरणाभरणत्वं किमिति न प्रकाशितमित्यर्थः} (वा॰रा॰ भू॰टी॰~४.३.१५–१६) इति गोविन्द\-राजाः।}  इति व्याख्यानं किमवैयाकरणेन सम्भवम्। एवं श्रीरामो हनुमन्तं प्रशंसन् कथयति यत्~–\end{sloppypar}
\centering\textcolor{red}{नूनं व्याकरणं कृत्स्नमनेन बहुधा श्रुतम्।\nopagebreak\\
बहु व्याहरताऽनेन न किञ्चिदपशब्दितम्॥}\nopagebreak\\
\raggedleft{–~वा॰रा॰~४.३.२९}\\
\begin{sloppypar}\hyphenrules{nohyphenation}\justifying\noindent अपरञ्च~–\end{sloppypar}
\centering\textcolor{red}{अनया चित्रया वाचा त्रिस्थानव्यञ्जनस्थया।\nopagebreak\\
कस्य नाराध्यते चित्तमुद्यतासेररेरपि॥}\nopagebreak\\
\raggedleft{–~वा॰रा॰~४.३.३३}\\
\begin{sloppypar}\hyphenrules{nohyphenation}\justifying\noindent अत्र \textcolor{red}{त्रि\-स्थान\-व्यञ्जनस्थया} इति शब्दस्य कथमवैयाकरणो भावमवगन्तुं पारयिष्यति। वैयाकरणस्तूच्चारणस्य कण्ठ\-तालु\-मूर्धान इति त्रि\-स्थानं\footnote{यद्वाऽभि\-व्यक्तेरुरःकण्ठ\-शिरांसीति त्रिस्थानम्।} ज्ञात्वा ततो वाचः समुद्भवं ज्ञास्यति। अगस्त्यो रामायण\-महामाला\-रत्नं श्रीमन्तं हनुमन्तं नव\-व्याकरणस्य वेत्तारं कथयति। \textcolor{red}{सोऽयं नव\-व्याकरणार्थवेत्ता} (वा॰रा॰~७.३६.४७) इत्यादि। वाल्मीकिरपि \textcolor{red}{तदुपगत\-समास\-सन्धि\-योगम्} (वा॰रा॰~१.२.४३) इति कथयति। एवमेव विबुध\-वाण्यां गीतानि कोटिशो रामायणानि व्याकरणं विना कथमपि ज्ञातुं न शक्यन्ते। मम त्वियं मान्यता संस्कृतवत्पुण्य\-जनकतावच्छेदिष्ठायां महादेव\-भाषायां भाषायां सङ्गीतं भक्ति\-शिरोमणि\-सकल\-कवि\-कुल\-शेखर\-हस्तामलकीकृत\-वेद\-शास्त्र\-पुराणेतिहास\-काव्य\-नाटक\-निखिल\-निगमागम\-सुदुर्गम\-विचार\-चतुर्दिक्चारु\-चातुरी\-विलोकित\-तुरीय\-महनीय\-कविता\-वनिता\-जीवन\-सीता\-रमण\-पद\-पद्म\-पराग\-सुराग\-रसमिलिन्द\-माहात्म्य\-श्रीमत्तुलसीदास\-कृत\-श्रीमद्राम\-चरित\-मानसमप्यवैयाकरणेन ज्ञातुं न शक्यते। यथा~–\end{sloppypar}
\centering\textcolor{red}{सरिस श्वान मघवान जुबानू}\footnote{एतद्रूपान्तरम्–\textcolor{red}{मघवा श्वा युवा चैव वर्तन्ते समतान्विताः} (मा॰भा॰~२.३०२.८)।}\nopagebreak\\
\raggedleft{–~रा॰च॰मा॰~२.३०२.८}\\
\begin{sloppypar}\hyphenrules{nohyphenation}\justifying\noindent इत्युपमा। तात्पर्यं येन \textcolor{red}{श्व\-युव\-मघोनामतद्धिते} (पा॰सू॰~६.४.१३३) इति सूत्रं न पठितं स किं बोद्धुं प्रभविष्यति। बहुत्र तुलसीदासेनापि व्याकरणस्य वेदान्तादि\-दर्शनानां स्वाभाविकतया रहस्यमुपन्यासि। तत्र व्याकरणमन्तरेणेतर\-दर्शन\-ज्ञानाभावे मानसस्योत्तर\-काण्डस्योत्तरार्धं कथमपि स्पष्टयितुं न शक्यते। तत्परम्परायां वर्तमानं श्रीमदध्यात्म\-रामायणमपि पाणिनीय\-व्याकरणेन पूर्णतः सम्बद्धम्। पाणिनीय\-व्याकरणं चतुर्दश\-सूत्राचार्यतया शिवोक्तम्। तस्मात्पाणिनिर्बहुश उकारानु\-बन्धानि सूत्राणि पपाठ। यथा \textcolor{red}{डः सि धुट्} (पा॰सू॰~८.३.२९) \textcolor{red}{ङ्णोः कुक् टुक् शरि} (पा॰सू॰~८.३.२८) \textcolor{red}{शि तुक्} (पा॰सू॰~८.३.३१) \textcolor{red}{आने मुक्} (पा॰सू॰~७.२.८२) \textcolor{red}{ह्रस्वनद्यापो नुट्} (पा॰सू॰~७.१.५४) इत्यादि। अन्याननु\-बन्धानुपेक्ष्योकारानुबन्ध\-बाहुल्यादुकार\-वाच्यस्य शम्भोः स्मरणं पाणिनि\-कृतं प्रतीयते।\end{sloppypar}
\begin{sloppypar}\hyphenrules{nohyphenation}\justifying\noindent\hspace{10mm} एवमध्यात्म\-रामायणमपि शिव\-भाषितं यथा~–\end{sloppypar}
\centering\textcolor{blue}{पुरारिगिरिसम्भूता श्रीरामार्णवसङ्गता।\nopagebreak\\
अध्यात्मरामगङ्गेयं पुनाति भुवनत्रयम्॥}\nopagebreak\\
\raggedleft{–~अ॰रा॰~१.१.५}\\
\begin{sloppypar}\hyphenrules{nohyphenation}\justifying\noindent तर्हि शम्भु\-भाषितस्यास्याध्यात्म\-रामायणस्य शैवेन पाणिनि\-व्याकरणेनैक\-वाक्यता\-पुरःसर\-सम्बन्धो भवेदेवेति वच्मि। अहं तूत्प्रेक्षे यत्स्वयमेव शिवो लोक\-वत्सलतया भगवतीं पार्वतीं भवानीं श्रोत्रीं मत्वा सम्भाष्य च भुवन\-पावनीं राम\-कथां तदर्थ\-बुबोधयिषया पाणिनि\-हृदयस्थो व्याकरण\-मूल\-भूतायां माहेश्वर्यां चतुर्दश\-सूत्र्यां समासतो राम\-कथां कथयति। तत्र द्वि\-चत्वारिंशदक्षराणां सङ्ग्रहो यश्च भगवतः श्रीरामस्य द्विचत्वारिंशद्वर्षीय\-चरित्रं ध्वनयति। \label{text:exileage1}जन्मतो विवाहं यावद्द्वादशाब्दावधिस्ततो द्वादश\-वर्षं यावदयोध्यायां वास एवं पञ्चविंशे वर्षे सीता\-लक्ष्मणाभ्यां सह वन\-गमनं चतुर्दशाब्दं यावदरण्य\-चरित्रं मैथिली\-हरण\-रावण\-संहरणादिकमन्तिम\-वर्षे राज्य\-लीलेति मिलित्वोनचत्वारिंशद्वर्षाणि। वर्षत्रयं राज्य\-व्यवस्थायाम्। एषु मुख्यं रामायणम्। अक्षरस्य ब्रह्मणः श्रीरामस्य दिव्या चर्चा सूत्रबद्धाक्षरैः सङ्केतिता। चतुर्दशसूत्र्यां विभाग\-द्वयं स्वर\-वर्गो व्यञ्जनवर्गश्च। स्वराणां चर्चा चतुर्भिः सूत्रैर्व्यञ्जनानाञ्च चर्चा दशभिः सूत्रैर्व्यधायि। अत्रापि राम\-कथायां विभाग\-द्वयं निर्गुण\-लीलायाः सगुण\-लीलायाश्च। ऐश्वर्यस्य माधुर्यस्य वा। ऐश्वर्यं चत्वारि फलानि ददात्यतश्चतुर्भिः सूत्रैस्तत्सङ्केतः सङ्गच्छते। ऐश्वर्यञ्च स्वरस्थानापन्नमतः स्वरैः कथ्यते। \textcolor{red}{स्वेन राजते स्वेन रमते वेति स्वरः}।\footnote{\textcolor{red}{स्व} उपपदे \textcolor{red}{राज्‌}\-धातोः (\textcolor{red}{राजृँ दीप्तौ} धा॰पा॰~८२२) \textcolor{red}{रम्‌}\-धातोर्वा (\textcolor{red}{रमुँ क्रीडायाम्} धा॰पा॰~८५३) \textcolor{red}{अन्येष्वपि दृश्यते} (पा॰सू॰~३.२.१०१) इत्यनेन \textcolor{red}{ड}\-प्रत्ययः। अनुबन्ध\-लोपे \textcolor{red}{डित्यभस्याप्यनु\-बन्धकरण\-सामर्थ्यात्} (वा॰~६.४.१४३) इत्यनेन टिलोपे \textcolor{red}{सुपो धातु\-प्रातिपदिकयोः} (पा॰सू॰~२.४.७१) इत्यनेन सुब्लुकि विभक्तिकार्ये \textcolor{red}{स्वरः}। महाभाष्ये च~– \textcolor{red}{अन्तरेणापि व्यञ्जनमच एवैते गुणा लक्ष्यन्ते। न पुनरन्तरेणाचं व्यञ्जनस्योच्चारणमपि भवति। अन्वर्थं खल्वपि निर्वचनम् – स्वयं राजन्त इति स्वराः। अन्वग्भवति व्यञ्जनमिति} (भा॰पा॰सू॰~१.१.२९–३०)। तथा च सङ्गीत\-रत्नाकरस्थस्य \textcolor{red}{योऽयं स्वयं राजते} (स॰र॰~१.१) इत्यस्य व्याख्यानावसरे सङ्गीत\-सुधाकर\-टीकायां सिंहभूपालः~– \textcolor{red}{“योऽयं स्वयं राजते” अन्यानपेक्षतयाऽऽनन्दयितृत्वात्। स्वरपदस्य निरुक्तिरप्यनेन कथ्यते} (स॰र॰ सु॰टी॰~१.१)। वाचस्पत्याभिधाने \textcolor{red}{अचः स्वयं विराजन्ते हलस्तु परगामिनः} इत्युद्धृतः।} स्वरा ये खलूच्चारणे किमपि वर्णान्तरं नापेक्षन्ते। तथैवेश्वरोऽनपेक्षकोऽपि स्वस्मिन्महीयते। स्वराणां सङ्ख्या नव। राघवस्य च प्रादुर्भावो नवम्यां यथा वाल्मीकीये~–\end{sloppypar}
\centering\textcolor{red}{ततो यज्ञे समाप्ते तु ऋतूनां षट् समत्ययुः।\nopagebreak\\
ततस्तु द्वादशे मासे चैत्रे नावमिके तिथौ॥}\nopagebreak\\
\raggedleft{–~वा॰रा॰~१.१८.८}\\
\begin{sloppypar}\hyphenrules{nohyphenation}\justifying\noindent अध्यात्म\-रामायणे च~–\end{sloppypar}
\centering\textcolor{blue}{मधुमासे सिते पक्षे नवम्यां कर्कटे शुभे।\nopagebreak\\
पुनर्वस्वृक्षसहिते उच्चस्थे ग्रहपञ्चके॥}\nopagebreak\\
\raggedleft{–~अ॰रा॰~१.३.१४}\\
\begin{sloppypar}\hyphenrules{nohyphenation}\justifying\noindent नवमी सङ्ख्या श्रेष्ठा पूर्णा। एतस्या वरीयसी काऽपि सङ्ख्या नास्ति। आद्यन्त\-निर्वाहिका सर्वथा गुणिताऽपि पूर्व\-पर\-सम्मेलनेन नवैव। यथा द्विगुणिता नव सङ्ख्याऽष्टादशतामुपैति। सा च दक्षिण एकं वामेऽष्टाविति लिखित्वा भवति।\footnote{\textcolor{red}{अङ्कानां वामतो गतिः} इति न्यायेन।} सम्मेलनेन द्वयोः पुनर्नव। एवं त्रि\-गुणिता सप्तविंशतिः। सा च दक्षिणे द्वौ वामे सप्तेति लिखित्वा भवति।\footnote{\textcolor{red}{अङ्कानां वामतो गतिः} इति न्यायेन।} सम्मेलनेन द्वयोः पुनर्नव। एवमन्यत्रापि। यथा सर्वथा गुणिताऽपि नवमी सङ्ख्या द्वयोर्योगेन नवत्वं न जहाति तथैव नवभिः स्वरैः सङ्केत्यो भगवान् गुणितोऽप्यर्थाद्भक्तेच्छया जनवत्सलत्व\-कारुणिकत्व\-कृपालुत्व\-प्रणतानुरागित्व\-भक्तेच्छा\-पालकत्व\-दीन\-बन्धुत्व\-सरलत्व\-सफलत्व\-स्वामित्व\-रघुनाथत्व\-प्रभृतिभिर्दशभिर्गुणैर्गुणितोऽपि नवमी सङ्ख्येवैश्वर्यं न त्यजति। नवमीतोऽधिका काऽपि सङ्ख्या नहि। तथैव रामतोऽधिकः कोऽपि देवो नास्ति। नवमी सङ्ख्या पूर्णा श्रीरामोऽपि पूर्णः। ऐश्वर्य\-लीलायां चतुष्पाद्विभूति\-दर्शनम्। विभूतेश्चत्वारः पादा यथा~–\end{sloppypar}
\centering\textcolor{red}{ए॒तावा॑नस्य महि॒माऽतो॒ ज्यायाँ॑श्च॒ पूरु॑षः॒।\nopagebreak\\
पादो॑ऽस्य॒ विश्वा॑ भू॒तानि॑ त्रि॒पाद॑स्या॒मृतं॑ दि॒वि॥\nopagebreak\\
त्रि॒पादू॒र्ध्व उदै॒त्पुरु॑षः॒ पादो॑ऽस्ये॒हाभ॑व॒त्पुनः॑।\nopagebreak\\
ततो॒ विष्व॒ङ्व्य॒क्रामत्साशनानश॒ने अ॒भि॥}\nopagebreak\\
\raggedleft{–~शु॰य॰वा॰मा॰~३१.३-४}\footnote{ऋग्वेद\-संहितायां (ऋ॰वे॰सं॰~१०.९०.३-४) तैत्तिरीयारण्यके (कृ॰य॰ तै॰आ॰~३.१२.२) अप्येतौ मन्त्रौ।}\\
\begin{sloppypar}\hyphenrules{nohyphenation}\justifying\noindent चतुष्पाद्विभूतेर्भगवतश्चतुर्भिः सूत्रैर्विभूति\-भूषणेन ढक्का\-माध्यमेन गानमति\-न्याय\-सङ्गतं युक्ति\-युक्तञ्च। ऐश्वर्य\-लीलायां श्रीरामस्य मिलन्ति चत्वारि सूत्राणि। विकारैः सह विग्रहः फलानां निग्रहः कारुण्यादीनां सङ्ग्रहो निज\-पद\-पद्म\-प्रपन्न\-भक्तेष्वनुग्रहश्च। इदमपि
चतुर्भिः सूत्रैः
स्वररूप\-श्री\-रामस्यैश्वर्य\-लीला\-प्रतिपादने तात्पर्यमूह्यम्। चतुर्णामार्त\-जिज्ञास्वर्थार्थि\-ज्ञानि\-भक्तेष्वनु\-ग्रहश्चास्मिन् तथ्ये दृढीकरणमापादयति। चत्वारो हि भगवतो भक्ता आर्तो जिज्ञासुरर्थार्थी ज्ञानी चेति।\end{sloppypar}
\begin{sloppypar}\hyphenrules{nohyphenation}\justifying\noindent\hspace{10mm} (१) \textcolor{red}{आर्तः} आर्ति\-ग्रस्तश्चतुर्णामपि पुरुषार्थानां लिप्सया परमात्मानं प्रपद्यते यथा विभीषणः।\end{sloppypar}
\begin{sloppypar}\hyphenrules{nohyphenation}\justifying\noindent\hspace{10mm} (२) \textcolor{red}{जिज्ञासुः} ज्ञातुमिच्छुरर्थाद्धर्म\-मोक्ष\-लिप्सया श्रीरामं शुश्रूषते यथा लक्ष्मणः।\end{sloppypar}
\begin{sloppypar}\hyphenrules{nohyphenation}\justifying\noindent\hspace{10mm} (३) \textcolor{red}{अर्थार्थी} भगवन्तं राघवमर्थ\-कामौ प्रार्थयते यथा सुग्रीवः।\end{sloppypar}
\begin{sloppypar}\hyphenrules{nohyphenation}\justifying\noindent\hspace{10mm} (४) \textcolor{red}{ज्ञानी} ज्ञानं प्राप्य पूर्वं तु मोक्षमिच्छति। विशेषो विज्ञानि\-रूपो निष्कामो हृदि रामं रमयितुमेव सदा चेष्टते यथा जटायुर्हनुमांश्च। गीतायां मानसे चापि चर्चा यथा~–\end{sloppypar}
\centering\textcolor{red}{चतुर्विधा भजन्ते मां जनाः सुकृतिनोऽर्जुन।\nopagebreak\\
आर्तो जिज्ञासुरथार्थी ज्ञानी च भरतर्षभ॥}\nopagebreak\\
\raggedleft{–~भ॰गी॰~७.१६}\\
\centering\textcolor{red}{राम भगत जग चारि प्रकारा। सुकृती चारिउ अनघ उदारा॥}\footnote{एतद्रूपान्तरम्–\textcolor{red}{इत्थं चतुर्विधास्सन्ति भक्ता रामस्य निश्चितम्। एते सर्वेऽप्युदाराश्च धन्याः कलुष\-वर्जिताः॥} (मा॰भा॰~१.२२.६)।}\nopagebreak\\
\raggedleft{–~रा॰च॰मा॰~१.२२.६}\\
\begin{sloppypar}\hyphenrules{nohyphenation}\justifying\noindent ऐश्वर्य\-लीलायाममीषु चतुर्षु कृपाऽतोऽपि चतुर्भिः सूत्रैः प्रतिपादनं सङ्गच्छते। ऐश्वर्य\-लीलायां चतस्रः मुख्या घटनाः कौसल्या\-समक्षं विराड्रूप\-प्रदर्शनं यथा~–\end{sloppypar}
\centering\textcolor{red}{देखरावा मातहि निज अद्भुतरूप अखंड।\nopagebreak\\
रोम रोम प्रति लागे कोटि कोटि ब्रह्मांड॥}\footnote{एतद्रूपान्तरम्–\textcolor{red}{स मातरं दर्शयति स्म नैजमखण्डमाश्चर्यमयञ्च रूपम्। यद्रोमरोमश्रिततामुपेता ब्रह्माण्ड\-कोट्यो गुणिता अनेकाः॥} (मा॰भा॰~१.२०१)।}\nopagebreak\\
\raggedleft{–~रा॰च॰मा॰~१.२०१}\\
\begin{sloppypar}\hyphenrules{nohyphenation}\justifying\noindent अहल्योद्धारः परशुराम\-समक्षं वैष्णव\-धनुः\-कर्षणं रावण\-वधश्च। अतोऽपि चतुर्भिः सूत्रैः स्वरच्छलेनैश्वर्यांश\-वर्णनं रमणीयं प्रतिभाति मे। नव स्वरास्तथैवैश्वर्य\-लीलाऽपि श्रवण\-कीर्तन\-स्मरण\-पाद\-सेवन\-पूजन\-वन्दन\-दास्य\-सख्यात्म\-निवेदनेति\-नव\-लक्षणां भक्तिं ददाति। नवधा भक्तिर्भागवतानु\-सारेणेत्थम्~–\end{sloppypar}
\centering\textcolor{red}{श्रवणं कीर्तनं विष्णोः स्मरणं पादसेवनम्।\nopagebreak\\
अर्चनं वन्दनं दास्यं सख्यमात्मनिवेदनम्॥\nopagebreak\\
इति पुंसाऽर्पिता विष्णौ भक्तिश्चेन्नवलक्षणा।\nopagebreak\\
क्रियते भगवत्यद्धा तन्मन्येऽधीतमुत्तमम्॥}\nopagebreak\\
\raggedleft{–~भा॰पु॰~७.५.२३–२४}\\
\begin{sloppypar}\hyphenrules{nohyphenation}\justifying\noindent रामायणेऽमीषां सङ्ग्रहश्च भक्तानाम्~–\end{sloppypar}
\centering\textcolor{red}{श्रीरामश्रवणे मता गिरिसुता काकः शिवः कीर्तने\nopagebreak\\
कौसल्या स्मरणे पदाब्जभजने सीता सुतीक्ष्णोऽर्चने।\nopagebreak\\
सौमित्रिः पदवन्दने च हनुमान्दास्ये च सख्येऽर्कजो\nopagebreak\\
लङ्केशो भरतः समर्पणविधौ रामाप्तिरेषां फलम्॥}\nopagebreak\\
\raggedleft{–~इति मम}\\
\begin{sloppypar}\hyphenrules{nohyphenation}\justifying\noindent एवं भागवतेऽपि~–\end{sloppypar}
\centering\textcolor{red}{श्रीकृष्णश्रवणे परीक्षिदभवद्वैयासकिः कीर्तने\nopagebreak\\
प्रह्लादः स्मरणे तदङ्घ्रिभजने लक्ष्मीः पृथुः पूजने।\nopagebreak\\
अक्रूरस्त्वथ वन्दने च हनुमान्दास्येऽथ सख्येऽर्जुनः\nopagebreak\\
सर्वस्वात्मनिवेदने बलिरभूत्कृष्णाप्तिरेषां फलम्॥}\footnote{मूलं गौडीय\-वैष्णव\-ग्रन्थेषु मृग्यम्।}\\ 
\begin{sloppypar}\hyphenrules{nohyphenation}\justifying\noindent एवं ऐश्वर्य\-लीलाऽपि नव\-विधा। तत्रैश्वर्यं धर्मो यशः श्रीर्ज्ञानं वैराग्यं निग्रहो विग्रहोऽनुग्रहश्चेति। अत एव नव\-स्वरात्मक\-नवैश्वर्य\-लीलानां चतुर्भिः सूत्रैः प्रतिपादनं सुस्पष्टमेव रामायणस्यैश्वर्य\-वर्णनम्। तत्र~–\end{sloppypar}
\begin{sloppypar}\hyphenrules{nohyphenation}\justifying\noindent\hspace{10mm} \textcolor{red}{(१) अइउण्}~– अकारो वासुदेवः। यथा~–\end{sloppypar}
\centering\textcolor{red}{अक्षराणामकारोऽस्मि द्वन्द्वः सामासिकस्य च।\nopagebreak\\
अहमेवाक्षयः कालो धाताऽहं विश्वतोमुखः॥}\nopagebreak\\
\raggedleft{–~भ॰गी॰~१०.३३}\\
\begin{sloppypar}\hyphenrules{nohyphenation}\justifying\noindent इत्थम् \textcolor{red}{अ} श्रीरामो वासुदेवः।\footnote{\textcolor{red}{अकारो वासुदेवः स्यात्} (ए॰को॰~१)।} \textcolor{red}{इ} महालक्ष्मी सीता।\footnote{\textcolor{red}{लक्ष्मीरीकार उच्यते} (ए॰को॰~२)। \textcolor{red}{ईः। स्त्री। अस्य विष्णोः पत्नी। ङीप्। लक्ष्मीः। इति विश्वमेदिन्यौ} इति शब्दकल्पद्रुमः। \textcolor{red}{ईः। स्त्री। अस्य विष्णोः पत्नी। ङीष्। लक्ष्म्याम्} इति वाचस्पत्यम्। ततः \textcolor{red}{ङ्यापोः सञ्ज्ञा\-छन्दसोर्बहुलम्} (पा॰सू॰~६.३.६३) इत्यनेन छान्दस\-ह्रस्वः। \textcolor{red}{छन्दोवत्सूत्राणि भवन्ति} (भा॰पा॰सू॰~१.१.१, १.४.३) इत्यनेन सूत्राणां छान्दसत्वं माहेश्वर\-सूत्रार्थ\-प्रकरणेऽस्मिन् सर्वेषु छान्दस\-कार्येषु बोध्यम्।} \textcolor{red}{उ} जीवाचार्यो लक्ष्मणः।\footnote{उपलक्षणत्वात्।} \textcolor{red}{ण्} निर्वृति\-वाचकः। \textcolor{red}{णश्च निर्वृतिवाचकः} (गो॰पू॰ता॰उ॰~१) इति गोपाल\-तापनीय\-श्रुतेः। अर्थाद्राम\-सीता\-लक्ष्मण\-ध्यानेन जीवो भव\-बन्धनान्निवर्तत\footnote{जीवस्य भव\-बन्धनान्निवृतिर्वा।} इति सूत्रार्थः।\footnote{\textcolor{red}{अइउ} इत्यत्र समासेऽप्यसन्धिश्छान्दसः। \textcolor{red}{अइउना ण्} इति विग्रहे \textcolor{red}{कर्तृकरणे कृता बहुलम्‌} (पा॰सू॰~२.१.३२) इत्यनेन तृतीया\-तत्पुरुष\-समासः। विभक्तिकार्ये \textcolor{red}{हल्ङ्याब्भ्यो दीर्घात्सुतिस्यपृक्तं हल्} (पा॰सू॰~६.१.६८) इत्यनेन सुलोपे सिद्धम्।} अकारेकारोकार\-वाचकान् राम\-सीता\-लक्ष्मणान्नयति प्रापयतीति \textcolor{red}{णीञ्‌}\-धातोः (\textcolor{red}{णीञ् प्रापणे} धा॰पा॰~९०१) \textcolor{red}{अइउण्} इत्यपि।\footnote{\textcolor{red}{अइउ} इत्यत्र समासेऽप्यसन्धिश्छान्दसः। \textcolor{red}{अइउ} इत्युपपदे \textcolor{red}{नी}\-धातोरौणादिको \textcolor{red}{ड्विन्‌}\-प्रत्ययः। \textcolor{red}{कार्याद्विद्यादनूबन्धम्} (भा॰पा॰सू॰~३.३.१) \textcolor{red}{केचिदविहिता अप्यूह्याः} (वै॰सि॰कौ॰~३१६९) इत्यनुसारमूह्योऽ\-यमविहित\-प्रत्ययः। सर्वापहारि\-लोपे \textcolor{red}{डित्यभस्याप्यनु\-बन्धकरण\-सामर्थ्यात्} (वा॰~६.४.१४३) इत्यनेन टिलोपे सौत्रत्वान्नकारस्य णकार उपपद\-समासे विभक्तिकार्ये \textcolor{red}{हल्ङ्याब्भ्यो दीर्घात्सुतिस्यपृक्तं हल्} (पा॰सू॰~६.१.६८) इत्यनेन सुलोपे सिद्धम्।}\end{sloppypar}
\begin{sloppypar}\hyphenrules{nohyphenation}\justifying\noindent\hspace{10mm} एवमेव~–\end{sloppypar}
\begin{sloppypar}\hyphenrules{nohyphenation}\justifying\noindent\hspace{10mm} \textcolor{red}{(२) ऋऌक्}~– \textcolor{red}{ऋ} ऋषिः।\footnote{\textcolor{red}{नामैकदेशग्रहणे नाममात्रग्रहणम्} इत्यनेन न्यायेन।} \textcolor{red}{ऌ} ऌकारवज्जटिल\-साधना\-रतो मुनिः।\footnote{\textcolor{red}{नामैकदेशग्रहणे नाममात्रग्रहणम्} इत्यनेन न्यायेन।} तावेव कथयत्यात्म\-सम्मुखं करोतीति \textcolor{red}{ऋऌक्}।\footnote{\textcolor{red}{ऋऌ} इत्यत्र समासेऽप्यसन्धिश्छान्दसः। \textcolor{red}{ऋऌ} इत्युपपदे \textcolor{red}{कथ्‌}\-धातोः (\textcolor{red}{कथँ वाक्य\-प्रबन्धने} धा॰पा॰~१८५१) पूर्ववदौणादिको \textcolor{red}{ड्विन्‌}प्रत्ययः। पूर्ववत्सर्वापहारि\-लोपे टिलोप उपपद\-समासे विभक्तिकार्ये \textcolor{red}{हल्ङ्याब्भ्यो दीर्घात्सुतिस्यपृक्तं हल्} (पा॰सू॰~६.१.६८) इत्यनेन सुलोपे सिद्धम्।}\end{sloppypar}
\begin{sloppypar}\hyphenrules{nohyphenation}\justifying\noindent\hspace{10mm} \textcolor{red}{(३) एओङ्}~– एवम् \textcolor{red}{ए} षडैश्वर्य\-वाचको रामः।\footnote{माहेश्वर\-सूत्रेषु \textcolor{red}{ए}\-कारस्य क्रमसङ्ख्यानुसारम्। ऐश्वर्यं धर्मो यशः श्रीर्ज्ञानं वैराग्यञ्चेति षडैश्वर्याणि। \textcolor{red}{ऐश्वर्यस्य समग्रस्य धर्मस्य यशसः श्रियः। ज्ञानवैराग्ययोश्चैव षण्णां भग इतीरणा॥} (वि॰पु॰~६.५.७४)।} \textcolor{red}{ओ} सप्तावरण\-नाशिनी सीता।\footnote{माहेश्वर\-सूत्रेषु \textcolor{red}{ओ}\-कारस्य क्रमसङ्ख्यानुसारम्। क्षिति\-जल\-पावक\-गगन\-समीर\-महत्प्रकृतयः सप्तावरणानि। यथा रामचरितमानस उत्तरकाण्डे काकभुशुण्डि\-गीतायां मानसकाराः \textcolor{red}{सप्ताबरन भेद करि जहँ लगि रहि गति मोरि। गयउँ तहाँ प्रभु भुज निरखि ब्याकुल भयउँ बहोरि॥} (रा॰च॰मा॰~७.७९ख)। एतद्रूपान्तरम्–\textcolor{red}{प्रभेद्य सप्तावरणानि यावद्गतिर्ममासीदगमञ्च तावत्। तत्रापि वीक्ष्येशभुजौ स्वपृष्ठे तीव्राकुलत्वेन युतोऽहमासम्॥} (मा॰भा॰~७.७९ख)।} तावञ्चति पूजयति इति \textcolor{red}{एओङ्}।\footnote{\textcolor{red}{एओ} इत्यत्र समासेऽप्यसन्धिश्छान्दसः। \textcolor{red}{एओ} इत्युपपदे \textcolor{red}{अञ्च्‌}\-धातोः (\textcolor{red}{अञ्चुँ गतिपूजनयोः} धा॰पा॰~१८८) \textcolor{red}{क्विप् च} (पा॰सू॰~३.२.७६) इत्यनेन \textcolor{red}{क्विप्} प्रत्ययः। सर्वापहारि\-लोपे उपपद\-समासे \textcolor{red}{संयोगान्तस्य लोपः} (पा॰सू॰~८.२.२३) इत्यनेन चकार\-लोपे \textcolor{red}{चोः कुः} (पा॰सू॰~८.२.३०) इत्यनेन ञकारस्य ङत्वे \textcolor{red}{एङः पदान्तादति} (पा॰सू॰~६.१.१०९) इत्यनेन पूर्वरूपैकादेशे विभक्तिकार्ये \textcolor{red}{हल्ङ्याब्भ्यो दीर्घात्सुतिस्यपृक्तं हल्} (पा॰सू॰~६.१.६८) इत्यनेन सुलोपे सिद्धम्।} सीता\-राम\-पूजकमिति तात्पर्यम्।\end{sloppypar}
\begin{sloppypar}\hyphenrules{nohyphenation}\justifying\noindent\hspace{10mm} \textcolor{red}{(४) ऐऔच्}~– \textcolor{red}{ऐ} अष्टप्रकृत्यात्मिका सीता।\footnote{माहेश्वर\-सूत्रेषु \textcolor{red}{ऐ}\-कारस्य क्रमसङ्ख्यानुसारम्। \textcolor{red}{भूमिरापोऽनलो वायुः खं मनो बुद्धिरेव च। अहङ्कार इतीयं मे भिन्ना प्रकृतिरष्टधा॥} (भ॰गी॰~७.४) इत्यष्टधा प्रकृतिः।} \textcolor{red}{औ} नवम\-सङ्ख्या\-वाच्यो रामः।\footnote{माहेश्वर\-सूत्रेषु \textcolor{red}{औ}\-कारस्य क्रमसङ्ख्यानुसारम्। राम\-नवमसङ्ख्ययोः साम्यं पूर्वमेव कथितं प्रणेतृभिः।} तौ चिनोत्यन्तर्भावित\-ण्यर्थतया निश्चाययतीति \textcolor{red}{ऐऔच्}।\footnote{\textcolor{red}{ऐऔ} इत्यत्र समासेऽप्यसन्धिश्छान्दसः। \textcolor{red}{ऐऔ} इत्युपपदे \textcolor{red}{चि}\-धातोः (\textcolor{red}{चिञ् चयने} धा॰पा॰~१२५१) पूर्ववदौणादिको \textcolor{red}{ड्विन्‌}प्रत्ययः। पूर्ववत्सर्वापहारि\-लोपे टिलोप उपपद\-समासे विभक्तिकार्ये \textcolor{red}{हल्ङ्याब्भ्यो दीर्घात्सुतिस्यपृक्तं हल्} (पा॰सू॰~६.१.६८) इत्यनेन सुलोपे सिद्धम्। \textcolor{red}{अयस्मयादीनि च्छन्दसि} (पा॰सू॰~१.४.२०) इत्यनेन छान्दस\-भसञ्ज्ञायां पदत्वाभावे \textcolor{red}{चोः कुः} (पा॰सू॰~८.२.३०) इत्यस्य प्रवृत्तिर्न। यद्वा प्रत्याहारेष्वसन्देहार्थं कुत्वाभावः।} सीता\-राम\-निश्चायकमित्यर्थः।\end{sloppypar}
\begin{sloppypar}\hyphenrules{nohyphenation}\justifying\noindent\hspace{10mm} इत्थं सूत्र\-चतुष्टयेनैश्वर्य\-लीलात्मकं रामायणं प्रतिपादितम्। इदानीं माधुर्य\-लीला\-परं रामायणं बाल\-बुद्ध्या विविच्यते दशभिः सूत्रैः। तत्र माधुर्ये पूर्व\-निर्दिष्ट\-दश\-गुण\-प्रतिपादकानि दश सूत्राणि \textcolor{red}{दशमस्त्वमसि}\footnote{मूलं वेदशाखासु मृग्यम्।} इति श्रुतेर्वाच्यतावच्छेदकस्य लक्ष्यतावच्छेदकस्य भगवतः श्रीरामस्य माधुर्य\-गुण\-बृंहितं सौन्दर्य\-सार\-सर्वस्वं दिव्यं चरित्रं शिवेन लोकोत्तर\-कौशल\-पुरःसरं सङ्केतितम्। अत्र वर्णानां द्वौ विभागौ स्वरो व्यञ्जनञ्च। एवमेव ब्रह्मणो द्वे स्वरूपे निर्गुणं सगुणञ्च। तत्र निर्गुण\-लीला प्रतिपादिता। साम्प्रतं सगुण\-लीलाऽपि प्रतिपाद्यते। पञ्चमात्सूत्राद्व्यञ्जनानां वर्णनं प्रारब्धम्। तत्र व्यञ्जनं सीताया रामेण मिश्रणम्। सगुण\-लीलायां सीतया रामोऽभिन्नः। यथा~–\end{sloppypar}
\centering\textcolor{red}{अनन्या राघवेणाहं भास्करेण यथा प्रभा॥}\nopagebreak\\
\raggedleft{–~वा॰रा॰~५.२१.१५}\\
\centering\textcolor{red}{अनन्या हि मया सीता भास्करेण यथा प्रभा ॥}\nopagebreak\\
\raggedleft{–~वा॰रा॰~६.११८.१९}\\
\begin{sloppypar}\hyphenrules{nohyphenation}\justifying\noindent इति वाल्मीकीये रामायणे श्री\-सीता\-रामाभ्यामुक्तत्वात्। उभयत्र तृतीया प्रकृत्यादित्वात्।\footnote{\textcolor{red}{प्रकृत्यादिभ्य उपसङ्ख्यानम्} (वा॰~२.३.१८) इत्यनेन।} व्यञ्जनानां षड्वर्गा यवर्गः कवर्गश्चवर्गष्टवर्गस्तवर्गः पवर्गश्चेति। एषां कीर्तनेन जीवानां विकार\-षड्वर्ग\-नाशनं सूचितम्। किं बहुना स्वर\-वर्गं मिलित्वा चतुर्दश\-सूत्रेषु सप्त\-वर्गाः। एवं सप्तभिर्वर्गैरक्षराणां सप्त\-काण्डात्मकं रामायणं सुस्पष्टं कीर्तितम्। सूत्राणि सार्वभौमानि विश्वतोमुखानि छन्दः\-स्वरूपाणि भवन्ति। तत्र स्थूलानामक्षराणां निर्देशत्वेऽप्यक्षराणामक्षरस्य भगवतो रामचन्द्रस्य रामायणी गाथा कथं न निर्दिष्टा स्यात्। भगवत्सङ्कीर्तनं विनैषु पुण्य\-जनकतावच्छेदकता कथं स्यात्। यतो रामायण\-कीर्तनं चतुर्दश\-भुवन\-व्यापकं चतुर्दश\-सूत्रेष्वतो लोकोत्तर\-पुण्य\-जनकता। तस्माद्भाष्यकारश्चतुर्दश\-सूत्राणां सानन्दं प्रशंसां कुर्वन्नाह \textcolor{red}{सोऽयमक्षर\-समाम्नायो वाक्समाम्नायः पुष्पितश्चन्द्र\-तारकवत्प्रतिमण्डितो वेदितव्यो ब्रह्म\-राशिः। सर्व\-वेद\-पुण्य\-फलावाप्तिश्चास्य ज्ञाने भवति} (भा॰शि॰सू॰)। तर्हि ब्रह्मणो रामस्य चर्चां विना पूर्वोक्त\-पुण्यजनकताऽन्येषु सन्दिग्धा स्यात्। यतो भाष्य\-प्रमाणम्। यया पुण्यजनकता यतश्च पुण्यजनकता। पूर्वं
चतुर्भिः सूत्रैर्बालकाण्डं चर्चितमिदानीमयोध्याकाण्डमुपक्रमते।\end{sloppypar}
\begin{sloppypar}\hyphenrules{nohyphenation}\justifying\noindent\hspace{10mm} \textcolor{red}{(५) हयवरट्}~– एवं वर्ण\-संयोजनेन हयेषु घोटकेषु वरा \textcolor{red}{हयवराः} श्रेष्ठ\-घोटकास्तैरटति वनमिति \textcolor{red}{हयवरट्}। शकन्ध्वादित्वात्पररूपं सौत्रत्वाद्वा।\footnote{\textcolor{red}{हयवर} उपपदे \textcolor{red}{अटँ गतौ} (धा॰पा॰~३३२) इति धातोः \textcolor{red}{क्विप् च} (पा॰सू॰~३.२.७६) इत्यनेन कर्तरि क्विप्। सर्वापहारि\-लोप उपपद\-समासे \textcolor{red}{शकन्ध्वादिषु पररूपं वाच्यम्} (वा॰~६.१.९४) इत्यनेन पूर्वरूपे विभक्तिकार्ये \textcolor{red}{हल्ङ्याब्भ्यो दीर्घात्सुतिस्यपृक्तं हल्} (पा॰सू॰~६.१.६८) इत्यनेन सुलोपे सिद्धम्।} अथवा \textcolor{red}{हयवर}\-संयोजित\-रथेन पित्रादिष्टः सीता\-लक्ष्मण\-सहायः श्रीरामो वनमटतीति \textcolor{red}{हयवरट्}।\footnote{प्रक्रिया पूर्ववत्।}\end{sloppypar}
\begin{sloppypar}\hyphenrules{nohyphenation}\justifying\noindent\hspace{10mm} \textcolor{red}{(६) लण्}~– लसतीति \textcolor{red}{लः}। सौत्रत्वाट्टिलोपः सलोपश्च।\footnote{\textcolor{red}{लसँ श्लेषण\-क्रीडनयोः} (धा॰पा॰~७१४)~\arrow लस्~\arrow \textcolor{red}{नन्दि\-ग्रहि\-पचादिभ्यो ल्युणिन्यचः} (पा॰सू॰~३.१.१३४)~\arrow लस् अच्~\arrow लस् अ~\arrow लस~\arrow सौत्रटिलोपः~\arrow लस्~\arrow सौत्रसलोपः~\arrow ल~\arrow विभक्तिकार्यम्~\arrow लः। यद्वा लस् धातोः \textcolor{red}{अन्येष्वपि दृश्यते} (पा॰सू॰~३.२.१०१) इत्यनेनोपपदाभावेऽपि डप्रत्ययः। लस् ड~\arrow लस् अ~\arrow \textcolor{red}{डित्यभस्याप्यनु\-बन्धकरण\-सामर्थ्यात्} (वा॰~६.४.१४३)~\arrow ल् अ~\arrow ल~\arrow विभक्तिकार्यम्~\arrow लः।} तस्मिन् ले प्रकृति\-सौन्दर्य\-लसिते चित्रकूटे भक्तानानन्दं नयतीति कष्टान्निवर्तयति वेति \textcolor{red}{लण्} चित्रकूटस्थो रामः। सौत्रत्वाण्णत्वम्।\footnote{\textcolor{red}{ल} इत्युपपदे \textcolor{red}{नी}\-धातोः (\textcolor{red}{णीञ् प्रापणे} धा॰पा॰~९०१) पूर्ववदौणादिको \textcolor{red}{ड्विन्‌}\-प्रत्ययः। पूर्ववत्सर्वापहारि\-लोपे टिलोपे सौत्रणत्व उपपद\-समासे विभक्तिकार्ये \textcolor{red}{हल्ङ्याब्भ्यो दीर्घात्सुतिस्यपृक्तं हल्} (पा॰सू॰~६.१.६८) इत्यनेन सुलोपे सिद्धम्।} यद्वा लाति भक्तिं यः स \textcolor{red}{लः} चित्र\-कूटः।\footnote{\textcolor{red}{ला आदाने। द्वावपि (रा ला) दाने इति चन्द्रः} (धा॰पा॰~१०५८)। ला~\arrow \textcolor{red}{आतश्चोपसर्गे} (पा॰सू॰~३.१.१३६)~\arrow बाहुलकादनुपसर्गे कः~\arrow ला~क~\arrow ला~अ~\arrow \textcolor{red}{आतो लोप इटि च} (पा॰सू॰~६.४.६४)~\arrow ल्~अ~\arrow ल~\arrow विभक्तिकार्यम्~\arrow लः। यद्वा ला~\arrow \textcolor{red}{अन्येष्वपि दृश्यते} (पा॰सू॰~३.२.१०१)~\arrow ला~ड~\arrow ला~अ~\arrow \textcolor{red}{डित्यभस्याप्यनु\-बन्धकरण\-सामर्थ्यात्} (वा॰~६.४.१४३)~\arrow ल्~अ~\arrow ल~\arrow विभक्तिकार्यम्~\arrow लः। ददातीति दः (\textcolor{red}{पुमांस्तु दातरि स्मृतः} मे॰को॰~१८.१) इतिवल्लातीति लः।} तस्मिन्नाशयति भक्तकष्टं जयन्त\-दर्पञ्च यः स \textcolor{red}{लण्} चित्रकूटस्थः श्रीरामः। सौत्रत्वाट्टिलोपो णत्वञ्च।\footnote{\textcolor{red}{ल} उपपदे \textcolor{red}{णशँ अदर्शने} (धा॰पा॰~११९४) इत्यतो णिजन्तात् \textcolor{red}{नाशि} धातोः \textcolor{red}{क्विप् च} (पा॰सू॰~३.२.७६) इत्यनेन कर्तरि क्विप्। सर्वापहारि\-लोपे \textcolor{red}{णेरनिटि} (पा॰सू॰~६.४.५१) इत्यनेन णिलोपे \textcolor{red}{नाश्} इति जाते सौत्रटिलोपे सौत्रणत्वे चोपपद\-समासे विभक्तिकार्ये \textcolor{red}{हल्ङ्याब्भ्यो दीर्घात्सुतिस्यपृक्तं हल्} (पा॰सू॰~६.१.६८) इत्यनेन सुलोपे सिद्धम्।}\end{sloppypar}
\begin{sloppypar}\hyphenrules{nohyphenation}\justifying\noindent\hspace{10mm} \textcolor{red}{(७) ञमङणनम्}~– प्रतिवर्गान्तिम\-भूतान् खर\-दूषण\-त्रिशीर्ष\-मारीच\-कबन्धान्मीनातीति \textcolor{red}{ञमङणनम्} श्रीरामः।\footnote{\textcolor{red}{ञमङणन} इत्युपपदे \textcolor{red}{मी}\-धातोः (\textcolor{red}{मीञ् हिंसायाम्} धा॰पा॰~१४७६) पूर्ववदौणादिको \textcolor{red}{ड्विन्‌}\-प्रत्ययः। पूर्ववत्सर्वापहारि\-लोपे टिलोप उपपदसमासे विभक्तिकार्ये \textcolor{red}{हल्ङ्याब्भ्यो दीर्घात्सुतिस्यपृक्तं हल्} (पा॰सू॰~६.१.६८) इत्यनेन सुलोपे सिद्धम्।}\end{sloppypar}
\begin{sloppypar}\hyphenrules{nohyphenation}\justifying\noindent\hspace{10mm} \textcolor{red}{(८) झभञ्}~– \textcolor{red}{झॄष् वयो\-हानौ} (धा॰पा॰~११३१)। झीर्यतीति \textcolor{red}{झः}। सुग्रीवः। वालि\-त्रासाज्जीर्णो भवतीति भावः।\footnote{\textcolor{red}{झॄष् वयो\-हानौ} (धा॰पा॰~११३१)~\arrow झॄ~\arrow \textcolor{red}{अन्येष्वपि दृश्यते} (पा॰सू॰~३.२.१०१)~\arrow झॄ~ड~\arrow झॄ~अ~\arrow \textcolor{red}{डित्यभस्याप्यनु\-बन्धकरण\-सामर्थ्यात्} (वा॰~६.४.१४३)~\arrow झ्~अ~\arrow झ~\arrow विभक्तिकार्यम्~\arrow झः।} भातीति \textcolor{red}{भः}। हनुमान्।\footnote{\textcolor{red}{भा दीप्तौ} (धा॰पा॰~१०५१)~\arrow भा~\arrow \textcolor{red}{अन्येष्वपि दृश्यते} (पा॰सू॰~३.२.१०१)~\arrow भा~ड~\arrow भा~अ~\arrow \textcolor{red}{डित्यभस्याप्यनु\-बन्धकरण\-सामर्थ्यात्} (वा॰~६.४.१४३)~\arrow भ्~अ~\arrow भ~\arrow विभक्तिकार्यम्~\arrow भः।} तौ झभौ सुग्रीव\-हनुमन्तौ यन्त्रयत्यानन्देनेति \textcolor{red}{झभञ्} श्रीरामः। सौत्रत्वाद्यकारस्य ञकारः।\footnote{\textcolor{red}{झभ} उपपदे \textcolor{red}{यत्रिँ सङ्कोचे} (धा॰पा॰~१५३६) इति धातोः पूर्ववदौणादिको \textcolor{red}{ड्विन्‌}\-प्रत्ययः। पूर्ववत्सर्वापहारि\-लोपे टिलोप उपपद\-समासे सौत्रत्वाद्यकारस्य ञकारे विभक्ति\-कार्ये \textcolor{red}{हल्ङ्याब्भ्यो दीर्घात्सुतिस्यपृक्तं हल्} (पा॰सू॰~६.१.६८) इत्यनेन सुलोपे सिद्धम्। \textcolor{red}{अयस्मयादीनि च्छन्दसि} (पा॰सू॰~१.४.२०) इत्यनेन छान्दस\-भसञ्ज्ञायां पदत्वाभावे \textcolor{red}{चोः कुः} (पा॰सू॰~८.२.३०) इत्यस्य प्रवृत्तिर्न।} सुग्रीव\-वायु\-पुत्र\-तोष\-कर्तेति तात्पर्यम्।\end{sloppypar}
\begin{sloppypar}\hyphenrules{nohyphenation}\justifying\noindent\hspace{10mm} \textcolor{red}{(९) घढधष्}~– घढमभिमानं\footnote{व्युत्पत्तिः साध्या।} दधातीति \textcolor{red}{घढधः} वाली।\footnote{\textcolor{red}{घढ} उपपदे \textcolor{red}{डुधाञ् धारण\-पोषणयोः} (धा॰पा॰~१०९२) इति धातोः \textcolor{red}{अन्येष्वपि दृश्यते} (पा॰सू॰~३.२.१०१) इत्यनेन \textcolor{red}{ड}\-प्रत्ययः। अनुबन्ध\-लोपे \textcolor{red}{डित्यभस्याप्यनु\-बन्धकरण\-सामर्थ्यात्} (वा॰~६.४.१४३) इत्यनेन टिलोप उपपदसमासे विभक्तिकार्ये सिद्धम्।} तमेव स्यति खण्डयतीति \textcolor{red}{घढधष्}।\footnote{\textcolor{red}{घढध} उपपदे \textcolor{red}{षो अन्तकर्मणि} (धा॰पा॰~११४७) इति धातोः पूर्ववदौणादिको \textcolor{red}{ड्विन्‌}\-प्रत्ययः। पूर्ववत्सर्वापहारि\-लोपे टिलोप उपपद\-समासे सौत्रत्वात्सकारस्य षकारे विभक्ति\-कार्ये \textcolor{red}{हल्ङ्याब्भ्यो दीर्घात्सुतिस्यपृक्तं हल्} (पा॰सू॰~६.१.६८) इत्यनेन सुलोपे सिद्धम्। \textcolor{red}{अयस्मयादीनि च्छन्दसि} (पा॰सू॰~१.४.२०) इत्यनेन छान्दस\-भसञ्ज्ञायां पदत्वाभावे \textcolor{red}{व्रश्चभ्रस्ज\-सृजमृज\-यजराज\-भ्राजच्छशां षः} (पा॰सू॰~८.२.३६) \textcolor{red}{झलां जशोऽन्ते} (पा॰सू॰~८.२.३९) इत्यनयोः प्रवृत्तिर्न।} वालि\-नाशको राम इति तात्पर्यम्।\end{sloppypar}
\begin{sloppypar}\hyphenrules{nohyphenation}\justifying\noindent\hspace{10mm} \textcolor{red}{(१०) जबगडदश्}~– जयतीति \textcolor{red}{जः}। बालयतीति \textcolor{red}{बः}। गर्जतीति \textcolor{red}{गः}। डम्बयतीति \textcolor{red}{डः}। दमयति निशाचरानिति \textcolor{red}{दः}।\footnote{नाना\-धातुभ्यः \textcolor{red}{अन्येष्वपि दृश्यते} (पा॰सू॰~३.२.१०१) इत्यनेन \textcolor{red}{ड}\-प्रत्यये \textcolor{red}{डित्यभस्याप्यनु\-बन्धकरण\-सामर्थ्यात्} (वा॰~६.४.१४३) इत्यनेन टिलोप उपपदसमासे विभक्तिकार्ये सिद्धानि पञ्चैतानि। \textcolor{red}{जि जये} (धा॰पा॰~५६१) इति धातोः \textcolor{red}{जः}। \textcolor{red}{बलँ प्राणने धान्यावरोधने च} (धा॰पा॰~८४०) इत्यतो णिचि \textcolor{red}{बालि} धातुः। ततः \textcolor{red}{ड}\-प्रत्यये \textcolor{red}{बः}। \textcolor{red}{णेरनिटि} (पा॰सू॰~६.४.५१) इत्यनेन णिलोपः शेषा प्रक्रिया पूर्ववत्। न च \textcolor{red}{बलँ प्राणने} (धा॰पा॰~१६२८) इति धातोर्णिचि डे णिलोपे टिलोपे विभक्तिकार्येऽपि सिद्धिः। \textcolor{red}{बलँ प्राणने} (धा॰पा॰~१६२८) इत्यस्य ज्ञपादित्वान्मित्त्वम्। ततो णिच्युपधा\-वृद्धौ \textcolor{red}{मितां ह्रस्वः} (पा॰सू॰~६.४.९२) इत्यनेन ह्रस्वे \textcolor{red}{बलयति} इति रूपं प्रणेतारस्तु \textcolor{red}{बालयति} इत्याहुः। \textcolor{red}{गर्जँ शब्दे} (धा॰पा॰~२२६) इति धातोः \textcolor{red}{गृजँ शब्दे} (धा॰पा॰~२४८) इत्यतो वा \textcolor{red}{गः}। \textcolor{red}{डिपँ सङ्घाते डबिँ डिबिँ इति चान्द्रः} (धा॰पा॰~१६७७) इत्यत्र \textcolor{red}{डम्ब्‌}\-(\textcolor{red}{डबिँ})\-धातोः \textcolor{red}{डः}। \textcolor{red}{दमुँ उपशमे} (धा॰पा॰~१२०३) इत्यतो णिचि \textcolor{red}{दमि} धातुः। \textcolor{red}{जनी\-जॄष्क्नसु\-रञ्जोऽमन्ताश्च} (धा॰पा॰ ग॰सू॰) इत्यनेन मित्त्वादुपधा\-वृद्धौ \textcolor{red}{मितां ह्रस्वः} (पा॰सू॰~६.४.९२) इत्यनेन ह्रस्वः। \textcolor{red}{दमि} धातोः \textcolor{red}{दः}। \textcolor{red}{णेरनिटि} (पा॰सू॰~६.४.५१) इत्यनेन णिलोपः शेषा प्रक्रिया पूर्ववत्।} एतत्पञ्च\-गुण\-सम्पन्न\-जबगडदस्य हृदये शेत इति \textcolor{red}{जबगडदश्}। वायु\-पुत्र\-हृदयस्थो रामः।\footnote{\textcolor{red}{जबगडद} उपपदे \textcolor{red}{शीङ् स्वप्ने} (धा॰पा॰~१०३२) इति धातोः पूर्ववदौणादिको \textcolor{red}{ड्विन्‌}\-प्रत्ययः। पूर्ववत्सर्वापहारि\-लोपे टिलोप उपपद\-समासे विभक्ति\-कार्ये \textcolor{red}{हल्ङ्याब्भ्यो दीर्घात्सुतिस्यपृक्तं हल्} (पा॰सू॰~६.१.६८) इत्यनेन सुलोपे सिद्धम्। \textcolor{red}{अयस्मयादीनि च्छन्दसि} (पा॰सू॰~१.४.२०) इत्यनेन छान्दस\-भसञ्ज्ञायां पदत्वाभावे \textcolor{red}{झलां जशोऽन्ते} (पा॰सू॰~८.२.३९) इत्यस्य प्रवृत्तिर्न।}\end{sloppypar}
\begin{sloppypar}\hyphenrules{nohyphenation}\justifying\noindent\hspace{10mm} \textcolor{red}{(११) खफछठथचटतव्}~– खनतीति \textcolor{red}{खः}। फलतीति \textcolor{red}{फः}। छ्यतीति \textcolor{red}{छः}।\footnote{नाना\-धातुभ्यः \textcolor{red}{अन्येष्वपि दृश्यते} (पा॰सू॰~३.२.१०१) इत्यनेन \textcolor{red}{ड}\-प्रत्यये \textcolor{red}{डित्यभस्याप्यनु\-बन्धकरण\-सामर्थ्यात्} (वा॰~६.४.१४३) इत्यनेन टिलोपे विभक्तिकार्ये सिद्धानि त्रीण्येतानि। \textcolor{red}{खनुँ अवदारणे} (धा॰पा॰~८७८) इति धातोः \textcolor{red}{खः}। \textcolor{red}{ञिफलाँ विशरणे} (धा॰पा॰~५१६) इति धातोः \textcolor{red}{फः}। \textcolor{red}{छो छेदने} (धा॰पा॰~११४६) इति धातोः \textcolor{red}{छः}।} लोठतीति \textcolor{red}{ठः}। लुकार\-लोपश्छान्दसः।\footnote{\textcolor{red}{लुठँ उपघाते} (धा॰पा॰~३३७) इति धातोः \textcolor{red}{नन्दि\-ग्रहि\-पचादिभ्यो ल्युणिन्यचः} (पा॰सू॰~३.१.१३४) इत्यनेन कर्तरि पचाद्यचि छान्दस\-लुकार\-लोपे विभक्तिकार्ये सिद्धम्।} थूर्वतीति \textcolor{red}{थः}। चिनोति दुर्गुणानिति \textcolor{red}{चः}। टङ्कयत्यसद्विचारानिति \textcolor{red}{टः}। ताम्यतीति \textcolor{red}{तः}।\footnote{नाना\-धातुभ्यः \textcolor{red}{अन्येष्वपि दृश्यते} (पा॰सू॰~३.२.१०१) इत्यनेन \textcolor{red}{ड}\-प्रत्यये \textcolor{red}{डित्यभस्याप्यनु\-बन्धकरण\-सामर्थ्यात्} (वा॰~६.४.१४३) इत्यनेन टिलोपे विभक्तिकार्ये सिद्धानि चत्वार्येतानि। \textcolor{red}{थुर्वीँ हिंसायाम्} (धा॰पा॰~५७१) इति धातोः \textcolor{red}{थः}। \textcolor{red}{चिञ् चयने} (धा॰पा॰~१२५१) इति धातोः \textcolor{red}{चः}। \textcolor{red}{टकिँ बन्धने} (धा॰पा॰~१६३८) इति धातोः \textcolor{red}{टः}। \textcolor{red}{णेरनिटि} (पा॰सू॰~६.४.५१) इत्यनेन णिलोपः शेषा प्रक्रिया पूर्ववत्। \textcolor{red}{तमुँ काङ्क्षायाम्} (धा॰पा॰~१२०२) इति धातोः \textcolor{red}{तः}।} एवं खफछठथचटतं रावणमपि वधतीति \textcolor{red}{खफछठथचटतव्}।\footnote{\textcolor{red}{वधति} इत्यत्र \textcolor{red}{वधँ हिंसायाम्} भौवादिकः सौत्रो धातुः। स च \textcolor{red}{जनिवध्योश्च} (पा॰सू॰~७.३.३५) इति सूत्रेण ज्ञापितः। \textcolor{red}{‘जनिवध्योश्च’। जनकः। ‘वधँ हिंसायाम्’। वधकः} (वै॰सि॰कौ॰~२८९५)। \textcolor{red}{‘वधँ हिंसायामिति’। धात्वन्तरं भौवादिकम्। भ्वादेराकृतिगणत्वात्} (बा॰म॰~२८९५)। एवं तर्हि \textcolor{red}{खफछठथचटत} उपपदे \textcolor{red}{वधँ हिंसायाम्} इति धातोः पूर्ववदौणादिको \textcolor{red}{ड्विन्‌}\-प्रत्ययः। पूर्ववत्सर्वापहारि\-लोपे  टिलोप उपपद\-समासे विभक्ति\-कार्ये \textcolor{red}{हल्ङ्याब्भ्यो दीर्घात्सुतिस्यपृक्तं हल्} (पा॰सू॰~६.१.६८) इत्यनेन सुलोपे सिद्धम्।} अथवा खफछठथचटत\-ध्वनि\-कुर्वाणं युद्धाय स्वकीयं चाप\-विशेषं वर्धयतीति \textcolor{red}{खफछठथचटतव्}।\footnote{\textcolor{red}{वृधुँ वृद्धौ} (धा॰पा॰~७५९) इति धातोर्णिचि \textcolor{red}{वर्धि} इति धातुः। \textcolor{red}{खफछठथचटत} उपपदे \textcolor{red}{वर्धि}\-धातोः पूर्ववदौणादिको \textcolor{red}{ड्विन्‌}\-प्रत्ययः। पूर्ववत्सर्वापहारि\-लोपे \textcolor{red}{णेरनिटि} (पा॰सू॰~६.४.५१) इत्यनेन णिलोपे टिलोप उपपद\-समासे विभक्ति\-कार्ये \textcolor{red}{हल्ङ्याब्भ्यो दीर्घात्सुतिस्यपृक्तं हल्} (पा॰सू॰~६.१.६८) इत्यनेन सुलोपे सिद्धम्।} अथवैतद्ध्वनि\-युक्तं वानर\-दलमवतीति \textcolor{red}{खफछठथचटतव्}। शकन्ध्वादित्वात्पूर्वरूपम्।\footnote{\textcolor{red}{खफछठथचटत} उपपदे \textcolor{red}{अवँ रक्षण\-गति\-कान्ति\-प्रीति\-तृप्त्यवगम\-प्रवेश\-श्रवण\-स्वाम्यर्थ\-याचन\-क्रियेच्छा\-दीप्त्यवाप्त्यालिङ्गन\-हिंसा\-दान\-भाग\-वृद्धिषु} (धा॰पा॰~६००) इति धातोः \textcolor{red}{क्विप् च} (पा॰सू॰~३.२.७६) इत्यनेन \textcolor{red}{क्विप्} प्रत्यये सर्वापहारि\-लोप उपपदसमासे \textcolor{red}{शकन्ध्वादिषु पर\-रूपं वाच्यम्} इत्यनेन पररूपे विभक्ति\-कार्ये \textcolor{red}{हल्ङ्याब्भ्यो दीर्घात्सुतिस्यपृक्तं हल्} (पा॰सू॰~६.१.६८) इत्यनेन सुलोपे सिद्धम्।}\end{sloppypar}
\begin{sloppypar}\hyphenrules{nohyphenation}\justifying\noindent\hspace{10mm} \textcolor{red}{(१२) कपय्}~– कायति विभीषणं राज्य\-दानाय शब्दापयति\footnote{\textcolor{red}{कायतीति कः} इत्यर्थः। \textcolor{red}{कै शब्दे} (धा॰पा॰~९१६)~\arrow कै~\arrow \textcolor{red}{अन्येष्वपि दृश्यते} (पा॰सू॰~३.२.१०१)~\arrow कै~ड~\arrow कै~अ~\arrow \textcolor{red}{डित्यभस्याप्यनु\-बन्धकरण\-सामर्थ्यात्} (वा॰~६.४.१४३)~\arrow क्~अ~\arrow क~\arrow विभक्तिकार्यम्~\arrow कः।} पाति वानर\-सेनां पुष्पकारोहेण\footnote{\textcolor{red}{पातीति पः} इत्यर्थः। \textcolor{red}{पा रक्षणे} (धा॰पा॰~१०५६)~\arrow पा~\arrow \textcolor{red}{अन्येष्वपि दृश्यते} (पा॰सू॰~३.२.१०१)~\arrow पा~ड~\arrow पा~अ~\arrow \textcolor{red}{डित्यभस्याप्यनु\-बन्धकरण\-सामर्थ्यात्} (वा॰~६.४.१४३)~\arrow प्~अ~\arrow प~\arrow विभक्तिकार्यम्~\arrow पः। \textcolor{red}{पः स्यात्पाने च पातरि} (ए॰को॰~२४) इति कोशादपि।} यात्ययोध्याम्\footnote{\textcolor{red}{यातीति य्} इत्यर्थः। \textcolor{red}{या प्रापणे} (धा॰पा॰~१०४९) इति धातोः पूर्ववदौणादिको \textcolor{red}{ड्विन्‌}\-प्रत्ययः। पूर्ववत्सर्वापहारि\-लोपे टिलोपे विभक्ति\-कार्ये \textcolor{red}{हल्ङ्याब्भ्यो दीर्घात्सुतिस्यपृक्तं हल्} (पा॰सू॰~६.१.६८) इत्यनेन सुलोपे \textcolor{red}{य्}। \textcolor{red}{याने यातरि यस्त्यागे} (ए॰को॰~२९) इति कोशादकारान्तोऽपि यातरि।} इति \textcolor{red}{कपय्}।\footnote{\textcolor{red}{कश्च पश्च य् चेति कपय्}।} दत्त\-विभीषण\-राज्य\-लक्ष्मीः पुष्पकारूढः सीताभिरामो रामोऽयोध्यां प्रति प्रतिष्ठत इति भावः।\end{sloppypar}
\begin{sloppypar}\hyphenrules{nohyphenation}\justifying\noindent\hspace{10mm} \textcolor{red}{(१३) शषसर्}~– तथा श्यति तनूकरोति सीता\-तापं\footnote{\textcolor{red}{श्यतीति शः} इत्यर्थः। \textcolor{red}{शो तनूकरणे} (धा॰पा॰~११४५)~\arrow शो~\arrow \textcolor{red}{अन्येष्वपि दृश्यते} (पा॰सू॰~३.२.१०१)~\arrow शो~ड~\arrow शो~अ~\arrow \textcolor{red}{डित्यभस्याप्यनु\-बन्धकरण\-सामर्थ्यात्} (वा॰~६.४.१४३)~\arrow श्~अ~\arrow श~\arrow विभक्तिकार्यम्~\arrow शः।} स्यति रावणं\footnote{\textcolor{red}{स्यतीति षः} इत्यर्थः। \textcolor{red}{षो अन्तकर्मणि} (धा॰पा॰~११४७)~\arrow षो~\arrow \textcolor{red}{धात्वादेः षः सः} (पा॰सू॰~६.१.६४)~\arrow सो~\arrow \textcolor{red}{अन्येष्वपि दृश्यते} (पा॰सू॰~३.२.१०१)~\arrow सो~ड~\arrow सो~अ~\arrow \textcolor{red}{डित्यभस्याप्यनु\-बन्धकरण\-सामर्थ्यात्} (वा॰~६.४.१४३)~\arrow स्~अ~\arrow स~\arrow सौत्रषकारः~\arrow ष~\arrow विभक्तिकार्यम्~\arrow षः।} सरत्ययोध्यां\footnote{\textcolor{red}{सरतीति सः} इत्यर्थः। \textcolor{red}{सृ गतौ} (धा॰पा॰~९३५)~\arrow सृ~\arrow \textcolor{red}{अन्येष्वपि दृश्यते} (पा॰सू॰~३.२.१०१)~\arrow सृ~ड~\arrow सृ~अ~\arrow \textcolor{red}{डित्यभस्याप्यनु\-बन्धकरण\-सामर्थ्यात्} (वा॰~६.४.१४३)~\arrow स्~अ~\arrow स~\arrow विभक्तिकार्यम्~\arrow सः।} रमयति सीतां रमते च स्वयं राजते वा राज\-सिंहासने\footnote{\textcolor{red}{रमयति रमते राजते वेति र्} इत्यर्थः। प्रथमपक्षे \textcolor{red}{रमुँ क्रीडायाम्} (धा॰पा॰~८५३) इत्यतो णिचि \textcolor{red}{रमि} धातुः। \textcolor{red}{जनी\-जॄष्क्नसु\-रञ्जोऽमन्ताश्च} (धा॰पा॰ ग॰सू॰) इत्यनेन मित्त्वादुपधा\-वृद्धौ \textcolor{red}{मितां ह्रस्वः} (पा॰सू॰~६.४.९२) इत्यनेन ह्रस्वः। द्वितीयपक्षे \textcolor{red}{रमुँ क्रीडायाम्} (धा॰पा॰~८५३) इति शुद्धो धातुः। तृतीयपक्षे \textcolor{red}{राजृँ दीप्तौ} (धा॰पा॰~८२२) इति धातुः। एतेभ्यः पूर्ववदौणादिको \textcolor{red}{ड्विन्‌}\-प्रत्ययः। प्रथमपक्षे \textcolor{red}{णेरनिटि} (पा॰सू॰~६.४.५१) इत्यनेन णिलोपः। पूर्ववत्सर्वापहारि\-लोपे टिलोपे विभक्ति\-कार्ये \textcolor{red}{हल्ङ्याब्भ्यो दीर्घात्सुतिस्यपृक्तं हल्} (पा॰सू॰~६.१.६८) इत्यनेन सुलोपे \textcolor{red}{र्}। \textcolor{red}{अयस्मयादीनि च्छन्दसि} (पा॰सू॰~१.४.२०) इत्यनेन छान्दस\-भसञ्ज्ञायां पदत्वाभावे \textcolor{red}{झलां जशोऽन्ते} (पा॰सू॰~८.२.३९) इत्यस्य प्रवृत्तिर्न।
} यः स \textcolor{red}{शषसर्}।\footnote{\textcolor{red}{शश्च षश्च सश्च र् चेति शषसर्}।} रावण\-वधं विधाय राज\-सिंहासनासीनो राम इति तात्पर्यम्। सौत्रत्वात्सकारस्य षकारः।\end{sloppypar}
\begin{sloppypar}\hyphenrules{nohyphenation}\justifying\noindent\hspace{10mm} \textcolor{red}{(१४) हल्}~– हरति भक्त\-तापं\footnote{\textcolor{red}{हरतीति हः} इत्यर्थः। \textcolor{red}{हृञ् हरणे} (धा॰पा॰~८९९)~\arrow हृ~\arrow \textcolor{red}{अन्येष्वपि दृश्यते} (पा॰सू॰~३.२.१०१)~\arrow हृ~ड~\arrow हृ~अ~\arrow \textcolor{red}{डित्यभस्याप्यनु\-बन्धकरण\-सामर्थ्यात्} (वा॰~६.४.१४३)~\arrow ह्~अ~\arrow ह~\arrow विभक्तिकार्यम्~\arrow हः।} लिङ्गति यः सीतां लिङ्ग्यते वा सीतया लीयते वा भक्तानां हृदि\footnote{\textcolor{red}{लिङ्गति लिङ्ग्यते लीयते वेति ल्} इत्यर्थः। प्रथम\-द्वितीय\-पक्षयोः \textcolor{red}{लिगिँ गतौ} (धा॰पा॰~१५५) इति धातुः। तृतीयपक्षे \textcolor{red}{लीङ् श्लेषणे} (धा॰पा॰~११३९) इति धातुः। प्रथम\-तृतीय\-पक्षयोः कर्तरि द्वितीय\-पक्षे च कर्मण्यौणादिको \textcolor{red}{ड्विन्‌}\-प्रत्ययः। \textcolor{red}{कार्याद्विद्यादनूबन्धम्} (भा॰पा॰सू॰~३.३.१) \textcolor{red}{केचिदविहिता अप्यूह्याः} (वै॰सि॰कौ॰~३१६९) इत्यनुसारमूह्योऽ\-यमविहित\-प्रत्ययः। सर्वापहारि\-लोपे \textcolor{red}{डित्यभस्याप्यनु\-बन्धकरण\-सामर्थ्यात्} (वा॰~६.४.१४३) इत्यनेन टिलोपे विभक्ति\-कार्ये \textcolor{red}{हल्ङ्याब्भ्यो दीर्घात्सुतिस्यपृक्तं हल्} (पा॰सू॰~६.१.६८) इत्यनेन सुलोपे \textcolor{red}{ल्}।} यः स \textcolor{red}{हल्}।\footnote{\textcolor{red}{हश्च ल् चेति हल्}।}\end{sloppypar}
\begin{sloppypar}\hyphenrules{nohyphenation}\justifying\noindent\hspace{10mm} इति बाल\-बुद्धि\-प्रतिपादित\-व्युत्पत्ति\-परक\-चतुर्दश\-सूत्री रामायण\-कथा।\end{sloppypar}
\begin{sloppypar}\hyphenrules{nohyphenation}\justifying\noindent\hspace{10mm} हकारेण प्रारभ्य हकारेणैवोपसंहारः। अपवर्गस्योभयत्र चर्चया चतुर्दश\-सूत्र\-रामायणमपवर्गायेति ध्वन्यते। एवमेव नव\-स्वरैर्नव\-सङ्ख्या\-वाच्य\-पूर्ण\-ब्रह्म\-श्रीरामस्य चर्चां विधाय पुनरेक\-त्रिंशद्वर्णै रावण\-वध\-काले भगवता धनुषि संहितानां रामस्यैक\-त्रिंशद्बाणानां\footnote{\textcolor{red}{आकरषेउ धनु श्रवण लगि छाँड़े शर एकतीस। रघुनायक सायक चले मानहुँ काल फनीस॥} (रा॰च॰मा॰~६.१०२)। एतद्रूपान्तरम्–\textcolor{red}{कर्णान्तमाकृष्य धनुः शरान् स त्रिंशन्मितानेकयुतान् व्यमुञ्चत्। शराश्चलन्ति स्म रघुप्रभोस्ते भुजङ्गमेशा ननु कालरूपाः॥} (मा॰भा॰~६.१०२)।} स्मरणेनैक\-त्रिंशद्विकार\-नाशोऽपि सूचितः। पुनर्द्वाभ्यां वर्णाभ्यां श्रीसीता\-राम\-स्मरणेन ज्ञानिनां मोक्षो भक्तानां भक्तिश्च सूचिता। व्यञ्जनेषु हकारेण प्रारभ्य हकारेणोपसंहारस्य तात्पर्यं यत् \textcolor{red}{हकारः} हरि\-वाचको हरिश्च पापानि हरत्यतः पूर्वम् \textcolor{red}{अइउण्} इत्यत्राकारेण वासुदेवस्य रामस्य स्मरणं\footnote{\textcolor{red}{अक्षराणामकारोऽस्मि} (भ॰गी॰~१०.३३)। \textcolor{red}{अकारो वासुदेवः स्यात्} (ए॰को॰~१)।} मध्ये हकारेण हरि\-स्मरणमन्ते च हकारेण हरि\-स्मरणमित्यादौ\footnote{\textcolor{red}{नामैकदेशग्रहणे नाममात्रग्रहणम्} इत्यनेन न्यायेन हकारो हरिवाचकः।} मध्येऽन्ते च मङ्गलाचरणं सङ्केत्य रामायणी मर्यादाऽपि सुरक्षिता। तथा चोक्तम्~–\end{sloppypar}
\centering\textcolor{red}{वेदे रामायणे चैव पुराणे भारते तथा।\nopagebreak\\
आदावन्ते च मध्ये च हरिः सर्वत्र गीयते॥}\nopagebreak\\
\raggedleft{–~क॰पु॰~२१.३७}\\
\begin{sloppypar}\hyphenrules{nohyphenation}\justifying\noindent अकारमुद्दिश्य श्रीराम\-स्मरणतः प्रारभ्यान्ते \textcolor{red}{हल्} इति लकारेण लक्ष्मणः स्मृतः।\footnote{\textcolor{red}{नामैकदेशग्रहणे नाममात्रग्रहणम्} इत्यनेन न्यायेन लकारो लक्ष्मणवाचकः।}\end{sloppypar}
\begin{sloppypar}\hyphenrules{nohyphenation}\justifying\noindent\hspace{10mm} यद्यपि चतुर्दश\-सूत्राणां द्विरुच्चारित\-हकारस्य सन्ति विविधेषु प्रयोगेषु प्रयोजनानि यथा \textcolor{red}{हयवरट्} इत्यत्र हकारः \textcolor{red}{अट्‌हश्‌अश्‌इण्‌}\-ग्रहणेषु सूत्रेषु हकार\-ग्रहणार्थः। तथा च \textcolor{red}{हयवरट्} इत्यत्र हकार\-ग्रहणाभावे तस्य च \textcolor{red}{अट्‌}\-प्रत्याहारे श्रवणाभावे \textcolor{red}{महाँ हि सः} इत्यत्र रुत्वानु\-नासिकत्वे न स्याताम्।\footnote{महान्~हि~सः~\arrow \textcolor{red}{दीर्घादटि समानपदे} (पा॰सू॰~८.३.९)~\arrow महारुँ~हि~सः~\arrow \textcolor{red}{आतोऽटि नित्यम्} (पा॰सू॰~८.३.३)~\arrow महाँरुँ~हि~सः~\arrow \textcolor{red}{भोभगोअघो\-अपूर्वस्य योऽशि} (पा॰सू॰~८.३.१७)~\arrow महाँय्~हि~सः~\arrow \textcolor{red}{लोपः शाकल्यस्य} (पा॰सू॰~८.३.१९)~\arrow महाँ~हि~सः।} \textcolor{red}{अर्हेण} इत्यत्र च णत्वं न स्यात्।\footnote{अर्ह~टा~\arrow \textcolor{red}{ टाङसिङसामिनात्स्याः} (पा॰सू॰~७.१.१२)~\arrow अर्ह~इन~\arrow \textcolor{red}{आद्गुणः} (पा॰सू॰~६.१.८७)~\arrow अर्हेन~\arrow \textcolor{red}{अट्कुप्वाङ्नुम्व्यवायेऽपि} (पा॰सू॰~८.४.२)~\arrow अर्हेण।} हयवरट्सूत्रे हकारो न स्यात्तदा \textcolor{red}{हश्‌}\-प्रत्याहार एव न स्यात्। \textcolor{red}{रामो हसति} इत्यत्र हकारस्य हश्परकत्वाभावे \textcolor{red}{हशि च} (पा॰सू॰~६.१.११४) इत्यनेनात उत्वं न स्यात्।\footnote{रामस्~हसति~\arrow \textcolor{red}{ससजुषो रुः} (पा॰सू॰~८.२.६६)~\arrow रामरुँ~हसति~\arrow \textcolor{red}{हशि च} (पा॰सू॰~६.१.११४)~\arrow राम~उ~हसति~\arrow \textcolor{red}{आद्गुणः} (पा॰सू॰~६.१.८७)~\arrow रामो~हसति।} एवमेव \textcolor{red}{भोभगोअघो\-अपूर्वस्य योऽशि} (पा॰सू॰~८.३.१७) इति सूत्रेण रोर्यत्वं न स्यात्।\footnote{अस्योदाहरणम्~– भोस्~हरे~\arrow \textcolor{red}{ससजुषो रुः} (पा॰सू॰~८.२.६६)~\arrow भोरुँ~हरे~\arrow भोभगोअघो\-अपूर्वस्य \textcolor{red}{योऽशि} (पा॰सू॰~८.३.१७)~\arrow भोय्~हरे~\arrow \textcolor{red}{हलि सर्वेषाम्} (पा॰सू॰~८.३.२२)~\arrow भो~हरे।} एवमेव \textcolor{red}{हयवरट्} इति सूत्रे हकार\-ग्रहणं विना \textcolor{red}{लिलिहिढ्वे} इत्यत्र \textcolor{red}{विभाषेटः} (पा॰सू॰~८.३.७९) इत्यनेनेण्लक्षणो ढकारो वैकल्पिको न स्यात्।\footnote{\textcolor{red}{लिहँ आस्वादने} (धा॰पा॰~१०१६)~\arrow लिह्~\arrow \textcolor{red}{स्वरितञितः कर्त्रभिप्राये क्रियाफले} (पा॰सू॰~१.३.७२)~\arrow \textcolor{red}{परोक्षे लिट्} (पा॰सू॰~३.२.११५)~\arrow लिह्~लिट्~\arrow लिह्~ध्वम्~\arrow \textcolor{red}{लिटि धातोरनभ्यासस्य} (पा॰सू॰~६.१.८)~\arrow लिह्~लिह्~ध्वम्~\arrow \textcolor{red}{हलादिः शेषः} (पा॰सू॰~७.४.६०)~\arrow लि~लिह्~ध्वम्~\arrow \textcolor{red}{आर्धधातुकस्येड्वलादेः} (पा॰सू॰~७.२.३५)~\arrow इट्प्राप्तिः~\arrow \textcolor{red}{एकाच उपदेशेऽनुदात्तात्‌} (पा॰सू॰~७.२.१०)~\arrow इण्निषेधः~\arrow \textcolor{red}{कृसृभृवृ\-स्तुद्रुस्रुश्रुवो लिटि} (पा॰सू॰~७.२.१३)~\arrow क्रादि\-नियमादिट्प्राप्तिः~\arrow \textcolor{red}{आद्यन्तौ टकितौ} (पा॰सू॰~१.१.४६)~\arrow लि~लिह्~इट्~ध्वम्~\arrow लि~लिह्~इ~ध्वम्~\arrow \textcolor{red}{विभाषेटः} (पा॰सू॰~८.३.७९)~\arrow वैकल्पिक\-ढत्वम्~\arrow लि~लिह्~इ~ढ्वम्~\arrow \textcolor{red}{टित आत्मनेपदानां टेरे} (पा॰सू॰~३.४.७९)~\arrow लि~लिह्~इ~ढ्वे~\arrow लिलिहिढ्वे। ढत्वाभावे~– लि~लिह्~इ~ध्वम्~\arrow \textcolor{red}{टित आत्मनेपदानां टेरे} (पा॰सू॰~३.४.७९)~\arrow लि~लिह्~इ~ध्वे~\arrow लिलिहिध्वे।} तस्माद्धयवरड्ढकारः सप्रयोजनकः। एवं \textcolor{red}{हल्} इत्यत्र हकार\-ग्रहणाभावे \textcolor{red}{रुदिहि}\footnote{\textcolor{red}{रुदिँर् अश्रुविमोचने} (धा॰पा॰~१०६७)~\arrow रुद्~\arrow \textcolor{red}{शेषात्कर्तरि परस्मैपदम्} (पा॰सू॰~१.३.७८)~\arrow \textcolor{red}{लोट् च} (पा॰सू॰~३.३.१६२)~\arrow रुद्~सिप्~\arrow \textcolor{red}{कर्तरि शप्‌} (पा॰सू॰~३.१.६८)~\arrow रुद्~शप्~सिप्~\arrow \textcolor{red}{अदिप्रभृतिभ्यः शपः} (पा॰सू॰~२.४.७२)~\arrow रुद्~सिप्~\arrow \textcolor{red}{सेर्ह्यपिच्च} (पा॰सू॰~३.४.८७)~\arrow रुद्~हि~\arrow \textcolor{red}{सार्वधातुकमपित्} (पा॰सू॰~१.२.४)~\arrow ङित्त्वम्~\arrow \textcolor{red}{ग्क्ङिति च} (पा॰सू॰~१.१.५)~\arrow लघूपध\-गुण\-निषेधः~\arrow \textcolor{red}{रुदादिभ्यः सार्वधातुके} (पा॰सू॰~७.२.७६)~\arrow \textcolor{red}{आद्यन्तौ टकितौ} (पा॰सू॰~१.१.४६)~\arrow रुद्~इट्~हि~\arrow रुद्~इ~हि~\arrow रुदिहि।} \textcolor{red}{स्वपिहि}\footnote{\textcolor{red}{ञिष्वपँ शये} (धा॰पा॰~१०६८)~\arrow ष्वप्~\arrow \textcolor{red}{धात्वादेः षः सः} (पा॰सू॰~६.१.६४)~\arrow स्वप्~\arrow \textcolor{red}{शेषात्कर्तरि परस्मैपदम्} (पा॰सू॰~१.३.७८)~\arrow \textcolor{red}{लोट् च} (पा॰सू॰~३.३.१६२)~\arrow स्वप्~सिप्~\arrow \textcolor{red}{कर्तरि शप्‌} (पा॰सू॰~३.१.६८)~\arrow स्वप्~शप्~सिप्~\arrow \textcolor{red}{अदिप्रभृतिभ्यः शपः} (पा॰सू॰~२.४.७२)~\arrow स्वप्~सिप्~\arrow \textcolor{red}{सेर्ह्यपिच्च} (पा॰सू॰~३.४.८७)~\arrow स्वप्~हि~\arrow \textcolor{red}{रुदादिभ्यः सार्वधातुके} (पा॰सू॰~७.२.७६)~\arrow \textcolor{red}{आद्यन्तौ टकितौ} (पा॰सू॰~१.१.४६)~\arrow स्वप्~इट्~हि~\arrow स्वप्~इ~हि~\arrow स्वपिहि।} \textcolor{red}{श्वसिहि}\footnote{\textcolor{red}{श्वसँ प्राणने} (धा॰पा॰~१०६९)~\arrow श्वस्~\arrow \textcolor{red}{शेषात्कर्तरि परस्मैपदम्} (पा॰सू॰~१.३.७८)~\arrow \textcolor{red}{लोट् च} (पा॰सू॰~३.३.१६२)~\arrow श्वस्~सिप्~\arrow \textcolor{red}{कर्तरि शप्‌} (पा॰सू॰~३.१.६८)~\arrow श्वस्~शप्~सिप्~\arrow \textcolor{red}{अदिप्रभृतिभ्यः शपः} (पा॰सू॰~२.४.७२)~\arrow श्वस्~सिप्~\arrow \textcolor{red}{सेर्ह्यपिच्च} (पा॰सू॰~३.४.८७)~\arrow श्वस्~हि~\arrow \textcolor{red}{रुदादिभ्यः सार्वधातुके} (पा॰सू॰~७.२.७६)~\arrow \textcolor{red}{आद्यन्तौ टकितौ} (पा॰सू॰~१.१.४६)~\arrow श्वस्~इट्~हि~\arrow श्वस्~इ~हि~\arrow श्वसिहि।} इत्यादौ \textcolor{red}{रुदादिभ्यः सार्वधातुके} (पा॰सू॰~७.२.७६) इत्यनेन हकारस्य वल्त्वाभावेन वलादि\-लक्षण इण्न स्यात्। एवं हल्सूत्रे हकार\-पाठं विना \textcolor{red}{स्नेहित्वा} \textcolor{red}{स्निहित्वा} इत्यत्र \textcolor{red}{रलो व्युपधाद्धलादेः संश्च} (पा॰सू॰~१.२.२६) इत्यनेन वैकल्पिकं कित्त्वं न स्यात्।\footnote{\textcolor{red}{ष्णिहँ प्रीतौ} (धा॰पा॰~१२००)~\arrow ष्णिह्~\arrow \textcolor{red}{धात्वादेः षः सः} (पा॰सू॰~६.१.६४)~\arrow निमित्तापाये नैमित्तिकस्याप्यपायः~\arrow स्निह्~\arrow \textcolor{red}{समान\-कर्तृकयोः पूर्वकाले} (पा॰सू॰~३.४.२१)~\arrow स्निह्~क्त्वा~\arrow स्निह्~त्वा~\arrow \textcolor{red}{रधादिभ्यश्च} (पा॰सू॰~७.२.४५)~\arrow वैकल्पिक इट्~\arrow स्निह्~इट्~त्वा~\arrow स्निह्~इ~त्वा~\arrow \textcolor{red}{रलो व्युपधाद्धलादेः संश्च} (पा॰सू॰~१.२.२६)~\arrow वैकल्पिकं कित्त्वम्। कित्त्वपक्षे – स्निह्~इ~त्वा~\arrow \textcolor{red}{ग्क्ङिति च} (पा॰सू॰~१.१.५)~\arrow लघूपध\-गुण\-निषेधः~\arrow स्निहित्वा। अकित्त्वपक्षे – स्निह्~इ~त्वा~\arrow \textcolor{red}{पुगन्त\-लघूपधस्य च} (पा॰सू॰~७.३.८६)~\arrow स्नेह्~इ~त्वा~\arrow स्नेहित्वा। इडभावपक्षे च~– स्निह्~त्वा~\arrow \textcolor{red}{वा द्रुहमुह\-ष्णुहष्णिहाम्} (पा॰सू॰~८.२.३३)~\arrow वैकल्पिकं घत्वम्। घत्वपक्षे~– स्निघ्~त्वा~\arrow \textcolor{red}{झषस्तथोर्धोऽधः} (पा॰सू॰~८.२.४०)~\arrow स्निघ्~ध्वा~\arrow \textcolor{red}{झलां जश् झशि} (पा॰सू॰~८.४.५३)~\arrow स्निग्~ध्वा~\arrow स्निग्ध्वा। घत्वाभावे~– स्निह्~त्वा~\arrow \textcolor{red}{हो ढः} (पा॰सू॰~८.२.३१)~\arrow स्निढ्~त्वा~\arrow \textcolor{red}{झषस्तथोर्धोऽधः} (पा॰सू॰~८.२.४०)~\arrow स्निढ्~ध्वा~\arrow \textcolor{red}{ष्टुना ष्टुः} (पा॰सू॰~८.४.४१)~\arrow स्निढ्~ढ्वा~\arrow \textcolor{red}{ढो ढे लोपः} (पा॰सू॰~८.३.१३)~\arrow स्नि~ढ्वा~\arrow \textcolor{red}{ढ्रलोपे पूर्वस्य दीर्घोऽणः} (पा॰सू॰~६.३.१११)~\arrow स्नी~ढ्वा~\arrow स्नीढ्वा।} एवं हल्सूत्रीय\-हकारमन्तरेण शलित्यस्य पाठाभावात् \textcolor{red}{अलिक्षत्} इत्यत्र \textcolor{red}{शल इगुपधादनिटः क्सः} (पा॰सू॰~३.१.४५) इत्यनेन क्सो न स्यात्।\footnote{\textcolor{red}{लिहँ आस्वादने} (धा॰पा॰~१०१६)~\arrow लिह्~\arrow \textcolor{red}{शेषात्कर्तरि परस्मैपदम्} (पा॰सू॰~१.३.७८)~\arrow \textcolor{red}{लुङ्} (पा॰सू॰~३.२.११०)~\arrow लिह्~लङ्~\arrow लिह्~तिप्~\arrow लिह्~ति~\arrow \textcolor{red}{लुङ्लङ्लृङ्क्ष्वडुदात्तः} (पा॰सू॰~६.४.७१)~\arrow \textcolor{red}{आद्यन्तौ टकितौ} (पा॰सू॰~१.१.४६)~\arrow अट्~लिह्~ति~\arrow अ~लिह्~ति~\arrow \textcolor{red}{च्लि लुङि} (पा॰सू॰~३.१.४३)~\arrow अ~लिह्~च्लि~ति~\arrow \textcolor{red}{शल इगुपधादनिटः क्सः} (पा॰सू॰~३.१.४५)~\arrow अ~लिह्~क्स~ति~\arrow अ~लिह्~स~ति~\arrow \textcolor{red}{हो ढः} (पा॰सू॰~८.२.३१)~\arrow अ~लिढ्~क्स~ति~\arrow \textcolor{red}{षढोः कः सि} (पा॰सू॰~८.२.४१)~\arrow अ~लिक्~स~ति~\arrow \textcolor{red}{आदेश\-प्रत्यययोः} (पा॰सू॰~८.३.५९)~\arrow अ~लिक्~ष~ति~\arrow \textcolor{red}{इतश्च} (पा॰सू॰~३.४.१००)~\arrow अ~लिक्~ष~त्~\arrow अलिक्षत्।} एवमेव हल्सूत्रस्थ\-हकारमन्तरा झल्प्रत्याहारे तस्य ग्रहणाभावे \textcolor{red}{अदाग्धाम्} इत्यत्र \textcolor{red}{अ~दाह्~स~ताम्} इति स्थिते \textcolor{red}{झलो झलि} (पा॰सू॰~८.२.२६) इत्यनेन सकार\-लोपो न स्यात्।\footnote{\textcolor{red}{दहँ भस्मीकरणे} (धा॰पा॰~९९१)~\arrow दह्~\arrow \textcolor{red}{शेषात्कर्तरि परस्मैपदम्} (पा॰सू॰~१.३.७८)~\arrow \textcolor{red}{लुङ्} (पा॰सू॰~३.२.११०)~\arrow दह्~लङ्~\arrow दह्~तस्~\arrow \textcolor{red}{लुङ्लङ्लृङ्क्ष्वडुदात्तः} (पा॰सू॰~६.४.७१)~\arrow \textcolor{red}{आद्यन्तौ टकितौ} (पा॰सू॰~१.१.४६)~\arrow अट्~दह्~तस्~\arrow अ~दह्~तस्~\arrow \textcolor{red}{च्लि लुङि} (पा॰सू॰~३.१.४३)~\arrow अ~दह्~च्लि~तस्~\arrow \textcolor{red}{च्लेः सिच्} (पा॰सू॰~३.१.४४)~\arrow अ~दह्~सिच्~तस्~\arrow अ~दह्~स्~तस्~\arrow \textcolor{red}{सिचि वृद्धिः परस्मैपदेषु} (पा॰सू॰~७.२.१)~\arrow अ~दाह्~स्~तस्~\arrow \textcolor{red}{तस्थस्थमिपां तान्तन्तामः} (पा॰सू॰~३.४.१०१)~\arrow अ~दाह्~स्~ताम्~\arrow \textcolor{red}{झलो झलि} (पा॰सू॰~८.२.२६)~\arrow अ~दाह्~ताम्~\arrow \textcolor{red}{दादेर्धातोर्घः} (पा॰सू॰~८.२.३२)~\arrow अ~दाघ्~ताम्~\arrow \textcolor{red}{झषस्तथोर्धोऽधः} (पा॰सू॰~८.२.४०)~\arrow अदाघ्~धाम्~\arrow \textcolor{red}{झलां जश् झशि} (पा॰सू॰~८.४.५३)~\arrow अदाग्~धाम्~\arrow अदाग्धाम्।} तस्माद्धल्सूत्रेऽपि हकारः सार्थक एवेति प्रौढमनोरमादौ प्रपञ्चितम्। एवं सम्भवति हकारोच्चारण\-द्वय\-प्रयोजनेऽपि लौकिके मम दृष्टौ हकारं द्विरुच्चार्य तस्य हरि\-शब्दाद्यक्षरतया चतुर्दश\-सूत्री हरि\-स्मरणेन सम्पुटिता। इदमलौकिकं प्रयोजनमिति मे प्रतिभाति।\end{sloppypar}
\begin{sloppypar}\hyphenrules{nohyphenation}\justifying\noindent\hspace{10mm} एवं राघवेण चतुर्दश\-वर्षीय\-वन\-वास\-काल एव सम्पूर्णा लीला कृता। यद्यपि चतुर्दश\-सूत्र्यां सङ्ख्या\-ग्रहे प्रयोजनं सम्भवति शम्भोः। अन्यथा त्रयोदशैव सूत्राणि क्रियेरन् किं जातं चतुर्दशेन सूत्रेण हलादि\-प्रत्याहारा रान्ताः करणीया अन्योऽन्याश्रय\-दोष\-वारणाय रन्त्यं \textcolor{red}{हर्} इति न्यासः करणीयः। तत्रत्येयं परिस्थितिः \textcolor{red}{हलन्त्यम्} (पा॰सू॰~१.३.३) इति सूत्रे। \textcolor{red}{वाक्यार्थ\-बोधे पदार्थ\-ज्ञानं कारणम्}। यथा \textcolor{red}{रामो गच्छति} इत्यत्र वाक्यार्थ\-ज्ञाने पदयोर्द्वयोः पृथग्बोधः करणीयोऽनन्तरं वाक्यार्थ\-बोधो भविष्यति। तथैवात्रापि। \textcolor{red}{उपदेशेऽजनुनासिक इत्} (पा॰सू॰~१.३.२) इत्यस्मात् \textcolor{red}{उपदेशे इत्} इदं पद\-द्वयमनुवर्तते। तथा \textcolor{red}{उपदेशे अन्त्यं हल् इत् स्यात्} इत्यर्थे सम्पन्ने हल्पदस्यार्थ\-ज्ञाने कर्तव्ये हल्सञ्ज्ञा\-विधायक\-सूत्र\-वाक्यार्थ\-बोधः कर्तव्यः। तत्र च पदार्थ\-ज्ञानमावश्यकम्। हल्सञ्ज्ञा\-सूत्रं \textcolor{red}{आदिरन्त्येन सहेता} (पा॰सू॰~१.१.७१) इत्येवम्। \textcolor{red}{अन्त्येनेता सहित आदिर्मध्यगानां स्वस्य च सञ्ज्ञा स्यात्} (ल॰सि॰कौ॰~४)। एवमत्रापीत्पदार्थ\-ज्ञानं जिज्ञासितम्। तच्चेत्सञ्ज्ञा\-विधायक\-सूत्रे \textcolor{red}{हलन्त्यम्} (पा॰सू॰~१.३.३) इत्यस्मिन्नधीनम्। तत्रापि वाक्यार्थ\-ज्ञानाय पदार्थ\-ज्ञानम्। इत्थम् \textcolor{red}{इत्‌}\-पदार्थ\-ज्ञानं \textcolor{red}{हल्‌}\-पदार्थ\-ज्ञानाधीनं \textcolor{red}{हल्‌}\-पदार्थज्ञानञ्च \textcolor{red}{इत्‌}\-पदार्थ\-ज्ञानाधीनम्। अयमेवान्योऽन्याश्रयः। \textcolor{red}{अन्योऽन्याश्रयत्वं नाम परस्परापेक्षित्वम्}। \textcolor{red}{तद्ग्रहसापेक्ष\-ग्रह\-सापेक्ष\-ग्रह\-विषयत्वमन्योऽन्याश्रयत्वम्}। तद्यथा तद्ग्रह इद्ग्रहस्तत्सापेक्ष\-ग्रहो हल्ग्रहस्तत्सापेक्ष\-ग्रह इद्ग्रहस्तद्विषयत्वमन्योऽ\-न्याश्रयत्वम्। अन्योऽन्याश्रयाणि कार्याणि न प्रकल्पन्ते। यथा नावि बद्धा नौर्न प्रचलति।\footnote{\textcolor{red}{इतरेतराश्रयाणि च कार्याणि न प्रकल्पन्ते। तद्यथा नौर्नावि बद्धा नेतरत्राणाय भवति} (भा॰पा॰सू॰~१.१.१)।} इमं दोषमाशङ्क्य भट्टोजिदीक्षितो हलन्त्यमिति सूत्रस्याऽवृत्तिं कृतवान्। ध्यातव्यमष्टाध्यायीस्थम् \textcolor{red}{हलन्त्यम्} इति सूत्रमसमस्तं किन्त्वावृत्तम् \textcolor{red}{हलन्त्यम्} इति सूत्रं समस्तम्। सप्तमी\-समासे शौण्डादिगण एतस्याभावाद्योगविभागस्य चाप्रमाणिकत्वात्सुप्सुपा समासस्यागतिक\-गतित्वादत्र षष्ठी\-तत्पुरुषोऽवयवावयवि\-भाव\-रूपः सम्बन्धः। हलो हल्सूत्रस्यावयवि\-भूतमन्त्यमिति। इत्थमेकत्रेत्पदार्थ\-ज्ञाने \textcolor{red}{आदिरन्त्येन सहेता} (पा॰सू॰~१.१.७१) इत्यनेन हल्पदार्थोऽपि विज्ञाप्यते हल्सूत्रेण चेत्पदार्थः। तत्रैव रन्त्यं \textcolor{red}{हर्} इति न्यासः करणीयो \textcolor{red}{हरन्त्यम्} इति वाऽऽवर्तनीयम्। हकारस्य च प्रयोजनान्युक्तानि। अतो \textcolor{red}{ल्}\-शब्दस्य दृष्ट\-प्रयोजनाभावेऽ\-दृष्टार्थं प्रयोजनं रामायण\-कथाछलेन लक्ष्मण\-स्मरणार्थं च। यद्यपि हकारः \textcolor{red}{शषसहर्} इति पठनीयो \textcolor{red}{हरिर्हसति} इत्यत्र विसर्गापत्तिः।\footnote{\textcolor{red}{शषसहर्} इति पाठे कृते हकारस्य \textcolor{red}{खर्} इत्यत्र ग्रहणात् \textcolor{red}{हरिस्~हसति} इति स्थिते \textcolor{red}{ससजुषो रुः} (पा॰सू॰~८.२.६६) इत्यनेन रुत्वे \textcolor{red}{हरिरुँ~हसति} इति जातेऽनुबन्ध\-लोपे \textcolor{red}{हरिर्~हसति} इति जाते \textcolor{red}{खरवसानयोर्विसर्जनीयः} (पा॰सू॰~८.३.१५) इत्यनेन \textcolor{red}{हरिः~हसति} इत्यनिष्ट\-रूपं स्यादिति भावः। \textcolor{red}{शषसर्} पाठे \textcolor{red}{खरवसानयोर्विसर्जनीयः} (पा॰सू॰~८.३.१५) इत्यस्याप्रवृत्तौ \textcolor{red}{हरिर्~हसति} इत्यतो वर्णसम्मेलने \textcolor{red}{हरिर्हसति} इति।} \textcolor{red}{प्रेष्य\-ब्रुवोर्हविषो देवतासम्प्रदाने} (पा॰सू॰~२.३.६१) इति
ज्ञापनेन यद्यपि दोषो दूरीकर्तुं शक्यते। अतो निरर्थकं हल्। \textcolor{red}{नासूया कर्तव्या यत्रानुगमः क्रियते सूत्रकारैः} (भा॰पा॰सू॰~५.१.५९) इति वचनेनालमसूयया। चतुर्दश\-सूत्र्यां रामायणी चर्चा। रामायणे चानसूयाया वर्णनम्। असूया नहि सूत्रे।
दृष्टभाजनमेव न फलमदृष्टमपि। हकारोच्चारणेन हरि\-स्मरणं सीता\-सहितस्य श्रीरामस्य प्रतिपाद्यत्वात्। तत्र व्यञ्जन\-रूपेण सीताऽकार\-रूपेण च रामो लकारस्यार्धत्वाल्लकारेणापूर्ण\-जीवाचार्यस्य लक्ष्मणस्य स्मरणम्। अतो बाह्यान्तरकरणानां चतुर्दशानां\footnote{चक्षुः\-श्रोत्र\-रसना\-घ्राण\-त्वगाख्यानां पञ्च\-ज्ञानेन्द्रियाणां पाणि\-पाद\-पायूपस्थ\-वागाख्यानां पञ्च\-कर्मेन्द्रियाणां मनो\-बुद्धि\-चिताहङ्काराख्यानां चतुरन्तः\-करणानां च।} पावन्याः चतुर्दश\-भुवन\-ख्यात\-राम\-कथायाः सूत्रतः सङ्केतितत्वाच्चतुर्दशत्वेऽतिविशेष आग्रहो महादेवस्येति प्रतीयते। इत्थं सम्पूर्णमपि व्याकरणं पाणिनीयं राम\-कथा\-परम्। अध्यात्म\-रामायणञ्च शिव\-प्रोक्तम्। अतः पाणिनि\-व्याकरणाध्यात्म\-रामायणयोरेकस्य शिवस्य वक्तृत्वाद्द्वयोः समीक्षायां प्रेरिता स्वयमेव भगवता रामचन्द्रेणाज्ञान\-विक्लवाऽनधीत\-शास्त्रा क्षपित\-चक्षुषो मे बाल\-मनीषा। यद्यप्येका सूक्तिर्यत्~–\end{sloppypar}
\centering\textcolor{red}{यान्युज्जहार माहेशाद्व्यासो व्याकरणार्णवात्।\nopagebreak\\
तानि किं पदरत्नानि मान्ति पाणिनिगोष्पदे॥}\footnote{मूलं मृग्यम्। शब्दकल्पद्रुम\-काराः \textcolor{red}{माहेशम्} इति शब्दे श्लोकमिमुद्धृत्य \textcolor{red}{इत्युद्भटः} इत्यूचुः।}\nopagebreak\\
\begin{sloppypar}\hyphenrules{nohyphenation}\justifying\noindent अर्थाच्छाङ्कर\-व्याकरण\-समुद्रस्य सम्पूर्णोऽपि विषयः पाणिनीये व्याकरणे कथं समाहर्तुं शक्यः। किन्त्वेतत्कथनं केवलं बौद्धिक\-विचारतः पलायन\-वाद\-मात्रम्। शिव एव पाणिनि\-हृदय\-स्थ इदं व्याकरणमस्मदर्थं व्याचकार। अतः सर्वमपि वाल्मीकि\-व्यास\-तुलसीदास\-प्रमुख\-शिष्ट\-प्रयुक्तं पाणिनि\-व्याकरण\-सम्मतं कर्तुं शक्यमिति गुरु\-सेवा\-लब्ध\-बुद्धि\-बलो गिरिधरः साटोपं घोषयति।\end{sloppypar}
\begin{sloppypar}\hyphenrules{nohyphenation}\justifying\noindent\hspace{10mm} अथ वेदना\-तान्त\-क्रौञ्च\-द्वन्द्व\-वियोगोत्थ\-शोको श्लोकत्वमागतो वाल्मीकेः।\footnote{\textcolor{red}{काव्यस्यात्मा स एवार्थस्तथा चादिकवेः पुरा। क्रौञ्चद्वन्द्ववियोगोत्थः शोकः श्लोकत्वमागतः॥} (ध्व॰~१.५)।} तद्वै ब्रह्मणाऽऽदिष्टः श्रीमद्रामायणं प्रोवाच। तच्च विपुलं चतुर्विंशति\-सहस्र\-श्लोकात्मकम्। सा च चतुर्विंशत्साहस्री चतुर्विंशति\-साहस्री संहितेति कथ्यते चतुर्विंशत्यक्षरात्मक\-गायत्री\-भाष्य\-भूता। गायत्री च सीता। वाल्मीकीयं रामायणं सीता\-चरित\-प्रधानं स्वयमेव वाल्मीकिः प्रतिजानीते यथा~–\end{sloppypar}
\centering\textcolor{red}{काव्यं रामायणं कृत्स्नं सीतायाश्चरितं महत्।\nopagebreak\\
पौलस्त्यवधमित्येवं चकार चरितव्रतः॥}\nopagebreak\\
\raggedleft{–~वा॰रा॰~१.४.७}\\
\begin{sloppypar}\hyphenrules{nohyphenation}\justifying\noindent ततः शत\-कोटि\-पर्यन्तानि रामायणानि भाषितानि वाल्मीकिना। तत्र प्रथमस्याऽदि\-काव्यस्य वक्ता स्वयं शेषाणाञ्च वक्तारं शशाङ्क\-शेखरं शिवं श्रोत्रीं च भगवतीं भवानीं समकल्पयत्। अस्य रामायणे माधुर्यं प्रधानमैश्वर्यमत्यल्पम्। यद्यप्ययमेव श्रीरामस्य भगवत्त्वं सर्व\-प्रथमं प्रतिजानीते यथा~–\end{sloppypar}
\centering\textcolor{red}{प्रोद्यमाने जगन्नाथं सर्वलोकनमस्कृतम्।\nopagebreak\\
कौसल्याऽजनयद्रामं दिव्यलक्षणसंयुतम्॥}\nopagebreak\\
\raggedleft{–~वा॰रा॰~१.१८.१०}\\
\begin{sloppypar}\hyphenrules{nohyphenation}\justifying\noindent इति। एवमयोध्या\-काण्डेऽपि~–\end{sloppypar}
\centering\textcolor{red}{स हि देवैरुदीर्णस्य रावणस्य वधार्थिभिः।\nopagebreak\\
अर्थितो मानुषे लोके जज्ञे विष्णुः सनातनः॥}\nopagebreak\\
\raggedleft{–~वा॰रा॰~२.१.७}\\
\begin{sloppypar}\hyphenrules{nohyphenation}\justifying\noindent एवं क्वचित्क्वचित्प्रति\-काण्डम्। इमे वैदिका ऋषयः। अतो वेद\-प्रतिपाद्य\-ब्रह्म\-रामस्यैव चर्चामकार्षुः। वेदे राम\-कथायाः कुत्र चर्चा कथं वा राम\-कथाया वेद\-मूलकता इति चेत्। \textcolor{red}{मन्त्र\-ब्राह्मणयोर्वेदनामधेयम्} (आ॰श्रौ॰सू॰~२४.१.३१) इत्यापस्तम्ब\-सूत्रानुसारं मन्त्र\-ब्राह्मणात्मको वेदः। मन्त्र\-भागे राम\-कथा नीलकण्ठाचार्य\-सङ्कलित\-मन्त्रात्मिका। तत् \textcolor{red}{मन्त्र\-रामायणम्} इति कथ्यते। ब्राह्मण\-भागे च प्रायश उपनिषदो यासु रामोत्तर\-तापनीयोप\-निषत्सीतोप\-निषदित्यादयः। इत्थं माधुर्य\-गुण\-बृंहित\-वाल्मीकि\-रामायणे बहुत्र स्थलेषु नरमनुकुर्वतो भगवतो रामस्य चरित्रं दृष्ट्वा श्रुत्वा वा सामान्य\-जनानां सन्देहः। यथा स्वयमेव दाक्षायणी भवानी सीता\-विरह\-विधुर\-हृदयं पामरमिव पृच्छन्तं लता\-काननानि श्रीरघुनन्दनं दृष्ट्वा मुमोहेति। जगती\-तल\-समुद्धार\-चिकीर्षया रामोपासना\-सुधया सुधारयितुं वसुधा\-तलं कैलास\-वट\-वृक्ष\-तले निषण्णो भवानी\-विहित\-विविध\-श्रीराम\-विषयक\-प्रश्न\-समाकर्णन\-प्रसन्नोऽपि शम्भुर्निज\-हृदय\-पुण्य\-वसुमती\-तले प्रवहन्तीं भक्ति\-वेदान्त\-सिद्धान्त\-पावन\-पूर\-पूर्णां कृत\-कलि\-पाप\-पाषाण\-चूर्णां श्रीराम\-सागर\-सङ्गमां निखिल\-लोक\-मनोरमां निहित\-सिद्धान्त\-तरल\-तरङ्गां समलङ्कृत\-वैष्णव\-हृदय\-रङ्गां विहित\-भगवल्लीला\-समागत\-सन्देह\-कश्मल\-भङ्गामुपनिषत्सिद्धान्त\-श्रुति\-प्रामाण्य\-सङ्गामध्यात्म\-रामायण\-गङ्गां पञ्चभ्योऽपि वक्त्रेभ्यः प्रवाहयामास। भगवान् शिवोऽनादिरत एतद्रचनाऽप्यनादिरेव। शङ्करो भगवान् देश\-काल\-परिस्थिति\-परिच्छेद\-रहितोऽतस्तत्कृतिरपि तथैव। श्रीराम\-कथा\-वर्णन\-प्रसङ्गेन भगवता शिवेन निगूढ\-वेदान्त\-रहस्यानां यादृक्सरलतया लालित्य\-पूर्णं प्रतिपादनमकार्येतत्कर्म तस्मिन्नेव सङ्घटते सच्चिदानन्द\-शान्ति\-निलये शिवे। अस्मिन् परम\-प्रत्ने विहित\-रामोपासना\-भावना\-मण्डन\-यत्ने ग्रन्थ\-रत्ने नवानां रसानामनुपमा छटा समालोक्यते। साहित्य\-शास्त्रानुसारं प्रत्येक\-महा\-काव्ये सप्ताधिकसर्गा अपेक्षन्ते। अत्र च चतुःषष्टि\-सर्गाः सन्ति। प्रति\-सर्गं छन्दः\-परिवर्तनं भवति यथा वाल्मीकीय\-रामायणे रघुवंशादौ च। वाल्मीकि\-रामायणस्य बाल\-काण्डस्य प्रथम\-सर्गे नव\-नवतिं यावच्छ्लोका अनुष्टुप एवं शततमः श्लोक उपजातिः। यथा~–\end{sloppypar}
\centering\textcolor{red}{पठन्द्विजो वागृषभत्वमीयात्स्यात्क्षत्रियो भूमिपतित्वमीयात्।\nopagebreak\\
वणिग्जनः पण्यफलत्वमीयाज्जनश्च शूद्रोऽपि महत्त्वमीयात्॥}\nopagebreak\\
\raggedleft{–~वा॰रा॰~१.१.१००}\\
\begin{sloppypar}\hyphenrules{nohyphenation}\justifying\noindent एवमेव रघुवंशे~–\end{sloppypar}
\centering\textcolor{red}{अथ नयनसमुत्थं ज्योतिरत्रेरिव द्यौः\nopagebreak\\
सुरसरिदिव तेजो वह्निनिष्ठ्यूतमैशम्।\nopagebreak\\
नरपतिकुलभूत्यै गर्भमाधत्त राज्ञी\nopagebreak\\
गुरुभिरभिनिविष्टं लोकपालानुभावैः॥}\nopagebreak\\
\raggedleft{–~र॰वं॰~२.७५}\footnote{रघुवंशस्य द्वितीयसर्गे चतुःसप्ततिं यावत्पद्यानीन्द्र\-वज्रोप\-जातिच्छन्दोबद्धानि पञ्चसप्ततितमं पद्यं च मालिनी\-वृत्त\-बद्धम्।}\\
\begin{sloppypar}\hyphenrules{nohyphenation}\justifying\noindent तथैवात्रापि प्रायः सर्गान्ते छन्दसः परिवर्तनम्। यथा~–\end{sloppypar}
\centering\textcolor{blue}{देवाश्च सर्वे हरिरूपधारिणः स्थिताः सहायार्थमितस्ततो हरेः।\nopagebreak\\
महाबलाः पर्वतवृक्षयोधिनः प्रतीक्षमाणा भगवन्तमीश्वरम्॥}\nopagebreak\\
\raggedleft{–~अ॰रा॰~१.२.३२}\\
\centering\textcolor{blue}{एवं परात्मा मनुजावतारो मनुष्यलोकाननुसृत्य सर्वम्।\nopagebreak\\
चक्रेऽविकारी परिणामहीनो विचार्यमाणे न करोति किञ्चित्॥}\nopagebreak\\
\raggedleft{–~अ॰रा॰~१.३.६६}\\
\begin{sloppypar}\hyphenrules{nohyphenation}\justifying\noindent एवमन्यत्रापि। महा\-काव्यस्य नियमानुसारेण प्रत्येकस्मिन्महाकाव्ये शृङ्गार\-शान्त\-वीर\-करुणेष्वन्यतमो रसः प्रधानोऽङ्गी वा शेषा अष्टौ रसा गौणा अङ्गभूताश्च। एवमस्मिन्नपि महा\-काव्ये शान्तो रसः प्रधानः। ज्ञातव्यमत्र वाल्मीकीय\-रामायणं करुण\-रस\-प्रधानं क्रौञ्च\-द्वन्द्व\-वियोग\-संवीक्षण\-सञ्जात\-करुणस्य महर्षेः काव्य\-सृष्टौ प्रवृत्तत्वात्। रघुवंश\-महा\-काव्यस्य चतुर्दशे सर्गे कविता\-कामिनी\-विलासः कवि\-कुल\-गुरुः कालिदासः सीता\-निर्वासन\-करुणां तरुणयन् गायति यथा~–\end{sloppypar}
\centering\textcolor{red}{तामभ्यगच्छद्रुदितानुसारी मुनिः कुशेध्माहरणाय यातः।\nopagebreak\\
निषादविद्धाण्डजदर्शनोत्थः श्लोकत्वमापद्यत यस्य शोकः॥}\nopagebreak\\
\raggedleft{–~र॰वं॰~१४.७०}\\
\begin{sloppypar}\hyphenrules{nohyphenation}\justifying\noindent एवं श्रीतुलसीदास\-कृतं श्रीराम\-चरित\-मानसं वीर\-रस\-प्रधानं रघुवंश\-महा\-काव्यं शृङ्गार\-रस\-प्रधानं तथैवाध्यात्म\-रामायणमिदं शान्त\-रस\-प्रधानम्। अस्य वक्ता स्वयं प्रशान्ति\-निलयः शिवः। स च काम\-शत्रुः। अत एव तुलसीदासोऽपि श्रीमानस इमं धृतशरीरं शान्त\-रसमिव प्रस्तौति। यथा~–\end{sloppypar}
\centering\textcolor{red}{बैठे सोह कामरिपु कैसे। धरे शरीर शांत रस जैसे॥}\footnote{एतद्रूपान्तरम्–\textcolor{red}{तत्रोपविष्टः शुशुभे कामारिः कीदृशस्तदा। शरीरधारी प्रत्यक्षं नूनं शान्तरसो यथा॥} (मा॰भा॰~१.१०७.१)।}\nopagebreak\\
\raggedleft{–~रा॰च॰मा॰~१.१०७.१}\\
\begin{sloppypar}\hyphenrules{nohyphenation}\justifying\noindent \textcolor{red}{फल\-भोक्ता तु नायकः} इति सिद्धान्तानुसारं काव्यस्य फलं नायक एव भुङ्क्ते। अतो नायक\-वृत्तानु\-सारमपि रस\-प्राधान्यं निर्णेतुं शक्यते। नायकः श्रीरामस्तथाऽत्राध्यात्म\-तत्त्वात्परम\-शान्ति\-निलयः शान्तस्तस्माच्छान्त\-रस\-प्रधानमिदम्। आद्यन्तघटनाभ्यामपि रसो निर्णीयते। एतस्याऽद्या घटना शिव\-पार्वती\-सम्बद्धाऽन्तिमाऽपि घटना भगवदन्तर्धानरूपोभे च शान्त\-प्रधाने तस्माच्छान्त\-रसः प्रधानः। अत्र स्थले स्थले सहस्रश उपनिषदां वेदान्त\-गूढ\-तत्त्वानां वैराग्य\-प्रतिपादक\-वाक्यानां प्रतिपादनं प्राचुर्येण मिलति तथाऽप्यध्यात्म\-चर्चा\-विषयत्वात्सुतरामिदमाध्यात्मिकीं पिपासां शमयितुं क्षमम्।\end{sloppypar}
\begin{sloppypar}\hyphenrules{nohyphenation}\justifying\noindent\hspace{10mm} चत्वारो हि नायका धीरोदात्तो धीरोद्धत्तो धीरललितो धीरप्रशान्तश्चेति।\footnote{\textcolor{red}{त्यागी कृती कुलीनः सुश्रीको रूपयौवनोत्साही। दक्षोऽनुरक्तलोकस्तेजोवैदग्ध्यशीलवान् नेता॥ धीरोदात्तो धीरोद्धतस्तथा धीरललितश्च। धीरप्रशान्त इत्ययमुक्तः प्रथमश्चतुर्भेदः॥} (सा॰द॰~३.३०-३१)।} तत्र धीरोदात्तः कुलीनः शान्तो दान्तश्चरित्र\-सद्गुण\-सम्पन्नस्तेजस्वी। स च पराक्रम\-मेधादीनां यष्टा।\footnote{\textcolor{red}{अविकत्थनः क्षमावानतिगम्भीरो महासत्त्वः। स्थेयान् निगूढमानो धीरोदात्तो दृढव्रतः कथितः॥} (सा॰द॰~३.३२)।} मर्यादा\-पुरुषोत्तमः पर\-ब्रह्म श्रीरामो यथा वाल्मीकीय\-रामायणे तुलसी\-कृते च प्रायश आत्म\-श्लाघा\-रहितः। स एवात्र धीरोदात्तः। भीमसेनादिर्धीरोद्धत्तः।\footnote{\textcolor{red}{मायापरः प्रचण्डश्चपलोऽहङ्कारदर्पभूयिष्ठः। आत्मश्लाघानिरतो धीरैर्धीरोद्धतः कथितः॥} (सा॰द॰~३.३३)।} धीरललितः कृष्णो जयदेवस्य।\footnote{\textcolor{red}{निश्चिन्तो मृदुरनिशं कलापरो धीरललितः स्यात्} (सा॰द॰~३.३४)।}
महाभारते युधिष्ठिरादयो धीर\-प्रशान्ताः।\footnote{\textcolor{red}{सामान्यगुणैर्भूयान्द्विजादिको धीरप्रशान्तः स्यात्} (सा॰द॰~३.३४)।} चतुर्षु नायकेषु श्रीरामो धीरोदात्तोऽध्यात्म\-रामायणानुसारम्। अत्र नायिका भगवती सीता मुग्धा।\footnote{\textcolor{red}{विनयार्जवादियुक्ता गृहकर्मपरा पतिव्रता स्वीया। साऽपि कथिता त्रिभेदा मुग्धा मध्या प्रगल्भेति॥ प्रथमावतीर्णयौवनमदनविकारा रतौ वामा। कथिता मृदुश्च माने समधिकलज्जावती मुग्धा॥} (सा॰द॰~३.५७-५८)} अत एव सा श्यामेति कथ्यते। \textcolor{red}{श्यामा मुग्धा हि नायिका}\footnote{मूलं मृग्यम्।} इति वचनात्। अत्र शान्त एव रसः प्रारम्भतः समाप्तिं यावत्। भक्ति\-ज्ञान\-वैराग्याणां मनोहारिणी चर्चा। अत्रत्यः श्रीरामः सुन्दरो मधुरः शिवश्च। इत्यध्यात्म\-रामायणी गङ्गा वाराणस्या गङ्गेव धारा\-त्रयी\-मिश्रिता। यथा वाराणस्या गङ्गायां हरिद्वारस्थ\-शुद्ध\-गङ्गायाः प्रयागस्थयोः यमुना\-सरस्वत्योर्मिलितः प्रवाहस्तथैवात्र राम\-भक्ति\-गङ्गायाः कर्म\-कथा\-रवि\-नन्दन्या ज्ञान\-सरसस्वत्याश्च प्रवाहा मिश्रिताः।\footnote{\textcolor{red}{रामभगति जहँ सुरसरि धारा। सरसइ ब्रह्मबिचार प्रचारा॥ बिधिनिषेधमय कलिमलहरनी। करमकथा रबिनन्दिनि बरनी॥} (रा॰च॰मा॰~१.२.८,९)। एतद्रूपान्तरम्–\textcolor{red}{रामभक्तिर्यत्र धारा सुरनद्याः प्रकीर्तिता। ब्रह्मतत्त्वविचारस्य प्रचारश्च सरस्वती॥ कलिजानां कल्मषाणां हन्त्री विधिनिषेधयुक्। श्रौती कर्मकथा यत्र यमुना परिकीर्तिता॥} (मा॰भा॰~१.२.८,९)। } वारणस्यां गङ्गा विष्णु\-प्रिया विश्वनाथ\-प्रिया तथैवेयमपि। धन्यैषा या स्वयं शशाङ्क\-शेखर\-हिमाद्रितः समुद्भवा श्रीराम\-सागर\-गामिनी च।\footnote{\textcolor{red}{पुरारिगिरिसम्भूता श्रीरामार्णवसङ्गता। अध्यात्मरामगङ्गेयं पुनाति भुवनत्रयम्॥} (अ॰रा॰~१.१.५)।}\end{sloppypar}
\begin{sloppypar}\hyphenrules{nohyphenation}\justifying\noindent\hspace{10mm} वाल्मीकीय\-रामायणं माधुर्य\-प्रधान\-श्रीराघव\-भगवत्त्ववर्णन\-परम्। तस्मात्सामान्यानां दृष्टौ बहुत्र माधुर्य\-धारायां तिरोहितत्वेन भाष्यमाणे श्रीराघवस्यैश्वर्ये सन्देहो जायते
श्रीरामस्य ब्रह्मत्वे। यथा जगच्छरण्यो रामः सुग्रीवं शरणं गत इति कथयति।\footnote{किष्किन्धा\-काण्डे चतुर्थे सर्गे। यथा \textcolor{red}{अहं चैव हि रामश्च सुग्रीवं शरणं गतौ} (वा॰रा॰~४.४.१७)।} अयमेव समुद्रं शरणं गत इति प्रतिजानीते।\footnote{युद्ध\-काण्ड एकोनविंश एकविंशे च सर्गे। यथा \textcolor{red}{समुद्रं राघवो राजा शरणं गन्तुमर्हति} (वा॰रा॰~६.१९.३०), \textcolor{red}{ततः सागरवेलायां दर्भानास्तीर्य राघवः। अञ्जलिं प्राङ्मुखः कृत्वा प्रतिशिश्ये महोदधेः॥} (वा॰रा॰~६.२१.१)।} अपहृतां सीतां स्त्रैण इव शरण्यो वरदो राघवेन्द्रो नदी\-निर्झर\-गिरि\-जन\-गुल्म\-तरु\-लताः पृच्छति।\footnote{अरण्य\-काण्डे षष्टितमे चतुःषष्टितमे च सर्गे।} इन्द्रजिता मेघनादेन ब्रह्मास्त्र\-मोहितो भव\-बन्धन\-हर्ताऽपि निबद्धो नाग\-पाशेन स्वयं मोक्ष\-रूपो मुमुक्ष्वभिमृग्य\-पदाब्ज\-पद्धतिर्मोक्ष\-दायको रघु\-नायको मुकुन्दो निज\-कृपा\-पात्र\-भूतेन निज\-चरण\-कमल\-कुन्त\-केतु\-कञ्ज\-ललित\-लक्ष्म\-सनाथित\-पृष्ठ\-भागेन सतत\-केतन\-संलिप्त\-योगि\-मुनि\-वृन्द\-परम\-हंस\-महात्म\-शिव\-विरिञ्चीन्दिरा\-वन्दित\-निखिल\-गुण\-गण\-सद्म\-शीर्ण\-सज्जन\-छद्म\-निश्छद्म\-पद\-पद्म\-परागाङ्ग\-रागेण समूढ\-पुण्डरीकाक्षेण तार्क्ष्येण मुच्यते।\footnote{युद्ध\-काण्डे पञ्च\-चत्वारिंशे पञ्चाशे च सर्गे। यथा \textcolor{red}{निरन्तरशरीरौ तौ भ्रातरौ रामलक्ष्मणौ। क्रुद्धेनेन्द्रजिता वीरौ पन्नगैः शरतां गतैः॥} (वा॰रा॰~६.४५.८), \textcolor{red}{तमागतमभिप्रेक्ष्य नागास्ते विप्रदुद्रुवुः} (वा॰रा॰~६.५०.३७)।} रावणाहतं लक्ष्मणं क्रोडीकृत्य विलपति।\footnote{युद्ध\-काण्डे द्व्युत्तर\-शततमे सर्गे। यथा \textcolor{red}{परं विषादमापन्नो विललापाकुलेन्द्रियः} (वा॰रा॰~६.१०२.१०)।} सर्वज्ञोऽपि सर्वेश्वरः सर्वाधिष्ठान\-रूपः सर्व\-शरण्यः सर्वशक्तिमान् सर्व\-श्रुति\-सिद्धान्त\-भूत\-सकल\-विद्या\-निकेतनं श्रीनिकेतनमीश्वरो भगवान् रामचन्द्रो ब्रह्मणा बोध्यमानः सन् विस्मृतमेवात्मनो भगवत्त्वं प्रतिपद्यमान इव विलोक्यते कथयति च \textcolor{red}{आत्मानं मानुषं मन्ये रामं दशरथात्मजम्} (वा॰रा॰~६.११७.११)।
सत्त्वानन्द\-विमल\-बोध\-मयस्य निरामयस्य परात्परमेश्वर\-परमात्मनो जगदात्मनो रामचन्द्रस्य माधुर्य\-लीला\-विलास\-दर्शनमदूर\-दर्शिनां
हृदये सन्देह\-दाह\-दर्शनमातनुते। इत्थमनादि\-काल\-पाप\-वासना\-दूषित\-शेमुषीकाणां वासना\-भुजङ्गिनी\-विषय\-दंष्ट्रा\-दष्ट\-मानसानां कामिनी\-कटाक्ष\-सम्पात\-पातित\-ब्रह्मचर्यादि\-सद्गुणानां भोगान्ध\-चक्षुषां मलिनं हृदय\-नभःस्थलं भासयितुं कश्यप इव कश्यप\-पितृव्यो गङ्गा\-तरल\-तरङ्ग\-भङ्गिम\-भक्त\-भव\-भय\-भीति\-समुल्लसित\-जटा\-कानन\-प्रान्तरो भगवान् भवानी\-भवो भवः शशाङ्क\-शेखरः शिवः सङ्कलित\-रमणीय\-राम\-रहस्य\-रश्मि\-राशिं मनीषि\-मनीषा\-कमलिनी\-वल्लभं बोधित\-सज्जन\-सुमनः\-सरोरुहं प्रमोदित\-भावुक\-भाव\-भृङ्गं क्षपित\-त्रिभुवन\-तमिस्र\-पटलं
राम\-भक्ति\-भावान्वितं विज्ञान\-ज्योतिर्मयमध्यात्म\-रामायण\-नामधेयमनस्तं प्रगुणं प्रभाकरं प्रादुर्भावयामास। श्रीरामस्य माधुर्य\-लीला\-स्थल\-समागत\-सन्देहानेव निवारयितुमेवेदं प्रवृत्तम्। अतः सन्दिग्ध\-विषयान् विस्पष्टयितुं भवानी\-कृत\-पूर्व\-पक्ष\-मिषेण प्रस्तौति संस्तौति च स्वयमेव तदुत्तर\-मिषेणाध्यात्म\-राम\-गाथामनाथ\-नाथो विश्व\-नाथोऽविनाशी कैलास\-वासी भगवान् शङ्करः। यथा पार्वती परमेश्वरं पूर्व\-पक्ष\-पुरःसरमभिमुखयति यद्यदि रामः सर्वज्ञस्तर्हि प्राकृत\-नर इव कथं सीताकृते विललाप यदि च सामान्य\-जीव इतरस्तर्हि कथमस्माभिः सेव्यताम्। नहि भिक्षुको भिक्षुकान्तरं याचतेऽभिक्षुक इति न्यायात्। तथा च गिरिजा स्वयमेव स्पष्टयति~–\end{sloppypar}
\centering\textcolor{blue}{यदि स्म जानाति कुतो विलापः सीताकृतेऽनेन कृतः परेण।\nopagebreak\\
जानाति नैवं यदि केन सेव्यः समो हि सर्वैरपि जीवजातैः॥}\nopagebreak\\
\raggedleft{–~अ॰रा॰~१.१.१४}\\
\begin{sloppypar}\hyphenrules{nohyphenation}\justifying\noindent इदं ग्रन्थ\-रत्नं नूनमेव कलि\-काल\-ग्रस्तानामस्मादृशां भवायैव परम\-कारुणिको भवो भावयाम्बभूव। अतः कालातीतत्वान्निटिल\-लोचनस्य त्रि\-लोचनस्याध्यात्म\-रामायण\-रचना\-कालस्तु न वक्तुं शक्यते किन्तु भूत\-नाथस्य सर्व\-भूत\-हृदयस्थत्वेऽपि सगुणोपासना\-दृष्ट्या व्याप्त्यवच्छेदकतया कैलासे सतत\-निवासस्य वेद\-रामायण\-पुराणादौ विश्रुतत्वादध्यात्म\-रामायण\-रचना कैलास एव। यथाऽत्रैव ग्रन्थे~–\end{sloppypar}
\centering\textcolor{blue}{कैलासाग्रे कदाचिद्रविशतविमले मन्दिरे रत्नपीठे\nopagebreak\\
संविष्टं ध्याननिष्ठं त्रिनयनमभयं सेवितं सिद्धसङ्घैः।\nopagebreak\\
देवी वामाङ्कसंस्था गिरिवरतनया पार्वती भक्तिनम्रा\nopagebreak\\
प्राहेदं देवमीशं सकलमलहरं वाक्यमानन्दकन्दम्॥}\nopagebreak\\
\raggedleft{–~अ॰रा॰~१.१.६}\\
\begin{sloppypar}\hyphenrules{nohyphenation}\justifying\noindent\hspace{10mm} अत्र शताधिकोपनिषच्छ्रुतीनां प्रामाण्यं समुपबृंहितम्। राम\-कथया सह जीव\-व्यथाया रमणीयः संयोगः। सर्वत्र ब्रह्म\-रूपस्य रामस्य सम्पन्नं प्रतिपादनम्। राम\-हृदयं राम\-गीता चेति द्वयमपि निखिल\-दर्शन\-सिद्धान्त\-नवनीतम्। भक्ति\-भागीरथी समुल्लसिता सोल्लासं प्रवहति। इदं चतुःषष्टि\-सर्गात्मकमध्यात्म\-रामायणं वेद\-व्यास\-प्रणीत\-ब्रह्माण्ड\-पुराणोत्तर\-खण्डे मन्यते जनैः। वयं तु शब्द\-नित्यत्व\-वादिनो वैयाकरणा नित्यमिमं सद्ग्रन्थ\-भास्करं शम्भु\-मुख\-प्रादुर्भूतं शिव\-प्रेरितेन व्यासेन पुराणे व्यवस्थापितमिति मन्यामहे। पाणिनीय\-व्याकरणस्य च महेश्वर एव आचार्यः। अध्यात्म\-रामायणञ्च महेश्वरोक्तम्। अतो द्वयोः साम्प्रदायिकीमेक\-वाक्यतामपि विभाव्याध्यात्म\-रामायणे समागतानामपाणिनीयानां प्रयोगाणां विमर्शं चिकीर्षन् कमपि बाल\-प्रयासमातनोमि। यद्यपि माहेश्वर\-व्याकरण\-सिन्धोरपेक्षया पाणिनीयं गोष्पदमिति पुराण\-विदो विदाङ्कुर्वन्तु तथाऽपि वयं तु पाणिनीयमपि माहेश्वर\-व्याकरण\-सिन्धु\-सार\-सर्वस्व\-सुधामिव मन्यामहे। पाणिनीय\-व्याकरण\-घटे शम्भु\-शब्द\-सागरः समाहित इति मे द्रढीयसी प्रतीतिः। अतः शिवोक्तेऽस्मिन्नध्यात्म\-रामायणे प्रायशः सप्त\-शत\-शब्दा अपाणिनीयाः प्रतीयन्ते। ते च सन्धि\-समास\-कारक\-कृदन्त\-तद्धित\-धातु\-प्रक्रिया\-लिङ्ग\-सम्बन्धिनः। त एव मया विमृश्यन्ते।\footnote{तेषु सप्त\-शत\-प्रयोगेषु सार्ध\-त्रिशत\-प्रयोगा ग्रन्थेऽस्मिन् विमृष्टाः।} श्रीराम\-कृपया निर्देशक\-परम\-वन्दनीय\-गुरु\-चरण\-पण्डित\-भूपेन्द्रपति\-त्रिपाठि\-महाभागैरेवमाधुनिक\-शब्द\-विद्या\-चुञ्चु\-परम\-श्रद्धेय\-वैयाकरण\-शिरोमणि\-पूज्य\-गुरुदेव\-डॉ॰\-राम\-प्रसाद\-त्रिपाठि\-चरणैर्दत्त\-बुद्धि\-वैभव एवं च परम\-सुशील\-शब्द\-सागर\-मन्दर\-मति\-काव्य\-कला\-कलाधर\-सरस\-हृदय\-सदय\-पण्डित\-प्रकाण्ड\-परम\-भावुक\-गुरु\-चरण\-डॉ॰\-कालिका\-प्रसाद\-शुक्ल\-वर्यैः सम्प्रति सम्पूर्णानन्द\-विश्व\-विद्यालय\-व्याकरण\-विभागाध्यक्ष\-पदमलङ्कुर्वद्भिः संवर्धित\-बुद्धि\-गाम्भीर्य\-गवेषणा\-गौरवो विश्वनाथ\-प्रसादतो वीणा\-वादिनी\-परमेश्वरी\-संस्मरण\-लब्धोत्साहो विगलित\-सकाम\-साधनोऽपि स्वकीय\-श्रीगुरु\-चरण\-कृपा\-धनः \textcolor{red}{अध्यात्म\-रामायणेऽपाणिनीय\-प्रयोगाणां विमर्शः} इति नामधेयस्य शोध\-प्रबन्धस्य भूमिकां प्रस्तौमि।\end{sloppypar}
