% Nityanand Misra: LaTeX code to typeset a book in Sanskrit
% Copyright (C) 2016 Nityanand Misra
%
% This program is free software: you can redistribute it and/or modify it under
% the terms of the GNU General Public License as published by the Free Software
% Foundation, either version 3 of the License, or (at your option) any later
% version.
%
% This program is distributed in the hope that it will be useful, but WITHOUT
% ANY WARRANTY; without even the implied warranty of  MERCHANTABILITY or FITNESS
% FOR A PARTICULAR PURPOSE. See the GNU General Public License for more details.
%
% You should have received a copy of the GNU General Public License along with
% this program.  If not, see <http://www.gnu.org/licenses/>.

\renewcommand\chaptername{}
\chapter[सम्पादकीयम्]{सम्पादकीयम्}
\markboth{सम्पादकीयम्}{}
\fontsize{14}{21}\selectfont
\renewcommand{\thefootnote}{\small{\engtextfont \arabic{footnote}}}
\begin{center}
शुद्धाद्धातोः प्रकृत्यां णिचि सनि यङि वा नामजात्सोपसर्गा-\nopagebreak\\
द्धातोर्वा तिङ्कृदन्तं विकरणविधिभिश्चागमादेशकार्यैः।\\
कृच्छब्दात्तद्धितान्तं प्रतिपदजनकात्साधयन् सुब्विभक्तिं\nopagebreak\\
व्याकुर्वन् सर्वशास्त्रं जयति गुरुवरः प्रक्रियाज्ञाननेत्रः॥\\
\end{center}
\begin{sloppypar}\hyphenrules{nohyphenation}\justifying\noindent\hspace{10mm} {\engtextfont \lqtwo Deriving the conjugational form (\textit{tiṅanta}) and the form with a primary suffix (\textit{kṛdanta}) using the rules of inserted conjugational affixes (\textit{vikaraṇa}‑s) and the operations of augmentation (\textit{āgama}) and substitution (\textit{ādeśa}) in the natural (\textit{prakṛti}), causative (\textit{ṇic}), desiderative (\textit{san}), or intensive (\textit{yaṅ}) sense from an original root (\textit{dhātu}), or from a denominative root (\textit{nāmadhātu}), or from a root with a prefix (\textit{upasarga}); [deriving] the form with a secondary suffix (\textit{taddhitānta}) from a form with a primary suffix (\textit{kṛdanta}); [and deriving] the inflected form (\textit{subanta}) from a lemma (\textit{prātipadika})—thus explaining the entire scripture [of \textit{Vyākaraṇa}]—the foremost Guru, one of whose eyes is the knowledge of the derivational process (\textit{prakriyā}) [of \textit{Vyākaraṇa}], is [ever] victorious.\rqtwo}\end{sloppypar}
\begin{center}
अष्टाध्यायीं सभाष्यां सवररुचिकृतिं स्वस्य कण्ठे दधानो\nopagebreak\\
न्यायान् सर्वान् विजानन् भरणहरिकृतौ ब्रह्म वाक्यं प्रकीर्णम्।\\
सिद्धान्तान् कारिकार्थान् फणिपतिरचनाः कौण्डभट्टार्यसारं\nopagebreak\\
चक्षाणः शाब्दबोधं जयति गुरुवरो दर्शनज्ञाननेत्रः॥\\
\end{center}
\begin{sloppypar}\hyphenrules{nohyphenation}\justifying\noindent\hspace{10mm} {\engtextfont \lqtwo Having committed to His memory the \textit{Aṣṭādhyāyī}, along with the work of Vararuci (the \textit{Vārttika}‑s) and the commentary (\textit{Mahābhāṣyam}); specially knowing all the axioms (\textit{nyāya}‑s), [the three books titled] \textit{Brahma}, \textit{Vākya} and \textit{Prakīrṇa} in the work of Bhartṛhari (\textit{Vākyapadīyam}), the principles (\textit{siddhānta}‑s) [of \textit{Vyākaraṇa}], the meanings of the \textit{kārikā}‑s [in the \textit{Vaiyakaraṇa\-siddhānta\-kārikāḥ}], the works of Nāgeśa, and the \textit{sāra} (\textit{Vaiyākaraṇa\-bhūṣaṇa\-sāraḥ}) of the noble Kauṇḍabhaṭṭa; and expounding on verbal cognition (\textit{śābdabodha})—the foremost Guru, one of whose eyes is the knowledge of the philosophy [of \textit{Vyākaraṇa}], is [ever] victorious.\rqtwo}\end{sloppypar}
\begin{sloppypar}\hyphenrules{nohyphenation}\justifying\noindent\hspace{10mm} {\engtextfont This phenomenal work is the result of spontaneous dictation over only thirteen days by my Gurudeva, the polymath saint Jagadguru Rāmānandācārya Svāmī Rāmabhadrācārya, earlier known as Ācārya Giridharalāla Miśra Prajñācakṣu. While Gurudeva and His life need no introduction, the events which led to the authoring of this work certainly deserve a mention. Gurudeva entered the academic world of Vārāṇasī in 1971, after four years of schooling in Jaunpur. He completed His bachelor’s (\textit{śāstrī}) and master’s (\textit{ācārya}) degrees from the Sampurnanand Sanskrit University in 1974 and 1976, respectively. In 1976, He was awarded the Chancellor’s gold medal, along with seven gold medals. He then registered for a doctoral (\textit{vidyāvāridhi} or PhD) degree at the same university, but hardly spent any time on research over the next five years during which, as a wandering \textit{kathā} artiste (\textit{kathāvācaka}) and ascetic (\textit{tapasvī}), He traversed across the length and breadth of northern and western India. In 1981, when Gurudeva wanted to be initiated as a \textit{virakta saṃnyāsī} in the order (\textit{sampradāya}) of Rāmānanda, one of His close associates suggested that He complete the task which He had set on five years ago before severing all ties of \textit{pūrvāśrama}. Gurudeva then came to Vārāṇasī, and dictated this work over thirteen days, as I have heard from Him and His close associates. It may sound incredible to some, but it does not take much to believe. All one needs to do is to take a look at the endless literary output of Gurudeva consisting of more than one hundred books and innumerable articles, speeches, songs, and verses—and consider the fact that all of this comes from somebody who has been without eyesight since the age of two months and had no formal education till the age of seventeen. Although I was not born in 1981, I can attest to another such miraculous feat in December 2013 when Gurudeva authored a 300-page book—the \textit{Mūlārthabodhinī} commentary on the \textit{Bhaktamāla} of Gosvāmī Nārayaṇadāsa (Nābhājī)—in just sixteen hours of recording spread over nine days.\footnote{{\engtextfont I co-edited the book. While proofreading, I listened to the complete recording twice. There was never a pause to think or reflect, and nobody recited or read the original text of the \textit{Bhaktamāla} for Gurudeva during the recording.}} Being a witness to this feat, I have no doubts that this PhD thesis was dictated by Gurudeva over thirteen days. On September 24 1981, the thesis was examined by Paṇḍita Bhūpendrapati Tripāṭhī, the former Head of Department of \textit{Vyākaraṇa} at the Sampurnanand Sanskrit University, and recommended for the viva voce examination. The signature of Paṇḍita Bhūpendrapati Tripāṭhī in green ink still adorns the first page of the typed manuscript. The thesis was then successfully defended in Sanskrit by Gurudeva in the viva voce examination where the examiners included stalwarts like Paṇḍita Kālikāprasāda Śukla and Paṇḍita Paṭṭābhirāma Śāstrī.\footnote{{\engtextfont It is unfortunate that no recording of the session exists—technology was not as ubiquitous in India then as today. We all can only imagine what a wonderful treat for the ears it would have been to hear the conversations between the examiners and the researcher.}}}\end{sloppypar}
\begin{sloppypar}\hyphenrules{nohyphenation}\justifying\noindent\hspace{10mm} {\engtextfont I had read about this work in the bibliography of Gurudeva published in many books authored by Him, as well as in \textit{Svarṇayātrā}, Gurudeva’s autobiographical account of His life till the year 2000. I was very curious to read this work given its rather interesting title. I unsuccesfully tried to search for this work at the Citrakūṭa \textit{āśrama} of Gurudeva during my several visits to the \textit{āśrama} between 2009 to 2012. Nobody at the \textit{āśrama} remembered where Gurudeva’s copy of the thesis was. In 2011, I tried to find it in the Rajkot \textit{āśrama} of Gurudeva with no success. I had almost given up on my search and had decided to go to the Sampurnanand Sanskrit University, hoping against hope to procure this thirty-year old thesis, when fortunately, in the month of June 2012, I paid my first visit with my \textit{gurubhrātā} Pavana Śarmā to \textit{Vasiṣṭhāyanam}, Gurudeva’s \textit{āśrama} at Haridvāra. There, while going through a stack of rare and old books, I found Gurudeva’s copy of the thesis—buried under many other books, and with a torn cover, weak binding, and yellow and brittle pages. I flipped through its pages and started reading—like a curious child, who has just learned the basics of linear algebra, going through derivations in the notes of Isaac Newton or Carl Friedrich Gauss. Though I had been teaching myself \textit{Vyākaraṇa} since 2007, my five years of self-learning was cast away by reading a few lines of the thesis. So engrossed I was in the two or three derivations of the work that the typographic mistakes became invisible to me. After around five minutes, I could not read it any further—it was too much for me to understand. I put down the thesis and said to myself, “This work has to be published, else one day it may be lost forever.” Gurudeva kindly gave me the permission to take the thesis with me, and also gave His blessing to my wish. I feel His blessing is what has materialized as this editor’s note after three years.}\end{sloppypar}
\begin{sloppypar}\hyphenrules{nohyphenation}\justifying\noindent\hspace{10mm} {\engtextfont For those who work in the investment banking industry, time is a luxury they can rarely afford. The typed manuscript was full of errors—unfortunately the scribe and the typist did not have even basic knowledge of Sanskrit. Due to the many mistakes, I abandoned my idea of typing the work myself as it was too time-consuming. I decided to scan the book and posted a message on several mailing lists asking for help in data-entry. I was enthused by the several responses, some of them offering to work for free, that I received.\footnote{{\engtextfont And also amused by a self-proclaimed Sanskrit scholar who tried his best to fleece me by quoting a ludicrously high price. Frustrated with his repeated questioning on who I was engaging and how much I was paying, when I told him even esteemed scholars were quoting one-fourth of his rate, he responded nonchalantly, “That is very reasonable [sic].”}} One of the responses was from the Bangalore-based scholar \textit{Vedavāridhi} Dr. P. Ramanujan, the \textit{āsthāna vidvān} (assembly scholar) of the Ahobilamaṭha, an authority on the \textit{Kṛṣna Yajurveda}’s \textit{Taittirīya Śākhā}, and currently a member of the Second Sanskrit Commission. Dr. Ramanujan’s Parankushachar Institute of Vedic Studies (PIVS) is engaged in the digitization and publication of many Sanskrit works. Dr. K. S. Mukundan of the institute ably typed the work in a few months and Dr. Ramanujan did the first round of proofreading. I offer my fervent thanks to both Dr. Mukundan and Dr. Ramanujan without whose help this work would not have been published.}\end{sloppypar}
\begin{sloppypar}\hyphenrules{nohyphenation}\justifying\noindent\hspace{10mm} {\engtextfont Over the next few months, I proofread and edited the work, adding original verses of \textit{Adhyātma Rāmāyaṇa} and citing the work’s around 1650 references in the process. Several members of the \textit{Bhāratīyavidvatpariṣad} mailing list helped me trace the original sources of many citations, and I thank all of them. The first edition of the work was released online in HTML format in June 2013, one year after I had first seen the work. The second edition, which made many corrections, was released online in HTML format in October 2013.}\end{sloppypar}
\begin{sloppypar}\hyphenrules{nohyphenation}\justifying\noindent\hspace{10mm} {\engtextfont Although the second edition was complete and many typographic mistakes in the typed manuscript had been corrected, I was nevertheless not satisfied as I knew I had not understood the work and may had not done complete editorial justice to the work due to the lack of my understanding. In addition, I knew that at some places words or sentences were missing in the manuscript, leading to a lack of continuity. The following words of Gosvāmī Tulasīdāsa in the \textit{Rāmacaritamānasa} (1.30A-B) were echoing in my mind—}\end{sloppypar}
\vspace{-2mm}
\begin{center}
मैं पुनि निज गुरु सन सुनी कथा सो सूकरखेत।\nopagebreak\\
समुझी नहिं तसि बालपन तब अति रहेउँ अचेत॥\\
श्रोता बक्ता ग्याननिधि कथा राम कै गूढ़।\nopagebreak\\
किमि समुझौं मैं जीव जड़ कलि मल ग्रसित बिमूढ़॥\nopagebreak\\
\end{center}
\vspace{-2mm}
\begin{sloppypar}\hyphenrules{nohyphenation}\justifying\noindent\hspace{10mm} {\engtextfont \lqtwo And then I heard the same \textit{kathā} from my Guru at \textit{Vārāhakṣetra}, but I could not understand it as I was extremly ignorant in that phase of childhood. The \textit{kathā} of Rāma is difficult to understand; the listeners and speakers are rich with knowledge. How could I, an especially foolish and unintelligent being seized by the vices of the \textit{Kali} age, understand it?\rqtwo}\end{sloppypar}
\begin{sloppypar}\hyphenrules{nohyphenation}\justifying\noindent\hspace{10mm} {\engtextfont The answer came from what Tulasīdāsa had said next in the 	\textit{Rāmacaritamānasa} (1.31.1)—}\end{sloppypar}
\vspace{-2mm}
\begin{center}
तदपि कही गुरु बारहिं बारा। समुझि परी कछु मति अनुसारा॥\nopagebreak\\
\end{center}
\vspace{-2mm}
\begin{sloppypar}\hyphenrules{nohyphenation}\justifying\noindent\hspace{10mm} {\engtextfont \lqtwo Still, the Guru narrated it again and again, and then I understood it somewhat, in accordance with my intellect.\rqtwo}\end{sloppypar}
\begin{sloppypar}\hyphenrules{nohyphenation}\justifying\noindent\hspace{10mm} {\engtextfont I decided to read the work, my edited version, again and again. After three iterations, I was confident of my understanding, and decided to publish a third edition with footnotes reflecting my insights. In addition, I decided that the this new edition would be typeset in \XeLaTeX{}. The typesetting was completed in December 2013, following which I started editing the work again. My understanding of grammatical process and philosophy increased manifold while editing the work second time, and I could successfully solve the derivations for the explanations. These derivations were also included as footnotes. The work on the third edition began in January 2014 and was completed in February 2015, interrupted by another book of Gurudeva which I proofread, typeset and co-edited;\footnote{{\engtextfont This book was the \textit{Mūlārthabodhinī} commentary on the \textit{Bhaktamāla}.}} and my move from Hong Kong to Mumbai in July 2014. From February to August 2015, I was occupied with authoring, typesetting, designing, and publishing my first English book, \textit{Mahāvīrī: Hanumān-Cālīsā Demystified}.\footnote{{\engtextfont The book is a translation and expansion, with notes, of the \textit{Mahāvīrī} commentary on the \textit{Hanumān-Cālīsā} by Gurudeva.}} This delayed the online publication of the third edition of this book, \textit{Adhyātmarāmāyaṇe’\-pāṇinīya\-prayogāṇāṃ Vimarśaḥ}.}\end{sloppypar}
\begin{sloppypar}\hyphenrules{nohyphenation}\justifying\noindent\hspace{10mm} {\engtextfont The text of the third edition with nearly 1,200 footnotes runs into nearly 375 pages now, as opposed to around 240 pages for the text of the first and second editions. The number of works cited in the second addition is close to 175 as opposed to 65 in the first and second editions. Hundreds of \textit{vyutpatti}‑s (etymologies) and \textit{prakriyā}‑s (derivations) have been included in footnotes. Most of the original text in this third edition is the same as that in the first and second editions except for orthographic and editorial corrections, and supplying the text missing from the manuscript after consultations with Gurudeva. Some minor additions have been made at some places following my discussions with Gurudeva regarding several usages and forms, in order to facilitate better understanding of the matter at hand.}\end{sloppypar}
\begin{sloppypar}\hyphenrules{nohyphenation}\justifying\noindent\hspace{10mm} {\engtextfont I am indebted to the many sources I have used for my understanding and editing of Gurudeva’s work. The Hindi commentaries on the \textit{Aṣtādhyāyi} by Devaprakāśa Pātañjala and Paṇḍita Īśvaracandra (each in two volumes), the six-volume English commentary by Prof. Ramānātha Śarmā, and the three-volume Hindi commentary on the \textit{Laghu\-siddhānta\-kaumudī} by Ācārya Govindaprasāda Śarmā have been of invaluable help. Their focus on the derivational process of Pāṇinian grammar is remarkable. Puṣpā Dīkṣita’s \textit{Aṣṭādhyāyī Sahajabodha}, T.~R.~Kṛṣṇācārya’s \textit{Bṛhaddhātu\-rūpāvalī}, and S.~Rāmasubrahmaṇya Śāstrī’s \textit{Kṛdanta\-rūpamālā} have been especially helpful in understanding the third chapter (\textit{Dhātuprakaraṇam}), arguably the most involved chapter of the work.\footnote{{\engtextfont I have heard that mastering verbal derivations is the most challenging part of Sanskrit grammar, and I can attest to the same—(s)he who can derive verbal forms can easily derive all other forms.}} The \textit{Kṛdanta\-rūpamālā} has also been helpful in understanding the second chapter (\textit{Kṛttaddhita\-prakaraṇam}). The \textit{Upasargārtha\-candrikā} of Cārudeva Śāstrī is another work which I have found very useful. The \textit{Śabda\-kalpa\-drumaḥ} and \textit{Vācaspatyam}—possibly the greatest dictionaries ever written in any language—and the \textit{Amarakoṣa} and its various commentaries (especially \textit{Vyākhyā\-sudhā}) have been helpful for many etymologies, as has been Apte’s Sanskrit-Hindi dictionary. Comments and responses over emails from members of the \textit{Bhāratīya\-vidvat\-pariṣad} mailing list and discussions over phone and email with Mātājī Puṣpā Dīkṣita, Prof. Ramānātha Śarmā, and Prof. George Cardona on specific forms and derivations have greatly benefitted my understanding. The meticulous proofreading by Dr.~H.~N.~Bhat has helped correct several errors in the draft and add some more insightful footnotes. Lastly, I am indebted to all the traditional commentaries including the great \textit{Mahābhāṣya}, the brilliant \textit{Kāśikā}, the useful \textit{Nyāsa}, the illuminating \textit{Siddhānta\-kaumudī}, the delightful \textit{Bāla\-manoramā}, the enlightening \textit{Tattva\-bodhinī}, and the small yet illuminating \textit{Laghu\-siddhānta\-kaumudī}.}\end{sloppypar}
\begin{sloppypar}\hyphenrules{nohyphenation}\justifying\noindent\hspace{10mm} {\engtextfont I cannot thank enough Prof. Mādhava Deśapāṇḍe, Devarṣi Kalānātha Śāstrī, Dr. Himāṃśu Poṭā, and Dr. Baladevānanda Sāgara for taking time out of their busy schedules to write \textit{praśasti}‑s for this work. I offer special thanks to Śatāvadhāni R.~Gaṇeśa for writing a \textit{citrakāvya praśasti} in praise of Gurudeva. All these are exemplary scholars in the field of Sanskrit, and it is a privilege for me that they have adorned my first edited work in Sanskrit with their words. I shall be forever indebted to all of them.}\end{sloppypar}
\begin{sloppypar}\hyphenrules{nohyphenation}\justifying\noindent\hspace{10mm} {\engtextfont Before I end this note, I would like to offer thanks to my near and dear ones who made this work possible. The cooperation of my wife was indispensable in completion of the work—her support has always been present for the digitization and editing of Gurudeva’s works. My children Nilayā and Nirāmāya often made me get up from long sessions spent with my books and laptop, and return afresh to solve intricate derivations. My elder sister and her children were often troubled by my early morning and late night work on the this edition for nearly a month in Hong Kong, and were very kind to adjust. Relentless and selfless hard work is what my mother has taught me, not by her words but by her actions, and the same spirit kept me going during the editing. The genius of my father and his indomitable self-belief has been and will continue to be my inspiration. Lastly, the work is dedicated to my late grandparents, Śrīmatī Kausalyā Devī Miśra and Śrī Śrīgopāla Miśra, to whom I owe my love of all the three treasures of humans—literature (\textit{sāhitya}), music (\textit{saṅgīta}), and art (\textit{kalā}).}\end{sloppypar}
\begin{sloppypar}\hyphenrules{nohyphenation}\justifying\noindent\hspace{10mm} {\engtextfont No part of this work would have been even conceivable without the divine grace of Gurudeva, the saint of our times and the personification of all of India’s traditional knowledge. I bow to His lotus-feet again and again for giving me this opportunity to serve Him and taking time out to adress doubts I had while editing the work.}\end{sloppypar}
\begin{sloppypar}\hyphenrules{nohyphenation}\justifying\noindent\hspace{10mm} {\engtextfont I respectfully offer this work to all scholars and students of Sanskrit grammar, and request them to excuse and point out any editorial mistakes that may have still remained. I summarize my humble effort as—}\end{sloppypar}
\vspace{-2mm}
\begin{center}
रामभद्रो हि जानाति रामभद्रसरस्वतीम्।\nopagebreak\\
रामभद्रो हि जानाति रामभद्रसरस्वतीम्॥\nopagebreak\\
तत्कृपयैव जानाति नित्यानन्दः क्वचित्क्वचित्॥\nopagebreak\\
\end{center}
\vspace{-2mm}
\begin{sloppypar}\hyphenrules{nohyphenation}\justifying\noindent\hspace{10mm} {\engtextfont \lqtwo Verily, [only] Rāmabhadrācārya understands the speech of the auspicious Rāma. Verily, [only] the auspicious Rāma understands the speech of Rāmabhadrācārya. By the divine grace of both, Nityānanda understands [the speech of Rāmabhadrācārya] here and there.\rqtwo}\end{sloppypar}
\vspace{5mm}
\raggedleft{{\engtextfont Nityānanda Miśra}}\\
\raggedleft{{\engtextfont Mumbai, August 22 2015}}\\
\vspace{5mm}
