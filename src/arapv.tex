% Nityanand Misra: LaTeX code to typeset a book in Sanskrit
% Copyright (C) 2016 Nityanand Misra
%
% This program is free software: you can redistribute it and/or modify it under
% the terms of the GNU General Public License as published by the Free Software
% Foundation, either version 3 of the License, or (at your option) any later
% version.
%
% This program is distributed in the hope that it will be useful, but WITHOUT
% ANY WARRANTY; without even the implied warranty of  MERCHANTABILITY or FITNESS
% FOR A PARTICULAR PURPOSE. See the GNU General Public License for more details.
%
% You should have received a copy of the GNU General Public License along with
% this program.  If not, see <http://www.gnu.org/licenses/>.

\documentclass[twoside]{book}
\usepackage[margin=2cm,paperwidth=180mm,paperheight=270mm]{geometry}
\usepackage{nameref}
\usepackage{titlesec}
\usepackage[usenames,dvipsnames]{xcolor}
\usepackage{ulem}
\usepackage{calc}
\usepackage{enumitem}
\usepackage{tikz}
\usetikzlibrary{positioning,shapes,shadows,arrows}
\usepackage{pdfpages}
\usepackage[titles]{tocloft}
\usepackage{changepage}
\usepackage{polyglossia}
\usepackage{fancyhdr}
\usepackage{fontspec}
\usepackage{metalogo}
\usepackage{array}
\usepackage{longtable}
\usepackage{setspace}
\usepackage{xstring}
\usepackage{ragged2e}
\usepackage{adjustbox}
\usepackage{afterpage}
\raggedbottom
\setmainlanguage{hindi}
\setotherlanguage{english}
\setmainfont[Script=Devanagari]{Sanskrit 2003 NM}
\newfontfamily\devanagarifont[Script=Devanagari]{Sanskrit 2003 NM}
\newfontfamily{\englishfont}{Times New Roman}
\newfontfamily{\engtextfont}{Charis SIL}
\newcommand{\arrow}{{\englishfont →}\ }
\newcommand{\devanagarinumeral}[1]{
  \devanagaridigits{\number\csname c@#1\endcsname}}
\makeatletter
\renewcommand*{\cleardoublepage}{\clearpage\if@twoside \ifodd\c@page\else
  \hbox{}
  \thispagestyle{empty}
  \newpage
  \if@twocolumn\hbox{}\newpage\fi\fi\fi}
\makeatother
\titleformat{\chapter}[display]
  {\bfseries\Large}
  {\centering\huge{\chaptertitlename}}
  {1ex}
  {\titlerule\vspace{1ex}\centering}
  [\vspace{1ex}\titlerule]
\renewcommand\cftchapfont{\LARGE\bfseries}
\renewcommand\cftsecfont{\Large}
\renewcommand\cftchappagefont{\LARGE\bfseries}
\renewcommand\cftsecpagefont{\Large}
\setlength{\cftsecnumwidth}{4em}
\usepackage{footmisc}
\makeatletter
  \newcommand\normallarge{\@setfontsize\normallarge{11pt}{13}}
\makeatother
\renewcommand{\footnotelayout}{\hspace{1mm}\large}
\usepackage{fixltx2e}
\newcommand{\lqone}{{\fontsize{36}{18}\selectfont \textsubscript{\textcolor{Gray}{\englishfont ‘\hspace{-1.5mm}‘}}}}
\newcommand{\lqtwo}{{\fontsize{36}{18}\selectfont \textsubscript{\textcolor{Gray}{\englishfont ‘\hspace{-1.5mm}‘}}}}
\newcommand{\rqone}{{\fontsize{36}{18}\selectfont \textsubscript{\textcolor{Gray}{\englishfont ’\hspace{-1.5mm}’}}}}
\newcommand{\rqtwo}{{\fontsize{36}{18}\selectfont \textsubscript{\textcolor{Gray}{\englishfont ’\hspace{-1.5mm}’}}}}
\interfootnotelinepenalty=10000
\newcounter{dummy}
\usepackage{hyperref,bookmark}
\hypersetup{
	pdftitle={Adhyātmarāmāyaṇe’pāṇinīyaprayogāṇāṃ Vimarśaḥ},
	pdfauthor={Jagadguru Rāmānandācārya Svāmī Rāmabhadrācārya},
	pdfkeywords={Adhyātma Rāmāyaṇa, Pāṇinian grammar, Sanskrit, Svāmī Rāmabhadrācārya},
	pdfsubject={Deliberation on the non-Pāṇinian usages in the Adhyātma Rāmāyaṇa},
	bookmarksnumbered,
	pdfpagelayout=TwoPageRight,
	bookmarksopen=true,
	pdfstartview={FitH},
	colorlinks,
	urlcolor=cyan,
	linkcolor=blue
}
\begin{document}
\includepdf[width=\paperwidth,trim=204mm 0mm 0mm 0mm]{arapvcover.pdf}
\mbox{}
\thispagestyle{empty}
\newpage

\pagenumbering{gobble}
\makeatletter
\renewcommand{\thesection}{
  \ifnum\c@chapter<0 \devanagarinumeral{section}
  \else \thechapter\hspace{-1mm}.\devanagarinumeral{section}\hspace{-1mm}
  \fi
}
\makeatother
\makeatletter
\renewcommand{\thechapter}{
  \ifnum\c@chapter<1{}
  \else \devanagarinumeral{chapter}
  \fi
}
\makeatother
\titleformat{\chapter}[display]
  {\bfseries\Large}
  {\centering\huge{\chaptertitlename}}
  {1ex}
  {\titlerule\vspace{1ex}\centering\Huge\bfseries\color{blue}}
  [\bfseries\color{black}\vspace{0.5ex}\titlerule]
\renewcommand{\chaptermark}[1]{\markboth{#1}{}}
\renewcommand{\sectionmark}[1]{\markright{\thesection \ #1}}
\begin{titlepage}
\begin{center}
\pdfbookmark[1]{आवरणपृष्ठम्}{CoverPage}
{\fontsize{14}{21} \selectfont \textcolor{BrickRed}{२०३८तमे वैक्रमेऽब्दे सम्पूर्णानन्दसंस्कृतविश्वविद्यालयस्य\\[2pt] \textcolor{red}{विद्यावारिधिः} (पीएच्.डी.) इत्युपाधये प्रस्तुतः शोधप्रबन्धः}}\\[10pt]
{\fontsize{27}{40.5} \selectfont \textcolor{blue}{अध्यात्मरामायणेऽपाणिनीयप्रयोगाणां विमर्शः}}\\[30pt]
 \vfill
{\fontsize{14}{21} \selectfont प्रणेतारः}\\[5pt]
{\fontsize{16}{24} \selectfont \textcolor{BrickRed}{पदवाक्यप्रमाणपारावारीणाः षड्दर्शनपरमप्रवीणाः सनातनधर्मसंरक्षणधुरीणाः}}\\[5pt]
{\fontsize{21}{31.5} \selectfont \textcolor{blue}{आचार्यगिरिधरलालमिश्राः प्रज्ञाचक्षुषः}}\\[5pt]
{\fontsize{16}{24} \selectfont \textcolor{BrickRed}{(तुर्याश्रमे \textcolor{blue}{जगद्गुरुरामानन्दाचार्यस्वामिरामभद्राचार्याः} इति ख्याताः)}}\\[7pt]
\vfill
{\fontsize{14}{21} \selectfont \textcolor{BrickRed}{स च विमर्शो गौतमान्वयेन पिपरागौतमग्राममूलकेन पण्डितगायत्रीभूषणमिश्रसूनुना}}\\[5pt]
{\fontsize{18}{24} \selectfont \textcolor{blue}{नित्यानन्दमिश्रेण}}\\[5pt]
{\fontsize{14}{21} \selectfont \textcolor{BrickRed}{यथामति सम्पादित उद्धरणमूलग्रन्थनामश्लोकसङ्ख्याटिप्पणीप्रक्रियाभिः संवर्धितश्च}}
\end{center}
\pagebreak
{\fontsize{14}{21} \selectfont \textcolor{red}{प्रकाशकः}}\\[-4pt]

{\fontsize{14}{21} \selectfont श्रीतुलसीपीठसेवान्यासः}\\[-4pt]

{\fontsize{14}{21} \selectfont चित्रकूटः, सतनाजनपदः, मध्यप्रदेशः, भारतम्}\\[-4pt]
\vfill
\noindent {\fontsize{14}{21} \selectfont \textcolor{red}{तृतीयसङ्गणकीयसंस्करणम्}}\\[-4pt]

{\fontsize{14}{21} \selectfont \bfseries विजयदशमी, विक्रमाब्दः २०७१}\\[-4pt]

{\fontsize{14}{21} \selectfont \begin{english}(October 22, 2015)\end{english}}\\[-4pt]
\vfill
\newfontfamily{\englishfont}{Arial}
\noindent {\fontsize{14}{21} \selectfont \textcolor{red}{\begin{english}©\end{english} सर्वाधिकारः}}\\[-4pt]
\newfontfamily{\englishfont}{Times New Roman}

{\fontsize{14}{21} \selectfont प्रणेतृसम्पादकायत्तः}\\[-4pt]
 \vfill
{\fontsize{14}{18} \selectfont \begin{sloppypar}\hyphenrules{nohyphenation}\justifying\begin{english} \noindent Cover art created using two images. The first image is of a manuscript of the work Śābdabodha (MS Add.2464) available from the UCDL website under \href{http://cudl.lib.cam.ac.uk/view/MS-ADD-02464/1}{http://cudl.lib.cam.ac.uk/view/MS-ADD-02464/1}. The second image is that of a painting from Himachal Pradesh (1775-1800) titled as ‘Rama’s Court, Folio from a Ramayana (Adventures of Rama)’ available from the LACMA website under \href{http://collections.lacma.org/node/198599}{http://collections.lacma.org/node/198599}.\end{english}\end{sloppypar}}
 \vfill
\noindent {\fontsize{14}{21} \selectfont \textcolor{red}{पुस्तकप्राप्तिस्थानम्}}\\[-4pt]

{\fontsize{14}{21} \selectfont \begin{english}\href{http://www.jagadgururambhadracharya.org}{http://www.jagadgururambhadracharya.org}\end{english}}\\[-4pt]
 \vfill
\noindent {\fontsize{14}{21} \selectfont \textcolor{red}{अक्षरसंयोजकः}}\\[-4pt]

{\fontsize{14}{21} \selectfont \bfseries नित्यानन्दमिश्रः}\\[-6pt]

{\fontsize{12}{18} \selectfont \begin{english}Typeset in \XeLaTeX{} using \XeTeX{} engine version 0.9998.\end{english}}\\[-4pt]
\end{titlepage}
\titleformat{\section}[block]{\color{blue}\Large\bfseries\filright}{\color{blue}{\Large\thesection}}{0.25em}{}
\pagestyle{plain}
\frontmatter
\fancypagestyle{plain}{
  \fancyhf{}
  \renewcommand{\headrulewidth}{0pt}
  \renewcommand{\footrulewidth}{0pt}
}
\renewcommand\cftchapfont{\LARGE}
\renewcommand\cftsecfont{\Large}
\renewcommand\cftsubsecfont{\large}
{\pagestyle{plain}
\renewcommand{\contentsname}{अनुक्रमणिका}
\pdfbookmark[1]{अनुक्रमणिका}{table}
\begin{spacing}{1.6}
\tableofcontents
\end{spacing}
\cleardoublepage}

\fancyhf{}
\fancyfoot[C]{\thepage}
\fancyhead[LE,RO]{\large \rightmark}
\fancyhead[LO,RE]{\large \leftmark}
\pagenumbering{arabic}
\renewcommand{\thepage}{\large \devanagarinumeral{page}}
\renewcommand{\thedummy}{\devanagarinumeral{dummy}}
\pagestyle{fancy}

% Nityanand Misra: LaTeX code to typeset a book in Sanskrit
% Copyright (C) 2016 Nityanand Misra
%
% This program is free software: you can redistribute it and/or modify it under
% the terms of the GNU General Public License as published by the Free Software
% Foundation, either version 3 of the License, or (at your option) any later
% version.
%
% This program is distributed in the hope that it will be useful, but WITHOUT
% ANY WARRANTY; without even the implied warranty of  MERCHANTABILITY or FITNESS
% FOR A PARTICULAR PURPOSE. See the GNU General Public License for more details.
%
% You should have received a copy of the GNU General Public License along with
% this program.  If not, see <http://www.gnu.org/licenses/>.

\renewcommand\chaptername{}
\chapter[सम्पादकीयम्]{सम्पादकीयम्}
\markboth{सम्पादकीयम्}{}
\fontsize{14}{21}\selectfont
\renewcommand{\thefootnote}{\small{\engtextfont \arabic{footnote}}}
\begin{center}
शुद्धाद्धातोः प्रकृत्यां णिचि सनि यङि वा नामजात्सोपसर्गा-\nopagebreak\\
द्धातोर्वा तिङ्कृदन्तं विकरणविधिभिश्चागमादेशकार्यैः।\\
कृच्छब्दात्तद्धितान्तं प्रतिपदजनकात्साधयन् सुब्विभक्तिं\nopagebreak\\
व्याकुर्वन् सर्वशास्त्रं जयति गुरुवरः प्रक्रियाज्ञाननेत्रः॥\\
\end{center}
\begin{sloppypar}\hyphenrules{nohyphenation}\justifying\noindent\hspace{10mm} {\engtextfont \lqtwo Deriving the conjugational form (\textit{tiṅanta}) and the form with a primary suffix (\textit{kṛdanta}) using the rules of inserted conjugational affixes (\textit{vikaraṇa}‑s) and the operations of augmentation (\textit{āgama}) and substitution (\textit{ādeśa}) in the natural (\textit{prakṛti}), causative (\textit{ṇic}), desiderative (\textit{san}), or intensive (\textit{yaṅ}) sense from an original root (\textit{dhātu}), or from a denominative root (\textit{nāmadhātu}), or from a root with a prefix (\textit{upasarga}); [deriving] the form with a secondary suffix (\textit{taddhitānta}) from a form with a primary suffix (\textit{kṛdanta}); [and deriving] the inflected form (\textit{subanta}) from a lemma (\textit{prātipadika})—thus explaining the entire scripture [of \textit{Vyākaraṇa}]—the foremost Guru, one of whose eyes is the knowledge of the derivational process (\textit{prakriyā}) [of \textit{Vyākaraṇa}], is [ever] victorious.\rqtwo}\end{sloppypar}
\begin{center}
अष्टाध्यायीं सभाष्यां सवररुचिकृतिं स्वस्य कण्ठे दधानो\nopagebreak\\
न्यायान् सर्वान् विजानन् भरणहरिकृतौ ब्रह्म वाक्यं प्रकीर्णम्।\\
सिद्धान्तान् कारिकार्थान् फणिपतिरचनाः कौण्डभट्टार्यसारं\nopagebreak\\
चक्षाणः शाब्दबोधं जयति गुरुवरो दर्शनज्ञाननेत्रः॥\\
\end{center}
\begin{sloppypar}\hyphenrules{nohyphenation}\justifying\noindent\hspace{10mm} {\engtextfont \lqtwo Having committed to His memory the \textit{Aṣṭādhyāyī}, along with the work of Vararuci (the \textit{Vārttika}‑s) and the commentary (\textit{Mahābhāṣyam}); specially knowing all the axioms (\textit{nyāya}‑s), [the three books titled] \textit{Brahma}, \textit{Vākya} and \textit{Prakīrṇa} in the work of Bhartṛhari (\textit{Vākyapadīyam}), the principles (\textit{siddhānta}‑s) [of \textit{Vyākaraṇa}], the meanings of the \textit{kārikā}‑s [in the \textit{Vaiyakaraṇa\-siddhānta\-kārikāḥ}], the works of Nāgeśa, and the \textit{sāra} (\textit{Vaiyākaraṇa\-bhūṣaṇa\-sāraḥ}) of the noble Kauṇḍabhaṭṭa; and expounding on verbal cognition (\textit{śābdabodha})—the foremost Guru, one of whose eyes is the knowledge of the philosophy [of \textit{Vyākaraṇa}], is [ever] victorious.\rqtwo}\end{sloppypar}
\begin{sloppypar}\hyphenrules{nohyphenation}\justifying\noindent\hspace{10mm} {\engtextfont This phenomenal work is the result of spontaneous dictation over only thirteen days by my Gurudeva, the polymath saint Jagadguru Rāmānandācārya Svāmī Rāmabhadrācārya, earlier known as Ācārya Giridharalāla Miśra Prajñācakṣu. While Gurudeva and His life need no introduction, the events which led to the authoring of this work certainly deserve a mention. Gurudeva entered the academic world of Vārāṇasī in 1971, after four years of schooling in Jaunpur. He completed His bachelor’s (\textit{śāstrī}) and master’s (\textit{ācārya}) degrees from the Sampurnanand Sanskrit University in 1974 and 1976, respectively. In 1976, He was awarded the Chancellor’s gold medal, along with seven gold medals. He then registered for a doctoral (\textit{vidyāvāridhi} or PhD) degree at the same university, but hardly spent any time on research over the next five years during which, as a wandering \textit{kathā} artiste (\textit{kathāvācaka}) and ascetic (\textit{tapasvī}), He traversed across the length and breadth of northern and western India. In 1981, when Gurudeva wanted to be initiated as a \textit{virakta saṃnyāsī} in the order (\textit{sampradāya}) of Rāmānanda, one of His close associates suggested that He complete the task which He had set on five years ago before severing all ties of \textit{pūrvāśrama}. Gurudeva then came to Vārāṇasī, and dictated this work over thirteen days, as I have heard from Him and His close associates. It may sound incredible to some, but it does not take much to believe. All one needs to do is to take a look at the endless literary output of Gurudeva consisting of more than one hundred books and innumerable articles, speeches, songs, and verses—and consider the fact that all of this comes from somebody who has been without eyesight since the age of two months and had no formal education till the age of seventeen. Although I was not born in 1981, I can attest to another such miraculous feat in December 2013 when Gurudeva authored a 300-page book—the \textit{Mūlārthabodhinī} commentary on the \textit{Bhaktamāla} of Gosvāmī Nārayaṇadāsa (Nābhājī)—in just sixteen hours of recording spread over nine days.\footnote{{\engtextfont I co-edited the book. While proofreading, I listened to the complete recording twice. There was never a pause to think or reflect, and nobody recited or read the original text of the \textit{Bhaktamāla} for Gurudeva during the recording.}} Being a witness to this feat, I have no doubts that this PhD thesis was dictated by Gurudeva over thirteen days. On September 24 1981, the thesis was examined by Paṇḍita Bhūpendrapati Tripāṭhī, the former Head of Department of \textit{Vyākaraṇa} at the Sampurnanand Sanskrit University, and recommended for the viva voce examination. The signature of Paṇḍita Bhūpendrapati Tripāṭhī in green ink still adorns the first page of the typed manuscript. The thesis was then successfully defended in Sanskrit by Gurudeva in the viva voce examination where the examiners included stalwarts like Paṇḍita Kālikāprasāda Śukla and Paṇḍita Paṭṭābhirāma Śāstrī.\footnote{{\engtextfont It is unfortunate that no recording of the session exists—technology was not as ubiquitous in India then as today. We all can only imagine what a wonderful treat for the ears it would have been to hear the conversations between the examiners and the researcher.}}}\end{sloppypar}
\begin{sloppypar}\hyphenrules{nohyphenation}\justifying\noindent\hspace{10mm} {\engtextfont I had read about this work in the bibliography of Gurudeva published in many books authored by Him, as well as in \textit{Svarṇayātrā}, Gurudeva’s autobiographical account of His life till the year 2000. I was very curious to read this work given its rather interesting title. I unsuccesfully tried to search for this work at the Citrakūṭa \textit{āśrama} of Gurudeva during my several visits to the \textit{āśrama} between 2009 to 2012. Nobody at the \textit{āśrama} remembered where Gurudeva’s copy of the thesis was. In 2011, I tried to find it in the Rajkot \textit{āśrama} of Gurudeva with no success. I had almost given up on my search and had decided to go to the Sampurnanand Sanskrit University, hoping against hope to procure this thirty-year old thesis, when fortunately, in the month of June 2012, I paid my first visit with my \textit{gurubhrātā} Pavana Śarmā to \textit{Vasiṣṭhāyanam}, Gurudeva’s \textit{āśrama} at Haridvāra. There, while going through a stack of rare and old books, I found Gurudeva’s copy of the thesis—buried under many other books, and with a torn cover, weak binding, and yellow and brittle pages. I flipped through its pages and started reading—like a curious child, who has just learned the basics of linear algebra, going through derivations in the notes of Isaac Newton or Carl Friedrich Gauss. Though I had been teaching myself \textit{Vyākaraṇa} since 2007, my five years of self-learning was cast away by reading a few lines of the thesis. So engrossed I was in the two or three derivations of the work that the typographic mistakes became invisible to me. After around five minutes, I could not read it any further—it was too much for me to understand. I put down the thesis and said to myself, “This work has to be published, else one day it may be lost forever.” Gurudeva kindly gave me the permission to take the thesis with me, and also gave His blessing to my wish. I feel His blessing is what has materialized as this editor’s note after three years.}\end{sloppypar}
\begin{sloppypar}\hyphenrules{nohyphenation}\justifying\noindent\hspace{10mm} {\engtextfont For those who work in the investment banking industry, time is a luxury they can rarely afford. The typed manuscript was full of errors—unfortunately the scribe and the typist did not have even basic knowledge of Sanskrit. Due to the many mistakes, I abandoned my idea of typing the work myself as it was too time-consuming. I decided to scan the book and posted a message on several mailing lists asking for help in data-entry. I was enthused by the several responses, some of them offering to work for free, that I received.\footnote{{\engtextfont And also amused by a self-proclaimed Sanskrit scholar who tried his best to fleece me by quoting a ludicrously high price. Frustrated with his repeated questioning on who I was engaging and how much I was paying, when I told him even esteemed scholars were quoting one-fourth of his rate, he responded nonchalantly, “That is very reasonable [sic].”}} One of the responses was from the Bangalore-based scholar \textit{Vedavāridhi} Dr. P. Ramanujan, the \textit{āsthāna vidvān} (assembly scholar) of the Ahobilamaṭha, an authority on the \textit{Kṛṣna Yajurveda}’s \textit{Taittirīya Śākhā}, and currently a member of the Second Sanskrit Commission. Dr. Ramanujan’s Parankushachar Institute of Vedic Studies (PIVS) is engaged in the digitization and publication of many Sanskrit works. Dr. K. S. Mukundan of the institute ably typed the work in a few months and Dr. Ramanujan did the first round of proofreading. I offer my fervent thanks to both Dr. Mukundan and Dr. Ramanujan without whose help this work would not have been published.}\end{sloppypar}
\begin{sloppypar}\hyphenrules{nohyphenation}\justifying\noindent\hspace{10mm} {\engtextfont Over the next few months, I proofread and edited the work, adding original verses of \textit{Adhyātma Rāmāyaṇa} and citing the work’s around 1650 references in the process. Several members of the \textit{Bhāratīyavidvatpariṣad} mailing list helped me trace the original sources of many citations, and I thank all of them. The first edition of the work was released online in HTML format in June 2013, one year after I had first seen the work. The second edition, which made many corrections, was released online in HTML format in October 2013.}\end{sloppypar}
\begin{sloppypar}\hyphenrules{nohyphenation}\justifying\noindent\hspace{10mm} {\engtextfont Although the second edition was complete and many typographic mistakes in the typed manuscript had been corrected, I was nevertheless not satisfied as I knew I had not understood the work and may had not done complete editorial justice to the work due to the lack of my understanding. In addition, I knew that at some places words or sentences were missing in the manuscript, leading to a lack of continuity. The following words of Gosvāmī Tulasīdāsa in the \textit{Rāmacaritamānasa} (1.30A-B) were echoing in my mind—}\end{sloppypar}
\vspace{-2mm}
\begin{center}
मैं पुनि निज गुरु सन सुनी कथा सो सूकरखेत।\nopagebreak\\
समुझी नहिं तसि बालपन तब अति रहेउँ अचेत॥\\
श्रोता बक्ता ग्याननिधि कथा राम कै गूढ़।\nopagebreak\\
किमि समुझौं मैं जीव जड़ कलि मल ग्रसित बिमूढ़॥\nopagebreak\\
\end{center}
\vspace{-2mm}
\begin{sloppypar}\hyphenrules{nohyphenation}\justifying\noindent\hspace{10mm} {\engtextfont \lqtwo And then I heard the same \textit{kathā} from my Guru at \textit{Vārāhakṣetra}, but I could not understand it as I was extremly ignorant in that phase of childhood. The \textit{kathā} of Rāma is difficult to understand; the listeners and speakers are rich with knowledge. How could I, an especially foolish and unintelligent being seized by the vices of the \textit{Kali} age, understand it?\rqtwo}\end{sloppypar}
\begin{sloppypar}\hyphenrules{nohyphenation}\justifying\noindent\hspace{10mm} {\engtextfont The answer came from what Tulasīdāsa had said next in the 	\textit{Rāmacaritamānasa} (1.31.1)—}\end{sloppypar}
\vspace{-2mm}
\begin{center}
तदपि कही गुरु बारहिं बारा। समुझि परी कछु मति अनुसारा॥\nopagebreak\\
\end{center}
\vspace{-2mm}
\begin{sloppypar}\hyphenrules{nohyphenation}\justifying\noindent\hspace{10mm} {\engtextfont \lqtwo Still, the Guru narrated it again and again, and then I understood it somewhat, in accordance with my intellect.\rqtwo}\end{sloppypar}
\begin{sloppypar}\hyphenrules{nohyphenation}\justifying\noindent\hspace{10mm} {\engtextfont I decided to read the work, my edited version, again and again. After three iterations, I was confident of my understanding, and decided to publish a third edition with footnotes reflecting my insights. In addition, I decided that the this new edition would be typeset in \XeLaTeX{}. The typesetting was completed in December 2013, following which I started editing the work again. My understanding of grammatical process and philosophy increased manifold while editing the work second time, and I could successfully solve the derivations for the explanations. These derivations were also included as footnotes. The work on the third edition began in January 2014 and was completed in February 2015, interrupted by another book of Gurudeva which I proofread, typeset and co-edited;\footnote{{\engtextfont This book was the \textit{Mūlārthabodhinī} commentary on the \textit{Bhaktamāla}.}} and my move from Hong Kong to Mumbai in July 2014. From February to August 2015, I was occupied with authoring, typesetting, designing, and publishing my first English book, \textit{Mahāvīrī: Hanumān-Cālīsā Demystified}.\footnote{{\engtextfont The book is a translation and expansion, with notes, of the \textit{Mahāvīrī} commentary on the \textit{Hanumān-Cālīsā} by Gurudeva.}} This delayed the online publication of the third edition of this book, \textit{Adhyātmarāmāyaṇe’\-pāṇinīya\-prayogāṇāṃ Vimarśaḥ}.}\end{sloppypar}
\begin{sloppypar}\hyphenrules{nohyphenation}\justifying\noindent\hspace{10mm} {\engtextfont The text of the third edition with nearly 1,200 footnotes runs into nearly 375 pages now, as opposed to around 240 pages for the text of the first and second editions. The number of works cited in the second addition is close to 175 as opposed to 65 in the first and second editions. Hundreds of \textit{vyutpatti}‑s (etymologies) and \textit{prakriyā}‑s (derivations) have been included in footnotes. Most of the original text in this third edition is the same as that in the first and second editions except for orthographic and editorial corrections, and supplying the text missing from the manuscript after consultations with Gurudeva. Some minor additions have been made at some places following my discussions with Gurudeva regarding several usages and forms, in order to facilitate better understanding of the matter at hand.}\end{sloppypar}
\begin{sloppypar}\hyphenrules{nohyphenation}\justifying\noindent\hspace{10mm} {\engtextfont I am indebted to the many sources I have used for my understanding and editing of Gurudeva’s work. The Hindi commentaries on the \textit{Aṣtādhyāyi} by Devaprakāśa Pātañjala and Paṇḍita Īśvaracandra (each in two volumes), the six-volume English commentary by Prof. Ramānātha Śarmā, and the three-volume Hindi commentary on the \textit{Laghu\-siddhānta\-kaumudī} by Ācārya Govindaprasāda Śarmā have been of invaluable help. Their focus on the derivational process of Pāṇinian grammar is remarkable. Puṣpā Dīkṣita’s \textit{Aṣṭādhyāyī Sahajabodha}, T.~R.~Kṛṣṇācārya’s \textit{Bṛhaddhātu\-rūpāvalī}, and S.~Rāmasubrahmaṇya Śāstrī’s \textit{Kṛdanta\-rūpamālā} have been especially helpful in understanding the third chapter (\textit{Dhātuprakaraṇam}), arguably the most involved chapter of the work.\footnote{{\engtextfont I have heard that mastering verbal derivations is the most challenging part of Sanskrit grammar, and I can attest to the same—(s)he who can derive verbal forms can easily derive all other forms.}} The \textit{Kṛdanta\-rūpamālā} has also been helpful in understanding the second chapter (\textit{Kṛttaddhita\-prakaraṇam}). The \textit{Upasargārtha\-candrikā} of Cārudeva Śāstrī is another work which I have found very useful. The \textit{Śabda\-kalpa\-drumaḥ} and \textit{Vācaspatyam}—possibly the greatest dictionaries ever written in any language—and the \textit{Amarakoṣa} and its various commentaries (especially \textit{Vyākhyā\-sudhā}) have been helpful for many etymologies, as has been Apte’s Sanskrit-Hindi dictionary. Comments and responses over emails from members of the \textit{Bhāratīya\-vidvat\-pariṣad} mailing list and discussions over phone and email with Mātājī Puṣpā Dīkṣita, Prof. Ramānātha Śarmā, and Prof. George Cardona on specific forms and derivations have greatly benefitted my understanding. The meticulous proofreading by Dr.~H.~N.~Bhat has helped correct several errors in the draft and add some more insightful footnotes. Lastly, I am indebted to all the traditional commentaries including the great \textit{Mahābhāṣya}, the brilliant \textit{Kāśikā}, the useful \textit{Nyāsa}, the illuminating \textit{Siddhānta\-kaumudī}, the delightful \textit{Bāla\-manoramā}, the enlightening \textit{Tattva\-bodhinī}, and the small yet illuminating \textit{Laghu\-siddhānta\-kaumudī}.}\end{sloppypar}
\begin{sloppypar}\hyphenrules{nohyphenation}\justifying\noindent\hspace{10mm} {\engtextfont I cannot thank enough Prof. Mādhava Deśapāṇḍe, Devarṣi Kalānātha Śāstrī, Dr. Himāṃśu Poṭā, and Dr. Baladevānanda Sāgara for taking time out of their busy schedules to write \textit{praśasti}‑s for this work. I offer special thanks to Śatāvadhāni R.~Gaṇeśa for writing a \textit{citrakāvya praśasti} in praise of Gurudeva. All these are exemplary scholars in the field of Sanskrit, and it is a privilege for me that they have adorned my first edited work in Sanskrit with their words. I shall be forever indebted to all of them.}\end{sloppypar}
\begin{sloppypar}\hyphenrules{nohyphenation}\justifying\noindent\hspace{10mm} {\engtextfont Before I end this note, I would like to offer thanks to my near and dear ones who made this work possible. The cooperation of my wife was indispensable in completion of the work—her support has always been present for the digitization and editing of Gurudeva’s works. My children Nilayā and Nirāmāya often made me get up from long sessions spent with my books and laptop, and return afresh to solve intricate derivations. My elder sister and her children were often troubled by my early morning and late night work on the this edition for nearly a month in Hong Kong, and were very kind to adjust. Relentless and selfless hard work is what my mother has taught me, not by her words but by her actions, and the same spirit kept me going during the editing. The genius of my father and his indomitable self-belief has been and will continue to be my inspiration. Lastly, the work is dedicated to my late grandparents, Śrīmatī Kausalyā Devī Miśra and Śrī Śrīgopāla Miśra, to whom I owe my love of all the three treasures of humans—literature (\textit{sāhitya}), music (\textit{saṅgīta}), and art (\textit{kalā}).}\end{sloppypar}
\begin{sloppypar}\hyphenrules{nohyphenation}\justifying\noindent\hspace{10mm} {\engtextfont No part of this work would have been even conceivable without the divine grace of Gurudeva, the saint of our times and the personification of all of India’s traditional knowledge. I bow to His lotus-feet again and again for giving me this opportunity to serve Him and taking time out to adress doubts I had while editing the work.}\end{sloppypar}
\begin{sloppypar}\hyphenrules{nohyphenation}\justifying\noindent\hspace{10mm} {\engtextfont I respectfully offer this work to all scholars and students of Sanskrit grammar, and request them to excuse and point out any editorial mistakes that may have still remained. I summarize my humble effort as—}\end{sloppypar}
\vspace{-2mm}
\begin{center}
रामभद्रो हि जानाति रामभद्रसरस्वतीम्।\nopagebreak\\
रामभद्रो हि जानाति रामभद्रसरस्वतीम्॥\nopagebreak\\
तत्कृपयैव जानाति नित्यानन्दः क्वचित्क्वचित्॥\nopagebreak\\
\end{center}
\vspace{-2mm}
\begin{sloppypar}\hyphenrules{nohyphenation}\justifying\noindent\hspace{10mm} {\engtextfont \lqtwo Verily, [only] Rāmabhadrācārya understands the speech of the auspicious Rāma. Verily, [only] the auspicious Rāma understands the speech of Rāmabhadrācārya. By the divine grace of both, Nityānanda understands [the speech of Rāmabhadrācārya] here and there.\rqtwo}\end{sloppypar}
\vspace{5mm}
\raggedleft{{\engtextfont Nityānanda Miśra}}\\
\raggedleft{{\engtextfont Mumbai, August 22 2015}}\\
\vspace{5mm}


% Nityanand Misra: LaTeX code to typeset a book in Sanskrit
% Copyright (C) 2016 Nityanand Misra
%
% This program is free software: you can redistribute it and/or modify it under
% the terms of the GNU General Public License as published by the Free Software
% Foundation, either version 3 of the License, or (at your option) any later
% version.
%
% This program is distributed in the hope that it will be useful, but WITHOUT
% ANY WARRANTY; without even the implied warranty of  MERCHANTABILITY or FITNESS
% FOR A PARTICULAR PURPOSE. See the GNU General Public License for more details.
%
% You should have received a copy of the GNU General Public License along with
% this program.  If not, see <http://www.gnu.org/licenses/>.

\setcounter{footnote}{0}
\renewcommand\chaptername{}
\chapter[परिचयः]{परिचयः}
\markboth{परिचयः}{}
\fontsize{14}{21}\selectfont
\begin{sloppypar}\hyphenrules{nohyphenation}\justifying\noindent\hspace{10mm} {\engtextfont \textbf{Adhyātmarāmāyaṇe’pāṇinīyaprayogāṇāṃ Vimarśaḥ} (English: \textit{Deliberation on non-Pāṇinian usages in the Adhyātma Rāmāyaṇa}) is a comprehensive Sanskrit essay (\textit{nibandha}) in around 50,000 words, authored in 1981 by my Gurudeva, Ācārya Giridharalāla Miśra Prajñācakṣu (known in his current Āśrama as Jagadguru Rāmanandācārya Svāmī Rāmabhadrācārya). The work was spontaneously dictated by Gurudeva over only thirteen days in 1981. A disciple of Gurudeva, Dayāśaṅkara Pāṇdeya, was the scribe who took dictation from Gurudeva. The entire work was authored by Gurudeva without any book or references, with all the text of \textit{Adhyātma Rāmāyaṇa}, \textit{Aṣṭādhyāyī}, \textit{Mahābhāṣya} and many other works in the memory (\textit{kaṇṭhastha}) and all the Pāṇinian \textit{prakriyā}‑s in the mind (\textit{buddhistha}). At the end of the work, Gurudeva says—}\end{sloppypar}
\vspace{-2mm}
\begin{center}
बुद्ध्या श्रीगुरुपादपद्मरजसा संशुद्धया सादरं\nopagebreak\\
कृत्वा लेखकमाप्तशीलयशसं शिष्यं शिशुं राघवम्।\nopagebreak\\
बालो नष्टविलोचनो गिरिधरः शब्दान् विभाव्याऽत्मना\nopagebreak\\
बध्नाति स्म निबन्धमेतममलं तोषाय सीतापतेः॥
\end{center}
\vspace{-2mm}
\begin{sloppypar}\hyphenrules{nohyphenation}\justifying\noindent\hspace{10mm} {\engtextfont \lqtwo Making the infant Rāma, his disciple endowed with moral conduct and fame, as the scribe, the child Giridhara, devoid of physical vision, after respectfully examining words with his intellect which was made especially pure by the pollen from the lotus-feet of the revered Guru, composed this work for the pleasure of the lord of Sītā.\rqtwo}\end{sloppypar}
\begin{sloppypar}\hyphenrules{nohyphenation}\justifying\noindent\hspace{10mm} {\engtextfont The work was then typed and presented as a doctoral thesis (\textit{śodha\-prabandha}) at the Sampurnanand Sanskrit University, for which the degree of \textit{Vidyāvāridhi} (Ph.D.) was conferred upon Gurudeva in 1981. The thesis was reviewed by the grammarian (\textit{vaiyākaraṇa}) and epic-poet (\textit{mahākavi}) Kālikāprasāda Śukla, who wrote the following verse in the \textit{Vasantatilakā} metre to describe it—}\end{sloppypar}
\vspace{-2mm}
\begin{center}
शोधप्रबन्धपरिशीलनतः समन्तात्सञ्जायते मतमिदं मम युक्तियुक्तम्।\nopagebreak\\
शोधप्रबन्धमकरन्दमधुव्रतोऽयं विद्वद्विमृग्यविरुदं लभतामिदानीम्॥
\end{center}
\vspace{-2mm}
\begin{sloppypar}\hyphenrules{nohyphenation}\justifying\noindent\hspace{10mm} {\engtextfont \lqtwo My logical conclusion, that arises from having thoroughly examined the dissertation, is that he (Giridhara Miśra) is the bumblebee for the nectar of literary compositions for purification. May he now [effortlessly] obtain the praise and fame which is especially sought after by the learned.\rqtwo}\end{sloppypar}
\begin{sloppypar}\hyphenrules{nohyphenation}\justifying\noindent\hspace{10mm} {\engtextfont The work is divided into four parts—\textit{Prastāvanā}, \textit{Sandhi\-kāraka\-samāsa\-prakaraṇam}, \textit{Kṛttaddhita\-prakaraṇam}, and \textit{Dhātu\-prakaraṇam}.}\end{sloppypar}
\begin{sloppypar}\hyphenrules{nohyphenation}\justifying\noindent\hspace{10mm} {\engtextfont \textbf{Prastāvanā}: The introduction begins with tracing the tradition of \textit{Vyākaraṇa}. A wide range of topics are first covered like nature of \textit{Veda}, origin of \textit{Veda}, \textit{apauruṣeyatā} of \textit{Veda}, \textit{vedatrayī} and \textit{vedacatuṣṭayī}, \textit{śruti} and \textit{Veda}, \textit{lakṣaṇa} of \textit{vidhi} and \textit{niṣedha}, relation between \textit{śruti} and \textit{smṛti}, five \textit{sampradāya}‑s based on \textit{smṛti}‑s, five types of \textit{Vaiṣṇava upāsanā}‑s, \textit{Vedānta}, \textit{Purāṇa}, \textit{Itihāsa}, fourteen \textit{vidyā}‑s including six \textit{darśana}‑s, three types of \textit{āgama}‑s and two types of \textit{mārga}‑s, and the six \textit{vedāṅga}‑s—the chief amongst which is \textit{Vyākaraṇa}. Next, the importance of \textit{Vyākaraṇa} is stressed. The nine \textit{vyākaraṇa}‑s are mentioned, following which some unique aspects of Pāṇini’s grammar are discussed. The terms \textit{āpta} and \textit{śiṣṭa} are defined, along with generic definitions of \textit{śiṣṭa\-prayoga} and \textit{sādhutva}, which are to be followed by \textit{Vyākaraṇa} and not the other way round. \textit{Sādhutva} is then redefined in the context of the \textit{prakriyā} of Pāṇini’s grammar. Deeper insights going beyond the realm of grammar are arrived at from some \textit{sūtra}‑s of Pāṇini. The essay then delves on \textit{sūtratva} and \textit{lakṣaṇa} of six types of \textit{sūtra}‑s and their interplay is explained with the comprehensive example of \textit{iko yaṇaci} (PS 6.1.77). This is followed by the position of works of Kātyāyana and Patañjali in the \textit{Vyākaraṇa} tradition. \textit{Mahābhāṣya} and its position is described in detail with examples. Following this, \textit{prakriyā} and \textit{darśana}—the two eyes of the \textit{Vyākaraṇa} tradition—are described with the works of both, and the philosophy of \textit{Vyākaraṇa} tradition is compared with that of other \textit{darśana}‑s. \textit{Śābdbodha} and \textit{śakti} are then explained as per \textit{Vyākaraṇa}. The relevance of Pāṇini’s grammar is discussed in the context of the \textit{Rāmāyaṇa} tradition, including the \textit{Adhyātma Rāmāyaṇa}. Some insights into the \textit{Rāmāyaṇa} tradition from Pāṇini’s grammar are discussed, and all the fourteen \textit{Śiva Sūtra}‑s are then interpreted in the context of \textit{Rāmāyaṇa}. The question of seemingly non-Pāṇinian usages in \textit{śiṣṭaprayoga}‑s, like those in \textit{Adhyātma Rāmāyaṇa}, is raised. Gurudeva says that it is very much possible to explain all \textit{śiṣṭaprayoga}‑s by Pāṇini’s grammar. After explaining the the relation between \textit{Vyākaraṇa} and \textit{Adhyātma Rāmāyaṇa}, the latter work is described in some detail. \textit{Adhyātma Rāmāyaṇa}’s origin, poetic features, \textit{rasa}‑s, \textit{nāyaka} and its relevance and usefulness in the context of the \textit{Rāmāyaṇa} tradition is discussed. The author stresses that since both Pāṇini’s grammar and \textit{Adhyātma Rāmāyaṇa} come from Lord Śiva, there must be consistency (\textit{ekavākyatā}) between the two. Gurudeva says that in the \textit{Adhyātma Rāmāyaṇa}, there are 700-odd usages that “appear to be non-Pāṇinian.” Around half of these usages are examined and explained using the Pāṇinian framework (others being similar to those explained).}\end{sloppypar}
\begin{sloppypar}\hyphenrules{nohyphenation}\justifying\noindent\hspace{10mm} {\engtextfont \textbf{I. Sandhi\-kāraka\-samāsa\-prakaraṇam}: The first chapter is split into two parts and examines 140 sequential usages in the \textit{Adhyātma Rāmāyaṇa} pertaining to \textit{sandhi}, \textit{kāraka} and \textit{samāsa}. It begins with an insightful grammatical explanation of the compound \textit{Adhyātma Rāmāyaṇa}. Quite often two or three solutions, and sometimes upto six or seven solutions in the Pāṇinian framework are given for the seemingly non-Pāṇinian forms.}\end{sloppypar}
\begin{sloppypar}\hyphenrules{nohyphenation}\justifying\noindent\hspace{10mm} {\engtextfont \textbf{II. Kṛttaddhita\-prakaraṇam}: The second chapter is also split into two parts and examines 90 sequential usages in the \textit{Adhyātma Rāmāyaṇa} pertaining to \textit{kṛt} and \textit{taddhita} affixes. Again, multiple Pāṇinian explanations are offered for many usages.}\end{sloppypar}
\begin{sloppypar}\hyphenrules{nohyphenation}\justifying\noindent\hspace{10mm} {\engtextfont \textbf{IIII. Dhātu\-prakaraṇam}: The third chapter explains 135 sequential \textit{tiṅanta} usages in the \textit{Adhyātma Rāmāyaṇa} that are seemingly non-Pāṇinian. Mostly one and sometimes two or three explanations are offered in the Pāṇinian framework for these usages.}\end{sloppypar}
\vspace{4mm}
\begin{sloppypar}\hyphenrules{nohyphenation}\justifying\noindent {\engtextfont \textbf{Features}}\end{sloppypar}
\begin{sloppypar}\hyphenrules{nohyphenation}\justifying\noindent\hspace{10mm} {\engtextfont Some unique features of the work include—}\end{sloppypar}
\begin{enumerate}[itemsep=0mm,label={{\engtextfont \arabic*.}}]
\item \begin{sloppypar}\hyphenrules{nohyphenation}\justifying\noindent {\engtextfont Pāṇinian explanations of 365 seemingly non-Pāṇinian usages using various Pāṇinian \textit{sūtra}‑s, \textit{vārttika}‑s, \textit{kārikā}‑s, \textit{niyama}‑s, \textit{pari\-bhāṣā}‑s, and \textit{jñāpaka}‑s, in accordance with traditional commentaries.}\end{sloppypar}
\item \begin{sloppypar}\hyphenrules{nohyphenation}\justifying\noindent {\engtextfont More than 1500 citations from works of diverse fields, and many more from oral traditions. On including the editor’s footnotes and derivations in the third edition, there are more than 6500 citations from around 175 works.}\end{sloppypar}
\item \begin{sloppypar}\hyphenrules{nohyphenation}\justifying\noindent {\engtextfont \textit{Śiva Sūtra}‑s explained in the context of \textit{Rāmāyaṇa}, with a \textit{Rāmāyaṇa}-centric interpretation of each \textit{sūtra}. This section of the work compares with the \textit{Nandikeśvara Kāśikā}.}\end{sloppypar}
\item \begin{sloppypar}\hyphenrules{nohyphenation}\justifying\noindent {\engtextfont Hundreds of Pāṇinian \textit{prakriyā}‑s, some abridged and some detailed. In the second addition, hundreds of complete Pāṇinian derivations are given in the editor’s footnotes, corresponding to the abridged and detailed derivations by the author.}\end{sloppypar}
\item \begin{sloppypar}\hyphenrules{nohyphenation}\justifying\noindent {\engtextfont Lucid explanations of complex grammatical concepts.}\end{sloppypar}
\item \begin{sloppypar}\hyphenrules{nohyphenation}\justifying\noindent {\engtextfont Vivid descriptions in poetic style, with some \textit{daṇḍaka}-style \textit{samāsa}‑s formed from several hundreds of words compounded together.}\end{sloppypar}
\item \begin{sloppypar}\hyphenrules{nohyphenation}\justifying\noindent {\engtextfont Didactic approach with many counter-questions and doubts raised by \textit{nanu}, \textit{na ca}, et cetera, and all of them resolved in favour of the proposed solutions.}\end{sloppypar}
\item \begin{sloppypar}\hyphenrules{nohyphenation}\justifying\noindent {\engtextfont \textit{Nyāya}-styled \textit{lakṣaṇa}‑s of many concepts, both grammatical and non-grammatical.}\end{sloppypar}
\item \begin{sloppypar}\hyphenrules{nohyphenation}\justifying\noindent {\engtextfont \textit{Vaiyākaraṇa śābdabodha}‑s of common terms (\textit{śruti}, \textit{veda}, \textit{anuśāsana}, \textit{vyākaraṇa}, et cetera) and involved grammatical usages (\textit{vāraṇārtha}, \textit{karmamūlaka\-sambandha}, \textit{samāsa}, \textit{śaiṣika\-ṣaṣṭhī}, \textit{tatkaroti} usage, \textit{tadiva ācarati} usage, \textit{tiṅanta} usage from a \textit{pacādyajanta\-kvibanta nāmadhātu}, \textit{ṇijanta} usage, \textit{svārtha\-ṇijanta} usage, \textit{vartamāna\-sāmīpya} usage, et cetera).}\end{sloppypar}
\item \begin{sloppypar}\hyphenrules{nohyphenation}\justifying\noindent {\engtextfont Critical insights into many original verses of \textit{Adhyātma Rāmāyaṇa} using traditional methods of interpretation. The work may be thought of as a mini-commentary on the \textit{Adhyātma Rāmāyaṇa}.}\end{sloppypar}
\end{enumerate}
\begin{sloppypar}\hyphenrules{nohyphenation}\justifying\noindent\hspace{10mm} {\engtextfont Besides satisfying scholars of \textit{Vyākaraṇa} and the connoisseurs of \textit{Adhyātma Rāmāyaṇa}, the work is very useful for students learning Pāṇini’s \textit{Vyākaraṇa}. Having acquired all my limited learning in Sanskrit grammar exclusively from self-study (\textit{svādhyāya}), I can personally attest to the extraordinary benefits the work offers for grammar students, especially those studying by \textit{svādhyāya}. Just like a picture is worth a thousand words, similarly a \textit{prakriyā} is worth the understanding of tens of \textit{sūtra}‑s, a grammatical insight is worth tens of such \textit{prakriyā}‑s, and a \textit{śābdabodha} is worth tens of such grammatical insights. The work—replete with \textit{prakriyā}‑s, insights and \textit{śābdabodha}‑s—offers a rare source of learning for students and scholars of Pāṇinian \textit{Vyākaraṇa}.}\end{sloppypar}
\begin{sloppypar}\hyphenrules{nohyphenation}\justifying\noindent\hspace{10mm} {\engtextfont I am a quantitative analyst by profession, and Applied Statistics is one of my areas of work. From the viewpoint of Statistics, I see the Pāṇinian grammar as a parsinomious statistical model which Pāṇini formulated to explain the variation in the infinitely many \textit{śiṣṭa prayoga}‑s which formed his dataset. No statistical model with finite predictors can explain all of the variation in an infinitely large dataset. Pāṇini’s model was the best model ever formulated for this purpose, and could account for a very large degree of variation. Vararuci’s \textit{vārttika}‑s and Patañjali’s \textit{bhāṣya} extended the model by introducing additional model complexity, and explaining further variation in the data of \textit{śiṣṭa prayoga}‑s. All of these great grammarians were mathematicians, and I see them as statisticians since like Statistics, Sanskrit grammar also combines mathematics and philosophy—the two eyes of Sanskrit grammar being \textit{prakriyā} (mathematical derivations) and \textit{darśana} (philosophy). This work is also largely mathematical in nature, interpreting or extending Pāṇini’s model to explain the variation of \textit{śiṣṭa prayoga}‑s seen in the \textit{Adhyātma Rāmāyaṇa}. With this conclusion, I believe students and scholars of computational liguistics also stand to gain from the study of this work.}\end{sloppypar}
\begin{sloppypar}\hyphenrules{nohyphenation}\justifying\noindent\hspace{10mm} {\engtextfont The original work and the editor’s footnotes in the third edition (except for some citations) are entirely in Sanskrit. For the benefit of students of Sanskrit and linguistics, I plan to translate the work into English and Hindi some day. The next edition of the work will hopefully come with an English or Hindi translation.}\end{sloppypar}
\vspace{5mm}
\raggedleft{{\engtextfont Nityānanda Miśra}}\\
\raggedleft{{\engtextfont Mumbai, August 22 2015}}\\


% Nityanand Misra: LaTeX code to typeset a book in Sanskrit
% Copyright (C) 2016 Nityanand Misra
%
% This program is free software: you can redistribute it and/or modify it under
% the terms of the GNU General Public License as published by the Free Software
% Foundation, either version 3 of the License, or (at your option) any later
% version.
%
% This program is distributed in the hope that it will be useful, but WITHOUT
% ANY WARRANTY; without even the implied warranty of  MERCHANTABILITY or FITNESS
% FOR A PARTICULAR PURPOSE. See the GNU General Public License for more details.
%
% You should have received a copy of the GNU General Public License along with
% this program.  If not, see <http://www.gnu.org/licenses/>.

\setcounter{footnote}{0}
\renewcommand{\thefootnote}{\footnotesize{\devanagarinumeral{footnote}}}
\renewcommand\chaptername{}
\chapter[ग्रन्थकर्तृजीवनवृत्तम्]{ग्रन्थकर्तृजीवनवृत्तम्}
\markboth{ग्रन्थकर्तृजीवनवृत्तम्}{}
\fontsize{14}{21}\selectfont
\vspace{-4mm}
\raggedleft{लेखकौ –~वाचस्पतिमिश्रः, तुलसीदासपरौहा\nopagebreak\\
(सम्पादकः~– नित्यानन्दमिश्रः)}\nopagebreak\\
\vspace{4mm}
\begin{sloppypar}\hyphenrules{nohyphenation}\justifying\noindent\hspace{10mm} संस्कृत\-साहित्य\-जगति प्रथित\-कीर्तयो विलक्षण\-प्रतिभा\-वन्तः पदवाक्य\-प्रमाण\-पारावारीणाः पद्मविभूषण\-साहित्याकादमी\-राष्ट्रिय\-पुरस्कारादि\-पुरस्कृताः कविकुलरत्नानि मध्यप्रदेशस्य सतना\-मण्डल\-वर्तिनि मन्दाकिनी\-सलिल\-विमल\-सलिलासक्ते चित्रकूटे कृतनिवासा अहोरात्रं काव्य\-रचना\-व्याजेन सुरभारती\-समर्चने समर्पित\-जीवनास्तथा च 
विकलाङ्ग\-जनतोपासनायां सततं संलग्नाः शतावधानि\-कवयः स्वामि\-रामभद्राचार्याः केन न ज्ञायन्ते।\end{sloppypar}
\vspace{-2mm}
\begin{center}
वदनं प्रसादसदनं सदयं हृदयं सुधामुचो वाचः।\nopagebreak\\
करणं परोपकरणं येषां केषां न ते वन्द्याः॥\\
\end{center}
\vspace{-2mm}
\begin{sloppypar}\hyphenrules{nohyphenation}\justifying\noindent\hspace{10mm} इति यद्भर्तृहरिणा 
साकूतं समान्वभाणि तदेतदशेषं समञ्जसं प्रस्तुतग्रन्थ\-रत्नस्य प्रणेतृषु धर्मचक्रवर्तिषु महामहोपाध्यायेषु वाचस्पतिषु महाकविषु सर्वतन्त्र\-स्वतन्त्रेषु प्रस्थानत्रयी\-भाष्यकारेषु सर्वाम्नाय\-श्रीचित्रकूट\-तुलसी\-पीठाधीश्वरेषु श्रीमज्जगद्गुरु\-श्रीरामानन्दाचार्यास्पद\-स्वामि\-श्रीरामभद्राचार्यचरणेषु।\end{sloppypar}
\begin{sloppypar}\hyphenrules{nohyphenation}\justifying\noindent\hspace{10mm} \textcolor{red}{जन्म}~– स्वामि\-रामभद्राचार्याणां जनिरुत्तरप्रदेशे जौनपुर\-जनपदान्तर्वर्तिनि शाण्डिखुर्द\-नामके ग्रामे द्विसहस्राधिक\-षष्ठे वैक्रमेऽब्दे माघकृष्णैकादश्यां तदनुसारं पञ्चाशदुत्तरैकोन\-विंशतिशत ईसवीयाब्दे मकर\-सङ्क्रान्तौ (१४ जनवरी १९५०) वैवस्वत\-वासरे निशि निशीथाभिमुख्यां 
मकरराशिगते सवितरि वसिष्ठ\-गोत्रीय\-सरयूपारीण\-विप्रामले कुले विश्व\-विश्रुत\-ब्राह्मण\-कुलालङ्कारस्य श्रीराजदेव\-मिश्रस्य चतुर्थापत्य\-रूपेणाखण्ड\-सौभाग्य\-वत्याः श्रीमत्याः शची\-देव्या दक्षिण\-कुक्षितः समभवत्। एतेषां नामकरणं \textcolor{red}{गिरिधरलाल} इति कृत्वा पितरौ मोदमावहन्तौ मासद्वयावधिमेव यापितवन्तौ हा हन्त तदानीमेव 
दैवदुर्विपाकात् \textcolor{red}{रोहुआ}\-रुजा\-प्रकोपाद्बालकस्य बाह्यदृष्टिर्गता। वस्तुतस्त्वसारमेतं संसारं नावलोकयितु\-कामेनानेन नवजात\-शिशुनैव द्विमासेन भौतिके चक्षुषी निमीलिते इत्युत्प्रेक्षामहे। एतान् दृष्टि\-बाधितान् विज्ञाय महद्दुःखमनुभवन्तः पितामहाः श्रीसूर्यबलि\-मिश्र\-महोदयास्तान् श्रीमद्भगवद्गीतां वारं वारं श्रावितवन्तः। तेन पञ्चवर्षस्यावस्थायामेव जन्मान्तरीय\-प्रतिभा\-प्रागल्भ्य\-भगवद्भजन\-साधन\-धना विलक्षण\-प्रतिभाशीलाः प्रज्ञाचक्षुष्मन्तो गिरिधरनामानोऽशेषां गीतां कण्ठस्थीकृतवन्तः।\end{sloppypar}
\begin{sloppypar}\hyphenrules{nohyphenation}\justifying\noindent\hspace{10mm} \textcolor{red}{शिक्षार्जनम्}~– बाल्ये गिरिधर\-लालाभिधेयानां जगद्गुरु\-रामभद्राचार्याणां प्राथमिकी शिक्षा पितामह\-सन्निधावेव सम्पन्ना। एते जानकी\-जीवन\-लीला\-लालित्य\-लुब्ध\-धियो पितामह\-चरणेभ्यो गोस्वामि\-तुलसीदास\-प्रणीतं श्रीमद्रामचरितमानसं साङ्गोपाङ्गं सानुपूर्वीकं पङ्क्तिसङ्ख्यासहितं द्वाभ्यामेव मासाभ्यां कण्ठस्थीकृत्याष्टम एव वर्षे रामनवम्यामश्रावयन्। एतेषां विशिष्टां प्रज्ञां लोचं लोचं पितामहाः सूर्यबलि\-महोदयाः पितरौ च संस्कृतस्य पारम्परिक\-शिक्षां दापयितुं निकटवर्तिनं सुजानपुरस्थं श्रीगौरीशङ्कर\-संस्कृत\-महाविद्यालयमेतान् प्रैषयन्। कुशाग्र\-बुद्धि\-सम्पन्ना एकश्रुता\footnote{एकपाठिन एकसन्धिग्राहिणो वा।} गिरिधरमिश्रास्तत्र प्रारम्भिक\-व्याकरण\-शिक्षां तत्रत्य\-व्याकरण\-विभागाध्यक्षेभ्यः श्रीशीतला\-प्रसाद\-मिश्रेभ्यः सम्प्राप्तवन्तः श्लोक\-संस्कार\-रचनां च समधिगतवन्तः तत्रत्य\-साहित्य\-विभागाध्यक्षेभ्यः श्रीराम\-मनोरथ\-त्रिपाठि\-महाभागेभ्यः। श्रीगौरीशङ्कर\-संस्कृत\-महाविद्यालये नव्यव्याकरणं मुख्यविषयत्वेन स्वीकृत्य प्रथमातो मध्यमाश्रेणी\-पर्यन्तं सर्वाः परीक्षाः सविषेशाङ्कं सर्वप्राथम्येन प्रथम\-श्रेण्या समुत्तीर्योच्च\-शिक्षार्थं वाराणस्यां सम्पूर्णानन्द\-संस्कृत\-विश्वविद्यालयं प्रविष्टवन्तः। तत्कालीन\-व्याकरण\-विभागाध्यक्षाणां श्रीभूपेन्द्र\-पति\-त्रिपाठिनां शीतल\-स्नेह\-च्छायायां \textcolor{red}{वैयाकरण\-भूषण\-सारम्} एवं शाब्दिक\-शिरोमणि\-श्रीपण्डित\-कालिका\-प्रसाद\-शुक्ल\-महाभागानां श्रीचरण\-कमल\-सन्निधावन्यांश्च व्याकरण\-टीका\-ग्रन्थान् समधीत्य पुनश्चातिरिक्तसमये प्रतिदिनं सायं षड्ढोरावधिं\footnote{\textcolor{red}{कालाध्वनोरत्यन्त\-संयोगे} (पा॰सू॰~२.३.५) इत्यनेन द्वितीया।} समुपविश्य शब्द\-सागर\-मन्दर\-मति\-व्याकरण\-विभागाध्यक्ष\-चराणानामभिनव\-पाणिनीनां डॉ॰\-रामप्रसाद\-त्रिपाठिनां सन्निधौ भाष्यान्त\-व्याकरण\-ग्रन्थानामन्येषां नव्यव्याकरणस्य पद\-वाक्य\-प्रमाण\-ग्रन्थानां सपरिष्कारं निगूढ\-तत्त्व\-बोध\-पुरःसरं विशेषमध्ययनं विधिवत्कुर्वाणाः शास्त्रिकक्षामीसवीयाब्दे चतुःसप्तत्युत्तरैकोन\-विंशतिशते विश्वविद्यालये सर्वोत्तमाङ्कैः सह समुत्तीर्य प्रथमं स्थानं लब्धवन्तः। आचार्यकक्षायामधीयानाः सन्तो भारत\-सर्वकार\-द्वारा प्रायोजितास्वखिल\-भारतीय\-प्रतियोगितासु सर्वानपि प्रतियोगिनः प्रतिभया परिभाव्य वाद\-विवादान्त्याक्षरी\-समस्यापूर्ति\-व्याकरण\-साङ्ख्येति\-पञ्च\-प्रतियोगितासु प्रथमं स्थानं लब्धवन्तः। तत्र पुरस्कर्तुं स्वयं समागता तत्कालीना प्रधान\-मन्त्रिणी श्रीमतीन्दिरागान्धी पञ्चपुरस्कारैः सहोत्तरप्रदेशस्य कृते विशाल\-रजत\-पट्टिका\-रूपं चलवैजयन्ती\-पुरस्कारमपि ससाधुवादमदात्। ईसवीयाब्दे षट्सप्तत्युत्तरैकोन\-विंशतिशत आचार्य\-कक्षां विश्वविद्यालयस्य सर्वेष्वपि विभागेषु सर्वाधिकाङ्कैः सह समुत्तीर्य विशिष्टां कीर्तिमाभजद्भ्यः स्वामि\-वर्येभ्यो गिरिधर\-लालाभिधेयेभ्यः सप्तस्वर्ण\-पदकानि कुलाधिपति\-स्वर्ण\-पदकं चामिलन्। सम्पूर्णानन्द\-संस्कृत\-विश्वविद्यालयस्यैव पी॰एच्॰डी॰ (विद्यावारिधिः) इत्युपाधये \textcolor{red}{अध्यात्म\-रामायणेऽपाणिनीय\-प्रयोगाणां विमर्शः} इति विषये शोध\-प्रबन्धमपि लिखित\-वन्तोऽनुसन्धान\-विधया केवलैस्त्रयोदशभिरेव दिवसैः। तथा च \textcolor{red}{पाणिनीयाष्टाध्याय्याः प्रतिसूत्रं शाब्दबोधसमीक्षा} इति विषये द्विसहस्रपृष्ठात्मकं शोधप्रबन्धं प्रणीय तस्मादेव विश्वविद्यालयतो डी॰लिट्॰ (वाचस्पतिः) इत्युपाधिमप्यलभन्त।\end{sloppypar}
\begin{sloppypar}\hyphenrules{nohyphenation}\justifying\noindent\hspace{10mm} \textcolor{red}{विरक्तदीक्षोत्तरजीवनम्}~– १९८३ ईसवीयाब्दे कार्त्तिक\-शुक्ल\-पौर्णमास्यां श्रीरामानन्द\-वैष्णव\-सम्प्रदाये विरक्तवेषो गृहीतो गिरिधर\-लाल\-मिश्र\-चरणैः श्रीमदलर्क\-पुरी\-प्रयाग\-निवासिभ्यः फलाहारि\-सञ्ज्ञया प्रसिद्धेभ्यः श्रीश्रीरामचरण\-दास\-महाभागेभ्यः। यद्यपि मिश्रचरणैः श्रीराममन्त्रो विरक्त\-वैष्णव\-सम्प्रदाय\-दीक्षा च गृहीते जीवनस्याष्टम एव वर्षे श्रीमद्राम\-वल्लभा\-शरण\-महाराज\-चरणारविन्द\-प्रमुख\-कृपा\-पात्रेभ्यः श्रीमदीश्वर\-दास\-महाभागेभ्यो विरक्तवेष\-समकालमेव पञ्चसंस्कार\-विधिनाऽमीषां पूर्वाश्रमौप\-चारिकताऽपीतिहासपृष्ठं समधिशिश्रिये। अतोऽच्युत\-गोत्राणामेषां महात्मनां \textcolor{red}{रामभद्रदासः} इति नाम ख्यातिमगमत्। रामभद्रदासाः समष्टि\-मङ्गलाय १९८७ ईसवीये तुलसीजयन्त्यां मन्दाकिनी\-विमल\-सलिलासक्त आमोदवने श्रीतुलसीपीठस्य प्रामाणिकीं प्रतिष्ठां विधाय श्रीचित्रकूट\-तुलसी\-पीठाधीश्वर\-पदालङ्कृता जाताः। नारदभक्तिसूत्र ईशाद्येकादशोपनिषत्सु श्रीमद्भगवद्गीतायां ब्रह्मसूत्रे च श्रीराघवकृपाभाष्यं विलिख्य प्रस्थानत्रयी\-भाष्य\-कारत्वेन प्रथितयशसः सारस्वत्या प्रतिभया योग्यतया स्वामिवर्यान् श्रीरामानन्द\-सम्प्रदायस्य जगद्गुरुपदे काशीविद्वत्परिषत्सम्मत्याऽखाड़ा\-परिषन्महान्तो मनीषिणो गण्यमान्या विद्वांसो जगद्गुरवश्च समवेतघोषणेन ससम्मानं प्रतिष्ठापितवन्तोऽभि\-षेचितवन्तश्च। एवं हि स्वामिवर्याः श्रीचित्रकूट\-तुलसी\-पीठाधीश्वर\-प्रस्थानत्रयी\-भाष्यकार\-जगद्गुरु\-रामानन्दाचार्य\-स्वामि\-रामभद्राचार्या इत्याख्येन भुवि लब्धयशोराशयः सततमेव विकलाङ्ग\-जनतोपासनायां समष्टि\-हिताय राष्ट्रसेवायै च समर्पितजीवना राजमाना जाताः।\end{sloppypar}
\begin{sloppypar}\hyphenrules{nohyphenation}\justifying\noindent\hspace{10mm} \textcolor{red}{विशेषः}~– निगूढानां शास्त्रविषयाणामवगमनं नव\-नवोन्मेष\-शालिनी प्रतिभा\-कल्पना\-कमनीयता तर्ककर्कश\-मस्तिष्कं प्रौढ\-पाण्डित्यं गुरुजने विनम्रता नैष्ठिकं ब्रह्मचर्यं श्रीरामोपासना\-रुचिश्चामीषां देशिकवर्याणां निसर्गसिद्धा विशेषता। काव्य\-रचना\-पाटवं त्वमीषां सार्धत्रिवर्षावस्थायामेव सुस्पष्टमुद्भूतम्। राष्ट्रभाषा\-संस्कृत\-भाषयोरप्याशु\-कवित्वं सहजसिद्धम्। एभिः श्रीचित्रकूटे नव पयोव्रतान्यापि विशालानि षण्मासपर्यन्तानि विहितानि। शरीरस्य वसिष्ठ\-गोत्रीयत्वादिमे शिशु\-रूप\-राघवमेव समुपासते लालयन्ति च वात्सल्य\-भावनया। देवतुल्यसमाराध्या अस्माकं सद्गुरवः जगद्गुरु\-स्वामि\-रामभद्राचार्य\-महाराजाः शाश्वतं भासन्तामिह लोके यावच्चन्द्र\-दिवाकराविति मङ्गलं कामयमानास्तेषां चरणकमलयोरर्पयन्तश्च प्रणतीः कथयन्त इदं विरमामः~–\end{sloppypar}
\vspace{-2mm}
\begin{center}
रामभद्रो हि जानाति रामभद्रसरस्वतीम्।\nopagebreak\\
रामभद्रो हि जानाति रामभद्रसरस्वतीम्॥\nopagebreak\\
\end{center}
\begin{sloppypar}\hyphenrules{nohyphenation}\justifying\noindent\hspace{10mm} \textcolor{red}{ग्रन्थसूची}~– जगद्गुरु\-रामभद्राचार्य\-प्रणीताः प्रमुख\-ग्रन्था अधोलिखिताः सन्ति।\end{sloppypar}
\begin{itemize}
\item पद्यकृतयः
	\begin{itemize}
	\item महाकाव्यानि
	\begin{itemize}
		\item अरुन्धती (१९९४)। हिन्द्याम्।
		\item श्रीभार्गवराघवीयम् (२००२)। संस्कृते। हिन्द्यनुवादसहितम्।
		\item अष्टावक्र (२०१०)। हिन्द्याम्।
		\item गीतरामायणम् (२०११)। संस्कृते। हिन्द्यनुवादसहितम्।
	\end{itemize}
	\item खण्डकाव्यानि
	\begin{itemize}
		\item काका विदुर (१९८०)। हिन्द्याम्।
		\item माँ शबरी (१९८२)। हिन्द्याम्।
		\item आजादचन्द्रशेखरचरितम् (१९९६)। संस्कृते। हिन्द्यनुवादसहितम्।
		\item लघुरघुवरम् (२००१)। संस्कृते। हिन्दीपद्यगद्यानुवादसहितम्।
		\item श्रीसरयूलहरी (२००१)। संस्कृते। हिन्द्यनुवादसहिता।
		\item भृङ्गदूतम् (२००४)। संस्कृते। हिन्द्यनुवादसहितम्।
		\item अवध कै अँजोरिया (२०११)। अवध्याम्। 
		\item श्रीसीतासुधानिधिः (२०११)। संस्कृते। हिन्द्यनुवादसहितः।
	\end{itemize}
	\item पत्रकाव्यम्
	\begin{itemize}
		\item कुब्जापत्रम् (२००३)। संस्कृते। हिन्द्यनुवादसहितम्।
	\end{itemize}
	\item गीतकाव्यानि
	\begin{itemize}
		\item राघवगीतगुञ्जन (१९९१)। हिन्द्याम्।
		\item भक्तिगीतसुधा (१९९३)। हिन्द्याम्।
	\end{itemize}
	\item रीतिकाव्यम्
	\begin{itemize}
		\item श्रीसीतारामकेलिकौमुदी (२००८)। ब्रजभाषायाम्। हिन्द्यनुवादसहिता।
	\end{itemize}
	\item शतककाव्यानि
	\begin{itemize}
		\item आर्याशतकम् (१९९६)। संस्कृते। हिन्द्यनुवादसहितम्।
		\item श्रीगणपतिशतकम् (१९९६)। संस्कृते। हिन्द्यनुवादसहितम्।
		\item श्रीचण्डीशतकम् (१९९६)। संस्कृते। हिन्द्यनुवादसहितम्।
		\item श्रीराघवेन्द्रशतकम् (१९९६)। संस्कृते। हिन्द्यनुवादसहितम्।
		\item श्रीरामभक्तिसर्वस्वम् (१९९७)। संस्कृते। हिन्द्यनुवादसहितम्।
		\item श्रीराघवचरणचिह्नशतकम् (२००१)। संस्कृते। हिन्द्यनुवादसहितम्।
		\item श्रीजानकीचरणचिह्नशतकम् (२००१)। संस्कृते। हिन्द्यनुवादसहितम्।
		\item मन्मथारिशतकम् (२००७)। संस्कृते। हिन्द्यनुवादसहितम्।
		\item श्रीरघुनाथशतकम् (२०११)। संस्कृते। हिन्द्यनुवादसहितम्।
	\end{itemize}
	\item स्तोत्रकाव्यानि
	\begin{itemize}
		\item श्रीहनुमच्चत्वारिंशिका (१९८३)। संस्कृते। हिन्द्यनुवादसहिता।
		\item श्लोकमौक्तिकम् (१९८३)। संस्कृते। हिन्द्यनुवादसहितम्।
		\item मुकुन्दस्मरणम् (१९९६)। संस्कृते। हिन्द्यनुवादसहितम्।
		\item श्रीजानकीकृपाकटाक्षस्तोत्रम् (१९९६)। संस्कृते।हिन्द्यनुवादसहितम्।
		\item भक्तिसारसर्वस्वम् (१९९७)। संस्कृते। हिन्द्यनुवादसहितम्।
		\item श्रीगङ्गामहिम्नःस्तोत्रम् (१९९८)। संस्कृते। हिन्द्यनुवादसहितम्।
		\item नमोराघवायाष्टकम् (२००१)। संस्कृते। हिन्द्यनुवादसहितम्।
		\item श्रीचित्रकूटविहार्यष्टकम् (२००१)। संस्कृते। हिन्द्यनुवादसहितम्।
		\item श्रीरामवल्लभास्तोत्रम् (२००१)। संस्कृते। हिन्द्यनुवादसहितम्।
		\item श्रीराघवभावदर्शनम् (२००२)। संस्कृते। हिन्द्यनुवादसहितम्।
		\item चरणपीडाहराष्टकम् (२००८)। संस्कृते। हिन्द्यनुवादसहितम्।
		\item सर्वरोगहराष्टकम् (२०१०)। संस्कृते। हिन्द्यनुवादसहितम्।
	\end{itemize}
	\item सुप्रभातकाव्यम्
	\begin{itemize}
		\item श्रीसीतारामसुप्रभातम् (२००९)। संस्कृते। हिन्द्यनुवादसहितम्।
	\end{itemize}
	\item वृत्तिकाव्यम्
	\begin{itemize}
		\item अष्टाध्याय्याः प्रतिसूत्रं शाब्दबोधसमीक्षा (१९९७)। संस्कृते।
	\end{itemize}
	\end{itemize}
\item नाटके
	\begin{itemize}
	\item उत्साह (१९९६)। हिन्द्याम्।
	\item श्रीराघवाभ्युदयम् (१९९६)। संस्कृते। हिन्द्यनुवादसहितम्।
	\end{itemize}
\item गद्यकृतयः
	\begin{itemize}
	\item प्रस्थानत्रय्यां श्रीराघवकृपाभाष्याणि
		\begin{itemize}
		\item श्रीब्रह्मसूत्रेषु श्रीराघवकृपाभाष्यम् (१९९८)। संस्कृते हिन्द्यां च।
		\item श्रीमद्भगवद्गीतासु श्रीराघवकृपाभाष्यम् (१९९८)। संस्कृते हिन्द्यां च।
		\item कठोपनिषदि श्रीराघवकृपाभाष्यम् (१९९८)। संस्कृते हिन्द्यां च।
		\item केनोपनिषदि श्रीराघवकृपाभाष्यम् (१९९८)। संस्कृते हिन्द्यां च।
		\item माण्डूक्योपनिषदि श्रीराघवकृपाभाष्यम् (१९९८)। संस्कृते हिन्द्यां च।
		\item ईशावास्योपनिषदि श्रीराघवकृपाभाष्यम् (१९९८)। संस्कृते हिन्द्यां च।
		\item प्रश्नोपनिषदि श्रीराघवकृपाभाष्यम् (१९९८)। संस्कृते हिन्द्यां च।
		\item तैत्तिरीयोपनिषदि श्रीराघवकृपाभाष्यम् (१९९८)। संस्कृते हिन्द्यां च।
		\item ऐतरेयोपनिषदि श्रीराघवकृपाभाष्यम् (१९९८)। संस्कृते हिन्द्यां च।
		\item श्वेताश्वतरोपनिषदि श्रीराघवकृपाभाष्यम् (१९९८)। संस्कृते हिन्द्यां च।
		\item छान्दोग्योपनिषदि श्रीराघवकृपाभाष्यम् (१९९८)। संस्कृते हिन्द्यां च।
		\item बृहदारण्यकोपनिषदि श्रीराघवकृपाभाष्यम् (१९९८)। संस्कृते हिन्द्यां च।
		\item मुण्डकोपनिषदि श्रीराघवकृपाभाष्यम् (१९९८)। संस्कृते हिन्द्यां च।
		\end{itemize}
	\item अन्ये श्रीराघवकृपाभाष्ये
		\begin{itemize}
		\item श्रीनारदभक्तिसूत्रेषु श्रीराघवकृपाभाष्यम् (१९९१)। संस्कृते हिन्द्यां च।
		\item श्रीरामस्तवराजस्तोत्रे श्रीराघवकृपाभाष्यम् (२०००)। संस्कृते हिन्द्यां च।
		\end{itemize}
	\item हिन्दीभाष्याणि
		\begin{itemize}
		\item श्रीहनुमानचालीसायां महावीरी टीका (१९८३)। हिन्द्याम्।
		\item श्रीरामचरितमानसे भावार्थबोधिनी टीका (२००५)। हिन्द्याम्।
		\item श्रीभक्तमाले मूलार्थबोधिनी टीका (२०१४)। हिन्द्याम्।
		\end{itemize}
	\item विमर्शौ
		\begin{itemize}
		\item अध्यात्मरामायणेऽपाणिनीयप्रयोगानां विमर्शः (१९८१)। संस्कृते।
		\item श्रीरासपञ्चाध्यायीविमर्शः (२००७)। हिन्द्याम्।
		\end{itemize}
	\item प्रवचनसङ्ग्रहाः
		\begin{itemize}
		\item मानस में तापस प्रसंग (१९८२)। हिन्द्याम्।
		\item सुग्रीव का अघ और विभीषण की करतूति (१९८५)। हिन्द्याम्।
		\item श्रीगीतातात्पर्य (१९८५)। हिन्द्याम्।
		\item सनातनधर्म की विग्रहस्वरूप गोमाता (१९८८)। हिन्द्याम्।
		\item श्रीतुलसीसाहित्य में कृष्णकथा (१९८८)। हिन्द्याम्।
		\item मानस में सुमित्रा (१९८९)। हिन्द्याम्।
		\item सीता निर्वासन नहीं (१९९०)। हिन्द्याम्।
		\item प्रभु करि कृपा पाँवरी दीन्ही (१९९२)। हिन्द्याम्।
		\item परम बड़भागी जटायु (१९९३)। हिन्द्याम्।
		\item श्रीसीताराम विवाह दर्शन (२००१)। हिन्द्याम्।
		\item तुम पावक मँह करहु निवासा (२००४)। हिन्द्याम्।
		\item अहल्योद्धार (२००६)। हिन्द्याम्।
		\item हर ते भे हनुमान (२००८)। हिन्द्याम्।
		\item सत्य रामप्रेमी श्रीदशरथ (२००९)। हिन्द्याम्।
		\item वेणुगीत (२०११)। हिन्द्याम्।
		\end{itemize}
	\end{itemize}
\end{itemize}


% Nityanand Misra: LaTeX code to typeset a book in Sanskrit
% Copyright (C) 2016 Nityanand Misra
%
% This program is free software: you can redistribute it and/or modify it under
% the terms of the GNU General Public License as published by the Free Software
% Foundation, either version 3 of the License, or (at your option) any later
% version.
%
% This program is distributed in the hope that it will be useful, but WITHOUT
% ANY WARRANTY; without even the implied warranty of  MERCHANTABILITY or FITNESS
% FOR A PARTICULAR PURPOSE. See the GNU General Public License for more details.
%
% You should have received a copy of the GNU General Public License along with
% this program.  If not, see <http://www.gnu.org/licenses/>.

\chapter[आचार्यचरणानां विरुदावली]{आचार्यचरणानां विरुदावली}
\markboth{आचार्यचरणानां विरुदावली}{}

\fontsize{14}{21}\selectfont
\begin{center}\hyphenrules{nohyphenation} \textcolor{blue}{॥ श्रीसीताराम\-पदपद्म\-मकरन्द\-मधुव्रत\-श्रीसम्प्रदाय\-प्रवर्तक\-सकल\-शास्त्रार्थ\-महार्णव\-मन्दरमति\-श्रीमदाद्य\-जगद्गुरु\-रामानन्दाचार्य\-चरणारविन्द\-चञ्चरीकाः समस्त\-वैष्णवालङ्कार\-भूता आर्षवाङ्मय\-निगमागम\-पुराणेतिहास\-सन्निहित\-गम्भीर\-तत्त्वान्वेषण\-तत्पराः पदवाक्य\-प्रमाण\-पारावारीणाः साङ्ख्ययोग\-न्यायवैशेषिक\-पूर्वमीमांसा\-वेदान्त\-नारद\-शाण्डिल्य\-भक्तिसूत्र\-गीता\-वाल्मीकीय\-रामायण\-भागवतादि\-सिद्धान्तबोध\-पुरःसर\-समधिकृताशेष\-तुलसीदास\-साहित्य\-सौहित्य\-स्वाध्याय\-प्रवचन\-व्याख्यान\-परमप्रवीणाः सनातन\-धर्म\-संरक्षण\-धुरीणाश्चतुराश्रम\-चातुर्वर्ण्य\-मर्यादा\-संरक्षण\-विचक्षणा अनाद्यविच्छिन्न\-सद्गुरु\-परम्परा\-प्राप्त\-श्रीमत्सीताराम\-भक्तिभागीरथी\-विगाहन\-विमलीकृत\-मानसाः श्रीमद्रामचरित्र\-मानस\-राजमरालाः सततं शिशुराघव\-लालनतत्पराः समस्त\-प्राच्य\-प्रतीच्य\-विद्या\-विनोदित\-विपश्चितो राष्ट्रभाषा\-गीर्वाणगिरा\-महाकवयो विद्वन्मूर्धन्याः श्रीमद्राम\-प्रेमसाधना\-धनधन्याः श्रोत्रिय\-ब्रह्मनिष्ठाः प्रस्थानत्रयी\-भाष्यकारा महामहोपाध्याया वाचस्पतयो जगद्गुरु\-रामभद्राचार्य\-विकलाङ्ग\-विश्वविद्यालयस्य जीवन\-पर्यन्त\-कुलाधिपतयः श्रीचित्रकूटस्थ\-मन्दाकिनी\-विमलपुलिन\-निवासिनः श्रीतुलसी\-पीठाधीश्वरा धर्मचक्रवर्तिनः श्रीमज्जगद्गुरु\-रामानन्दाचार्या अनन्तश्री\-समलङ्कृत\-स्वामिरामभद्राचार्य\-महाराजा विजयन्तेतराम् ॥}\end{center}


% Nityanand Misra: LaTeX code to typeset a book in Sanskrit
% Copyright (C) 2016 Nityanand Misra
%
% This program is free software: you can redistribute it and/or modify it under
% the terms of the GNU General Public License as published by the Free Software
% Foundation, either version 3 of the License, or (at your option) any later
% version.
%
% This program is distributed in the hope that it will be useful, but WITHOUT
% ANY WARRANTY; without even the implied warranty of  MERCHANTABILITY or FITNESS
% FOR A PARTICULAR PURPOSE. See the GNU General Public License for more details.
%
% You should have received a copy of the GNU General Public License along with
% this program.  If not, see <http://www.gnu.org/licenses/>.

\renewcommand\chaptername{}
\chapter[प्रशस्तिवाचः]{प्रशस्तिवाचः}
\markboth{प्रशस्तिवाचः}{}
\fontsize{18}{27}\selectfont
\centering\textcolor{blue}{\underline{नवीना संशोधनदिक्}}\nopagebreak\\
\vspace{4mm}
\fontsize{14}{21}\selectfont
\raggedleft{–~प्राचार्या माधवशर्माणो देशपाण्डे इत्युपाह्वाः}\nopagebreak\\
\vspace{4mm}
\fontsize{14}{21}\selectfont
\begin{sloppypar}\hyphenrules{nohyphenation}\justifying\noindent\hspace{10mm} १९८१ ईसवी\-संवत्सरे सम्पूर्णानन्द\-संस्कृत\-विश्वविद्यालये \textcolor{red}{विद्यावारिधि}\-उपाधि\-प्राप्तये महापण्डितै रामभद्राचार्यैर्लिखितः प्रबन्धः \textcolor{blue}{अध्यात्मरामायणेऽपाणिनीय\-प्रयोगाणां विमर्शः} इति नाम्ना सम्प्रति तच्छिष्येण पण्डित\-नित्यानन्द\-मिश्रेण संस्कृत्य प्राकाश्यं नीयत इति ज्ञात्वा नितरां मोदते मे चेतः। व्यास\-वाल्मीकि\-प्रभृतिभिर्मुनिभिर्विरचितेषु महाभारत\-रामायणादि\-महाकाव्येषु सर्वत्रापाणिनीयाः प्रयोगा दृश्यन्त इति नाविदितं विदुषाम्। तेषामुपपत्तिः कथं दर्शयितव्येति चिन्तां कुर्वद्भिः प्राचीनैरेते प्रयोगा यद्यप्यपाणिनीयास्तथाऽपि तेषामार्षत्वात्तेषु दोषो नारोपणीयः किन्तु तेषां यद्यप्यार्षत्वात्साधुत्वं तथाऽपि \textcolor{red}{न~देवचरितं चरेत्} इतिवत् \textcolor{red}{न~ऋषिचरितं चरेत्} इति नियमेनार्ष\-प्रयोगाणां लौकिकैर्नानुकरणं कर्तव्यमित्यभि\-प्रायो दर्शितः। अन्ये तु~–\end{sloppypar}
\centering\textcolor{red}{यान्युज्जहार माहेन्द्राद्व्यासो व्याकरणार्णवात्।\nopagebreak\\
पदरत्नानि किं तानि सन्ति पाणिनिगोष्पदे॥}\nopagebreak\\
\begin{sloppypar}\hyphenrules{nohyphenation}\justifying\noindent इत्यूचुः। अनया दिशा पाणिनीय\-व्याकरणादपि प्राचीनतराणि माहेन्द्रादि\-व्याकरणान्यासन् यान्यनुसृत्य व्यासादिभिः कृताः शब्दप्रयोगा अपाणिनीया अपि साधव एव तेषां च साधुत्वं पाणिनिं प्रमाणीकृत्य न परीक्षितव्यमिति केषाञ्चिन्मतम्। अध्यात्म\-रामायण\-स्थानीयानपाणिनीय\-प्रयोगानधिकृत्य रामभद्राचार्यैर्लिखितोऽ\-यमधुना प्रकाश्यमानः प्रबन्धः स्वीयं वैशिष्ट्यमावहति। अस्य ग्रन्थस्यानया दिशा परीक्षणं न केनापि कृतपूर्वम्। \textcolor{red}{काव्येषु कोमलधियो वयमेव नान्ये तर्केषु कर्कशधियो वयमेव नान्ये} इति पण्डितराज\-जगन्नाथ\-सरणिमनुकुर्वद्भी रामभद्राचार्यैः काव्य\-विषये शास्त्र\-विषये च बहु पराक्रान्तम्। अस्मिन् प्रबन्धे ते सर्वत्र प्राचीनां निरुक्त\-व्याकरणादि\-शास्त्र\-परम्परामनुसृत्य शब्द\-व्युत्पत्त्यादिप्रदर्शनं कुर्वन्ति यथा तैः स्वीय\-प्रबन्ध\-प्रस्तावनायाम् \textcolor{red}{अथर्व}\-शब्दः \textcolor{red}{अथ}\-शब्देन \textcolor{red}{अर्व}\-शब्दं संयोज्य व्युत्पादितः। नवीने विषये प्राचीना शास्त्रसरणिः कथमुप\-युज्येतेत्यस्यायं प्रबन्धः परमं निदर्शनम्। आपाततोऽपाणिनीयत्वेन प्रतीयमानानां प्रयोगाणामेते रीत्यन्तरेण पाणिनीयत्वं साधयन्ति तदेतेषां सर्वतन्त्र\-स्वतन्त्रत्वं प्रमाणयति। एतद्ग्रन्थरत्नं प्रणीय रामभद्राचार्यैर्व्याकरण\-शास्त्रस्य नवीना संशोधनदिगुद्घाटिता। अतस्ते सर्वथा धन्यवादानर्हन्ति। इति शम्।\end{sloppypar}
\vspace{4mm}
\fontsize{14}{19}\selectfont
\raggedleft{माधवशर्मा देशपाण्डे इत्युपाह्वः\nopagebreak\\
३१/०८/२०१४}\nopagebreak\\
\fontsize{12}{16}\selectfont
\raggedleft{प्राचार्यः, मिशिगन्‌‌विश्वविद्यालयः\nopagebreak\\
ऍन् आर्बर्, मिशिगन्, यू.एस्.ए.}\\
\vspace{8mm}
\fontsize{18}{27}\selectfont
\centering\textcolor{blue}{\underline{अभिनन्दनवाचः}}\nopagebreak\\
\vspace{4mm}
\fontsize{14}{21}\selectfont
\raggedleft{–~देवर्षि\-कलानाथ\-शास्त्रिणः}\nopagebreak\\
\vspace{4mm}
\fontsize{14}{21}\selectfont
\begin{sloppypar}\hyphenrules{nohyphenation}\justifying\noindent\hspace{10mm} सकल\-शास्त्र\-पारगैः संस्कृत\-वाङ्मय\-वारिधि\-मन्दरायमाणैः पद\-वाक्य\-प्रमाण\-पताका\-वाहकैः सरस्वती\-पुत्रैस्तुलसी\-पीठाधिपतिभिर्जगद्गुरु\-रामानन्दाचार्य\-महालङ्कार\-भूतैर्महा\-महोपाध्यायैः श्री१०८स्वामि\-श्रीरामभद्राचार्यैः पूर्वाश्रमे विलिखितं शोधप्रबन्धं वीक्ष्य सुमहान्तं प्रमोदमन्वभवम्। यद्यपि शोध\-प्रबन्धस्यास्य प्रधान\-विषयत्वेन तैरध्यात्म\-रामायणीयास्ते प्रयोगा गहन\-विमर्श\-दृशा परीक्षिता ये \textcolor{red}{अपाणिनीयाः} इति व्यपदिश्यन्ते किन्त्वेतेन विमर्श\-व्याजेन तैः पाणिनीयस्य बहवः सिद्धान्ताः सूक्ष्मदृशा विवेचितास्तेषु तेषु च प्रसङ्गेष्वनेके संस्कृत\-ग्रन्थाः श्रीतुलसीदास\-ग्रन्था अन्ये च शास्त्र\-विधयोऽलौकिक\-विश्लेषण\-पद्धत्याऽनुशीलिता इति व्यापकस्य शोध\-प्रबन्धस्य फलकं संवृत्तम्।\end{sloppypar}
\begin{sloppypar}\hyphenrules{nohyphenation}\justifying\noindent\hspace{10mm} बहुमूल्यस्यास्य विमर्श\-ग्रन्थस्य बहोः कालादनन्तरं ससम्पादनमुपस्थापनं कुर्वाणो बह्वायामि\-प्रतिभा\-धनः श्रीनित्यानन्द\-मिश्रोऽस्माकं शतशः साधुवाचां पात्रमिति श्रीरामभद्राचार्य\-चरणान् सप्रश्रयं प्रणमन् सभाजयामि श्रीनित्यानन्द\-मिश्रं धन्यवादैः।\end{sloppypar}
\fontsize{14}{21}\selectfont\centering\textcolor{blue}{अध्यात्मरामायणवाक्प्रयोगानाधाररूपेण सकृद्गृहीत्वा।\nopagebreak\\
यो व्यापकं पाणिनिशब्दशास्त्रं ममन्थ साधुत्वपरीक्षणाय॥\\
स्थले स्थले प्रातिभदृष्टिशोधैः प्रातिष्ठिपन्नूतनशास्त्रयुक्तीः।\nopagebreak\\
गभीरमन्वेषणकृत्यमित्थं शोधप्रबन्धे निबबन्ध सम्यक्॥\\
शोधोपाधिं गिरिधरवरो मिश्रलक्ष्मा गृहीत्वा\nopagebreak\\
सन्न्यस्तोऽभूदथ च तुलसीपीठकं चित्रकूटे।\\
पीठाधीशो विमलधिषणो रामभद्रेति नाम्ना\nopagebreak\\
सम्भूष्यात्र प्रथयति नवं विश्वविद्यालयं तम्॥\\
शोधग्रन्थं तमिह गहनप्रौढशैलीनिबद्धं\nopagebreak\\
नित्यानन्दः कुशलधिषणः साधु सम्पाद्य धत्ते।\\
सम्प्रेक्षायै सकलविदुषां सन्निधौ देववाणी-\nopagebreak\\
ग्रन्थस्यास्य प्रमुदितहृदा स्वागतं व्याहरामः॥}\\
\vspace{4mm}
\fontsize{14}{19}\selectfont
\raggedleft{देवर्षि\-कलानाथ\-शास्त्री\nopagebreak\\
१४/१०/२०१४}\nopagebreak\\
\fontsize{12}{16}\selectfont
\raggedleft{अध्यक्षचरः, राजस्थान\-संस्कृताकादमी\nopagebreak\\
निदेशकः, संस्कृतशिक्षाभाषाविभागः, राजस्थानसर्वकारः\nopagebreak\\
प्रधानसम्पादकः, भारती (संस्कृतमासिकपत्रिका)\nopagebreak\\
सदस्यः, संस्कृतायोगः, भारतसर्वकारः\nopagebreak\\
अध्यक्षः, आधुनिकसंस्कृतपीठम्, ज॰रा॰रा॰संस्कृतविश्वविद्यालयः\nopagebreak\\
जयपुरम्, भारतम्}\\
\vspace{8mm}
\fontsize{18}{27}\selectfont
\centering\textcolor{blue}{\underline{प्रशस्तिपद्यानि}}\nopagebreak\\
\vspace{4mm}
\fontsize{14}{21}\selectfont
\raggedleft{–~शतावधानिनो रा.गणेशाः}\nopagebreak\\
\vspace{4mm}
\fontsize{14}{21}\selectfont
\begin{sloppypar}\hyphenrules{nohyphenation}\justifying\noindent\hspace{10mm} इयमिह मदुपज्ञा काचिल्पीयसी 
तथाऽपि भक्तिमकरन्दभरनम्रा सुवर्णमयी च प्रशस्तिपद्यसुमनोमालिका। अनया प्रज्ञाविलोकनवतां तत्र भवतां महतां सतां श्रीश्रीरामभद्राचार्यमस्करिणां मत्परतया वरिवस्या भवेदित्याशासे। चित्रावहा खलु श्रीचरणानां सारस्वतसाधना शास्त्रे काव्ये च। अतो हि चित्रकवितैव विहिताऽत्र मया। ईदृशानां वैरल्येन च साम्प्रतिके युगे प्रायेण कामपि प्रतिनवां सुषमामर्हतीयं मम वागुपक्रमप्रक्रियेति मन्ये। अन्यच्च कवितल्लजा 
नैके ध्वन्यध्वन्यध्वनीनाः खलु कुर्युरन्यया चानन्यया रसमयरीत्या पद्यानीति मत्वा मदीयमिदं तार्तीयकं वर्त्म समादृतम्।\end{sloppypar}
\begin{sloppypar}\hyphenrules{nohyphenation}\justifying\noindent\hspace{10mm} गोमूत्रिकाबन्धः—\end{sloppypar}
\fontsize{14}{21}\selectfont\centering\textcolor{blue}{रामभद्रयतिर्जीयाच्चलुकीकृतवाग्विधिः।\nopagebreak\\
धीमद्भद्रकृतिर्भूयाद्वालुकीकृतवारिधिः॥}\\
\vspace{4mm}
\tikzstyle{line} = [draw, -stealth, thick]
\begin{tikzpicture}[every node/.append style={circle, text height=2ex, text depth=.25ex, text width=1.5em, text centered, draw=black!80, inner sep=0pt},->,>=stealth',auto,thick]
\node [minimum width=5mm] (01) {\strut रा};
\node [minimum width=5mm, below right=1em and 0.25em of 01] (02) {\strut म};
\node [minimum width=5mm, right=1em of 01] (03) {\strut भ};
\node [minimum width=5mm, below right=1em and 0.25em of 03] (04) {\strut द्र};
\node [minimum width=5mm, right=1em of 03] (05) {\strut य};
\node [minimum width=5mm, below right=1em and 0.25em of 05] (06) {\strut ति};
\node [minimum width=5mm, right=1em of 05] (07) {\strut र्जी};
\node [minimum width=5mm, below right=1em and 0.25em of 07] (08) {\strut या};
\node [minimum width=5mm, right=1em of 07] (09) {\strut च्च};
\node [minimum width=5mm, below right=1em and 0.25em of 09] (10) {\strut लु};
\node [minimum width=5mm, right=1em of 09] (11) {\strut की};
\node [minimum width=5mm, below right=1em and 0.25em of 11] (12) {\strut कृ};
\node [minimum width=5mm, right=1em of 11] (13) {\strut त};
\node [minimum width=5mm, below right=1em and 0.25em of 13] (14) {\strut वा};
\node [minimum width=5mm, right=1em of 13] (15) {\strut ग्वि};
\node [minimum width=5mm, below right=1em and 0.25em of 15] (16) {\strut धिः};
\node [minimum width=5mm, below=2.75em of 01] (17) {\strut धी};
\node [minimum width=5mm, below=2.75em of 03] (19) {\strut द्भ};
\node [minimum width=5mm, below=2.75em of 05] (21) {\strut कृ};
\node [minimum width=5mm, below=2.75em of 07] (23) {\strut र्भू};
\node [minimum width=5mm, below=2.75em of 09] (25) {\strut द्वा};
\node [minimum width=5mm, below=2.75em of 11] (27) {\strut की};
\node [minimum width=5mm, below=2.75em of 13] (29) {\strut त};
\node [minimum width=5mm, below=2.75em of 15] (31) {\strut रि};
\path [color=red, line] (01) -- (02);
\path [color=red, line] (02) -- (03);
\path [color=red, line] (03) -- (04);
\path [color=red, line] (04) -- (05);
\path [color=red, line] (05) -- (06);
\path [color=red, line] (06) -- (07);
\path [color=red, line] (07) -- (08);
\path [color=red, line] (08) -- (09);
\path [color=red, line] (09) -- (10);
\path [color=red, line] (10) -- (11);
\path [color=red, line] (11) -- (12);
\path [color=red, line] (12) -- (13);
\path [color=red, line] (13) -- (14);
\path [color=red, line] (14) -- (15);
\path [color=red, line] (15) -- (16);
\path [color=blue, line] (17) -- (02);
\path [color=blue, line] (02) -- (19);
\path [color=blue, line] (19) -- (04);
\path [color=blue, line] (04) -- (21);
\path [color=blue, line] (21) -- (06);
\path [color=blue, line] (06) -- (23);
\path [color=blue, line] (23) -- (08);
\path [color=blue, line] (08) -- (25);
\path [color=blue, line] (25) -- (10);
\path [color=blue, line] (10) -- (27);
\path [color=blue, line] (27) -- (12);
\path [color=blue, line] (12) -- (29);
\path [color=blue, line] (29) -- (14);
\path [color=blue, line] (14) -- (31);
\path [color=blue, line] (31) -- (16);
\end{tikzpicture}
\vspace{4mm}
\fontsize{14}{21}\selectfont
\begin{sloppypar}\hyphenrules{nohyphenation}\justifying\noindent\hspace{10mm} गतप्रत्यागतम्—\end{sloppypar}
\fontsize{14}{21}\selectfont\centering\textcolor{blue}{जय हे यजने ध्याने भजनेऽजभणेक्षणे।\nopagebreak\\
महिमन् हिमहस्तोह स्तवनावस्त शस्त्यश॥}\\
\vspace{4mm}
\tikzstyle{line} = [draw, -stealth, thick]
\begin{center}
\begin{tikzpicture}[every node/.append style={circle, text height=2ex, text depth=.25ex, text width=1.5em, text centered, draw=black!80, inner sep=0pt},->,>=stealth',auto,thick]
\node [minimum width=5mm] (01) {\strut ज};
\node [minimum width=5mm, right=0.8em of 01] (02) {\strut य};
\node [minimum width=5mm, right=0.8em of 02] (03) {\strut हे};
\node [minimum width=5mm, right=0.8em of 03] (04) {\strut ने};
\node [minimum width=5mm, right=0.8em of 04] (05) {\strut ध्या};
\node [minimum width=5mm, right=0.8em of 05] (06) {\strut भ};
\node [minimum width=5mm, right=0.8em of 06] (07) {\strut ज};
\node [minimum width=5mm, right=0.8em of 07] (08) {\strut नेऽ};
\node [minimum width=5mm, right=0.8em of 08] (09) {\strut णे};
\node [minimum width=5mm, right=0.8em of 09] (10) {\strut क्ष};
\node [minimum width=5mm, below=2em of 01] (11) {\strut म};
\node [minimum width=5mm, right=0.8em of 11] (12) {\strut हि};
\node [minimum width=5mm, right=0.8em of 12] (13) {\strut मन्};
\node [minimum width=5mm, right=0.8em of 13] (14) {\strut ह};
\node [minimum width=5mm, right=0.8em of 14] (15) {\strut स्तो};
\node [minimum width=5mm, right=0.8em of 15] (16) {\strut स्त};
\node [minimum width=5mm, right=0.8em of 16] (17) {\strut व};
\node [minimum width=5mm, right=0.8em of 17] (18) {\strut ना};
\node [minimum width=5mm, right=0.8em of 18] (19) {\strut श};
\node [minimum width=5mm, right=0.8em of 19] (20) {\strut स्त्य};
\draw[bend left,->]  (01) to (02);
\draw[bend left,->]  (02) to (03);
\draw[bend left,->]  (03) to (02);
\draw[bend left,->]  (02) to (01);
\draw[bend left,->]  (01.north) to (04.north west);
\draw[bend left,->]  (04) to (05);
\draw[bend left,->]  (05) to (04);
\draw[bend left,->]  (04.north) to (06.north west);
\draw[bend left,->]  (06) to (07);
\draw[bend left,->]  (07) to (08);
\draw[bend left,->]  (08) to (07);
\draw[bend left,->]  (07) to (06);
\draw[bend left,->]  (06.north) to (09.north west);
\draw[bend left,->]  (09) to (10);
\draw[bend left,->]  (10) to (09);
\draw[bend left,->]  (11) to (12);
\draw[bend left,->]  (12) to (13);
\draw[bend left,->]  (13) to (12);
\draw[bend left,->]  (12) to (11);
\draw[bend left,->]  (11.north) to (14.north west);
\draw[bend left,->]  (14) to (15);
\draw[bend left,->]  (15) to (14);
\draw[bend left,->]  (14.north) to (16.north west);
\draw[bend left,->]  (16) to (17);
\draw[bend left,->]  (17) to (18);
\draw[bend left,->]  (18) to (17);
\draw[bend left,->]  (17) to (16);
\draw[bend left,->]  (16.north) to (19.north west);
\draw[bend left,->]  (19) to (20);
\draw[bend left,->]  (20) to (19);
\end{tikzpicture}
\end{center}
\vspace{4mm}
\begin{sloppypar}\hyphenrules{nohyphenation}\justifying\noindent\hspace{10mm} पद्मबन्धः—\end{sloppypar}
\fontsize{14}{21}\selectfont\centering\textcolor{blue}{नमस्ते मनसा भासा नव्यकाव्यनयप्रिय।\nopagebreak\\
नभश्शोभनसत्त्वास नन्दनन्दनवल्लव॥}\\
\vspace{4mm}
\begin{tikzpicture}[every node/.append style={circle, text height=2ex, text depth=.25ex, text width=1.5em, text centered, draw=black, inner sep=0pt},>=stealth',auto,thick]
\foreach \a/\t in {1/भा,2/स्ते,3/ल्ल,4/न,5/त्त्वा,6/श्शो,7/प्रि,8/का}{
\draw (\a*360/8: 4cm) node (\a) {\t};
}
\foreach \a/\t in {11/व,12/न्द,13/स,14/भ,15/य,16/व्य,17/सा,18/म}{
\draw (\a*360/8: 2cm) node[draw=none] (\a) {\t};
}
\node [draw=black,below=3.075cm of 2] (21) {\strut न};
\draw[color=blue,bend left=22.5,-]  (21.112.5) to (2.south west);
\draw[color=blue,bend right=22.5,-]  (21.67.5) to (2.south east);
\draw[color=blue,bend left=22.5,-]  (21.67.5) to (1.west);
\draw[color=blue,bend right=22.5,-]  (21.22.5) to (1.south);
\draw[color=blue,bend left=22.5,-]  (21.22.5) to (8.north west);
\draw[color=blue,bend right=22.5,-]  (21.337.5) to (8.south west);
\draw[color=blue,bend left=22.5,-]  (21.337.5) to (7.north);
\draw[color=blue,bend right=22.5,-]  (21.292.5) to (7.west);
\draw[color=blue,bend left=22.5,-]  (21.292.5) to (6.north east);
\draw[color=blue,bend right=22.5,-]  (21.247.5) to (6.north west);
\draw[color=blue,bend left=22.5,-]  (21.247.5) to (5.east);
\draw[color=blue,bend right=22.5,-]  (21.202.5) to (5.north);
\draw[color=blue,bend left=22.5,-]  (21.202.5) to (4.south east);
\draw[color=blue,bend right=22.5,-]  (21.157.5) to (4.north east);
\draw[color=blue,bend left=22.5,-]  (21.157.5) to (3.south);
\draw[color=blue,bend right=22.5,-]  (21.112.5) to (3.east);
\draw[->]  (21.101.25) to (18.258.75);
\draw[->]  (18.101.25) to (2.258.75);
\draw[->]  (2.281.25) to (18.78.75);
\draw[->]  (18.281.25) to (21.78.75);
\draw[->]  (21.56.25) to (17.213.75);
\draw[->]  (17.56.25) to (1.213.75);
\draw[->]  (1.236.25) to (17.33.75);
\draw[->]  (17.236.25) to (21.33.75);
\draw[->]  (21.11.25) to (16.168.75);
\draw[->]  (16.11.25) to (8.168.75);
\draw[->]  (8.191.25) to (16.348.75);
\draw[->]  (16.191.25) to (21.348.75);
\draw[->]  (21.326.25) to (15.123.75);
\draw[->]  (15.326.25) to (7.123.75);
\draw[->]  (7.146.25) to (15.303.75);
\draw[->]  (15.146.25) to (21.303.75);
\draw[->]  (21.281.25) to (14.78.75);
\draw[->]  (14.281.25) to (6.78.75);
\draw[->]  (6.101.25) to (14.258.75);
\draw[->]  (14.101.25) to (21.258.75);
\draw[->]  (21.236.25) to (13.33.75);
\draw[->]  (13.236.25) to (5.33.75);
\draw[->]  (5.56.25) to (13.213.75);
\draw[->]  (13.56.25) to (21.213.75);
\draw[->]  (21.191.25) to (12.348.75);
\draw[->]  (12.191.25) to (4.348.75);
\draw[->]  (4.11.25) to (12.168.75);
\draw[->]  (12.11.25) to (21.168.75);
\draw[->]  (21.146.25) to (11.303.75);
\draw[->]  (11.146.25) to (3.303.75);
\draw[->]  (3.326.25) to (11.123.75);
\draw[->]  (11.326.25) to (21.123.75);
\end{tikzpicture}
\vspace{4mm}
\begin{sloppypar}\hyphenrules{nohyphenation}\justifying\noindent\hspace{10mm} मृदङ्गबन्धः—\end{sloppypar}
\fontsize{14}{21}\selectfont\centering\textcolor{blue}{भजे महितमादर्शं व्रजेम परमादरम्।\nopagebreak\\
विन्देम गरिमागल्भं वन्देमहि तमागमम्॥}\\
\vspace{4mm}
\tikzstyle{line} = [draw, -stealth, thick]
\begin{center}
\begin{tikzpicture}[every node/.append style={circle, text height=2ex, text depth=.25ex, text width=1.75em, text centered, draw=black!80, inner sep=0pt},->,>=stealth',auto,thick]
\node [minimum width=5mm] (01) {\strut भ};
\node [minimum width=5mm, right=0.8em of 01] (02) {\strut जे};
\node [minimum width=5mm, right=0.8em of 02] (03) {\strut म};
\node [minimum width=5mm, right=0.8em of 03] (04) {\strut हि};
\node [minimum width=5mm, right=0.8em of 04] (05) {\strut त};
\node [minimum width=5mm, right=0.8em of 05] (06) {\strut मा};
\node [minimum width=5mm, right=0.8em of 06] (07) {\strut द};
\node [minimum width=5mm, right=0.8em of 07] (08) {\strut र्शं};
\node [minimum width=5mm, below=1em of 01] (09) {\strut व्र};
\node [minimum width=5mm, right=0.8em of 09] (10) {\strut जे};
\node [minimum width=5mm, right=0.8em of 10] (11) {\strut म};
\node [minimum width=5mm, right=0.8em of 11] (12) {\strut प};
\node [minimum width=5mm, right=0.8em of 12] (13) {\strut र};
\node [minimum width=5mm, right=0.8em of 13] (14) {\strut मा};
\node [minimum width=5mm, right=0.8em of 14] (15) {\strut द};
\node [minimum width=5mm, right=0.8em of 15] (16) {\strut रम्};
\node [minimum width=5mm, below=1em of 09] (17) {\strut वि};
\node [minimum width=5mm, right=0.8em of 17] (18) {\strut न्दे};
\node [minimum width=5mm, right=0.8em of 18] (19) {\strut म};
\node [minimum width=5mm, right=0.8em of 19] (20) {\strut ग};
\node [minimum width=5mm, right=0.8em of 20] (21) {\strut रि};
\node [minimum width=5mm, right=0.8em of 21] (22) {\strut मा};
\node [minimum width=5mm, right=0.8em of 22] (23) {\strut ग};
\node [minimum width=5mm, right=0.8em of 23] (24) {\strut ल्भम्};
\node [minimum width=5mm, below=1em of 17] (25) {\strut व};
\node [minimum width=5mm, right=0.8em of 25] (26) {\strut न्दे};
\node [minimum width=5mm, right=0.8em of 26] (27) {\strut म};
\node [minimum width=5mm, right=0.8em of 27] (28) {\strut हि};
\node [minimum width=5mm, right=0.8em of 28] (29) {\strut त};
\node [minimum width=5mm, right=0.8em of 29] (30) {\strut मा};
\node [minimum width=5mm, right=0.8em of 30] (31) {\strut ग};
\node [minimum width=5mm, right=0.8em of 31] (32) {\strut मम्};
\draw[->]  (01) to (02); \draw[->]  (02) to (03); \draw[->]  (03) to (04); \draw[->]  (04) to (05); \draw[->]  (05) to (06); \draw[->]  (06) to (07); \draw[->]  (07) to (08); 
\draw[->]  (09) to (10); \draw[->]  (10) to (11); \draw[->]  (11) to (12); \draw[->]  (12) to (13); \draw[->]  (13) to (14); \draw[->]  (14) to (15); \draw[->]  (15) to (16); 
\draw[->]  (17) to (18); \draw[->]  (18) to (19); \draw[->]  (19) to (20); \draw[->]  (20) to (21); \draw[->]  (21) to (22); \draw[->]  (22) to (23); \draw[->]  (23) to (24); 
\draw[->]  (25) to (26);\draw[->]  (26) to (27);  \draw[->]  (27) to (28); \draw[->]  (28) to (29); \draw[->]  (29) to (30); \draw[->]  (30) to (31); \draw[->]  (31) to (32);
\draw[color=blue,->]  (01.south east) to (10.north west); \draw[color=blue,->]  (10.south east) to (19.north west); \draw[color=blue,->]  (19.south east) to (28.north west); \draw[color=blue,->]  (28.22.5) to (29.157.5); \draw[color=blue,->]  (29.north east) to (22.south west); \draw[color=blue,->]  (22.north east) to (15.south west); \draw[color=blue,->]  (15.north east) to (08.south west); 
\draw[color=blue,->]  (25.north east) to (18.south west); \draw[color=blue,->]  (18.north east) to (11.south west); \draw[color=blue,->]  (11.north east) to (04.south west); \draw[color=blue,->]  (04.337.5) to (05.202.5); \draw[color=blue,->]  (05.south east) to (14.north west); \draw[color=blue,->]  (14.south east) to (23.north west); \draw[color=blue,->]  (23.south east) to (32.north west); 
\draw[color=red,->]  (09.north east) to (02.south west); \draw[color=red,->]  (02.337.5) to (03.202.5); \draw[color=red,->]  (03.south east) to (12.north west); \draw[color=red,->]  (12.22.5) to (13.157.5); \draw[color=red,->]  (13.north east) to (06.south west); \draw[color=red,->]  (06.337.5) to (07.202.5); \draw[color=red,->]  (07.south east) to (16.north west); 
\draw[color=red,->]  (17.south east) to (26.north west); \draw[color=red,->]  (26.22.5) to (27.157.5); \draw[color=red,->]  (27.north east) to (20.south west); \draw[color=red,->]  (20.337.5) to (21.202.5); \draw[color=red,->]  (21.south east) to (30.north west); \draw[color=red,->]  (30.22.5) to (31.157.5); \draw[color=red,->]  (31.north east) to (24.south west); 
\end{tikzpicture}
\end{center}
\vspace{4mm}
\begin{sloppypar}\hyphenrules{nohyphenation}\justifying\noindent\hspace{10mm} पुष्पगुच्छबन्धः—\end{sloppypar}
\fontsize{14}{21}\selectfont\centering\textcolor{blue}{अलोकलोक्यलोकनं  दयाऽभयास्मयाग्रहं वचोरुचोऽधिचोदनं कवित्ववित्वविग्रहम्।\nopagebreak\\
श्रयाम्ययाम्ययाजनं नमामि मापमास्पृहं भवेऽभवेड्यवेतनं जनावनात्मना महम्॥}\\
\vspace{4mm}
\begin{tikzpicture}[every node/.append style={circle, text height=2ex, text depth=.25ex, text width=1.5em, text centered, draw=black, inner sep=0pt},>=stealth',auto,thick]
\foreach \a/\t in {1/वे,2/या,3/चो,4/लो}{
\draw (33.75 + \a*360/16: 9.75cm) node (\a) {\strut \t};
}
\draw (50: 7.5cm) node (a1) {\strut नं};
\draw (15 + 2*360/12: 7.5cm) node (a2) {\strut नं};
\draw (15 + 3*360/12: 7.5cm) node (a3) {\strut नं};
\draw (130: 7.5cm) node (a4) {\strut नं};
\draw (1*360/10: 5cm) node (b1) {\strut ना};
\draw (2*360/10: 5.25cm) node (b2) {\strut मा};
\draw (3*360/10: 5.25cm) node (b3) {\strut वि};
\draw (4*360/10: 5cm) node (b4) {\strut या};
\draw (90: 0.5cm) node (c1) {\strut हम्};
\node [draw=none,above=1mm of 1] (11) {\strut भ};
\node [draw=none,right=1mm of 1] (12) {\strut ऽभ};
\node [draw=none,below=1mm of 1] (13) {\strut त};
\node [draw=none,left=1mm of 1] (14) {\strut ड्य};
\node [draw=none,above=1mm of 2] (21) {\strut श्र};
\node [draw=none,right=1mm of 2] (22) {\strut म्य};
\node [draw=none,below=1mm of 2] (23) {\strut ज};
\node [draw=none,left=1mm of 2] (24) {\strut म्य};
\node [draw=none,above=1mm of 3] (31) {\strut व};
\node [draw=none,right=1mm of 3] (32) {\strut रु};
\node [draw=none,below=1mm of 3] (33) {\strut द};
\node [draw=none,left=1mm of 3] (34) {\strut ऽधि};
\node [draw=none,above=1mm of 4] (41) {\strut अ};
\node [draw=none,right=1mm of 4] (42) {\strut क};
\node [draw=none,below=1mm of 4] (43) {\strut क};
\node [draw=none,left=1mm of 4] (44) {\strut क्य};
\node [draw=none,above=1mm of b1] (b11) {\strut ज};
\node [draw=none,right=1mm of b1] (b12) {\strut व};
\node [draw=none,below=1mm of b1] (b13) {\strut म};
\node [draw=none,left=1mm of b1] (b14) {\strut त्म};
\node [draw=none,above=1mm of b2] (b21) {\strut न};
\node [draw=none,right=1mm of b2] (b22) {\strut मि};
\node [draw=none,below=1mm of b2] (b23) {\strut स्पृ};
\node [draw=none,left=1mm of b2] (b24) {\strut प};
\node [draw=none,above=1mm of b3] (b31) {\strut क};
\node [draw=none,right=1mm of b3] (b32) {\strut त्व};
\node [draw=none,below=1mm of b3] (b33) {\strut ग्र};
\node [draw=none,left=1mm of b3] (b34) {\strut त्व};
\node [draw=none,above=1mm of b4] (b41) {\strut द};
\node [draw=none,right=1mm of b4] (b42) {\strut ऽभ};
\node [draw=none,below=1mm of b4] (b43) {\strut ग्र};
\node [draw=none,left=1mm of b4] (b44) {\strut स्म};
\draw[color=blue,bend left=67.5,-]  (1.north west) to (11.north);
\draw[color=blue,bend right=67.5,-]  (1.north east) to (11.north);
\draw[color=blue,bend left=67.5,-]  (1.north east) to (12.east);
\draw[color=blue,bend right=67.5,-]  (1.south east) to (12.east);
\draw[color=blue,bend left=67.5,-]  (1.south east) to (13.south);
\draw[color=blue,bend right=67.5,-]  (1.south west) to (13.south);
\draw[color=blue,bend left=67.5,-]  (1.south west) to (14.west);
\draw[color=blue,bend right=67.5,-]  (1.north west) to (14.west);
\draw[color=blue,bend left=67.5,-]  (2.north west) to (21.north);
\draw[color=blue,bend right=67.5,-]  (2.north east) to (21.north);
\draw[color=blue,bend left=67.5,-]  (2.north east) to (22.east);
\draw[color=blue,bend right=67.5,-]  (2.south east) to (22.east);
\draw[color=blue,bend left=67.5,-]  (2.south east) to (23.south);
\draw[color=blue,bend right=67.5,-]  (2.south west) to (23.south);
\draw[color=blue,bend left=67.5,-]  (2.south west) to (24.west);
\draw[color=blue,bend right=67.5,-]  (2.north west) to (24.west);
\draw[color=blue,bend left=67.5,-]  (3.north west) to (31.north);
\draw[color=blue,bend right=67.5,-]  (3.north east) to (31.north);
\draw[color=blue,bend left=67.5,-]  (3.north east) to (32.east);
\draw[color=blue,bend right=67.5,-]  (3.south east) to (32.east);
\draw[color=blue,bend left=67.5,-]  (3.south east) to (33.south);
\draw[color=blue,bend right=67.5,-]  (3.south west) to (33.south);
\draw[color=blue,bend left=67.5,-]  (3.south west) to (34.west);
\draw[color=blue,bend right=67.5,-]  (3.north west) to (34.west);
\draw[color=blue,bend left=67.5,-]  (4.north west) to (41.north);
\draw[color=blue,bend right=67.5,-]  (4.north east) to (41.north);
\draw[color=blue,bend left=67.5,-]  (4.north east) to (42.east);
\draw[color=blue,bend right=67.5,-]  (4.south east) to (42.east);
\draw[color=blue,bend left=67.5,-]  (4.south east) to (43.south);
\draw[color=blue,bend right=67.5,-]  (4.south west) to (43.south);
\draw[color=blue,bend left=67.5,-]  (4.south west) to (44.west);
\draw[color=blue,bend right=67.5,-]  (4.north west) to (44.west);
\draw[color=blue,bend left=67.5,-]  (b1.north west) to (b11.north);
\draw[color=blue,bend right=67.5,-]  (b1.north east) to (b11.north);
\draw[color=blue,bend left=67.5,-]  (b1.north east) to (b12.east);
\draw[color=blue,bend right=67.5,-]  (b1.south east) to (b12.east);
\draw[color=blue,bend left=67.5,-]  (b1.south east) to (b13.south);
\draw[color=blue,bend right=67.5,-]  (b1.south west) to (b13.south);
\draw[color=blue,bend left=67.5,-]  (b1.south west) to (b14.west);
\draw[color=blue,bend right=67.5,-]  (b1.north west) to (b14.west);
\draw[color=blue,bend left=67.5,-]  (b2.north west) to (b21.north);
\draw[color=blue,bend right=67.5,-]  (b2.north east) to (b21.north);
\draw[color=blue,bend left=67.5,-]  (b2.north east) to (b22.east);
\draw[color=blue,bend right=67.5,-]  (b2.south east) to (b22.east);
\draw[color=blue,bend left=67.5,-]  (b2.south east) to (b23.south);
\draw[color=blue,bend right=67.5,-]  (b2.south west) to (b23.south);
\draw[color=blue,bend left=67.5,-]  (b2.south west) to (b24.west);
\draw[color=blue,bend right=67.5,-]  (b2.north west) to (b24.west);
\draw[color=blue,bend left=67.5,-]  (b3.north west) to (b31.north);
\draw[color=blue,bend right=67.5,-]  (b3.north east) to (b31.north);
\draw[color=blue,bend left=67.5,-]  (b3.north east) to (b32.east);
\draw[color=blue,bend right=67.5,-]  (b3.south east) to (b32.east);
\draw[color=blue,bend left=67.5,-]  (b3.south east) to (b33.south);
\draw[color=blue,bend right=67.5,-]  (b3.south west) to (b33.south);
\draw[color=blue,bend left=67.5,-]  (b3.south west) to (b34.west);
\draw[color=blue,bend right=67.5,-]  (b3.north west) to (b34.west);
\draw[color=blue,bend left=67.5,-]  (b4.north west) to (b41.north);
\draw[color=blue,bend right=67.5,-]  (b4.north east) to (b41.north);
\draw[color=blue,bend left=67.5,-]  (b4.north east) to (b42.east);
\draw[color=blue,bend right=67.5,-]  (b4.south east) to (b42.east);
\draw[color=blue,bend left=67.5,-]  (b4.south east) to (b43.south);
\draw[color=blue,bend right=67.5,-]  (b4.south west) to (b43.south);
\draw[color=blue,bend left=67.5,-]  (b4.south west) to (b44.west);
\draw[color=blue,bend right=67.5,-]  (b4.north west) to (b44.west);
\draw[color=blue,bend left=20,-]  (13.south) to (a1.north east);
\draw[color=blue,bend left=10,-]  (23.south) to (a2.north);
\draw[color=blue,bend right=10,-]  (33.south) to (a3.north);
\draw[color=blue,bend right=20,-]  (43.south) to (a4.north west);
\draw[color=blue,bend left=20,-]  (a1.south) to (b11.north east);
\draw[color=blue,bend left=10,-]  (a2.south) to (b21.67.5);
\draw[color=blue,bend right=10,-]  (a3.south) to (b31.112.5);
\draw[color=blue,bend right=20,-]  (a4.south) to (b41.north west);
\draw[color=blue,bend left=20,-]  (b13.south west) to (c1.east);
\draw[color=blue,bend left=20,-]  (b23.south) to (c1.north east);
\draw[color=blue,bend right=20,-]  (b33.south) to (c1.north west);
\draw[color=blue,bend right=20,-]  (b43.south east) to (c1.west);
\end{tikzpicture}

\vspace*{4mm}
\fontsize{14}{19}\selectfont
\raggedleft{शतावधानी रा.गणेशः\nopagebreak\\
२६/१०/२०१४}\nopagebreak\\
\fontsize{12}{16}\selectfont
\raggedleft{बेङ्गलूरुनगरम्, भारतम्}\\
\pagebreak
\fontsize{18}{27}\selectfont
\centering\textcolor{blue}{\underline{अपि अपि रक्षति डुकृञ्करणे}}\nopagebreak\\
\vspace{4mm}
\fontsize{14}{21}\selectfont
\raggedleft{–~पोटोपाख्या हिमांशवः}\nopagebreak\\
\vspace{4mm}
\fontsize{14}{21}\selectfont\centering\textcolor{blue}{भज गोविन्दं भज गोविन्दं गोविन्दं भज मूढमते।\nopagebreak\\
सम्प्राप्ते सन्निहिते काले नहि नहि रक्षति डुकृञ्करणे॥}\\
\raggedleft{–~चर्पटपञ्जरिकास्तोत्रे १}\nopagebreak\\
\fontsize{14}{21}\selectfont
\begin{sloppypar}\hyphenrules{nohyphenation}\justifying\noindent\hspace{10mm} आदि\-शङ्कराचार्य उद्घोषयति \textcolor{red}{नहि नहि रक्षति डुकृञ्करणे} इति। अस्मिन् ग्रन्थे जगद्गुरु\-रामभद्राचार्य उद्घोषयति \textcolor{red}{अपि अपि रक्षति डुकृञ्करणे} इति। अस्मिन् ग्रन्थे जगद्गुरुरामभद्राचार्यः शुष्कं व्याकरणं रामरसेनौतप्रोतं कृत्वा पाठकायोपहरति। एकैकः शब्दो रामरसेन लिप्तोऽस्ति। व्याकरण\-व्याजेन जगद्गुरु\-रामभद्राचार्यो रामचन्द्रेणास्मान्मुक्तिं प्रदापयति। \textcolor{red}{अपि अपि रक्षति डुकृञ्करणे}।\end{sloppypar}
\vspace{4mm}
\fontsize{14}{19}\selectfont
\raggedleft{संस्कृतज्ञानां गोविन्द\-लाल\-मेहता\-महाराजानां दौहित्रः\nopagebreak\\
हिमांशुः पोटा\nopagebreak\\
१२/८/२०१५}\nopagebreak\\
\fontsize{12}{16}\selectfont
\raggedleft{प्राचार्यः, अभियान्त्रिकीसूचनाप्रौद्योगिकीविद्यालयः\nopagebreak\\
ऑस्ट्रेलिया\-रक्षा\-बलाकादमी\nopagebreak\\
न्यू\-साऊथ्‌वेल्स्‌विश्वविद्यालयः\nopagebreak\\
कॅनबेरा, ऑस्ट्रेलिया}\\
\vspace{8mm}
\fontsize{18}{27}\selectfont
\centering\textcolor{blue}{\underline{लोकोपकारकोऽनुसन्धानग्रन्थः}}\nopagebreak\\
\vspace{4mm}
\fontsize{14}{21}\selectfont
\raggedleft{–~बलदेवानन्दसागराः}\nopagebreak\\
\vspace{4mm}
\fontsize{14}{21}\selectfont
\begin{sloppypar}\hyphenrules{nohyphenation}\justifying\noindent\hspace{10mm} कतिपय\-दिवसेभ्यः प्राक् स्वसुहृदा हिमांशु\-पोटा\-वर्येण सार्धं वैद्युतान्तर्जाल\-माध्यमेन सम्भाष\-माणोऽहं जगद्गुरु\-वर्याणां श्रीमतां रामानन्दाचार्याणां रामभद्राचार्य\-स्वामिपादानां विषये श्रीमद्भगवद्गीताया दशमाध्यायस्य श्लोकमिमम्–\end{sloppypar}
\fontsize{14}{21}\selectfont\centering\textcolor{blue}{यद्यद्विभूतिमत्सत्त्वं श्रीमदूर्जितमेव वा।\nopagebreak\\
तत्तदेवावगच्छ त्वं मम तेजोंऽशसम्भवम्॥}\\
\raggedleft{–~भ॰गी॰~१०.४१}\nopagebreak\\
\begin{sloppypar}\hyphenrules{nohyphenation}\justifying\noindent\noindent उद्धरन्न्यगादिषं यदेता विभूतयः मानव\-समाजस्य कल्याणार्थं समुद्धारार्थञ्च जगती\-तलेऽवतरन्ति।\end{sloppypar}
\pagebreak
\begin{sloppypar}\hyphenrules{nohyphenation}\justifying\noindent\hspace{10mm} \textcolor{red}{अध्यात्म\-रामायणेऽ\-पाणिनीय\-प्रयोगाणां विमर्शः} इति महतो लोकोपकारस्यानु\-सन्धान\-ग्रन्थस्य रचनायां प्रकाशने च गुरु\-शिष्ययोः परिनिष्ठित\-प्रयासा नूनं वर्तमानानां भाविनाञ्च जनानां कृते प्रेरणास्पदी\-भूताः सेत्स्यन्तीति मे द्रढीयान् विश्वासः।\end{sloppypar}
\begin{sloppypar}\hyphenrules{nohyphenation}\justifying\noindent\hspace{10mm} अनयोः सर्व\-विधानामय\-हेतोः भगवन्तं राघवं प्रार्थये।\end{sloppypar}
\vspace{4mm}
\fontsize{14}{19}\selectfont
\raggedleft{विदुषां वशंवदः\nopagebreak\\
बलदेवानन्दसागरः\nopagebreak\\
२१/८/२०१५}\nopagebreak\\
\fontsize{12}{16}\selectfont
\raggedleft{महासचिवः, भारतीयसंस्कृतपत्रकारसङ्घः\nopagebreak\\
देहलीनगरम्, भारतम्}\\

\chapter[प्रशंसा]{प्रशंसा}
\markboth{प्रशंसा}{}
\fontsize{16}{24}\selectfont\centering\textcolor{blue}{शोधप्रबन्धपरिशीलनतः समन्तात्\\
सञ्जायते मतमिदं मम युक्तियुक्तम्।\\
शोधप्रबन्धमकरन्दमधुव्रतोऽयं\\
विद्वद्विमृग्यविरुदं लभतामिदानीम्॥}\\
\vspace{4mm}
\fontsize{14}{19}\selectfont
\raggedleft{कालिकाप्रसादशुक्लः\\
१४/१०/८१\\
\vspace{4mm}
आचार्योऽध्यक्षश्च\\
व्याकरणविभागः\\
सम्पूर्णानन्दसंस्कृतविश्वविद्यालयः\\
वाराणसी}\\

% Nityanand Misra: LaTeX code to typeset a book in Sanskrit
% Copyright (C) 2016 Nityanand Misra
%
% This program is free software: you can redistribute it and/or modify it under
% the terms of the GNU General Public License as published by the Free Software
% Foundation, either version 3 of the License, or (at your option) any later
% version.
%
% This program is distributed in the hope that it will be useful, but WITHOUT
% ANY WARRANTY; without even the implied warranty of  MERCHANTABILITY or FITNESS
% FOR A PARTICULAR PURPOSE. See the GNU General Public License for more details.
%
% You should have received a copy of the GNU General Public License along with
% this program.  If not, see <http://www.gnu.org/licenses/>.

\renewcommand\chaptername{}
\chapter[आलोकनम्]{आलोकनम्}
\markboth{आलोकनम्}{}
\fontsize{14}{19}\selectfont
\begin{sloppypar}\hyphenrules{nohyphenation}\justifying\noindent\hspace{10mm} सम्पूर्णानन्दसंस्कृतविश्वविद्यालये विद्यावारिधिः (पी-एच्.डी.) इत्युपाधिप्राप्तये गिरिधरलालमिश्रः (प्रज्ञाचक्षुः) \textcolor{red}{अध्यात्मरामायणेऽपाणिनीयप्रयोगाणां विमर्शः} इति विषयमधिकृत्य मम निर्देशने शोधप्रबन्धं प्रस्तुतवान्। शोधप्रबन्धोऽयं मया सम्यगालोकितः। परीक्षणार्होऽयमिति संस्तौति~– \end{sloppypar}
\vspace{4mm}
\raggedright{दिनाङ्कः २४/९/८१}\\
\vspace{4mm}
\raggedleft{भूपेन्द्रपतित्रिपाठी\\
भूतपूर्वव्याकरणविभागाध्यक्षः\\
सम्पूर्णानन्दसंस्कृतविश्वविद्यालयस्य}


% Nityanand Misra: LaTeX code to typeset a book in Sanskrit
% Copyright (C) 2016 Nityanand Misra
%
% This program is free software: you can redistribute it and/or modify it under
% the terms of the GNU General Public License as published by the Free Software
% Foundation, either version 3 of the License, or (at your option) any later
% version.
%
% This program is distributed in the hope that it will be useful, but WITHOUT
% ANY WARRANTY; without even the implied warranty of  MERCHANTABILITY or FITNESS
% FOR A PARTICULAR PURPOSE. See the GNU General Public License for more details.
%
% You should have received a copy of the GNU General Public License along with
% this program.  If not, see <http://www.gnu.org/licenses/>.

\renewcommand\chaptername{}
\chapter[आभारप्रदर्शनम्]{आभारप्रदर्शनम्}
\markboth{आभारप्रदर्शनम्}{}
\fontsize{14}{21}\selectfont
\begin{center}
(१)\nopagebreak\\
उर्वीं गुर्वीं प्रकुर्वन्निजपदकमलान्मेखलां यज्ञसूत्रं\nopagebreak\\
बिभ्राणः शिक्षमाणः सनियमनिगमां बाणविद्यां वसिष्ठात्।\nopagebreak\\
कन्दश्यामोऽभिरामो विहितजनमनस्सद्मवासो वसानः\nopagebreak\\
पातु श्रीरामचन्द्रस्त्वचमथ रुचिरां रौरवीं रौरवान्माम्॥\\
(२)\nopagebreak\\
साकार जगदाधार कृष्ण कृष्णाजिनाम्बर।\nopagebreak\\
तुष्टसान्दीपने चित्तं ज्ञानदीपेन दीपय॥\\
(३)\nopagebreak\\
मन्दरौ भवपाथोधेः काशीकैलासमन्दिरौ।\nopagebreak\\
रातां रामे रतिं रम्यां भवानीभूतभावनौ॥\\
(४)\nopagebreak\\
भागीरथीविमलवीचिविलासरम्या\nopagebreak\\
वृन्दारकाविबुधवन्दिवरप्रणम्या।\nopagebreak\\
श्रीविश्वनाथपदपङ्कजपूतरेणु-\nopagebreak\\
र्वाराणसी विजयते जनकामधेनुः॥\\
(५)\nopagebreak\\
दिव्योल्लसद्गिरिधरान्वितगाङ्गधारा\nopagebreak\\
चण्डांशुजासितजलाश्रितचारुगीता।\nopagebreak\\
बिभ्रत्सुगुप्तसलिला जनरञ्जना सा\nopagebreak\\
धारात्रयी लसति पूतमनःप्रयागे॥\\
(६)\nopagebreak\\
श्यामप्रिया श्याममुखप्रसूता जगत्कृते पार्थमिषेण भूता।\nopagebreak\\
श्यामावदाता धृतशास्त्रजाता मन्मानसे सा विचकास्तु गीता॥\\
(७)\nopagebreak\\
मुखात्पूज्यगोस्वामिनः सम्प्रवृत्ते भवानीपतेः काकवर्यस्य वित्ते।\nopagebreak\\
लसद्रामसीतायशोवारिपूरे मनो राजहंसो भवेन्मानसे मे॥\\
(८)\nopagebreak\\
यत्प्रेरणा समभवन्मम जीवनाय\nopagebreak\\
पीयूषयूषविलसन्नवचन्द्रलेखा।\nopagebreak\\
तां रामचन्द्रचरणाम्बुजचञ्चरीकां\nopagebreak\\
गीतां भजे बहुमतां भगिनीं स्वकीयाम्॥\\
(९)\nopagebreak\\
श्रीराजदेवं पितरं शचीं तथा स्वां मातरं पूज्यपितामहं तथा।\nopagebreak\\
स्वभ्रातरं ज्येष्ठमहं रमापतिं नमामि पुष्टोऽस्मि शिशुर्यदाशिषा॥\\
(१०)\nopagebreak\\
श्रीचन्द्रकान्तः सशशाङ्कशेखरो हिमांशुरम्यो महितो मनीषया।\nopagebreak\\
सदोम्प्रकाशः भवतात्प्रसन्नधीर्यत्सेवयाऽहं विपदो न्यवारयम्॥\\
(११)\nopagebreak\\
पिण्डीजनिं सज्जनशिष्यवश्यं विद्वद्वरं शाब्दिकपूज्यपादम्।\nopagebreak\\
चरित्रमूर्तिं स्वगुरुं सुशीलं त्रिपाठिभूपेन्द्रपतिं प्रपद्ये॥\\
(१२)\nopagebreak\\
अध्यात्मरामायणशब्दजातमपाणिनीयं यदनुग्रहेण।\nopagebreak\\
व्यमर्शयं स्वीयनिदेशकं तं सदैव भूपेन्द्रपतिं स्मरामि॥\\
(१३)\nopagebreak\\
यदाशिषा नष्टविलोचनोऽप्यहं शोधप्रबन्धाब्धिमहोऽवतीर्णवान्।\nopagebreak\\
निर्देशकं स्वं सरलं त्रिपाठिनं नमामि भूपेन्द्रपतिं समादरात्॥\\
(१४)\nopagebreak\\
शब्दाटवीसिंहमदभ्रबुद्धिं शिष्यप्रियं मञ्जुलभाषणञ्च।\nopagebreak\\
त्रिपाठिनं शास्त्रसुपाठिनं तं रामप्रसादं गुरुमानतोऽस्मि॥\\
(१५)\nopagebreak\\
नव्यप्राच्यसुशब्दशासनविधाप्रारब्धसङ्काययो-\nopagebreak\\
राध्यक्ष्यं कलयन्मुदा सुरगवीश्रीविश्वविद्यालये।\nopagebreak\\
श्रीराधाचरणारविन्दमधुपो वैदुष्यभूषान्वितो\nopagebreak\\
नित्यं शुक्लवरो गुरुर्विजयते वात्सल्यरत्नाकरः॥\\
(१६)\nopagebreak\\
शब्दाटवीकेसरिणं महामतिं छात्त्रप्रियं शिष्यदयावशंवदम्।\nopagebreak\\
कविं च शुक्लं प्रणमामि कालिकाप्रसादमार्यं स्वगुरुं मुहुर्मुहुः॥\\
(१७)\nopagebreak\\
दोषाः शोधप्रबन्धे चेन्ममैते नष्टचक्षुषः।\nopagebreak\\
गुणाश्चेत्परिलोक्येरन्मद्गुरूणां च ते स्फुटम्॥\\
(१८)\nopagebreak\\
प्रायः प्रबन्धेऽत्र विमृश्य बुद्ध्या मयैव भावा निहिताः समग्राः।\nopagebreak\\
त्रुटिर्यदि स्यान्मनुजस्वभावान्मत्वा शिशुं तां सुधियः क्षमन्ताम्॥\\
(१९)\nopagebreak\\
हिन्द्युदाहरणं दत्तं मौलिकत्वान्मया क्वचित्।\nopagebreak\\
तद्दृष्ट्वा चापलं क्षान्त्वा भावं गृह्णन्तु साधवः॥\\
(२०)\nopagebreak\\
अध्यायत्रितये चास्मिन् बहिर्भूताश्च पाणिनेः।\nopagebreak\\
शब्दाः स्वीयोपपत्त्या च यथाबुद्धि समाहिताः॥\\
(२१)\nopagebreak\\
गतदृक्स्वल्पबुद्धिश्च शास्त्रसाधनवर्जितः।\nopagebreak\\
प्रयासमाचरं चैनं दृष्ट्वा नन्दन्तु सज्जनाः॥\\
(२२)\nopagebreak\\
निरीक्ष्य भावुका बन्धं परीक्षन्तां परीक्षकाः।\nopagebreak\\
सकृन्मां करुणादृष्ट्या रामभद्रोऽपि वीक्षताम्॥\\
(२३)\nopagebreak\\
मद्भावान् परिधाप्य वर्णवसनं शोधप्रबन्धेऽत्र हि\nopagebreak\\
संस्थाप्य प्रगुणान्स्म संवितनुते रम्या च यल्लेखनी।\nopagebreak\\
सेवाभावपरायणं कृतधियं वात्सल्यभाजं मुदा\nopagebreak\\
स्वाशीर्भिः परिलालयामि सततं शिष्यं दयाशङ्करम्॥\\
(२४)\nopagebreak\\
नीलनीरजसङ्काशकान्तये दिव्यकान्तये।\nopagebreak\\
रामाय पूर्णकामाय जानकीजानये नमः॥\\
\end{center}
\raggedleft{(गिरिधरलालमिश्रः प्रज्ञाचक्षुः)}

% Nityanand Misra: LaTeX code to typeset a book in Sanskrit
% Copyright (C) 2016 Nityanand Misra
%
% This program is free software: you can redistribute it and/or modify it under
% the terms of the GNU General Public License as published by the Free Software
% Foundation, either version 3 of the License, or (at your option) any later
% version.
%
% This program is distributed in the hope that it will be useful, but WITHOUT
% ANY WARRANTY; without even the implied warranty of  MERCHANTABILITY or FITNESS
% FOR A PARTICULAR PURPOSE. See the GNU General Public License for more details.
%
% You should have received a copy of the GNU General Public License along with
% this program.  If not, see <http://www.gnu.org/licenses/>.

\renewcommand\chaptername{}
\chapter[मङ्गलाचरणम्]{मङ्गलाचरणम्}
\markboth{मङ्गलाचरणम्}{}
\fontsize{14}{20}\selectfont
\begin{center}
(क)\nopagebreak\\
कौसल्यास्तनपानलालसमना मन्दस्मितोऽव्यक्तवा-\nopagebreak\\
गेकं ब्रह्म गुडालकावृतमुखाम्भोजो घनश्यामलः।\nopagebreak\\
खेलन्पङ्क्तिरथाजिरे रघुपतिर्बालानुजैः सुन्दरो\nopagebreak\\
देवो धूलिविधूसरो विजयते रामो मुकुन्दः शिशुः॥\\
(ख)\nopagebreak\\
श्रीजानकीनयननीरजनीरजेशो\nopagebreak\\
बिभ्रद्धनुर्नवतमालतनुर्नरेशः।\nopagebreak\\
सौमित्रिमारुतिनुतो भरताग्रजन्मा\nopagebreak\\
अस्मन्मनोवियति राजतु रामचन्द्रः॥\\
(ग)\nopagebreak\\
श्रीरामचन्द्रमुखचन्द्रमसश्चकोरी\nopagebreak\\
भूनन्दिनी च मिथिलानृपतेः किशोरी।\nopagebreak\\
श्रीहंसवंशवरवैभववैजयन्ती\nopagebreak\\
सा जानकी जयति मे जननी च सीता॥\\
(घ)\nopagebreak\\
वाणीं गौरिगिरीश्वरौ गणपतिं गङ्गां च काशीं गुरून्\nopagebreak\\
पाणिन्यादिमुनीन्स्वकीयपितरौ देवोपमौ सादरम्।\nopagebreak\\
नत्वा सङ्कटमोचनं कपिपतिं मिश्रो वसिष्ठान्वयः\nopagebreak\\
सानन्दं कुरुते शिशुर्गिरिधरः शोधं सतां तुष्टये॥\\
(ङ)\nopagebreak\\
शर्वाण्यै गिरिशेन सन्निगदिते अध्यात्मरामायणे\nopagebreak\\
शब्दाः पाणिनिशासनाद्बहिरथो ये वै प्रयुक्ताः श्रुताः।\nopagebreak\\
तानेवात्र विमर्शये गुरुकृपासंलब्धया मेधया\nopagebreak\\
दोषं मे परिशोधयन्तु सुधियः शोधप्रबन्धे कृतम्॥\\
\end{center}


\setcounter{footnote}{0}
% Nityanand Misra: LaTeX code to typeset a book in Sanskrit
% Copyright (C) 2016 Nityanand Misra
%
% This program is free software: you can redistribute it and/or modify it under
% the terms of the GNU General Public License as published by the Free Software
% Foundation, either version 3 of the License, or (at your option) any later
% version.
%
% This program is distributed in the hope that it will be useful, but WITHOUT
% ANY WARRANTY; without even the implied warranty of  MERCHANTABILITY or FITNESS
% FOR A PARTICULAR PURPOSE. See the GNU General Public License for more details.
%
% You should have received a copy of the GNU General Public License along with
% this program.  If not, see <http://www.gnu.org/licenses/>.

\renewcommand\chaptername{अथ प्रस्तावना}
\chapter[प्रस्तावना]{प्रस्तावना}
\fontsize{14}{21}\selectfont
\begin{sloppypar}\hyphenrules{nohyphenation}\justifying\noindent\hspace{10mm} वेदो वै विगलित\-विभेदो विखात\-खेदो विच्छिन्न\-दोषच्छेदो निरस्ताशेष\-भ्रम\-प्रमाद\-विप्रलिप्सा\-करणापाटवादि\-पुंदोष\-शङ्का\-पङ्क\-कलङ्कावकाशो विलसित\-भगवल्लीला\-विलासो विलसति लीला\-गृहीत\-दिव्य\-विग्रहस्य विहित\-खल\-गण\-निग्रहस्य निखिल\-विभूति\-सङ्ग्रहस्य सीता\-परिग्रहस्य नारायणादि\-सहस्र\-पर\-नाम\-लक्ष्यतावच्छेदकतया सम्प्रतीतस्य दाशरथेः सन्निहित\-सच्चिदानन्द\-शर्मणो धर्म\-वर्मणो व्यापक\-ब्रह्मणः पर\-ब्रह्मणो निश्श्वास\-भूतो लोकाभिरामस्य श्रीरामस्येति निखिल\-तार्किक\-विपश्चिदपश्चिम\-मनस्वि\-मानिता मान्यता। अतो \textcolor{red}{यस्य निश्श्वसितं वेदाः} (ऋ॰वे॰सं॰ सा॰भा॰ उ॰प्र॰~२) इत्यभियुक्तोक्तिः। भाषा\-विद्वांसोऽपि तुलसीदास\-प्रभृति\-मनीषिणः कवयो मान्यतामिमां नतमस्तका मानयन्तो महीयन्ते। यथा रामचरितमानसे श्रीतुलसीदासः कथयति~–\end{sloppypar}
\centering\textcolor{red}{जाकी सहज श्वास श्रुति चारी। सो हरि पढ़ यह कौतुक भारी॥}\footnote{एतद्रूपान्तरम्–\textcolor{red}{यस्य स्वाभाविकः श्वासश्चतस्रः श्रुतयो मताः। स हरिः पठतीत्येतत्परमं कौतुकं स्थितम्॥} (मा॰भा॰~१.२०४.५)।}\nopagebreak\\
\raggedleft{–~रा॰च॰मा॰~१.२०४.५}\\
\begin{sloppypar}\hyphenrules{nohyphenation}\justifying\noindent\hspace{10mm} अस्यापौरुषेयताऽपि निर्विवादा। यद्यपि वैयाकरण\-धौरेयाः शब्दं ब्रह्मेत्यामनन्ति तथा च पातञ्जल\-महाभाष्य\-पस्पशाह्निके प्राञ्जलिः पतञ्जलिः शब्द\-नित्यत्वं प्रतिपादयन् वार्त्तिकमाह \textcolor{red}{सिद्धे शब्दार्थ\-सम्बन्धे} (भा॰प॰) इति तथाऽपि शब्द\-योजना\-क्रमोल्लङ्घन\-धिया वेदातिरिक्त\-वाङ्मयस्य पौरुषेयता। अर्थाद्भगवतो वेदस्य शब्द\-क्रमोऽपि सनातनः। यथा पूर्वं मन्त्राणां पाठस्तथाऽस्मिन् कल्पे तद्वत्पश्चाद्भाविनि कल्पेऽपि। कल्पादौ जीवैः सह स्वस्मिन्नेवोपसंहृत\-पूर्वो वेदोऽपि यथाऽऽनुपूर्व्या भगवतो ब्रह्मणो हृदये समुद्भासितो भगवता नारायणेन। तथा चास्य राद्धान्तस्य पुष्टौ \textcolor{red}{गति\-बुद्धि\-प्रत्यवसानार्थ\-शब्द\-कर्माकर्मकाणामणि कर्ता स णौ} (पा॰सू॰~१.४.५२) इति सूत्रस्य प्रयोग\-दिग्दर्शनावसरे कारक\-प्रकरणे वैयाकरण\-सिद्धान्त\-कौमुद्यां श्रीभट्टोजि\-दीक्षितोऽपि लिखति यत्~–\end{sloppypar}
\centering\textcolor{red}{शत्रूनगमयत्स्वर्गं वेदार्थं स्वानवेदयत्। \nopagebreak\\
आशयच्चामृतं देवान्वेदमध्यापयद्विधिम्॥ \nopagebreak\\
आसयत्सलिले पृथ्वीं यः स मे श्रीहरिर्गतिः।}\nopagebreak\\
\raggedleft{–~वै॰सि॰कौ॰~५४०}\\
\begin{sloppypar}\hyphenrules{nohyphenation}\justifying\noindent\hspace{10mm} इति। एवं च पुराणेष्वपि समाख्यायत इयमाख्यायिका। कल्प\-क्षये सलिल\-सम्प्लव\-परिप्लाविते सागरीभूते निखिलेऽवनि\-तले चराचरे च निमग्ने लग्ने च निद्रायां भगवति विष्णौ शेष\-शायिनि दैत्य एको ब्रह्म\-मुख\-च्युतं वेदराशिमहार्षीत्। ततो निरीक्ष्य वेदापहरणं मत्स्य\-रूपेण समवतीर्य हत्वा दिति\-सुतं पृथ्वीं शृङ्ग\-बद्ध\-नौकामारोप्य समुपदिश्य ब्रह्मविद्यां वैवस्वतमनवे प्रत्यर्पयामास भूयो वेदं विधात्रे भगवान् वासुदेवः। तथा च भागवते~–\end{sloppypar}
\centering\textcolor{red}{प्रलयपयसि धातुः सुप्तशक्तेर्मुखेभ्यः श्रुतिगणमपनीतं प्रत्युपादत्त हत्वा।\nopagebreak\\
दितिजमकथयद्यो ब्रह्म सत्यव्रतानां तमहमखिलहेतुं जिह्ममीनं नतोऽस्मि॥}\nopagebreak\\
\raggedleft{–~भा॰पु॰~८.२४.६१}\\
\begin{sloppypar}\hyphenrules{nohyphenation}\justifying\noindent\hspace{10mm} अनुमानरीत्याऽपि वैदिक्यपौरुषेयताऽवगन्तुं शक्यते। यथा वेदोऽपौरुषेयोऽविच्छिन्न\-गुरुपरम्परा\-पाठक्रमत्वात्। यन्नैवं तन्नैवमेष व्यतिरेकि\-व्याप्ति\-मूलकानुमान\-प्रकारः। अस्य माहात्म्यं सकल\-शाब्दिक\-शिरोमणि\-र्दार्शनिक\-कुञ्जर\-हरिः श्रीभर्तृहरिः पाणिनीय\-व्याकरण\-दार्शनिक\-संस्करण\-भूते स्वकीय\-वाक्यपदीय\-नामके ग्रन्थे ब्रह्मकाण्डे शब्दब्रह्म\-प्राप्ति\-मुख्य\-माध्यमतया श्लोक\-षट्केन मुक्तकण्ठं सञ्जगौ। यथा~–\end{sloppypar}
\centering\textcolor{red}{प्राप्त्युपायोऽनुकारश्च तस्य वेदो महर्षिभिः। \nopagebreak\\
एकोऽप्यनेकवर्त्मेव समाम्नातः पृथक्पृथक्॥\\
भेदानां बहुमार्गत्वं कर्मण्येकत्र चाङ्गता।\nopagebreak\\
शब्दानां यतशक्तित्वं तस्य शाखासु दृश्यते॥\\
स्मृतयो बहुरूपाश्च दृष्टादृष्टप्रयोजनाः।\nopagebreak\\
तमेवाश्रित्य लिङ्गेभ्यो वेदविद्भिः प्रकाशिताः॥\\
तस्यार्थवादरूपाणि निश्रिताः स्वविकल्पजाः।\nopagebreak\\
एकत्विनां द्वैतिनां च प्रवादा बहुधा मताः॥\\
सत्या विशुद्धिस्तत्रोक्ता विद्यैवैकपदागमा।\nopagebreak\\
युक्ता प्रणवरूपेण सर्ववादाविरोधिना॥\\
विधातुस्तस्य लोकानामङ्गोपाङ्गनिबन्धनाः।\nopagebreak\\
विद्याभेदाः प्रतायन्ते ज्ञानसंस्कारहेतवः॥}\nopagebreak\\
\raggedleft{–~वा॰प॰~१.५–१०}\\
\begin{sloppypar}\hyphenrules{nohyphenation}\justifying\noindent\hspace{10mm} एवं निखिल\-ज्ञान\-प्रादुर्भूति\-स्थानमखिल\-दर्शन\-निधानं स्वर्गापवर्ग\-सोपानं चतुर्वर्ग\-साधनं भागवत\-परम\-धनं वर्णाश्रम\-धर्म\-सदनं सकल\-कलि\-कलुष\-कदनं वदनं कमल\-वदनस्य वेदमिमं चतुर्धा व्याचक्रिरे वेद\-व्यास\-वर्या ऋग्यजुः\-सामाथर्व\-भेदेन। यथा भागवते~–\end{sloppypar}
\centering\textcolor{red}{ऋग्यजुःसामाथर्वाख्या वेदाश्चत्वार उद्धृताः।\nopagebreak\\
इतिहासपुराणं च पञ्चमो वेद उच्यते॥\\
तत्रर्ग्वेदधरः पैलः सामगो जैमिनिः कविः।\nopagebreak\\
वैशम्पायन एवैको निष्णातो यजुषामभूत्॥\\
अथर्वाङ्गिरसामासीत्सुमन्तुर्दारुणो मुनिः।\nopagebreak\\
इतिहासपुराणानां पिता मे रोमहर्षणः॥\\
त एत ऋषयो वेदं स्वं स्वं व्यस्यन्ननेकधा। \nopagebreak\\
शिष्यैः प्रशिष्यैस्तच्छिष्यैर्वेदास्ते शाखिनोऽभवन्॥\\
तथैव वेदा दुर्मेधैर्धार्यन्ते पुरुषैर्यथा।\nopagebreak\\
एवं चकार भगवान् व्यासः कृपणवत्सलः॥}\nopagebreak\\
\raggedleft{–~भा॰पु॰~१.४.२०–२४}\\
\begin{sloppypar}\hyphenrules{nohyphenation}\justifying\noindent\hspace{10mm} तत्रर्ग्वेद ऋचां सङ्ग्रह एवं च यजुर्वेदे याज्ञिक\-प्रयोगाणां समुपग्रह एवं सामवेदे ललित\-स्तोत्र\-द्वारा भगवद्यशो\-गान\-परिग्रहस्तथाऽथर्व\-वेदे सामरिक\-शास्त्र\-सम्प्रतिग्रहः। यद्वा मुख्या वेद\-त्रयी। अत एव शिवमहिम्नःस्तोत्रादौ त्रयीत्वेन चर्चा। तद्यथा~–\end{sloppypar}
\centering\textcolor{red}{त्रयी साङ्ख्यं योगः पशुपतिमतं वैष्णवमिति\nopagebreak\\
प्रभिन्ने प्रस्थाने परमिदमदः पथ्यमिति च।\nopagebreak\\
रुचीनां वैचित्र्यादृजुकुटिलनानापथजुषां\nopagebreak\\
नृणामेको गम्यस्त्वमसि पयसामर्णव इव॥}\nopagebreak\\
\raggedleft{–~शि॰म॰स्तो॰~७}\\
\begin{sloppypar}\hyphenrules{nohyphenation}\justifying\noindent\hspace{10mm} एवं च महर्षि\-वाल्मीकि\-प्रणीत आदि\-काव्ये श्रीमद्वाल्मीकीय\-रामायणे किष्किन्धा\-काण्डस्य तृतीय\-सर्गे भगवद्दर्शनेन शङ्कमानेन सुग्रीवेण प्रेषितं परिचय\-मनीषया पृष्टवन्तं धीमन्तं हनुमन्तं प्रशंसता लक्ष्मणं प्रति वदमानेन\footnote{\textcolor{red}{वदमानेन} इत्यत्र \textcolor{red}{भासनोप\-सम्भाषा\-ज्ञान\-यत्न\-विमत्युपमन्त्रणेषु वदः} (पा॰सू॰~१.३.४७) इत्यनेन भासने (भासमानेन वदता) यत्ने (उत्साहं प्रकटयता वदता) उपमन्त्रणे (लक्ष्मणं प्रति रहसि वदता) वाऽऽत्मनेपदम्।} विगत\-मानेन भगवता राघवेन्द्रेण श्रीरामचन्द्रेण त्रयाणामेव वेदानां 
चर्चा समचर्चि।\footnote{\textcolor{red}{चर्चा समचर्चि} इति कर्मणि प्रयोगः। \textcolor{red}{षिद्भिदादिभ्योऽङ्} (पा॰सू॰~३.३.१०४) इत्यनेन \textcolor{red}{चिन्ति\-पूजि\-कथि\-कुम्बि\-चर्चश्च} (पा॰सू॰~३.३.१०५) इत्यनेन वा निष्पन्नस्य \textcolor{red}{चर्चा} शब्दस्य कर्मत्वं \textcolor{red}{यतोऽनुजीविना पराधिकारचर्चा सर्वथा न कर्तव्या} (हि॰~२.३१) \textcolor{red}{मम नियोगस्य चर्चा त्वया न कर्तव्या} (हि॰~२.३२) \textcolor{red}{स्वनियोगचर्चा क्रियताम्} (हि॰~२.३५) इतिवत्।} यथा~–\end{sloppypar}
\centering\textcolor{red}{नानृग्वेदविनीतस्य नायजुर्वेदधारिणः।\nopagebreak\\
नासामवेदविदुषः शक्यमेवं प्रभाषितुम्॥}\nopagebreak\\
\raggedleft{–~वा॰रा॰~४.३.२८}\\
\begin{sloppypar}\hyphenrules{nohyphenation}\justifying\noindent\hspace{10mm} मार्कण्डेय\-पुराणान्तर्भूतायाः स्तोत्र\-रत्न\-दुर्गा\-सप्तशत्याश्चतुर्थाध्याये शक्रोऽपि त्रयीमयीमेव भगवतीं स्तौति यथा~–\end{sloppypar}
\centering\textcolor{red}{शब्दात्मिका सुविमलर्ग्यजुषां निधानमुद्गीथरम्यपदपाठवतां च साम्नाम्। \nopagebreak\\
देवी त्रयी भगवती भवभावनाय वार्ता च सर्वजगतां परमार्तिहन्त्री॥}\nopagebreak\\
\raggedleft{–~दु॰स॰श॰~४.१०}\\
\begin{sloppypar}\hyphenrules{nohyphenation}\justifying\noindent\hspace{10mm} श्रीमद्भागवतेऽपि श्रीकृष्ण\-बाल\-लीला\-प्रसङ्गे मृद्भक्षण\-लीलायां भगवन्मुखे निखिल\-ब्रह्माण्डमवलोक्य स्तब्धा यशोदा पुनस्तत्कृपया प्राप्त\-भगवदैश्वर्य\-विस्मृतिर्वात्सल्य\-धिया श्रीकृष्ण\-मुख\-कमलं लालयन्ती साश्चर्यं शुकाचार्येणाऽक्षिप्ता। अत्रापि च त्रयीशब्देनैव चर्चां चकार भगवान् बादरायणिः। यथा~–\end{sloppypar}
\centering\textcolor{red}{त्रय्या चोपनिषद्भिश्च साङ्ख्ययोगैश्च सात्वतैः। \nopagebreak\\
उपगीयमानमाहात्म्यं हरिं साऽमन्यतात्मजम्॥}\nopagebreak\\
\raggedleft{–~भा॰पु॰~१०.८.४५}\\
\begin{sloppypar}\hyphenrules{nohyphenation}\justifying\noindent\hspace{10mm} एवमेव श्रीमद्भगवद्गीतायामपि श्रीकृष्ण\-चन्द्रस्त्रयीमिमां सानुरागं समगायद्यथा~–\end{sloppypar}
\centering\textcolor{red}{त्रैविद्या मां सोमपाः पूतपापा यज्ञैरिष्ट्वा स्वर्गतिं प्रार्थयन्ते। \nopagebreak\\
ते पुण्यमासाद्य सुरेन्द्रलोकमश्नन्ति दिव्यान्दिवि देवभोगान्॥}\nopagebreak\\
\raggedleft{–~भ॰गी॰~९.२०}\\
\begin{sloppypar}\hyphenrules{nohyphenation}\justifying\noindent\hspace{10mm} गद्य\-काव्य\-निष्णातो महाकविर्बाणोऽपि कादम्बर्या मङ्गलाचरणे प्रथम\-श्लोके भगवन्तं त्रयीमयमेव तुष्टाव। यथा~–\end{sloppypar}
\centering\textcolor{red}{रजोजुषे जन्मनि सत्त्ववृत्तये स्थितौ प्रजानां प्रलये तमःस्पृशे।\nopagebreak\\
अजाय सर्गस्थितिनाशहेतवे त्रयीमयाय त्रिगुणात्मने नमः॥}\nopagebreak\\
\raggedleft{–~का॰~१.१}\\
\begin{sloppypar}\hyphenrules{nohyphenation}\justifying\noindent\hspace{10mm} उपबृंहणेनैतेन वेदानां त्रैविध्यं सुस्पष्टम्। तथा च ज्ञान\-सिद्धान्त\-वर्णन\-प्रधान ऋग्वेदो यज्ञनिर्देशकतया कर्म\-काण्ड\-प्राधान्य\-प्रतिपादन\-परो यजुर्वेदः स्तवन\-गान\-परायणतयोपासना\-सिद्धान्त\-सङ्कीर्तन\-तत्परः सामवेदः। अथर्व\-वेदस्त्रिष्वत्रान्तर्भवति यद्यप्यस्मिन् युद्ध\-विद्यायाः प्राधान्येन वर्णनमवलोक्यते। यतो ह्यथर्वशब्दः शब्दद्वयस्य संहितरूपः। \textcolor{red}{अर्व}\-शब्दो घोटकपरः। \textcolor{red}{अर्वणस्त्रसावनञः} (पा॰सू॰~६.४.१२७) इति सूत्रमपि प्रमाणम्। \textcolor{red}{अथ}\-शब्दोऽधिकार\-वाची यथा पातञ्जल\-महाभाष्ये \textcolor{red}{अथेत्ययं शब्दोऽधिकारार्थः प्रयुज्यते। शब्दानुशासनं नाम शास्त्रमधिकृतं वेदितव्यम्} (भा॰प॰) इति। इत्थमधिकार\-वाचकाथ\-शब्देन सहार्व\-शब्दस्य सन्धिः। तत्राकृति\-गणत्वादथर्व\-शब्दः शकन्ध्वादिगण आकृत्या गण्यते मनीषादिवत्। एवं \textcolor{red}{शकन्ध्वादिषु पर\-रूपं वाच्यम्} (वा॰~६.१.९१) इति वार्त्तिक\-सह\-कारेण \textcolor{red}{अचोऽन्त्यादि टि} (पा॰सू॰~१.१.६४) इति टिसञ्ज्ञया\footnote{\textcolor{red}{‘शकन्ध्वादिषु पररूपं वाच्यम्।’ तच्च टेः} (वै॰सि॰कौ॰~७९)। शकन्ध्वादित्वात् \textcolor{red}{अथर्व}\-शब्द\-सिद्धौ टिसञ्ज्ञाया उपयोगिता नास्ति परन्तु \textcolor{red}{मनीषा पतञ्जलि} इत्यादिषु टिसञ्ज्ञया विना शब्दसिद्धिर्न।} पर\-रूपम्। इत्थमथर्वशब्दस्य सिद्धिः। अन्यथाऽथ\-शब्द\-घटकाकारेण सहार्व\-शब्द\-घटकाकारस्य सन्धौ दीर्घः स्यात्। अतोऽधिकृतोऽर्ववेद इति सरलार्थः। सङ्क्षेपत ऋग्वेदे मन्त्र\-देवता\-ध्यानं यजुर्वेदे यज्ञोपयोगि\-देवतानामाह्वानं सामवेदे भगवद्विभूति\-भूत\-देवानां भगवतश्च यशो\-गानमथर्व\-वेदे रण\-प्रयाणं वर्णितं पर्यवसीयते। इमे वेदा एव निखिल\-ज्ञान\-विज्ञान\-कला\-कौशल\-सकल\-सृष्टि\-समुद्भवस्थानानि। भगवानपि सर्व\-तन्त्र\-स्वतन्त्रः सर्वद्रष्टा सर्वान्तर्यामी सत्यकामः सत्यसङ्कल्पः सत्यसन्धः सर्वाधिष्ठान\-स्वरूपः कर्तुमकर्तुमन्यथा\-कर्तुं समर्थोऽपि कदाऽपि स्वेच्छया जगदिदन्तनं सिसृक्षति। सोऽपि वेद\-प्रतिपादन\-प्रकारमेवानुकरोति। किं बहुना परमात्मनः प्रामाण्यमपि वेदमेवाधिशेते। अत एव वेदमान्यता\-बहिर्भूतमीश्वरोक्तं वयमास्तिका नाद्रियामहे यथा बुद्धोक्तम्। ईश्वरोऽपि कदाचिद्धर्मसमुद्दिधीर्षया भक्त\-हृदय\-विजिहीर्षयाऽसुर\-दर्पाप\-जिहीर्षया चिकीर्षया च सुर\-कार्यस्य समवतीर्णोऽवनितले विडम्बयन्मनुज\-लोकं मानवोचित\-च्छल\-प्रपञ्चादिकमङ्गीकरोति। अस्यां युक्तौ जालन्धर\-वध\-प्रसङ्गः प्रमाणम्। किन्तु वेदः कदाचिदपि भ्रम\-प्रमाद\-विप्रलिप्सा\-करणापाटवादि\-पुंदोषान्दूरतो न परामृशति। अत एव वेदानां स्वतः प्रामाण्यमिति मीमांसकसिद्धान्तः सन्नपि सकलपण्डितसम्मतः। वेदाः स्वतः प्रमाणम्। ततः पूर्वं कस्यापि ग्रन्थस्यानुपलब्धत्वाद्वेदस्य स्वतः प्रामाण्यमपर\-शास्त्राणां च परतः प्रामाण्यम्। इमे वेदाः परम्परयेतर\-देवताः प्रतिपादयन्तोऽपि परमात्मानमेव प्रतिपादयन्ति यथा काचित्कुलाङ्गना परिवार\-जनाञ्छ्वश्रु\-प्रभृतीञ्छुश्रूष\-माणाऽपि साक्षात्पतिमेव परिचरति तथैव वेदापरनामधेयाः श्रुतयोऽपि परिवार\-जनानिवेतर\-देवान् गायन्त्योऽपि साक्षान्महा\-तात्पर्यतया स्वकीयं पतिं परमेश्वरमेव प्रतिपद्यन्ते। इयं भगवती श्रुतिरेव समेषां माता परमात्मा च परा\-पराणां पिता। यथा च पुत्रस्य कृते पितुः सत्तायां माता प्रमाणं तथैव परमेश्वर\-सत्तायामस्मत्कृते श्रुतिः प्रमाणम्। अतोऽभियुक्ता आमनन्ति \textcolor{red}{सर्वे वेदा यत्पदमामनन्ति} (क॰उ॰~१.२.१५)। भगवती श्रुतिरपि श्रावयति \textcolor{red}{अ॒मृत॑स्य पु॒त्राः} (शु॰य॰वा॰मा॰~११.५, श्वे॰उ॰~२.५) तमेव \textcolor{red}{विजिज्ञासस्व} (तै॰उ॰~३.१.१) तमेव \textcolor{red}{ब्राह्मणाः व्रतेन तपसाऽनाशकेन च विविदिषन्ति} (बृ॰उ॰~४.४.२२) \textcolor{red}{सत्यं ज्ञानमनन्तं ब्रह्म} (तै॰उ॰~२.१.१) \textcolor{red}{इ॒दं विष्णु॒र्वि च॑क्रमे} (ऋ॰वे॰सं॰~१.२२.१७, शु॰य॰वा॰मा॰~५.१५) \textcolor{red}{खं ब्रह्म} (शु॰य॰वा॰मा॰~४०.१७, बृ॰उ॰~५.१.१) इत्यादि\-मन्त्रैः। गीताऽपि परमात्मन एव वेद\-वेद्यत्वं तस्यैव च वेद\-वेत्तृत्वं गायति। यथा~–\end{sloppypar}
\centering\textcolor{red}{सर्वस्य चाहं हृदि सन्निविष्टो मत्तः स्मृतिर्ज्ञानमपोहनं च।\nopagebreak\\
वेदैश्च सर्वैरहमेव वेद्यो वेदान्तकृद्वेदविदेव चाहम्॥}\nopagebreak\\
\raggedleft{–~भ॰गी॰~१५.१५}\\
\begin{sloppypar}\hyphenrules{nohyphenation}\justifying\noindent\hspace{10mm} \textcolor{red}{श्रुति}\-शब्दः श्रवणार्थक\-\textcolor{red}{श्रु}\-धातोर्निष्पद्यते (\textcolor{red}{श्रु श्रवणे} धा॰पा॰~९४२)। एवं च \textcolor{red}{श्रूयत आनुपूर्व्याऽनादि\-कालाद्गुरु\-परम्परया या सा श्रुतिः} इति व्युत्पत्तौ कर्मणि क्तिन्प्रत्ययेऽनुबन्ध\-लोपे कृत्प्रत्ययान्तत्वाद्विभक्ति\-कार्ये प्रथमैकवचने श्रुतिः। इत्थं हि \textcolor{red}{अस्मदभिन्न\-कर्तृक\-वर्तमान\-कालिक\-श्रवणानुकूल\-व्यापार\-विशिष्ट\-निःसीम\-शब्द\-कर्मक\-श्रवण\-कर्म\-रूपं फलम्} इति फलमुख्य\-विशेष्यकः शाब्द\-बोधः। यद्वा \textcolor{red}{श्रूयते परमात्मा यया सा श्रुतिः} इति व्युत्पत्तौ \textcolor{red}{स्त्रियां क्तिन्} (पा॰सू॰~३.३.९४) इति सूत्रेण करणे क्तिन्। एवं \textcolor{red}{वर्तमान\-कालिक\-श्रवणानुकूल\-व्यापारावच्छिन्न\-श्रवण\-कर्मत्वाभिन्न\-फलत्व\-निष्ठ\-प्रकारता\-निरूपित\-फल\-निष्ठ\-विशेष्यता\-
शालि\-फल\-करणम्} एव श्रुतिः। अथवा \textcolor{red}{श्रावयति परमात्मानं पितरं पुत्रैः परिचाययति या सा श्रुतिः} इति व्युत्पत्तावन्तर्भावित\-ण्यर्थता\-वच्छेदक\-धातुत्वावच्छिन्न\-\textcolor{red}{श्रु}\-धातोर्बाहुलक\-\textcolor{red}{क्तिन्‌}प्रत्ययः। एवं च \textcolor{red}{वर्तमान\-कालावच्छिन्न\-ब्रह्म\-विषयक\-श्रवणानुकूल\-व्या\-पारानुकूल\-व्यापाराश्रयः} श्रुतिरिति दिक्। एवमेव \textcolor{red}{वेद}\-शब्दोऽपि चतुर्भिर्धातुभिर्निष्पादयितुं
शक्यते। तत्र \textcolor{red}{विदँ सत्तायाम्} (धा॰पा॰~११७१) इत्यस्मात्सत्तार्थक\-धातोः \textcolor{red}{विद्यते निरन्तरं वर्तते यद्ब्रह्म\-विषयक\-ज्ञानं तदेव वेद} इति कर्तरि घञ्‌प्रत्ययेऽचि वा गुणे कृते सति वेद\-शब्दः सिध्यति। एवं \textcolor{red}{वर्तमान\-कालावच्छिन्न\-भवन\-फलानुकूल\-व्यापाराश्रयः} इति शाब्द\-बोधः। द्वितीयस्मिन् कल्पे \textcolor{red}{विदँ विचारणे} (धा॰पा॰~१४५०) इत्यस्माद्धातोः \textcolor{red}{विन्ते ब्रह्मतत्त्वं विचारयति जनो यस्मिन् स वेदः} इति विग्रहे \textcolor{red}{हलश्च} (पा॰सू॰~३.३.१२१) इत्यनेनाधिकरणे घञ्‌प्रत्ययः। तथा चानुबन्ध\-कार्ये \textcolor{red}{पुगन्तलघूपधस्य च} (पा॰सू॰~७.३.८६) इत्यनेन गुणे निष्पन्नो वेद\-शब्दः। तथा च \textcolor{red}{विचारकाभिन्न\-कर्तृक\-वर्तमान\-कालावच्छिन्न\-मनो\-बुद्धि\-संयोगानुकूल\-व्यापाराधिकरणं वेदः} इति निर्गलितोऽर्थः। तृतीयस्मिन् कल्पे \textcolor{red}{विदँ ज्ञाने} (धा॰पा॰~१०६४) इत्यस्माद्धातोः \textcolor{red}{वेत्ति जानाति सनातनं परमं ब्रह्म निरञ्जनं निर्गुणं निराकारं निष्प्रकारं निर्लेपं निर्मानं निर्मोहं सच्चिदानन्द\-सन्दोहं निर्विकल्पं तथा चानन्द\-कन्द\-मुकुन्द\-सतत\-मुनिजन\-परिपीत\-चरणारविन्दामन्द\-मकरन्द\-सच्चिदानन्द\-चिन्मय\-समलङ्कृत\-प्रणयि\-हृदय\-कोसलेन्द्र\-सरयू\-तरल\-तरङ्ग\-भङ्गि\-विक्षालित\-चरण\-सरोज\-मनोज\-वैरि\-वन्दित\-पाद\-पयोज\-कोशल\-वीचि\-विघूर्ण\-धूलि\-धूसर\-सरस\-शिरोज\-कपोलावलम्बि\-रोलम्ब\-निन्दक\-कुटिल\-कुन्तल\-संस्मारित\-पाटल\-कोश\-विहारि\-भृङ्ग\-सौभग\-कौसल्या\-क्रोड\-वर्त्ति\-दशरथ\-नयन\-वर्ति\-वर्तुल\-मुख\-मृगाङ्क\-संव्रीडित\-शारद\-शशाङ्क\-परम\-करुण\-नील\-तरुण\-तामरस\-शरीर\-परम\-रणधीर\-रघुवीर\-पूर्णचन्द्र\-निभानन\-निहत\-दशानन\-श्याम\-सुन्दर\-पर\-ब्रह्म\-सगुण\-विग्रह\-साकार\-भगवन्तं श्रीरामचन्द्रं वा येन स वेदः}। अथ \textcolor{red}{साधकाभिन्न\-कर्तृक\-वर्तमान\-कालावच्छिन्न\-ज्ञानानुकूल\-व्यापार\-विशिष्ट\-निर्गुण\-सगुण\-ब्रह्म\-विषयक\-ज्ञान\-रूप\-फल\-मुख्य\-व्यापाराख्यं करणं वेदः} इति फलितार्थः। चतुर्थकल्पे \textcolor{red}{विदॢँ लाभे} (धा॰पा॰~१४३२) इति लाभार्थक\-धातोः \textcolor{red}{विन्दति लभते सगुण\-ब्रह्म श्रीरामचन्द्रं येन स वेदः} इति व्युत्पत्तौ करणे घञ्यनुबन्ध\-कार्ये गुणे च सति वेदशब्दस्य सिद्धिः। एवं \textcolor{red}{वर्तमान\-कालावच्छिन्न\-लाभानुकूल\-व्यापारविशिष्ट\-फलकरणं वेदः}। निर्गुणं सगुणं वा ब्रह्म वेद\-प्रतिपाद्यैरुपायैरेवानु\-भवितुं द्रष्टुं वा शक्यते। तत्र वेदे त्रय उपायास्त्रिभिः काण्डैः प्रतिपादिताः। कर्म उपासना ज्ञानं चेति। कर्मकाण्डं कर्म\-प्रति\-पादकमशीति\-सहस्रमन्त्रात्मकम्। उपासना\-काण्डमुपासनां बोधयति। तच्च षोडश\-सहस्र\-मन्त्रात्मकम्। ज्ञान\-काण्डं ज्ञान\-निरूपण\-परम्। तच्च चतुस्सहस्र\-मन्त्रात्मकम्। तत्र \textcolor{red}{वेद\-प्रणिहितो धर्मः} (भा॰पु॰~६.१.४०) इति वचनेन वेदप्रतिपाद्यमेव कर्म। तच्च द्विविधं नित्यं नैमित्तिकञ्च। \textcolor{red}{नित्यत्वं नाम कृते सत्यदृष्ट\-फलाजनकत्वेऽपि न कृते पापजनकत्वम्}।
\textcolor{red}{नैमित्तिकत्वं नाम निमित्ते विधीयमानत्वे सति पुण्य\-जनकतावच्छेदकत्वम्}। नित्यं कर्म सन्ध्यादिकं श्रौतं स्मार्तं च। तस्यावश्यं करणीयता यथा गीतायामशेष\-विशेषातीतोऽतसी\-कुसुमोपमेय\-कान्तिः पार्थ\-सारथिर्निखिल\-गुण\-गण\-निलय उत्तम\-श्लोको भक्तानुग्रह\-कातरः श्रीकृष्ण\-चन्द्रः साटोपं पार्थं प्रति प्रत्य\-पादयद्यत्~–\end{sloppypar}
\centering\textcolor{red}{नियतं कुरु कर्म त्वं कर्म ज्यायो ह्यकर्मणः। \nopagebreak\\
शरीरयात्राऽपि च ते न प्रसिद्ध्येदकर्मणः॥}\nopagebreak\\
\raggedleft{–~भ॰गी॰~३.८}\\
\centering\textcolor{red}{एवं ज्ञात्वा कृतं कर्म पूर्वैरपि मुमुक्षुभिः।\nopagebreak\\
कुरु कर्मैव तस्मात्त्वं पूर्वैः पूर्वतरं कृतम्॥}\nopagebreak\\
\raggedleft{–~भ॰गी॰~४.१५}\\
\centering\textcolor{red}{कर्मण्येवाधिकारस्ते मा फलेषु कदाचन।\nopagebreak\\
मा कर्मफलहेतुर्भूर्मा ते सङ्गोऽस्त्वकर्मणि॥}\nopagebreak\\
\raggedleft{–~भ॰गी॰~२.४७}\\
\begin{sloppypar}\hyphenrules{nohyphenation}\justifying\noindent\hspace{10mm} किं बहुना कार्याकार्य\-व्यवस्थायां शास्त्रमेव प्रमाणं गीता\-कारः प्रमिमीते यथा~–\end{sloppypar}
\centering\textcolor{red}{यः शास्त्रविधिमुत्सृज्य वर्तते कामकारतः। \nopagebreak\\
न स सिद्धिमवाप्नोति न सुखं न परां गतिम्॥\nopagebreak\\
तस्माच्छास्त्रं प्रमाणं ते कार्याकार्यव्यवस्थितौ।\\
ज्ञात्वा शास्त्रविधानोक्तं कर्म कर्तुमिहार्हसि॥}\nopagebreak\\
\raggedleft{–~भ॰गी॰~१६.२३–२४}\\
\begin{sloppypar}\hyphenrules{nohyphenation}\justifying\noindent\hspace{10mm} इत्थं कर्मणोऽस्य द्वौ भागौ विधिर्निषेधश्चेति। तत्रान्यकर्मणोऽत्यन्ताप्राप्तौ विधीयमान\-कर्मतावच्छेदकत्वं विधित्वम्। तद्यथा \textcolor{red}{अहरहः सन्ध्यामुपासीत}\footnote{मूलं मृग्यम्।} इति विधिः। अत्रैव निषेधस्यैव क्रोडे द्वावपरावप्यंशौ नियमः परिसङ्ख्या च। \textcolor{red}{नियमः पाक्षिके सति} (त॰वा॰~१.२.४२)। यथा \textcolor{red}{व्रीहीनवहन्ति}। अत्रावहननमावश्यकं नखैर्वाऽन्यैः साधनैर्वा। परिसङ्ख्या। तत्र विधेयांशेऽन्यत्राविधेयांशे चाप्राप्तौ परिसङ्ख्या। यथा विद्वांसो निगदन्ति~–\end{sloppypar}
\centering\textcolor{red}{विधिरत्यन्तमप्राप्तौ नियमः पाक्षिके सति। \nopagebreak\\
तत्र चान्यत्र चाप्राप्तौ परिसङ्ख्येति कीर्त्यते॥}\nopagebreak\\
\raggedleft{–~त॰वा॰~१.२.४२}\\
\begin{sloppypar}\hyphenrules{nohyphenation}\justifying\noindent\hspace{10mm} निषेधो नाम कृते सति पापजनकतावच्छेदकः शास्त्र\-समभिव्याहृत\-नञर्थावच्छेदक\-वदभित्याज्यः कर्म\-विशेषः। यथा \textcolor{red}{स्वाध्याय\-प्रवचनाभ्यां न प्रमदितव्यम्} (तै॰उ॰~१.११.१) \textcolor{red}{मा हिंस्याः सर्वभूतानि},\footnote{मूलं मृग्यम्।} \textcolor{red}{मा गृधः कस्य\-स्विद्धनम्} (ई॰उ॰~१) इत्यादि। एवमुपासना\-काण्ड\-प्रतिपादनं भगवन्तं वेदमाश्रित्यैवोपास्यमानः परमात्मा साधकानां कामदो भवति। शास्त्रमतिरिच्य किमपि 
कृतं नैव फलदम्। अतो गीतायाः पूर्व\-निर्दिष्टे \textcolor{red}{यः शास्त्रविधिम्} (भ॰गी॰~१६.२३) इत्यादि\-श्लोके शास्त्र\-विधेरेव सफलत्वं सकलं प्रत्यपादि भगवता श्रीकृष्णेन। उपासनायां श्रुति\-सम्मतत्वं नितरामपेक्षितम्। तत्र वेदार्थ\-प्रतिपादक\-स्मृत्यनुसारं पञ्च सम्प्रदायाः। स्मृतेर्हि वेद\-मूलकत्वेन प्रामाण्यम्। वेदस्य निगूढार्थान् स्मृतिः स्मरति। स्मृति\-कर्तार ऋषयो मन्वादयः। स्मृतेः प्रामाण्यं पूर्वमेव वेदः प्रशंसति यथा \textcolor{red}{यद्वै किं च॒ मनु॒रव॑द॒त्तद्भे॑ष॒जम्} (कृ॰य॰ तै॰सं॰~२.२.१०.२)। महाकवि\-कालिदासोऽपि स्मृतेर्वेदानु\-गन्तृत्वमुपमया स्तौति। यथा रघुवंश\-महाकाव्ये~–\end{sloppypar}
\centering\textcolor{red}{तस्याः खुरन्यासपवित्रपांसुमपांसुलानां धुरि कीर्तनीया। \nopagebreak\\
मार्गं मनुष्येश्वरधर्मपत्नी श्रुतेरिवार्थं स्मृतिरन्वगच्छत्॥}\nopagebreak\\
\raggedleft{–~र॰वं॰~२.२}\\
\begin{sloppypar}\hyphenrules{nohyphenation}\justifying\noindent\hspace{10mm} तस्माच्छ्रुत्यनुमोदितानेवार्थान् स्मृतिः स्मरति। तस्यां गाणपत्य\-सौर\-शाक्त\-शैव\-वैष्णव\-नामधेयाः पञ्च सम्प्रदायाः। गाण\-पत्य\-सम्प्रदायो गणेशमेव मुख्यमाद्रियतेऽन्यान् गौणत्वेनैवोपास्ते। तस्य बाहुल्यमद्यापि महाराष्ट्रे विलोक्यते। सौर\-सम्प्रदायः सूर्यमेव प्रधानत्वेन समर्थयतीतरान् तदङ्गत्वेनैवाञ्चति। अस्य प्राधान्यं गुर्जर\-प्रान्ते। शाक्त\-सम्प्रदाये शक्तेः प्राधान्यम्। अयं सम्प्रदायश्च कश्मीर\-बङ्गयोः प्रचुरः प्रसरति। शैव\-सम्प्रदाये प्रामुख्येन शशाङ्क\-शेखरो भूत\-भावनोऽभयङ्करः प्रलयङ्करः 
काशीश्वरः प्रचण्ड\-ताण्डवोपसंहृत\-भूत\-वरूथ\-भसित\-भासित\-भाल\-पट्टिकः 
कराल\-भाल\-क्रतु\-वेदिका\-जाज्वल्यमान\-समिन्धन\-घन\-धर्षित\-निखिल\-जन्तु\-जलारि\-धनञ्जय\-समुच्छलत्स्फुलिङ्ग\-समाहुतीकृत\-निखिल\-जन\-मनः\-क्षोभ\-कर\-मकर\-केतनः कैलास\-निकेतनो विरसित\-वैराग्य\-रस\-निमग्न\-जगती\-तल\-भीषण\-भोग\-लालसा\-मनो\-निकेतन\-भवानी\-समलङ्कृत\-वाम\-विग्रहः कृत\-जालन्धर\-गजासुर\-त्रिपुर\-प्रभृति\-प्रचण्ड\-दानव\-निग्रहः शान्त\-वासनाग्रहो भगवाञ्छिवः प्राधान्येन परिगण्यते। एतत्प्राचुर्यं केरल\-कर्णाटकादौ। वैष्णव\-सम्प्रदाये प्रमुखतया निखिल\-कोटि\-ब्रह्माण्डाधीशो जगदीशः पुरन्दर\-त्रिपुर\-हर\-वसु\-विरिञ्चि\-पावक\-पवमान\-हिमभानु\-चित्रभानु\-कृशानु\-निखिल\-नक्षत्र\-गण\-सकल\-सुरासुर\-यक्ष\-गन्धर्व\-किन्नर\-चारण\-सिद्ध\-साध्य\-नर\-नाग\-लोक\-पाल\-नाक\-पाल\-काल\-कराल\-व्याल\-माल\-मौलि\-सम्पूजित\-पादारविन्द\-लेखाधीश\-मुकुटमणि\-संलालित\-चरण\-सरोज\-रजो\-मकरन्द\-त्रैलोक्य\-लक्ष्मी\-नवनलिन\-ललित\-लोचन\-लसित\-कमन\-कटाक्ष\-संवीक्षित\-शुभेक्षित\-भव्य\-सौन्दर्य\-सार\-सर्वस्व\-सकल\-सुषमा\-सार\-भूत\-मनोरम\-सार्वभौम\-शर्वरी\-प्राण\-वल्लभ\-संव्रीडक\-मदन\-मान\-पीडक\-वदन\-सरसीरुह\-निसर्ग\-निहित\-सुधित\-सुधा\-माधुरीक\-श्रीवत्स\-लाञ्छन\-शरच्छशाङ्कानन\-मधुर\-मन्द\-स्मित\-तुहिनांशु\-दीधिति\-विहित\-प्रपन्न\-हृदय\-सरसी\-कैरव\-विकास\-ललित\-लीला\-विलास\-महा\-लक्ष्मी\-निवास\-क्षीर\-सागर\-सम्मन्थन\-सञ्जात\-कण\-संस्पर्श\-भग्न\-भक्त\-भूरि\-भव\-भय\-व्रण\-पाणि\-पल्लव\-जगती\-तल\-पाथोधि\-प्लव\-नाशित\-प्रणिपात\-क्लव\-विरचित\-गणिका\-गजाजामिल\-पङ्किल\-पङ्क\-कलङ्क\-भव\-विहित\-भव\-संस्तव\-करुणार्णव\-सकल\-शरण्य\-वरेण्य\-तरुणारुण\-सरसिज\-चरण\-तरुण\-तमाल\-नील\-सरसीरुह\-मरकत\-मणि\-कालिन्दी\-कीलाल\-विनिन्दक\-भुवनाभि\-राम\-श्याम\-शरीर\-दामिनी\-द्युति\-विनिन्दक\-संवीत\-पीत\-परिधान\-म्लान\-मदमत्त\-मनोभव\-धीर\-मधु\-कैटभ\-नरक\-मुर\-प्रचुर\-महासुर\-भूरि\-मद\-गर्वित\-गजेन्द्र\-गण्डस्थल\-विनिर्गत\-शोणित\-करकमल\-कलित\-कौमोदकी\-धर\-शङ्ख\-चक्र\-धर\-कमल\-कलित\-कमल\-कर\-केयूर\-कुण्डल\-कटक\-वनमाला\-नूपुरादि\-भूषण\-मण्डित\-सकल\-कला\-कलाप\-पण्डित\-कमलामल\-मुख\-चन्द्र\-चकोर\-वैकुण्ठ\-विहरण\-परायण\-नारायण\-प्रभविष्णु\-विष्णुर्हृदय\-मन्दिरे महीयते। अस्य प्रायशो विश्वस्मिन् विश्वे प्रचारः प्रसारश्च। अस्य 
प्रवर्तका रामानुज\-रामानन्द\-वल्लभ\-निम्बार्क\-मध्व\-सूर\-तुलसीदास\-प्रमुखाः। उपासना\-प्रतिपादक\-शास्त्रेषु वेद\-मूलकेषु नारद\-पाञ्चरात्र\-शाण्डिल्य\-सूत्र\-नारद\-भक्ति\-सूत्र\-भक्ति\-रसायन\-भक्ति\-रसामृत\-सिन्धु\-भक्तिसुधा\-प्रभृति\-नामानि। तत्रोपासनायां भावानुसारिणः पञ्च प्रकाराः। ते सन्ति शान्त\-वात्सल्य\-दास्य\-सख्य\-मधुराख्याः। शान्ते शान्त\-परम\-प्रकाश\-रूपं परम\-ज्योतिर्मयमखण्डमनन्तं भगवन्तं हृदय\-सहस्रदल\-कमले समुपासते योगिनः शिवादयः। वात्सल्ये पुत्रं शिष्यं वा मत्वा शिशु\-रूपं निर्गुणमपि सगुणं निरञ्जनमपि साञ्जनं व्यापकमपि व्याप्यं पितरमपि पुत्रं पुराणमपि नूतनं प्राचीनमप्यार्वाचीनं निराकारमपि साकारमकलमपि सकलं निरामयमपि सामयं दुर्लभमपि सुलभमचलमपि चलं निर्लेपमपि धूलिलेपं निरावरणमपि साभरणावरणमवर्णमप्यनुगत\-पितृवर्णमव्यक्तमपि व्यक्त\-कौतुकमविषादमपि स्तन्यपानार्थ\-विलसित\-विषादं निराधारमपि स्वीकृतकौसल्यकाधारं श्रीराममरूपमपि समाश्रित\-नीलनीरधरश्यामं भुवनाभिराम\-सकल\-विश्राम\footnote{\textcolor{red}{विश्रामो विश्रमश्चापि} इति द्विरूपकोशे श्रीहर्षः।}\-धाम\-निखिल\-लावण्य\-ललाम\-लोक\-लोचनाभिराम\-बालरूपं परमेश्वरं सम्भजन्ते दशरथ\-कौसल्या\-विश्वामित्र\-वसिष्ठादयः। दास्ये जगदात्मानं परमात्मानमैश्वर्य\-बुद्ध्या सकल\-निर्मातारं रचयितारं त्रिभुवनस्य निखिल\-प्रपञ्च\-भूतं त्रिलोकपितरं कर्तुमकुर्तुमन्यथाकर्तुं समर्थं सत्य\-सङ्कल्प\-सत्य\-काम\-सौन्दर्येश्वर\-माधुर्य\-सौशील्य\-सौजन्य\-सौलभ्य\-सौरभ्य\-सारल्य\-तारल्य\-कारुण्य\-तारुण्य\-वात्सल्यादि\-गुण\-गण\-रत्नाकरं करुणाकरं श्री\-राघवेन्द्रं स्वामिनं स्व\-सर्वस्वं मत्वा सर्व\-भावेन भावयन्ति हनुमत्प्रमुखाः सन्तः। सख्ये कूटस्थमपि भूयिष्ठं हृदयेशमपि विडम्बित\-नरेश\-वेषमदेहमपि लीलागृहीत\-देहं सर्व\-गेह\-मपि स्वीकृत\-पर्णकुटीर\-गेहं सच्चिदानन्द\-सन्दोहं कोशलेन्द्रं प्रणत\-सहायं सखायं मत्वा सञ्चक्षते सुग्रीवादयः। माधुर्येण निखिल\-रसामृत\-मूर्तिं सौन्दर्य\-क्षीरनिधि\-रोहिणीशं भुवन\-सुन्दरं सीता\-पाणिग्रहण\-लालसं धृत\-वर\-वेषं धनुर्धरमपि कामारि\-कार्मुक\-खण्डन\-परं परमात्मानं मधुरं तत्तत्सम्बन्ध\-समाश्रयं मत्वा भजन्ते भावुका मिथिला\-निवासिनः। एषु सर्वेषु भावेषु मौलिकतया दास्यमेव। भावनावैषम्यदृष्ट्या प्रकाराः प्रादर्शिषत। अतो विपश्चितो ब्रुवन्ति \textcolor{red}{दासभूतास्स्वतः सर्वे ह्यात्मानः परमात्मनः} (अहि॰सं॰~११) इति। उपासनाऽपि श्रुतिमूलिका। अर्वाचीन\-कल्पितायां भक्तौ नैव फलजनकतावच्छेदकता। अतो मान्यतैषा मान्यानां महात्मनाम्~–\end{sloppypar}
\centering\textcolor{red}{श्रुतिस्मृतिपुराणादिपञ्चरात्रविधिं विना। \nopagebreak\\
ऐकान्तिकी हरेर्भक्तिरुत्पातायैव कल्पते॥}\footnote{मूलं मृग्यम्।}\\
\begin{sloppypar}\hyphenrules{nohyphenation}\justifying\noindent\hspace{10mm} तुलसीदासोऽपि वेदानुमोदितामेव भक्तिं वरीयसीं वरयाम्बभूव। यथा~–\end{sloppypar}
\centering\textcolor{red}{श्रुति सम्मत हरि भक्ति पथ संजुत बिरति बिबेक।\nopagebreak\\
तेहिं न चलहिं नर मोहबश कल्पहिं पंथ अनेक॥}\footnote{एतद्रूपान्तरम्–\textcolor{red}{यो मार्ग आस्ते श्रुतिसम्मतः श्रीनाथस्य भक्तेः सविरक्तिबोधः। तस्मिन्नरा यान्ति न मोहनिघ्ना मार्गाननेकानपि कल्पयन्ति॥} (मा॰भा॰~७.१००ख)।}\nopagebreak\\
\raggedleft{–~रा॰च॰मा॰~७.१००ख}\\
\begin{sloppypar}\hyphenrules{nohyphenation}\justifying\noindent\hspace{10mm} ज्ञान\-काण्डं ज्ञान\-प्रतिपादकं चतुः\-सहस्त्र\-मन्त्रात्मकम्। इदमेव वेदान्त\-दर्शनमिति कथ्यते। तत्र \textcolor{red}{वेदस्यान्तः सिद्धान्तः परम\-तात्पर्यं वा यस्मिन् सः} इति विग्रहे \textcolor{red}{अतिँ बन्धने} (धा॰पा॰~६१) इत्यस्माद्धातोरन्तति बध्नाति व्यवस्थापयति वेदानां महा\-तात्पर्यं यः स वेदान्तः। \textcolor{red}{अन्तः} इत्यत्र पचादित्वात् \textcolor{red}{अच्‌}\-प्रत्ययः।\footnote{\textcolor{red}{नन्दि\-ग्रहि\-पचादिभ्यो ल्युणिन्यचः} (पा॰सू॰~३.१.१३४) इत्यनेन कर्तरि।} तस्य \textcolor{red}{वेद}\-शब्देन सह व्यधिकरण\-बहुव्रीहिः। यद्यपि \textcolor{red}{अनेकमन्यपदार्थे} (पा॰सू॰~२.२.२४) इति सूत्रं समानाधिकरण\-बहुव्रीहि\-परम्। तस्य ह्यर्थोऽन्य\-पदार्थे वर्तमानमनेक\-सुबन्तं समस्यते स च बहुव्रीहिः। \textcolor{red}{अन्य\-पदार्थे} इत्यत्र सप्तमी। सा चाधिकरण\-सञ्ज्ञा\-परा। तत्र च विषयता\-रूपाऽधिकरणता। सा च वर्तमानत्व\-रूपा। तस्मादन्य\-पदार्थ\-निष्ठ\-वर्तमानत्व\-विषयता\-परमनेक\-सुबन्तं तदेव परस्परं समस्यत इत्यर्थानुरोधेन प्रथमान्तानामेव समासः। तत्र समानाधिकरणत्वं नाम शब्द\-विशिष्टत्वम्। वैशिष्ट्यं च स्व\-समभि\-व्याहृतत्व\-स्व\-समान\-विभक्तिकत्वमित्येतदुभय\-सम्बन्धेन। तथाऽपि \textcolor{red}{सप्तमी\-विशेषणे बहुव्रीहौ} (पा॰सू॰~२.२.३५) इति सूत्रे सप्तमी\-शब्दस्य विशेषणात्पृथगुपादानेन व्यधिकरण\-बहुव्रीहेः कल्पना। अन्यथा विशेषण\-शब्देनैव समानाधिकरण\-बहुव्रीहेः सङ्ग्रहः किमेतेन सप्तमीपदविन्यासेन। अत्र सप्तमीशब्दस्य पृथक्श्रवणाद्विशेषणं विशेष्य\-समान\-विभक्तिकमेव तत्र प्रथमान्तमेव। तेन \textcolor{red}{पीतमम्बरं यस्य स पीताम्बरः} इत्यादि\-स्थलानि सुसङ्गतानि। पीत\-शब्दो ह्यम्बर\-शब्दस्य विशेषणत्वावच्छेदकतया पूर्वं प्रयुक्तः। यद्यपि \textcolor{red}{भूतले घटः} इत्यादौ सप्तम्यन्तस्यापि विशेषणता। अत एव गदाधर\-भट्टाचार्या व्युत्पत्ति\-वादे \textcolor{red}{वेदाः प्रमाणम्} इत्यस्य स्थलस्य\footnote{\textcolor{red}{स्थल}\-शब्दः प्रसङ्गे विषये भागे चेत्याप्टे\-कोशः।} व्याख्यावसरे विशेषण\-विशेष्य\-पदयोः समान\-विभक्तिकत्व\-नियमं सार्वकालिकं न स्वीचक्रुः। अतस्ते व्युत्पत्तिमिमां प्रणिन्युर्यत् \textcolor{red}{यत्र विशेष्य\-वाचक\-पदोत्तर\-विभक्ति\-तात्पर्य\-विषय\-सङ्ख्या\-विरुद्ध\-सङ्ख्याया अविवक्षितत्वं तत्र विशेष्य\-विशेषण\-वाचक\-पदयोः समान\-वचनकत्व\-नियमः} (व्यु॰वा॰ का॰प्र॰)। तद्यथा \textcolor{red}{सुन्दरो रामः} इत्यत्र विशेष्य\-वाचक\-पदं रामस्तदुत्तर\-विभक्तिः सु\-विभक्तिस्तत्तात्पर्य\-विषय\-सङ्ख्यैकत्व\-सङ्ख्या तद्विरुद्ध\-सङ्ख्यायार्द्वित्वादेः सुन्दर\-पदोत्तर\-सु\-विभक्त्या विवक्षा नास्ति। अतोऽत्र समान\-वचनकत्वम्। किन्तु \textcolor{red}{वेदाः प्रमाणम्} इत्यत्र विशेष्य\-वाचक\-पदं वेदास्तदुत्तर\-विभक्तिर्जस्विभक्तिस्तत्तात्पर्य\-विषय\-सङ्ख्या बहुत्व\-सङ्ख्या तद्विरुद्ध\-सङ्ख्याया एकत्व\-नाम\-सङ्ख्यायाः प्रमाण\-पदोत्तर\-सु\-विभक्त्या विवक्षितत्वमेव। अतोऽत्र विषम\-वचनता। विगत\-सप्तमी\-पदस्य पृथगुपादानादस्मिन्नियमेऽत्र सङ्कोचं चकार भगवान् पाणिनिः। अतः सिद्धान्ते व्यधिकरण\-बहुव्रीहौ \textcolor{red}{कण्ठेकालः} (कण्ठे कालो यस्य सः)\footnote{\textcolor{red}{कण्ठेकालः} इत्यत्र \textcolor{red}{अमूर्ध\-मस्तकात्स्वाङ्गादकामे} (पा॰सू॰~६.३.१२) इति सूत्रेण काशिका\-कौमुदी\-शब्दकल्पद्रुम\-वाचस्पत्यादिषु समासः प्रादर्शि। \textcolor{red}{कण्ठे कालोऽस्य कण्ठेकालः} (का॰वृ॰~६.३.१२) इति काशिका। \textcolor{red}{कण्ठेकालः} (वै॰सि॰कौ॰~९७०) इति कौमुदी। \textcolor{red}{कण्ठे कालः क्षीरोद\-मन्थनोद्भव\-विषपान\-निदर्शन\-रूप\-नील\-वर्णो यस्य। सप्तम्या अलुक्} इति शब्दकल्पद्रुमः। \textcolor{red}{कण्टे कालः कण्ठे कालोऽस्य वा सप्तम्या अलुक्} इति वाचस्पत्यम्। भाष्ये च \textcolor{red}{समानाधिकरण\-समासाद्बहुव्रीहिः} (वा॰~२.१.६९) इति वार्त्तिके \textcolor{red}{बहुव्रीहेरवकाशः~–कण्ठेकालः} इति बहुव्रीहि\-समासमुदाहृत्य \textcolor{red}{सप्तम्युपमान\-पूर्वपदस्योत्तर\-पदलोपश्च} (वा॰~२.२.२४) इति वार्त्तिके \textcolor{red}{कण्ठेस्थः कालोऽस्य कण्ठेकालः} इत्युक्तम्। इत्युभयथा विग्रहः।} \textcolor{red}{पाणौमहा\-सायक\-चारु\-चापम्} (पाणौ महान् सायकश्चारु चापं च यस्य तम्) इत्यादि\-स्थलमिवात्रापि समाधातव्यम्। वेदान्तेऽस्मिन्माया\-ब्रह्म\-जीवानां निरूपणम्। \textcolor{red}{वेदान्तो नामोपनिषत्प्रमाणं तदुपकारीणि शारीरकसूत्रादीनि च} (वे॰सा॰~३) इति वेदान्तसारे। वेदान्तेऽप्यद्वैत\-वादो विशिष्टाद्वैत\-वादः शुद्धाद्वैत\-वादो द्वैताद्वैत\-वादोऽचिन्त्याद्वैत\-वादो द्वैत\-वाद इति मतानि प्रस्तुतान्याचार्यैः। इदमेव दर्शनमुत्तर\-मीमांसा\-शब्देन व्यवह्रियते। \end{sloppypar}
\begin{sloppypar}\hyphenrules{nohyphenation}\justifying\noindent\hspace{10mm} इममेव वेदमधिकृत्याष्टादश पुराणान्युप\-बृंहितानि वेदव्यासेन। तत्रैषां सङ्ग्रह\-श्लोकः~–\end{sloppypar}
\centering\textcolor{red}{मद्वयं भद्वयं चैव ब्रत्रयं वचतुष्टयम्। \nopagebreak\\
अनापकूष्कलिङ्गानि पुराणानि प्रचक्षते॥}\nopagebreak\\
\raggedleft{–~सङ्ग्रहश्लोकः}\footnote{\textcolor{red}{मद्वयं भद्वयं चैव ब्रत्रयं वचतुष्टयम्। अनापलिङ्गकूस्कानि पुराणानि पृथक्पृथक्॥} (दे॰भा॰पु॰~१.३.२)।}\\
\begin{sloppypar}\hyphenrules{nohyphenation}\justifying\noindent इति। आचार्या इति शेषः। यथा मत्स्य\-पुराणं मार्कण्डेय\-पुराणं भविष्योत्तर\-पुराणं भागवत\-पुराणं ब्रह्म\-वैवर्त\-पुराणं ब्रह्माण्ड\-पुराणं ब्रह्म\-पुराणं वाराह\-पुराणं वामन\-पुराणं विष्णु\-पुराणं वायव्य\-पुराणमग्नि\-पुराणं नारद\-पुराणं पद्म\-पुराणं कूर्म\-पुराणं स्कन्द\-पुराणं लिङ्ग\-पुराणं गरुड\-पुराणमित्यष्टादश पुराणानि। तत्र देवी\-भागवतमेषु। श्रीमद्भागवत\-महा\-पुराणं त्वेभ्यः पृथक्। महा\-पुराणत्वात्। इदमेवोन\-विंशं वेद\-व्यास\-देवस्य चरमा रचनेति विपश्चितां मनीषा।\footnote{श्रीमद्भागवतमष्टादश\-पुराणेभ्यो भिन्नं महापुराणं व्यासदेवस्यान्तिमा रचनेति विपश्चितां मतमिति भावः। उपपुराणानि तु नात्र परामृष्टानि।} इतिहासश्च द्वेधा रामायणं महाभारतं चेति। रामायणस्य प्रवर्तक आदि\-कवि\-वाल्मीकिर्यः खलु पुरा लुण्ठकोऽपि सप्तर्षि\-प्रसाद\-संलब्ध\-लालित्य\-पूर्ण\-प्रतिमा\-लवित्र\-सञ्छिन्न\-मोह\-महा\-महीरुहो हृदय\-सरसी\-समुल्लसित\-सीता\-निवास\-चरण\-सरोरुहो मुनि\-व्रत\-स्वीकार\-संवर्धित\-ललाट\-तपस्तपन\-दीधिति\-समिद्ध\-तीव्र\-तर\-ताप\-सहिष्णु\-तपोऽनुष्ठानोप\-योगि\-योगि\-जन\-दुर्लभ\-जटी\-भूत\-शिरोरुहो 
विमल\-वैराग्य\-विहित\-तृणायमान\-कुटुम्ब\-त्याग\-परिपीत\-रामचन्द्र\-पद\-पाथोज\-पराग\-रस\-सरस\-मानस\-कूजित\-मङ्गलमय\-काव्य\-गान\-संस्मृत\-मुकुन्द\-करुणा\-वत्सलत्व\-सौशील्य\-सौलभ्यादि\-मनुजेतर\-गुण\-गण\-विहित\-भूति\-वल्मीक\-वरूथ\-शरीर\-व्रण\-रघुपति\-चिन्तन\-
विचक्षण\-कविकुल\-शरण\-सम्मण्डितारण्य\-वरेण्य\-वागीश्वर\-प्रसाद\-बृंहित\-वाचस्पति\-शेमुषी\-दुर्लभ\-मनीषोत्कर्ष\-चरण\-पङ्कजभावित\-भारत\-वर्ष\-प्रस्तुतोत्कटादर्श\-रघुनाथ\-लीला\-तरलतर\-सुर\-तरङ्गिणी\-निस्यन्दमान\-प्रेम\-सुधापान\-सञ्जन्यमान\-हर्षोत्कर्ष\-पुण्य\-पुलकित\-तनूरुहः श्रीराम\-विपरीत\-नाम\-जप\-प्रभाव\-संलब्ध\-ब्रह्म\-समान\-वैभवोऽपि\footnote{विपरीत\-नाम\-जपोऽध्यात्म\-रामायणे वर्णितः~– \textcolor{red}{इत्युक्त्वा राम ते नाम व्यत्यस्ताक्षरपूर्वकम्। एकाग्रमनसाऽत्रैव मरेति जप सर्वदा॥} (अ॰रा॰~२.६.८०)।} सीता\-पति\-पद\-पद्म\-भक्ति\-मदिरया मदिरेक्षणो नरपिशाचित\-गृहदारादि\-जगतीतल\-भोग\-सम्भारं समुल्लसित\-षड्विकारमनुभूत\-पत्नी\-पुत्र\-भ्रातृ\-कठोर\-तिरस्कारं स्वार्थ\-शृङ्गारं पतितं परिवारं तृणमिव तिरश्चकार। यस्य खलु 
धारित\-तपोव्रतस्य कवि\-कुल\-गुरोः प्राचेतसस्य भगवतो वाल्मीकेः केकोपमा निरुपमा कोविद\-कुल\-मनोरमा गति\-विभ्रमा विलसित\-त्रिविक्रम\-विक्रमा परिष्कृत\-पण्डित\-परिश्रमा भग्नभव\-भयभ्रमा समुल्लसित\-काव्यकला\-कलाप\-सर्वोत्तम\-संस्करण\-सम्भ्रमा श्री\-रामलीला\-सुरधुनी\-विमल\-वीचि\-कीलाल\-क्रीडा\-गीत\-कवि\-कुल\-व्यतिक्रमा मङ्गल\-मयी लोक\-हितैषिणी जगदुप\-कारिणी कविता\-कानन\-स्वच्छन्द\-विहारिणी धूर्जटि\-जटा\-चारिणीव त्रिपथ\-विहारिणी मुनि\-मनोहारिणी पतित\-तारिणी तरणि\-किरण\-मालिकेव निर्मला मनीषा क्रौञ्च\-द्वन्द्व\-निधन\-निरीक्षण\-सञ्जात\-तरुण\-करुण\-शोक\-समालोकित\-जगदीशा प्रथमं काव्य\-सर्गं व्यधत्त करतल\-चतुर्वर्गाप\-वर्ग\-सम्मोदित\-भर्ग\-सुलभीकृत\-स्वर्गं श्लोकमेव विरचित\-सर्गमिव समुत्ससर्ज च। स च श्लोकोऽनुष्टुप्छन्दसा समगीयत~–\end{sloppypar}
\centering\textcolor{red}{मा निषाद प्रतिष्ठां त्वमगमः शाश्वतीः समाः। \nopagebreak\\
यत्क्रौञ्चमिथुनादेकमवधीः काममोहितम्॥}\nopagebreak\\
\raggedleft{–~वा॰रा॰~१.२.१५}\\
\begin{sloppypar}\hyphenrules{nohyphenation}\justifying\noindent\hspace{10mm} अयमेव निखिल\-काव्य\-जगति प्रथमो रचना\-विशेषः। शापात्मकोऽस्यार्थः~– हे \textcolor{red}{निषाद} नीचैः सीदति तिष्ठतीति निषादस्तत्सम्बुद्धौ निषाद। निम्न\-गामिन्निति भावः। \textcolor{red}{त्वं शाश्वतीः समाः} अनन्ताः समा वर्षाणि यावत् \textcolor{red}{प्रतिष्ठां} सम्मानं \textcolor{red}{मा अगमः} मा प्राप्नुहि \textcolor{red}{यत्} यतो हि \textcolor{red}{क्रौञ्च\-मिथुनात्} क्रौञ्च\-युग्मतः \textcolor{red}{काम\-मोहितमेकं} पुरुषमिति भावस्त्वम् \textcolor{red}{अवधीः}। पुरुषं हत्वा शृङ्गार\-रसमपि करुण\-रसे परिवर्तितवानिति भावः। विचारे कृतेऽयं श्लोकः सर्व\-काव्य\-प्राथम्य\-भाक्तयाऽमङ्गलशापोक्तिमत्तया कथं परिणमितः। अतोऽपरा व्याख्या~– मा सीता। \textcolor{red}{इन्दिरा लोक\-माता मा} (अ॰को॰~१.१.२७क) इति कोशात्। सैव \textcolor{red}{मा सीता नितरां सीदति तिष्ठति यस्मिन् स मानिषादः}। अधिकरणे घञ्।\footnote{\textcolor{red}{नितरां सीदति यस्मिन्स निषादः}। निपूर्वकात् \textcolor{red}{सद्‌}\-धातोः (\textcolor{red}{षद्ऌँ विशरण\-गत्यवसादनेषु} धा॰पा॰~८५४, १४२७) \textcolor{red}{हलश्च} (पा॰सू॰~३.३.१२१) इत्यनेनाधिकरणे घञि \textcolor{red}{धात्वादेः षः सः} (पा॰सू॰~६.१.६४) इत्यनेन सत्वे \textcolor{red}{अत उपधायाः} (पा॰सू॰~७.२.११६) इत्यनेन वृद्धौ \textcolor{red}{सदिरप्रतेः} (पा॰सू॰~८.३.६६) इत्यनेन षत्वे विभक्तिकार्ये \textcolor{red}{निषादः}। \textcolor{red}{माया निषाद इति मानिषादस्तत्सम्बुद्धौ मानिषाद}।}
अथवा \textcolor{red}{मायां सीतायां नितरां सीदति तिष्ठति यः स मानिषादः}। कर्तरि घञ्।\footnote{\textcolor{red}{नितरां सीदति निषीदतीति निषादः}। सञ्ज्ञायां बाहुलकात्कर्तरि घञ्। प्रक्रिया पूर्ववत्। \textcolor{red}{मायां निषाद इति मानिषादस्तत्सम्बुद्धौ मानिषाद}।} श्रीरामचन्द्र इति भावः। अर्थात्~– हे \textcolor{red}{मानिषाद} सीतानिवास श्रीराम \textcolor{red}{त्वं} भवान् \textcolor{red}{शाश्वतीः समाः} अनन्त\-वर्षाणि यावत् \textcolor{red}{प्रतिष्ठां} सम्मानं पूजाम् \textcolor{red}{अगमः} प्राप्तवान् \textcolor{red}{यत्} यतो हि \textcolor{red}{क्रौञ्च\-मिथुनात्} क्रौञ्चयोः पक्षिवद्रावण\-मन्दोदर्यो\-र्मिथुनाद्युग्मात् \textcolor{red}{काम\-मोहितं} कामिनम् \textcolor{red}{एकं} रावणम् \textcolor{red}{अवधीः} हतवान्। रावणं हत्वा भवता महती प्रतिष्ठा समर्जितेति श्लोकस्य भावः। अयमेव सर्व\-प्रथमः श्लोकोऽनन्तरं नारदोपदिष्ट\-सङ्क्षिप्त\-सीताराम\-कथां विगतव्यथां भगवान् वाल्मीकिरादि\-काव्यरूपां पञ्चशतसर्गात्मिकां षट्काण्डां रामायण\-नामधेयां चतुर्विंशति\-साहस्री\-संहितामिमां\footnote{\textcolor{red}{चतुर्विंशत्सहस्राणि श्लोकानामुक्तवानृषिः। तथा सर्गशतान्पञ्च षट् काण्डानि तथोत्तरम्॥} (वा॰रा॰~१.४.२)।} काव्यवसन्त\-कोकिलः कोकिल\-काकलीमिव चुकूज। इदमितिहास\-भूतं वेद\-सम्मतम्। तत्रेत्थं ज्ञातव्यम्। यथा विज्ञाने सम्प्रति \textcolor{red}{साईन्स्‌}\-नामधेये योग\-प्रयोगौ \textcolor{red}{थिअरी\-प्रैक्टिकल्}\-शब्दाभ्यां व्यवह्रियेते। तथैव वेदः \textcolor{red}{थिअरी} इति। अर्थात्सिद्धान्त\-सङ्ग्रहः। रामायणं \textcolor{red}{प्रैक्टिकल्} इति। अर्थात्प्रयोग\-सङ्ग्रहः। अत्र वेदस्य सम्पूर्णाः सिद्धान्ता अनेक\-कथा\-व्याजेन सविस्तरं प्रतिपादिताः। अयमेवेतिहासः। इतिहास\-शब्दो हि त्रिभिः शब्दैर्निष्पद्यते \textcolor{red}{इति ह आस}। \textcolor{red}{इति}\-शब्दः पूर्वकाल\-घटना\-क्रम\-सूचको \textcolor{red}{ह}\-शब्दो निश्चयवाचक \textcolor{red}{आस}\-शब्दो भूतकाल\-क्रिया\-सूचकः। एवमिदं निश्चितमासीदित्येवमितिहास\-शब्दार्थः। अस्य सम्पूर्णतो वेदमूलतां स्वयं वाल्मीकिरकथयत्~–\end{sloppypar}
\centering\textcolor{red}{इदं पवित्रं पापघ्नं पुण्यं वेदैश्च सम्मितम्।\nopagebreak\\
यः पठेद्रामचरितं सर्वपापैः प्रमुच्यते॥}\nopagebreak\\
\raggedleft{–~वा॰रा॰~१.१.९८}\\
\begin{sloppypar}\hyphenrules{nohyphenation}\justifying\noindent\hspace{10mm} द्वितीय इतिहास\-ग्रन्थो महाभारत\-नामधेयः। अयमष्टादश\-पर्वयुक्तो लक्षश्लोकात्मकः। अस्य रचयिता श्रीमन्नारायण\-कलावतारः साक्षाद्भगवान् वेदव्यासः। यथा रामायणे महर्षिणा वाल्मीकिना परब्रह्मणः सनातस्य मर्यादा\-पुरुषोत्तमस्य चारु\-चरित्रं सजीव\-चित्रमिव चित्रितं तस्य रामो महाविष्णुः परं ब्रह्म सुरकार्य\-चिकीर्षया दशरथ\-पुत्रतां स्वीकृत्य कौसल्यायां प्रकटयाम्बभूव यथा~–\end{sloppypar}
\centering\textcolor{red}{स हि देवैरुदीर्णस्य रावणस्य वधार्थिभिः। 	\nopagebreak\\
अर्थितो मानुषे लोके जज्ञे विष्णुः सनातनः॥}\nopagebreak\\
\raggedleft{–~वा॰रा॰~२.१.७}\\
\begin{sloppypar}\hyphenrules{nohyphenation}\justifying\noindent\hspace{10mm} तथैव महाभारतेऽपि परम\-भागवतस्य पाण्डु\-वंशस्य वर्णनच्छलेनातसी\-कुसुमोपमेय\-कान्ते रुक्मिणी\-रमणस्य भगवतो गृहीत\-नराकारस्य पार्थ\-सूतस्य वासुदेवस्य श्री\-कृष्ण\-चन्द्रस्य भक्त\-वश्यतोप\-बृंहिता वेदव्यासेन। अत्र भगवान् कृपण\-वत्सलो व्यासः स्त्री\-शूद्र\-द्विज\-बन्धूनां कृतेऽपि वेदार्थो यथा सरलः स्यादिति हेतोः
समस्त\-वेदे वर्णित\-विषयाणां कथोदाहरण\-प्रसङ्गेषु निबन्धनमकार्षीत्। अतः श्रूयते \textcolor{red}{यन्न भारते तन्न भारते} इति। श्रीमद्भागवतेऽपि महाभारत\-रचना\-कारणमिदमेव प्रादर्शि	~–\end{sloppypar}
\centering\textcolor{red}{स्त्रीशूद्रद्विजबन्धूनां त्रयी न श्रुतिगोचरा। \nopagebreak\\
कर्मश्रेयसि मूढानां श्रेय एवं भवेदिह। \nopagebreak\\
इति भारतमाख्यानं कृपया मुनिना कृतम्॥}\nopagebreak\\
\raggedleft{–~भा॰पु॰~१.४.२५}\\
\begin{sloppypar}\hyphenrules{nohyphenation}\justifying\noindent\hspace{10mm} एवं निगूढा वेदार्था इतिहास\-पुराणाभ्यामुप\-बृंह्यन्ते। स्वयमेव वेदव्यासः कथयति~–\end{sloppypar}
\centering\textcolor{red}{इतिहासपुराणाभ्यां वेदार्थमुपबृंहयेत्। \nopagebreak\\
बिभेत्यल्पश्रुताद्वेदो मामसौ प्रहरेदिति॥}\nopagebreak\\
\raggedleft{–~म॰भा॰~१.१.२६७, १.१.२९३, १.१.२९४}\\
\begin{sloppypar}\hyphenrules{nohyphenation}\justifying\noindent\hspace{10mm} इममेव वेदमाधारी\-कृत्य प्रवर्तिताश्चतुर्दश विद्याः। तत्र षड्दर्शनान्यान्वीक्षिकी त्रयी वार्ता दण्डनीतिः पुराणं मीमांसा धर्मशास्त्रं सङ्गीतं चेति। तत्र षड्दर्शनेषु साङ्ख्यं योगो वैशेषिकं न्यायः पूर्व\-मीमांसोत्तर\-मीमांसा चेति। \textcolor{red}{दर्शनं नाम दृश्यते ज्ञायते परमात्म\-तत्त्वं येन तत्} इति व्युत्पत्तौ \textcolor{red}{दृश्‌}\-धातोः (\textcolor{red}{दृशिँर् प्रेक्षणे} धा॰पा॰~९८८) \textcolor{red}{करणाधि\-करणयोश्च} (पा॰सू॰~३.३.११७) इति सूत्रेण ल्युट्। टकारस्येत्सञ्ज्ञा \textcolor{red}{हलन्त्यम्} (पा॰सू॰~१.३.३) इति सूत्रेण \textcolor{red}{तस्य लोपः} (पा॰सू॰~१.३.९) इत्यनेन लोपश्च। लकारस्य \textcolor{red}{लशक्वतद्धिते} (पा॰सू॰~१.३.८) इति सूत्रेणेत्सञ्ज्ञा पुनर्लोपः।\footnote{\textcolor{red}{दृशिँर्} इत्यत्र \textcolor{red}{इर उपसङ्ख्यानम्} (वा॰~१.३.७) इत्यनेनेर इत्सञ्ज्ञा \textcolor{red}{तस्य लोपः} (पा॰सू॰~१.३.९) इत्यनेन लोपश्च बोध्यः।} उकारस्य विधान\-सामर्थ्यादुप\-देशानु\-नासिकत्वेऽपि नेत्त्वम्। लकारस्य लित्स्वरार्थः प्रयोगः।\footnote{\textcolor{red}{लिति} (पा॰सू॰~६.१.१९३) इत्यनेन लित्स्वरे प्रत्ययात्पूर्वः स्वर उदात्तः। दर्शनम्~\arrow \textcolor{red}{लिति} (पा॰सू॰~६.१.१९३)~\arrow \textcolor{red}{अनुदात्तौ सुप्पितौ} (पा॰सू॰~३.१.४)~\arrow \textcolor{red}{अनुदात्तं पदमेकवर्जम्‌} (पा॰सू॰~६.१.१५८)~\arrow दर्श॒न॒म्~\arrow \textcolor{red}{उदात्तादनुदात्तस्य स्वरितः} (पा॰सू॰~८.४.६६)~\arrow दर्श॑न॒म्~\arrow \textcolor{red}{स्वरितात्संहितायामनु\-दात्तानाम्} (पा॰सू॰~१.२.३९)~\arrow दर्श॑नम्। ऋग्वेद\-संहितायां च \textcolor{red}{दर्श॑नाय} (ऋ॰वे॰सं॰~१.११६.२३) इति। न च \textcolor{red}{उदात्तादनुदात्तस्य स्वरितः} (पा॰सू॰~८.४.६६) इत्यस्य त्रिपादीस्थत्वात् \textcolor{red}{स्वरितात्संहितायामनु\-दात्तानाम्} (पा॰सू॰~१.२.३९) इत्यस्य सपाद\-सप्ताध्यायी\-सूत्रस्यासिद्धे स्वरितकार्ये कथं प्रवृत्तिरिति चेत्। विधान\-सामर्थ्यात्त्रिपादी\-सूत्रकार्यं सिद्धम्। यद्वा \textcolor{red}{स्वरितस्यार्ध\-ह्रस्वोदात्तादोदात्त\-स्वरित\-परस्य सन्नतरादूर्ध्वमुदात्तादनुदात्तस्य स्वरितात्कार्यं स्वरितादिति सिद्ध्यर्थम्} (पा॰सू॰~१.२.३२) इति वार्त्तिक\-वचनात्सिद्धम्। यद्वा \textcolor{red}{न मु ने} (पा॰सू॰~८.२.३) इत्यत्र \textcolor{red}{न} इति योगविभागात् \textcolor{red}{पूर्वत्रासिद्धम्} (पा॰सू॰~८.२.१) इति सूत्रस्याप्रवृत्तिः क्वाचित्का। अन्तिम\-पक्षस्य विस्तराय \pageref{sec:jaayeti_siiteti}तमे पृष्ठे \ref{sec:jaayeti_siiteti} \nameref{sec:jaayeti_siiteti} इति प्रयोगस्य विमर्शं पश्यन्तु।} तस्मिंश्चादृष्ट\-जनकत्वरूपं फलम्। ततो \textcolor{red}{युवोरनाकौ} (पा॰सू॰~७.१.१) इत्यनेनानादेशे गुणे\footnote{\textcolor{red}{पुगन्त\-लघूपधस्य च} (पा॰सू॰~७.३.८६) इत्यनेन।} रपरत्वे\footnote{\textcolor{red}{उरण् रपरः} (पा॰सू॰~१.१.५१) इत्यनेन।} कृते \textcolor{red}{दर्शनम्} इति सिद्धम्। तत्रास्तिक\-दर्शनानि षट्। \textcolor{red}{आस्तिको नाम वेद\-प्रतिपादक\-वेदप्रतिपाद्य\-मान्यतास्वास्थावान्}। \textcolor{red}{अस्ति\-नास्ति\-दिष्टं मतिः} (पा॰सू॰~४.४.६०) इति सूत्रेण \textcolor{red}{अस्ति}\-शब्दात् \textcolor{red}{ठक्‌}\-प्रत्ययः। विधान\-सामर्थ्यात् \textcolor{red}{चूटू} (पा॰सू॰~१.३.७) इत्यनेन ठकारस्य नेत्त्वं कस्येत्सञ्ज्ञा\-लोपयोः ठस्येकादेशे \textcolor{red}{ठस्येकः} (पा॰सू॰~७.३.५०) इति सूत्रेण \textcolor{red}{यचि भम्} (पा॰सू॰~१.४.१८) इत्यनेन भसञ्ज्ञायां \textcolor{red}{यस्येति च} (पा॰सू॰~६.४.१४८) इत्यनेनास्तिघटकेकारलोपे \textcolor{red}{तद्धितेष्वचामादेः} (पा॰सू॰~७.२.११७) इत्यनेन वृद्धौ विभक्ति\-कार्ये \textcolor{red}{आस्तिकः} इति सिद्धम्। अर्थाद्वेद\-शास्त्रेतिहास\-पुराण\-स्मृति\-धर्मशास्त्रे सिद्धान्ते श्रद्धधानत्वे 
सतीश्वर\-पूजकत्वे सति गो\-विप्र\-प्रतिमा\-पूजकत्वमास्तिकत्वम्। अत एव व्यास आस्तिकानुत्साहयति~–\end{sloppypar}
\centering\textcolor{red}{सन्दिग्धेऽपि परे लोके कर्तव्यो धर्मसङ्ग्रहः। \nopagebreak\\
नास्ति चेदस्ति का हानिरस्ति चेन्नास्तिको हतः॥}\footnote{मूलं भारते शान्ति\-पर्वणि मृग्यम्।}\nopagebreak\\
\begin{sloppypar}\hyphenrules{nohyphenation}\justifying\noindent इति महाभारते शान्तिपर्वणि। तत्र साङ्ख्य\-दर्शन\-प्रवर्तको भगवदंशावतारः साक्षाद्भगवान् कपिलः। अत्र त्रीणि तत्त्वानि व्यक्तमव्यक्तं ज्ञ इति। व्यक्तं 
त्रयोविंशति\-तत्त्व\-निकरः। महदहङ्कारो दशेन्द्रियाणि चक्षुः\-श्रोत्र\-रसना\-घ्राण\-त्वक्पाणि\-पाद\-पायूपस्थ\-वागाख्यानि मनः शब्द\-स्पर्श\-रूप\-रस\-गन्धाः क्षिति\-जल\-पावक\-गगन\-समीरा इति त्रयोविंशति\-तत्त्वानि व्यक्तानि। अव्यक्तं प्रकृतिः पुरुषश्च। ज्ञः षड्विंशो हि पुरुषः। इदं तावत्साङ्ख्ये \textcolor{red}{सम्यक्ख्यायन्ते गण्यन्ते योद्धारो यस्मिन् तादृशे दुःख\-त्रयाभिघात\-रूपे युद्धे सङ्ख्ये प्रवर्तितमिति साङ्ख्यम्}। सङ्ख्यं बहुशो युद्धादौ प्रयुक्तं यथा \textcolor{red}{एवमुक्त्वाऽर्जुनः सङ्ख्ये} (भ॰गी॰~१.४७) इति।\footnote{\textcolor{red}{युद्धमायोधनं जन्यं प्रधनं प्रविदारणम्॥ मृधमास्कन्दनं सङ्ख्यं समीकं साम्परायिकम्।} (अ॰को॰~२.८.१०३-१०४) इत्यमरः।} आधिभौतिकाधिदैविकाध्यात्मिक\-नामधेयानां त्रयाणां दुःखानां प्रतिकूलतयाऽऽत्मनि व्याघातात्तदभिघातक\-हेतु\-विषयक\-जिज्ञासायां व्यक्ताव्यक्त\-ज्ञ\-ज्ञान\-दीपकरूपमिदं प्राचीनतमम्। तद्यथा~–\end{sloppypar}
\centering\textcolor{red}{दुःखत्रयाभिघाताज्जिज्ञासा तदपघातके हेतौ। \nopagebreak\\
दृष्टे साऽपार्था चेन्नैकान्तात्यन्ततोऽभावात्॥}\nopagebreak\\
\raggedleft{–~सा॰का॰~१}\\
\begin{sloppypar}\hyphenrules{nohyphenation}\justifying\noindent\hspace{10mm} अत्र प्रकृति\-पुरुष\-संयोगेन सृष्टिः। तयोः संयोगस्तत्र पङ्ग्वन्धवत्संयोगः। अत्र धाराद्वयी। केचिन्निरीश्वरवादं केचिच्च सेश्वरवादं व्यवस्थापयन्ति। इदं दर्शनं द्वैतवाद\-परम्। अत्र पञ्चविंशतितत्त्वानां सङ्ग्रहः। प्रकृतिः कर्त्री पुरुषः पुष्कर\-पलाशवन्निर्लेपः। पुरुषेण संयुक्तायां प्रकृतौ पुरुष\-निष्ठ\-चेतनत्वमारोप्यते। पुरुषे च प्रकृति\-गत\-कर्तृत्वमध्यस्यते। तथा चोक्तम्~–\end{sloppypar}
\centering\textcolor{red}{मूलप्रकृतिरविकृतिर्महदाद्याः प्रकृतिविकृतयः सप्त। \nopagebreak\\
षोडशकस्तु विकारो न प्रकृतिर्न विकृतिः पुरुषः॥}\nopagebreak\\
\raggedleft{–~सा॰का॰~३}\\
\begin{sloppypar}\hyphenrules{nohyphenation}\justifying\noindent\hspace{10mm} अस्मिन्दर्शने त्रीण्येव प्रमाणानि स्वीचक्रिरे दृष्टमनु\-मानमाप्त\-वचनं च।\footnote{\textcolor{red}{दृष्टमनु\-मानमाप्त\-वचनं च सर्व\-प्रमाण\-सिद्धत्वात्। त्रिविधं प्रमाणमिष्टं प्रमेय\-सिद्धिः प्रमाणाद्धि॥} (सा॰का॰~४)।} अनुमानमपि त्रिविधं पूर्ववच्छेष\-वत्सामान्यतो\-दृष्ट\-भेदेन।\footnote{\textcolor{red}{त्रिविधमनु\-मानमाख्यातम्} (सा॰का॰~५)। \textcolor{red}{त्रिविधमनु\-मानमाख्यातम्। पूर्ववच्छेषवत्सामान्यतो\-दृष्टं चेति} (सा॰का॰ गौ॰भा॰~५)। न्यायशास्त्रेऽपि~– \textcolor{red}{अथ तत्पूर्वकं त्रिविधमनुमानं पूर्ववच्छेषवत्सामान्यतो दृष्टं च} (न्या॰सू॰~१.१.५)।} यत्र हेतुं दृष्ट्वा व्याप्तिस्मरण\-पुरःसरं भूतपूर्वं कार्यमनुमीयते तत्र पूर्ववद्यथा जलसङ्कुलं जलाशयं दृष्ट्वा पूर्ववर्षाऽनुमिता।\footnote{वर्षर्त्वर्थे वर्षा\-शब्दो बहुवचनान्तो वृष्ट्यर्थे चैकवचनान्त इत्याप्टे\-कोशः।} यत्र शेषं दृष्ट्वा कार्यमनुमीयते तत्र शेषवद्यथा बिन्दुमात्रं समुद्रजलं निपीय शेषं क्षारमनुमीयते। यत्र सामान्य\-परिस्थित्या विशेषस्यानुमानं तत्र सामान्यतो\-दृष्टम्। यथा कुत्रचिदाम्रफलानां दर्शनेनेतरत्र रसालफलत्वानुमानम्। शास्त्रेऽस्मिन्नात्मैव पुरुष\-शब्देन व्यवह्रियते। तथा च \textcolor{red}{पुरि शरीरे शेते साक्षित्वेन तिष्ठति यः स पुरुष आत्मा}।\footnote{पृषोदरादित्वात्साधुः। \textcolor{red}{पुरि देहे शेते शी–ड पृषो॰} इति वाचस्पत्यकारः। भागवते च~– \textcolor{red}{पुराण्येन सृष्टानि नृतिर्यगृषिदेवताः। शेते जीवेन रूपेण पुरेषु पुरुषो ह्यसौ॥} (भा॰पु॰~७.१४.३७)।} साङ्ख्याः पुरुषबहुत्वमपि साधयन्ति।\footnote{\textcolor{red}{जनन\-मरण\-करणानां प्रति\-नियमादयुगपत्प्रवृत्तेश्च। पुरुष\-बहुत्वं सिद्धं त्रैगुण्य\-विपर्ययाच्चैव॥} (सा॰का॰~१८)।} यतो ह्येकस्मिन् सति पुरुषे सर्वेषां सहैव युगपज्जन्ममरणे स्याताम्।\footnote{\textcolor{red}{यद्येक एवाऽत्मा स्यात्तत एकस्य जन्मनि सर्व एव जायेरन्नेकस्य मरणे सर्वेऽपि म्रियेरन्} (सा॰का॰ गौ॰भा॰~१८)।} यथैका विद्युद्यदा गच्छति तदा सहैव सर्वेषु कक्षेष्वन्धकारो भवति तदागमने युगपदेव सकल\-धामसु सुप्रकाशः सहैव विद्युद्व्यजन\-वीजनम्। तस्मात्पुरुषाणां बहुत्वम्। एकत्वे सति जनानां सुख\-दुःख\-प्रभृतीनां स्वभावस्य च वैषम्यं कथमपि न सङ्गच्छेत। राद्धान्तोऽयं समीचीनो युक्तियुक्तो वैष्णव\-सम्मतश्च। अन्यथा यदि सर्वेष्वेक एवाऽत्मा तर्हि कथमेकस्मिन्नेव क्षणे कोऽपि म्रियते कोऽपि जायते कोऽपि तिष्ठति कोऽपि वर्धते कश्चन विपरिणमति कश्चिद्ध्रसति। कथं वा कोऽपि विद्वान् कश्चिन्मूर्खः केचन खलाः केचित्साधवः। अस्तु। इदं शास्त्रं चापि वेदमूलकमेव। पुरुषबहुत्वे श्रुतिरपि प्रमाणं यथा \textcolor{red}{इन्द्रो॑ मा॒याभि॑ पुरु॒रूप॑ ईयते} (ऋ॰वे॰सं॰~६.४७.१८)।\end{sloppypar}
\begin{sloppypar}\hyphenrules{nohyphenation}\justifying\noindent\hspace{10mm} योगश्च चित्त\-वृत्ति\-निरोधपरः। यथा योगसूत्रे \textcolor{red}{योगश्चित्तवृत्तिनिरोधः} (यो॰सू॰~१.२)। चित्त\-वृत्ति\-निरोधायाष्टाङ्गयोगस्य चर्चा। यम\-नियमासन\-प्राणायाम\-प्रत्याहार\-धारणा\-ध्यान\-समाधयोऽष्टाङ्गानि। एतैश्चित्तवृत्तिर्निरुध्यते। अयं च हठयोग इति कथ्यते। राजयोगस्त्वीश्वर\-प्रणिधानात्मकः। यथा योग\-सूत्रे \textcolor{red}{ईश्वरप्रणिधानाद्वा} (यो॰सू॰~१.२३)। हठ\-योगे प्राण\-जयेन मनो जयन्ति योगिनः। राजयोगे च मनोजय\-द्वारेण प्राणमतिशेरते संयमिनः। अस्याऽचार्यो भगवान् पतञ्जलिः। स एव पाणिनि\-व्याकरण\-महाभाष्यकारः। स एव चाऽयुर्वेदे चरक\-संहिता\-रचयितेति श्रूयते गुरुभ्यः। तथा च सङ्कीर्तयन्ति गुरुचरणाः~–\end{sloppypar}
\centering\textcolor{red}{योगेन चित्तस्य पदेन वाचां मलं शरीरस्य च वैद्यकेन।\nopagebreak\\
योऽपाकरोत्तं प्रवरं मुनीनां पतञ्जलिं प्राञ्जलिरानतोऽस्मि॥}\footnote{मूलं मृग्यम्।}\\
\begin{sloppypar}\hyphenrules{nohyphenation}\justifying\noindent अर्थाद्योगसूत्रं 
निर्माय चित्तमलममलितवन्तं व्याकरण\-शास्त्रे पद\-पदार्थ\-प्रतिपादक\-परम\-प्रमाणभूत\-महाभाष्य\-सागर\-संरचनया वाणीं निर्दोषयन्तं चरकसंहिता\-प्रतिपादनेन शरीरं भूषयन्तं प्राञ्जलिः पतञ्जलिं प्रणिपतामीति तात्पर्यम्। \label{text:patanjali} शब्दस्यास्य व्युत्पत्ति\-प्रकारश्च \textcolor{red}{अञ्जलौ पतन्} इति विग्रहे सप्तम्यन्ताञ्जलिशब्दस्य शतृ\-प्रत्ययान्त\-प्रथमान्त\-\textcolor{red}{पतन्‌}\-शब्देन मयूर\-व्यंसकादित्वात्समासः\footnote{\textcolor{red}{मयूर\-व्यंसकादयश्च} (पा॰सू॰~२.१.७२) इत्यनेन।} \textcolor{red}{पतन्‌}\-शब्दस्य च पूर्वनिपातः।\footnote{\textcolor{red}{पतन्तः अञ्जलयोऽस्मिन् नमस्कार्यत्वादिति पतञ्जलिः} (त॰बो॰~७९)। \textcolor{red}{पतञ्जलिरिति। पतन् अञ्जलिर्यस्मिन् नमस्कार्यत्वादिति विग्रहः। अत्र ‘अत्’ इति टेरकारस्य च स्थाने पररूपमकारः। केचित्तु गोर्नदाख्य\-देशे कस्यचिदृषेस्सन्ध्योपासन\-समयेऽञ्जलेर्निर्गत इत्यैतिह्यात् ‘अञ्जलेः पतन्’ इति विगृह्णन्ति। मयूर\-व्यंसकादित्वात्समासः} (बा॰म॰~७९)।} एवं शकन्ध्वादि\-गणमाकृति\-गणं मत्वा \textcolor{red}{शकन्ध्वादिषु पररूपं वाच्यम्} (वा॰~६.१.९४) इति वार्त्तिकेन पररूपे विभक्तिकार्ये \textcolor{red}{पतञ्जलिः}। इत्थमाख्यायिका श्रुता गुरुमुखेभ्यो यदीसा\-पूर्वान्तिम\-शताब्द्यां पाणिनि\-सूत्रार्थ\-परिश्चिकीर्षयाऽशेष\-विशेषातीतो भगवाञ्छेष एकस्यचित्तपो\-निष्ठ\-ब्राह्मणस्य सन्ध्यार्थं गतस्य सरोवराभ्यासमञ्जलिं प्रसार्य सहस्र\-रश्मि\-मालिनं परम\-प्रकाश\-शालिनं कमलिनी\-कुल\-वल्लभं भगवत्साकाररूपं दिव्य\-रोचिषा विभासित\-भुवन\-मण्डलं दीप्तिमय\-मण्डलं जगदाखण्डलं भास्वन्तं विवस्वन्तं निखिल\-जगदुपादान\-भूतं शास्त्र\-स्व\-स्यन्दनं कश्यप\-नन्दनमदिति\-कीर्ति\-केतुं वैदिक\-धर्म\-सेतुमलौकिक\-तेजसं प्रशान्तं भगवन्तं भास्करं समुप\-तिष्ठमानस्य समपतत्सर्पशावकी\-भूयाञ्जलि\-पुटे। आकस्मिक\-करपुट\-भोगि\-तोक\-पतन\-सञ्जात\-प्रबल\-कुतूहलतया द्विजन्मना \textcolor{red}{कोर्भवान्} इति पृष्टः सन् \textcolor{red}{सप्पोऽहम्} इति समुदतीतरत्परिकलित\-सकल\-विद्याविशेषो भगवाञ्छेषः। सर्पघटकरेफः कुत्रेति प्रतिपृष्टः सन् \textcolor{red}{को भवान् कोर्भवान्} इति निरर्थकं मध्येरेफं
व्यवहरता भवता पूर्वमेवोक्त इति वयङ्ग्य\-कटाक्षं समार्पिपत्। स एव सर्प\-शावकः शाब्दिक\-सम्प्रदाय\-नभोमण्डल\-समागत\-पाणिनि\-सूत्रार्थ\-ज्ञान\-तरुण\-तल\-तिमिर\-पटल\-पाटन\-प्रबल\-प्रभञ्जनोपमो निखिल\-पण्डित\-मनोरमा\-परिणत\-परिश्रमो विगत\-भ्रमः प्रस्तुताष्टाध्यायी\-मौलिक\-पाठक्रमः पण्डित\-जगदुद्धार\-चिकीर्षया विप्रपुत्रतामङ्गीचकार। स एव भगवान् पतञ्जलिः पश्चाद्योगिजन\-सेवितं दुराराध्यं सकल\-काम\-कल्याण\-कल्पद्रुमं स्वर्गापवर्ग\-चतुर्वर्ग\-सोपानं साधना\-जाटिल्य\-कटु\-कण्टक\-विषमं सुदुर्गमं योगि\-वर्त्म योग\-दर्शन\-दीपकेन परिश्चकार। इदं दर्शनमपि द्वैतवादपरमीश्वरवाद\-प्रतिपादकमास्तिकं वेदमूलकञ्च।\end{sloppypar}
\begin{sloppypar}\hyphenrules{nohyphenation}\justifying\noindent\hspace{10mm} वैशेषिकं दर्शनं तावत्सप्त\-पदार्थात्मकम्। अत्र द्रव्य\-गुण\-कर्म\-सामान्य\-विशेष\-समवायाभाव\-नामधेय\-सप्त\-पदार्थानां ज्ञानान्मोक्षः। एतस्याऽचार्यः काणादः। इदं सकल\-शास्त्रोपकारकम्। तथा च गुरु\-चरणाः प्राहुर्यत् \textcolor{red}{काणादं पाणिनीयञ्च सर्वशास्त्रोपकारकम्} इति।\footnote{मूलं मृग्यम्।} इदमपि वेद\-मूलकमास्तिक\-दर्शनमीश्वर\-वाद\-परम्। न्याय\-दर्शनं तावत्प्रमाण\-प्रमेयादि\-षोडश\-वस्तु\-मीमांसापरम्। अत्र षोडश\-तत्त्व\-ज्ञानादेव मोक्षः। न्यायदर्शनस्याऽचार्यो भगवान् गौतमः। इदमपीश्वरं कर्तृत्वेन स्वीकरोति। तद्यथा यद्यत्कार्यं तत्तत्सकर्तृकमिति व्याप्तिं स्मरन्तो नैयायिका इत्थमनुमान्ति यत्क्षित्यङ्कुरादिकं कर्तृजन्यं कार्यत्वाद्घटादिवत्। इदं तु परमास्तिक\-दर्शनम्। एतस्य प्राचीनाचार्या मिथिला\-भुवः शेखरीभूताः श्रीमदुदयनाचार्या ईश्वर\-सिद्धावेव न्याय\-सिद्धान्त\-कुसुमाञ्जलि\-नामक\-ग्रन्थं प्रतुष्टुवुः। न्याय\-शास्त्रं परम\-नास्तिक\-विद्या\-मद\-मत्त\-गजेन्द्र\-गण्ड\-स्थल\-भेदन\-शीलं तरल\-तर्क\-तीक्ष्ण\-नख\-युक्त\-सिंहोपमम्। बौद्धादीनां खण्डनायेदमेव धारित\-व्यसनम्। आकर्ण्यते प्राचीनेभ्यो गुरुचरणेभ्यो यत्कदाचित्तीर्थ\-यात्रा\-व्यपदेशेन चरण\-सरोज\-रजः\-सनाथितोत्कल\-प्रदेशाः स्वीकृत\-शास्त्रार्थ\-व्यसन\-रत\-पण्डितेन्द्र\-वेशाः श्रीमदुदयनाचार्य\-चरणाः शरीर\-सन्निहित\-नास्तिक\-कृतेश्वर\-खण्डन\-सन्ताप\-गम्भीर\-व्रणाः निम्नगा\-वल्लभ\-तरस्वि\-तरल\-तरङ्ग\-सङ्क्षालित\-पाद\-कमलाममलां जगन्नाथपुरीं समलञ्चक्रुः। तत्र 
श्रीमज्जगन्नाथ\-भुवन\-पावन\-पादारविन्द\-निस्स्यन्द\-प्रेम\-मकरन्द\-रोलम्बायमान\-मानसतया
भगवन्मुख\-मृगाङ्क\-सौन्दर्य\-सुधा\-माधुरीं पिपासवो जिज्ञासवश्च तत्पिहित\-द्वारोद्घाटन\-कालमर्चकाद्विलम्ब\-श्रवण\-सञ्जात\-रोष\-ज्वाला\-जालं नाल्पं धैर्यं धरन्तो व्याहार्षुर्व्यङ्ग्य\-मिश्रं श्लोकमिमम्। यत्~–\end{sloppypar}
\centering\textcolor{red}{ऐश्वर्यमदमत्तोऽसि मामनादृत्य तिष्ठसि।\nopagebreak\\
उपस्थितेषु बौद्धेषु मदधीना तव स्थितिः॥}\nopagebreak\\
\raggedleft{–~मुक्तकम्}\\
\begin{sloppypar}\hyphenrules{nohyphenation}\justifying\noindent\hspace{10mm} आकर्ण्य तत्प्रेम\-विह्वलां वाणीं पण्डित\-मचर्चिकाया भक्त\-वत्सलो भगवानकालमुद्घाट्य द्वारमुदयनमुदय\-गिरि\-रविमिवाऽत्मानं दर्शयामास। अस्मिन्न्याये सम्प्रदाय\-द्वयं प्रावर्तत। प्राचीन\-न्यायस्य नव्य\-न्यायस्य च। प्राचीन\-न्याय\-प्रवर्तका उदयनाचार्य\-प्रमुख\-मैथिलाः। नव्य\-न्यायाचार्या गङ्गेशोपाध्याय\-गोकुलचन्द्र\-रघुनाथशिरोमणि\-विश्वनाथ\-तर्कपञ्चानन\-गदाधर\-भट्टाचार्य\-प्रभृतयो मैथिलबङ्गीयाः।\end{sloppypar}
\begin{sloppypar}\hyphenrules{nohyphenation}\justifying\noindent\hspace{10mm} तत्र पूर्व\-मीमांसा\-दर्शनं कर्म\-काण्ड\-प्रतिपादकम्। इदं वेदस्य प्रयोग\-पद्धति\-परिष्कारं मीमांसते। एतस्याऽचार्या जैमिनि\-महाभागाः। मीमांसा\-सूत्रं द्वादशाध्याय\-परम्। अस्मिन् धर्म\-जिज्ञासैव मीमांसिता। यथा \textcolor{red}{अथातो धर्मजिज्ञासा} (मी॰सू॰~१.१.१)। इदं तु सर्वभावेन वेदमाश्रयति। अत्रापि द्वौ सम्प्रदायौ भाट्टः प्राभाकरश्च। एतन्मतेऽपौरुषेयवेद एव निखिल\-लोक\-नियन्ता। तत्र कर्मैव ब्रह्मत्वेन प्रतिपादितम्।\footnote{\textcolor{red}{यं शैवाः समुपासते शिव इति ब्रह्मेति वेदान्तिनो बौद्धा बुद्ध इति प्रमाणपटवः कर्तेति नैयायिकाः। अर्हन्नित्यथ जैनशासनरताः कर्मेति मीमांसकाः सोऽयं वो विदधातु वाञ्छितफलं त्रैलोक्यनाथो हरिः॥} (ह॰ना॰~१.३)।}\end{sloppypar}
\begin{sloppypar}\hyphenrules{nohyphenation}\justifying\noindent\hspace{10mm} वेदान्त\-दर्शनं षष्ठम्। इदं वेदस्य ज्ञानकाण्डं सुस्पष्टयति। एतस्य प्रवर्तका नारायण\-कलावतारा हृदय\-सन्निहित\-सकल\-श्रुति\-सारा वेद\-सिद्धान्तानुशीलन\-भागवत\-धर्म\-परिशीलन\-परिष्कृत\-विचारा विहित\-परम\-धर्म\-रस\-तरङ्गिणी\-प्रचारा निखिल\-कोविद\-कवि\-कुलालङ्कारा निरहङ्काराः सन्दर्शित\-समाजोचित\-मुनि\-मनो\-दुर्लभ\-दुरासद\-दुस्सह\-दुरन्त\-दुर्गम\-दुष्प्रेक्ष्य\-दुष्प्राप्य\-सामग्री\-सम्भारा निखिल\-योगीन्द्र\-मुनीन्द्र\-यतीन्द्र\-सुरेन्द्रासुरेन्द्र\-नरेन्द्र\-साधक\-वृन्दवन्दित\-चरण\-कमल\-कल्हाराः परमोदारा बादरायणापर\-नामधेया महर्षि\-वेद\-व्यास\-वर्याः। इदं ब्रह्मसूत्रमध्याय\-चतुष्टयात्मकम्। एतदवलम्ब्यैव विभिन्न\-वेदान्त\-मत\-प्रवर्तकाः स्व\-स्व\-सम्प्रदायानुसारं भाष्याणि बभाषिरे। अत्र ब्रह्मैव निरूपितम्। \textcolor{red}{अथातो ब्रह्म\-जिज्ञासा} (ब्र॰सू॰~१.१.१) इति। इदं वेदस्य शिरोभागव्याख्यानभूतम्। अत्र ब्रह्मणो निर्गुण\-सगुण\-पक्षौ निरूपितौ। \end{sloppypar}
\begin{sloppypar}\hyphenrules{nohyphenation}\justifying\noindent\hspace{10mm} एवमेव वेदखण्डन\-पराणि त्रीणि नास्तिक\-दर्शनानि। बोद्धं जैनं चार्वाकमिति। प्रकारान्तरेण सर्वस्यापि वाङ्मयस्य वेद एवाऽश्रयः। सर्वत्र च भगवान् स्तुत्या क्वचिन्निन्दया च प्रस्तुतः। यथा बुद्धो महावीर\-स्वामी च वेदं शिष्टतया निन्दतः किं च चार्वाकोऽशिष्टतया निन्दति।\end{sloppypar}
\begin{sloppypar}\hyphenrules{nohyphenation}\justifying\noindent\hspace{10mm} वेदमाश्रित्य त्रय आगमाः। आगमो नाम प्राचीन\-विचार\-सङ्ग्रहः। स च त्रिविधः। शैवो वैष्णवः शाक्तश्च। तत्राऽगम\-विषयक एकः पारम्परिकः श्लोकः श्रुतो गुरुभ्यः~–\end{sloppypar}
\centering\textcolor{red}{आगतं शिववक्त्रेभ्यो गतं च गिरिजाश्रुतौ।\nopagebreak\\
मतञ्च वासुदेवस्य तस्मादागम उच्यते॥}\nopagebreak\\
\raggedleft{–~इति साम्प्रदायिकाः}\\
\begin{sloppypar}\hyphenrules{nohyphenation}\justifying इदमेव तन्त्र\-शास्त्रमिति कथ्यते। अत्र द्वौ मार्गौ दक्षिणो वामश्च। दक्षिण\-मार्ग आत्म\-शुद्धि\-पूर्वक\-सिद्धिः स्वीकृता। वाममार्गे चाऽडम्बर\-पूर्णा चमत्कार\-मयी सिद्धिः साध्यते। मार्गोऽयं पञ्च\-मकार\-सेवन\-परतया हेयप्रायः। \end{sloppypar}
\begin{sloppypar}\hyphenrules{nohyphenation}\justifying\noindent\hspace{10mm} इममेव वेदमधिगन्तुं षडङ्गानि शिक्षा कल्पो निरुक्तं ज्यौतिषं छन्दो व्याकरणं च। षडङ्ग\-वेदाध्ययनं पुराऽस्माकं विधेयमासीत्। वाल्मीकिरपि सुन्दरकाण्डे षडङ्गानां प्रशंसां सोत्साहमकार्षीत्। यथा~–\end{sloppypar}
\centering\textcolor{red}{षडङ्गवेदविदुषां क्रतुप्रवरयाजिनाम्।\nopagebreak\\
शुश्राव ब्रह्मघोषान् स विरात्रे ब्रह्मरक्षसाम्॥}\nopagebreak\\
\raggedleft{–~वा॰रा॰~५.१८.२}\\
\begin{sloppypar}\hyphenrules{nohyphenation}\justifying\noindent\hspace{10mm} शिक्षायां वेद\-स्वर\-पाठ\-प्रक्रिया। बहव आचार्याः पृथक्पृथक्शिक्षा\-ग्रन्थान् विलिलिखुः। पाणिनीय\-शिक्षाऽपि महत्त्वपूर्णा। कल्पेषु वैदिक\-मन्त्र\-प्रयोगाणां नियमाः। निरुक्ते वेदस्य गहन\-शब्दानां वैदुष्य\-पूर्णा व्युत्पत्तिः। एतदाचार्यो भगवान् यास्कः। छन्दः\-शास्त्रं वेद\-च्छन्दसां व्यवस्था निर्विशति।
गायत्र्युष्णिगनुष्टुब्बृहती पङ्क्तिस्त्रिष्टुब्जगतीति सप्त च्छन्दांसि। शेषेषु यद्यपि वैदिकी रचना मिलति तथाऽपि प्रधानतया लोके तेषामेव प्रयोगः। इदमेव पिङ्गल\-शास्त्रं कथ्यते। एतत्प्रवर्तकः पिङ्गल\-नामधेयः सर्पः। श्रूयते महात्ममुखेभ्यो यत्कदाचिद्भगवाञ्छ्रीलक्ष्मी\-रमण\-रमणीय\-चरण\-कमल\-समलङ्कृत\-पृष्ठभाग\-भोगि\-वरूथ\-शत्रु\-महा\-पराक्रम\-समूह\-त्रिविक्रम\-हिरण्मय\-पक्ष\-व्याघात\-विमदीकृत\-दैत्य\-दानवाराति\-पक्ष\-लक्षो वैनतेयो भगवान् गरुडो बुभुक्षा\-परः सर्पमाजिर्हीषुरभ्यधावत्। स च दुद्राव। अनिवर्तमान\-हरियानमालोक्य प्रार्थयाञ्चक्रे सर्पो यद्देव मा जहि विद्यामेकां गोपित\-पूर्वां मत्तः प्राप्नुहीति निगद्य भगवान् सर्पभूतः पिङ्गलः पलायमानो गरुडं विद्यामुवाच। तदेव पिङ्गल\-शास्त्रं कथ्यते। अन्तिमं छन्दो भुजङ्गप्रयातं शिक्षयन् शीघ्रमेव दुरापं स्वकीयं विलमाविवेश। तदेव पिङ्गल\-शास्त्रम्। \end{sloppypar}
\begin{sloppypar}\hyphenrules{nohyphenation}\justifying\noindent\hspace{10mm} व्याकरणमेव वेदस्य मुख्याङ्गम्। यद्यपि षडङ्गवेदोऽध्येय इति स्मृति\-श्रुतिर्यथा पतञ्जलिः प्रमाणयन् प्रोवाच \textcolor{red}{ब्राह्मणेन निष्कारणो धर्मः षडङ्गो वेदोऽध्येयो ज्ञेयः} (भा॰प॰)। \textcolor{red}{निष्कारणः} इत्यत्र निर्गतं कारणमर्थ\-धर्म\-काम\-मोक्ष\-प्रभृति\-प्रवृत्ति\-प्रयोजकं यस्मात्तथा\-भूतः। इत्थं सर्वतो\-भावेन वेद\-ज्ञानाय षडङ्गानां सत्यामुपयोगितायामपि सर्व\-प्राथम्येन व्याकरणाध्ययनं नितान्तोपयोगि। विद्वद्भिर्व्याकरणं भगवतो वेदस्य मुखमित्यभ्य\-धायि। \textcolor{red}{मुखं व्याकरणं प्रोक्तम्} इति प्राचीनोक्तेः।
\textcolor{red}{प्रथमं छन्दसामङ्गं प्राहुर्व्याकरणं बुधाः} (वा॰प॰~१.११) इति भर्तृहरि\-सूक्तेश्च। व्याकरणमन्तरा व्यवस्थित\-भाषा\-ज्ञानमसम्भवम्। ऋते भाषा\-ज्ञानं वेदार्थ\-ज्ञानमपि सर्वतो\-भावेनासम्भवम्। व्याकरणं भाषाया अलङ्कारः। यथा प्राचीनैरुक्तम्~–\end{sloppypar}
\centering\textcolor{red}{यद्यपि बहु नाधीषे तथाऽपि पठ पुत्र व्याकरणम्। \nopagebreak\\
स्वजनः श्वजनो माऽभूत्सकलः शकलः सकृच्छकृत्॥}\nopagebreak\\
\raggedleft{–~प्राचीनोक्तिः}\\
\begin{sloppypar}\hyphenrules{nohyphenation}\justifying\noindent अर्थाद्बहु\-शास्त्राध्ययनेऽलभ्य\-रुचेऽपि पुत्र व्याकरणमधिगच्छ। व्याकरणं विना सन्धि\-समास\-ज्ञान\-कोष\-ज्ञान\-शून्यतया सकार\-शकारयोरुच्चारण\-भ्रमेण त्वं \textcolor{red}{स्व\-जनम्} स्व\-कुटुम्बं \textcolor{red}{श्वजनम्} इति शुनः कुक्कुरस्य जनं
मा मंस्थाः। दन्त्य\-सकार\-घटितः \textcolor{red}{स्व}\-शब्द आत्मीय\-वाची तालव्य\-शकार\-घटितः \textcolor{red}{श्व}\-शब्दः कुक्कुरवाची। व्याकरण\-ज्ञानं विनाऽयं विवेकः कथं सम्भवेत्। इत्थमेव दन्त्य\-सकार\-घटित\-\textcolor{red}{सकल}\-शब्दस्य समुदायोऽर्थस्तालव्य\-शकार\-घटित\-\textcolor{red}{शकल}\-शब्दः खण्ड\-वाचक इति कथं निर्णीयेत। तथैकवार\-वाचि\-दन्त्य\-सकार\-घटित\-\textcolor{red}{सकृत्‌}\-शब्दस्य\footnote{\textcolor{red}{एकस्य सकृच्च} (पा॰सू॰~५.४.१९)।} पुरीष\-वाचि\-तालव्य\-शकार\-घटित\-\textcolor{red}{शकृत्‌}\-शब्दात्\footnote{\textcolor{red}{शकेरृतिन्} (प॰उ॰~४.५८)। \textcolor{red}{उच्चारावस्करौ शमलं शकृत्} (अ॰को॰~२.६.६७)।} का भिन्नतेति ज्ञानं कथं स्यात्। व्याकरणं विना विधवा\-ललाट\-चर्चित\-सिन्दूरमिव मुखारविन्द\-निहित\-शास्त्रमपि नैव शोभामाटीकते। तथा वृद्धाः प्राहुः~–\end{sloppypar}
\centering\textcolor{red}{अङ्गीकृतं कोटिमितं च शास्त्रं नाङ्गीकृतं व्याकरणं च येन। \nopagebreak\\
न शोभते तस्य मुखारविन्दे सिन्दूरबिन्दुर्विधवाललाटे॥}\nopagebreak\\
\raggedleft{–~वृद्धोक्तिः}\\
\begin{sloppypar}\hyphenrules{nohyphenation}\justifying\noindent किं बहुना सत्यपि पुण्य\-जनकतावच्छेदकतावती निखिल\-निगमागम\-शास्त्र\-दर्शन\-पुराणेतिहास\-नाटक\-चम्पू\-गद्य\-पद्य\-प्रहेलिकादि\-चित्र\-बन्ध\-मणि\-बन्ध\-बहु\-विध\-स्वच्छन्द\-च्छन्दः\-प्रवाह\-कमन\-काव्य\-ज्योतिष्मती निखिल\-संस्कार\-शोधन\-शीला भगवती देव\-भारती व्याकरण\-ज्ञानमन्तरेणापुण्य\-कला कल्प\-लतिकेव भाति। यतो हि प्रयोग\-विधि\-ज्ञान\-पूर्वक\-शब्द एव ददाति समीप्सितं फलम्। यथा श्रुतिः \textcolor{red}{एकः शब्दः सम्यग्ज्ञातः शास्त्रान्वितः सुप्रयुक्तः स्वर्गे लोके कामधुग्भवति} (भा॰पा॰सू॰~६.१.८४)। तस्माज्ज्ञान\-पूर्वक\-प्रयोग एव पुण्य\-जनकतावच्छेदकः। तस्मादितर\-भाषेव व्याकरण\-ज्ञान\-शून्य\-जनोच्चरित\-संस्कृत\-भाषाऽपि वन्ध्या गौरिव निष्प्रयोजना। तथा च~–\end{sloppypar}
\centering\textcolor{red}{यथा शशाङ्केन विना विभावरी जलं विना भाति न निम्नगा यथा।\nopagebreak\\
सुबोधभाषा सरला रसान्विता न शोभते व्याकरणं विना तथा॥}\nopagebreak\\
\raggedleft{–~इति मम}\\
\begin{sloppypar}\hyphenrules{nohyphenation}\justifying\noindent संस्कृता वाण्येव पुरुषमलङ्करोति। उक्तं नीतिशतके~–\end{sloppypar}
\centering\textcolor{red}{केयूराणि न भूषयन्ति पुरुषं हारा न चन्द्रोज्ज्वला\nopagebreak\\
न स्नानं न विलेपनं न कुसुमं नालङ्कृता मूर्धजाः।\nopagebreak\\
वाण्येका समलङ्करोति पुरुषं या संस्कृता धार्यते\nopagebreak\\
क्षीयन्ते खलु भूषणानि सततं वाग्भूषणं भूषणम्॥}\nopagebreak\\
\raggedleft{–~भ॰नी॰~१९}\\
\begin{sloppypar}\hyphenrules{nohyphenation}\justifying\noindent इत्थं निरस्त\-दूषणं भूषण\-भूषणं वाग्भूषणं भूषणत्वसम्पत्तये संस्कारमपेक्षते। स च संस्कारो व्याकरणमन्तरेणाऽकाश\-पुष्पमिव काल्पनिकः। तस्मात्पुण्य\-जनकता\-पुरःसर\-यथेष्ट\-भाषा\-ज्ञान\-सम्पत्त्यै व्याकरणं रत्नदीप\-मालिकेव। ममायं प्रत्ययः साधु\-शब्दानां प्रयोगेणैव धर्मः। \end{sloppypar}
\begin{sloppypar}\hyphenrules{nohyphenation}\justifying\noindent\hspace{10mm} व्याकरणाध्ययने मुख्यतया पञ्च प्रयोजनानि प्रतिपादितान्यशेष\-विद्या\-विशेषेण भगवता शेषेण प्राञ्जलिना पतञ्जलिना। यथा तद्वार्तिकं \textcolor{red}{रक्षोहागमलघ्वसन्देहाः प्रयोजनम्} (भा॰प॰)। इमानि पञ्च प्रयोजनानि। अत्रोद्देश्य\-दलानुरोधेन विधेय\-दले कथं न बहु\-वचनतेति चेदेक\-शेष\-महिम्नेत्यवधेयम्। तत्र रक्षा\-शब्दः स्त्रीलिङ्ग ऊहागमासन्देहा उभये लघु नपुंसकलिङ्गे। एवं च रक्षा प्रयोजन्यूहः प्रयोजन आगमः प्रयोजनो लघु प्रयोजनमसन्देहः प्रयोजन इति प्रयोजनी च प्रयोजनश्च प्रयोजनश्च प्रयोजनं च प्रयोजनश्चेति विग्रहे \textcolor{red}{नपुंसकमनपुंसकेनैकवच्चास्यान्यतरस्याम्} (पा॰सू॰~१.२.६९) इति सूत्रेणानपुंसकानां चतुर्णामपि लोपे सति वैकल्पिकमेकवचनम्। यद्वा \textcolor{red}{वेदाः प्रमाणम्} इतिवद्विशेष्य\-वाचक\-पदोत्तर\-सु\-विभक्ति\-तात्पर्य\-सङ्ख्याया बहुत्वाख्यायाः प्रयोजन\-पदोत्तर\-सु\-विभक्ति\-तात्पर्य\-ग्राहकैकत्व\-सङ्ख्यया सत्यपि विरोधे तस्या एकस्यैव प्रयोजनत्वस्य सकलेष्वन्वयाद्विवक्षितत्वमेव।\end{sloppypar}
\begin{sloppypar}\hyphenrules{nohyphenation}\justifying\noindent\hspace{10mm} वेदानां रक्षार्थं व्याकरणमध्येयम्। व्याकरण\-ज्ञानं विनाऽर्थाभावे कथं वेदेषु श्रद्धा कथं वाऽवगमनम्। सर्वेऽपि मन्त्रा नैव विभक्तिषु निर्दिष्टाः। व्याकरण\-ज्ञानं विना यथायथं परिणमयितुं कथं शक्ष्यत्यवैयाकरणः। यथा \textcolor{red}{अग्नये स्वाहा} इति निर्दिष्टमिन्द्र\-प्रसङ्गे चतुर्थी\-ज्ञानं विनाऽवैयाकरण \textcolor{red}{इन्द्राय} कथं कथयिष्यति।\end{sloppypar}
\begin{sloppypar}\hyphenrules{nohyphenation}\justifying\noindent\hspace{10mm} जीवनावधिरल्पः शास्त्रञ्च विशालम्। श्रूयते यत् \textcolor{red}{बृहस्पतिरिन्द्राय दिव्यं वर्ष\-सहस्रं प्रतिपदोक्तानां शब्दानां शब्द\-पारायणं प्रोवाच नान्तं जगाम। बृहस्पतिश्च प्रवक्ता इन्द्रश्चाध्येता दिव्यं वर्षसहस्रमध्ययन\-कालो न चान्तं जगाम। किं पुनरद्यत्वे। यः सर्वथा चिरं जीवति सोऽपि शतं जीवति। चतुर्भिश्च प्रकारैर्विद्योपयुक्ता भवति। आगम\-कालेन स्वाध्याय\-कालेन प्रवचन\-कालेन व्यवहार\-कालेनेति} (भा॰प॰)। एकस्मिन्नेव काले यदि समस्तस्याऽयुषः क्षयस्तर्ह्यन्य\-प्रकाराणां का गतिः। एवमेव \textcolor{red}{बृहस्पतिश्च प्रवक्ता इन्द्रश्चाध्येता दिव्यं वर्ष\-सहस्रमध्ययन\-कालो न चान्तं जगाम} (भा॰प॰) तर्ह्यस्माकं का कथा। अतोऽल्पेन कालेन विपुलस्य शब्द\-सागरस्य यथा पारं व्रजेम लघुनोपायेनेत्यध्येयं व्याकरणम्। व्याकरणे च वृत्तयो व्यक्तं सामासिक\-सिद्धान्त\-पर्यवसायिन्य इति विपश्चितां मनीषितम्।\end{sloppypar}
\begin{sloppypar}\hyphenrules{nohyphenation}\justifying\noindent\hspace{10mm} समस्त\-वाङ्मयस्य यथार्थावगमनं कथं स्यादित्यध्येयं व्याकरणम्। \textcolor{red}{असन्देहार्थं चाध्येयं व्याकरणम्} (भा॰प॰)। यतो हि वेदे स्वर\-विचारः प्रधानः। श्रुतिरस्ति~–\end{sloppypar}
\centering\textcolor{red}{दुष्टः शब्दः स्वरतो वर्णतो वा मिथ्याप्रयुक्तो न तमर्थमाह। \nopagebreak\\
स वाग्वज्रो यजमानं हिनस्ति यथेन्द्रशत्रुः स्वरतोऽपराधात्॥}\nopagebreak\\
\raggedleft{–~भा॰प॰}\\
\begin{sloppypar}\hyphenrules{nohyphenation}\justifying\noindent\hspace{10mm} अर्थात्स्वरतो वर्णतश्चाशुद्धो मन्त्र इष्टमर्थं न प्रयच्छत्यपि तु वज्रमिव यजमानं हन्ति। वेद आख्यानमिदं प्रसिद्धम्। कूट्यात्स्वार्थ\-साधन\-निपुणेन पुरन्दरेण हते पुरोधसि विश्वरूपे स्वसुत\-निधन\-दीप्त\-क्रोधानलस्त्वष्टेन्द्र\-नाशन\-परं यज्ञं समीजे। तत्र \textcolor{red}{इन्द्रशत्रो विवर्धस्व}\footnote{\textcolor{red}{हतपुत्रस्ततस्त्वष्टा जुहावेन्द्राय शत्रवे। इन्द्रशत्रो विवर्धस्व मा चिरं जहि विद्विषम्॥} (भा॰पु॰~६.९.११)। \textcolor{red}{इन्द्रस्याभिचारो वृत्रेणारब्धस्तत्र ‘इन्द्रशत्रुर्वर्धस्व’ इति मन्त्र ऊहितः} (भा॰प्र॰) इति पस्पशायां कैयटः।} अस्मिन्मन्त्र\-खण्डे स्वर\-व्यतिक्रमादिन्द्रस्य विजयो वृत्रस्य च पराजयो जातः। व्याकरणं विना कथं समास\-सन्देह\-निवारणं स्यात्। वैयाकरणस्तु प्रकृति\-स्वरं दृष्ट्वा बहुव्रीहिं समासान्तोदात्तं दृष्ट्वा तत्पुरुषं सारल्येनावगन्तुं पारयिष्यति। अतोऽसन्देहार्थं व्याकरणं पठितव्यम्। \end{sloppypar}
\begin{sloppypar}\hyphenrules{nohyphenation}\justifying\noindent\hspace{10mm} इत्थं समस्त\-शास्त्रोपकारक\-तया पण्डित\-मचर्चिका\-चर्चिते समभ्यर्चिते च विद्वत्तल्लजैर्विबुध\-भारती\-संस्करणे शब्द\-ब्रह्म\-विहरणे सकल\-विद्यालङ्करणे व्याकरणेऽस्माभिः श्रद्धेयता निरापदं धारणीया विचारणीया च तस्य शब्द\-सागर\-समुन्मथन\-मनोरम\-मञ्जुतर\-म्रदिम\-गाम्भीर्य\-सरणिः। इदमेव सकल\-विद्यानां सोपान\-भूतं परम\-पूतं पाटवमयं निरामयं रसमयं ध्वस्त\-संसारामयं विगत\-भव\-भयं दत्ताभयं परम\-प्रकाशरूपं शब्द\-ब्रह्म\-प्रापकं मोक्ष\-द्वारं सकल\-सौष्ठवमयम्। एवं राद्धान्तिते व्याकरण\-महिमनि सकल\-विद्वन्मनोरम\-तया प्रति\-पादिते परम\-रमणीय\-शास्त्र\-विलोचनेऽस्मिञ्छब्द\-शास्त्रे समभवन् बहवः प्रवर्तकाः। \end{sloppypar}
\begin{sloppypar}\hyphenrules{nohyphenation}\justifying\noindent\hspace{10mm} तत्र पुरा नव व्याकरणानि प्रसिद्धान्यासन्। पुरा यथा वाल्मीकीय\-रामायणस्योत्तरकाण्डे भगवन्तमेव सम्बोधयन्नगस्त्यो हनुमद्विषये प्राह यत्~–\end{sloppypar}
\centering\textcolor{red}{सोऽयं नवव्याकरणार्थवेत्ता ब्रह्मा भविष्यत्यपि ते प्रसादात्॥}\nopagebreak\\
\raggedleft{–~वा॰रा॰~७.३६.४७}\\
\begin{sloppypar}\hyphenrules{nohyphenation}\justifying\noindent अथ काल\-क्रमेण लुप्त\-प्रायेषु नवसु व्याकरणेष्वनवस्था जाता। अलोक\-लोक\-लोचन\-निभ\-शास्त्र\-विषयेषु जटिलतापन्नेषु च शास्त्रीय\-विचारेषु शिथिली\-भूतेषु च धर्म\-सिद्धान्त\-प्रचारेषु सकल\-कला\-कलाप\-कलनो निखिल\-जगदुद्धरण\-शील\-ताण्डव\-प्रथित\-लीलो लीलावती\-हैमवती\-ललित\-लोचन\-विलास\-लालित\-मकर\-केतन\-वदन\-तामरस\-रस\-लुब्ध\-शैल\-सुता\-मनोमिलिन्दो मुररिपु\-पाद\-पयोज\-नख\-मणि\-चन्द्र\-कान्त\-द्रवीभूत\-परम\-पूत\-निखिल\-निगम\-सार\-सर्वस्व\-वस्तु\-भिक्षण\-समादृत\-योगीन्द्र\-मुनीन्द्र\-वृन्द\-वेद\-पुराण\-पुरस्कृत\-भगीरथ\-जटिल\-साधन\-धृत\-शरीर\-फल\-भुवन\-पावन\-नीर\-नीरज\-नयन\-नयन\-वल्लभा\-नीराजित\-जित\-मुनि\-मनो\-मलय\-समीरण\-समीरित\-भक्त\-भय\-सङ्ग\-भङ्ग\-निरङ्ग\-रिपु\-जटा\-जूट\-जटिल\-जाम्बूनद\-प्रख्य\-तरलतम\-तरङ्ग\-भग्नान्तरङ्गामङ्गल\-स्फटिक\-निर्मल\-मृदित\-कश्मल\-शमित\-संसारानल\-क्षपित\-सगर\-सूनु\-शाप\-ताप\-पाप\-प्रचण्ड\-दावानल\-दमित\-षड्विकार\-गरल\-धारा\-पङ्क्ति\-विजित\-सरल\-परम\-विमल\-परम\-तरल\-मौक्तिक\-महोज्ज्वल\-सुधा\-सम्मित\-जल\-जीर्ण\-जगज्जरा\-जन्म\-जाल\-मालती\-मालोपम\-दिव्य\-नव्य\-भव्य\-धृत\-भूरिमान\-भागीरथी\-भावित\-सुविशाल\-भालदेशो वलयीकृत\-शेषः शशाङ्क\-मौलिर्व्याकरण\-मन्तरा जगतः कल्याणमसम्भवमिति कृत्वा परम\-सन्तं प्रतिभया विलसन्तं पाणिनिमेव प्रादुर्भावयामास।\end{sloppypar}
\begin{sloppypar}\hyphenrules{nohyphenation}\justifying\noindent\hspace{10mm} स च शिव\-लब्ध\-वेद\-प्रसादो विगत\-विषादो दर्भ\-पवित्र\-पाणिरुदङ्मुखः सकल\-व्याकरण\-समन्वय\-भूतं लघु\-कायं सकल\-शास्त्राध्ययन\-सहायं निहिताष्टाध्यायं विशाल\-विशद\-सरलीकृत\-सङ्क्षिप्त\-शब्द\-शास्त्र\-स्वाध्यायमष्टाध्यायी\-सूत्र\-पाठं सुपठं पपाठ। इदं शिव\-चतुर्दश\-सूत्राणि समाश्रित्य प्रावर्तत। तानि च (१)~अइउण् (२)~ऋऌक् (३)~एओङ् (४)~ऐऔच् (५)~हयवरट् (६)~लण् (७)~ञमङणनम् (८)~झभञ् (९)~घढधष् (१०)~जबगडदश् (११)~खफछठथचटतव् (१२)~कपय् (१३)~शषसर् (१४)~हल्। \textcolor{red}{इति माहेश्वराणि सूत्राण्यणादिसञ्ज्ञार्थानि। एषामन्त्या इतः। लण्सूत्रेऽकारश्च। हकारादिष्वकार उच्चारणार्थः} (वै॰सि॰कौ॰ सञ्ज्ञाप्रकरणे)।\end{sloppypar}
\begin{sloppypar}\hyphenrules{nohyphenation}\justifying\noindent\hspace{10mm} तत्र \textcolor{red}{माहेश्वराणि} इत्यत्र \textcolor{red}{महेश्वरादागतानि}।
शङ्कर\-वर\-प्रसादात्पाणिनिना लब्धानीति भावः। पञ्चम्यन्त\-महेश्वर\-शब्दात् \textcolor{red}{तत आगतः} (पा॰सू॰~४.३.७४) इति सूत्रेण \textcolor{red}{अण्‌}\-प्रत्ययः। भत्वान्महेश्वर\-घटकाकार\-लोपो वृद्धिर्विभक्ति\-कार्यञ्च।\footnote{महेश्वर~\arrow \textcolor{red}{तत आगतः} (पा॰सू॰~४.३.७४)~\arrow महेश्वर~अण्~\arrow महेश्वर~अ~\arrow \textcolor{red}{अचो ञ्णिति} (पा॰सू॰~७.२.११५)~\arrow माहेश्वर~अ~\arrow \textcolor{red}{यचि भम्} (पा॰सू॰~१.४.१८)~\arrow भसञ्ज्ञा~\arrow \textcolor{red}{यस्येति च} (पा॰सू॰~६.४.१४८)~\arrow माहेश्वर्~अ~\arrow माहेश्वर~\arrow विभक्ति\-कार्यम्~\arrow माहेश्वर~जस्~\arrow \textcolor{red}{जश्शसोः शिः} (पा॰सू॰~७.१.२०)~\arrow माहेश्वर~शि~\arrow माहेश्वर~इ~\arrow शि सर्वनामस्थानम् (पा॰सू॰~१.१.४२)~\arrow सर्वनामस्थान\-सञ्ज्ञा~\arrow \textcolor{red}{नपुंसकस्य झलचः} (पा॰सू॰~७.१.७२)~\arrow \textcolor{red}{मिदचोऽन्त्यात्परः} (पा॰सू॰~१.१.४७)~\arrow माहेश्वर~नुँम्~इ~\arrow माहेश्वर~न्~इ~\arrow \textcolor{red}{सर्वनामस्थाने चासम्बुद्धौ} (पा॰सू॰~६.४.८)~\arrow माहेश्वरा~न्~इ~\arrow \textcolor{red}{अट्कुप्वाङ्नुम्व्यवायेऽपि} (पा॰सू॰~८.४.२)~\arrow माहेश्वरा~ण्~इ~\arrow माहेश्वराणि।} इत्थं श्रूयते भगवान् पाणिनिः शालातुरीयः कदाचिद्गुरुकुलेऽधीयानः प्रकृति\-मन्द\-बुद्धिरासीत्। \textcolor{red}{होनहार बिरवान के होत चीकने पात} इति ग्रामीण\-सूक्त्यनुसारं भविष्णु\-जनानां जीवनमपि प्रायशो विषमतामयं नूनमेव। महा\-पुरुषो भीषण\-परिस्थिति\-प्रचण्ड\-झञ्झा\-वातेन सहैवावतरति। आघातं विना व्यक्तित्व\-परिष्कारो न भवति। को जानीयाद्गुरुकुलस्थः पाणिनिर्विद्यार्थि\-जीवने मूर्ख\-चक्र\-चूडामणिः पश्चान्निखिल\-विद्वज्जन\-समर्चित\-चरण\-कमलो भवितेति। तदा कस्य हृदीत्थमनुमानं भूतं स्यादधुना यः परीक्षार्थि\-विद्यार्थिनां परिहास\-भाजनतामुपेतः स एव पाणिनिः कदाचिद्वाचस्पतेरपि सम्मान\-पात्रतां पात्रयिष्यतीति। अतो नीतिश्लोकः पठ्यते~–\end{sloppypar}
\centering\textcolor{red}{नृपस्य चित्तं कृपणस्य वित्तं मनोरथं दुर्जनमानवानाम्।	\nopagebreak\\
स्त्रीणां चरित्रं पुरुषस्य भाग्यं देवो न जानाति कुतो मनुष्यः॥}\nopagebreak\\
\raggedleft{–~नीतिश्लोकः}\\
\begin{sloppypar}\hyphenrules{nohyphenation}\justifying\noindent सम्भवतोऽनयैव धारणयैतस्य नाम पाणिनिरिति। \textcolor{red}{पाणिभ्यां गृहीत्वा नीयते गुरु\-सन्निधिं यः स पाणिनिः}। अर्थात्स्वयमध्ययन\-दुर्बलतया गुरु\-सन्निधिं न गच्छति स्म तदा गुरु\-प्रेषितैर्विद्यार्थिभिः पाणिभ्यां गृहीत्वा केशेषु सावज्ञोऽयं नीयते स्म। साम्प्रतं तु \textcolor{red}{पाणिं गृहीत्वाऽन्यानप्यूर्ध्वं नयति स्व\-रचित\-व्याकरण\-ज्ञान\-द्वारा मोक्षं प्रापयति यः स पाणिनिः}। पूर्व\-व्युत्पत्तौ तृतीयान्त\-पाणि\-शब्दोपपद\-पूर्वकात् \textcolor{red}{नी}\-धातोः (\textcolor{red}{णीञ् प्रापणे} धा॰पा॰~९०१) कर्मणि क्विप्।
ततः सर्वापहारि\-लोपे विभक्ति\-कार्ये पाणिनिः।\footnote{\textcolor{red}{पृषोदरादीनि यथोपदिष्टम्} (पा॰सू॰~६.३.१०९) इत्यनेन पृषोदरादित्वाद्ध्रस्व इति शेषः।} द्वितीयस्मिन् कल्पे द्वितीयान्त\-पाणि\-शब्दोपपदात् \textcolor{red}{नी}\-धातोः कर्तरि क्विप्।
सर्वापहारि\-लोपे सुप्कार्ये पाणिनिः।\footnote{\textcolor{red}{पृषोदरादीनि यथोपदिष्टम्} (पा॰सू॰~६.३.१०९) इत्यनेन पृषोदरादित्वाद्ध्रस्व इति शेषः।} स एव कदाचिच्छास्त्रार्थे सतीर्थ्यैः पराजितो ग्लानि\-झञ्झावात\-विलुलित\-कोमल\-हृदयतः स्वीकृत\-मुनि\-व्रतः सुर\-धुनी\-तरल\-तरङ्ग\-सङ्क्षालित\-शृङ्गमालं गौरी\-तपश्चीर\-संश्रय\-पवित्रीकृत\-विटप\-जालं शशाङ्क\-मौलि\-मञ्जुल\-विलास\-समुल्लास\-भग्न\-भक्त\-भव\-भय\-ज्वालं सकल\-शैल\-शिरोमणिं हिमाचलं समाश्रित्य निखिल\-विद्या\-निकेतं गिरिजा\-समेतं रोष\-रुक्ष\-निटिलाक्ष\-समुद्भूत\-भीषण\-पावक\-स्फुलिङ्ग\-भस्मीकृत\-मीन\-केतं शिशु\-शशि\-मधुर\-मयूख\-सुधा\-यूष\-सम्पोषित\-पद\-पाथोज\-प्रपन्न\-वर्गं स्व\-सङ्कल्प\-सृष्ट\-सकल\-सर्गमुमावन्तं भगवन्तं परिमहित\-महसा प्रबलतर\-तपसा सन्तोषयामास शिवमाशु\-तोषम्। स वै चन्द्रावतंसः श्रुति\-विहित\-प्रशंसो ढक्का\-निनाद\-च्छलेन चतुर्दश सूत्र\-रत्नानि समुन्मथ्य स्वकीय\-व्याकरण\-महा\-सागरात्समर्प्य पाणिनये सकल\-विश्वोपकारकं स्वकरुणा\-विग्रहं परिष्कर्तुं व्याकरण\-शास्त्रं प्रणेतुं चाष्टाध्यायीं सम्प्रेरयाम्बभूव। अतः श्लोको गीयते~–\end{sloppypar}
\centering\textcolor{red}{नृत्तावसाने नटराजराजो ननाद ढक्कां नव पञ्च वारम्।\nopagebreak\\
उद्धर्तुकामः सनकादिसिद्धानेतद्विमर्शे शिवसूत्रजालम्॥}\nopagebreak\\
\raggedleft{–~न॰का॰~१}\\
\begin{sloppypar}\hyphenrules{nohyphenation}\justifying\noindent\hspace{10mm} स एव पाणिनिः शिव\-वर\-प्रभाव\-परिष्कृत\-विमल\-मनीषो मनीषीशश्चतुः\-सहस्र\-सूत्रात्मकमभूतपूर्वं ग्रन्थ\-रत्नं लौकिक\-वैदिक\-सकल\-शब्द\-साधुत्व\-परं परायणञ्च विदुषां पारावारमिव प्रणिनाय।\footnote{\textcolor{red}{अथ कालेन बहवः शिष्या वर्षमुपाययुः। एकोऽपि पाणिनिर्नाम जडबुद्धिरुपाययौ॥ शिष्यान्तरोपहासेन सावमानः स पाणिनिः। शुश्रूषाक्लेशतो यातः कदाचित्तुहिनाचलम्॥ आराध्य तपसा तत्र विद्याकामः स शङ्करम्। प्राप व्याकरणं दिव्यं स च विद्यामुखं शुभम्॥} (ह॰च॰चि॰~२७.७२-७४)।} यद्यप्येतस्मात्पूर्वमप्यैन्द्र\-चान्द्रादीनि व्याकरणान्यासन्नेवञ्च काश्यप\-शाकटायन\-स्फोटायनापिशलि\-भारद्वाज\-प्रभृतीनामाचार्याणां नामानि चर्चितानि यथा \textcolor{red}{त्रिप्रभृतिषु शाकटायनस्य} (पा॰सू॰~८.४.५०) \textcolor{red}{अवङ् स्फोटायनस्य} (पा॰सू॰~६.१.१२३) \textcolor{red}{सर्वत्र शाकल्यस्य} (पा॰सू॰~८.४.५१) \textcolor{red}{ऋतो भारद्वाजस्य} (पा॰सू॰~७.२.६३) \textcolor{red}{दीर्घादाचार्याणाम्} (पा॰सू॰~८.४.५२) \textcolor{red}{वा सुप्यापिशलेः} (पा॰सू॰~६.१.९२) \textcolor{red}{हलि सर्वेषाम्} (पा॰सू॰~८.३.२२) इत्यादिषु तथाऽप्येषां विशालत्वादसमन्वयाच्च लौकिक\-वैदिक\-शब्दानां भगवान् पाणिनिः परम\-सरलं परम\-लघु\-व्याकरणं व्याचकार। यथोदाहरणमेकं द्रष्टव्यम्। इकारोकार\-ऋकार\-ऌकाराणां\footnote{अत्र \textcolor{red}{ऋत्यकः} (पा॰सू॰~६.१.१२८) इत्यनेन प्रकृतिभावः।} स्थाने स्वरे परतः क्रमाद्यकार\-वकार\-रकार\-लकार\-व्यवस्थापनार्थमाचार्याश्चत्वारि सूत्राणि पेठुः। तानि यथा~– (१)~\textcolor{red}{इ यं स्वरे} (२)~\textcolor{red}{उ वं स्वरे} (३)~\textcolor{red}{ऋ रं स्वरे} (४)~\textcolor{red}{ऌ लं स्वरे} इति।\footnote{मूलं मृगयम्।} किन्तु भगवान् पाणिनिश्चतुर्णां सूत्राणां स्थाने प्रत्याहार\-सरण्या लाघवार्थमेकमेव सूत्रमसूत्रयत्कार्यमपि सम्पूर्णं चकार। यथा \textcolor{red}{इको यणचि} (पा॰सू॰~६.१.७७)। \textcolor{red}{इक्‌}\-शब्देन चतुर्णां स्थानिनां चर्चा \textcolor{red}{यण्‌}\-शब्देन च चतुर्णामादेशानाम्। तत्रापि सवर्णयोरिगचोर्यण्निवृत्तये बाध्य\-बाधको भावः प्रास्तावि। यथा \textcolor{red}{मुनीशः} अत्रेश\-घटक ईकारेऽचि परतः मुनि\-घटकस्येकारस्य स्थाने यण् प्राप्तः स च \textcolor{red}{अकः सवर्णे दीर्घः} (पा॰सू॰~६.१.१०१) इत्यनेन बाधितः। अर्थादसवर्णयोरेवेगचोर्यण्संहिता।\end{sloppypar}
\begin{sloppypar}\hyphenrules{nohyphenation}\justifying\noindent\hspace{10mm} विस्तार\-भीत्याऽनुवृत्तिरपि पाणिनेः पटुतायाः पण्डित\-विस्मापनं प्रमाणम्। यथा \textcolor{red}{अतो भिस ऐस्} (पा॰सू॰~७.१.९) इति सूत्रम्। अदन्तादङ्गात्परस्य भिस ऐसिति सूत्रार्थः। किन्तु \textcolor{red}{तपरस्तत्कालस्य} (पा॰सू॰~१.१.७०) इति सूत्रेण तपरस्तकारात्परस्तकार\-पूर्व\-वर्ती वा सम\-कालस्य बोधकः समानोच्चारण\-सदृशोच्चारण\-कालस्य प्रत्यायक इत्यर्थ इति निर्दिश्यते। अत्र \textcolor{red}{अतो भिस ऐस्} (पा॰सू॰~७.१.९) इति सूत्रेऽप्यकारस्तकार\-पूर्व\-वर्त्यपि स्व\-समान\-ह्रस्वोच्चारणस्येकारस्यापि बोधकः स्यात्। तथा च हरिभिरित्यत्र भिस ऐस्स्यात्। अस्मिन्नसामञ्जस्ये \textcolor{red}{अणुदित्सवर्णस्य चाप्रत्ययः} (पा॰सू॰~१.१.६९) इति पूर्व\-वर्ति\-सूत्रात् \textcolor{red}{सवर्णस्य} इतिपदमनुवृत्तम्। अर्थात्तपरः सवर्णस्य सम\-कालस्य सञ्ज्ञेत्यर्थे जात इकाराकारयोः सावर्ण्याभावाद्बोधकत्वावच्छिन्न\-प्रतियोगिकत्वाभावेन नैव दोषः। अतो लघु\-सिद्धान्त\-कौमुद्यां वरदराजाचार्याः प्राहुः \textcolor{red}{सूत्रेष्वदृष्टं पदं सूत्रान्तरादनुवर्तनीयं सर्वत्र} (ल॰सि॰कौ॰~१)।\end{sloppypar}
\begin{sloppypar}\hyphenrules{nohyphenation}\justifying\noindent\hspace{10mm} एवमेव बहुत्र। एतद्दृश्यते सङ्क्षेप\-प्रक्रिया\-व्यवस्थार्थमेव महा\-मुनिना सपाद\-सप्ताध्याय्यां त्रिपाद्यामप्युत्तरोत्तर\-सम्बन्ध\-व्यवस्था कृता। अतो लक्ष्यानुरोधेन पृथक्सूत्र\-निर्माणमन्तराऽपि \textcolor{red}{पूर्वत्रासिद्धम्} (पा॰सू॰~८.२.१) इति सूत्र\-बलेन \textcolor{red}{सपाद\-सप्ताध्यायीं प्रति त्रिपाद्यसिद्धा त्रिपाद्यामपि पूर्वं प्रति परं शास्त्रमसिद्धम्} (ल॰सि॰कौ॰~३१)। \textcolor{red}{पूर्वत्र} इत्यत्र सप्तम्यन्त\-\textcolor{red}{त्रल्‌}\-प्रत्ययः। तथा \textcolor{red}{पूर्वस्मिन्} इति विग्रहे सप्तम्यन्त\-\textcolor{red}{पूर्व}\-शब्दात् \textcolor{red}{सप्तम्यास्त्रल्} (पा॰सू॰~५.३.१०) इति सूत्रेण \textcolor{red}{त्रल्‌}\-प्रत्यये विभक्ति\-कार्ये \textcolor{red}{तद्धितश्चासर्वविभक्तिः} (पा॰सू॰~१.१.३८) इत्यनेनाव्यय\-सञ्ज्ञायाम् \textcolor{red}{अव्ययादाप्सुपः} (पा॰सू॰~२.४.८२) इत्यनेन विभक्ति\-लोपे \textcolor{red}{पूर्वत्र} इति सिद्धम्। तथा च पूर्वत्र परमसिद्धमिति फलितार्थः।
पूर्वस्मिन् परमसिद्धमित्यर्थे कृतेऽत्र सप्तमी विषयता\-रूपा। विषयता च कर्तव्यता\-रूपा। एवं पूर्वस्मिन् कर्तव्ये शास्त्रे परमसिद्धमिति जातम्। कस्मात्परमिति जिज्ञासायां प्रकरणमवलोक्य निश्चीयते। सूत्रमिदमष्टाध्याय्या अष्टमाध्यायस्य द्वितीयस्य पादस्य प्रथमम्। पूर्व\-पर\-शब्दयोस्तस्मात्पूर्वस्मिन्निति फलितमेतस्माच्च परम्। इत्थं निर्दिष्ट\-सूत्रात्पूर्वा सपाद\-सप्ताध्यायी परा च त्रिपादी। अतः सपादसप्ताध्याय्यां कर्तव्यायां पूर्व\-शब्द\-सङ्केत्यायां परा त्रिपाद्यसिद्धा। अस्मिन्नंशे सूत्रस्यास्य विधित्वं प्रतिभाति। त्रिपाद्यामपि पूर्वं प्रति परशास्त्रमसिद्धम्। अयमर्थोऽधिकारगम्यः। यथा कश्चिन्निर्देशकः प्रतिजन\-समुदायं गच्छन् दण्डेन निर्दिशति यत्त्वं स्व\-पूर्वं दृष्ट्वाऽसिद्ध\-कार्यो भव तथैव सूत्रमिदं त्रिपाद्यां प्रतिसूत्रं गत्वा निर्दिशति त्वं सपादसप्ताध्यायीं प्रति स्वपूर्वत्रिपादीं प्रति चासिद्धं भवेति। त्रिपाद्यंश एतस्य अधिकार\-सूत्रता। इयं मे मनीषा। यद्यप्यन्य आचार्या भट्टोजि\-दीक्षित\-हरि\-दीक्षित\-नागेश\-भट्टपाद\-प्रमुखा इदं सामग्र्येणाधिकारमेव स्वीकुर्वन्ति। यथा प्रयत्न\-निर्देश\-प्रसङ्गे श्री\-भट्टोजि\-दीक्षितश्चतुर एव प्रयत्नानाभ्यन्तरान् स्वीकुर्वन् प्रतिपादयति~– \textcolor{red}{ह्रस्वस्यावर्णस्य प्रयोगे संवृतम्। प्रक्रिया\-दशायां तु विवृतमेव} (वै॰सि॰कौ॰~१०)। तथा च सूत्रं \textcolor{red}{अ अ} (पा॰सू॰~८.४.६८) इत्यनुसारेण \textcolor{red}{विवृतमनूद्य संवृतोऽनेन विधीयते। अस्य चाष्टाध्यायीं सम्पूर्णां प्रत्यसिद्धात्वाच्छास्त्र\-दृष्ट्या विवृतत्वमस्त्येव} (वै॰सि॰कौ॰~११)। \textcolor{red}{तथा हि सूत्रं “पूर्वत्रासिद्धम्” (पा॰सू॰~८.२.१)। अधिकारोऽयम्} (वै॰सि॰कौ॰~१२) इति। अर्थादकारं वर्जयित्वा सर्वेषामपि स्वराणां विवृत आभ्यन्तरः प्रयत्नः। केवलमकारस्य संवृत इति। स च \textcolor{red}{अ अ} (पा॰सू॰~८.४.६८) इति सूत्रेण विधीयते। तत्र प्रथमः अ इति लुप्त\-षष्ठीकः। अस्य अ इति तात्पर्यं सामान्यम्। विवृतस्याकारस्य स्थाने संवृतो भवत्वकार इति भावः। तदा \textcolor{red}{रामागमनम्} इत्यत्र राम\-घटकस्याकारस्य संवृतत्वादागमन\-घटकाऽऽकारस्य विवृतत्वाच्चोभयो\-र्विषमाभ्यन्तर\-प्रयत्नतया सावर्ण्य\-प्रतियोगिकाभावेन \textcolor{red}{अकः सवर्णे दीर्घः} (पा॰सू॰~६.१.१०१) इत्यनेन दीर्घानुपपत्तौ \textcolor{red}{पूर्वत्रासिद्धम्} (पा॰सू॰~८.२.१) इत्यनेनाष्टाध्याय्या अन्तिम\-सूत्रत्वात्सम्पूर्णं प्रत्येतस्यासिद्धौ सिद्धायां दीर्घ\-विधायक\-सूत्र\-कर्तव्यतायामेतस्य चासिद्धौ द्वयोर्विवृत\-प्रयत्नतयाऽकाराकारयोः सावर्ण्ये सति दीर्घे \textcolor{red}{रामागमनम्} इत्यपि प्रयोगः सुसिद्धः। इयं तत्रत्या परिस्थितिः। किन्त्वत्र पूर्व\-शब्दस्यावधि\-सापेक्षत्वेनोक्त\-सूत्रस्याष्टमाध्याय\-द्वितीय\-पाद\-प्रथमत्वात्ततः पूर्वस्मिन् कर्तव्य इतः परमसिद्धमित्यर्थः सुस्पष्ट एव। ततः सपाद\-सप्ताध्यायीं प्रति विधित्वेनैव कार्य\-सिद्धौ तदंशेऽलमधिकार\-कल्पनया। अधिकारत्वं नाम स्व\-देशे वाक्यार्थ\-ज्ञान\-शून्यत्वे सति पर\-देशे वाक्यार्थ\-ज्ञान\-बोधकत्वमुत्तरोत्तर\-सम्बन्धकत्वं वा। त्रिपाद्यामधिकारत्वं भवतूत्तरोत्तर\-सम्बन्धत्वात्सपाद\-सप्ताध्याय्यामन्योऽन्यं प्रत्यन्यो\-ऽन्यमसिद्धमित्यस्यापेक्षैव नास्ति। अतस्तत्रोत्तरोत्तर\-सम्बन्धत्व\-मूलकाधिकारत्व\-सूचकासिद्धि\-विधायकस्य \textcolor{red}{पूर्वत्रासिद्धम्} (पा॰सू॰~८.२.१) इत्यस्य प्रसर\-परिसर एव नास्ति। तत्र हि \textcolor{red}{विप्रतिषेधे परं कार्यम्} (पा॰सू॰~१.४.२) इति सूत्रस्य जागरूकत्वात्। विप्रतिषेधो नामान्यत्रान्यत्र\-लब्धावकाशयोरेकत्र युगपत्प्रवृत्तिः। यथा \textcolor{red}{इको यणचि} (पा॰सू॰~६.१.७७) इति \textcolor{red}{सुध्युपास्यः} इत्यादौ चरितार्थम्। \textcolor{red}{अकः सवर्णे दीर्घः} (पा॰सू॰~६.१.१०१) इति च \textcolor{red}{दैत्यारिः} इत्यादौ चरितार्थम्। उभयोरपि \textcolor{red}{श्री ईशः} इत्यादौ युगपत्प्राप्तिः। परत्वाद्दीर्घः। तस्मात्सपादसप्ताध्याय्यां पूर्वत्रासिद्धमित्यस्य प्रसरणाभावेन तत्राधिकारस्यानुपयोगात्सपाद\-सप्ताध्यायीं प्रति त्रिपाद्यसिद्धत्वस्य शब्द\-महिम्नैव सिद्धतया केवलं त्रिपाद्यामेवैतस्याधिकारता। सोपयोगा प्रतिभाति मे बाल\-मनीषया। स्वं प्रत्यप्येतस्याधिकारता नास्ति। इदमस्त्यस्यासिद्धि\-सूचकम्। स्वयं नासिद्धिमाटीकते। यथा कश्चित्प्राण\-दण्ड\-दाता नैव स्वयं दण्ड्यो भवति। एकस्मिन्नेव दण्ड\-दातृत्व\-दण्ड्यत्व\-धर्मयोरसम्भवमिव। एकस्मिन् सूत्रेऽसिद्धि\-सूचकत्वासिद्धत्व\-धर्मौ न घटेताम्। नह्यत्र चतुरोऽपि नटः स्वेनैव स्वस्य स्कन्धमारोढुं शक्नोति। यद्यप्येतस्याधिकार\-सूत्रत्व\-चर्चा सिद्धान्त\-कौमुदी\-प्रौढ\-मनोरमा\-शब्दरत्न\-शब्देन्दुशेखरेषु साटोपं प्रतिपादिता। इत्यलमतिपल्लवितेन।\end{sloppypar}
\begin{sloppypar}\hyphenrules{nohyphenation}\justifying\noindent\hspace{10mm} विविध\-लक्ष्याणामल्पैरेव सूत्रैर्यादृश्या चातुर्य\-पूर्ण\-तूर्ण\-प्रतिभया व्यवस्था दत्ता भगवता पाणिनिना सा नूनं स्वस्यामलौकिक्यनुपमा मनीषि\-मनोरमा च। अन्येभ्यो व्याकरणेभ्योऽमुष्मिन् पाणिनीय\-व्याकरण एकं वैलक्षण्यं यत्केषुचिदपि वैदिक\-शब्दानामनुशासनं न वर्तते। किन्त्वस्मिन् लौकिक\-वैदिकानामुभयेषामप्यनुशासनं वैदिक\-शब्द\-प्रक्रिया\-सङ्कलनम्। सिद्धान्त\-कौमुद्या वैदिक\-प्रकरणं स्वर\-प्रकरणं च सर्वेषामज्ञानावरणं निस्स्पृणोति। सूत्रेषु त्रिशत\-प्राय\-सूत्राणि लौकिक\-शब्दतो\-विलक्षण\-वैदिक\-शब्द\-साधुत्व\-प्रक्रिया\-प्रकार\-प्रतिपादकानि।\footnote{\textcolor{red}{लौकिक\-शब्दतो\-विलक्षण} इत्यत्र सुप्सुपा\-समासः।} यथा प्रथमाध्यायस्य द्वितीयपादे~–\end{sloppypar}
\centering\textcolor{red}{फल्गुनीप्रोष्ठपदानां च नक्षत्रे।\nopagebreak\\
छन्दसि पुनर्वस्वोरेकवचनम्।\\
विशाखयोश्च।\nopagebreak\\
तिष्यपुनर्वस्वोर्नक्षत्रद्वन्द्वे बहुवचनस्य द्विवचनं नित्यम्।}\nopagebreak\\
\raggedleft{–~पा॰सू॰~१.२.६०–१.२.६३}\\
\begin{sloppypar}\hyphenrules{nohyphenation}\justifying\noindent\hspace{10mm} पातञ्जल\-महाभाष्येऽपि \textcolor{red}{अथ शब्दानुशासनम्। अथेत्ययं शब्दोऽधिकारार्थः प्रयुज्यते। शब्दानुशासनं नाम शास्त्रमधिकृतं वेदितव्यम्। केषां शब्दानाम्। लौकिकानां वैदिकानां च इति} (भा॰प॰)। \textcolor{red}{शासुँ अनुशिष्टौ} (धा॰पा॰~१०७५) इत्यस्माद्धातोः \textcolor{red}{ल्युट् च} (पा॰सू॰~३.३.११५) इति सूत्रेण नपुंसके भावे ल्युट्। अनुबन्ध\-कार्येऽनादेशे\footnote{\textcolor{red}{युवोरनाकौ} (पा॰सू॰~७.१.१) इत्यनेन।} विभक्ति\-कार्ये \textcolor{red}{अनु}\-उपसर्ग\-जुष्टतया \textcolor{red}{अनुशासनम्} इति सिद्धम्। यद्यपि शासन\-शब्द एवाऽनुपूर्वी\-पूर्ण\-शास्त्ररूपोऽर्थो निर्गलति तदाऽनूपसर्गस्योप\-योगिताऽऽनुकूल्य\-प्रतिपादनार्था। एवं च \textcolor{red}{पाणिनिः शब्दाननुशास्ति} इत्यत्र \textcolor{red}{पाणिन्यभिन्नैक\-कर्तृक\-वर्तमान\-कालतावच्छेदकतावच्छिन्न\-शब्द\-कर्मानुकूल्य\-पुरःसरानुपूर्वी\-समन्वित\-शब्दानुशिष्ट्यनुकूलो व्यापार} इति शाब्द\-बोध\-प्रकारः। धन्यो भगवान् पाणिनिर्यः खलु निखिल\-पण्डित\-भयावहं भग्न\-विपश्चिद्धैर्य\-राशिमगाधं सागरमिव साक्षात्पर\-ब्रह्म दुरासदं दुर्धर्षं दुर्ज्ञेयं दुरवगमं सर्वतन्त्र\-स्वतन्त्रमनन्तं शब्द\-कण्ठीरवमपि सर्वमनुशशास। आनुकूल्यञ्च साधुत्वम्। तच्च शिष्ट\-प्रयुक्तत्वे सति पुण्य\-जनकतावच्छेदकत्वम्। लक्षणेनानेन तुलसी\-कृतादौ नाव्याप्तिः। तुलसी\-कृत\-मानसमपि शिष्ट\-प्रयुक्तं पुण्य\-जनकतावच्छेदकमपि। यद्यपि सामान्य\-भाषा\-भणितत्वादिदं न पुण्यजनकं ये स्वीकुर्वन्ति विभ्रान्तास्ते। अत्राद्याप्येतत्पाठेन लौकिक\-पारलौकिक\-सिद्धयो दृश्यन्ते। \end{sloppypar}
\begin{sloppypar}\hyphenrules{nohyphenation}\justifying\noindent\hspace{10mm} समानाधिकरणयोरेवावच्छेद्यावच्छेदक\-भाव\-नियमः। यथा गोत्वावच्छेदकं सास्नादिमत्त्वमेकस्मिन्नेव गवि गोत्व\-सास्नादिमत्त्व\-धर्मौ तिष्ठतः। अतो \textcolor{red}{गोत्वावच्छिन्नो गौः} इति प्रयोगो गोत्व\-युक्त\-परः। किन्त्वयं नियमोऽपि प्रायिक एव। अवच्छेद्यावच्छेदक\-भाव\-व्यवस्थाया विशेषं दर्शनमवच्छेदकत्व\-निरुक्तौ निरूपितम्। शिष्टत्वं नामाप्तत्वम्। तच्च सकल\-दुर्गुण\-शून्यत्वे सति दिव्य\-विज्ञान\-विध्वस्त\-कल्मषत्वे सति त्रिकाल\-दर्शित्वे सति यथार्थ\-वक्तृत्वम्। अत एव न्याय\-प्रवर्तकाः श्रीगौतम\-पादा आप्त\-वाक्यमेव शब्द\-प्रमाणं मन्यन्ते। यथा \textcolor{red}{आप्तोपदेशः शब्दः} (न्या॰सू॰~१.१.७)। \textcolor{red}{आपॢँ व्याप्तौ} (धा॰पा॰~१२६०) इत्यस्माद्धातोः \textcolor{red}{आप्नोति त्रिकालज्ञतया सर्वमपि चराचरं व्याप्नोति} इति विग्रहे कर्तरि क्तः।\footnote{\textcolor{red}{गत्यर्थाकर्मक\-श्लिष\-शीङ्स्थास\-वस\-जन\-रुह\-जीर्यतिभ्यश्च} (पा॰सू॰~३.४.७२) इत्यनेन। कर्मणोऽविवक्षाया अकर्मकत्वम्। \textcolor{red}{धातोरर्थान्तरे वृत्तेर्धात्वर्थेनोपसङ्ग्रहात्। प्रसिद्धेरविवक्षातः कर्मणोऽकर्मिका क्रिया॥} (वा॰प॰~३.७.८८)।} अयमेवाऽप्तशब्दो भाषायामप\-भ्रंशतया \textcolor{red}{आप} इति कथ्यते। आप्तो नाम यथार्थ\-वक्ता। \textcolor{red}{रागादि\-वशादपि नानन्यथा\-वादी यः स इति चरके पतञ्जलिः} (ल॰म॰, प॰ल॰म॰) इति वैयाकरण\-सिद्धान्त\-लघु\-मञ्जूषायां वैयाकरण\-सिद्धान्त\-परम\-लघु\-मञ्जूषायां च नागेश\-भट्टपादाः। अपि च~–\end{sloppypar}
\centering\textcolor{red}{रजस्तमोभ्यां निर्मुक्ता नित्यज्ञानबलेन ये।\nopagebreak\\
येषां त्रिकालममलं ज्ञानमव्याहतं सदा॥\nopagebreak\\
आप्ताः शिष्टा विबुधास्ते तेषां वाक्यमसंशयम्।\nopagebreak\\
सत्यं वक्ष्यन्ति ते कस्मादसत्यं नीरजस्तमाः॥}\nopagebreak\\
\raggedleft{–~च॰सं॰ सू॰स्था॰~११.१८,१९}\\
\begin{sloppypar}\hyphenrules{nohyphenation}\justifying\noindent इति। भ्रम\-प्रमाद\-विप्रलिप्सा\-करणापाटवादि\-दोष\-दूषितान्तः\-करण\-स्वार्थ\-नाशित\-चक्षुषो\-ऽप्रमाणं वक्तुं पारयन्ति। किन्तु ये दिव्य\-विज्ञान\-सम्पन्नाः प्रपन्नाश्च हरेः पदमुपपन्नास्ते कथं ब्रूयुर्विपन्ना इव दूषितम्। तुलसीदासोऽपि मानस आप्त\-वाक्यस्यैव प्रामाण्यं बाभाष्यते~–\end{sloppypar}
\centering\textcolor{red}{ते श्रोता बक्ता सम शीला। समदरशी जानहिं हरि लीला॥\nopagebreak\\
जानहिं तीनि काल निज ग्याना। करतल गत आमलक समाना॥}\footnote{एतद्रूपान्तरम्–\textcolor{red}{तावुभौ श्रोतृवक्तारौ समशीलौ समेक्षणौ। जानीतश्च हरेर्लीलां सकलां सर्वभावतः॥ निजज्ञानप्रभावेण तौ कालत्रितयं पुनः। करामलकवत्साक्षात्सर्वदैवावगच्छतः॥} (मा॰भा॰~१.३०.६,७)।}\nopagebreak\\
\raggedleft{–~रा॰च॰मा॰~१.३०.६,७}\\
\begin{sloppypar}\hyphenrules{nohyphenation}\justifying\noindent ऋषय एवाऽप्ताः। ते खल्वितर\-राज\-कवय इव राज्ञो न किमपि गृह्णन्ति स्म। रामायण इदमाख्यानं प्रसिद्धं यत्~–\end{sloppypar}
\centering\textcolor{red}{निर्माय रामायणमादिकाव्यं श्रीमैथिलीरामरसायनञ्च।\nopagebreak\\
अध्याप्य सैतेयकुशीलवौ तौ वाल्मीकिवर्यः किल निर्दिदेश॥\\
गत्वा रामायणं काव्यं गायतां रामसन्निधौ।\nopagebreak\\
युवाभ्यां कर्हिचिद्राज्ञो ग्रहीतव्यं न किञ्चन॥\\
गृहीते द्रविणे राज्ञो बुद्धिमालिन्यकारणात्।\nopagebreak\\
व्याहन्येताप्तता वत्सौ निर्लोभं गायतामतः॥}\footnote{मूलं रामायणेषु मृग्यम्।}\nopagebreak\\
\begin{sloppypar}\hyphenrules{nohyphenation}\justifying\noindent एतादृश\-वीत\-राग\-माहात्म्य\-समानाधिकरणमेवाप्तत्वम्। एषामेव वाक्यं प्रामाणिकम्। अत एव~–\end{sloppypar}
\centering\textcolor{red}{इक्ष्वाकुवंशप्रभवो रामो नाम जनैः श्रुतः। \nopagebreak\\
नियतात्मा महावीर्यो द्युतिमान् धृतिमान् वशी॥\nopagebreak\\
बुद्धिमान्नीतिमान् वाग्मी श्रीमाञ्छत्रुनिबर्हणः। \nopagebreak\\
विपुलांसो महाबाहुः कम्बुग्रीवो महाहनुः॥}\nopagebreak\\
\raggedleft{–~वा॰रा॰~१.१.८–९}\\
\begin{sloppypar}\hyphenrules{nohyphenation}\justifying\noindent इत्यादि वाल्मीकीयं प्रमाणम्। अन्यथा रामो राजा बभूवेति को मन्येत। वेदव्यासोऽपि निर्लोभस्यैवाऽप्ततां समामनति~–\end{sloppypar}
\centering\textcolor{red}{चीराणि किं पथि न सन्ति दिशन्ति भिक्षां\nopagebreak\\
नैवाङ्घ्रिपाः परभृतः सरितोऽप्यशुष्यन्।\nopagebreak\\
रुद्धा गुहाः किमजितोऽवति नोपसन्नान्\nopagebreak\\
कस्माद्भजन्ति कवयो धनदुर्मदान्धान्॥}\nopagebreak\\
\raggedleft{–~भा॰पु॰~२.२.५}\\
\begin{sloppypar}\hyphenrules{nohyphenation}\justifying\noindent ईदृशाप्तोच्चरितत्वमेव साधुत्वावच्छेदकम्। साधुत्वञ्च जातिः। \textcolor{red}{जातित्वं नामैकत्वे सति नित्यत्वे सत्यनेक\-समवेतत्वम्}। यथा घटत्वम्। तच्चैकं नित्यमनेक\-घट\-समवेतञ्च। मञ्जूषायामेतस्य चर्चा।\footnote{\textcolor{red}{साधुत्वं च व्याकरण\-व्यङ्ग्योऽर्थ\-विशिष्ट\-शब्द\-निष्ठ\-पुण्य\-जनकतावच्छेदको जातिविशेषः} (ल॰म॰)।} साधुष्वसाधुषु च वाचकत्वाविशेषः पुण्य\-पापयोरेव तत्र नियमः।\footnote{\textcolor{red}{वाचकत्वाविशेषे वा नियमः पुण्यपापयोः} (वा॰प॰~३.३.३०)।} साधूनां शब्दानामुच्चारणे पुण्यं जायते। यथा भाष्यकारोऽपि \textcolor{red}{भ॒द्रैषां॑ ल॒क्ष्मीर्निहि॒ताधि॑ वा॒चि} (ऋ॰वे॰सं॰~१०.७१.२, भा॰प॰) इति कथयति। नैयायिकानां नये साधु\-शब्देषु वर्तमानमर्थमसाधु\-शब्देषु स्मृत्वैवार्थं प्रतिपद्यते।\footnote{\textcolor{red}{असाधुरनुमानेन वाचकः कैश्चिदिष्यते} (वा॰प॰~३.३.३०)।} यथा कश्चित् \textcolor{red}{गगरी}\-शब्दमुच्चारयति। तत्र घटे वर्तमानं कम्बु\-ग्रीवादिमानित्यर्थं स्मृत्वा भावं प्रतिपद्यते। किन्तु वैयाकरणानां नय इयं मान्यता नहि। यतो हि वर्ष\-कल्पो बालः शुद्धं घट\-शब्दं न जानन्नपि \textcolor{red}{गगली} इत्युच्चारणेनैव तमर्थं प्रतिपद्यते। अतो वाचकत्वमुभयत्र किन्तु पुण्य\-जनकता साधुष्वेव। साधूनामन्वाख्यानं व्याकरणं ह्यसाधून् साधु\-शब्देभ्यो व्याकरोति। एवं \textcolor{red}{व्याकरणाभिन्नैक\-कर्तृक\-वर्तमान\-कालावच्छिन्नासाधु\-शब्द\-कर्मक\-साधु\-शब्दावधिक\-पृथक्करणानुकूल\-व्यापारः}। \textcolor{red}{व्याक्रियन्ते साधु\-शब्दा येन तद्व्याकरणम्} इति व्युत्पत्तौ \textcolor{red}{वि आङ्} पूर्वक \textcolor{red}{डुकृञ् करणे} (धा॰पा॰~१४७२) इति धातोः \textcolor{red}{करणाधिकरणयोश्च} (पा॰सू॰~३.३.११७) इत्यनेन ल्युटि लटोश्चेतोर्लोपे योरनादेशे\footnote{\textcolor{red}{युवोरनाकौ} (पा॰सू॰~७.१.१) इत्यनेन।} णत्वे\footnote{\textcolor{red}{अट्कुप्वाङ्नुम्व्यवायेऽपि} (पा॰सू॰~८.४.२) इत्यनेन।} यणि\footnote{\textcolor{red}{इको यणचि} (पा॰सू॰~६.१.७७) इत्यनेन।} विभक्ति\-कार्ये च \textcolor{red}{व्याकरणम्} इति सिद्धम्। व्युत्पत्तिर्हि शब्दार्थ\-फल\-प्रयोजन\-ज्ञानम्। तत्र कृधातुना साकं \textcolor{red}{वि आङ्} इत्युपसर्ग\-द्वयस्य समभिव्याहारोऽपूर्वमर्थं व्यनक्ति। तत्र \textcolor{red}{वि} इत्यस्यार्थो विवेचनम्। \textcolor{red}{आ} इत्यस्याऽसमन्तात्। अर्थाद्व्याकरणेन विविच्याऽसमन्तात्क्रियते साधु\-शब्दः प्रकट्यत\footnote{\textcolor{red}{सम्प्रोदश्च कटच्} (पा॰सू॰~५.२.२९) इत्यनेन निष्पन्नात् \textcolor{red}{प्रकट}\-शब्दात् \textcolor{red}{तत्करोति तदाचष्टे} (धा॰पा॰ ग॰सू॰~१८७) इत्यनेन णिचि धातुसञ्ज्ञायां कर्मणि लटि यकि \textcolor{red}{णेरनिटि} (पा॰सू॰~६.४.५१) इति णिलोपे तिपि \textcolor{red}{टित आत्मनेपदानां टेरे} (पा॰सू॰~३.४.७९) इत्यनेनैत्वे \textcolor{red}{प्रकट्यते} इति सिद्धम्। प्रकटीक्रियते इत्यर्थः।} इति तात्पर्यम्। तदेव लक्ष्य\-लक्षणे व्याकरणमिति कथयति। \textcolor{red}{व्याकरणत्वं नाम पाणिनि\-प्रभृति\-त्रिमुन्युच्चरितत्वे सत्यसाधु\-शब्द\-पृथक्कर्तृत्वे सति साधु\-शब्दानुख्यातृत्वम्} इति मे मतम्।\end{sloppypar}
\begin{sloppypar}\hyphenrules{nohyphenation}\justifying\noindent\hspace{10mm} एवं \textcolor{red}{शिष्टोच्चरितत्वे सति पुण्य\-जनकतावच्छेदकत्वरूपं साधुत्वं} सार्वभौमम्। अत्र व्याकरण\-सम्मतत्वे नाग्रहः। तेन वाल्मीकि\-रामायणे पुराणेषु त बहुत्र सत्यपि पाणिनि\-सिद्धान्त\-विरुद्ध\-प्रयोगे नैव साधुत्वोच्छित्तिः। शिष्ट\-प्रयुक्तत्वात्। यथा \textcolor{red}{सीतायाः पतये नमः} (रा॰र॰स्तो॰~२७) अत्रासमस्त\-पति\-शब्दाद्घि\-सञ्ज्ञा\-फल\-रूपो \textcolor{red}{घेर्ङिति} (पा॰सू॰~७.३.१११) इति गुण अपाणिनीय एव। पाणिनिस्तु समास एव यत्र पति\-शब्दं घिसञ्ज्ञं वाञ्छति। यथा तत्सूत्रं \textcolor{red}{पतिः समास एव} (पा॰सू॰~१.४.८)। किन्त्वपाणिनीयत्वेऽपि शिष्ट\-प्रयुक्तत्वादत्र साधुत्वम्। एवमेव \textcolor{red}{साधवो हृदयं मह्यम्} (भा॰पु॰~९.४.६८)। अत्र व्यासो ममेत्यस्य स्थाने मह्यमिति लिखति। किन्तु शिष्ट\-प्रयुक्तत्वेनात्र साधुता। ईदृशेषु स्थलेष्वार्षत्वादित्येव समाधानं समादधति सुधियः। अत्र दृष्टादौ ह्यृषयः परमात्म\-चिन्तका नैवमुपगच्छन्ति व्याकरणम्। अपि तु स्वसफलतार्थं व्याकरणमेव ताननुगच्छेत्। अतः शिष्ट\-प्रयुक्तानेव शब्दान् साधुत्व\-प्रतिपादनरूप\-सुमनोभिः सम्पूज्य कृतकृत्यतां व्रजति व्याकरणम्। अत इदानीं पाणिनीय\-प्रक्रियोपयोगि\-साधुत्वं मीमांसामहे। पूर्वोक्त\-साधुत्वं तु सार्वत्रिकम्। इदानीं पाणिनीय\-प्रक्रियायां साधुत्वमनुसन्दधे। \textcolor{red}{साधुत्वं नामाप्रवृत्त\-नित्य\-विध्युद्देश्यतावच्छेदकतानाक्रान्तत्वम्}।
यथा \textcolor{red}{सुध्युपास्यः} इत्यत्राप्रवृत्तो यो नित्य\-विधिर्दीर्घ\-गुणादिस्तदुद्देश्यतावच्छेदकता याऽक्त्वाच्त्व\-रूपा तदनाक्रान्तत्वं \textcolor{red}{सुध्युपास्यः} इत्यत्र साधुत्वमस्त्येव। अप्रवृत्तेति पदं \textcolor{red}{बाभवति} इत्याद्यसाधुत्व\-निरासार्थम्।
अन्यथा प्रवृत्तस्य विघात\-रूपस्य विधेरिक्त्व\-रूपोद्देश्यतावच्छेदकतया भवतीति पदमाक्रान्तमेव। अतोऽप्रवृत्तेति। नित्य\-विधि\-शब्दोपादानं विकल्प\-स्थलेऽपि साधुत्व\-प्रतिपादनार्थं यथा \textcolor{red}{चक्री अत्र} इति स्थले \textcolor{red}{इकोऽसवर्णे शाकल्यस्य ह्रस्वश्च} (पा॰सू॰~६.१.१२७) इत्यनेनासवर्णाजुपश्लिष्टाः पदान्ता इकः शाकल्यमते ह्रस्व\-समुचितं प्रकृति\-भावं भजन्त इत्यर्थानुसारमत्र घटकेऽकारेऽसवर्णेऽच्परे पदान्तश्चक्रीघटक ईकारः सह्रस्वं प्रकृति\-भावमभजद्विकल्पेन \textcolor{red}{चक्रि अत्र} इति पक्षे। \textcolor{red}{चक्री अत्र} इत्यवस्थायां यणि \textcolor{red}{चक्र्यत्र}। अत्रापि विकल्प\-विध्युद्देश्यतावच्छेदकताक्रान्तत्वेऽपि साधुत्वे न क्षतिः। इदमेव साधुत्वं पाणिनीय\-प्रक्रियार्थमुपयोगि। अतः शब्द\-नित्यत्व\-पक्षेऽयमर्थः क्रियते \textcolor{red}{इको यणचि} (पा॰सू॰~६.१.७७) इत्यादेः। अजुपश्लिष्टेग्घटितस्य स्थाने यण्घटितः प्रयोक्तव्यः स च साधुः। अतो भाष्यकारः कथयति~–\end{sloppypar}
\centering\textcolor{red}{सर्वे सर्वपदादेशा दाक्षीपुत्रस्य पाणिनेः।\nopagebreak\\
एकदेशविकारे हि नित्यत्वं नोपपद्यते॥}\nopagebreak\\
\raggedleft{–~भा॰पा॰सू॰~१.१.२०, ७.१.२७}\\
\begin{sloppypar}\hyphenrules{nohyphenation}\justifying\noindent अतः प्रक्रियार्थं शब्देषु काल्पनिको विकारः। प्रकृति\-प्रत्यय\-कल्पना तत्तदर्थ\-विकार\-कल्पना सर्वाऽप्यौपचारिकी। यथाऽऽह श्रीहरिः~–\end{sloppypar}
\centering\textcolor{red}{उपायाः शिक्षमाणानां बालानामुपलालनाः।\nopagebreak\\
असत्ये वर्त्मनि स्थित्वा ततः सत्यं समीहते॥}\nopagebreak\\
\raggedleft{–~वा॰प॰~२.२३८}\\
\begin{sloppypar}\hyphenrules{nohyphenation}\justifying\noindent अतः सत्यस्य शब्द\-ब्रह्मणः परिचयार्थमसत्याऽपि व्याकरण\-प्रक्रिया नितरामुपयोगिनी। यथा सोपानमन्तरा कोऽपि प्रासादमारोढुं न शक्नोति तथैव शब्द\-ब्रह्म\-ज्ञानमन्तरेण कश्चनापि पर\-ब्रह्म न साक्षात्कर्तुमीष्टे। अतो गुरवः पठन्ति~–\end{sloppypar}
\centering\textcolor{red}{शब्दब्रह्मणि निष्णातः परं ब्रह्माधिगच्छति॥}\nopagebreak\\
\raggedleft{–~ब्र॰उ॰~१७}\\
\begin{sloppypar}\hyphenrules{nohyphenation}\justifying\noindent\hspace{10mm} वस्तुतस्तु स्फोट एव मुख्यः। स चाष्ट\-विधः।\footnote{\textcolor{red}{वर्णस्फोटः पदस्फोटो वाक्य\-स्फोटोऽखण्ड\-पदवाक्य\-स्फोटौ वर्णपदवाक्य\-भेदेन त्रयो जातिस्फोटा इत्यष्टौ पक्षाः सिद्धान्त\-सिद्धाः} (वै॰भू॰सा॰~१४.६१)।} यथा स्वच्छं स्फटिकं जपाकुसुम\-संयोगे तद्गत\-रक्तिम्ना रक्ततामुपैति तथा निर्मलं चैतन्यमात्मा
कत्व\-गत्वादि\-ध्वनि\-रूप\-रूषितान्तः\-करणावच्छिन्नः सन् स्फोट\-सञ्ज्ञां लभते। अयमेव मुख्यः। अत्र व्युत्पत्ति\-द्वयं \textcolor{red}{स्फुटँ विकसने} (धा॰पा॰~२६०, १३७३) इत्यस्माद्धातोः \textcolor{red}{स्फुट्यते प्रकाश्यते} इति कर्म\-व्युत्पत्तौ कर्मणि घञ्।\footnote{\textcolor{red}{अकर्तरि च कारके सञ्ज्ञायाम्} (पा॰सू॰~३.३.१९) इत्यनेन।} अनुबन्धकार्ये गुणे\footnote{\textcolor{red}{पुगन्त\-लघूपधस्य च} (पा॰सू॰~७.३.८६) इत्यनेन।} विभक्तिकार्ये च स्फोटः। द्वितीये च \textcolor{red}{स्फुटत्यर्थो येन} इति विग्रहे बाहुलकाल्ल्युटं प्रबाध्य पुनः करणे घञ्।\footnote{सोऽपि \textcolor{red}{अकर्तरि च कारके सञ्ज्ञायाम्} (पा॰सू॰~३.३.१९) इत्यनेन।} इत्थं \textcolor{red}{ध्वनि\-व्यङ्ग्यत्वे सत्यर्थ\-विषयक\-बोध\-जनकतावच्छेदकत्वं स्फोटत्वम्} इति लक्षणम्। इमे द्वे व्युत्पत्ती भाष्य\-प्रदीप\-सम्मते। अथ \textcolor{red}{गौरित्यत्र कः शब्दः} इति जिज्ञासायां \textcolor{red}{येनोच्चारितेन सास्ना\-लाङ्गूल\-ककुद\-खुर\-विषाणिनां सम्प्रत्ययो भवति} (भा॰प॰) इति सिद्धान्तितं भगवता भाष्यकृता। अत्रोच्चारितेनेत्यस्य व्याख्यानं व्याचक्षते कैयटोपाध्याया~– \textcolor{red}{उच्चारितेन प्रकाशितेनेत्यर्थः}।\footnote{अत्र नागेशभट्ट\-पादाश्च~– \textcolor{red}{प्रकाशितेनेत्यभि\-व्यञ्जकैरीति शेषः}।} यथा तत्रत्य\-भाष्य\-प्रदीपौ। एतस्य वाक्यपदीय\-वैयाकरण\-भूषण\-सार\-वैयाकरण\-सिद्धान्त\-मञ्जूषा\-वैयाकरण\-सिद्धान्त\-परम\-लघु\-मञ्जूषादौ सविस्तरं चर्चा।\end{sloppypar}
\begin{sloppypar}\hyphenrules{nohyphenation}\justifying\noindent\hspace{10mm} अस्याञ्चाष्टाध्याय्यां प्रत्यध्यायं चत्वारः पादाः। प्रायश्चतुःसहस्रशो विचित्राणि सूत्राणि। अहो आश्चर्यमेतत्। कोटि\-कोटि\-प्रयोगाणां किं बहुना निखिलस्यापि वाङ्मय\-वारिधेश्चतुःसहस्र\-सूत्रैरेवानुशासनमित्येवास्य व्याकरणस्य मुख्यं वैशिष्ट्यम्। अत एव सर्व\-मान्यम्। एकैक\-सूत्रे विविध\-विषयाणां समन्वयः। स च शास्त्र\-विशेषः शब्दानुशासनं कुर्वन्नपि मानव\-मनो\-वृत्तिमपि व्याचष्टे। यथा सूत्रं स्पष्टं \textcolor{red}{श्वयुव\-मघोनामतद्धिते} (पा॰सू॰~६.४.१३३)। तद्धितं विहाय यजादि\-प्रत्यये परे\footnote{भसञ्ज्ञायामिति भावः। \textcolor{red}{यचि भम्} (पा॰सू॰~१.४.१८) इत्यनेन यजादि\-प्रत्यये परे पूर्वस्य भसञ्ज्ञा भवति।} \textcolor{red}{श्वन् युवन् मघवन्} शब्दाः सम्प्रसारणं लभन्ते। यद्यप्यनेन \textcolor{red}{शुनः यूनः मघोनः} इत्यादयः प्रयोगाः सिध्यन्ति तथाऽपि सहोक्त्या त्रयाणां प्रवृत्ति\-साम्यमपि प्रतीयते। तद्यथा श्वा कुक्कुरो युवा युवको मघवेन्द्र इमे त्रयः समानमेव विषय\-लोलुपाः कामुकाः स्वार्थान्धाश्च। अतस्तुलसी\-दासो रामचरितमानसेऽपि~–\end{sloppypar}
\centering\textcolor{red}{लखि हिय हँसि कह कृपानिधानू। सरिस श्वान मघवान जुबानू॥}\footnote{एतद्रूपान्तरम्–\textcolor{red}{दशां वीक्ष्य विहस्यापि हृद्यवोचत्कृपानिधिः। मघवा श्वा युवा चैव वर्तन्ते समतान्विताः॥} (मा॰भा॰~२.३०२.८)।}
\nopagebreak\\
\raggedleft{–~रा॰च॰मा॰~२.३०२.८}\\
\begin{sloppypar}\hyphenrules{nohyphenation}\justifying\noindent इममेवार्थं प्रतिपादयन् कविरेकोऽकथयद्यत्~–\end{sloppypar}
\centering\textcolor{red}{काचं मणिं काञ्चनमेकसूत्रे ग्रथ्नासि बाले किमु चित्रमेतत्।\nopagebreak\\
अशेषवित्पाणिनिरेकसूत्रे श्वानं युवानं मघवानमाह॥}\nopagebreak\\
\raggedleft{–~मौक्तिकम्}\\
\begin{sloppypar}\hyphenrules{nohyphenation}\justifying\noindent एवमेव बहुत्र व्यवहार\-पक्षस्याप्रत्यक्ष\-रूपेण चर्चा कृता वर्तते। यथा सावर्ण्यं व्याचक्षाणः पाणिनिः \textcolor{red}{तुल्यास्य\-प्रयत्नं सवर्णम्} (पा॰सू॰~१.१.९)। सत्यपि सवर्ण\-सञ्ज्ञा\-विधायकेऽस्मिन्नन्योऽप्यर्थो निर्गलति यत्तयोरेव वर्ण\-साम्यं ययोरास्य\-प्रयत्नावर्थादाकार\-प्रकारौ सदृशौ भवेताम्। \end{sloppypar}
\begin{sloppypar}\hyphenrules{nohyphenation}\justifying\noindent\hspace{10mm} इत्थमेव प्रातिपदिक\-सञ्ज्ञा\-विधायक\-सूत्र\-विषयेऽप्येका किंवदन्ती प्रहेलिका यत्~–\end{sloppypar}
\centering\textcolor{red}{धीरः कीदृग्वचो ब्रूते को रोगी कश्च नास्तिकः। \nopagebreak\\
कीदृक्चन्द्रं न पश्यन्ति तत्सूत्रं पाणिनेर्वद॥}\nopagebreak\\
\raggedleft{–~मौक्तिकम्}\\
\begin{sloppypar}\hyphenrules{nohyphenation}\justifying\noindent इति प्रश्ने। अर्थाद्धीरोऽर्थवद्वचो ब्रूते। अधातू रोगी भवति। अप्रत्ययो नास्तिकः कथ्यते। प्रातिपदिकं चन्द्रं न पश्यन्ति। सम्पूर्णं सूत्रं चतुर्णामपि प्रश्नानामुत्तर\-रूपं यत् \textcolor{red}{अर्थवदधातुरप्रत्ययः प्रातिपदिकम्} (पा॰सू॰~१.२.४५)। \end{sloppypar}
\begin{sloppypar}\hyphenrules{nohyphenation}\justifying\noindent\hspace{10mm} सङ्केतेनाव्युत्पन्नस्य चापि व्याख्या कृता। अव्युत्पन्नः प्रायशोऽर्थवान् धातु\-रहित ईश्वरे प्रत्यय\-रहितो भवति। शास्त्रे यद्यपि डित्थ\-डवित्थ\-साम्प्रतिक\-नाम\-शब्दानामेवाव्युत्पन्नत्वम्। तदर्थमेव सूत्रमिदम्। किन्तु व्यवहारेऽप्यनेनैवाव्युत्पन्न\-लक्षणं सङ्गमयितुं शक्यते। व्युत्पन्न\-प्रातिपदिक\-सञ्ज्ञा\-विधायकं सूत्रं \textcolor{red}{कृत्तद्धितसमासाश्च} (पा॰सू॰~१.२.४६)। अनेन \textcolor{red}{कर्ता वाराणसेयः रामदासः} इत्यादौ कृत्तद्धित\-समासानां प्रातिपदिक\-सञ्ज्ञा। तत्रापि व्यवहार\-व्युत्पन्न\-लक्षणं द्रष्टव्यम्। कृदर्थाद्यः सक्रियः। तद्धितोऽर्थात्तस्मै हितः परोपकारी। समासा अर्थात्समन्वय\-वादिनः। त एव व्युत्पन्नाः। यद्यपि शाकटायन\-मते सर्वमपि प्रातिपदिकं व्युत्पन्नमर्थाद्धातुजं प्रकृति\-प्रत्यय\-विभाग\-पूर्वकम्। यथोक्तम्~–\end{sloppypar}
\centering\textcolor{red}{नाम च धातुजमाह निरुक्ते व्याकरणे शकटस्य च तोकम्।\nopagebreak\\
यन्न विशेषपदार्थसमुत्थं प्रत्ययतः प्रकृतेश्च तदूह्यम्॥}\nopagebreak\\
\raggedleft{–~भा॰पा॰सू॰~३.३.१}\\
\begin{sloppypar}\hyphenrules{nohyphenation}\justifying\noindent किन्तु भगवान् पाणिनिर्महा\-सञ्ज्ञामन्वर्थां मन्यते। अतोऽर्थमविचार्यार्वाचीन\-विहित\-नामसु भाषान्तरीय\-शब्देषु च व्युत्पत्तिमस्वीकुर्वन् तेषां साधुत्वार्थ\-विभक्ति\-प्रतिपत्तयेऽव्युत्पन्न\-प्रातिपदिकं स्वीकृत्योक्त्वा चान्वर्थं नाम रामादीनां साधुत्वाय व्युत्पन्न\-प्रातिपदिक\-सञ्ज्ञार्थं कृत्तद्धितेति सूत्रं सूत्रयामास। \textcolor{red}{अव्युत्पन्न\-प्रातिपदिकत्वं नाम धातु\-प्रत्यय\-प्रत्ययान्त\-रहितत्वे सति लोकेऽर्थ\-बोधकत्वे सति प्रातिपदिक\-सञ्ज्ञावत्त्वम्}। यथा डित्थादौ। \textcolor{red}{व्युत्पन्न\-प्रातिपदिकत्वं नाम प्रकृति\-प्रत्यय\-जन्य\-लौकिकार्थ\-बोधकत्वे सति कृत्तद्धित\-समासान्यतमत्वे सति प्रातिपदिक\-सञ्ज्ञा\-भाक्त्वम्}। यथा कस्यचित्कुरूपस्य नाम मदन\-मोहन इति। व्युत्पत्तिः क्रियते मदनं कामं मोहयति। तर्हि कुरूप इयं व्युत्पत्तिर्घटिष्यते। अत एतादृशेषु स्थलेषु सम्भवायां व्युत्पत्तौ व्युत्पत्त्यनुसारमर्थाभावे प्रकृति\-प्रत्ययार्थ\-कल्पना\-त्याग एव स्वीकार्य इति। रूढानां पूर्वं कथित\-स्थलानां कृते चाव्युत्पन्न\-प्रातिपदिकतैवेति पाणिनि\-मनीषितं मे प्रतिभाति। व्युत्पन्न\-प्रातिपदिकता च रामादीनामन्वर्थानां कृते। यथा \textcolor{red}{रामः}। \textcolor{red}{रमन्ते योगिनो यस्मिन् स रामः} अथवा \textcolor{red}{रमयति सर्वाणि भूतानि यः स रामः} अथवा \textcolor{red}{रमते सर्वेषु भूतेषु यः स रामः} इत्यादयः सहस्रशोऽपि व्युत्पत्तयः सार्था अर्थापयितुं शक्यन्ते। राम\-शब्दस्य वाच्यतावच्छेदकत्वं लक्ष्यतावच्छेदकत्वञ्च परमात्मनि सच्चिदानन्द\-घने पर\-ब्रह्मणि दशरथात्मजे। अतोऽस्य कृते व्युत्पन्न\-प्रातिपदिकता। यथा कस्यचिद्भोजनार्थं म्रियमाणस्य परम\-दरिद्रस्य पुत्रस्य नाम \textcolor{red}{राज\-कुमारः} इति। सत्यपि सुलभतया तत्पुरुष\-समास\-सम्भवे \textcolor{red}{राज्ञः कुमारः} इति दरिद्र\-कुमारे जन्य\-जनक\-भाव\-सम्बन्धावच्छिन्न\-राज\-प्रतियोगि\-कुमारता नान्वर्था। अत एतत्कृते \textcolor{red}{अर्थवत्} (पा॰सू॰~१.२.४५) इति सूत्रमेव। रामस्तु धातूनां धातुः प्रत्ययानाञ्च प्रत्ययस्तस्य कृतेऽधातु\-घटित\-सूत्रं सञ्ज्ञार्थं न मे रोचते।\end{sloppypar}
\begin{sloppypar}\hyphenrules{nohyphenation}\justifying\noindent\hspace{10mm} अर्थवदिति सूत्रं चतुष्पदम्। एतत्सूत्रेण चतुष्पदेन चतुष्पद\-तुल्यानामव्युत्पन्नानामेव सञ्ज्ञा करणीया। तथा च कृत्तद्धितेऽति सूत्रञ्च श्रूयते चतुष्पदम्। पूर्व\-सूत्रतोऽर्थवत्प्रातिपदिकञ्चेति द्वे पदे अनुवर्त्येते। एवं श्रुतानुवृत्त\-सम्मेलनेन षट्पदम्~–\end{sloppypar}
\begin{sloppypar}\hyphenrules{nohyphenation}\justifying\noindent\hspace{10mm} (१) कृत्\end{sloppypar}
\begin{sloppypar}\hyphenrules{nohyphenation}\justifying\noindent\hspace{10mm} (२) तद्धित\end{sloppypar}
\begin{sloppypar}\hyphenrules{nohyphenation}\justifying\noindent\hspace{10mm} (३) समासाः\end{sloppypar}
\begin{sloppypar}\hyphenrules{nohyphenation}\justifying\noindent\hspace{10mm} (४) च\end{sloppypar}
\begin{sloppypar}\hyphenrules{nohyphenation}\justifying\noindent\hspace{10mm} (५) अर्थवत्\end{sloppypar}
\begin{sloppypar}\hyphenrules{nohyphenation}\justifying\noindent\hspace{10mm} (६) प्रातिपदिकम् \end{sloppypar}
\begin{sloppypar}\hyphenrules{nohyphenation}\justifying\noindent षडैश्वर्यञ्च~–\end{sloppypar}
\centering\textcolor{red}{ऐश्वर्यस्य समग्रस्य धर्मस्य यशसः श्रियः। \nopagebreak\\
ज्ञानवैराग्ययोश्चैव षण्णां भग इतीरणा॥}\nopagebreak\\
\raggedleft{–~वि॰पु॰~६.५.७४}\\
\begin{sloppypar}\hyphenrules{nohyphenation}\justifying\noindent अतः षडैश्वर्य\-सम्पन्नस्य रामस्य भगवतो वाचकस्य \textcolor{red}{रामः} इति शब्दस्य द्वितीय\-सूत्रेण प्रातिपदिक\-सञ्ज्ञा करणीया। यद्वा समास\-महिम्ना \textcolor{red}{कृत्तद्धित\-समासाः} इत्येकं पदं \textcolor{red}{च} इति द्वितीयम्। अतो द्विपद\-सूत्रेण द्विपदं मनुष्यमनुकुर्वतो रामभद्रस्य वाचकस्य \textcolor{red}{रामः} इति शब्दस्य द्वितीय\-सूत्रेणैव प्रातिपदिक\-सञ्ज्ञा मेऽतिरुचि\-करा लगति। द्विपद\-सूत्रेण सीता\-समेत\-रामचन्द्र\-वाचकस्य \textcolor{red}{रामः} इति शब्दस्य द्वितीय\-सूत्रेण प्रातिपदिक\-सञ्ज्ञा युक्ति\-युक्ता भक्ति\-सहकृता हृदय\-रमणीया च। तस्मात्प्रथम\-सूत्रेणार्वाचीन\-नाम्नां भाषान्तरीय\-शब्दानां डित्थादीनां रूढतावच्छेदकवतां प्रातिपदिक\-सञ्ज्ञा राम\-मुख्यानां च द्वितीय\-सूत्रेण। एवं लघु\-सिद्धान्त\-कौमुदी\-वैयाकरण\-सिद्धान्त\-कौमुद्यादावर्थवत्सूत्रोदाहरणं राम\-कृष्ण\-मुकुन्दादि किमपि चेतस्तुदति। कदाचिदिमान्यर्वाचीन\-जन\-साधारण\-वाचकानि तदा सुष्ठु। किं बहुना। दशरथापत्य\-ब्रह्म\-वाचक\-राम\-शब्दमव्युत्पन्न\-प्रातिपदिकमित्यङ्गीकर्तुमहमाशिरसि च्छेदमपि नोत्सह इति विद्वांसो बाल\-चापलं क्षमन्ताम्। एवं\-विधानि बहूनि स्थलानि सन्ति येषु पदे पदे व्यवहारिकता सामाजिकता वैज्ञानिकताऽऽध्यात्मिकता च। किं बहुना। अष्टाध्याय्या राम\-कथया समन्वयः। यथा राम\-कथाया वृद्धौ तात्पर्यं रामायणस्यान्ते कवि\-कोकिलो भगवान् वाल्मीकिः कथयति~–\end{sloppypar}
\centering\textcolor{red}{बलं विष्णोः प्रवर्धताम्॥}\nopagebreak\\
\raggedleft{–~वा॰रा॰~६.१२८.१२१}\\
\begin{sloppypar}\hyphenrules{nohyphenation}\justifying\noindent एवमेव भगवान् पाणिनिरप्यष्टाध्याय्याः प्रथमं सूत्रं लिखन् वृद्धि\-शब्दं रामायणस्य सार\-रूपं महा\-मन्त्रं स्मरति। \textcolor{red}{वृद्धिरादैच्} (पा॰सू॰~१.१.१) इति। \textcolor{red}{भूवादयो धातवः} (पा॰सू॰~१.३.१) इत्यत्र भाष्यं भाषमाणाः पतञ्जलयः प्राहुर्यत् \textcolor{red}{मङ्गलादीनि मङ्गलमध्यानि मङ्गलान्तानि शास्त्राणि प्रथन्ते वीरपुरुषाणि भवन्त्यायुष्मत्पुरुषाणि चाध्येतारश्च सिद्धार्था यथा स्युरिति} (भा॰पा॰सू॰~१.३.१)। माङ्गलिक आचार्यो मङ्गलार्थं वृद्धिशब्दं प्रयुङ्क्ते। तत्रेत्थं विचारश्चलितो यद्वृद्धिः सञ्ज्ञाऽऽदैच्च सञ्ज्ञी। नियमोऽयमुद्देश्यं पूर्वमतः कथयन्त्याचार्याः \textcolor{red}{उद्देश्य\-शब्दः पूर्वं विधेयश्च ततः परम्}। किन्त्वत्र कथं विधेयं पूर्वमुद्देश्यं परमिति। तदेत्थं समाधानं कर्तुं शक्यते यदुद्देश्यस्य पूर्वं प्रयोगः प्रायिकः। यथा \textcolor{red}{अदेङ्गुणः} (पा॰सू॰~१.१.२) अत्रोद्देश्यः पूर्वं विधेयश्च परम्।\end{sloppypar}
\begin{sloppypar}\hyphenrules{nohyphenation}\justifying\noindent\hspace{10mm} पुनः विप्रतिपत्तिर्यथा \textcolor{red}{घु टि भ घि} इत्यादयो लघवः सञ्ज्ञाः। तथैव कथं नात्र लघु\-सञ्ज्ञा किं महा\-सञ्ज्ञया। तदाऽऽचार्यो मङ्गलार्थमिति समुच्चारयामास। तच्च वृद्धिरूपं मङ्गलं रामकथायाः सिद्धान्त\-भूतम्। अष्टाध्यायी राम\-चरित\-मानसञ्चोभावपि ग्रन्थौ वकारतः प्रारब्धौ।\footnote{\textcolor{red}{वृद्धिरादैच्} (पा॰सू॰~१.१.१) इत्यनेन सूत्रेणाष्टाध्यायी प्रारब्धा। \textcolor{red}{वर्णानामर्थ\-सङ्घानां रसानां छन्दसामपि} (रा॰च॰मा॰~१/म॰श्लो॰१) इत्यनेनानुष्टुपा रामचरितमानसं प्रारब्धम्। अयं ग्रन्थोऽपि वकारतः प्रारब्धः।} वकारञ्चामृत\-बीजम्।\footnote{यथा शिवपुराणे~– \textcolor{red}{शं नित्यं सुखमानन्दमिकारः पुरुषः स्मृतः॥ वकारः शक्तिरमृतं मेलनं शिव उच्यते। तस्मादेवं स्वमात्मानं शिवं कृत्वाऽर्चयेच्छिवम्॥} (शि॰पु॰~१८.७६–७७)। तन्त्रशास्त्रेऽपि~– \textcolor{red}{“ब्रह्मरन्ध्रे यवादूर्ध्वं कुलपद्मं महेश्वरि। श्वेतं सुकेसरोपेतं सहस्रारमधोमुखम्॥” इत्यादिना “व्यापिनी केवला शक्तिरमृतौघप्रवर्षिणी।” इत्यन्तेन स्वच्छन्द\-सङ्ग्रहोक्त\-रीत्याऽमृतमयो वकारः} (यो॰हृ॰ दी॰टी॰~३.१३७)।} सम्पूर्णेऽस्मिन् पाणिनीये शब्द\-ब्रह्मामृतस्य चर्चा। \textcolor{red}{न म्रियत इत्यमृतम्}। \textcolor{red}{मृङ् प्राण\-त्यागे} (धा॰पा॰~१४०३) इत्यस्माद्धातोः कर्तरि क्तो नञ्समासश्च। शब्द\-ब्रह्म जन्म\-मरण\-रहितं यथा~–\end{sloppypar}
\centering\textcolor{red}{अनादिनिधनं ब्रह्म शब्दतत्त्वं यदक्षरम्।\nopagebreak\\
विवर्ततेऽर्थभावेन प्रक्रिया जगतो यतः॥}\nopagebreak\\
\raggedleft{–~वा॰प॰~१.१}\\
\begin{sloppypar}\hyphenrules{nohyphenation}\justifying\noindent एवमादौ वृद्धिरूपं मङ्गलाचरणं कृत्वा अन्ते \textcolor{red}{अ अ} (पा॰सू॰~८.४.६८) इति विलिख्य \textcolor{red}{अकारो वासुदेवः} इति श्रुत्यनुसारं पुनरकार\-वाच्यं वासुदेवं श्रीरामचन्द्रं एव स्मरन् विरमति रामे। अस्यां प्रत्यध्यायं चत्वारः पादाः समग्रस्य वाङ्मयस्य सङ्क्षिप्त\-परिचयो भौगोलिक\-परिस्थितेः परिशीलनमार्ष\-चक्षुषा शब्दानां परिलोकनम्। यथा \textcolor{red}{उदक्च विपाशः} (पा॰सू॰~४.२.७४) इत्यादि।\end{sloppypar}
\begin{sloppypar}\hyphenrules{nohyphenation}\justifying\noindent\hspace{10mm} किं नाम सूत्रत्वमित्यपेक्षायाम् \textcolor{red}{अल्पाक्षरत्वे सति बह्वर्थ\-बोधकत्वम्} इति।\footnote{पूर्वपक्षोऽयम्।} अस्मिल्लँक्षणे स्वीकृते \textcolor{red}{हरि}\-शब्देऽति\-व्याप्तिः। तत्राऽप्यल्पाक्षर\-त्वादिन्द्र\-सूर्य\-सर्प\-सिंह\-विष्णु\-प्रभृति\-बह्वर्थ\-बोधकत्वाच्च।\footnote{\textcolor{red}{यमानिलेन्द्र\-चन्द्रार्क\-विष्णु\-सिंहांशु\-वाजिषु॥ शुकाहि\-कपि\-भेकेषु हरिर्ना कपिले त्रिषु।} (अ॰को॰~३.३.१७४-१७५) इत्यमरः। अपि च~– \textcolor{red}{हरिर्विष्णावहाविन्द्रे भेके सिंहे हये रवौ। चन्द्रे कोले प्लवङ्गे च यमे वाते च कीर्तितः। वारि वारिदके वाऽपि नवपञ्चार्थकः स्मृतः॥} (मूलं मृग्यम्)।} हरि\-शब्दस्यानेकार्थ\-तामाशङ्क्य कवि\-कुल\-गुरुः कविता\-कामिनी\-विलासो महा\-कवि\-कालिदासः स्वकीय\-रघुवंश\-महा\-काव्यस्य त्रयोदशे सर्गे प्रथमे श्लोके पुष्पकारूढ\-रामं सीतायायाशंसन्तं समुद्रं हरि\-शब्दस्य \textcolor{red}{रामाभिधानः} इति विशेषणं प्रयुञ्जानस्तं दशरथापत्य\-श्रीराम\-रूप\-ब्रह्म\-वाचकं व्यवस्थापयन्नितरार्थेभ्यो व्यावर्तयति~–\end{sloppypar}
\centering\textcolor{red}{अथात्मनः शब्दगुणं गुणज्ञः पदं विमानेन विगाहमानः।\nopagebreak\\
रत्नाकरं वीक्ष्य मिथः स जायां रामाभिधानो हरिरित्युवाच॥}\nopagebreak\\
\raggedleft{–~र॰वं॰~१३.१}\\
\begin{sloppypar}\hyphenrules{nohyphenation}\justifying\noindent \textcolor{red}{रामाभिधानः} इति विशेषणमेव हरि\-शब्दस्यानेकार्थत्वं प्रमाणयति। तस्मादुक्तं लक्षणमत्रातिव्याप्तम्। \textcolor{red}{तदेव हि लक्षणं यदव्याप्त्यति\-व्याप्त्यसम्भवरूप\-दोष\-त्रय\-शून्यम्}। तथा च  \textcolor{red}{अव्याप्त्यति\-व्याप्त्यसम्भव\-रूप\-दोष\-त्रय\-शून्यत्वे सत्यसाधारण\-धर्मत्वं लक्षणत्वम्}। अस्यैक\-देशावृत्तित्वमव्याप्तित्वम्। यथा कपिलत्वं गोत्वम्। कपिलत्वं हि लक्ष्यस्य गोः सकल\-देशे न वर्तत इत्यव्याप्तिः। लक्ष्य\-वृत्तित्वे सत्यलक्ष्य\-वृत्तित्वमतिव्याप्तित्वम्। यथा शृङ्गित्वं गोत्वम्। शृङ्गित्वं हि गवि च लक्ष्येऽलक्ष्य\-भूते गवेतरे महिषादौ चातितिष्ठतीत्यतिव्याप्तम्। असम्भवत्वं लक्ष्यमात्रावृत्तित्वम्। यथा पुष्पवत्त्वमाकाशत्वम्। लक्ष्य\-भूत आकाशे पुष्पाभावादसम्भवमत्र। इत्थमल्पाक्षरत्वे सति बह्वर्थ\-बोधकत्वं सूत्रत्वमिति सूत्र\-लक्षणस्य हरि\-शब्दोऽतिव्याप्तत्वे स्वरूपं लक्षयामः~–\end{sloppypar}
\centering\textcolor{red}{अल्पाक्षरमसन्दिग्धं सारवद्विश्वतोमुखम्। \nopagebreak\\
अस्तोभमनवद्यञ्च सूत्रं सूत्रविदो विदुः॥}\nopagebreak\\
\raggedleft{–~परा॰उ॰~१८.१३,१४}\\
\begin{sloppypar}\hyphenrules{nohyphenation}\justifying\noindent\hspace{10mm} इत्थम् \textcolor{red}{असन्दिग्ध\-बह्वर्थ\-बोधकाल्पाक्षरत्वे सति सिद्धान्त\-प्रतिपादकत्वं सूत्रत्वम्} इति मे शिशु\-मतिः। इमानि सूत्राणि षड्विधानि। सञ्ज्ञा\-सूत्रं परिभाषा\-सूत्रं विधि\-सूत्रं नियम\-सूत्रमतिदेश\-सूत्रमधिकार\-सूत्रञ्च। \textcolor{red}{सञ्ज्ञा\-सूत्रत्वं नाम साक्षाच्छक्ति\-ग्राहकत्वे सति पाणिन्युच्चरितत्वम्}। यथा \textcolor{red}{वृद्धिरादैच्} (पा॰सू॰~१.१.१) \textcolor{red}{अदेङ्गुणः} (पा॰सू॰~१.१.२) इत्यादि। अत्रादैचि वृद्धिरूपा साक्षाच्छक्तिर्ग्राहिता। आदैज्वृद्धि\-निष्ठ\-शक्तिमान् भवत्विति। \textcolor{red}{परिभाषा\-सूत्रत्वं नामानियमे नियमकारित्वम्}। यथा \textcolor{red}{सुधी उपास्य} इति स्थिते \textcolor{red}{इको यणचि} (पा॰सू॰~६.१.७७) इत्यनेन यणि विधीयमानेऽनियमः। \textcolor{red}{अचि इकः यण् स्यात्} इत्येव सूत्रार्थः। अत्र षष्ठ्याः कोऽर्थः कोऽनुयोगी कः प्रतियोगी यतो हीक्शब्दस्य व्यवहार\-जडस्य केन सम्बन्ध इत्यनियमे \textcolor{red}{षष्ठी स्थानेयोगा} (पा॰सू॰~१.१.४९) इति परिभाषा\-सूत्रमागतम्। \textcolor{red}{स्थानेयोगा} इत्यत्र स्थानेन प्रसङ्गेन योगो यस्याः सा। एकार आर्षः। अथवा \textcolor{red}{कण्ठेकालः} इतिवत् \textcolor{red}{स्थाने योगो यस्याः सा} इति सप्तम्या अलुक्। अनिर्धारित\-सम्बन्ध\-विशेषा षष्ठी स्थानानुयोगिक\-सम्बन्धार्थवती भवेदिति तात्पर्यम्। सम्बन्धश्च विषय\-विषयि\-भावः। इत्थं \textcolor{red}{स्थाने} इति पदेन \textcolor{red}{इकः स्थाने यण्} इत्यर्थः। \textcolor{red}{अचि} इत्यत्र पुनरनियमः। \textcolor{red}{अचि} इत्यत्र सप्तमी। सा च \textcolor{red}{सप्तम्यधिकरणे च} (पा॰सू॰~२.३.३६) इति सूत्रेणाधिकरणे। अधिकरणं नाम \textcolor{red}{आधारोऽधिकरणम्} (पा॰सू॰~१.४.४५) इति सूत्रेणाधारनामकम्। आधारश्च यथाऽभिव्यापक औपश्लेषिको वैषयिकश्च। यत्र सम्पूर्णमाधेयमाधारो व्याप्नोति तत्रैवाभिव्यापकः। यथा दध्नि घृतम्। सर्वस्मिन् आत्मा। वैषयिको यदाऽऽधार आधेयं विषयतया गृह्णाति। यथा मोक्ष इच्छा। रामे प्रेम। गुरौ श्रद्धा। चरित्रे निष्ठा। सिद्धान्ते दृढता। भक्तौ हठः। संसारे नीरसतेत्यादि। औपश्लेषिको यदाऽऽधार आधेयमुपश्लिष्यति तेन सह सम्बद्धो भवति। उपश्लेषश्च संयोगेन समवायेन सामीप्येन। संयोगेन यथा \textcolor{red}{कटे शेते}। समवायेन यथा \textcolor{red}{शरीरे चेष्टा}। सामीप्येन यथा \textcolor{red}{गुरौ वसति}।\footnote{गुरोः समीपे वसतीत्यर्थः।} अतः \textcolor{red}{अचि} इत्यत्रौपश्लेषिकी। सा च संयोगात्मिका। इत्थम् \textcolor{red}{अजुपश्लिष्टस्येकः स्थाने यण् स्यात्} इत्यर्थः। पुनरनियमः सम्बन्धस्तु संयोगात्मकः पूर्वेण परेण च भवति तदा कुत्र यण् यथा \textcolor{red}{सुधी उपास्य} इत्यत्र सुघटक उकार इत्यच्पुनर्धी\-घटक ईकारोऽच्पुनर्धी\-घटक ईकार इगुपास्य\-घटक उकारोऽच्पुनरुपास्य\-घटक उकार इक्पाकार\-घटक आकारोऽजित्थं व्यवहितेऽव्यवहिते पूर्वत्र परत्र च यणि प्रसक्ते नियमः कृतः परिभाषया। अच्यव्यवहित उच्चरिते पूर्वस्याव्यवहितस्यैव। यथा सूत्रं \textcolor{red}{तस्मिन्निति निर्दिष्टे पूर्वस्य} (पा॰सू॰~१.१.६६)। अस्यार्थः सप्तमी\-निर्देशेन विधीयमानं कार्यं वर्णान्तरेणाव्यवहितस्य पूर्वस्य बोध्यम्। पुनरनियमश्चतुर्षु यण्सु को भवेत्तदा परिभाषा\-सूत्रं न्ययमयत्। \textcolor{red}{स्थानेऽन्तरतमः} (पा॰सू॰~१.१.५०)। स्थानञ्च प्रसङ्गः। तस्मिन् सत्यन्तरतम आदेशः स्यात्। यद्यप्यान्तरतम्यं साधयितुं शक्यम्। तत्र हि स्थान\-कृतमान्तर्यं यथा \textcolor{red}{कृष्णैकत्वम्}। अत्राकारैकारयोः स्थानिनोः सदृशतमः कण्ठतालु\-स्थानी \textcolor{red}{ऐ} इति। द्वितीयमर्थ\-कृतमान्तर्यम्।
क्रोष्टु\-शब्दोऽर्थद्वये प्रसिद्धो राजर्षौ\footnote{\textcolor{red}{क्रोष्टोः शृणुत राजर्षेर्वंशमुत्तमपूरुषम्॥ यस्यान्ववाये संभूतो वृष्णिर्वृष्णिकुलोद्वहः।} (ब्रह्मा॰पु॰~३.७०.१४-१५)। } वृके\footnote{\textcolor{red}{स्त्रियां शिवा भूरिमायगोमायुमृगधूर्तकाः। शृगाल\-वञ्चक\-क्रोष्टु\-फेरुफेरव\-जम्बुकाः॥} (अ॰को॰~२.५.५)।} च। 
तत्र \textcolor{red}{तृज्वत्क्रोष्टुः} (पा॰सू॰~७.१.९५) इत्यत्रार्थ\-कृतान्तर्यानुरोधेन शृगाल\-वाचक\-क्रोष्टु\-शब्द एवाऽदेशत्वेन जातः। गुण\-कृतमान्तर्यं यथा \textcolor{red}{वाग्घरिः} इत्यत्र \textcolor{red}{वाग् हरिः} इत्यवस्थायां \textcolor{red}{झयो होऽन्यतरस्याम्} (पा॰सू॰~८.४.६२) इत्यनेन हकारस्य पूर्व\-सवर्णे प्राप्ते पूर्वत्र गकारस्तस्य सन्ति चत्वारः सवर्णाः स्वयं च मिलित्वा पञ्च हकारस्य स्थाने पञ्चसु को भवेत्तदा गुण\-कृतान्तर्यानुरोधेन नाद\-घोष\-संवार\-महा\-प्राणवतो हकारस्य स्थाने पूर्वत्र पञ्चसु तादृङ्नाद\-घोष\-संवार\-महा\-प्राणवाञ्चतुर्थो घकारः। प्रमाण\-कृतमान्तर्यं यथा \textcolor{red}{अदसोऽसेर्दादु दो मः} (पा॰सू॰~८.२.८०)। अत्र क्रमशो ह्रस्वस्य स्थाने ह्रस्व उकारो दीर्घस्य स्थाने दीर्घः \textcolor{red}{अमू} इति सिद्धमिदं प्रमाण\-कृतमान्तर्यम्। एषु दर्शितेषु चतुर्विधेष्वान्तर्येषु \textcolor{red}{सुधी उपास्य} इत्यत्र किं स्यादित्यपेक्षायां \textcolor{red}{यत्रानेक\-विधमान्तर्यं तत्र स्थानत आन्तर्यं बलीयः} (वै॰सि॰कौ॰~३९) इति परिभाषयेतरेषु व्यावर्तितेषु स्थान\-कृतमान्तर्यमालम्ब्य चतुर्षु यण्स्विकार\-सदृश\-तालु\-स्थानवान् यकारः। इत्थं परिभाषा\-त्रय\-योग\-दानेन \textcolor{red}{इको यणचि} (पा॰सू॰~६.१.७७) इत्यस्य निष्कृष्टोऽर्थः \textcolor{red}{अजुपश्लिष्ट\-पूर्वत्व\-विशिष्टस्येकः स्थानेऽन्तरतमो यण् स्यात्स च प्रयोक्तव्यः स्यात्स च साधु स्यात्}। एवमिग्घटित\-स्थाने यण्घटितः स्यात्स च प्रयोक्तव्यः स च साधु इति शब्द\-नित्यत्व\-पक्षीयोऽर्थः। \textcolor{red}{विधि\-सूत्रं नाम मुख्य\-लक्ष्य\-संस्कारकमर्थाल्लक्ष्य\-संस्कारे मुख्यतया सहायकतावच्छेदकम्}। यथा \textcolor{red}{इको यणचि} (पा॰सू॰~६.१.७७) \textcolor{red}{आद्गुणः} (पा॰सू॰~६.१.८७) \textcolor{red}{वृद्धिरेचि} (पा॰सू॰~६.१.८८) \textcolor{red}{अकः सवर्णे दीर्घः} (पा॰सू॰~६.१.१०१) \textcolor{red}{मोऽनुस्वारः} (पा॰सू॰~८.३.२३) इति। एवमेव \textcolor{red}{नियमो विधौ संशोधन\-रूपः}। यथा \textcolor{red}{कृत्तद्धित\-समासाश्च} (पा॰सू॰~१.२.४६) \textcolor{red}{धातोस्तन्निमित्तस्यैव} (पा॰सू॰~६.१.८०) इत्यादि। अयं प्रायशो लक्ष्यं नियमयति। यथा \textcolor{red}{एचोऽयवायावः} (पा॰सू॰~६.१.७८) इत्यनेन सर्वत्रैचोऽच्ययादौ प्राप्ते नियमो जातो धातु\-घटकैचो यद्यवावादेशस्तर्हि धातु\-निमित्तस्यैव। तेन लव्यमित्यत्र \textcolor{red}{लूञ् छेदने} (धा॰पा॰~१४८३) इति धातोः \textcolor{red}{अचो यत्} (पा॰सू॰~३.१.९७) इत्यनेन यति \textcolor{red}{सार्वधातुकार्धधातुकयोः} (पा॰सू॰~७.३.८४) इत्यनेन गुणे धातु\-निमित्तकस्यौकारस्यैवावादेश एवमन्यत्रापि। \textcolor{red}{अतिदेशत्वं नाम वति\-घटितत्वम्}। अर्थात् \textcolor{red}{वति\-घटितत्वे सति समारोपित\-धर्म\-प्रतिपादकतावच्छेदकतावत्त्वम्}। यथा \textcolor{red}{स्थानिवदादेशोऽनल्विधौ} (पा॰सू॰~१.१.५६) \textcolor{red}{अचः परस्मिन् पूर्वविधौ} (पा॰सू॰~१.१.५७) \textcolor{red}{आद्यन्तवदेकस्मिन्} (पा॰सू॰~१.१.२१) इत्यादि।
\textcolor{red}{स्थानिवदादेशोऽनल्विधौ} (पा॰सू॰~१.१.५६) इदं सूत्रमादेशे स्थानि\-धर्ममारोपयति। \textcolor{red}{अधिकारत्वं नाम स्व\-देशे वाक्यार्थ\-जनकता\-शून्यत्वे सति पर\-देशे बोधकत्वमुत्तरोत्तर\-सम्बन्धत्वं वा}। यथा \textcolor{red}{संहितायाम्} (पा॰सू॰~६.१.७२, ६.३.११४) \textcolor{red}{धातोः} (पा॰सू॰~३.१.९१) \textcolor{red}{प्रत्ययः} (पा॰सू॰~३.१.१) \textcolor{red}{परश्च} (पा॰सू॰~३.१.२) इत्यादि। सङ्ग्रहश्चामीषां सूत्र\-प्रकाराणामित्थम्~–\end{sloppypar}
\centering\textcolor{red}{सञ्ज्ञा च परिभाषा च विधिर्नियम एव च। \nopagebreak\\
अतिदेशोऽधिकारश्च षड्विधं सूत्रलक्षणम्॥}\footnote{मूलं मृग्यम्। श्लोकमिमुद्धृत्य \textcolor{red}{मुग्धबोधटीकायां दुर्गादासः} इति वाचस्पत्य\-काराः।}\\
\begin{sloppypar}\hyphenrules{nohyphenation}\justifying\noindent\hspace{10mm} इमामेवाष्टाध्यायीमाश्रित्य पाणिनीयं व्याकरणं प्रावर्तत। लोकेऽस्मिन्नपराणि गौरव\-प्रधानानीदं च लाघव\-प्रधानम्। अतः परिभाषामिमां भाषन्ते भाष्यकाराः~– \textcolor{red}{अर्धमात्रालाघवेन पुत्रोत्सवं मन्यन्ते वैयाकरणाः}।\footnote{\textcolor{red}{‘एओङ्-ऐऔच्’सूत्रयोर्ध्वनितैषा भाष्ये} (प॰शे॰~१३३)।} अतो यावत्सम्भवमासीत्तावल्लघुता वर्तिता पाणिनिना। अल्पान्येव सन्ति सूत्राणि। कानिचिद्विशालानि यथा \textcolor{red}{न पदान्त\-द्विर्वचन\-वरेय\-लोप\-स्वर\-सवर्णानुस्वार\-दीर्घ\-जश्चर्विधिषु} (पा॰सू॰~१.१.५८)। बहूनि च सूत्राणि पञ्चाक्षराणि चतुरक्षराणि त्र्यक्षराणि द्व्यक्षराण्येकाक्षराणि च सन्ति। यथा \textcolor{red}{इको यणचि} (पा॰सू॰~६.१.७७) \textcolor{red}{अनचि च} (पा॰सू॰~८.४.४७) \textcolor{red}{नाज्झलौ} (पा॰सू॰~१.१.१०) \textcolor{red}{चोः कुः} (पा॰सू॰~८.२.३०) \textcolor{red}{अचः} (पा॰सू॰~६.४.१३८) \textcolor{red}{हलः} (पा॰सू॰~६.४.२) \textcolor{red}{चौ} (पा॰सू॰~६.१.२२२, ६.३.१३८) \textcolor{red}{टेः} (पा॰सू॰~६.४.१४३, ६.४.१४५) इत्यादीनि। इत्थमेव महता प्रयासेन लघु\-तममिदं व्याकरणं निर्माय लोकमिममुप\-चकार शालातुरीयः। भाष्यकाराणां प्रमाणेन महापुरुषस्यास्य जन्म \textcolor{red}{शलातुर}\-नामके स्थाने सूच्यते।\footnote{मूलं भाष्य\-संस्करणेषु मृग्यम्।} सूत्रेऽपि \textcolor{red}{शलातुर}\-शब्दस्य चर्चा कृता (\textcolor{red}{तूदी\-शलातुर\-वर्मती\-कूच\-वाराड्ढक्छण्ढञ्यकः} पा॰सू॰~४.३.९४)।\end{sloppypar}
\begin{sloppypar}\hyphenrules{nohyphenation}\justifying\noindent\hspace{10mm} एतस्य पितुर्नाम \textcolor{red}{पणिनः}। अत एव \textcolor{red}{पणिनस्यापत्यं पुमान् पाणिनिः} इति कथ्यते। एतस्य मातुर्नामासीत् \textcolor{red}{दाक्षी} इति भाष्य\-वचनादवगम्यते। भाष्य\-कारो भगवन्तं पाणिनिं \textcolor{red}{दाक्षीपुत्र} इति सम्बोधयति यथा~–\end{sloppypar}
\centering\textcolor{red}{सर्वे सर्वपदादेशा दाक्षीपुत्रस्य पाणिनेः। \nopagebreak\\
एकदेशविकारे हि नित्यत्वं नोपपद्यते॥}\nopagebreak\\
\raggedleft{–~भा॰पा॰सू॰~१.१.२०, ७.१.२७}\\
\begin{sloppypar}\hyphenrules{nohyphenation}\justifying\noindent अयं च दाक्षी\-पुत्रो भगवान् पाणिनिः पाटलिपुत्रेऽधीतवानिति कथा\-सरित्सागरे लिखितं\footnote{क॰स॰सा॰~१.२.४५–४६, १.४.२०–२५।} किन्तु दृढतमं प्रमाणं किमपि न लभ्यते। वररुचिरेतस्य मित्रं सहाध्यायी च तत्परिस्पर्धी। अनेनाष्टाध्यायी लिखिता वररुचिना कात्यायनापर\-नामधेयेन सूत्रेषु त्रुटीरन्विष्य वार्त्तिकानि लिखितानि। यथा सूत्रं \textcolor{red}{न पदान्ताट्टोरनाम्} (पा॰सू॰~८.४.४२) तदुपरि वार्त्तिकम् \textcolor{red}{अनाम्नवति\-नगरीणामिति वाच्यम्}। सूत्रम् \textcolor{red}{आलजाटचौ बहुभाषिणि} (पा॰सू॰~५.२.१२५) वार्त्तिकं \textcolor{red}{कुत्सित\-ग्रहणं कर्तव्यम्}। एवमन्यत्रापि। यद्यपि पर\-वर्ती भाष्यकारः सूत्रमेव प्रमाणं मत्वा वार्त्तिकानि प्रायो निरर्थकानि स्वीचक्रे। इत्थमपि श्रूयते यद्द्वयोरप्याचार्ययोरीर्ष्या पराकाष्ठामाञ्चत्। बहुत्र सूत्राणि निराधारमाक्षेप्य वार्त्तिक\-रचनेन पाणिनिं प्रति कात्यायनो द्वेषं प्रमाणयति। अन्ते चोभौ मृत्यवे परस्परं शप्तवन्तौ। प्रातः कात्यायनो दिवं गतः पश्चात्पाणिनिः स्वर्गमाससाद त्रयोदश्याम्। अतोऽस्मत्सम्प्रदाये त्रयोदश्यां व्याकरणं न पाठ्यते। अतः पठन्ति गुरवः~–\end{sloppypar}
\centering\textcolor{red}{काणादं पाणिनीयं च त्रयोदश्यां न पाठयेत्।}\nopagebreak\\
\raggedleft{–~इत्यस्मद्गुरवः}\\
\begin{sloppypar}\hyphenrules{nohyphenation}\justifying\noindent कुत्रचिदित्थं मिलति यद्वने वसन्तं तमारण्यको व्याघ्रोऽकालयत्। विशेषेणासमन्ताज्जिघ्रतीति व्युत्पत्त्यनुसारं व्याघ्रस्य हिंसा\-कर्म न प्रामाणिकं किन्त्वन्ते तदेव जातम्। अतः पठ्यते~– \textcolor{red}{व्याघ्रो व्याकरणस्य शीघ्रमहरत्प्राणान् प्रियान् पाणिनेः}।\footnote{मूलं मृग्यम्। \textcolor{red}{सिंहो व्याकरणस्य कर्तुरहरत्प्राणान् प्रियान् पाणिनेः} (प॰त॰~२.३६) इति पञ्चतन्त्रकाराः।} इदमपि स्पष्ट\-प्रमाणाभावे निःसारमेव प्रतिभाति मे। अस्तु महा\-पुरुषाणां जन्म यत्र कुत्रापि स्यात्किन्तु तेषां कलित\-धर्म\-वर्म\-संवर्धित\-नम्र\-मनीषि\-नर्म\-समुद्घाटित\-वेदान्त\-निगूढ\-मर्म\-सकल\-लोकालङ्कार\-परमोदार\-लोकोत्तर\-कर्म शर्मणे कल्पते निखिल\-भुवनानाम्। इदमेव व्याकरणं समभवद्विबुध\-भारती\-वल्लभालङ्करणं तरुणतर\-मनीषा\-समुल्लसित\-विकसित\-कञ्ज\-माला\-लसल्ललाम\-भुवनाभिराम\-नव\-नवोन्मेष\-शालि\-प्रतिभा\-सम्भासुर\-सकल\-सद्गुण\-प्रचुर\-निखिल\-कला\-पुर\-विद्या\-नव\-वनिता\-नूपुर\-मुखर\-मधुर\-भाव\-विलसदुरःस्थल\-ललित\-कम्र\-कल्पना\-कलेवर\-वरेण्य\-वन्दित\-पाणि\-विलास\-समुल्लसच्चित्तानां विदुषां विहरणम्।\end{sloppypar}
\begin{sloppypar}\hyphenrules{nohyphenation}\justifying\noindent\hspace{10mm} पाणिनेरनन्तरं कात्यायन\-नामधेय आचार्यः पाणिनीय\-व्याकरण\-श्रियं समर्चयामास स्वकीय\-बुद्धि\-कौशल\-कुसुमावलीभिः। अनेन द्वि\-सहस्र\-प्राय\-वार्त्तिकानि समावर्तितानि। बहुशः प्रयोगाश्च वार्त्तिक\-द्वारा साधिताः। यथा \textcolor{red}{अभिवादि\-दृशोरात्मनेपदे वेति वाच्यम्} (वा॰~१.४.५३) \textcolor{red}{संपुंकानां सो वक्तव्यः} (वा॰~८.३.५) इत्यादि। पश्चात्पाणिनीय\-व्याकरणस्यान्तिम आचार्याः शेषावताराः पाणिनीय\-व्याकरणालङ्काराः पाणिनि\-चरण\-कमल\-बद्धाञ्जलयः शब्द\-सागर\-सम्पूर्ण\-सलिल\-सलील\-मण्डिताञ्जलयः पतञ्जलयः प्रादुर्बभूवुर्येषां जीवन\-कृतं योग\-दर्शनाचार्य\-प्रसङ्गे सङ्क्षेपतो वर्णितम्। यः पाणिनीय\-व्याकरण\-परिश्चिकीर्षया पितृ\-चरणस्य सन्ध्या\-तर्पणार्थं बद्धाञ्जलेरञ्जलौ पतन् पतञ्जलिरिति विश्रुतः। \textcolor{red}{पतन्तो नमस्कार्यत्वेन जनानामञ्जलयो यस्मिन् विषये स पतञ्जलिः}\footnote{\textcolor{red}{“पतञ्जलिरिति” पतन्नञ्जलिर्यस्मिन् नमस्कार्यत्वादिति विग्रहः} (बा॰म॰~७९)। \textcolor{red}{पतन्तोऽञ्जलयोऽस्मिन् नमस्कार्यत्वादिति पतञ्जलिः} (त॰बो॰~७९)। \textcolor{red}{नमस्कार्यत्वेन नमस्कार्यत्वात्} इत्यनयोः \textcolor{red}{विभाषा गुणेऽस्त्रियाम्‌} (पा॰सू॰~२.३.२५) इत्यनेन तृतीया\-पञ्चम्यौ। पूर्वत्रापि \pageref{text:patanjali}तमे पृष्ठे व्युत्पत्तिर्विमृष्टा।} इति वा बहुव्रीहौ शकन्ध्वादि\-गणस्याकृति\-गणत्वात्पर\-रूपम्। इमे शेषावतारा आसन्नित्यास्तिक\-वैयाकरणानां हृदयम्। यथा~–\end{sloppypar}
\centering\textcolor{red}{अशेषफलदातारं भवाब्धितरणे तरिम्।\nopagebreak\\
शेषाशेषार्थलाभार्थं प्रार्थये शेषभूषणम्॥}\nopagebreak\\
\raggedleft{–~वै॰भू॰सा॰ मङ्गलाचरणे २}\\
\begin{sloppypar}\hyphenrules{nohyphenation}\justifying\noindent इति वैयाकरण\-भूषण\-सारे कौण्डभट्टः। भट्टोजिदीक्षितश्च स्वकीय\-वैयाकरण\-सिद्धान्त\-कारिकावल्यामिमं \textcolor{red}{फणि}\-शब्देन सम्बोधयति। यथा~–\end{sloppypar}
\centering\textcolor{red}{फणिभाषितभाष्याब्धेः शब्दकौस्तुभ उद्धृतः। \nopagebreak\\
तत्र निर्णीत एवार्थः सङ्क्षेपेणेह कथ्यते॥}\nopagebreak\\
\raggedleft{–~वै॰सि॰का॰~१}\\
\begin{sloppypar}\hyphenrules{nohyphenation}\justifying\noindent एवमेव हरि\-दीक्षित एनं शब्दरत्ने शेषमेव स्वीकरोति~–\end{sloppypar}
\centering\textcolor{red}{शेषविभूषणमीडे शेषाशेषार्थलाभाय।\nopagebreak\\
दातुं सकलमभीष्टं फलमीष्टे यत्कृपादृष्टिः॥}\nopagebreak\\
\raggedleft{–~श॰र॰ मङ्गलाचरणे १}\\
\begin{sloppypar}\hyphenrules{nohyphenation}\justifying\noindent नागोजिभट्टस्त्वात्मनो नाम \textcolor{red}{नागेश} इति लिखति परिभाषेन्दुशेखरे शब्देन्दुशेखरे च। यथा परिभाषेन्दुशेखरे मङ्गलाचरणं कुर्वन्नागोजिभट्टो लिखति~–\end{sloppypar}
\centering\textcolor{red}{नत्वा साम्बं शिवं ब्रह्म नागेशः कुरुते सुधीः।\nopagebreak\\
बालानां सुखबोधाय परिभाषेन्दुशेखरम्॥}\nopagebreak\\
\raggedleft{–~प॰शे॰~मङ्गलाचरणम्}\\
\begin{sloppypar}\hyphenrules{nohyphenation}\justifying\noindent तत्र \textcolor{red}{नागेश}\-शब्दे बहुव्रीहि\-समासः। \textcolor{red}{नागो नागावतारः पतञ्जलिरीशो यस्य स नागेशः} इति। नागोजिभट्ट आत्मन ईश्वरं पतञ्जलिमेव मन्यते। एवमेव लघु\-शब्देन्दु\-शेखरस्य मङ्गलाचरणे पतञ्जलिं \textcolor{red}{फणीश}\-शब्देन स्तौति। यथा~–\end{sloppypar}
\centering\textcolor{red}{नत्वा फणीशं नागेशस्तनुतेऽर्थप्रकाशकम्। \nopagebreak\\
मनोरमोमार्द्धदेहं लघुशब्देन्दुशेखरम्॥}\nopagebreak\\
\raggedleft{–~ल॰शे॰ म॰~३}\\
\begin{sloppypar}\hyphenrules{nohyphenation}\justifying\noindent तस्य वाल्मीकि\-रामयणे टीकाऽपि नागेश्वरी नाम्ना प्रसिद्धिं गता। लघुत्रयी\-बृहत्त्रय्योः प्रबुद्ध\-टीकाकारो मल्लिनाथोऽपि पातञ्जल\-महाभाष्यं \textcolor{red}{पन्नगगवी}\-शब्देन तुष्टाव।\footnote{कि॰ घ॰व्या॰म॰~४, र॰वं॰ स॰व्या॰म॰~४, शि॰व॰ स॰व्या॰म॰~४।} एवमेव कवि\-तार्किक\-चूडामणिः कल्पना\-कानन\-कण्ठीरवः शब्द\-रचना\-विन्यास\-प्रगल्भः सर्व\-तन्त्र\-स्वतन्त्रः कविता\-कामिनी\-हर्षः श्रीहर्षो भगवन्तं भाष्यकारं शेषमेव मन्यते। कथयति \textcolor{red}{फणि\-भाषित\-भाष्य\-फक्किका\-विषमा कुण्डल\-नामवापिता} (नै॰च॰~२.९५) इति। यः खलु भगवान् श्रीहर्ष एतावदात्मनो वैदुष्यं प्रकटयन् सगर्वं कथयति यत्~–\end{sloppypar}
\centering\textcolor{red}{ग्रन्थग्रन्थिरिह क्वचित्क्वचिदपि न्यासि प्रयत्नान्मया \nopagebreak\\
प्राज्ञंमन्यमना हठेन पठिती माऽस्मिन् खलः खेलतु।\\
श्रद्धाराद्धगुरुश्लथीकृतदृढग्रन्थिः समासादय-\nopagebreak\\
त्वेतत्काव्यरसोर्मिमज्जनसुखव्यासज्जनं सज्जनः॥}\nopagebreak\\
\raggedleft{–~नै॰च॰~२२.१५४}\\
\begin{sloppypar}\hyphenrules{nohyphenation}\justifying\noindent एतादृग्वैदुष्य\-सम्पन्नोऽपि श्रीहर्षो भाष्यकारं शेषं मत्वा प्रपद्यत एतदधिकं किं ब्रूमहे। अधुनाऽपि वाराणस्यां प्रसिद्धो नागकूपो यस्य परिसरे भगवान् भाष्यकारः स्थित्वा सम्पूर्ण\-महा\-भाष्यं बभाषे। साम्प्रतं भाषायां स \textcolor{red}{नागकुआँ} इति कथ्यते। प्रतिवर्षं सर्वेऽपि वाराणसेया विद्वांसो मादृशा विद्यार्थिनश्च श्रावण\-शुक्ल\-पञ्चम्यामाशीर्वाद\-लिप्सया शास्त्रार्थ\-चिकीर्षया च सोत्साहं गुरुभिः सह गच्छन्ति। तत्र च सानन्दं शास्त्रार्थ\-माध्यमेन शास्त्रं गुरुभ्यः शिक्षन्ते शिक्षयन्ति च शिष्यान्। श्रूयते गुरुभ्यो यदस्य नाग\-कूपस्य परिसरे भगवान् पतञ्जलिः प्रायशो भाष्य\-पारायणं प्रोवाच। स च यवनिका\-पट\-खण्डेन मुखमाच्छाद्य पाठयति स्म। महाभाष्यकाराः प्रतिदिवसं यावद्विषयं स्पष्टयन्ति स्म तदेव \textcolor{red}{आह्निकम्} इति कथ्यते। \textcolor{red}{अह्ना निर्वृत्तं जातमाह्निकम्}।\footnote{\textcolor{red}{तेन निर्वृत्तम्} (पा॰सू॰~५.१.७९) इत्यनेन \textcolor{red}{अहन्‌}\-प्रातिपदिकात् \textcolor{red}{ठञ्‌}\-प्रत्ययः। \textcolor{red}{ठस्येकः} (पा॰सू॰~७.३.५०) इत्यनेनेकादेशे \textcolor{red}{अल्लोपोऽनः} (पा॰सू॰~६.४.१३४) इत्यनेनालोपे \textcolor{red}{तद्धितेष्वचामादेः} (पा॰सू॰~७.२.११७) इत्यनेनादि\-वृद्धौ विभक्ति\-कार्ये सिद्धम्।} एवं पञ्चाशीति दिवसान् यावत्पाणिनि\-सूत्राणामुपरि भाष्य\-प्रवचनं चक्रुः। तानि पञ्चाशीत्याह्निकानि जातानि। तानि च भूर्जपत्रेषु कदल्यादिपत्रेषु च लिख्यन्ते स्म।
कुत्रत्यानि\-चित्स्थलानि पत्रेषु लिखितानि विद्यार्थिभिस्तान्यजाभिर्भक्षितानि। अद्याप्यजा\-भक्षित\-स्थलानि चर्चन्तेऽस्मद्गुरु\-परम्परायाम्। गुरवः श्रावितवन्तो यत्कदाचिदेको विद्यार्थी महाभाष्य\-काराणां भाष्य\-प्रवचन\-वेलायां किञ्चित्कार्य\-गौरवात्तानपृष्ट्वा पाठं त्यक्त्वा बहिर्गतः। आगते च तस्मिन् क्रोध\-वशेन भगवता वास्तवं रूपं सहस्र\-शिरो\-युक्तं सहस्र\-फणा\-समुल्लसित\-दिव्य\-कोटि\-कोटि\-सहस्र\-मरीचि\-मालि\-निन्दक\-सहस्र\-मणि\-फण\-प्रकाश\-निरस्त\-निखिल\-भुवन\-तिमिर\-पटलं फूत्कार\-समुच्छलित\-सहस्र\-वदन\-ज्वाला\-जाल\-मात्र\-तिरस्कृत\-प्रलय\-कालानलमनलं भयानकं नारायणांश\-शिरो\-लसित\-रजः\-कणायित\-शैल\-सरित्समुद्र\-कानन\-वृक्ष\-जड\-जङ्गम\-सचराचर\-धर\-धरणि\-गौरवं शेषाख्यं प्रकटयाम्बभूवे। विद्यार्थी च सर्वैश्छात्त्रैः सहानुनिनाय। अनन्तरमुपसंहृत्य रूपमलौकिकं विससर्ज तान् भगवान् विरराम च पाठात्। भगवतः पतञ्जलेर्भाषा सुवासा योषेव कलित\-भाषा विचक्षण\-पोषा विहित\-विपश्चित्सत्तोषा निरस्त\-निखिल\-दोषा दोषाकर\-कौमुदीव मोदयति सज्जन\-चकोर\-निकुरम्बकम्। सरल\-भाषायां गम्भीरतम\-विषयाणां तात्पर्यं यादृक्चतुरतया स्पष्टीकृतं भगवता नूनं तदभूतपूर्वम्। विविध\-दृष्टान्त\-दार्ष्टान्तिक\-व्याज\-भङ्गिम्ना प्रतिपादनं काञ्चने मणिरिव सम्यक्स्थिरतां याति। भाषा\-भाव\-शैली\-प्राञ्जल्यं दृष्ट्वा लगति यद्भगवाननायासं शब्द\-समुद्रमवगाहमानोऽमन्द\-मन्दरतामुपेतः। प्रश्न\-भाष्यमाक्षेप\-भाष्यं सिद्धान्त\-भाष्यं विषये स्वर्ण\-सौरभ\-योगमापादयन्ति। कुत्रचिदैश्वर्य\-मुद्रायां निभ्रान्तं पाणिनिं प्रशंसतो भगवतो वाक्यं विदुषां मनो हरति। यथा \textcolor{red}{सामर्थ्य\-योगान्नहि किञ्चिदत्र पश्यामि शास्त्रे यदनर्थकं स्यात्} (भा॰पा॰सू॰~६.१.७७)। एवमेव \textcolor{red}{वृद्धिरादैच्} (पा॰सू॰~१.१.१) सूत्रे भगवान् साटोपं घोषयति यत्~– \textcolor{red}{प्रमाणभूत आचार्यो दर्भपवित्रपाणिः शुचाववकाशे प्राङ्मुख उपविश्य महता यत्नेन सूत्रं प्रणयति स्म। तत्राशक्यं वर्णेनाप्यनर्थकेन भवितुम्। किं पुनरियता सूत्रेण} (भा॰पा॰सू॰~१.१.१) इति। स्वकीयया प्रतिभया बहुत्र सूत्राणि प्रत्याख्यातानि भगवता। तत्र भगवत इत्थं तात्पर्यं फलं द्विविधं दृष्टमदृष्टञ्च। सूत्राणां लक्ष्य\-सिद्धौ सहायकत्वं दृष्टं फलम्। पुण्य\-जनकता चादृष्टं फलम्। अप्रत्याख्यात\-सूत्राणां द्विविधम्। प्रत्याख्यात\-सूत्राणामदृष्टमात्रं फलम्। अतो यानि प्रत्याख्यातानि तेषां दृष्टं फलं नहि। तदभावेऽपि लक्ष्याणां सिद्धेः। यथा \textcolor{red}{नाज्झलौ} (पा॰सू॰~१.१.१०) इति सूत्रमचां हलाञ्च मिथः सावर्ण्यं निषेधयति। \textcolor{red}{विश्वपाभिः} इत्यत्राऽकारेण हकारस्य सावर्ण्यात्प्राप्त\-ढत्व\-निषेधाय\footnote{ \textcolor{red}{हो ढः} (पा॰सू॰~८.२.३१) इत्यनेन हकारस्य झलि ढत्वं प्राप्तम्।}
\textcolor{red}{हे यियासो} इत्यत्र प्लुताकारेण च हकार\-सावर्ण्य\-वारणाय। दीर्घाकार\-प्लुताकारयोः प्रश्लेषं विधाय दीर्घाकार\-प्लुताकार\-सहितानामचां हलां च न मिथः सावर्ण्यमिति व्यवस्थायामेव \textcolor{red}{दधि हरति दधि शीतलम् दधि षष्ठम् दधि सान्द्रम्} इत्यादौ सावर्ण्याद्यण्दीर्घो न। अन्यथोष्म\-सञ्ज्ञक\-हकार\-शकार\-षकार\-सकाराणां स्वर\-सञ्ज्ञकेकारेण समान\-विवृत\-प्रयत्नत्वात्स्थान\-साम्याच्च सावर्ण्येन दोषो दुर्वार एव। तथा च दीक्षितः \textcolor{red}{तेन दधीत्यस्य हरति शीतलं षष्ठं सान्द्रमित्येतेषु परेषु यणादिकं न। अन्यथा दीर्घादीनामिव हकारादीनामपि ग्रहणक\-शास्त्र\-बलादच्त्वं स्यात्} (वै॰सि॰कौ॰~१३)। दीक्षितादयः प्राचीनाश्चतुर एवाभ्यन्तर\-प्रयत्नान् प्रतियन्ति। अत ऊष्मणां स्वराणाञ्च विवृतमेव प्रयत्नं मानयन्ति। यथा \textcolor{red}{आद्यश्चतुर्धा। स्पृष्टेषत्स्पृष्ट\-विवृत\-संवृत\-भेदात्। तत्र स्पृष्टं प्रयत्नं स्पर्शानाम्। ईषत्स्पृष्टमन्तःस्थानाम्। विवृतमूष्मणां स्वराणां च। ह्रस्वस्यावर्णस्य प्रयोगे संवृतम्। प्रक्रिया\-दशायां तु विवृतमेव} (वै॰सि॰कौ॰~१०) इति। एवं यदा चत्वारः प्रयत्नाः स्वीक्रियन्ते तदा \textcolor{red}{नाज्झलौ} (पा॰सू॰~१.१.१०) इत्यस्याऽवश्यकता। यदा चेषद्विवृतमूष्मणां विवृतञ्च स्वराणामिति प्रयत्न\-भेदः क्रियते तदा हकारादीनामिकारादिभिः प्रयत्न\-वैषम्यात्सावर्ण्याभावेन दीर्घादीनामप्रसक्त्या तन्निषेधार्थं नाज्झलावित्यलब्ध\-लौकिक\-फलतया प्रत्याख्यायि भगवता भाष्यकृता।
अदृष्टं तु फलमस्त्येवेति कृत्वा प्रत्याख्यातं किन्तु निष्कासितं न सूत्रपाठात्। अत एव कस्यचिदपि वर्णस्य व्यर्थता नहि स्वीकृता भगवता। इममेव पन्थानमनुसरद्भिः श्री\-वरद\-राजाचार्यैर्लघुसिद्धान्त\-कौमुद्यां पञ्च\-प्रयत्नता स्वीकृता। यथा \textcolor{red}{आद्यः पञ्चधा स्पृष्टेषत्स्पृष्टेषद्विवृत\-विवृत\-संवृत\-भेदात्। ईषद्विवृतमूष्मणाम्। विवृतं स्वराणाम्} (ल॰सि॰कौ॰~१०) इति। \end{sloppypar}
\begin{sloppypar}\hyphenrules{nohyphenation}\justifying\noindent\hspace{10mm} अस्मिन् प्रक्रिया\-लाघवं कार्य\-सिद्धिश्च बहुत्रान्य\-प्रकारैरपि दर्शयन्तः साधनोपायानपाणिनीय\-भीत्या श्रद्धावनत\-मस्तकाः पुनः प्राञ्चः प्राञ्जलयः प्रार्थयन्ते पतञ्जलयो यच्छक्यमेवमपि वक्तुं किन्तु \textcolor{red}{अपाणिनीयं तु भवति} (भा॰प॰, भा॰पा॰सू॰~१.१.३)। वीक्ष्यतां कीदृशी श्रद्धा भगवति पाणिनौ भगवतां भाष्यकाराणाम्। कात्यायनस्य बहुत्र त्रुटि\-संशोधनात्मक\-वार्त्तिक\-प्रयासो द्वेष\-ग्रस्त\-धियैवेति भाष्यकाराणां मनीषितम्। पस्पशाह्निके यथा लौकिकवैदिकेषु इति वार्त्तिकं भाषयन्तो भाषन्ते भाष्यकृतो यत् \textcolor{red}{प्रिय\-तद्धिता दाक्षिणात्याः। “यथा लोके वेदे च” इति प्रयोक्तव्ये “यथा लौकिक\-वैदिकेषु” इति प्रयुञ्जते} (भा॰प॰) इति। \textcolor{red}{संयोगान्तस्य लोपः} (पा॰सू॰~८.२.२३) इति सूत्रेण \textcolor{red}{सुद्ध्य् उपास्य} इत्यादौ संयोगान्तस्य यणो लोपो मा भूदिति \textcolor{red}{यणः प्रतिषेधो वाच्यः} वचनमिदमुपन्यस्तं वार्त्तिक\-कृता। एतस्य भाष्यकारः प्रत्याख्यानं करोति यत् \textcolor{red}{झलो झलि} (पा॰सू॰~८.२.२६) इति सूत्रात् \textcolor{red}{झलः} इति पञ्चम्यन्तं पदं षष्ठ्या विपरिणम्यात्राप\-कर्षणीयम्।\footnote{\textcolor{red}{न वा वक्तव्यम्। किं कारणम्। झलो लोपात्। झलो लोपः संयोगान्त\-लोपो वक्तव्यः} (भा॰पा॰सू॰~८.२.२३)। \textcolor{red}{न वेति। “झलो झली”त्यतः सिंहाव\-लोकित\-न्यायेन झल्ग्रहणमिहानु\-वर्तते। तत्षष्ठ्या विपरिणम्यत इति यणो लोपाभावः} (भा॰प्र॰ पा॰सू॰~८.२.२३)।} एवं च झलः संयोगान्तस्य लोप इत्यर्थे यणो लोपः सुतरामसम्भवस्तदर्थं वचनारम्भो व्यर्थः। स्वकीय\-दृष्टान्त\-प्रवाह\-प्रसङ्गे मनन\-शीलेनानेन महा\-मुनिना समग्रा भारतीय\-संस्कृति\-वाङ्मयी दर्शिता। धर्मशास्त्र\-राजशास्त्र\-नीतिशास्त्रार्थशास्त्र\-वेदान्त\-लोक\-व्यवहारादि\-मानव\-जीवनोपयोगि\-निगूढ\-विषयाणां मञ्जुलं चित्रणं दृश्यते कृतम्। दृष्टान्तेषु शास्त्र\-प्रतिपादन\-व्याजेन शिक्षाऽप्यतिचतुरतया प्रत्ता। यथा पस्पशाह्निके \textcolor{red}{समानश्च खेद\-विगमो गम्यायां चागम्यायां च। तत्र नियमः क्रियते। इयं गम्येयमगम्येति} (भा॰प॰)। एवं यथा \textcolor{red}{यो ह्यजानन् वै ब्राह्मणं हन्यात्सुरां वा पिबेत्सोऽपि मन्ये पतितः स्यात्} (भा॰प॰) इति।\end{sloppypar}
\begin{sloppypar}\hyphenrules{nohyphenation}\justifying\noindent\hspace{10mm} भगवतां भाष्यकृतां भगवती भागीरथीव भास्वती भासुरा भाषा विषयाणां सुस्पष्ट\-प्रतिपादनं भाव\-व्यक्तीकरण\-सामर्थ्यं निसर्ग\-सिद्ध\-प्रवाहो वाचस्पति\-मति\-रञ्जने गहनतमा विचार\-सरणिः कुशाग्र\-तीव्र\-प्रतिभैकैकस्य प्रश्नस्यानेकान्युत्तराणि स्वस्थो भाव\-बोध\-प्रकारः सिंहवद्विक्रान्ति\-युक्ति\-प्रदर्शनमिदं सर्वमलौकिकमेव।\end{sloppypar}
\centering\textcolor{red}{अन्यानि सन्ति भाष्याणि आचार्यैर्विहितानि वै।\nopagebreak\\
महाभाष्यमिदं प्रोक्तं दिव्यं शेषेण धीमता॥}\nopagebreak\\
\raggedleft{–~इति मम}\\
\begin{sloppypar}\hyphenrules{nohyphenation}\justifying\noindent अन्यान्य\-भाष्याणामपेक्षयाऽस्य वैशिष्ट्यं यदन्यानि सूत्र\-व्याख्यानानि भूतानीदं परिष्कृत\-सूत्र\-व्याख्यानं सदपीष्ट्यादिना महत्त्व\-पूर्णम्। क्वचित्क्वचिद्भगवता स्वतन्त्राऽपीष्टिर्दत्ता। यथा \textcolor{red}{त्यदादीनामः} (पा॰सू॰~७.२.१०२) इदं सूत्रं विभक्तौ परतस्त्यदादीनामकारान्तादेश\-विधायकमिति पाणिनि\-मतं त्यदमारभ्य। किमन्ताः सन्ति त्यदादयः किं पर्यन्तं चेदकारान्तादेशस्तदा \textcolor{red}{युष्मत् अस्मत् भवत्} इत्यादावकारान्तादेशे कृते \textcolor{red}{युवाम् आवाम् भवन्तौ} इत्यादिरूपाणि न सिद्धानि स्युः। अतो भगवतेष्टिर्दत्ता~– \textcolor{red}{द्विपर्यन्तानामेवेष्टिः} इति।\footnote{\textcolor{red}{तस्माद्द्वि\-पर्यन्तानामत्त्वं वक्तव्यम्} (भा॰पा॰सू॰~७.२.१०२)। भैमीकारा अपि~– \textcolor{red}{इसकी अवधि भाष्यकार ने ‘द्वि’ शब्द पर्यन्त नियत की है} (ल॰सि॰कौ॰ भै॰टी॰~१९३)। केषुचित्संस्करणेषु \textcolor{red}{द्विपर्यन्तानामेवेष्टिः} इति वार्त्तिकम्।} बहुत्र कल्पना\-बलेन संसाध्य सूत्र\-प्रयोग आर्ष\-दृष्ट्या तेषां लोकेऽनभिधानं पश्यन् स्वत एव व्यरंसीन्निरर्थक\-बुद्धि\-विलासात्। अत एव महत्त्वञ्चेष्ट्यादिना भाष्य\-प्रदीप\-टीकायां कैयटेन लिखितम्। कैयटः स्वयं कथयति~–\end{sloppypar}
\centering\textcolor{red}{भाष्याब्धिः क्वातिगम्भीरः क्वाहं मन्दमतिस्ततः।\nopagebreak\\
छात्त्राणामुपहास्यत्वं यास्यामि पिशुनात्मनाम्॥}\nopagebreak\\
\raggedleft{–~भा॰प्र॰ मङ्गलाचरणे~६}\\
\begin{sloppypar}\hyphenrules{nohyphenation}\justifying\noindent काशिकायां पूर्व\-वर्तिनां पाणिनि\-कात्यायान\-पतञ्जलीनां कृते मुनि\-सञ्ज्ञा कथिता लोकान्धकार\-निनाशयिषया।\footnote{का॰वृ॰~२.१.१९।} \textcolor{red}{यथोत्तरं मुनीनां प्रामाण्यम्}। तस्मात् \textcolor{red}{त्रिमुनि} व्याकरणं कथ्यते। अन्तिमं प्रामाण्यं महाभाष्यकृतामेव। इत्थं शिवप्रेरणतया पाणिनि\-भाषितस्य कात्यायन\-पतञ्जलि\-परिष्कृतस्यैतस्य लौकिक\-वैदिक\-शब्द\-साधुत्व\-परायणस्य पाणिनीय\-व्याकरणस्य द्वे नेत्रे प्रक्रिया दर्शनञ्च। प्रक्रियायां त्रयाणां मुनीनामनन्तरं मुख्या आचार्याः काशिकाकाराः कैयटोपाध्याया वृत्तिकारा भट्टोजिदीक्षित\-महाभागा वरदराजाचार्या नागोजिभट्ट\-महा\-भागाश्च येषु भट्टोजिदीक्षित\-प्रक्रिया\-प्रकारः सरलः सुष्ठुर्लोके चलितश्च। अनेन प्रक्रिया\-प्रकरणमनुसृत्य सूत्राणि सङ्कलितानि। यथा प्रथमं सन्धिः शब्दानां पश्चात् षड्लिङ्ग\-विभक्ति\-रूपाणि स्त्री\-प्रत्यया एतत्पर्यन्तं शब्द\-विवेचनम्। पुनर्वाक्यार्थं कारक\-निर्देशः समास\-वर्णनं तद्धितीय\-प्रयोग\-दिग्दर्शनं धातु\-प्रक्रिया\-निर्देशः कृदन्त\-निर्देशश्चेति। प्रक्रियायां टीका\-ग्रन्थ एतस्य प्रौढ\-मनोरमा। कुत्रचिन्नागेश एतन्मतं विरुणद्धि। अनुबन्ध\-विषयेऽयमित्सञ्ज्ञकत्वमनुबन्धत्वं मन्यते नागेशश्चेत्सञ्ज्ञा\-योगत्वमनुबन्धत्वं स्वीकरोति।
अन्तिम आचार्यो भगवान्नागेशः। प्रधानतया परिभाषेन्दुशेखर\-लघुशब्देन्दुशेखरौ प्रक्रिया\-प्रकार\-परिष्कारकौ। श्रूयते यन्नागेशभट्टः पुरा कुब्ज आसीत्कदाचिद्बाल\-स्वभावतयोच्चासनमधिश्रितश्चरण\-ताडन\-पुरःसरं तिरस्कृतः। ग्लानि\-खिन्न\-चेतास्त्रिरात्रेण वागीश्वरीमाराध्य वर्ष\-त्रयेण सकल\-शास्त्रं समधिगम्य पण्डित\-चक्र\-चूडामणिर्जातो नागेशः। विवाहे सति शास्त्र\-चिन्तन\-तन्मयतया भोग\-वासनातो विरतः सन्तानार्थं समभ्यर्थितो भार्यया सरल\-भावेनोत्तरयन्नाह नागेशो यत् \textcolor{red}{पुत्र्यश्चैता हि मञ्जूषाः पुत्रौ चैतौ हि शेखरौ}।\footnote{\textcolor{red}{शब्देन्दुशेखरः पुत्रो मञ्जूषा चैव कन्यका। स्वमतौ सम्यगुत्पाद्य शिवयोरर्पितौ मया॥} इत्यपि तेनोक्तं शब्देन्दु\-शेखरे।} गुरु\-चरणा वदन्ति यत्कुड्ये गर्तं कृत्वा तस्मिन्निवेश्य कुब्ज\-भागं लिखन्ति स्म ग्रन्थानाचार्या नागेशा अहो। स्वशरीर\-चर्म\-निर्मित\-पदत्राण\-समर्पणेनाऽपि वयं किममीषां महामहिम\-त्यागशीलानां प्रत्युपकारं कर्तुं क्षमेमहि। कदाचिद्राम\-सिंहेन शृङ्गपुराधीशेन पृष्टो न्यूनतार्थं बहुशो ग्रन्थ\-समागता अनुपपत्तीरेव दर्शयति। ईदृशं शास्त्र\-व्यसनम्। तेभ्यः परं परिष्कार\-प्रक्रिया\-शास्त्रार्थे व्याकरणस्य न्याय\-वासनया वैशिष्ट्यादि\-धारा\-मयी भ्रामक\-जटिल\-विशाल\-शब्द\-जाल\-युक्ता पण्डित\-मनोरमा परम्परा प्रावर्तत यस्यां गण्यन्ते श्री\-शिवकुमार\-शास्त्रि\-दामोदर\-शास्त्रि\-बाल\-शास्त्रि\-तात्य\-शास्त्रि\-जयदेव\-मिश्रास्मद्गुरु\-चरण\-प्रभृतयः।\end{sloppypar}
\centering\textcolor{red}{पाणिन्यादि मुनीन्नत्वा आचार्यान् स्वगुरूंस्तथा।\nopagebreak\\
साञ्जलिर्याचते दिव्यां मतिं गिरिधरः शिशुः॥}\nopagebreak\\
\raggedleft{–~इति मम}\\
\begin{sloppypar}\hyphenrules{nohyphenation}\justifying\noindent\hspace{10mm} अथ व्याकरणस्य दर्शन\-रूप\-नेत्र\-विषये सङ्क्षेपतश्च चर्चयामः। समस्त\-वैयाकरण\-सिद्धान्तानां मूल\-भूतं तु पाणिनि\-व्याकरणमेव। अष्टाध्याय्येको महान् रत्नाकरो यस्मिन्ननेकान्युपयोगीनि रत्नान्यनायासं समुपलब्धुं शक्यन्तेऽहो। विधीयमाने विचारेऽष्टाध्यायी\-रूपो महा\-सागरः क्षीर\-सागरमप्यतिशेते। क्षीर\-सागरे चतुर्दश रत्नान्यत्रापि माहेश्वर\-सूत्र\-रूप\-चतुर्दश\-रत्नानि। तत्र सुन्दराण्यमृतादीन्यभद्राणि विष\-वारुणी\-प्रभृतीनि रत्नानि राजन्ते। अत्र तु सर्वाण्यपि प्रकृति\-रमणीयानि। अत्र महा\-सागरे जल\-स्थानीयं शब्द\-ब्रह्म विचकास्ति। गम्भीरतमानामस्ति दर्शन\-सिद्धान्तानां समुपबृंहणम्। सम्पूर्णान्यपि दर्शनानि रत्नानीव विराजमानानि दिव्यां सुषमामञ्चन्ति। वेदान्तिनां यथाऽद्वैत\-वादस्तथैव वैयाकरणानां शब्द\-विशिष्टाद्वैत\-वादः। वैयाकरणाः शब्दमेव ब्रह्म मन्यन्ते। तत्र शब्देऽद्वैतम्। द्वाभ्यां भेद\-प्रतिपादकाभ्यामितं प्राप्तं ज्ञानं द्वीतं तस्य भावो द्वैतम्। न द्वैतमित्यद्वैतम्। शब्द\-विशिष्टमद्वैतं शब्द\-विशिष्टाद्वैतम्। शब्द\-विशिष्टाद्वैतं ब्रह्म वैयाकरणानां प्रबला मान्यता। यद्यपि नैयायिकाः शब्दमनित्यं मन्यन्ते किन्तु वैयाकरणा नित्यमेव शब्दमुद्घोषयन्ति। किं बहुना शब्दतो निखिल\-प्रपञ्चस्योद्भवं मन्यन्ते यथा~–\end{sloppypar}
\centering\textcolor{red}{अनादिनिधनं ब्रह्म शब्दतत्त्वं यदक्षरम्।\nopagebreak\\
विवर्ततेऽर्थभावेन प्रक्रिया जगतो यतः॥}\nopagebreak\\
\raggedleft{–~वा॰प॰~१.१}\\
\begin{sloppypar}\hyphenrules{nohyphenation}\justifying\noindent आदि\-निधन\-रहितं यदक्षरं शब्द\-ब्रह्म यतो जगतः प्रक्रिया तदेवार्थ\-भावेन विवर्तते। \textcolor{red}{अनादि\-निधनम्} इति कथयित्वाऽपि \textcolor{red}{अक्षरम्} इति कथयन् ब्रह्मणोऽक्षर\-शीलतां प्रतिपादयन् ब्रह्मणो व्यापकतां संस्तौति। \textcolor{red}{न क्षरतीति अक्षरम्}।\footnote{क्षरतीति क्षरम्। \textcolor{red}{क्षरँ सञ्चलने} (धा॰पा॰~८८१)~\arrow क्षर्~\arrow\textcolor{red}{ नन्दि\-ग्रहि\-पचादिभ्यो ल्युणिन्यचः} (पा॰सू॰~३.१.१३४)~\arrow क्षर् अच्~\arrow क्षर् अ~\arrow क्षर~\arrow विभक्तिकार्यम्~\arrow क्षरम्। न क्षरमित्यक्षरम्। \textcolor{red}{नञ्‌} (पा॰सू॰~२.२.६) इत्यनेन तत्पुरुष\-समासे \textcolor{red}{नलोपो नञः} (पा॰सू॰~६.३.७३) इत्यनेन नलोपे \textcolor{red}{अक्षर} इति प्रातिपदिके जाते विभक्तिकार्ये सौ \textcolor{red}{अतोऽम्} (पा॰सू॰~७.१.२४) इत्यनेनामि \textcolor{red}{अमि पूर्वः} (पा॰सू॰~६.१.१०७) इत्यनेन पूर्वरूपे \textcolor{red}{अक्षरम्} इति सिद्धम्।} अथवा \textcolor{red}{अश्नुते सम्पूर्णं जगदिदं व्याप्नोति} इत्यर्थे \textcolor{red}{अशेः सरन्} (प॰उ॰~३.७०) इत्युणादि\-सूत्रेण \textcolor{red}{अश्‌}\-धातोः (\textcolor{red}{अशँ भोजने} धा॰पा॰~१५२३) \textcolor{red}{सरन्‌}\-प्रत्यये षत्वे कत्वे षत्वे चाक्षरमिति।\footnote{\textcolor{red}{अशँ भोजने} (धा॰पा॰~१५२३)~\arrow अश्~\arrow \textcolor{red}{उणादयो बहुलम्} (पा॰सू॰~३.३.१)~\arrow \textcolor{red}{अशेः सरन्} (प॰उ॰~३.७०)~\arrow अश् सरन्~\arrow अश् सर~\arrow \textcolor{red}{व्रश्च\-भ्रस्ज\-सृज\-मृज\-यज\-राज\-भ्राजच्छशां षः} (पा॰सू॰~८.२.३६)~\arrow अष् सर~\arrow \textcolor{red}{षढोः कः सि} (पा॰सू॰~८.२.४१)~\arrow अक् सर~\arrow \textcolor{red}{आदेश\-प्रत्यययोः} (पा॰सू॰~८.३.५९)~\arrow अक् षर~\arrow अक्षर~\arrow विभक्तिकार्यम्~\arrow अक्षरम्।} \textcolor{red}{प्रक्रिया} इति प्रकृष्टा क्रिया। प्रकृष्टत्वञ्च सर्ग\-स्थिति\-विनाश\-रूपम्। \textcolor{red}{यतः} इत्यत्र सार्वविभक्तिकस्तसिः।\footnote{\setcounter{dummy}{\value{footnote}}\addtocounter{dummy}{-1}\refstepcounter{dummy}\label{fn:yatah}\textcolor{red}{इतराभ्योऽपि दृश्यन्ते} (पा॰सू॰~५.३.१४) इत्यनेन। यद्वा \textcolor{red}{तसि\-प्रकरण आद्यादिभ्य उपसङ्ख्यानम्} (वा॰~५.४.४४) इत्यनेन। तसिप्रत्ययान्तानामाकृति\-गणत्वम्। यथा \textcolor{red}{आकृतिगणश्चायम्} (का॰वृ॰~५.४.४४) \textcolor{red}{आकृतिगणोऽयम्} (वै॰सि॰कौ॰~२११२)। न चास्मिन् वार्तिके \textcolor{red}{प्रतियोगे पञ्चम्यास्तसिः} (का॰वृ॰~५.४.४४) इत्यतः \textcolor{red}{पञ्चम्याः} इत्यनुवृत्तम्। \textcolor{red}{अयं सार्वविभक्तिकस्तसिः} (बा॰म॰~२११२) इति बाल\-मनोरमायां वासुदेव\-दीक्षिताः। यथा \textcolor{red}{आदौ आदितः। मध्यतः। पार्श्वतः। पृष्ठतः} (का॰वृ॰~५.४.४४) इत्यादौ सप्तम्यां तसिः। \textcolor{red}{स्वरेण स्वरतः। वर्णतः} (वै॰सि॰कौ॰~२११२) इत्यादौ तृतीयायां तसिः।} अर्थात् \textcolor{red}{यतः} इति पञ्चमी\-तृतीया\-सप्तमी\-विभक्तिषु व्याख्यातुं शक्यते। यतो जगतो जन्म यस्मादिति तात्पर्यं यस्मादुत्पत्तिर्येन पालनं यस्मिल्लँयः। \textcolor{red}{यतो वा इमानि भूतानि जायन्ते येन जातानि जीवन्ति यत्प्रयन्त्यभिसंविशन्ति} (तै॰उ॰~३.१.१) यस्मिल्लीँयन्ते वेति श्रुतेः। तदेव जन्म\-मरण\-रहितमतिशय\-वर्धन\-शीलं सर्व\-व्यापकं शब्द\-ब्रह्म \textcolor{red}{अर्थ\-भावेन} अर्थानां पदार्थानां शक्त्या घट\-पटादि\-रूपेण \textcolor{red}{विवर्तते} विवृतं भवति परिणमतीति तात्पर्यम्। श्लोकेऽस्मिन् गुरु\-चरणा विवर्तत इति पदानुसारं व्याकरण\-सिद्धान्ते विवर्त\-वादं निश्चिन्वन्ति। ब्रह्मणो जगद्रूप\-परिवर्तने दार्शनिक\-जगति धारा\-द्वयी। नैयायिका जगद्ब्रह्म\-परिणाम\-भूतं स्वीकुर्वन्ति। अद्वैत\-वादिनो वेदान्तिनश्च ब्रह्मणो विवर्तं जगन्मन्यन्ते। परिणाम\-वादः प्रकृतेर्विकृति\-रूपेण सतात्त्विक\-परिवर्तनं यथा दुग्धस्य परिणामो दधि। विवर्त\-वादो वस्तुनोऽतात्त्विक\-परिवर्तनं यथा रज्जौ सर्पः। अद्वैत\-वादिनां मते जगदसदिति हेतोस्ते ब्रह्मणोऽन्यथा\-भावरूपं जगत्स्वीकुर्वन्ति। ते कथयन्ति ब्रह्म सत्यं जगन्मिथ्या। द्वैत\-वाद\-प्रतिपादक\-श्रुतीस्तेऽर्थवाद\-रूपेण स्वीकुर्वन्ति। तर्कतः सिद्धान्तितोऽपि सिद्धान्तोऽयं हृदय\-वीणा\-तन्त्रीं न स्पृशति। प्रत्यक्ष\-जगतोऽसत्त्वेन लापनं कर्तुं न शक्यते। वेदे तस्थिवच्छब्देन जगद्व्यवह्रियते। यथा \textcolor{red}{सूर्य॑ आ॒त्मा जग॑तस्त॒स्थुष॑श्च} (शु॰य॰वा॰मा॰~७.४२)।
एवमेव जीव\-ब्रह्मणोरैक्यं प्रतिपादयन्ति तदप्यसमीचीनं लगति। ब्रह्मविद्ब्रह्मैव भवतीति यदि तेषां मनीषा तर्हि कथं घटज्ञो न घटो भवति। यदि चेदात्मा परमात्मैव तर्हि कथं नित्य\-चेतन\-घने सर्व\-तन्त्र\-स्वतन्त्रेऽपरिच्छिन्ने परम\-प्रकाशे विशुद्ध\-ज्योतिषि निर्विकल्पे ब्रह्मण्यसत्य\-माया\-प्रसरः। कथं तस्मिन्नज्ञानं किमनेक\-योजन\-मण्डलं नक्षत्राखण्डलं भुवन\-भास्करं तिमिरता वृणुयात्। आत्मैव परमात्मा चेत्स्वयमेव गुरूणां गुरुस्तर्हि कः शिष्यः कं गुरुं गच्छेत्पठितुम्। कथं वा कोऽपि कामी लोभी क्रोधी किं निष्कलुषं विमल\-बोधमखण्डं ब्रह्म कामादयो मलिनयितुं क्षमन्ते। व्यावहारिकं तदिति चेदलं निरर्थक\-व्यवहार\-प्रपञ्चेन। \textcolor{red}{नित्यो नित्यानां चेतनश्चेतनानामेको बहूनां यो विदधाति कामान्} (श्वे॰उ॰~६.१३) इति श्रुतेः \textcolor{red}{ममैवांशो जीव\-लोके जीव\-भूतः सनातनः} (भ॰गी॰~१५.७) इति स्मृतेश्च का गतिः। तस्माद्विषयान्तरीय\-चर्चां समाप्य प्रकृत इदमेव कथनं पर्याप्तं यज्जीवो ब्रह्मणोंऽश\-भूतः स च सनातनः। अत एव गीतायाम्~–\end{sloppypar}
\centering\textcolor{red}{ममैवांशो जीवलोके जीवभूतः सनातनः।\nopagebreak\\
मनःषष्ठानीन्द्रियाणि प्रकृतिस्थानि कर्षति॥}\nopagebreak\\
\raggedleft{–~भ॰गी॰~१५.७}\\
\begin{sloppypar}\hyphenrules{nohyphenation}\justifying\noindent एकता च तयोः सम्बन्ध\-निबन्धनात्मिका यथा वाल्मीकीये \textcolor{red}{राम\-सुग्रीवयोरैक्यं देव्येवं समजायत} (वा॰रा॰~५.३५.५२)। उभावपि नित्यौ। इदमेव व्याकरणमतमपि प्रतिभाति मे। परिणाम\-वाद\-विवर्त\-वाद\-परिभाषानुसारं विवर्तत इति नास्ति। विवर्तत इत्यस्यार्थो निर्विशेषमपि नाम\-रूपात्मकतया विशिष्टं सद्वर्तते पर\-व्यूह\-विभवान्तर्याम्यर्चा\-रूपेणेति तात्पर्यम्। परिभाषा च परिणाम\-विवर्त\-वादयोरित्थम्~–\end{sloppypar}
\centering\textcolor{red}{सतत्त्वतोऽन्यथाप्रथा विकार इत्युदीरितः। \nopagebreak\\
अतत्त्वतोऽन्यथाप्रथा विवर्त इत्युदीरितः॥}\nopagebreak\\
\raggedleft{–~वे॰सा॰~१३८}\\
\begin{sloppypar}\hyphenrules{nohyphenation}\justifying\noindent तेषामद्वैत\-वादिनां ब्रह्म तु निर्विशेषः। किन्त्वस्मद्ब्रह्म शब्दत्व\-विशिष्टम्। अतस्तेषां ब्रह्म\-तत्त्वमस्माकं शब्द\-तत्त्वम्। तद्ब्रह्मणि निर्धर्मिताऽस्मद्ब्रह्मणि च शब्द\-धर्मिता। तत्रैकमेव नित्यमत्र शब्दार्थ\-सम्बन्धास्त्रयोऽपि नित्याः। अतो वाक्यपदीये \textcolor{red}{नित्याः शब्दार्थसम्बन्धाः} (वा॰प॰~१.२३) इति। जगदपि सम्बन्ध\-कल्पना\-दृष्ट्याऽसत्यं तेषां क्षण\-भङ्गुरत्वादर्वाचीन\-स्वीकृतत्वाच्च। किन्त्वस्माकं जगदीश्वर\-सम्बन्ध\-दृष्ट्या सत्यम्। अतः \textcolor{red}{ममैवांशो जीव\-लोके} (भ॰गी॰~१५.७) इत्यत्रांश\-शब्दो न विभाग\-परः। अन्यथाऽखण्डे ब्रह्मणि सखण्डतापत्तिः। महाकाशे घटाकाशमिवोपाध्यवच्छिन्नतयांऽशः कल्पित इति चेन्न। \textcolor{red}{जीव\-भूतः सनातनः} (भ॰गी॰~१५.७) इति सनातन\-शब्दस्य जीव\-नित्यता\-प्रतिपादकत्वात्कल्पने मानाभावाच्च। \textcolor{red}{नित्यो नित्यानां} (श्वे॰उ॰~६.१३) इति श्रुतेश्च स्वारस्याज्जीवस्य नित्यताऽनेकता च निर्विवादा। तस्मादंश\-शब्दोऽत्र पुत्र\-परः। \textcolor{red}{अमृतस्य पुत्राः} (श्वे॰उ॰~२.५) इति श्रुतेः। एवं शब्दोऽर्थः सम्बन्धश्चास्माकं मते नित्यः।
अतो नित्य\-ब्रह्मणो परिणामोऽपि नित्यमेव तस्माज्जगत्। शब्द\-ब्रह्मणः परिणाम\-भूतं जगदिदं वाक्यपदीय\-कारा अपि समर्थयन्ति~–\end{sloppypar}
\centering\textcolor{red}{शब्दस्य परिणामोऽयमित्याम्नायविदो विदुः। \nopagebreak\\
छन्दोभ्य एव प्रथममेतद्विश्वं प्रवर्तते ॥}\nopagebreak\\
\raggedleft{–~वा॰प॰~१.१२४}\\
\begin{sloppypar}\hyphenrules{nohyphenation}\justifying\noindent शब्दो नित्यः। \textcolor{red}{सिद्धे शब्दार्थ\-सम्बन्धे} (भा॰प॰) इति वार्त्तिकेऽपि सिद्ध\-शब्दो नित्य\-पर्यायः। नित्य\-शब्दञ्च \textcolor{red}{त्यब्नेर्ध्रुव इति वक्तव्यम्} (वा॰~४.२.१०४) इति वार्त्तिकमपि\footnote{सिद्धान्त\-कौमुदी\-संस्करणेषु लघु\-सिद्धान्त\-कौमुदी\-संस्करणेषु वार्त्तिकमिदम्। केषुचिन्महाभाष्य\-संस्करणेषु \textcolor{red}{अमेहक्वतसित्रेभ्यः} इति परिगणनानन्तरं भाष्य\-वचनमिदम्।} \textcolor{red}{ध्रुव} इत्यर्थे
\textcolor{red}{नि}\-अव्ययात् \textcolor{red}{त्यप्}\-प्रत्ययतया ध्रुवार्थं साधयति। शब्दो निरन्तरं ध्रुवो वेदोद्भवत्वात्। शब्द\-नित्यतायां त्रयाणामपि मुनीनां सम्मतिः। नित्यत्वञ्च \textcolor{red}{ध्वंस\-भिन्नत्वे सति ध्वंसाप्रतियोगित्वम्} (त॰स॰ प॰व्या॰~१०)। यथा ध्वंसस्य प्रतियोगिनो घटादयोऽनित्या अप्रतियोगि ब्रह्म नित्यम्। नैयायिकानां मते शब्दो गुणः।\footnote{\textcolor{red}{श्रोत्रग्राह्यो गुणः शब्दः} (त॰स॰~३३)।}
अस्मन्मतेऽगुणोऽद्रव्यं शब्दः।\footnote{\textcolor{red}{अथ गौरित्यत्र कः शब्दः। किं यत्तत्सास्ना\-लाङ्गूल\-ककुद\-खुर\-विषाण्यर्थरूपं स शब्दः। नेत्याह। द्रव्यं नाम तत्। ... यत्तर्हि तच्छुक्लो नीलः कपिलः कपोत इति स शब्दः। नेत्याह। गुणो नाम सः} (भा॰प॰)।} द्रव्यञ्च \textcolor{red}{सत्त्वम्} इति कथ्यते।\footnote{\textcolor{red}{द्रव्यासुव्यवसायेषु सत्त्वम्} (अ॰को॰~३.३.२१३)।} यथा \textcolor{red}{चादयोऽसत्त्वे} (पा॰सू॰~१.४.५७)।\footnote{\textcolor{red}{सत्त्वमिति द्रव्यमुच्यते} (का॰वृ॰~१.४.५७)। \textcolor{red}{अद्रव्यार्थाश्चादयो निपातसञ्ज्ञाः स्युः} (वै॰सि॰कौ॰~२०)। \textcolor{red}{‘सत्त्व’शब्देन द्रव्यमुच्यते} (बा॰म॰~२०)। \textcolor{red}{अद्रव्यार्थाश्चादयो निपाताः स्युः} (ल॰सि॰कौ॰~५३)।} \textcolor{red}{सीदतस्तिष्ठतो लिङ्ग\-सङ्ख्ये यस्मिन् तल्लिङ्ग\-सङ्ख्यान्वयि द्रव्यम्}।\footnote{\textcolor{red}{लिङ्गसङ्ख्याकारकान्वितं द्रव्यम्} (बा॰म॰~२०)। \textcolor{red}{लिङ्गसङ्ख्यान्वितं द्रव्यम्} (त॰बो॰~२०)।} शब्द\-नित्यत्व\-पक्षे धातु\-प्रातिपदिक\-प्रकृति\-प्रत्यय\-विभाग\-तत्तदर्थ\-विभाग\-कल्पनाऽपि सर्वा निर्मूला बाल\-बोधनाय कल्पिता। परमार्थतस्तु वाक्य\-स्फोट एव। तस्मात् \textcolor{red}{वाक्य\-स्फोटोऽति\-निष्कर्षे तिष्ठतीति व्यवस्थितिः} (वै॰सि॰का॰~५९)।\footnote{\textcolor{red}{व्यवस्थितिः} इति \textcolor{red}{मतस्थितिः} इत्यस्य पाठभेदः। वैयाकरण\-सिद्धान्त\-कारिकाः (ख्रिस्ताब्दः १९०१), आनन्दाश्रम\-मुद्रणालयः, पुण्याख्य\-पत्तनम्, ५६तमे पृष्ठे।} अत एव भगवान् भाष्यकार इकः स्थाने यणित्याद्येकदेश\-विकारेषु 
सत्सु शब्द\-नित्यतानुपपत्तिमाशङ्क्य स्थानिनि सर्वपदादेशं प्रतिजानीते। लिखति च~–\end{sloppypar}
\centering\textcolor{red}{सर्वे सर्वपदादेशा दाक्षीपुत्रस्य पाणिनेः।\nopagebreak\\
एकदेशविकारे हि नित्यत्वं नोपपद्यते॥}\nopagebreak\\
\raggedleft{–~भा॰पा॰सू॰~१.१.२०, ७.१.२७}\\
\begin{sloppypar}\hyphenrules{nohyphenation}\justifying\noindent अत इग्घटित\-स्थाने यण्घटितो बोध्यः स च प्रयोक्तव्यः स च साधुरित्येव तात्पर्यं निर्दिशन्ति। शब्दं परम\-प्रमाणतया वयं मन्यामहे निराकारमपि ब्रह्म शब्दाकारतया वयं समर्थयामहे।\end{sloppypar}
\centering\textcolor{red}{अद्वैतास्तु निराकारं नराकारञ्च द्वैतिनः।\nopagebreak\\
वैयाकरणा वयं ब्रह्म शब्दाकारमुपास्महे॥}\nopagebreak\\
\raggedleft{–~इति मम}\\
\begin{sloppypar}\hyphenrules{nohyphenation}\justifying\noindent निराकार\-वादिनां ब्रह्म मनसा दृश्यते साकार\-वादिनां नयन\-चरतामाटीकते शब्दाकार\-वादिनामस्माकं वैयाकरणानां ब्रह्म तु श्रवण\-गोचरतामापन्नं द्वाभ्यां श्रवणाभ्यां शब्दतनु तनोति हृदयम्। लिखितं श्रीराम\-चरित\-मानसे यथा त्रयोविंशति\-सहस्र\-वर्षाणि यावद्भगवद्दर्शनार्थमुग्रं तपस्तप्य\-मानयोर्मनु\-शतरूपयोः श्रीरामभद्रस्य भगवतः प्राकट्यं प्रथमं शब्द\-पुरःसरमेव समभवत्। यथा~–\end{sloppypar}
\centering\textcolor{red}{प्रभु सर्वग्य दास निज जानी। गति अनन्य तापस नृप रानी॥\nopagebreak\\
माँगु माँगु बर भइ नभ बानी। परम गँभीर कृपामृत सानी॥}\footnote{एतद्रूपान्तरम्–\textcolor{red}{अजानान्निजदासौ तौ सर्वज्ञः परमेश्वरः। अनन्यगतिकौ चापि नृपं राज्ञीञ्च तापसौ॥ याचतं याचतं चेति नभोवागभवत्ततः। अत्यन्तमेव गम्भीरा कृपामृतसुमेलिता॥} (मा॰भा॰~१.१४५.५,६)।}
\nopagebreak\\
\raggedleft{–~रा॰च॰मा॰~१.१४५.५,६}\\
\begin{sloppypar}\hyphenrules{nohyphenation}\justifying\noindent स एव ब्रह्मभूतः शब्दः पुनरर्थभावेन सीता\-राम\-रूपेण प्रकटयाम्बभूव। \textcolor{red}{वाग्वै ब्रह्म} (श॰ब्रा॰~२.१.४.१०, बृ॰उ॰~१.३.२१) इति श्रुतिरपि शब्दमेव ब्रह्मतया श्रावयति। अस्यैव शब्द\-ब्रह्मणो भगवतो देवस्य वृषभ\-रूपता श्रुतौ प्रतिपादिता।\footnote{\textcolor{red}{च॒त्वारि॒ शृङ्गा॒ त्रयो॑ अस्य॒ पादा॒ द्वे शी॒र्षे स॒प्त हस्ता॑सो अस्य। त्रिधा॑ ब॒द्धो वृ॑ष॒भो रो॑रवीति म॒हो दे॒वो मर्त्याँ॒ आ वि॑वेश॥} (ऋ॰वे॰सं॰~४.५८.३)। अत्र पतञ्जलयः~– \textcolor{red}{‘चत्वारि शृङ्गाणि’ पदजातानि नामाख्यातोप\-सर्गनिपाताश्च। ‘त्रयो अस्य पादाः’ त्रयः काला भूतभविष्यद्वर्तमानाः। ‘द्वे शीर्षे द्वौ’ शब्दात्मानौ नित्यः कार्यश्च। ‘सप्त हस्तासो अस्य’ सप्त विभक्तयः। ‘त्रिधा बद्धः’ त्रिषु स्थानेषु बद्ध उरसि कण्ठे शिरसीति। ‘वृषभः’ वर्षणात्। ‘रोरवीति’ शब्दं करोति। कुत एतत्। रौतिः शब्दकर्मा। ‘महो देवो मर्त्याँ आविवेश’ इति। महान् देवः शब्दः। मर्त्या मरणधर्माणो मनुष्याः। तानाविवेश। महता देवेन नः साम्यं यथा स्यादित्यध्येयं व्याकरणम्} (भा॰प॰)।} इदमेव शब्दं ब्रह्म ज्ञातुं व्याकरणं प्रवृत्तं पाणिनीयम्। त्रिमुनि\-व्याख्यानेषु बहुत्र
दार्शनिक\-विषय\-चर्चा। तत्र शब्द\-ब्रह्मणः प्रतिपादनम्। तस्यैव शृङ्ग\-भूतानां नामाख्यातोपसर्ग\-निपातानां पाद\-भूतानाञ्च भूत\-भविष्यद्वर्तमान\-कालानां हस्त\-भूतानां सप्त\-विभक्तीनां शिरसोर्व्युत्पन्नाव्युत्पन्न\-प्रातिपदिकयोः\footnote{यद्वा नित्यकार्य\-शब्दात्मनोर्व्यङ्ग्य\-व्यञ्जक\-ध्वन्योः।} बन्धन\-भूतानां कण्ठ\-तालु\-शिरसां\footnote{उरःकण्ठशिरसां वा।} विशद\-वर्णनेनैव समृद्धमिदं शब्द\-शास्त्रम्। स्वीकृते शब्द\-विशिष्टाद्वैत\-राद्धान्ते ब्रह्मणो निराकार\-साकार\-रूपे मन्यमाना वैयाकरणाः साधुत्वमेव शब्दानां ब्रह्म\-प्राप्ति\-करं निर्णयन्ति। तत्र तत्तद्धातूनां कोऽर्थो लकारार्थ\-निर्णये समासे च स्वतन्त्र\-परतन्त्र\-शक्ति\-पर्यालोचनं नञर्थ\-विचारो निपातानां द्योतकत्वं वाचकत्वं वा शक्तेः स्वरूप\-विमर्शः स्फोटस्य व्यवस्थापनमित्यादयः सन्ति परम\-गभीरा दार्शनिका विषयाः।\end{sloppypar}
\begin{sloppypar}\hyphenrules{nohyphenation}\justifying\noindent\hspace{10mm} त्रिमुनि\-व्याख्यानानन्तरं व्याकरण\-दर्शनस्य परम\-प्राचीना आचार्याः श्रीभर्तृहरयः। तेषां मुख्यो ग्रन्थो \textcolor{red}{वाक्यपदीयम्} इति। तत्र \textcolor{red}{वाक्यञ्च पदञ्चेति वाक्यपदे}। लघ्वक्षरत्वाद्यद्यपि \textcolor{red}{पद}\-शब्दस्य पूर्वं प्रयोगः कर्तव्य आसीत्\footnote{\textcolor{red}{लघ्वक्षरम्} (वा॰~२.२.३४)। \textcolor{red}{लघ्वक्षरं पूर्वं निपततीति वक्तव्यम्। कुशकाशम्। शरशीर्यम्} (भा॰पा॰सू॰~२.२.३४)।} तथाऽपि वाक्यस्याभ्यर्हितत्वात्पूर्वं प्रयोगः।\footnote{\textcolor{red}{अभ्यर्हितम्} (वा॰~२.२.३४)। \textcolor{red}{अभ्यर्हितं च पूर्वं निपततीति वक्तव्यम्। मातापितरौ। श्रद्धामेधे ... अपर आह – सर्वत्र एवाभ्यर्हितं पूर्वं निपततीति वक्तव्यम्। लघ्वक्षरादपीति। दीक्षातपसी। श्रद्धातपसी} (भा॰पा॰सू॰~२.२.३४)।} इत्थं \textcolor{red}{वाक्यञ्च पदञ्चेति वाक्य\-पदे ते अधिकृत्य कृतमिति वाक्यपदीयम्} इति विग्रहे द्वन्द्व\-\textcolor{red}{वाक्यपद}\-शब्दात् \textcolor{red}{शिशु\-क्रन्द\-यम\-सभ\-द्वन्द्वेन्द्र\-जननादिभ्यश्छः} (पा॰सू॰~४.३.८८) इत्यनेन \textcolor{red}{छ}\-प्रत्यय ईयादेशे\footnote{\textcolor{red}{आयनेयीनीयियः फढखच्छघां प्रत्ययादीनाम्‌} (पा॰सू॰~७.१.२) इत्यनेन।} विभक्तिकार्ये च सिद्धम्। अत्र शब्द\-ब्रह्मणः प्रतिपादनमस्यैव जगतश्च कारणता सूचिता। अस्य शब्द\-ब्रह्माण्ड\-निखिल\-कलावतीं कारण\-शक्तिमुपाश्रित्य जन्मादयो विकारा भाव\-भेदं भावयन्तीति प्रतिपादितम्। यथा~–\end{sloppypar}
\centering\textcolor{red}{अध्याहितकला यस्य कालशक्तिमुपाश्रिताः।\nopagebreak\\
जन्मादयो विकाराः षड्भावभेदस्य योनयः॥}\nopagebreak\\
\raggedleft{–~वा॰प॰~१.३}\\
\begin{sloppypar}\hyphenrules{nohyphenation}\justifying\noindent\hspace{10mm} सर्व\-बीज\-रूपमिदमेव शब्द\-ब्रह्म भोक्तृ\-भोक्तव्य\-रूपेण भोग\-रूपेण च विपरिणमति। एवमनेके दार्शनिका विषयाः प्रतिपादिताः। वाक्यपदीये काण्ड\-त्रयं ब्रह्म\-काण्डं वाक्य\-काण्डं पद\-काण्डं चेति।\end{sloppypar}
\begin{sloppypar}\hyphenrules{nohyphenation}\justifying\noindent\hspace{10mm} एतदेवोपजीव्य पुनः श्रीकौण्डभट्टो वैयाकरण\-भूषणसारं लिखित्वा विचारैर्व्याकरणं समभूषयत्। तेन धात्वर्थ\-निर्णयेऽति\-चातुरी प्रदर्शिता। व्यापार\-मुख्य\-विशेष्यक\-शाब्दबोधस्तेषामतीव पाण्डित्य\-पूर्णोऽन्वेषण\-विशेषः। यथा कारिका~–\end{sloppypar}
\centering\textcolor{red}{फलव्यापारयोर्धातुराश्रये तु तिङः स्मृताः।\nopagebreak\\
फले प्रधानं व्यापारस्तिङर्थस्तु विशेषणम्॥}\nopagebreak\\
\raggedleft{–~वै॰सि॰का॰~२}\\
\begin{sloppypar}\hyphenrules{nohyphenation}\justifying\noindent धातुः फल\-व्यापार\-वाचक आश्रय\-भूत\-कर्म\-कर्तृ\-वाचकस्तिङ्। फलापेक्षया व्यापारः प्रधानम्। तिङर्थाः कर्तृ\-कर्म\-संख्या\-कारका विशेषणमिति। \textcolor{red}{रामो हरिं भजति} इत्यत्र \textcolor{red}{रामाभिन्नैक\-कर्तृक\-हरि\-कर्मक\-वर्तमान\-कालावच्छिन्नो भजनानुकूल\-व्यापार} इति शाब्दबोधः। नैयायिक\-सम्मत\-प्रथमान्त\-मुख्य\-विशेष्यक\-शाब्द\-बोधस्य \textcolor{red}{पश्य मृगो धावति} इत्यत्र भाष्य\-सम्मत\-शाब्दबोध\-महास्त्रेण खण्डनं मीमांसक\-निर्णीत\-शाब्दबोध\-खण्डनञ्चामीषां पाण्डित्य\-परिचायकम्।\footnote{\textcolor{red}{अपि चाख्यातार्थ\-प्राधान्ये तस्य देवदत्तादिभिः सममभेदान्वयात्प्रथमान्तार्थस्य प्राधान्यापत्तिः। तथा च “पश्य मृगो धावति” इत्यत्र भाष्य\-सिद्धैक\-वाक्यता न स्यात्। प्रथमान्त\-मृगस्य धावन\-क्रिया\-विशेष्यस्य दृशिक्रियायां कर्मत्वापत्तौ द्वितीयापत्तेः। न चैवमप्रथमा\-सामानाधि\-करण्याच्छतृप्रसङ्गः। एवमपि द्वितीयाया दुर्वारत्वेन “पश्य मृगः” इत्यादि\-वाक्यस्यैवाऽसम्भवापत्तेः} (वै॰भू॰सा॰~१.२)। भाष्ये च \textcolor{red}{क्रियाऽपि क्रिययेप्सिततमा भवति। कया क्रियया। सन्दर्शनक्रियया वा प्रार्थयतिक्रियया वाऽध्यवस्यतिक्रियया वा। इह य एष मनुष्यः प्रेक्षापूर्वकारी भवति स बुद्ध्या तावत्कञ्चिदर्थं सम्पश्यति सन्दृष्टे प्रार्थना प्रार्थनायामध्यवसायोऽध्यवसाय आरम्भ आरम्भे निर्वृत्तिर्निर्वृत्तौ फलावाप्तिः। एवं क्रियाऽपि कृत्रिमं कर्म} (भा॰पा॰सू॰~१.४.३२)।} अकर्मक\-सकर्मक\-धातु\-लक्षणे निर्णयोऽपि महत्त्वपूर्णः। तत्र \textcolor{red}{स्वार्थ\-व्यापार\-समानाधिकरण\-फल\-वाचकत्वमकर्मकत्वं स्वार्थ\-व्यापार\-व्यधिकरण\-फल\-वाचकत्वञ्च सकर्मकत्वम्} इति।\end{sloppypar}
\begin{sloppypar}\hyphenrules{nohyphenation}\justifying\noindent\hspace{10mm} समासे विशिष्ट\-शक्ति\-निर्णयो यथा \textcolor{red}{रामस्य} इत्यस्य \textcolor{red}{पुत्रः} इत्यनेन च व्यस्त\-दशायां कोऽपि सम्बन्धो नास्ति। रामस्येति षष्ठ्यन्तस्य पृथगर्थः पुत्र इति प्रथमान्तस्य पृथक्। समासे सति जात एकार्थी\-भावेऽन्योऽर्थो द्वयोः स्वतन्त्रः कोऽप्यर्थो न पङ्कज\-शब्दवत्। \textcolor{red}{एकार्थीभावो नाम पृथगर्थानामेकोपस्थित्योप\-स्थापनम्}। अतः \textcolor{red}{समासे खलु भिन्नैव शक्तिः पङ्कज\-शब्दवत्} (वै॰सि॰का॰~३१)। अयं च समासोऽव्ययीभाव\-तत्पुरुष\-द्विगु\-कर्मधारय\-बहुव्रीहि\-द्वन्द्व\-भेदैः षड्विधः। अयं च सुपां सुपा तिङा प्रातिपदिकेन कदाचिद्धातुना च तिङन्तस्य तिङन्तेन सुबन्तेन च षड्विधो भवति। तथा च सिद्धान्तकौमुद्याम्~–\end{sloppypar}
\centering\textcolor{red}{सुपां सुपा तिङा नाम्ना धातुनाऽथ तिङां तिङा।\nopagebreak\\
सुबन्तेनेति विज्ञेयः समासः षड्विधो बुधैः॥}\nopagebreak\\
\raggedleft{–~वै॰सि॰कौ॰ सर्वसमासशेष\-प्रकरणे}\\
\begin{sloppypar}\hyphenrules{nohyphenation}\justifying\noindent एवं शक्ति\-निर्णये कौण्डभट्टो बोध\-जनकता\-रूपां शक्तिं मन्यते।\footnote{\textcolor{red}{इन्द्रियाणां चक्षुरादीनां स्वविषयेषु चाक्षुषेषु घटादिषु यथाऽनादिर्योग्यता तदीय\-चाक्षुषादि\-कारणता तथा शब्दानामप्यर्थैः सह तद्बोध\-कारणतैव योग्यता सैव शक्तिरित्यर्थः} (वै॰भू॰सा॰~६.३७)।} बोध\-जनकता च नैयायिक\-परम्परातः किमपि प्रभाविता प्रतीयते।\end{sloppypar}
\begin{sloppypar}\hyphenrules{nohyphenation}\justifying\noindent\hspace{10mm} एतस्मादनन्तरं व्याकरण\-दर्शन\-विचारे क्रान्ति\-पूर्ण\-परिवर्तन\-कर्तारः श्रीनागेशभट्टाः। अमी लघुशब्देन्दौ परिभाषेन्दौ च बहुत्र दर्शन\-विचारान् कृत्वाऽपि स्वतन्त्रान् वैयाकरण\-सिद्धान्त\-मञ्जूषा\-वैयाकरण\-लघु\-सिद्धान्त\-मञ्जूषा\-वैयाकरण\-परम\-लघु\-सिद्धान्त\-मञ्जूषा\-नामकांस्त्रीन्दर्शन\-ग्रन्थान् व्यरचयन्। तत्र शब्दार्थयोः कार्य\-कारण\-भावं भावयन् नागेशो लिखति \textcolor{red}{तद्धर्मावच्छिन्न\-विषयक\-शाब्द\-बुद्धित्वावच्छिन्नं प्रति तद्धर्मावच्छिन्न\-निरूपित\-वृत्ति\-विशिष्ट\-ज्ञानं हेतुः} (प॰ल॰म॰~६) इति। पुरातनैः स्वीकृतां खण्डयन् तस्या एकस्थतया सम्बन्धत्वाभावात्सम्बन्धस्य द्विष्ठत्वाद्वाच्य\-वाचक\-भाव\-रूपां शक्तिं मन्यते।\footnote{\textcolor{red}{तस्मात्पद\-पदार्थयोः सम्बन्धान्तरमेव शक्तिर्वाच्य\-वाचक\-भावापर\-पर्याया} (प॰ल॰म॰~१०)।} इयं नवीना गवेषणा। धात्वर्थ\-निर्णयेऽपि प्राचीनाः\footnote{भट्टोजिदीक्षित\-कौण्ड\-भट्टादयः।} फले व्यापारे च पृथक्शक्तिं कल्पयन्ति किन्त्विमे फल\-विशिष्ट\-व्यापार\footnote{व्यापार\-विशिष्ट\-फले च।} एकामेव शक्तिं मन्यन्ते।\footnote{\textcolor{red}{धातोरर्थद्वये शक्तिद्वयकल्पनं ... चातिगौरवम्। तस्मात्फलावच्छिन्ने व्यापारे व्यापारावच्छिन्ने फले च धातूनां शक्तिः कर्तृ\-कर्मार्थक\-तत्तत्प्रत्यय\-समभि\-व्याहारश्च ततद्बोधे नियामक इत्याहुः} (प॰ल॰म॰~४७)।} यद्यपि \textcolor{red}{पदार्थः पदार्थेनैवान्वेति न तु तदेक\-देशेन}\footnote{मूलं मृग्यम्। उद्द्योते तु \textcolor{red}{सविशेषणानां वृत्तिर्न वृत्तस्य वा विशेषणयोगो न। अगुरुपुत्रादीनाम्} (वा॰~२.१.१) इति वार्त्तिके समाधान\-भाष्ये स्थिते \textcolor{red}{महत्कष्टश्रितः} (भा॰पा॰सू॰~२.१.१) इत्युदाहरणे प्रदीपे स्थितं \textcolor{red}{न तु कष्टविशेषणम्} (भा॰प्र॰ पा॰सू॰~२.१.१) इत्युक्तिं व्याचक्षाणा नागेशभट्टपादाः~– \textcolor{red}{इदमेव “पदार्थः पदार्थेने”ति व्युत्पत्तेर्मूलमिति दिक्} (भा॰उ॰ पा॰सू॰~२.१.१)।} इति व्युत्पत्त्या फलस्य स्वतन्त्रं पदार्थत्वं नास्ति तर्हि कथं व्यापारेऽन्वयः। अत्रैकदेशान्वयस्यापि कल्पना क्रियतेऽत उभयोरपि दोषः। तस्मादस्मद्गुरु\-चरणा एक\-वृन्तावलम्बि\-फलद्वयवद्द्वयोरेव शक्तिं स्वीकुर्वन्ति।\footnote{\textcolor{red}{मम त्वेक\-वृन्ताव\-लम्बि\-फल\-द्वय\-वदुभयांश एका खण्डशश्शक्तिरिति न शक्ति\-द्वय\-कल्पनं न वा बोध\-जनकत्व\-सम्बन्ध\-द्वय\-कल्पनम्} (प॰ल॰म॰ ज्यो॰टी॰~४७) इति प्रणेतॄणां गुरुचरणाः कालिका\-प्रसाद\-शुक्ल\-वर्याः परम\-लघु\-मञ्जूषाया ज्योत्स्ना\-टीकायाम्।} शाब्द\-बोध\-विषयेऽपि प्राचीनाः सर्वत्र व्यापार\-मुख्य\-विशेष्यकं शाब्द\-बोधं स्वीकुर्वन्ति परन्तु नागेशास्ततः पृथग्विचारयन्ति। कर्तृ\-वाच्य\-स्थले भाव\-वाच्ये चेमे व्यापार\-मुख्य\-विशेष्यक\-शाब्द\-बोधं मन्यन्ते कर्मवाच्ये च फल\-मुख्य\-विशेष्यकं शाब्दबोधमङ्गीकुर्वन्ति।\footnote{\textcolor{red}{“कर्तृ\-कर्मार्थेति”~– कर्तृ\-भाव\-प्रत्यय\-समभि\-व्याहारः फल\-विशिष्ट\-व्यापार\-बोधे नियामकः। कर्म\-प्रत्यय\-समभि\-व्याहारो व्यापार\-विशिष्ट\-फल\-बोध\-नियामकः} (प॰ल॰म॰ ज्यो॰टी॰~४७)।} एवं बहुत्रामी भाष्यं पर्यालोच्य नवीन\-दृष्ट्यैव विचारयन्तो विलोक्यन्ते। लिखति च स्वयं \textcolor{red}{पातञ्जले महाभाष्ये कृत\-भूरि\-परिश्रमः} (ल॰शे॰ म॰~१) इति। तादात्म्य\-विषयेऽपि व्याकरण\-दर्शने नागेश\-सम्मतो द्वैतवादः। अद्वैत\-वेदान्तिनस्तादात्म्यस्य लक्षणं कुर्वन्तः \textcolor{red}{तदभिन्नत्वे सति तद्भेदेन प्रतीयमानत्वम्} इति भेदं पारमार्थिकं न मन्यन्त इमे चाभेदमेव व्यवहारिकतया स्वीकुर्वन्ति~– \textcolor{red}{तादात्म्यञ्च तद्भिन्नत्वे सति तदभेदेन प्रतीयमानत्वम्} (ल॰म॰, प॰ल॰म॰~१६)। एवमेवान्यैः शक्ति\-लक्षणा\-व्यञ्जनासु स्वीकृतास्विमे शक्ति\-व्यञ्जने एव स्वीकृत्य लक्षणां शक्यतावच्छेदकारोप\-रूपामेव मत्वा लक्षणां शक्तावन्तर्भावयन्ति। यथा \textcolor{red}{गङ्गायां घोषः} इत्यत्र भगीरथ\-रथ\-खातावच्छिन्न\-जल\-प्रवाहे शक्ये घोषस्यासम्भवात्तात्पर्यानुपपत्त्या शक्य\-सम्बन्धतया गङ्गा\-पदस्य गङ्गा\-तीरे लक्षणेत्यन्य\-दार्शनिकानां धीः। किन्तु नागेशो विप्रतिपद्यते यत् \textcolor{red}{गङ्गायां मकर\-घोषौ} इत्यत्र शक्यार्थे गङ्गा\-जले मकरस्यान्वयः सम्भवति तत्तीरे घोषस्य तर्हि गङ्गा\-पदस्य धर्म\-द्वयावच्छिन्नत्वादेक\-धर्मावच्छिन्नत्वाभाव एक\-धर्मावच्छिन्न एकत्रावच्छिन्नस्यैकस्यार्थ\-भावावच्छिन्न\-संसर्गेण साहित्ये सह विवक्षा तदा च द्वन्द्व इति कथं द्वन्द्व\-समासः मकर\-घोष\-पदयोर्विरुद्ध\-धर्मावच्छिन्नेऽन्वयात्। अतः शक्यतावच्छेदकस्य गङ्गा\-रूपार्थस्य गङ्गा\-तीर आरोप एवं द्वयोरप्येक\-धर्मावच्छिन्ने गङ्गा\-प्रवाह एवार्थेऽन्वयः। इयं नवीना क्रान्तिः। इत्थमेव बौद्धार्थः स्वीकृतः नागेशेन। असम्भवार्थ\-शब्दानामपि वन्ध्या\-सुतादीनां साधुत्वम्।
एवमिमे व्याकरण\-दर्शनं चरम\-शिखरे प्रतिष्ठापयामासुरिति।\end{sloppypar}
\centering\textcolor{red}{व्याकरणस्य द्वे नेत्रे प्रक्रिया दर्शनं तथा।\nopagebreak\\
सङ्क्षेपतो निर्दिष्टोऽत्रोभयोः परिचयो मया॥}\\ 
\begin{sloppypar}\hyphenrules{nohyphenation}\justifying\noindent\hspace{10mm} एवं भुवन\-विदित\-महिम्नो शब्द\-लघिम्नो विद्वद्गणित\-गरिम्नो पाणिनीय\-व्याकरणस्य रामकथया सह पूर्ण\-सम्बन्धो वर्तते। यतो हि पाणिनीयं व्याकरणं श्रीराम\-कथा चेति द्वे अपि वेद\-मूलके। पाणिनीय\-व्याकरणस्य वेद\-मूलकत्वमुपपादितम्। इदानीं रामायणस्य वेद\-मूलकत्वं मीमांस्यते। जगदुद्धार\-चिकीर्षया भुवन\-विदित\-लीलो दिव्य\-शीलो जगदात्मा स्वयं परमात्मा भक्त\-रक्षण\-परायणो नारायणः सकल\-मङ्गलायने दशरथायने राजीव\-नयनो नयन\-गोचरतां समागमद्भुवो भार\-जिहीर्षया। वेद\-वेद्यं परं तत्त्वं लोकाभिरामः श्रीरामः। तदा स्व\-प्रतिपाद्यं परं ब्रह्म सगुणं साकारं कोशलेन्द्र\-कुमारं तनु\-विजित\-कोटि\-मारं परमोदारं श्रीरामं दृष्ट्वा प्राचेतसं वाल्मीकिं माध्यमं कृत्वा रामायण\-रूपेण स्वयमेव वेदः प्रादुर्बभूव। अतोऽभियुक्ता वर्णयन्ति~–\end{sloppypar}
\centering\textcolor{red}{वेदवेद्ये परे पुंसि जाते दशरथात्मजे। \nopagebreak\\
वेदः प्राचेतसादासीत्साक्षाद्रामायणात्मना॥\nopagebreak\\
तस्माद्रामायणं देवि वेद एव न संशयः॥}\nopagebreak\\
\raggedleft{–~अग॰सं॰}\\
\begin{sloppypar}\hyphenrules{nohyphenation}\justifying\noindent इति। यदि चेद्वेदस्य रक्षार्थं व्याकरणमध्येयं तदा वेदावतारस्य रामायणस्यापि रक्षार्थं व्याकरणमध्येयम्। किं बहुना वेदस्य षट्स्वङ्गेषु व्याकरणं मुख्यतयोपादेयम्। तथैव वेद\-सम्मितस्य\footnote{\textcolor{red}{इदं पवित्रं पापघ्नं पुण्यं वेदैश्च सम्मितम्। यः पठेद्रामचरितं सर्वपापैः प्रमुच्यते॥} (वा॰रा॰~१.१.९८)।} रामायणस्यापि सम्यग्ज्ञानार्थं पाणिनीयं व्याकरणमुपादेयम्। व्याकरणं विना गूढा भावाः कथं ज्ञातुं शक्यन्ते। यथा वाल्मीकीये हनुमान् श्रीरामं पृच्छति यत्~–\end{sloppypar}
\centering\textcolor{red}{आयताश्च सुवृत्ताश्च बाहवः परिघोपमाः॥\nopagebreak\\
सर्वभूषणभूषार्हाः किमर्थम् न विभूषिताः।}\nopagebreak\\
\raggedleft{–~वा॰रा॰~४.३.१५–१६}\\
\begin{sloppypar}\hyphenrules{nohyphenation}\justifying\noindent अत्र \textcolor{red}{सर्व\-भूषणानि भूषयितुमर्हन्ति}\footnote{\textcolor{red}{“सर्वभूषणभूषार्हाः” आभरणस्याभरणमित्युक्त\-रीत्या भूषणान्यपि भूषयितुमर्हाः किमर्थं न विभूषिताः। इमान् भूषणैरलङ्कृत्याऽभरणाभरणत्वं किमिति न प्रकाशितमित्यर्थः} (वा॰रा॰ भू॰टी॰~४.३.१५–१६) इति गोविन्द\-राजाः।}  इति व्याख्यानं किमवैयाकरणेन सम्भवम्। एवं श्रीरामो हनुमन्तं प्रशंसन् कथयति यत्~–\end{sloppypar}
\centering\textcolor{red}{नूनं व्याकरणं कृत्स्नमनेन बहुधा श्रुतम्।\nopagebreak\\
बहु व्याहरताऽनेन न किञ्चिदपशब्दितम्॥}\nopagebreak\\
\raggedleft{–~वा॰रा॰~४.३.२९}\\
\begin{sloppypar}\hyphenrules{nohyphenation}\justifying\noindent अपरञ्च~–\end{sloppypar}
\centering\textcolor{red}{अनया चित्रया वाचा त्रिस्थानव्यञ्जनस्थया।\nopagebreak\\
कस्य नाराध्यते चित्तमुद्यतासेररेरपि॥}\nopagebreak\\
\raggedleft{–~वा॰रा॰~४.३.३३}\\
\begin{sloppypar}\hyphenrules{nohyphenation}\justifying\noindent अत्र \textcolor{red}{त्रि\-स्थान\-व्यञ्जनस्थया} इति शब्दस्य कथमवैयाकरणो भावमवगन्तुं पारयिष्यति। वैयाकरणस्तूच्चारणस्य कण्ठ\-तालु\-मूर्धान इति त्रि\-स्थानं\footnote{यद्वाऽभि\-व्यक्तेरुरःकण्ठ\-शिरांसीति त्रिस्थानम्।} ज्ञात्वा ततो वाचः समुद्भवं ज्ञास्यति। अगस्त्यो रामायण\-महामाला\-रत्नं श्रीमन्तं हनुमन्तं नव\-व्याकरणस्य वेत्तारं कथयति। \textcolor{red}{सोऽयं नव\-व्याकरणार्थवेत्ता} (वा॰रा॰~७.३६.४७) इत्यादि। वाल्मीकिरपि \textcolor{red}{तदुपगत\-समास\-सन्धि\-योगम्} (वा॰रा॰~१.२.४३) इति कथयति। एवमेव विबुध\-वाण्यां गीतानि कोटिशो रामायणानि व्याकरणं विना कथमपि ज्ञातुं न शक्यन्ते। मम त्वियं मान्यता संस्कृतवत्पुण्य\-जनकतावच्छेदिष्ठायां महादेव\-भाषायां भाषायां सङ्गीतं भक्ति\-शिरोमणि\-सकल\-कवि\-कुल\-शेखर\-हस्तामलकीकृत\-वेद\-शास्त्र\-पुराणेतिहास\-काव्य\-नाटक\-निखिल\-निगमागम\-सुदुर्गम\-विचार\-चतुर्दिक्चारु\-चातुरी\-विलोकित\-तुरीय\-महनीय\-कविता\-वनिता\-जीवन\-सीता\-रमण\-पद\-पद्म\-पराग\-सुराग\-रसमिलिन्द\-माहात्म्य\-श्रीमत्तुलसीदास\-कृत\-श्रीमद्राम\-चरित\-मानसमप्यवैयाकरणेन ज्ञातुं न शक्यते। यथा~–\end{sloppypar}
\centering\textcolor{red}{सरिस श्वान मघवान जुबानू}\footnote{एतद्रूपान्तरम्–\textcolor{red}{मघवा श्वा युवा चैव वर्तन्ते समतान्विताः} (मा॰भा॰~२.३०२.८)।}\nopagebreak\\
\raggedleft{–~रा॰च॰मा॰~२.३०२.८}\\
\begin{sloppypar}\hyphenrules{nohyphenation}\justifying\noindent इत्युपमा। तात्पर्यं येन \textcolor{red}{श्व\-युव\-मघोनामतद्धिते} (पा॰सू॰~६.४.१३३) इति सूत्रं न पठितं स किं बोद्धुं प्रभविष्यति। बहुत्र तुलसीदासेनापि व्याकरणस्य वेदान्तादि\-दर्शनानां स्वाभाविकतया रहस्यमुपन्यासि। तत्र व्याकरणमन्तरेणेतर\-दर्शन\-ज्ञानाभावे मानसस्योत्तर\-काण्डस्योत्तरार्धं कथमपि स्पष्टयितुं न शक्यते। तत्परम्परायां वर्तमानं श्रीमदध्यात्म\-रामायणमपि पाणिनीय\-व्याकरणेन पूर्णतः सम्बद्धम्। पाणिनीय\-व्याकरणं चतुर्दश\-सूत्राचार्यतया शिवोक्तम्। तस्मात्पाणिनिर्बहुश उकारानु\-बन्धानि सूत्राणि पपाठ। यथा \textcolor{red}{डः सि धुट्} (पा॰सू॰~८.३.२९) \textcolor{red}{ङ्णोः कुक् टुक् शरि} (पा॰सू॰~८.३.२८) \textcolor{red}{शि तुक्} (पा॰सू॰~८.३.३१) \textcolor{red}{आने मुक्} (पा॰सू॰~७.२.८२) \textcolor{red}{ह्रस्वनद्यापो नुट्} (पा॰सू॰~७.१.५४) इत्यादि। अन्याननु\-बन्धानुपेक्ष्योकारानुबन्ध\-बाहुल्यादुकार\-वाच्यस्य शम्भोः स्मरणं पाणिनि\-कृतं प्रतीयते।\end{sloppypar}
\begin{sloppypar}\hyphenrules{nohyphenation}\justifying\noindent\hspace{10mm} एवमध्यात्म\-रामायणमपि शिव\-भाषितं यथा~–\end{sloppypar}
\centering\textcolor{blue}{पुरारिगिरिसम्भूता श्रीरामार्णवसङ्गता।\nopagebreak\\
अध्यात्मरामगङ्गेयं पुनाति भुवनत्रयम्॥}\nopagebreak\\
\raggedleft{–~अ॰रा॰~१.१.५}\\
\begin{sloppypar}\hyphenrules{nohyphenation}\justifying\noindent तर्हि शम्भु\-भाषितस्यास्याध्यात्म\-रामायणस्य शैवेन पाणिनि\-व्याकरणेनैक\-वाक्यता\-पुरःसर\-सम्बन्धो भवेदेवेति वच्मि। अहं तूत्प्रेक्षे यत्स्वयमेव शिवो लोक\-वत्सलतया भगवतीं पार्वतीं भवानीं श्रोत्रीं मत्वा सम्भाष्य च भुवन\-पावनीं राम\-कथां तदर्थ\-बुबोधयिषया पाणिनि\-हृदयस्थो व्याकरण\-मूल\-भूतायां माहेश्वर्यां चतुर्दश\-सूत्र्यां समासतो राम\-कथां कथयति। तत्र द्वि\-चत्वारिंशदक्षराणां सङ्ग्रहो यश्च भगवतः श्रीरामस्य द्विचत्वारिंशद्वर्षीय\-चरित्रं ध्वनयति। \label{text:exileage1}जन्मतो विवाहं यावद्द्वादशाब्दावधिस्ततो द्वादश\-वर्षं यावदयोध्यायां वास एवं पञ्चविंशे वर्षे सीता\-लक्ष्मणाभ्यां सह वन\-गमनं चतुर्दशाब्दं यावदरण्य\-चरित्रं मैथिली\-हरण\-रावण\-संहरणादिकमन्तिम\-वर्षे राज्य\-लीलेति मिलित्वोनचत्वारिंशद्वर्षाणि। वर्षत्रयं राज्य\-व्यवस्थायाम्। एषु मुख्यं रामायणम्। अक्षरस्य ब्रह्मणः श्रीरामस्य दिव्या चर्चा सूत्रबद्धाक्षरैः सङ्केतिता। चतुर्दशसूत्र्यां विभाग\-द्वयं स्वर\-वर्गो व्यञ्जनवर्गश्च। स्वराणां चर्चा चतुर्भिः सूत्रैर्व्यञ्जनानाञ्च चर्चा दशभिः सूत्रैर्व्यधायि। अत्रापि राम\-कथायां विभाग\-द्वयं निर्गुण\-लीलायाः सगुण\-लीलायाश्च। ऐश्वर्यस्य माधुर्यस्य वा। ऐश्वर्यं चत्वारि फलानि ददात्यतश्चतुर्भिः सूत्रैस्तत्सङ्केतः सङ्गच्छते। ऐश्वर्यञ्च स्वरस्थानापन्नमतः स्वरैः कथ्यते। \textcolor{red}{स्वेन राजते स्वेन रमते वेति स्वरः}।\footnote{\textcolor{red}{स्व} उपपदे \textcolor{red}{राज्‌}\-धातोः (\textcolor{red}{राजृँ दीप्तौ} धा॰पा॰~८२२) \textcolor{red}{रम्‌}\-धातोर्वा (\textcolor{red}{रमुँ क्रीडायाम्} धा॰पा॰~८५३) \textcolor{red}{अन्येष्वपि दृश्यते} (पा॰सू॰~३.२.१०१) इत्यनेन \textcolor{red}{ड}\-प्रत्ययः। अनुबन्ध\-लोपे \textcolor{red}{डित्यभस्याप्यनु\-बन्धकरण\-सामर्थ्यात्} (वा॰~६.४.१४३) इत्यनेन टिलोपे \textcolor{red}{सुपो धातु\-प्रातिपदिकयोः} (पा॰सू॰~२.४.७१) इत्यनेन सुब्लुकि विभक्तिकार्ये \textcolor{red}{स्वरः}। महाभाष्ये च~– \textcolor{red}{अन्तरेणापि व्यञ्जनमच एवैते गुणा लक्ष्यन्ते। न पुनरन्तरेणाचं व्यञ्जनस्योच्चारणमपि भवति। अन्वर्थं खल्वपि निर्वचनम् – स्वयं राजन्त इति स्वराः। अन्वग्भवति व्यञ्जनमिति} (भा॰पा॰सू॰~१.१.२९–३०)। तथा च सङ्गीत\-रत्नाकरस्थस्य \textcolor{red}{योऽयं स्वयं राजते} (स॰र॰~१.१) इत्यस्य व्याख्यानावसरे सङ्गीत\-सुधाकर\-टीकायां सिंहभूपालः~– \textcolor{red}{“योऽयं स्वयं राजते” अन्यानपेक्षतयाऽऽनन्दयितृत्वात्। स्वरपदस्य निरुक्तिरप्यनेन कथ्यते} (स॰र॰ सु॰टी॰~१.१)। वाचस्पत्याभिधाने \textcolor{red}{अचः स्वयं विराजन्ते हलस्तु परगामिनः} इत्युद्धृतः।} स्वरा ये खलूच्चारणे किमपि वर्णान्तरं नापेक्षन्ते। तथैवेश्वरोऽनपेक्षकोऽपि स्वस्मिन्महीयते। स्वराणां सङ्ख्या नव। राघवस्य च प्रादुर्भावो नवम्यां यथा वाल्मीकीये~–\end{sloppypar}
\centering\textcolor{red}{ततो यज्ञे समाप्ते तु ऋतूनां षट् समत्ययुः।\nopagebreak\\
ततस्तु द्वादशे मासे चैत्रे नावमिके तिथौ॥}\nopagebreak\\
\raggedleft{–~वा॰रा॰~१.१८.८}\\
\begin{sloppypar}\hyphenrules{nohyphenation}\justifying\noindent अध्यात्म\-रामायणे च~–\end{sloppypar}
\centering\textcolor{blue}{मधुमासे सिते पक्षे नवम्यां कर्कटे शुभे।\nopagebreak\\
पुनर्वस्वृक्षसहिते उच्चस्थे ग्रहपञ्चके॥}\nopagebreak\\
\raggedleft{–~अ॰रा॰~१.३.१४}\\
\begin{sloppypar}\hyphenrules{nohyphenation}\justifying\noindent नवमी सङ्ख्या श्रेष्ठा पूर्णा। एतस्या वरीयसी काऽपि सङ्ख्या नास्ति। आद्यन्त\-निर्वाहिका सर्वथा गुणिताऽपि पूर्व\-पर\-सम्मेलनेन नवैव। यथा द्विगुणिता नव सङ्ख्याऽष्टादशतामुपैति। सा च दक्षिण एकं वामेऽष्टाविति लिखित्वा भवति।\footnote{\textcolor{red}{अङ्कानां वामतो गतिः} इति न्यायेन।} सम्मेलनेन द्वयोः पुनर्नव। एवं त्रि\-गुणिता सप्तविंशतिः। सा च दक्षिणे द्वौ वामे सप्तेति लिखित्वा भवति।\footnote{\textcolor{red}{अङ्कानां वामतो गतिः} इति न्यायेन।} सम्मेलनेन द्वयोः पुनर्नव। एवमन्यत्रापि। यथा सर्वथा गुणिताऽपि नवमी सङ्ख्या द्वयोर्योगेन नवत्वं न जहाति तथैव नवभिः स्वरैः सङ्केत्यो भगवान् गुणितोऽप्यर्थाद्भक्तेच्छया जनवत्सलत्व\-कारुणिकत्व\-कृपालुत्व\-प्रणतानुरागित्व\-भक्तेच्छा\-पालकत्व\-दीन\-बन्धुत्व\-सरलत्व\-सफलत्व\-स्वामित्व\-रघुनाथत्व\-प्रभृतिभिर्दशभिर्गुणैर्गुणितोऽपि नवमी सङ्ख्येवैश्वर्यं न त्यजति। नवमीतोऽधिका काऽपि सङ्ख्या नहि। तथैव रामतोऽधिकः कोऽपि देवो नास्ति। नवमी सङ्ख्या पूर्णा श्रीरामोऽपि पूर्णः। ऐश्वर्य\-लीलायां चतुष्पाद्विभूति\-दर्शनम्। विभूतेश्चत्वारः पादा यथा~–\end{sloppypar}
\centering\textcolor{red}{ए॒तावा॑नस्य महि॒माऽतो॒ ज्यायाँ॑श्च॒ पूरु॑षः॒।\nopagebreak\\
पादो॑ऽस्य॒ विश्वा॑ भू॒तानि॑ त्रि॒पाद॑स्या॒मृतं॑ दि॒वि॥\nopagebreak\\
त्रि॒पादू॒र्ध्व उदै॒त्पुरु॑षः॒ पादो॑ऽस्ये॒हाभ॑व॒त्पुनः॑।\nopagebreak\\
ततो॒ विष्व॒ङ्व्य॒क्रामत्साशनानश॒ने अ॒भि॥}\nopagebreak\\
\raggedleft{–~शु॰य॰वा॰मा॰~३१.३-४}\footnote{ऋग्वेद\-संहितायां (ऋ॰वे॰सं॰~१०.९०.३-४) तैत्तिरीयारण्यके (कृ॰य॰ तै॰आ॰~३.१२.२) अप्येतौ मन्त्रौ।}\\
\begin{sloppypar}\hyphenrules{nohyphenation}\justifying\noindent चतुष्पाद्विभूतेर्भगवतश्चतुर्भिः सूत्रैर्विभूति\-भूषणेन ढक्का\-माध्यमेन गानमति\-न्याय\-सङ्गतं युक्ति\-युक्तञ्च। ऐश्वर्य\-लीलायां श्रीरामस्य मिलन्ति चत्वारि सूत्राणि। विकारैः सह विग्रहः फलानां निग्रहः कारुण्यादीनां सङ्ग्रहो निज\-पद\-पद्म\-प्रपन्न\-भक्तेष्वनुग्रहश्च। इदमपि
चतुर्भिः सूत्रैः
स्वररूप\-श्री\-रामस्यैश्वर्य\-लीला\-प्रतिपादने तात्पर्यमूह्यम्। चतुर्णामार्त\-जिज्ञास्वर्थार्थि\-ज्ञानि\-भक्तेष्वनु\-ग्रहश्चास्मिन् तथ्ये दृढीकरणमापादयति। चत्वारो हि भगवतो भक्ता आर्तो जिज्ञासुरर्थार्थी ज्ञानी चेति।\end{sloppypar}
\begin{sloppypar}\hyphenrules{nohyphenation}\justifying\noindent\hspace{10mm} (१) \textcolor{red}{आर्तः} आर्ति\-ग्रस्तश्चतुर्णामपि पुरुषार्थानां लिप्सया परमात्मानं प्रपद्यते यथा विभीषणः।\end{sloppypar}
\begin{sloppypar}\hyphenrules{nohyphenation}\justifying\noindent\hspace{10mm} (२) \textcolor{red}{जिज्ञासुः} ज्ञातुमिच्छुरर्थाद्धर्म\-मोक्ष\-लिप्सया श्रीरामं शुश्रूषते यथा लक्ष्मणः।\end{sloppypar}
\begin{sloppypar}\hyphenrules{nohyphenation}\justifying\noindent\hspace{10mm} (३) \textcolor{red}{अर्थार्थी} भगवन्तं राघवमर्थ\-कामौ प्रार्थयते यथा सुग्रीवः।\end{sloppypar}
\begin{sloppypar}\hyphenrules{nohyphenation}\justifying\noindent\hspace{10mm} (४) \textcolor{red}{ज्ञानी} ज्ञानं प्राप्य पूर्वं तु मोक्षमिच्छति। विशेषो विज्ञानि\-रूपो निष्कामो हृदि रामं रमयितुमेव सदा चेष्टते यथा जटायुर्हनुमांश्च। गीतायां मानसे चापि चर्चा यथा~–\end{sloppypar}
\centering\textcolor{red}{चतुर्विधा भजन्ते मां जनाः सुकृतिनोऽर्जुन।\nopagebreak\\
आर्तो जिज्ञासुरथार्थी ज्ञानी च भरतर्षभ॥}\nopagebreak\\
\raggedleft{–~भ॰गी॰~७.१६}\\
\centering\textcolor{red}{राम भगत जग चारि प्रकारा। सुकृती चारिउ अनघ उदारा॥}\footnote{एतद्रूपान्तरम्–\textcolor{red}{इत्थं चतुर्विधास्सन्ति भक्ता रामस्य निश्चितम्। एते सर्वेऽप्युदाराश्च धन्याः कलुष\-वर्जिताः॥} (मा॰भा॰~१.२२.६)।}\nopagebreak\\
\raggedleft{–~रा॰च॰मा॰~१.२२.६}\\
\begin{sloppypar}\hyphenrules{nohyphenation}\justifying\noindent ऐश्वर्य\-लीलायाममीषु चतुर्षु कृपाऽतोऽपि चतुर्भिः सूत्रैः प्रतिपादनं सङ्गच्छते। ऐश्वर्य\-लीलायां चतस्रः मुख्या घटनाः कौसल्या\-समक्षं विराड्रूप\-प्रदर्शनं यथा~–\end{sloppypar}
\centering\textcolor{red}{देखरावा मातहि निज अद्भुतरूप अखंड।\nopagebreak\\
रोम रोम प्रति लागे कोटि कोटि ब्रह्मांड॥}\footnote{एतद्रूपान्तरम्–\textcolor{red}{स मातरं दर्शयति स्म नैजमखण्डमाश्चर्यमयञ्च रूपम्। यद्रोमरोमश्रिततामुपेता ब्रह्माण्ड\-कोट्यो गुणिता अनेकाः॥} (मा॰भा॰~१.२०१)।}\nopagebreak\\
\raggedleft{–~रा॰च॰मा॰~१.२०१}\\
\begin{sloppypar}\hyphenrules{nohyphenation}\justifying\noindent अहल्योद्धारः परशुराम\-समक्षं वैष्णव\-धनुः\-कर्षणं रावण\-वधश्च। अतोऽपि चतुर्भिः सूत्रैः स्वरच्छलेनैश्वर्यांश\-वर्णनं रमणीयं प्रतिभाति मे। नव स्वरास्तथैवैश्वर्य\-लीलाऽपि श्रवण\-कीर्तन\-स्मरण\-पाद\-सेवन\-पूजन\-वन्दन\-दास्य\-सख्यात्म\-निवेदनेति\-नव\-लक्षणां भक्तिं ददाति। नवधा भक्तिर्भागवतानु\-सारेणेत्थम्~–\end{sloppypar}
\centering\textcolor{red}{श्रवणं कीर्तनं विष्णोः स्मरणं पादसेवनम्।\nopagebreak\\
अर्चनं वन्दनं दास्यं सख्यमात्मनिवेदनम्॥\nopagebreak\\
इति पुंसाऽर्पिता विष्णौ भक्तिश्चेन्नवलक्षणा।\nopagebreak\\
क्रियते भगवत्यद्धा तन्मन्येऽधीतमुत्तमम्॥}\nopagebreak\\
\raggedleft{–~भा॰पु॰~७.५.२३–२४}\\
\begin{sloppypar}\hyphenrules{nohyphenation}\justifying\noindent रामायणेऽमीषां सङ्ग्रहश्च भक्तानाम्~–\end{sloppypar}
\centering\textcolor{red}{श्रीरामश्रवणे मता गिरिसुता काकः शिवः कीर्तने\nopagebreak\\
कौसल्या स्मरणे पदाब्जभजने सीता सुतीक्ष्णोऽर्चने।\nopagebreak\\
सौमित्रिः पदवन्दने च हनुमान्दास्ये च सख्येऽर्कजो\nopagebreak\\
लङ्केशो भरतः समर्पणविधौ रामाप्तिरेषां फलम्॥}\nopagebreak\\
\raggedleft{–~इति मम}\\
\begin{sloppypar}\hyphenrules{nohyphenation}\justifying\noindent एवं भागवतेऽपि~–\end{sloppypar}
\centering\textcolor{red}{श्रीकृष्णश्रवणे परीक्षिदभवद्वैयासकिः कीर्तने\nopagebreak\\
प्रह्लादः स्मरणे तदङ्घ्रिभजने लक्ष्मीः पृथुः पूजने।\nopagebreak\\
अक्रूरस्त्वथ वन्दने च हनुमान्दास्येऽथ सख्येऽर्जुनः\nopagebreak\\
सर्वस्वात्मनिवेदने बलिरभूत्कृष्णाप्तिरेषां फलम्॥}\footnote{मूलं गौडीय\-वैष्णव\-ग्रन्थेषु मृग्यम्।}\\ 
\begin{sloppypar}\hyphenrules{nohyphenation}\justifying\noindent एवं ऐश्वर्य\-लीलाऽपि नव\-विधा। तत्रैश्वर्यं धर्मो यशः श्रीर्ज्ञानं वैराग्यं निग्रहो विग्रहोऽनुग्रहश्चेति। अत एव नव\-स्वरात्मक\-नवैश्वर्य\-लीलानां चतुर्भिः सूत्रैः प्रतिपादनं सुस्पष्टमेव रामायणस्यैश्वर्य\-वर्णनम्। तत्र~–\end{sloppypar}
\begin{sloppypar}\hyphenrules{nohyphenation}\justifying\noindent\hspace{10mm} \textcolor{red}{(१) अइउण्}~– अकारो वासुदेवः। यथा~–\end{sloppypar}
\centering\textcolor{red}{अक्षराणामकारोऽस्मि द्वन्द्वः सामासिकस्य च।\nopagebreak\\
अहमेवाक्षयः कालो धाताऽहं विश्वतोमुखः॥}\nopagebreak\\
\raggedleft{–~भ॰गी॰~१०.३३}\\
\begin{sloppypar}\hyphenrules{nohyphenation}\justifying\noindent इत्थम् \textcolor{red}{अ} श्रीरामो वासुदेवः।\footnote{\textcolor{red}{अकारो वासुदेवः स्यात्} (ए॰को॰~१)।} \textcolor{red}{इ} महालक्ष्मी सीता।\footnote{\textcolor{red}{लक्ष्मीरीकार उच्यते} (ए॰को॰~२)। \textcolor{red}{ईः। स्त्री। अस्य विष्णोः पत्नी। ङीप्। लक्ष्मीः। इति विश्वमेदिन्यौ} इति शब्दकल्पद्रुमः। \textcolor{red}{ईः। स्त्री। अस्य विष्णोः पत्नी। ङीष्। लक्ष्म्याम्} इति वाचस्पत्यम्। ततः \textcolor{red}{ङ्यापोः सञ्ज्ञा\-छन्दसोर्बहुलम्} (पा॰सू॰~६.३.६३) इत्यनेन छान्दस\-ह्रस्वः। \textcolor{red}{छन्दोवत्सूत्राणि भवन्ति} (भा॰पा॰सू॰~१.१.१, १.४.३) इत्यनेन सूत्राणां छान्दसत्वं माहेश्वर\-सूत्रार्थ\-प्रकरणेऽस्मिन् सर्वेषु छान्दस\-कार्येषु बोध्यम्।} \textcolor{red}{उ} जीवाचार्यो लक्ष्मणः।\footnote{उपलक्षणत्वात्।} \textcolor{red}{ण्} निर्वृति\-वाचकः। \textcolor{red}{णश्च निर्वृतिवाचकः} (गो॰पू॰ता॰उ॰~१) इति गोपाल\-तापनीय\-श्रुतेः। अर्थाद्राम\-सीता\-लक्ष्मण\-ध्यानेन जीवो भव\-बन्धनान्निवर्तत\footnote{जीवस्य भव\-बन्धनान्निवृतिर्वा।} इति सूत्रार्थः।\footnote{\textcolor{red}{अइउ} इत्यत्र समासेऽप्यसन्धिश्छान्दसः। \textcolor{red}{अइउना ण्} इति विग्रहे \textcolor{red}{कर्तृकरणे कृता बहुलम्‌} (पा॰सू॰~२.१.३२) इत्यनेन तृतीया\-तत्पुरुष\-समासः। विभक्तिकार्ये \textcolor{red}{हल्ङ्याब्भ्यो दीर्घात्सुतिस्यपृक्तं हल्} (पा॰सू॰~६.१.६८) इत्यनेन सुलोपे सिद्धम्।} अकारेकारोकार\-वाचकान् राम\-सीता\-लक्ष्मणान्नयति प्रापयतीति \textcolor{red}{णीञ्‌}\-धातोः (\textcolor{red}{णीञ् प्रापणे} धा॰पा॰~९०१) \textcolor{red}{अइउण्} इत्यपि।\footnote{\textcolor{red}{अइउ} इत्यत्र समासेऽप्यसन्धिश्छान्दसः। \textcolor{red}{अइउ} इत्युपपदे \textcolor{red}{नी}\-धातोरौणादिको \textcolor{red}{ड्विन्‌}\-प्रत्ययः। \textcolor{red}{कार्याद्विद्यादनूबन्धम्} (भा॰पा॰सू॰~३.३.१) \textcolor{red}{केचिदविहिता अप्यूह्याः} (वै॰सि॰कौ॰~३१६९) इत्यनुसारमूह्योऽ\-यमविहित\-प्रत्ययः। सर्वापहारि\-लोपे \textcolor{red}{डित्यभस्याप्यनु\-बन्धकरण\-सामर्थ्यात्} (वा॰~६.४.१४३) इत्यनेन टिलोपे सौत्रत्वान्नकारस्य णकार उपपद\-समासे विभक्तिकार्ये \textcolor{red}{हल्ङ्याब्भ्यो दीर्घात्सुतिस्यपृक्तं हल्} (पा॰सू॰~६.१.६८) इत्यनेन सुलोपे सिद्धम्।}\end{sloppypar}
\begin{sloppypar}\hyphenrules{nohyphenation}\justifying\noindent\hspace{10mm} एवमेव~–\end{sloppypar}
\begin{sloppypar}\hyphenrules{nohyphenation}\justifying\noindent\hspace{10mm} \textcolor{red}{(२) ऋऌक्}~– \textcolor{red}{ऋ} ऋषिः।\footnote{\textcolor{red}{नामैकदेशग्रहणे नाममात्रग्रहणम्} इत्यनेन न्यायेन।} \textcolor{red}{ऌ} ऌकारवज्जटिल\-साधना\-रतो मुनिः।\footnote{\textcolor{red}{नामैकदेशग्रहणे नाममात्रग्रहणम्} इत्यनेन न्यायेन।} तावेव कथयत्यात्म\-सम्मुखं करोतीति \textcolor{red}{ऋऌक्}।\footnote{\textcolor{red}{ऋऌ} इत्यत्र समासेऽप्यसन्धिश्छान्दसः। \textcolor{red}{ऋऌ} इत्युपपदे \textcolor{red}{कथ्‌}\-धातोः (\textcolor{red}{कथँ वाक्य\-प्रबन्धने} धा॰पा॰~१८५१) पूर्ववदौणादिको \textcolor{red}{ड्विन्‌}प्रत्ययः। पूर्ववत्सर्वापहारि\-लोपे टिलोप उपपद\-समासे विभक्तिकार्ये \textcolor{red}{हल्ङ्याब्भ्यो दीर्घात्सुतिस्यपृक्तं हल्} (पा॰सू॰~६.१.६८) इत्यनेन सुलोपे सिद्धम्।}\end{sloppypar}
\begin{sloppypar}\hyphenrules{nohyphenation}\justifying\noindent\hspace{10mm} \textcolor{red}{(३) एओङ्}~– एवम् \textcolor{red}{ए} षडैश्वर्य\-वाचको रामः।\footnote{माहेश्वर\-सूत्रेषु \textcolor{red}{ए}\-कारस्य क्रमसङ्ख्यानुसारम्। ऐश्वर्यं धर्मो यशः श्रीर्ज्ञानं वैराग्यञ्चेति षडैश्वर्याणि। \textcolor{red}{ऐश्वर्यस्य समग्रस्य धर्मस्य यशसः श्रियः। ज्ञानवैराग्ययोश्चैव षण्णां भग इतीरणा॥} (वि॰पु॰~६.५.७४)।} \textcolor{red}{ओ} सप्तावरण\-नाशिनी सीता।\footnote{माहेश्वर\-सूत्रेषु \textcolor{red}{ओ}\-कारस्य क्रमसङ्ख्यानुसारम्। क्षिति\-जल\-पावक\-गगन\-समीर\-महत्प्रकृतयः सप्तावरणानि। यथा रामचरितमानस उत्तरकाण्डे काकभुशुण्डि\-गीतायां मानसकाराः \textcolor{red}{सप्ताबरन भेद करि जहँ लगि रहि गति मोरि। गयउँ तहाँ प्रभु भुज निरखि ब्याकुल भयउँ बहोरि॥} (रा॰च॰मा॰~७.७९ख)। एतद्रूपान्तरम्–\textcolor{red}{प्रभेद्य सप्तावरणानि यावद्गतिर्ममासीदगमञ्च तावत्। तत्रापि वीक्ष्येशभुजौ स्वपृष्ठे तीव्राकुलत्वेन युतोऽहमासम्॥} (मा॰भा॰~७.७९ख)।} तावञ्चति पूजयति इति \textcolor{red}{एओङ्}।\footnote{\textcolor{red}{एओ} इत्यत्र समासेऽप्यसन्धिश्छान्दसः। \textcolor{red}{एओ} इत्युपपदे \textcolor{red}{अञ्च्‌}\-धातोः (\textcolor{red}{अञ्चुँ गतिपूजनयोः} धा॰पा॰~१८८) \textcolor{red}{क्विप् च} (पा॰सू॰~३.२.७६) इत्यनेन \textcolor{red}{क्विप्} प्रत्ययः। सर्वापहारि\-लोपे उपपद\-समासे \textcolor{red}{संयोगान्तस्य लोपः} (पा॰सू॰~८.२.२३) इत्यनेन चकार\-लोपे \textcolor{red}{चोः कुः} (पा॰सू॰~८.२.३०) इत्यनेन ञकारस्य ङत्वे \textcolor{red}{एङः पदान्तादति} (पा॰सू॰~६.१.१०९) इत्यनेन पूर्वरूपैकादेशे विभक्तिकार्ये \textcolor{red}{हल्ङ्याब्भ्यो दीर्घात्सुतिस्यपृक्तं हल्} (पा॰सू॰~६.१.६८) इत्यनेन सुलोपे सिद्धम्।} सीता\-राम\-पूजकमिति तात्पर्यम्।\end{sloppypar}
\begin{sloppypar}\hyphenrules{nohyphenation}\justifying\noindent\hspace{10mm} \textcolor{red}{(४) ऐऔच्}~– \textcolor{red}{ऐ} अष्टप्रकृत्यात्मिका सीता।\footnote{माहेश्वर\-सूत्रेषु \textcolor{red}{ऐ}\-कारस्य क्रमसङ्ख्यानुसारम्। \textcolor{red}{भूमिरापोऽनलो वायुः खं मनो बुद्धिरेव च। अहङ्कार इतीयं मे भिन्ना प्रकृतिरष्टधा॥} (भ॰गी॰~७.४) इत्यष्टधा प्रकृतिः।} \textcolor{red}{औ} नवम\-सङ्ख्या\-वाच्यो रामः।\footnote{माहेश्वर\-सूत्रेषु \textcolor{red}{औ}\-कारस्य क्रमसङ्ख्यानुसारम्। राम\-नवमसङ्ख्ययोः साम्यं पूर्वमेव कथितं प्रणेतृभिः।} तौ चिनोत्यन्तर्भावित\-ण्यर्थतया निश्चाययतीति \textcolor{red}{ऐऔच्}।\footnote{\textcolor{red}{ऐऔ} इत्यत्र समासेऽप्यसन्धिश्छान्दसः। \textcolor{red}{ऐऔ} इत्युपपदे \textcolor{red}{चि}\-धातोः (\textcolor{red}{चिञ् चयने} धा॰पा॰~१२५१) पूर्ववदौणादिको \textcolor{red}{ड्विन्‌}प्रत्ययः। पूर्ववत्सर्वापहारि\-लोपे टिलोप उपपद\-समासे विभक्तिकार्ये \textcolor{red}{हल्ङ्याब्भ्यो दीर्घात्सुतिस्यपृक्तं हल्} (पा॰सू॰~६.१.६८) इत्यनेन सुलोपे सिद्धम्। \textcolor{red}{अयस्मयादीनि च्छन्दसि} (पा॰सू॰~१.४.२०) इत्यनेन छान्दस\-भसञ्ज्ञायां पदत्वाभावे \textcolor{red}{चोः कुः} (पा॰सू॰~८.२.३०) इत्यस्य प्रवृत्तिर्न। यद्वा प्रत्याहारेष्वसन्देहार्थं कुत्वाभावः।} सीता\-राम\-निश्चायकमित्यर्थः।\end{sloppypar}
\begin{sloppypar}\hyphenrules{nohyphenation}\justifying\noindent\hspace{10mm} इत्थं सूत्र\-चतुष्टयेनैश्वर्य\-लीलात्मकं रामायणं प्रतिपादितम्। इदानीं माधुर्य\-लीला\-परं रामायणं बाल\-बुद्ध्या विविच्यते दशभिः सूत्रैः। तत्र माधुर्ये पूर्व\-निर्दिष्ट\-दश\-गुण\-प्रतिपादकानि दश सूत्राणि \textcolor{red}{दशमस्त्वमसि}\footnote{मूलं वेदशाखासु मृग्यम्।} इति श्रुतेर्वाच्यतावच्छेदकस्य लक्ष्यतावच्छेदकस्य भगवतः श्रीरामस्य माधुर्य\-गुण\-बृंहितं सौन्दर्य\-सार\-सर्वस्वं दिव्यं चरित्रं शिवेन लोकोत्तर\-कौशल\-पुरःसरं सङ्केतितम्। अत्र वर्णानां द्वौ विभागौ स्वरो व्यञ्जनञ्च। एवमेव ब्रह्मणो द्वे स्वरूपे निर्गुणं सगुणञ्च। तत्र निर्गुण\-लीला प्रतिपादिता। साम्प्रतं सगुण\-लीलाऽपि प्रतिपाद्यते। पञ्चमात्सूत्राद्व्यञ्जनानां वर्णनं प्रारब्धम्। तत्र व्यञ्जनं सीताया रामेण मिश्रणम्। सगुण\-लीलायां सीतया रामोऽभिन्नः। यथा~–\end{sloppypar}
\centering\textcolor{red}{अनन्या राघवेणाहं भास्करेण यथा प्रभा॥}\nopagebreak\\
\raggedleft{–~वा॰रा॰~५.२१.१५}\\
\centering\textcolor{red}{अनन्या हि मया सीता भास्करेण यथा प्रभा ॥}\nopagebreak\\
\raggedleft{–~वा॰रा॰~६.११८.१९}\\
\begin{sloppypar}\hyphenrules{nohyphenation}\justifying\noindent इति वाल्मीकीये रामायणे श्री\-सीता\-रामाभ्यामुक्तत्वात्। उभयत्र तृतीया प्रकृत्यादित्वात्।\footnote{\textcolor{red}{प्रकृत्यादिभ्य उपसङ्ख्यानम्} (वा॰~२.३.१८) इत्यनेन।} व्यञ्जनानां षड्वर्गा यवर्गः कवर्गश्चवर्गष्टवर्गस्तवर्गः पवर्गश्चेति। एषां कीर्तनेन जीवानां विकार\-षड्वर्ग\-नाशनं सूचितम्। किं बहुना स्वर\-वर्गं मिलित्वा चतुर्दश\-सूत्रेषु सप्त\-वर्गाः। एवं सप्तभिर्वर्गैरक्षराणां सप्त\-काण्डात्मकं रामायणं सुस्पष्टं कीर्तितम्। सूत्राणि सार्वभौमानि विश्वतोमुखानि छन्दः\-स्वरूपाणि भवन्ति। तत्र स्थूलानामक्षराणां निर्देशत्वेऽप्यक्षराणामक्षरस्य भगवतो रामचन्द्रस्य रामायणी गाथा कथं न निर्दिष्टा स्यात्। भगवत्सङ्कीर्तनं विनैषु पुण्य\-जनकतावच्छेदकता कथं स्यात्। यतो रामायण\-कीर्तनं चतुर्दश\-भुवन\-व्यापकं चतुर्दश\-सूत्रेष्वतो लोकोत्तर\-पुण्य\-जनकता। तस्माद्भाष्यकारश्चतुर्दश\-सूत्राणां सानन्दं प्रशंसां कुर्वन्नाह \textcolor{red}{सोऽयमक्षर\-समाम्नायो वाक्समाम्नायः पुष्पितश्चन्द्र\-तारकवत्प्रतिमण्डितो वेदितव्यो ब्रह्म\-राशिः। सर्व\-वेद\-पुण्य\-फलावाप्तिश्चास्य ज्ञाने भवति} (भा॰शि॰सू॰)। तर्हि ब्रह्मणो रामस्य चर्चां विना पूर्वोक्त\-पुण्यजनकताऽन्येषु सन्दिग्धा स्यात्। यतो भाष्य\-प्रमाणम्। यया पुण्यजनकता यतश्च पुण्यजनकता। पूर्वं
चतुर्भिः सूत्रैर्बालकाण्डं चर्चितमिदानीमयोध्याकाण्डमुपक्रमते।\end{sloppypar}
\begin{sloppypar}\hyphenrules{nohyphenation}\justifying\noindent\hspace{10mm} \textcolor{red}{(५) हयवरट्}~– एवं वर्ण\-संयोजनेन हयेषु घोटकेषु वरा \textcolor{red}{हयवराः} श्रेष्ठ\-घोटकास्तैरटति वनमिति \textcolor{red}{हयवरट्}। शकन्ध्वादित्वात्पररूपं सौत्रत्वाद्वा।\footnote{\textcolor{red}{हयवर} उपपदे \textcolor{red}{अटँ गतौ} (धा॰पा॰~३३२) इति धातोः \textcolor{red}{क्विप् च} (पा॰सू॰~३.२.७६) इत्यनेन कर्तरि क्विप्। सर्वापहारि\-लोप उपपद\-समासे \textcolor{red}{शकन्ध्वादिषु पररूपं वाच्यम्} (वा॰~६.१.९४) इत्यनेन पूर्वरूपे विभक्तिकार्ये \textcolor{red}{हल्ङ्याब्भ्यो दीर्घात्सुतिस्यपृक्तं हल्} (पा॰सू॰~६.१.६८) इत्यनेन सुलोपे सिद्धम्।} अथवा \textcolor{red}{हयवर}\-संयोजित\-रथेन पित्रादिष्टः सीता\-लक्ष्मण\-सहायः श्रीरामो वनमटतीति \textcolor{red}{हयवरट्}।\footnote{प्रक्रिया पूर्ववत्।}\end{sloppypar}
\begin{sloppypar}\hyphenrules{nohyphenation}\justifying\noindent\hspace{10mm} \textcolor{red}{(६) लण्}~– लसतीति \textcolor{red}{लः}। सौत्रत्वाट्टिलोपः सलोपश्च।\footnote{\textcolor{red}{लसँ श्लेषण\-क्रीडनयोः} (धा॰पा॰~७१४)~\arrow लस्~\arrow \textcolor{red}{नन्दि\-ग्रहि\-पचादिभ्यो ल्युणिन्यचः} (पा॰सू॰~३.१.१३४)~\arrow लस् अच्~\arrow लस् अ~\arrow लस~\arrow सौत्रटिलोपः~\arrow लस्~\arrow सौत्रसलोपः~\arrow ल~\arrow विभक्तिकार्यम्~\arrow लः। यद्वा लस् धातोः \textcolor{red}{अन्येष्वपि दृश्यते} (पा॰सू॰~३.२.१०१) इत्यनेनोपपदाभावेऽपि डप्रत्ययः। लस् ड~\arrow लस् अ~\arrow \textcolor{red}{डित्यभस्याप्यनु\-बन्धकरण\-सामर्थ्यात्} (वा॰~६.४.१४३)~\arrow ल् अ~\arrow ल~\arrow विभक्तिकार्यम्~\arrow लः।} तस्मिन् ले प्रकृति\-सौन्दर्य\-लसिते चित्रकूटे भक्तानानन्दं नयतीति कष्टान्निवर्तयति वेति \textcolor{red}{लण्} चित्रकूटस्थो रामः। सौत्रत्वाण्णत्वम्।\footnote{\textcolor{red}{ल} इत्युपपदे \textcolor{red}{नी}\-धातोः (\textcolor{red}{णीञ् प्रापणे} धा॰पा॰~९०१) पूर्ववदौणादिको \textcolor{red}{ड्विन्‌}\-प्रत्ययः। पूर्ववत्सर्वापहारि\-लोपे टिलोपे सौत्रणत्व उपपद\-समासे विभक्तिकार्ये \textcolor{red}{हल्ङ्याब्भ्यो दीर्घात्सुतिस्यपृक्तं हल्} (पा॰सू॰~६.१.६८) इत्यनेन सुलोपे सिद्धम्।} यद्वा लाति भक्तिं यः स \textcolor{red}{लः} चित्र\-कूटः।\footnote{\textcolor{red}{ला आदाने। द्वावपि (रा ला) दाने इति चन्द्रः} (धा॰पा॰~१०५८)। ला~\arrow \textcolor{red}{आतश्चोपसर्गे} (पा॰सू॰~३.१.१३६)~\arrow बाहुलकादनुपसर्गे कः~\arrow ला~क~\arrow ला~अ~\arrow \textcolor{red}{आतो लोप इटि च} (पा॰सू॰~६.४.६४)~\arrow ल्~अ~\arrow ल~\arrow विभक्तिकार्यम्~\arrow लः। यद्वा ला~\arrow \textcolor{red}{अन्येष्वपि दृश्यते} (पा॰सू॰~३.२.१०१)~\arrow ला~ड~\arrow ला~अ~\arrow \textcolor{red}{डित्यभस्याप्यनु\-बन्धकरण\-सामर्थ्यात्} (वा॰~६.४.१४३)~\arrow ल्~अ~\arrow ल~\arrow विभक्तिकार्यम्~\arrow लः। ददातीति दः (\textcolor{red}{पुमांस्तु दातरि स्मृतः} मे॰को॰~१८.१) इतिवल्लातीति लः।} तस्मिन्नाशयति भक्तकष्टं जयन्त\-दर्पञ्च यः स \textcolor{red}{लण्} चित्रकूटस्थः श्रीरामः। सौत्रत्वाट्टिलोपो णत्वञ्च।\footnote{\textcolor{red}{ल} उपपदे \textcolor{red}{णशँ अदर्शने} (धा॰पा॰~११९४) इत्यतो णिजन्तात् \textcolor{red}{नाशि} धातोः \textcolor{red}{क्विप् च} (पा॰सू॰~३.२.७६) इत्यनेन कर्तरि क्विप्। सर्वापहारि\-लोपे \textcolor{red}{णेरनिटि} (पा॰सू॰~६.४.५१) इत्यनेन णिलोपे \textcolor{red}{नाश्} इति जाते सौत्रटिलोपे सौत्रणत्वे चोपपद\-समासे विभक्तिकार्ये \textcolor{red}{हल्ङ्याब्भ्यो दीर्घात्सुतिस्यपृक्तं हल्} (पा॰सू॰~६.१.६८) इत्यनेन सुलोपे सिद्धम्।}\end{sloppypar}
\begin{sloppypar}\hyphenrules{nohyphenation}\justifying\noindent\hspace{10mm} \textcolor{red}{(७) ञमङणनम्}~– प्रतिवर्गान्तिम\-भूतान् खर\-दूषण\-त्रिशीर्ष\-मारीच\-कबन्धान्मीनातीति \textcolor{red}{ञमङणनम्} श्रीरामः।\footnote{\textcolor{red}{ञमङणन} इत्युपपदे \textcolor{red}{मी}\-धातोः (\textcolor{red}{मीञ् हिंसायाम्} धा॰पा॰~१४७६) पूर्ववदौणादिको \textcolor{red}{ड्विन्‌}\-प्रत्ययः। पूर्ववत्सर्वापहारि\-लोपे टिलोप उपपदसमासे विभक्तिकार्ये \textcolor{red}{हल्ङ्याब्भ्यो दीर्घात्सुतिस्यपृक्तं हल्} (पा॰सू॰~६.१.६८) इत्यनेन सुलोपे सिद्धम्।}\end{sloppypar}
\begin{sloppypar}\hyphenrules{nohyphenation}\justifying\noindent\hspace{10mm} \textcolor{red}{(८) झभञ्}~– \textcolor{red}{झॄष् वयो\-हानौ} (धा॰पा॰~११३१)। झीर्यतीति \textcolor{red}{झः}। सुग्रीवः। वालि\-त्रासाज्जीर्णो भवतीति भावः।\footnote{\textcolor{red}{झॄष् वयो\-हानौ} (धा॰पा॰~११३१)~\arrow झॄ~\arrow \textcolor{red}{अन्येष्वपि दृश्यते} (पा॰सू॰~३.२.१०१)~\arrow झॄ~ड~\arrow झॄ~अ~\arrow \textcolor{red}{डित्यभस्याप्यनु\-बन्धकरण\-सामर्थ्यात्} (वा॰~६.४.१४३)~\arrow झ्~अ~\arrow झ~\arrow विभक्तिकार्यम्~\arrow झः।} भातीति \textcolor{red}{भः}। हनुमान्।\footnote{\textcolor{red}{भा दीप्तौ} (धा॰पा॰~१०५१)~\arrow भा~\arrow \textcolor{red}{अन्येष्वपि दृश्यते} (पा॰सू॰~३.२.१०१)~\arrow भा~ड~\arrow भा~अ~\arrow \textcolor{red}{डित्यभस्याप्यनु\-बन्धकरण\-सामर्थ्यात्} (वा॰~६.४.१४३)~\arrow भ्~अ~\arrow भ~\arrow विभक्तिकार्यम्~\arrow भः।} तौ झभौ सुग्रीव\-हनुमन्तौ यन्त्रयत्यानन्देनेति \textcolor{red}{झभञ्} श्रीरामः। सौत्रत्वाद्यकारस्य ञकारः।\footnote{\textcolor{red}{झभ} उपपदे \textcolor{red}{यत्रिँ सङ्कोचे} (धा॰पा॰~१५३६) इति धातोः पूर्ववदौणादिको \textcolor{red}{ड्विन्‌}\-प्रत्ययः। पूर्ववत्सर्वापहारि\-लोपे टिलोप उपपद\-समासे सौत्रत्वाद्यकारस्य ञकारे विभक्ति\-कार्ये \textcolor{red}{हल्ङ्याब्भ्यो दीर्घात्सुतिस्यपृक्तं हल्} (पा॰सू॰~६.१.६८) इत्यनेन सुलोपे सिद्धम्। \textcolor{red}{अयस्मयादीनि च्छन्दसि} (पा॰सू॰~१.४.२०) इत्यनेन छान्दस\-भसञ्ज्ञायां पदत्वाभावे \textcolor{red}{चोः कुः} (पा॰सू॰~८.२.३०) इत्यस्य प्रवृत्तिर्न।} सुग्रीव\-वायु\-पुत्र\-तोष\-कर्तेति तात्पर्यम्।\end{sloppypar}
\begin{sloppypar}\hyphenrules{nohyphenation}\justifying\noindent\hspace{10mm} \textcolor{red}{(९) घढधष्}~– घढमभिमानं\footnote{व्युत्पत्तिः साध्या।} दधातीति \textcolor{red}{घढधः} वाली।\footnote{\textcolor{red}{घढ} उपपदे \textcolor{red}{डुधाञ् धारण\-पोषणयोः} (धा॰पा॰~१०९२) इति धातोः \textcolor{red}{अन्येष्वपि दृश्यते} (पा॰सू॰~३.२.१०१) इत्यनेन \textcolor{red}{ड}\-प्रत्ययः। अनुबन्ध\-लोपे \textcolor{red}{डित्यभस्याप्यनु\-बन्धकरण\-सामर्थ्यात्} (वा॰~६.४.१४३) इत्यनेन टिलोप उपपदसमासे विभक्तिकार्ये सिद्धम्।} तमेव स्यति खण्डयतीति \textcolor{red}{घढधष्}।\footnote{\textcolor{red}{घढध} उपपदे \textcolor{red}{षो अन्तकर्मणि} (धा॰पा॰~११४७) इति धातोः पूर्ववदौणादिको \textcolor{red}{ड्विन्‌}\-प्रत्ययः। पूर्ववत्सर्वापहारि\-लोपे टिलोप उपपद\-समासे सौत्रत्वात्सकारस्य षकारे विभक्ति\-कार्ये \textcolor{red}{हल्ङ्याब्भ्यो दीर्घात्सुतिस्यपृक्तं हल्} (पा॰सू॰~६.१.६८) इत्यनेन सुलोपे सिद्धम्। \textcolor{red}{अयस्मयादीनि च्छन्दसि} (पा॰सू॰~१.४.२०) इत्यनेन छान्दस\-भसञ्ज्ञायां पदत्वाभावे \textcolor{red}{व्रश्चभ्रस्ज\-सृजमृज\-यजराज\-भ्राजच्छशां षः} (पा॰सू॰~८.२.३६) \textcolor{red}{झलां जशोऽन्ते} (पा॰सू॰~८.२.३९) इत्यनयोः प्रवृत्तिर्न।} वालि\-नाशको राम इति तात्पर्यम्।\end{sloppypar}
\begin{sloppypar}\hyphenrules{nohyphenation}\justifying\noindent\hspace{10mm} \textcolor{red}{(१०) जबगडदश्}~– जयतीति \textcolor{red}{जः}। बालयतीति \textcolor{red}{बः}। गर्जतीति \textcolor{red}{गः}। डम्बयतीति \textcolor{red}{डः}। दमयति निशाचरानिति \textcolor{red}{दः}।\footnote{नाना\-धातुभ्यः \textcolor{red}{अन्येष्वपि दृश्यते} (पा॰सू॰~३.२.१०१) इत्यनेन \textcolor{red}{ड}\-प्रत्यये \textcolor{red}{डित्यभस्याप्यनु\-बन्धकरण\-सामर्थ्यात्} (वा॰~६.४.१४३) इत्यनेन टिलोप उपपदसमासे विभक्तिकार्ये सिद्धानि पञ्चैतानि। \textcolor{red}{जि जये} (धा॰पा॰~५६१) इति धातोः \textcolor{red}{जः}। \textcolor{red}{बलँ प्राणने धान्यावरोधने च} (धा॰पा॰~८४०) इत्यतो णिचि \textcolor{red}{बालि} धातुः। ततः \textcolor{red}{ड}\-प्रत्यये \textcolor{red}{बः}। \textcolor{red}{णेरनिटि} (पा॰सू॰~६.४.५१) इत्यनेन णिलोपः शेषा प्रक्रिया पूर्ववत्। न च \textcolor{red}{बलँ प्राणने} (धा॰पा॰~१६२८) इति धातोर्णिचि डे णिलोपे टिलोपे विभक्तिकार्येऽपि सिद्धिः। \textcolor{red}{बलँ प्राणने} (धा॰पा॰~१६२८) इत्यस्य ज्ञपादित्वान्मित्त्वम्। ततो णिच्युपधा\-वृद्धौ \textcolor{red}{मितां ह्रस्वः} (पा॰सू॰~६.४.९२) इत्यनेन ह्रस्वे \textcolor{red}{बलयति} इति रूपं प्रणेतारस्तु \textcolor{red}{बालयति} इत्याहुः। \textcolor{red}{गर्जँ शब्दे} (धा॰पा॰~२२६) इति धातोः \textcolor{red}{गृजँ शब्दे} (धा॰पा॰~२४८) इत्यतो वा \textcolor{red}{गः}। \textcolor{red}{डिपँ सङ्घाते डबिँ डिबिँ इति चान्द्रः} (धा॰पा॰~१६७७) इत्यत्र \textcolor{red}{डम्ब्‌}\-(\textcolor{red}{डबिँ})\-धातोः \textcolor{red}{डः}। \textcolor{red}{दमुँ उपशमे} (धा॰पा॰~१२०३) इत्यतो णिचि \textcolor{red}{दमि} धातुः। \textcolor{red}{जनी\-जॄष्क्नसु\-रञ्जोऽमन्ताश्च} (धा॰पा॰ ग॰सू॰) इत्यनेन मित्त्वादुपधा\-वृद्धौ \textcolor{red}{मितां ह्रस्वः} (पा॰सू॰~६.४.९२) इत्यनेन ह्रस्वः। \textcolor{red}{दमि} धातोः \textcolor{red}{दः}। \textcolor{red}{णेरनिटि} (पा॰सू॰~६.४.५१) इत्यनेन णिलोपः शेषा प्रक्रिया पूर्ववत्।} एतत्पञ्च\-गुण\-सम्पन्न\-जबगडदस्य हृदये शेत इति \textcolor{red}{जबगडदश्}। वायु\-पुत्र\-हृदयस्थो रामः।\footnote{\textcolor{red}{जबगडद} उपपदे \textcolor{red}{शीङ् स्वप्ने} (धा॰पा॰~१०३२) इति धातोः पूर्ववदौणादिको \textcolor{red}{ड्विन्‌}\-प्रत्ययः। पूर्ववत्सर्वापहारि\-लोपे टिलोप उपपद\-समासे विभक्ति\-कार्ये \textcolor{red}{हल्ङ्याब्भ्यो दीर्घात्सुतिस्यपृक्तं हल्} (पा॰सू॰~६.१.६८) इत्यनेन सुलोपे सिद्धम्। \textcolor{red}{अयस्मयादीनि च्छन्दसि} (पा॰सू॰~१.४.२०) इत्यनेन छान्दस\-भसञ्ज्ञायां पदत्वाभावे \textcolor{red}{झलां जशोऽन्ते} (पा॰सू॰~८.२.३९) इत्यस्य प्रवृत्तिर्न।}\end{sloppypar}
\begin{sloppypar}\hyphenrules{nohyphenation}\justifying\noindent\hspace{10mm} \textcolor{red}{(११) खफछठथचटतव्}~– खनतीति \textcolor{red}{खः}। फलतीति \textcolor{red}{फः}। छ्यतीति \textcolor{red}{छः}।\footnote{नाना\-धातुभ्यः \textcolor{red}{अन्येष्वपि दृश्यते} (पा॰सू॰~३.२.१०१) इत्यनेन \textcolor{red}{ड}\-प्रत्यये \textcolor{red}{डित्यभस्याप्यनु\-बन्धकरण\-सामर्थ्यात्} (वा॰~६.४.१४३) इत्यनेन टिलोपे विभक्तिकार्ये सिद्धानि त्रीण्येतानि। \textcolor{red}{खनुँ अवदारणे} (धा॰पा॰~८७८) इति धातोः \textcolor{red}{खः}। \textcolor{red}{ञिफलाँ विशरणे} (धा॰पा॰~५१६) इति धातोः \textcolor{red}{फः}। \textcolor{red}{छो छेदने} (धा॰पा॰~११४६) इति धातोः \textcolor{red}{छः}।} लोठतीति \textcolor{red}{ठः}। लुकार\-लोपश्छान्दसः।\footnote{\textcolor{red}{लुठँ उपघाते} (धा॰पा॰~३३७) इति धातोः \textcolor{red}{नन्दि\-ग्रहि\-पचादिभ्यो ल्युणिन्यचः} (पा॰सू॰~३.१.१३४) इत्यनेन कर्तरि पचाद्यचि छान्दस\-लुकार\-लोपे विभक्तिकार्ये सिद्धम्।} थूर्वतीति \textcolor{red}{थः}। चिनोति दुर्गुणानिति \textcolor{red}{चः}। टङ्कयत्यसद्विचारानिति \textcolor{red}{टः}। ताम्यतीति \textcolor{red}{तः}।\footnote{नाना\-धातुभ्यः \textcolor{red}{अन्येष्वपि दृश्यते} (पा॰सू॰~३.२.१०१) इत्यनेन \textcolor{red}{ड}\-प्रत्यये \textcolor{red}{डित्यभस्याप्यनु\-बन्धकरण\-सामर्थ्यात्} (वा॰~६.४.१४३) इत्यनेन टिलोपे विभक्तिकार्ये सिद्धानि चत्वार्येतानि। \textcolor{red}{थुर्वीँ हिंसायाम्} (धा॰पा॰~५७१) इति धातोः \textcolor{red}{थः}। \textcolor{red}{चिञ् चयने} (धा॰पा॰~१२५१) इति धातोः \textcolor{red}{चः}। \textcolor{red}{टकिँ बन्धने} (धा॰पा॰~१६३८) इति धातोः \textcolor{red}{टः}। \textcolor{red}{णेरनिटि} (पा॰सू॰~६.४.५१) इत्यनेन णिलोपः शेषा प्रक्रिया पूर्ववत्। \textcolor{red}{तमुँ काङ्क्षायाम्} (धा॰पा॰~१२०२) इति धातोः \textcolor{red}{तः}।} एवं खफछठथचटतं रावणमपि वधतीति \textcolor{red}{खफछठथचटतव्}।\footnote{\textcolor{red}{वधति} इत्यत्र \textcolor{red}{वधँ हिंसायाम्} भौवादिकः सौत्रो धातुः। स च \textcolor{red}{जनिवध्योश्च} (पा॰सू॰~७.३.३५) इति सूत्रेण ज्ञापितः। \textcolor{red}{‘जनिवध्योश्च’। जनकः। ‘वधँ हिंसायाम्’। वधकः} (वै॰सि॰कौ॰~२८९५)। \textcolor{red}{‘वधँ हिंसायामिति’। धात्वन्तरं भौवादिकम्। भ्वादेराकृतिगणत्वात्} (बा॰म॰~२८९५)। एवं तर्हि \textcolor{red}{खफछठथचटत} उपपदे \textcolor{red}{वधँ हिंसायाम्} इति धातोः पूर्ववदौणादिको \textcolor{red}{ड्विन्‌}\-प्रत्ययः। पूर्ववत्सर्वापहारि\-लोपे  टिलोप उपपद\-समासे विभक्ति\-कार्ये \textcolor{red}{हल्ङ्याब्भ्यो दीर्घात्सुतिस्यपृक्तं हल्} (पा॰सू॰~६.१.६८) इत्यनेन सुलोपे सिद्धम्।} अथवा खफछठथचटत\-ध्वनि\-कुर्वाणं युद्धाय स्वकीयं चाप\-विशेषं वर्धयतीति \textcolor{red}{खफछठथचटतव्}।\footnote{\textcolor{red}{वृधुँ वृद्धौ} (धा॰पा॰~७५९) इति धातोर्णिचि \textcolor{red}{वर्धि} इति धातुः। \textcolor{red}{खफछठथचटत} उपपदे \textcolor{red}{वर्धि}\-धातोः पूर्ववदौणादिको \textcolor{red}{ड्विन्‌}\-प्रत्ययः। पूर्ववत्सर्वापहारि\-लोपे \textcolor{red}{णेरनिटि} (पा॰सू॰~६.४.५१) इत्यनेन णिलोपे टिलोप उपपद\-समासे विभक्ति\-कार्ये \textcolor{red}{हल्ङ्याब्भ्यो दीर्घात्सुतिस्यपृक्तं हल्} (पा॰सू॰~६.१.६८) इत्यनेन सुलोपे सिद्धम्।} अथवैतद्ध्वनि\-युक्तं वानर\-दलमवतीति \textcolor{red}{खफछठथचटतव्}। शकन्ध्वादित्वात्पूर्वरूपम्।\footnote{\textcolor{red}{खफछठथचटत} उपपदे \textcolor{red}{अवँ रक्षण\-गति\-कान्ति\-प्रीति\-तृप्त्यवगम\-प्रवेश\-श्रवण\-स्वाम्यर्थ\-याचन\-क्रियेच्छा\-दीप्त्यवाप्त्यालिङ्गन\-हिंसा\-दान\-भाग\-वृद्धिषु} (धा॰पा॰~६००) इति धातोः \textcolor{red}{क्विप् च} (पा॰सू॰~३.२.७६) इत्यनेन \textcolor{red}{क्विप्} प्रत्यये सर्वापहारि\-लोप उपपदसमासे \textcolor{red}{शकन्ध्वादिषु पर\-रूपं वाच्यम्} इत्यनेन पररूपे विभक्ति\-कार्ये \textcolor{red}{हल्ङ्याब्भ्यो दीर्घात्सुतिस्यपृक्तं हल्} (पा॰सू॰~६.१.६८) इत्यनेन सुलोपे सिद्धम्।}\end{sloppypar}
\begin{sloppypar}\hyphenrules{nohyphenation}\justifying\noindent\hspace{10mm} \textcolor{red}{(१२) कपय्}~– कायति विभीषणं राज्य\-दानाय शब्दापयति\footnote{\textcolor{red}{कायतीति कः} इत्यर्थः। \textcolor{red}{कै शब्दे} (धा॰पा॰~९१६)~\arrow कै~\arrow \textcolor{red}{अन्येष्वपि दृश्यते} (पा॰सू॰~३.२.१०१)~\arrow कै~ड~\arrow कै~अ~\arrow \textcolor{red}{डित्यभस्याप्यनु\-बन्धकरण\-सामर्थ्यात्} (वा॰~६.४.१४३)~\arrow क्~अ~\arrow क~\arrow विभक्तिकार्यम्~\arrow कः।} पाति वानर\-सेनां पुष्पकारोहेण\footnote{\textcolor{red}{पातीति पः} इत्यर्थः। \textcolor{red}{पा रक्षणे} (धा॰पा॰~१०५६)~\arrow पा~\arrow \textcolor{red}{अन्येष्वपि दृश्यते} (पा॰सू॰~३.२.१०१)~\arrow पा~ड~\arrow पा~अ~\arrow \textcolor{red}{डित्यभस्याप्यनु\-बन्धकरण\-सामर्थ्यात्} (वा॰~६.४.१४३)~\arrow प्~अ~\arrow प~\arrow विभक्तिकार्यम्~\arrow पः। \textcolor{red}{पः स्यात्पाने च पातरि} (ए॰को॰~२४) इति कोशादपि।} यात्ययोध्याम्\footnote{\textcolor{red}{यातीति य्} इत्यर्थः। \textcolor{red}{या प्रापणे} (धा॰पा॰~१०४९) इति धातोः पूर्ववदौणादिको \textcolor{red}{ड्विन्‌}\-प्रत्ययः। पूर्ववत्सर्वापहारि\-लोपे टिलोपे विभक्ति\-कार्ये \textcolor{red}{हल्ङ्याब्भ्यो दीर्घात्सुतिस्यपृक्तं हल्} (पा॰सू॰~६.१.६८) इत्यनेन सुलोपे \textcolor{red}{य्}। \textcolor{red}{याने यातरि यस्त्यागे} (ए॰को॰~२९) इति कोशादकारान्तोऽपि यातरि।} इति \textcolor{red}{कपय्}।\footnote{\textcolor{red}{कश्च पश्च य् चेति कपय्}।} दत्त\-विभीषण\-राज्य\-लक्ष्मीः पुष्पकारूढः सीताभिरामो रामोऽयोध्यां प्रति प्रतिष्ठत इति भावः।\end{sloppypar}
\begin{sloppypar}\hyphenrules{nohyphenation}\justifying\noindent\hspace{10mm} \textcolor{red}{(१३) शषसर्}~– तथा श्यति तनूकरोति सीता\-तापं\footnote{\textcolor{red}{श्यतीति शः} इत्यर्थः। \textcolor{red}{शो तनूकरणे} (धा॰पा॰~११४५)~\arrow शो~\arrow \textcolor{red}{अन्येष्वपि दृश्यते} (पा॰सू॰~३.२.१०१)~\arrow शो~ड~\arrow शो~अ~\arrow \textcolor{red}{डित्यभस्याप्यनु\-बन्धकरण\-सामर्थ्यात्} (वा॰~६.४.१४३)~\arrow श्~अ~\arrow श~\arrow विभक्तिकार्यम्~\arrow शः।} स्यति रावणं\footnote{\textcolor{red}{स्यतीति षः} इत्यर्थः। \textcolor{red}{षो अन्तकर्मणि} (धा॰पा॰~११४७)~\arrow षो~\arrow \textcolor{red}{धात्वादेः षः सः} (पा॰सू॰~६.१.६४)~\arrow सो~\arrow \textcolor{red}{अन्येष्वपि दृश्यते} (पा॰सू॰~३.२.१०१)~\arrow सो~ड~\arrow सो~अ~\arrow \textcolor{red}{डित्यभस्याप्यनु\-बन्धकरण\-सामर्थ्यात्} (वा॰~६.४.१४३)~\arrow स्~अ~\arrow स~\arrow सौत्रषकारः~\arrow ष~\arrow विभक्तिकार्यम्~\arrow षः।} सरत्ययोध्यां\footnote{\textcolor{red}{सरतीति सः} इत्यर्थः। \textcolor{red}{सृ गतौ} (धा॰पा॰~९३५)~\arrow सृ~\arrow \textcolor{red}{अन्येष्वपि दृश्यते} (पा॰सू॰~३.२.१०१)~\arrow सृ~ड~\arrow सृ~अ~\arrow \textcolor{red}{डित्यभस्याप्यनु\-बन्धकरण\-सामर्थ्यात्} (वा॰~६.४.१४३)~\arrow स्~अ~\arrow स~\arrow विभक्तिकार्यम्~\arrow सः।} रमयति सीतां रमते च स्वयं राजते वा राज\-सिंहासने\footnote{\textcolor{red}{रमयति रमते राजते वेति र्} इत्यर्थः। प्रथमपक्षे \textcolor{red}{रमुँ क्रीडायाम्} (धा॰पा॰~८५३) इत्यतो णिचि \textcolor{red}{रमि} धातुः। \textcolor{red}{जनी\-जॄष्क्नसु\-रञ्जोऽमन्ताश्च} (धा॰पा॰ ग॰सू॰) इत्यनेन मित्त्वादुपधा\-वृद्धौ \textcolor{red}{मितां ह्रस्वः} (पा॰सू॰~६.४.९२) इत्यनेन ह्रस्वः। द्वितीयपक्षे \textcolor{red}{रमुँ क्रीडायाम्} (धा॰पा॰~८५३) इति शुद्धो धातुः। तृतीयपक्षे \textcolor{red}{राजृँ दीप्तौ} (धा॰पा॰~८२२) इति धातुः। एतेभ्यः पूर्ववदौणादिको \textcolor{red}{ड्विन्‌}\-प्रत्ययः। प्रथमपक्षे \textcolor{red}{णेरनिटि} (पा॰सू॰~६.४.५१) इत्यनेन णिलोपः। पूर्ववत्सर्वापहारि\-लोपे टिलोपे विभक्ति\-कार्ये \textcolor{red}{हल्ङ्याब्भ्यो दीर्घात्सुतिस्यपृक्तं हल्} (पा॰सू॰~६.१.६८) इत्यनेन सुलोपे \textcolor{red}{र्}। \textcolor{red}{अयस्मयादीनि च्छन्दसि} (पा॰सू॰~१.४.२०) इत्यनेन छान्दस\-भसञ्ज्ञायां पदत्वाभावे \textcolor{red}{झलां जशोऽन्ते} (पा॰सू॰~८.२.३९) इत्यस्य प्रवृत्तिर्न।
} यः स \textcolor{red}{शषसर्}।\footnote{\textcolor{red}{शश्च षश्च सश्च र् चेति शषसर्}।} रावण\-वधं विधाय राज\-सिंहासनासीनो राम इति तात्पर्यम्। सौत्रत्वात्सकारस्य षकारः।\end{sloppypar}
\begin{sloppypar}\hyphenrules{nohyphenation}\justifying\noindent\hspace{10mm} \textcolor{red}{(१४) हल्}~– हरति भक्त\-तापं\footnote{\textcolor{red}{हरतीति हः} इत्यर्थः। \textcolor{red}{हृञ् हरणे} (धा॰पा॰~८९९)~\arrow हृ~\arrow \textcolor{red}{अन्येष्वपि दृश्यते} (पा॰सू॰~३.२.१०१)~\arrow हृ~ड~\arrow हृ~अ~\arrow \textcolor{red}{डित्यभस्याप्यनु\-बन्धकरण\-सामर्थ्यात्} (वा॰~६.४.१४३)~\arrow ह्~अ~\arrow ह~\arrow विभक्तिकार्यम्~\arrow हः।} लिङ्गति यः सीतां लिङ्ग्यते वा सीतया लीयते वा भक्तानां हृदि\footnote{\textcolor{red}{लिङ्गति लिङ्ग्यते लीयते वेति ल्} इत्यर्थः। प्रथम\-द्वितीय\-पक्षयोः \textcolor{red}{लिगिँ गतौ} (धा॰पा॰~१५५) इति धातुः। तृतीयपक्षे \textcolor{red}{लीङ् श्लेषणे} (धा॰पा॰~११३९) इति धातुः। प्रथम\-तृतीय\-पक्षयोः कर्तरि द्वितीय\-पक्षे च कर्मण्यौणादिको \textcolor{red}{ड्विन्‌}\-प्रत्ययः। \textcolor{red}{कार्याद्विद्यादनूबन्धम्} (भा॰पा॰सू॰~३.३.१) \textcolor{red}{केचिदविहिता अप्यूह्याः} (वै॰सि॰कौ॰~३१६९) इत्यनुसारमूह्योऽ\-यमविहित\-प्रत्ययः। सर्वापहारि\-लोपे \textcolor{red}{डित्यभस्याप्यनु\-बन्धकरण\-सामर्थ्यात्} (वा॰~६.४.१४३) इत्यनेन टिलोपे विभक्ति\-कार्ये \textcolor{red}{हल्ङ्याब्भ्यो दीर्घात्सुतिस्यपृक्तं हल्} (पा॰सू॰~६.१.६८) इत्यनेन सुलोपे \textcolor{red}{ल्}।} यः स \textcolor{red}{हल्}।\footnote{\textcolor{red}{हश्च ल् चेति हल्}।}\end{sloppypar}
\begin{sloppypar}\hyphenrules{nohyphenation}\justifying\noindent\hspace{10mm} इति बाल\-बुद्धि\-प्रतिपादित\-व्युत्पत्ति\-परक\-चतुर्दश\-सूत्री रामायण\-कथा।\end{sloppypar}
\begin{sloppypar}\hyphenrules{nohyphenation}\justifying\noindent\hspace{10mm} हकारेण प्रारभ्य हकारेणैवोपसंहारः। अपवर्गस्योभयत्र चर्चया चतुर्दश\-सूत्र\-रामायणमपवर्गायेति ध्वन्यते। एवमेव नव\-स्वरैर्नव\-सङ्ख्या\-वाच्य\-पूर्ण\-ब्रह्म\-श्रीरामस्य चर्चां विधाय पुनरेक\-त्रिंशद्वर्णै रावण\-वध\-काले भगवता धनुषि संहितानां रामस्यैक\-त्रिंशद्बाणानां\footnote{\textcolor{red}{आकरषेउ धनु श्रवण लगि छाँड़े शर एकतीस। रघुनायक सायक चले मानहुँ काल फनीस॥} (रा॰च॰मा॰~६.१०२)। एतद्रूपान्तरम्–\textcolor{red}{कर्णान्तमाकृष्य धनुः शरान् स त्रिंशन्मितानेकयुतान् व्यमुञ्चत्। शराश्चलन्ति स्म रघुप्रभोस्ते भुजङ्गमेशा ननु कालरूपाः॥} (मा॰भा॰~६.१०२)।} स्मरणेनैक\-त्रिंशद्विकार\-नाशोऽपि सूचितः। पुनर्द्वाभ्यां वर्णाभ्यां श्रीसीता\-राम\-स्मरणेन ज्ञानिनां मोक्षो भक्तानां भक्तिश्च सूचिता। व्यञ्जनेषु हकारेण प्रारभ्य हकारेणोपसंहारस्य तात्पर्यं यत् \textcolor{red}{हकारः} हरि\-वाचको हरिश्च पापानि हरत्यतः पूर्वम् \textcolor{red}{अइउण्} इत्यत्राकारेण वासुदेवस्य रामस्य स्मरणं\footnote{\textcolor{red}{अक्षराणामकारोऽस्मि} (भ॰गी॰~१०.३३)। \textcolor{red}{अकारो वासुदेवः स्यात्} (ए॰को॰~१)।} मध्ये हकारेण हरि\-स्मरणमन्ते च हकारेण हरि\-स्मरणमित्यादौ\footnote{\textcolor{red}{नामैकदेशग्रहणे नाममात्रग्रहणम्} इत्यनेन न्यायेन हकारो हरिवाचकः।} मध्येऽन्ते च मङ्गलाचरणं सङ्केत्य रामायणी मर्यादाऽपि सुरक्षिता। तथा चोक्तम्~–\end{sloppypar}
\centering\textcolor{red}{वेदे रामायणे चैव पुराणे भारते तथा।\nopagebreak\\
आदावन्ते च मध्ये च हरिः सर्वत्र गीयते॥}\nopagebreak\\
\raggedleft{–~क॰पु॰~२१.३७}\\
\begin{sloppypar}\hyphenrules{nohyphenation}\justifying\noindent अकारमुद्दिश्य श्रीराम\-स्मरणतः प्रारभ्यान्ते \textcolor{red}{हल्} इति लकारेण लक्ष्मणः स्मृतः।\footnote{\textcolor{red}{नामैकदेशग्रहणे नाममात्रग्रहणम्} इत्यनेन न्यायेन लकारो लक्ष्मणवाचकः।}\end{sloppypar}
\begin{sloppypar}\hyphenrules{nohyphenation}\justifying\noindent\hspace{10mm} यद्यपि चतुर्दश\-सूत्राणां द्विरुच्चारित\-हकारस्य सन्ति विविधेषु प्रयोगेषु प्रयोजनानि यथा \textcolor{red}{हयवरट्} इत्यत्र हकारः \textcolor{red}{अट्‌हश्‌अश्‌इण्‌}\-ग्रहणेषु सूत्रेषु हकार\-ग्रहणार्थः। तथा च \textcolor{red}{हयवरट्} इत्यत्र हकार\-ग्रहणाभावे तस्य च \textcolor{red}{अट्‌}\-प्रत्याहारे श्रवणाभावे \textcolor{red}{महाँ हि सः} इत्यत्र रुत्वानु\-नासिकत्वे न स्याताम्।\footnote{महान्~हि~सः~\arrow \textcolor{red}{दीर्घादटि समानपदे} (पा॰सू॰~८.३.९)~\arrow महारुँ~हि~सः~\arrow \textcolor{red}{आतोऽटि नित्यम्} (पा॰सू॰~८.३.३)~\arrow महाँरुँ~हि~सः~\arrow \textcolor{red}{भोभगोअघो\-अपूर्वस्य योऽशि} (पा॰सू॰~८.३.१७)~\arrow महाँय्~हि~सः~\arrow \textcolor{red}{लोपः शाकल्यस्य} (पा॰सू॰~८.३.१९)~\arrow महाँ~हि~सः।} \textcolor{red}{अर्हेण} इत्यत्र च णत्वं न स्यात्।\footnote{अर्ह~टा~\arrow \textcolor{red}{ टाङसिङसामिनात्स्याः} (पा॰सू॰~७.१.१२)~\arrow अर्ह~इन~\arrow \textcolor{red}{आद्गुणः} (पा॰सू॰~६.१.८७)~\arrow अर्हेन~\arrow \textcolor{red}{अट्कुप्वाङ्नुम्व्यवायेऽपि} (पा॰सू॰~८.४.२)~\arrow अर्हेण।} हयवरट्सूत्रे हकारो न स्यात्तदा \textcolor{red}{हश्‌}\-प्रत्याहार एव न स्यात्। \textcolor{red}{रामो हसति} इत्यत्र हकारस्य हश्परकत्वाभावे \textcolor{red}{हशि च} (पा॰सू॰~६.१.११४) इत्यनेनात उत्वं न स्यात्।\footnote{रामस्~हसति~\arrow \textcolor{red}{ससजुषो रुः} (पा॰सू॰~८.२.६६)~\arrow रामरुँ~हसति~\arrow \textcolor{red}{हशि च} (पा॰सू॰~६.१.११४)~\arrow राम~उ~हसति~\arrow \textcolor{red}{आद्गुणः} (पा॰सू॰~६.१.८७)~\arrow रामो~हसति।} एवमेव \textcolor{red}{भोभगोअघो\-अपूर्वस्य योऽशि} (पा॰सू॰~८.३.१७) इति सूत्रेण रोर्यत्वं न स्यात्।\footnote{अस्योदाहरणम्~– भोस्~हरे~\arrow \textcolor{red}{ससजुषो रुः} (पा॰सू॰~८.२.६६)~\arrow भोरुँ~हरे~\arrow भोभगोअघो\-अपूर्वस्य \textcolor{red}{योऽशि} (पा॰सू॰~८.३.१७)~\arrow भोय्~हरे~\arrow \textcolor{red}{हलि सर्वेषाम्} (पा॰सू॰~८.३.२२)~\arrow भो~हरे।} एवमेव \textcolor{red}{हयवरट्} इति सूत्रे हकार\-ग्रहणं विना \textcolor{red}{लिलिहिढ्वे} इत्यत्र \textcolor{red}{विभाषेटः} (पा॰सू॰~८.३.७९) इत्यनेनेण्लक्षणो ढकारो वैकल्पिको न स्यात्।\footnote{\textcolor{red}{लिहँ आस्वादने} (धा॰पा॰~१०१६)~\arrow लिह्~\arrow \textcolor{red}{स्वरितञितः कर्त्रभिप्राये क्रियाफले} (पा॰सू॰~१.३.७२)~\arrow \textcolor{red}{परोक्षे लिट्} (पा॰सू॰~३.२.११५)~\arrow लिह्~लिट्~\arrow लिह्~ध्वम्~\arrow \textcolor{red}{लिटि धातोरनभ्यासस्य} (पा॰सू॰~६.१.८)~\arrow लिह्~लिह्~ध्वम्~\arrow \textcolor{red}{हलादिः शेषः} (पा॰सू॰~७.४.६०)~\arrow लि~लिह्~ध्वम्~\arrow \textcolor{red}{आर्धधातुकस्येड्वलादेः} (पा॰सू॰~७.२.३५)~\arrow इट्प्राप्तिः~\arrow \textcolor{red}{एकाच उपदेशेऽनुदात्तात्‌} (पा॰सू॰~७.२.१०)~\arrow इण्निषेधः~\arrow \textcolor{red}{कृसृभृवृ\-स्तुद्रुस्रुश्रुवो लिटि} (पा॰सू॰~७.२.१३)~\arrow क्रादि\-नियमादिट्प्राप्तिः~\arrow \textcolor{red}{आद्यन्तौ टकितौ} (पा॰सू॰~१.१.४६)~\arrow लि~लिह्~इट्~ध्वम्~\arrow लि~लिह्~इ~ध्वम्~\arrow \textcolor{red}{विभाषेटः} (पा॰सू॰~८.३.७९)~\arrow वैकल्पिक\-ढत्वम्~\arrow लि~लिह्~इ~ढ्वम्~\arrow \textcolor{red}{टित आत्मनेपदानां टेरे} (पा॰सू॰~३.४.७९)~\arrow लि~लिह्~इ~ढ्वे~\arrow लिलिहिढ्वे। ढत्वाभावे~– लि~लिह्~इ~ध्वम्~\arrow \textcolor{red}{टित आत्मनेपदानां टेरे} (पा॰सू॰~३.४.७९)~\arrow लि~लिह्~इ~ध्वे~\arrow लिलिहिध्वे।} तस्माद्धयवरड्ढकारः सप्रयोजनकः। एवं \textcolor{red}{हल्} इत्यत्र हकार\-ग्रहणाभावे \textcolor{red}{रुदिहि}\footnote{\textcolor{red}{रुदिँर् अश्रुविमोचने} (धा॰पा॰~१०६७)~\arrow रुद्~\arrow \textcolor{red}{शेषात्कर्तरि परस्मैपदम्} (पा॰सू॰~१.३.७८)~\arrow \textcolor{red}{लोट् च} (पा॰सू॰~३.३.१६२)~\arrow रुद्~सिप्~\arrow \textcolor{red}{कर्तरि शप्‌} (पा॰सू॰~३.१.६८)~\arrow रुद्~शप्~सिप्~\arrow \textcolor{red}{अदिप्रभृतिभ्यः शपः} (पा॰सू॰~२.४.७२)~\arrow रुद्~सिप्~\arrow \textcolor{red}{सेर्ह्यपिच्च} (पा॰सू॰~३.४.८७)~\arrow रुद्~हि~\arrow \textcolor{red}{सार्वधातुकमपित्} (पा॰सू॰~१.२.४)~\arrow ङित्त्वम्~\arrow \textcolor{red}{ग्क्ङिति च} (पा॰सू॰~१.१.५)~\arrow लघूपध\-गुण\-निषेधः~\arrow \textcolor{red}{रुदादिभ्यः सार्वधातुके} (पा॰सू॰~७.२.७६)~\arrow \textcolor{red}{आद्यन्तौ टकितौ} (पा॰सू॰~१.१.४६)~\arrow रुद्~इट्~हि~\arrow रुद्~इ~हि~\arrow रुदिहि।} \textcolor{red}{स्वपिहि}\footnote{\textcolor{red}{ञिष्वपँ शये} (धा॰पा॰~१०६८)~\arrow ष्वप्~\arrow \textcolor{red}{धात्वादेः षः सः} (पा॰सू॰~६.१.६४)~\arrow स्वप्~\arrow \textcolor{red}{शेषात्कर्तरि परस्मैपदम्} (पा॰सू॰~१.३.७८)~\arrow \textcolor{red}{लोट् च} (पा॰सू॰~३.३.१६२)~\arrow स्वप्~सिप्~\arrow \textcolor{red}{कर्तरि शप्‌} (पा॰सू॰~३.१.६८)~\arrow स्वप्~शप्~सिप्~\arrow \textcolor{red}{अदिप्रभृतिभ्यः शपः} (पा॰सू॰~२.४.७२)~\arrow स्वप्~सिप्~\arrow \textcolor{red}{सेर्ह्यपिच्च} (पा॰सू॰~३.४.८७)~\arrow स्वप्~हि~\arrow \textcolor{red}{रुदादिभ्यः सार्वधातुके} (पा॰सू॰~७.२.७६)~\arrow \textcolor{red}{आद्यन्तौ टकितौ} (पा॰सू॰~१.१.४६)~\arrow स्वप्~इट्~हि~\arrow स्वप्~इ~हि~\arrow स्वपिहि।} \textcolor{red}{श्वसिहि}\footnote{\textcolor{red}{श्वसँ प्राणने} (धा॰पा॰~१०६९)~\arrow श्वस्~\arrow \textcolor{red}{शेषात्कर्तरि परस्मैपदम्} (पा॰सू॰~१.३.७८)~\arrow \textcolor{red}{लोट् च} (पा॰सू॰~३.३.१६२)~\arrow श्वस्~सिप्~\arrow \textcolor{red}{कर्तरि शप्‌} (पा॰सू॰~३.१.६८)~\arrow श्वस्~शप्~सिप्~\arrow \textcolor{red}{अदिप्रभृतिभ्यः शपः} (पा॰सू॰~२.४.७२)~\arrow श्वस्~सिप्~\arrow \textcolor{red}{सेर्ह्यपिच्च} (पा॰सू॰~३.४.८७)~\arrow श्वस्~हि~\arrow \textcolor{red}{रुदादिभ्यः सार्वधातुके} (पा॰सू॰~७.२.७६)~\arrow \textcolor{red}{आद्यन्तौ टकितौ} (पा॰सू॰~१.१.४६)~\arrow श्वस्~इट्~हि~\arrow श्वस्~इ~हि~\arrow श्वसिहि।} इत्यादौ \textcolor{red}{रुदादिभ्यः सार्वधातुके} (पा॰सू॰~७.२.७६) इत्यनेन हकारस्य वल्त्वाभावेन वलादि\-लक्षण इण्न स्यात्। एवं हल्सूत्रे हकार\-पाठं विना \textcolor{red}{स्नेहित्वा} \textcolor{red}{स्निहित्वा} इत्यत्र \textcolor{red}{रलो व्युपधाद्धलादेः संश्च} (पा॰सू॰~१.२.२६) इत्यनेन वैकल्पिकं कित्त्वं न स्यात्।\footnote{\textcolor{red}{ष्णिहँ प्रीतौ} (धा॰पा॰~१२००)~\arrow ष्णिह्~\arrow \textcolor{red}{धात्वादेः षः सः} (पा॰सू॰~६.१.६४)~\arrow निमित्तापाये नैमित्तिकस्याप्यपायः~\arrow स्निह्~\arrow \textcolor{red}{समान\-कर्तृकयोः पूर्वकाले} (पा॰सू॰~३.४.२१)~\arrow स्निह्~क्त्वा~\arrow स्निह्~त्वा~\arrow \textcolor{red}{रधादिभ्यश्च} (पा॰सू॰~७.२.४५)~\arrow वैकल्पिक इट्~\arrow स्निह्~इट्~त्वा~\arrow स्निह्~इ~त्वा~\arrow \textcolor{red}{रलो व्युपधाद्धलादेः संश्च} (पा॰सू॰~१.२.२६)~\arrow वैकल्पिकं कित्त्वम्। कित्त्वपक्षे – स्निह्~इ~त्वा~\arrow \textcolor{red}{ग्क्ङिति च} (पा॰सू॰~१.१.५)~\arrow लघूपध\-गुण\-निषेधः~\arrow स्निहित्वा। अकित्त्वपक्षे – स्निह्~इ~त्वा~\arrow \textcolor{red}{पुगन्त\-लघूपधस्य च} (पा॰सू॰~७.३.८६)~\arrow स्नेह्~इ~त्वा~\arrow स्नेहित्वा। इडभावपक्षे च~– स्निह्~त्वा~\arrow \textcolor{red}{वा द्रुहमुह\-ष्णुहष्णिहाम्} (पा॰सू॰~८.२.३३)~\arrow वैकल्पिकं घत्वम्। घत्वपक्षे~– स्निघ्~त्वा~\arrow \textcolor{red}{झषस्तथोर्धोऽधः} (पा॰सू॰~८.२.४०)~\arrow स्निघ्~ध्वा~\arrow \textcolor{red}{झलां जश् झशि} (पा॰सू॰~८.४.५३)~\arrow स्निग्~ध्वा~\arrow स्निग्ध्वा। घत्वाभावे~– स्निह्~त्वा~\arrow \textcolor{red}{हो ढः} (पा॰सू॰~८.२.३१)~\arrow स्निढ्~त्वा~\arrow \textcolor{red}{झषस्तथोर्धोऽधः} (पा॰सू॰~८.२.४०)~\arrow स्निढ्~ध्वा~\arrow \textcolor{red}{ष्टुना ष्टुः} (पा॰सू॰~८.४.४१)~\arrow स्निढ्~ढ्वा~\arrow \textcolor{red}{ढो ढे लोपः} (पा॰सू॰~८.३.१३)~\arrow स्नि~ढ्वा~\arrow \textcolor{red}{ढ्रलोपे पूर्वस्य दीर्घोऽणः} (पा॰सू॰~६.३.१११)~\arrow स्नी~ढ्वा~\arrow स्नीढ्वा।} एवं हल्सूत्रीय\-हकारमन्तरेण शलित्यस्य पाठाभावात् \textcolor{red}{अलिक्षत्} इत्यत्र \textcolor{red}{शल इगुपधादनिटः क्सः} (पा॰सू॰~३.१.४५) इत्यनेन क्सो न स्यात्।\footnote{\textcolor{red}{लिहँ आस्वादने} (धा॰पा॰~१०१६)~\arrow लिह्~\arrow \textcolor{red}{शेषात्कर्तरि परस्मैपदम्} (पा॰सू॰~१.३.७८)~\arrow \textcolor{red}{लुङ्} (पा॰सू॰~३.२.११०)~\arrow लिह्~लङ्~\arrow लिह्~तिप्~\arrow लिह्~ति~\arrow \textcolor{red}{लुङ्लङ्लृङ्क्ष्वडुदात्तः} (पा॰सू॰~६.४.७१)~\arrow \textcolor{red}{आद्यन्तौ टकितौ} (पा॰सू॰~१.१.४६)~\arrow अट्~लिह्~ति~\arrow अ~लिह्~ति~\arrow \textcolor{red}{च्लि लुङि} (पा॰सू॰~३.१.४३)~\arrow अ~लिह्~च्लि~ति~\arrow \textcolor{red}{शल इगुपधादनिटः क्सः} (पा॰सू॰~३.१.४५)~\arrow अ~लिह्~क्स~ति~\arrow अ~लिह्~स~ति~\arrow \textcolor{red}{हो ढः} (पा॰सू॰~८.२.३१)~\arrow अ~लिढ्~क्स~ति~\arrow \textcolor{red}{षढोः कः सि} (पा॰सू॰~८.२.४१)~\arrow अ~लिक्~स~ति~\arrow \textcolor{red}{आदेश\-प्रत्यययोः} (पा॰सू॰~८.३.५९)~\arrow अ~लिक्~ष~ति~\arrow \textcolor{red}{इतश्च} (पा॰सू॰~३.४.१००)~\arrow अ~लिक्~ष~त्~\arrow अलिक्षत्।} एवमेव हल्सूत्रस्थ\-हकारमन्तरा झल्प्रत्याहारे तस्य ग्रहणाभावे \textcolor{red}{अदाग्धाम्} इत्यत्र \textcolor{red}{अ~दाह्~स~ताम्} इति स्थिते \textcolor{red}{झलो झलि} (पा॰सू॰~८.२.२६) इत्यनेन सकार\-लोपो न स्यात्।\footnote{\textcolor{red}{दहँ भस्मीकरणे} (धा॰पा॰~९९१)~\arrow दह्~\arrow \textcolor{red}{शेषात्कर्तरि परस्मैपदम्} (पा॰सू॰~१.३.७८)~\arrow \textcolor{red}{लुङ्} (पा॰सू॰~३.२.११०)~\arrow दह्~लङ्~\arrow दह्~तस्~\arrow \textcolor{red}{लुङ्लङ्लृङ्क्ष्वडुदात्तः} (पा॰सू॰~६.४.७१)~\arrow \textcolor{red}{आद्यन्तौ टकितौ} (पा॰सू॰~१.१.४६)~\arrow अट्~दह्~तस्~\arrow अ~दह्~तस्~\arrow \textcolor{red}{च्लि लुङि} (पा॰सू॰~३.१.४३)~\arrow अ~दह्~च्लि~तस्~\arrow \textcolor{red}{च्लेः सिच्} (पा॰सू॰~३.१.४४)~\arrow अ~दह्~सिच्~तस्~\arrow अ~दह्~स्~तस्~\arrow \textcolor{red}{सिचि वृद्धिः परस्मैपदेषु} (पा॰सू॰~७.२.१)~\arrow अ~दाह्~स्~तस्~\arrow \textcolor{red}{तस्थस्थमिपां तान्तन्तामः} (पा॰सू॰~३.४.१०१)~\arrow अ~दाह्~स्~ताम्~\arrow \textcolor{red}{झलो झलि} (पा॰सू॰~८.२.२६)~\arrow अ~दाह्~ताम्~\arrow \textcolor{red}{दादेर्धातोर्घः} (पा॰सू॰~८.२.३२)~\arrow अ~दाघ्~ताम्~\arrow \textcolor{red}{झषस्तथोर्धोऽधः} (पा॰सू॰~८.२.४०)~\arrow अदाघ्~धाम्~\arrow \textcolor{red}{झलां जश् झशि} (पा॰सू॰~८.४.५३)~\arrow अदाग्~धाम्~\arrow अदाग्धाम्।} तस्माद्धल्सूत्रेऽपि हकारः सार्थक एवेति प्रौढमनोरमादौ प्रपञ्चितम्। एवं सम्भवति हकारोच्चारण\-द्वय\-प्रयोजनेऽपि लौकिके मम दृष्टौ हकारं द्विरुच्चार्य तस्य हरि\-शब्दाद्यक्षरतया चतुर्दश\-सूत्री हरि\-स्मरणेन सम्पुटिता। इदमलौकिकं प्रयोजनमिति मे प्रतिभाति।\end{sloppypar}
\begin{sloppypar}\hyphenrules{nohyphenation}\justifying\noindent\hspace{10mm} एवं राघवेण चतुर्दश\-वर्षीय\-वन\-वास\-काल एव सम्पूर्णा लीला कृता। यद्यपि चतुर्दश\-सूत्र्यां सङ्ख्या\-ग्रहे प्रयोजनं सम्भवति शम्भोः। अन्यथा त्रयोदशैव सूत्राणि क्रियेरन् किं जातं चतुर्दशेन सूत्रेण हलादि\-प्रत्याहारा रान्ताः करणीया अन्योऽन्याश्रय\-दोष\-वारणाय रन्त्यं \textcolor{red}{हर्} इति न्यासः करणीयः। तत्रत्येयं परिस्थितिः \textcolor{red}{हलन्त्यम्} (पा॰सू॰~१.३.३) इति सूत्रे। \textcolor{red}{वाक्यार्थ\-बोधे पदार्थ\-ज्ञानं कारणम्}। यथा \textcolor{red}{रामो गच्छति} इत्यत्र वाक्यार्थ\-ज्ञाने पदयोर्द्वयोः पृथग्बोधः करणीयोऽनन्तरं वाक्यार्थ\-बोधो भविष्यति। तथैवात्रापि। \textcolor{red}{उपदेशेऽजनुनासिक इत्} (पा॰सू॰~१.३.२) इत्यस्मात् \textcolor{red}{उपदेशे इत्} इदं पद\-द्वयमनुवर्तते। तथा \textcolor{red}{उपदेशे अन्त्यं हल् इत् स्यात्} इत्यर्थे सम्पन्ने हल्पदस्यार्थ\-ज्ञाने कर्तव्ये हल्सञ्ज्ञा\-विधायक\-सूत्र\-वाक्यार्थ\-बोधः कर्तव्यः। तत्र च पदार्थ\-ज्ञानमावश्यकम्। हल्सञ्ज्ञा\-सूत्रं \textcolor{red}{आदिरन्त्येन सहेता} (पा॰सू॰~१.१.७१) इत्येवम्। \textcolor{red}{अन्त्येनेता सहित आदिर्मध्यगानां स्वस्य च सञ्ज्ञा स्यात्} (ल॰सि॰कौ॰~४)। एवमत्रापीत्पदार्थ\-ज्ञानं जिज्ञासितम्। तच्चेत्सञ्ज्ञा\-विधायक\-सूत्रे \textcolor{red}{हलन्त्यम्} (पा॰सू॰~१.३.३) इत्यस्मिन्नधीनम्। तत्रापि वाक्यार्थ\-ज्ञानाय पदार्थ\-ज्ञानम्। इत्थम् \textcolor{red}{इत्‌}\-पदार्थ\-ज्ञानं \textcolor{red}{हल्‌}\-पदार्थ\-ज्ञानाधीनं \textcolor{red}{हल्‌}\-पदार्थज्ञानञ्च \textcolor{red}{इत्‌}\-पदार्थ\-ज्ञानाधीनम्। अयमेवान्योऽन्याश्रयः। \textcolor{red}{अन्योऽन्याश्रयत्वं नाम परस्परापेक्षित्वम्}। \textcolor{red}{तद्ग्रहसापेक्ष\-ग्रह\-सापेक्ष\-ग्रह\-विषयत्वमन्योऽन्याश्रयत्वम्}। तद्यथा तद्ग्रह इद्ग्रहस्तत्सापेक्ष\-ग्रहो हल्ग्रहस्तत्सापेक्ष\-ग्रह इद्ग्रहस्तद्विषयत्वमन्योऽ\-न्याश्रयत्वम्। अन्योऽन्याश्रयाणि कार्याणि न प्रकल्पन्ते। यथा नावि बद्धा नौर्न प्रचलति।\footnote{\textcolor{red}{इतरेतराश्रयाणि च कार्याणि न प्रकल्पन्ते। तद्यथा नौर्नावि बद्धा नेतरत्राणाय भवति} (भा॰पा॰सू॰~१.१.१)।} इमं दोषमाशङ्क्य भट्टोजिदीक्षितो हलन्त्यमिति सूत्रस्याऽवृत्तिं कृतवान्। ध्यातव्यमष्टाध्यायीस्थम् \textcolor{red}{हलन्त्यम्} इति सूत्रमसमस्तं किन्त्वावृत्तम् \textcolor{red}{हलन्त्यम्} इति सूत्रं समस्तम्। सप्तमी\-समासे शौण्डादिगण एतस्याभावाद्योगविभागस्य चाप्रमाणिकत्वात्सुप्सुपा समासस्यागतिक\-गतित्वादत्र षष्ठी\-तत्पुरुषोऽवयवावयवि\-भाव\-रूपः सम्बन्धः। हलो हल्सूत्रस्यावयवि\-भूतमन्त्यमिति। इत्थमेकत्रेत्पदार्थ\-ज्ञाने \textcolor{red}{आदिरन्त्येन सहेता} (पा॰सू॰~१.१.७१) इत्यनेन हल्पदार्थोऽपि विज्ञाप्यते हल्सूत्रेण चेत्पदार्थः। तत्रैव रन्त्यं \textcolor{red}{हर्} इति न्यासः करणीयो \textcolor{red}{हरन्त्यम्} इति वाऽऽवर्तनीयम्। हकारस्य च प्रयोजनान्युक्तानि। अतो \textcolor{red}{ल्}\-शब्दस्य दृष्ट\-प्रयोजनाभावेऽ\-दृष्टार्थं प्रयोजनं रामायण\-कथाछलेन लक्ष्मण\-स्मरणार्थं च। यद्यपि हकारः \textcolor{red}{शषसहर्} इति पठनीयो \textcolor{red}{हरिर्हसति} इत्यत्र विसर्गापत्तिः।\footnote{\textcolor{red}{शषसहर्} इति पाठे कृते हकारस्य \textcolor{red}{खर्} इत्यत्र ग्रहणात् \textcolor{red}{हरिस्~हसति} इति स्थिते \textcolor{red}{ससजुषो रुः} (पा॰सू॰~८.२.६६) इत्यनेन रुत्वे \textcolor{red}{हरिरुँ~हसति} इति जातेऽनुबन्ध\-लोपे \textcolor{red}{हरिर्~हसति} इति जाते \textcolor{red}{खरवसानयोर्विसर्जनीयः} (पा॰सू॰~८.३.१५) इत्यनेन \textcolor{red}{हरिः~हसति} इत्यनिष्ट\-रूपं स्यादिति भावः। \textcolor{red}{शषसर्} पाठे \textcolor{red}{खरवसानयोर्विसर्जनीयः} (पा॰सू॰~८.३.१५) इत्यस्याप्रवृत्तौ \textcolor{red}{हरिर्~हसति} इत्यतो वर्णसम्मेलने \textcolor{red}{हरिर्हसति} इति।} \textcolor{red}{प्रेष्य\-ब्रुवोर्हविषो देवतासम्प्रदाने} (पा॰सू॰~२.३.६१) इति
ज्ञापनेन यद्यपि दोषो दूरीकर्तुं शक्यते। अतो निरर्थकं हल्। \textcolor{red}{नासूया कर्तव्या यत्रानुगमः क्रियते सूत्रकारैः} (भा॰पा॰सू॰~५.१.५९) इति वचनेनालमसूयया। चतुर्दश\-सूत्र्यां रामायणी चर्चा। रामायणे चानसूयाया वर्णनम्। असूया नहि सूत्रे।
दृष्टभाजनमेव न फलमदृष्टमपि। हकारोच्चारणेन हरि\-स्मरणं सीता\-सहितस्य श्रीरामस्य प्रतिपाद्यत्वात्। तत्र व्यञ्जन\-रूपेण सीताऽकार\-रूपेण च रामो लकारस्यार्धत्वाल्लकारेणापूर्ण\-जीवाचार्यस्य लक्ष्मणस्य स्मरणम्। अतो बाह्यान्तरकरणानां चतुर्दशानां\footnote{चक्षुः\-श्रोत्र\-रसना\-घ्राण\-त्वगाख्यानां पञ्च\-ज्ञानेन्द्रियाणां पाणि\-पाद\-पायूपस्थ\-वागाख्यानां पञ्च\-कर्मेन्द्रियाणां मनो\-बुद्धि\-चिताहङ्काराख्यानां चतुरन्तः\-करणानां च।} पावन्याः चतुर्दश\-भुवन\-ख्यात\-राम\-कथायाः सूत्रतः सङ्केतितत्वाच्चतुर्दशत्वेऽतिविशेष आग्रहो महादेवस्येति प्रतीयते। इत्थं सम्पूर्णमपि व्याकरणं पाणिनीयं राम\-कथा\-परम्। अध्यात्म\-रामायणञ्च शिव\-प्रोक्तम्। अतः पाणिनि\-व्याकरणाध्यात्म\-रामायणयोरेकस्य शिवस्य वक्तृत्वाद्द्वयोः समीक्षायां प्रेरिता स्वयमेव भगवता रामचन्द्रेणाज्ञान\-विक्लवाऽनधीत\-शास्त्रा क्षपित\-चक्षुषो मे बाल\-मनीषा। यद्यप्येका सूक्तिर्यत्~–\end{sloppypar}
\centering\textcolor{red}{यान्युज्जहार माहेशाद्व्यासो व्याकरणार्णवात्।\nopagebreak\\
तानि किं पदरत्नानि मान्ति पाणिनिगोष्पदे॥}\footnote{मूलं मृग्यम्। शब्दकल्पद्रुम\-काराः \textcolor{red}{माहेशम्} इति शब्दे श्लोकमिमुद्धृत्य \textcolor{red}{इत्युद्भटः} इत्यूचुः।}\nopagebreak\\
\begin{sloppypar}\hyphenrules{nohyphenation}\justifying\noindent अर्थाच्छाङ्कर\-व्याकरण\-समुद्रस्य सम्पूर्णोऽपि विषयः पाणिनीये व्याकरणे कथं समाहर्तुं शक्यः। किन्त्वेतत्कथनं केवलं बौद्धिक\-विचारतः पलायन\-वाद\-मात्रम्। शिव एव पाणिनि\-हृदय\-स्थ इदं व्याकरणमस्मदर्थं व्याचकार। अतः सर्वमपि वाल्मीकि\-व्यास\-तुलसीदास\-प्रमुख\-शिष्ट\-प्रयुक्तं पाणिनि\-व्याकरण\-सम्मतं कर्तुं शक्यमिति गुरु\-सेवा\-लब्ध\-बुद्धि\-बलो गिरिधरः साटोपं घोषयति।\end{sloppypar}
\begin{sloppypar}\hyphenrules{nohyphenation}\justifying\noindent\hspace{10mm} अथ वेदना\-तान्त\-क्रौञ्च\-द्वन्द्व\-वियोगोत्थ\-शोको श्लोकत्वमागतो वाल्मीकेः।\footnote{\textcolor{red}{काव्यस्यात्मा स एवार्थस्तथा चादिकवेः पुरा। क्रौञ्चद्वन्द्ववियोगोत्थः शोकः श्लोकत्वमागतः॥} (ध्व॰~१.५)।} तद्वै ब्रह्मणाऽऽदिष्टः श्रीमद्रामायणं प्रोवाच। तच्च विपुलं चतुर्विंशति\-सहस्र\-श्लोकात्मकम्। सा च चतुर्विंशत्साहस्री चतुर्विंशति\-साहस्री संहितेति कथ्यते चतुर्विंशत्यक्षरात्मक\-गायत्री\-भाष्य\-भूता। गायत्री च सीता। वाल्मीकीयं रामायणं सीता\-चरित\-प्रधानं स्वयमेव वाल्मीकिः प्रतिजानीते यथा~–\end{sloppypar}
\centering\textcolor{red}{काव्यं रामायणं कृत्स्नं सीतायाश्चरितं महत्।\nopagebreak\\
पौलस्त्यवधमित्येवं चकार चरितव्रतः॥}\nopagebreak\\
\raggedleft{–~वा॰रा॰~१.४.७}\\
\begin{sloppypar}\hyphenrules{nohyphenation}\justifying\noindent ततः शत\-कोटि\-पर्यन्तानि रामायणानि भाषितानि वाल्मीकिना। तत्र प्रथमस्याऽदि\-काव्यस्य वक्ता स्वयं शेषाणाञ्च वक्तारं शशाङ्क\-शेखरं शिवं श्रोत्रीं च भगवतीं भवानीं समकल्पयत्। अस्य रामायणे माधुर्यं प्रधानमैश्वर्यमत्यल्पम्। यद्यप्ययमेव श्रीरामस्य भगवत्त्वं सर्व\-प्रथमं प्रतिजानीते यथा~–\end{sloppypar}
\centering\textcolor{red}{प्रोद्यमाने जगन्नाथं सर्वलोकनमस्कृतम्।\nopagebreak\\
कौसल्याऽजनयद्रामं दिव्यलक्षणसंयुतम्॥}\nopagebreak\\
\raggedleft{–~वा॰रा॰~१.१८.१०}\\
\begin{sloppypar}\hyphenrules{nohyphenation}\justifying\noindent इति। एवमयोध्या\-काण्डेऽपि~–\end{sloppypar}
\centering\textcolor{red}{स हि देवैरुदीर्णस्य रावणस्य वधार्थिभिः।\nopagebreak\\
अर्थितो मानुषे लोके जज्ञे विष्णुः सनातनः॥}\nopagebreak\\
\raggedleft{–~वा॰रा॰~२.१.७}\\
\begin{sloppypar}\hyphenrules{nohyphenation}\justifying\noindent एवं क्वचित्क्वचित्प्रति\-काण्डम्। इमे वैदिका ऋषयः। अतो वेद\-प्रतिपाद्य\-ब्रह्म\-रामस्यैव चर्चामकार्षुः। वेदे राम\-कथायाः कुत्र चर्चा कथं वा राम\-कथाया वेद\-मूलकता इति चेत्। \textcolor{red}{मन्त्र\-ब्राह्मणयोर्वेदनामधेयम्} (आ॰श्रौ॰सू॰~२४.१.३१) इत्यापस्तम्ब\-सूत्रानुसारं मन्त्र\-ब्राह्मणात्मको वेदः। मन्त्र\-भागे राम\-कथा नीलकण्ठाचार्य\-सङ्कलित\-मन्त्रात्मिका। तत् \textcolor{red}{मन्त्र\-रामायणम्} इति कथ्यते। ब्राह्मण\-भागे च प्रायश उपनिषदो यासु रामोत्तर\-तापनीयोप\-निषत्सीतोप\-निषदित्यादयः। इत्थं माधुर्य\-गुण\-बृंहित\-वाल्मीकि\-रामायणे बहुत्र स्थलेषु नरमनुकुर्वतो भगवतो रामस्य चरित्रं दृष्ट्वा श्रुत्वा वा सामान्य\-जनानां सन्देहः। यथा स्वयमेव दाक्षायणी भवानी सीता\-विरह\-विधुर\-हृदयं पामरमिव पृच्छन्तं लता\-काननानि श्रीरघुनन्दनं दृष्ट्वा मुमोहेति। जगती\-तल\-समुद्धार\-चिकीर्षया रामोपासना\-सुधया सुधारयितुं वसुधा\-तलं कैलास\-वट\-वृक्ष\-तले निषण्णो भवानी\-विहित\-विविध\-श्रीराम\-विषयक\-प्रश्न\-समाकर्णन\-प्रसन्नोऽपि शम्भुर्निज\-हृदय\-पुण्य\-वसुमती\-तले प्रवहन्तीं भक्ति\-वेदान्त\-सिद्धान्त\-पावन\-पूर\-पूर्णां कृत\-कलि\-पाप\-पाषाण\-चूर्णां श्रीराम\-सागर\-सङ्गमां निखिल\-लोक\-मनोरमां निहित\-सिद्धान्त\-तरल\-तरङ्गां समलङ्कृत\-वैष्णव\-हृदय\-रङ्गां विहित\-भगवल्लीला\-समागत\-सन्देह\-कश्मल\-भङ्गामुपनिषत्सिद्धान्त\-श्रुति\-प्रामाण्य\-सङ्गामध्यात्म\-रामायण\-गङ्गां पञ्चभ्योऽपि वक्त्रेभ्यः प्रवाहयामास। भगवान् शिवोऽनादिरत एतद्रचनाऽप्यनादिरेव। शङ्करो भगवान् देश\-काल\-परिस्थिति\-परिच्छेद\-रहितोऽतस्तत्कृतिरपि तथैव। श्रीराम\-कथा\-वर्णन\-प्रसङ्गेन भगवता शिवेन निगूढ\-वेदान्त\-रहस्यानां यादृक्सरलतया लालित्य\-पूर्णं प्रतिपादनमकार्येतत्कर्म तस्मिन्नेव सङ्घटते सच्चिदानन्द\-शान्ति\-निलये शिवे। अस्मिन् परम\-प्रत्ने विहित\-रामोपासना\-भावना\-मण्डन\-यत्ने ग्रन्थ\-रत्ने नवानां रसानामनुपमा छटा समालोक्यते। साहित्य\-शास्त्रानुसारं प्रत्येक\-महा\-काव्ये सप्ताधिकसर्गा अपेक्षन्ते। अत्र च चतुःषष्टि\-सर्गाः सन्ति। प्रति\-सर्गं छन्दः\-परिवर्तनं भवति यथा वाल्मीकीय\-रामायणे रघुवंशादौ च। वाल्मीकि\-रामायणस्य बाल\-काण्डस्य प्रथम\-सर्गे नव\-नवतिं यावच्छ्लोका अनुष्टुप एवं शततमः श्लोक उपजातिः। यथा~–\end{sloppypar}
\centering\textcolor{red}{पठन्द्विजो वागृषभत्वमीयात्स्यात्क्षत्रियो भूमिपतित्वमीयात्।\nopagebreak\\
वणिग्जनः पण्यफलत्वमीयाज्जनश्च शूद्रोऽपि महत्त्वमीयात्॥}\nopagebreak\\
\raggedleft{–~वा॰रा॰~१.१.१००}\\
\begin{sloppypar}\hyphenrules{nohyphenation}\justifying\noindent एवमेव रघुवंशे~–\end{sloppypar}
\centering\textcolor{red}{अथ नयनसमुत्थं ज्योतिरत्रेरिव द्यौः\nopagebreak\\
सुरसरिदिव तेजो वह्निनिष्ठ्यूतमैशम्।\nopagebreak\\
नरपतिकुलभूत्यै गर्भमाधत्त राज्ञी\nopagebreak\\
गुरुभिरभिनिविष्टं लोकपालानुभावैः॥}\nopagebreak\\
\raggedleft{–~र॰वं॰~२.७५}\footnote{रघुवंशस्य द्वितीयसर्गे चतुःसप्ततिं यावत्पद्यानीन्द्र\-वज्रोप\-जातिच्छन्दोबद्धानि पञ्चसप्ततितमं पद्यं च मालिनी\-वृत्त\-बद्धम्।}\\
\begin{sloppypar}\hyphenrules{nohyphenation}\justifying\noindent तथैवात्रापि प्रायः सर्गान्ते छन्दसः परिवर्तनम्। यथा~–\end{sloppypar}
\centering\textcolor{blue}{देवाश्च सर्वे हरिरूपधारिणः स्थिताः सहायार्थमितस्ततो हरेः।\nopagebreak\\
महाबलाः पर्वतवृक्षयोधिनः प्रतीक्षमाणा भगवन्तमीश्वरम्॥}\nopagebreak\\
\raggedleft{–~अ॰रा॰~१.२.३२}\\
\centering\textcolor{blue}{एवं परात्मा मनुजावतारो मनुष्यलोकाननुसृत्य सर्वम्।\nopagebreak\\
चक्रेऽविकारी परिणामहीनो विचार्यमाणे न करोति किञ्चित्॥}\nopagebreak\\
\raggedleft{–~अ॰रा॰~१.३.६६}\\
\begin{sloppypar}\hyphenrules{nohyphenation}\justifying\noindent एवमन्यत्रापि। महा\-काव्यस्य नियमानुसारेण प्रत्येकस्मिन्महाकाव्ये शृङ्गार\-शान्त\-वीर\-करुणेष्वन्यतमो रसः प्रधानोऽङ्गी वा शेषा अष्टौ रसा गौणा अङ्गभूताश्च। एवमस्मिन्नपि महा\-काव्ये शान्तो रसः प्रधानः। ज्ञातव्यमत्र वाल्मीकीय\-रामायणं करुण\-रस\-प्रधानं क्रौञ्च\-द्वन्द्व\-वियोग\-संवीक्षण\-सञ्जात\-करुणस्य महर्षेः काव्य\-सृष्टौ प्रवृत्तत्वात्। रघुवंश\-महा\-काव्यस्य चतुर्दशे सर्गे कविता\-कामिनी\-विलासः कवि\-कुल\-गुरुः कालिदासः सीता\-निर्वासन\-करुणां तरुणयन् गायति यथा~–\end{sloppypar}
\centering\textcolor{red}{तामभ्यगच्छद्रुदितानुसारी मुनिः कुशेध्माहरणाय यातः।\nopagebreak\\
निषादविद्धाण्डजदर्शनोत्थः श्लोकत्वमापद्यत यस्य शोकः॥}\nopagebreak\\
\raggedleft{–~र॰वं॰~१४.७०}\\
\begin{sloppypar}\hyphenrules{nohyphenation}\justifying\noindent एवं श्रीतुलसीदास\-कृतं श्रीराम\-चरित\-मानसं वीर\-रस\-प्रधानं रघुवंश\-महा\-काव्यं शृङ्गार\-रस\-प्रधानं तथैवाध्यात्म\-रामायणमिदं शान्त\-रस\-प्रधानम्। अस्य वक्ता स्वयं प्रशान्ति\-निलयः शिवः। स च काम\-शत्रुः। अत एव तुलसीदासोऽपि श्रीमानस इमं धृतशरीरं शान्त\-रसमिव प्रस्तौति। यथा~–\end{sloppypar}
\centering\textcolor{red}{बैठे सोह कामरिपु कैसे। धरे शरीर शांत रस जैसे॥}\footnote{एतद्रूपान्तरम्–\textcolor{red}{तत्रोपविष्टः शुशुभे कामारिः कीदृशस्तदा। शरीरधारी प्रत्यक्षं नूनं शान्तरसो यथा॥} (मा॰भा॰~१.१०७.१)।}\nopagebreak\\
\raggedleft{–~रा॰च॰मा॰~१.१०७.१}\\
\begin{sloppypar}\hyphenrules{nohyphenation}\justifying\noindent \textcolor{red}{फल\-भोक्ता तु नायकः} इति सिद्धान्तानुसारं काव्यस्य फलं नायक एव भुङ्क्ते। अतो नायक\-वृत्तानु\-सारमपि रस\-प्राधान्यं निर्णेतुं शक्यते। नायकः श्रीरामस्तथाऽत्राध्यात्म\-तत्त्वात्परम\-शान्ति\-निलयः शान्तस्तस्माच्छान्त\-रस\-प्रधानमिदम्। आद्यन्तघटनाभ्यामपि रसो निर्णीयते। एतस्याऽद्या घटना शिव\-पार्वती\-सम्बद्धाऽन्तिमाऽपि घटना भगवदन्तर्धानरूपोभे च शान्त\-प्रधाने तस्माच्छान्त\-रसः प्रधानः। अत्र स्थले स्थले सहस्रश उपनिषदां वेदान्त\-गूढ\-तत्त्वानां वैराग्य\-प्रतिपादक\-वाक्यानां प्रतिपादनं प्राचुर्येण मिलति तथाऽप्यध्यात्म\-चर्चा\-विषयत्वात्सुतरामिदमाध्यात्मिकीं पिपासां शमयितुं क्षमम्।\end{sloppypar}
\begin{sloppypar}\hyphenrules{nohyphenation}\justifying\noindent\hspace{10mm} चत्वारो हि नायका धीरोदात्तो धीरोद्धत्तो धीरललितो धीरप्रशान्तश्चेति।\footnote{\textcolor{red}{त्यागी कृती कुलीनः सुश्रीको रूपयौवनोत्साही। दक्षोऽनुरक्तलोकस्तेजोवैदग्ध्यशीलवान् नेता॥ धीरोदात्तो धीरोद्धतस्तथा धीरललितश्च। धीरप्रशान्त इत्ययमुक्तः प्रथमश्चतुर्भेदः॥} (सा॰द॰~३.३०-३१)।} तत्र धीरोदात्तः कुलीनः शान्तो दान्तश्चरित्र\-सद्गुण\-सम्पन्नस्तेजस्वी। स च पराक्रम\-मेधादीनां यष्टा।\footnote{\textcolor{red}{अविकत्थनः क्षमावानतिगम्भीरो महासत्त्वः। स्थेयान् निगूढमानो धीरोदात्तो दृढव्रतः कथितः॥} (सा॰द॰~३.३२)।} मर्यादा\-पुरुषोत्तमः पर\-ब्रह्म श्रीरामो यथा वाल्मीकीय\-रामायणे तुलसी\-कृते च प्रायश आत्म\-श्लाघा\-रहितः। स एवात्र धीरोदात्तः। भीमसेनादिर्धीरोद्धत्तः।\footnote{\textcolor{red}{मायापरः प्रचण्डश्चपलोऽहङ्कारदर्पभूयिष्ठः। आत्मश्लाघानिरतो धीरैर्धीरोद्धतः कथितः॥} (सा॰द॰~३.३३)।} धीरललितः कृष्णो जयदेवस्य।\footnote{\textcolor{red}{निश्चिन्तो मृदुरनिशं कलापरो धीरललितः स्यात्} (सा॰द॰~३.३४)।}
महाभारते युधिष्ठिरादयो धीर\-प्रशान्ताः।\footnote{\textcolor{red}{सामान्यगुणैर्भूयान्द्विजादिको धीरप्रशान्तः स्यात्} (सा॰द॰~३.३४)।} चतुर्षु नायकेषु श्रीरामो धीरोदात्तोऽध्यात्म\-रामायणानुसारम्। अत्र नायिका भगवती सीता मुग्धा।\footnote{\textcolor{red}{विनयार्जवादियुक्ता गृहकर्मपरा पतिव्रता स्वीया। साऽपि कथिता त्रिभेदा मुग्धा मध्या प्रगल्भेति॥ प्रथमावतीर्णयौवनमदनविकारा रतौ वामा। कथिता मृदुश्च माने समधिकलज्जावती मुग्धा॥} (सा॰द॰~३.५७-५८)} अत एव सा श्यामेति कथ्यते। \textcolor{red}{श्यामा मुग्धा हि नायिका}\footnote{मूलं मृग्यम्।} इति वचनात्। अत्र शान्त एव रसः प्रारम्भतः समाप्तिं यावत्। भक्ति\-ज्ञान\-वैराग्याणां मनोहारिणी चर्चा। अत्रत्यः श्रीरामः सुन्दरो मधुरः शिवश्च। इत्यध्यात्म\-रामायणी गङ्गा वाराणस्या गङ्गेव धारा\-त्रयी\-मिश्रिता। यथा वाराणस्या गङ्गायां हरिद्वारस्थ\-शुद्ध\-गङ्गायाः प्रयागस्थयोः यमुना\-सरस्वत्योर्मिलितः प्रवाहस्तथैवात्र राम\-भक्ति\-गङ्गायाः कर्म\-कथा\-रवि\-नन्दन्या ज्ञान\-सरसस्वत्याश्च प्रवाहा मिश्रिताः।\footnote{\textcolor{red}{रामभगति जहँ सुरसरि धारा। सरसइ ब्रह्मबिचार प्रचारा॥ बिधिनिषेधमय कलिमलहरनी। करमकथा रबिनन्दिनि बरनी॥} (रा॰च॰मा॰~१.२.८,९)। एतद्रूपान्तरम्–\textcolor{red}{रामभक्तिर्यत्र धारा सुरनद्याः प्रकीर्तिता। ब्रह्मतत्त्वविचारस्य प्रचारश्च सरस्वती॥ कलिजानां कल्मषाणां हन्त्री विधिनिषेधयुक्। श्रौती कर्मकथा यत्र यमुना परिकीर्तिता॥} (मा॰भा॰~१.२.८,९)। } वारणस्यां गङ्गा विष्णु\-प्रिया विश्वनाथ\-प्रिया तथैवेयमपि। धन्यैषा या स्वयं शशाङ्क\-शेखर\-हिमाद्रितः समुद्भवा श्रीराम\-सागर\-गामिनी च।\footnote{\textcolor{red}{पुरारिगिरिसम्भूता श्रीरामार्णवसङ्गता। अध्यात्मरामगङ्गेयं पुनाति भुवनत्रयम्॥} (अ॰रा॰~१.१.५)।}\end{sloppypar}
\begin{sloppypar}\hyphenrules{nohyphenation}\justifying\noindent\hspace{10mm} वाल्मीकीय\-रामायणं माधुर्य\-प्रधान\-श्रीराघव\-भगवत्त्ववर्णन\-परम्। तस्मात्सामान्यानां दृष्टौ बहुत्र माधुर्य\-धारायां तिरोहितत्वेन भाष्यमाणे श्रीराघवस्यैश्वर्ये सन्देहो जायते
श्रीरामस्य ब्रह्मत्वे। यथा जगच्छरण्यो रामः सुग्रीवं शरणं गत इति कथयति।\footnote{किष्किन्धा\-काण्डे चतुर्थे सर्गे। यथा \textcolor{red}{अहं चैव हि रामश्च सुग्रीवं शरणं गतौ} (वा॰रा॰~४.४.१७)।} अयमेव समुद्रं शरणं गत इति प्रतिजानीते।\footnote{युद्ध\-काण्ड एकोनविंश एकविंशे च सर्गे। यथा \textcolor{red}{समुद्रं राघवो राजा शरणं गन्तुमर्हति} (वा॰रा॰~६.१९.३०), \textcolor{red}{ततः सागरवेलायां दर्भानास्तीर्य राघवः। अञ्जलिं प्राङ्मुखः कृत्वा प्रतिशिश्ये महोदधेः॥} (वा॰रा॰~६.२१.१)।} अपहृतां सीतां स्त्रैण इव शरण्यो वरदो राघवेन्द्रो नदी\-निर्झर\-गिरि\-जन\-गुल्म\-तरु\-लताः पृच्छति।\footnote{अरण्य\-काण्डे षष्टितमे चतुःषष्टितमे च सर्गे।} इन्द्रजिता मेघनादेन ब्रह्मास्त्र\-मोहितो भव\-बन्धन\-हर्ताऽपि निबद्धो नाग\-पाशेन स्वयं मोक्ष\-रूपो मुमुक्ष्वभिमृग्य\-पदाब्ज\-पद्धतिर्मोक्ष\-दायको रघु\-नायको मुकुन्दो निज\-कृपा\-पात्र\-भूतेन निज\-चरण\-कमल\-कुन्त\-केतु\-कञ्ज\-ललित\-लक्ष्म\-सनाथित\-पृष्ठ\-भागेन सतत\-केतन\-संलिप्त\-योगि\-मुनि\-वृन्द\-परम\-हंस\-महात्म\-शिव\-विरिञ्चीन्दिरा\-वन्दित\-निखिल\-गुण\-गण\-सद्म\-शीर्ण\-सज्जन\-छद्म\-निश्छद्म\-पद\-पद्म\-परागाङ्ग\-रागेण समूढ\-पुण्डरीकाक्षेण तार्क्ष्येण मुच्यते।\footnote{युद्ध\-काण्डे पञ्च\-चत्वारिंशे पञ्चाशे च सर्गे। यथा \textcolor{red}{निरन्तरशरीरौ तौ भ्रातरौ रामलक्ष्मणौ। क्रुद्धेनेन्द्रजिता वीरौ पन्नगैः शरतां गतैः॥} (वा॰रा॰~६.४५.८), \textcolor{red}{तमागतमभिप्रेक्ष्य नागास्ते विप्रदुद्रुवुः} (वा॰रा॰~६.५०.३७)।} रावणाहतं लक्ष्मणं क्रोडीकृत्य विलपति।\footnote{युद्ध\-काण्डे द्व्युत्तर\-शततमे सर्गे। यथा \textcolor{red}{परं विषादमापन्नो विललापाकुलेन्द्रियः} (वा॰रा॰~६.१०२.१०)।} सर्वज्ञोऽपि सर्वेश्वरः सर्वाधिष्ठान\-रूपः सर्व\-शरण्यः सर्वशक्तिमान् सर्व\-श्रुति\-सिद्धान्त\-भूत\-सकल\-विद्या\-निकेतनं श्रीनिकेतनमीश्वरो भगवान् रामचन्द्रो ब्रह्मणा बोध्यमानः सन् विस्मृतमेवात्मनो भगवत्त्वं प्रतिपद्यमान इव विलोक्यते कथयति च \textcolor{red}{आत्मानं मानुषं मन्ये रामं दशरथात्मजम्} (वा॰रा॰~६.११७.११)।
सत्त्वानन्द\-विमल\-बोध\-मयस्य निरामयस्य परात्परमेश्वर\-परमात्मनो जगदात्मनो रामचन्द्रस्य माधुर्य\-लीला\-विलास\-दर्शनमदूर\-दर्शिनां
हृदये सन्देह\-दाह\-दर्शनमातनुते। इत्थमनादि\-काल\-पाप\-वासना\-दूषित\-शेमुषीकाणां वासना\-भुजङ्गिनी\-विषय\-दंष्ट्रा\-दष्ट\-मानसानां कामिनी\-कटाक्ष\-सम्पात\-पातित\-ब्रह्मचर्यादि\-सद्गुणानां भोगान्ध\-चक्षुषां मलिनं हृदय\-नभःस्थलं भासयितुं कश्यप इव कश्यप\-पितृव्यो गङ्गा\-तरल\-तरङ्ग\-भङ्गिम\-भक्त\-भव\-भय\-भीति\-समुल्लसित\-जटा\-कानन\-प्रान्तरो भगवान् भवानी\-भवो भवः शशाङ्क\-शेखरः शिवः सङ्कलित\-रमणीय\-राम\-रहस्य\-रश्मि\-राशिं मनीषि\-मनीषा\-कमलिनी\-वल्लभं बोधित\-सज्जन\-सुमनः\-सरोरुहं प्रमोदित\-भावुक\-भाव\-भृङ्गं क्षपित\-त्रिभुवन\-तमिस्र\-पटलं
राम\-भक्ति\-भावान्वितं विज्ञान\-ज्योतिर्मयमध्यात्म\-रामायण\-नामधेयमनस्तं प्रगुणं प्रभाकरं प्रादुर्भावयामास। श्रीरामस्य माधुर्य\-लीला\-स्थल\-समागत\-सन्देहानेव निवारयितुमेवेदं प्रवृत्तम्। अतः सन्दिग्ध\-विषयान् विस्पष्टयितुं भवानी\-कृत\-पूर्व\-पक्ष\-मिषेण प्रस्तौति संस्तौति च स्वयमेव तदुत्तर\-मिषेणाध्यात्म\-राम\-गाथामनाथ\-नाथो विश्व\-नाथोऽविनाशी कैलास\-वासी भगवान् शङ्करः। यथा पार्वती परमेश्वरं पूर्व\-पक्ष\-पुरःसरमभिमुखयति यद्यदि रामः सर्वज्ञस्तर्हि प्राकृत\-नर इव कथं सीताकृते विललाप यदि च सामान्य\-जीव इतरस्तर्हि कथमस्माभिः सेव्यताम्। नहि भिक्षुको भिक्षुकान्तरं याचतेऽभिक्षुक इति न्यायात्। तथा च गिरिजा स्वयमेव स्पष्टयति~–\end{sloppypar}
\centering\textcolor{blue}{यदि स्म जानाति कुतो विलापः सीताकृतेऽनेन कृतः परेण।\nopagebreak\\
जानाति नैवं यदि केन सेव्यः समो हि सर्वैरपि जीवजातैः॥}\nopagebreak\\
\raggedleft{–~अ॰रा॰~१.१.१४}\\
\begin{sloppypar}\hyphenrules{nohyphenation}\justifying\noindent इदं ग्रन्थ\-रत्नं नूनमेव कलि\-काल\-ग्रस्तानामस्मादृशां भवायैव परम\-कारुणिको भवो भावयाम्बभूव। अतः कालातीतत्वान्निटिल\-लोचनस्य त्रि\-लोचनस्याध्यात्म\-रामायण\-रचना\-कालस्तु न वक्तुं शक्यते किन्तु भूत\-नाथस्य सर्व\-भूत\-हृदयस्थत्वेऽपि सगुणोपासना\-दृष्ट्या व्याप्त्यवच्छेदकतया कैलासे सतत\-निवासस्य वेद\-रामायण\-पुराणादौ विश्रुतत्वादध्यात्म\-रामायण\-रचना कैलास एव। यथाऽत्रैव ग्रन्थे~–\end{sloppypar}
\centering\textcolor{blue}{कैलासाग्रे कदाचिद्रविशतविमले मन्दिरे रत्नपीठे\nopagebreak\\
संविष्टं ध्याननिष्ठं त्रिनयनमभयं सेवितं सिद्धसङ्घैः।\nopagebreak\\
देवी वामाङ्कसंस्था गिरिवरतनया पार्वती भक्तिनम्रा\nopagebreak\\
प्राहेदं देवमीशं सकलमलहरं वाक्यमानन्दकन्दम्॥}\nopagebreak\\
\raggedleft{–~अ॰रा॰~१.१.६}\\
\begin{sloppypar}\hyphenrules{nohyphenation}\justifying\noindent\hspace{10mm} अत्र शताधिकोपनिषच्छ्रुतीनां प्रामाण्यं समुपबृंहितम्। राम\-कथया सह जीव\-व्यथाया रमणीयः संयोगः। सर्वत्र ब्रह्म\-रूपस्य रामस्य सम्पन्नं प्रतिपादनम्। राम\-हृदयं राम\-गीता चेति द्वयमपि निखिल\-दर्शन\-सिद्धान्त\-नवनीतम्। भक्ति\-भागीरथी समुल्लसिता सोल्लासं प्रवहति। इदं चतुःषष्टि\-सर्गात्मकमध्यात्म\-रामायणं वेद\-व्यास\-प्रणीत\-ब्रह्माण्ड\-पुराणोत्तर\-खण्डे मन्यते जनैः। वयं तु शब्द\-नित्यत्व\-वादिनो वैयाकरणा नित्यमिमं सद्ग्रन्थ\-भास्करं शम्भु\-मुख\-प्रादुर्भूतं शिव\-प्रेरितेन व्यासेन पुराणे व्यवस्थापितमिति मन्यामहे। पाणिनीय\-व्याकरणस्य च महेश्वर एव आचार्यः। अध्यात्म\-रामायणञ्च महेश्वरोक्तम्। अतो द्वयोः साम्प्रदायिकीमेक\-वाक्यतामपि विभाव्याध्यात्म\-रामायणे समागतानामपाणिनीयानां प्रयोगाणां विमर्शं चिकीर्षन् कमपि बाल\-प्रयासमातनोमि। यद्यपि माहेश्वर\-व्याकरण\-सिन्धोरपेक्षया पाणिनीयं गोष्पदमिति पुराण\-विदो विदाङ्कुर्वन्तु तथाऽपि वयं तु पाणिनीयमपि माहेश्वर\-व्याकरण\-सिन्धु\-सार\-सर्वस्व\-सुधामिव मन्यामहे। पाणिनीय\-व्याकरण\-घटे शम्भु\-शब्द\-सागरः समाहित इति मे द्रढीयसी प्रतीतिः। अतः शिवोक्तेऽस्मिन्नध्यात्म\-रामायणे प्रायशः सप्त\-शत\-शब्दा अपाणिनीयाः प्रतीयन्ते। ते च सन्धि\-समास\-कारक\-कृदन्त\-तद्धित\-धातु\-प्रक्रिया\-लिङ्ग\-सम्बन्धिनः। त एव मया विमृश्यन्ते।\footnote{तेषु सप्त\-शत\-प्रयोगेषु सार्ध\-त्रिशत\-प्रयोगा ग्रन्थेऽस्मिन् विमृष्टाः।} श्रीराम\-कृपया निर्देशक\-परम\-वन्दनीय\-गुरु\-चरण\-पण्डित\-भूपेन्द्रपति\-त्रिपाठि\-महाभागैरेवमाधुनिक\-शब्द\-विद्या\-चुञ्चु\-परम\-श्रद्धेय\-वैयाकरण\-शिरोमणि\-पूज्य\-गुरुदेव\-डॉ॰\-राम\-प्रसाद\-त्रिपाठि\-चरणैर्दत्त\-बुद्धि\-वैभव एवं च परम\-सुशील\-शब्द\-सागर\-मन्दर\-मति\-काव्य\-कला\-कलाधर\-सरस\-हृदय\-सदय\-पण्डित\-प्रकाण्ड\-परम\-भावुक\-गुरु\-चरण\-डॉ॰\-कालिका\-प्रसाद\-शुक्ल\-वर्यैः सम्प्रति सम्पूर्णानन्द\-विश्व\-विद्यालय\-व्याकरण\-विभागाध्यक्ष\-पदमलङ्कुर्वद्भिः संवर्धित\-बुद्धि\-गाम्भीर्य\-गवेषणा\-गौरवो विश्वनाथ\-प्रसादतो वीणा\-वादिनी\-परमेश्वरी\-संस्मरण\-लब्धोत्साहो विगलित\-सकाम\-साधनोऽपि स्वकीय\-श्रीगुरु\-चरण\-कृपा\-धनः \textcolor{red}{अध्यात्म\-रामायणेऽपाणिनीय\-प्रयोगाणां विमर्शः} इति नामधेयस्य शोध\-प्रबन्धस्य भूमिकां प्रस्तौमि।\end{sloppypar}


% Redefines \pagenumbering to remove the part which resets numbering to 1
% TODO: Verify how removing this block affects the page numbering.
\makeatletter
\def\pagenumbering#1{
  \gdef\thepage{\csname @#1\endcsname \c@page}}
\makeatother
\mainmatter
\renewcommand{\chaptermark}[1]{\markboth{\thechapter\ #1}{}}
\renewcommand{\thepage}{\large \devanagarinumeral{page}}
\renewcommand{\chaptermark}[1]{\markboth{#1}{}}

% Nityanand Misra: LaTeX code to typeset a book in Sanskrit
% Copyright (C) 2016 Nityanand Misra
%
% This program is free software: you can redistribute it and/or modify it under
% the terms of the GNU General Public License as published by the Free Software
% Foundation, either version 3 of the License, or (at your option) any later
% version.
%
% This program is distributed in the hope that it will be useful, but WITHOUT
% ANY WARRANTY; without even the implied warranty of  MERCHANTABILITY or FITNESS
% FOR A PARTICULAR PURPOSE. See the GNU General Public License for more details.
%
% You should have received a copy of the GNU General Public License along with
% this program.  If not, see <http://www.gnu.org/licenses/>.

\renewcommand\chaptername{अथ प्रथमोऽध्यायः}
\chapter[\texorpdfstring{सन्धिकारकसमासप्रकरणम्‌}{प्रथमोऽध्यायः}]{सन्धिकारकसमासप्रकरणम्‌}
\vspace{-5mm}
\fontsize{16}{24}\selectfont\centering\hyphenrules{nohyphenation}\textcolor{blue}{अथ नत्वा घनश्यामं रामं सीतासमन्वितम्।\nopagebreak\\
अपाणिनीयानध्यात्मरामायणगतान् किल॥\nopagebreak\\
शब्दान् श्रीगुरुपादाब्जरजसा बुद्धया धिया।\nopagebreak\\
विमर्शये\footnote{\textcolor{red}{विमर्शये} इत्यत्र स्वार्थे णिच्। \textcolor{red}{निवृत्त\-प्रेषणाद्धातोः प्राकृतेऽर्थे णिजुच्यते} (वा॰प॰~३.७.६०)। स्वान्तःसुखाय विमृशामीति कर्त्रभिप्रायं ध्वनयितुमात्मने\-पदप्रयोगः। \textcolor{red}{णिचश्च} (पा॰सू॰~१.३.७४) इत्यनेन। वि~\textcolor{red}{मृशँ आमर्शने} (धा॰पा॰~१४२५)~\arrow वि~मृश्~\arrow स्वार्थे णिच्~\arrow वि~मृश्~णिच्~\arrow वि~मृश्~इ~\arrow \textcolor{red}{पुगन्त\-लघूपधस्य च} (पा॰सू॰~७.३.८६)~\arrow \textcolor{red}{उरण् रपरः} (पा॰सू॰~१.१.५१)~\arrow वि~मर्श्~इ~\arrow विमर्शि~\arrow \textcolor{red}{सनाद्यन्ता धातवः} (पा॰सू॰~३.१.३२)~\arrow धातु\-सञ्ज्ञा~\arrow \textcolor{red}{णिचश्च} (पा॰सू॰~१.३.७४)~\arrow \textcolor{red}{वर्तमाने लट्} (पा॰सू॰~३.२.१२३)~\arrow विमर्शि~इट्~\arrow विमर्शि~इ~\arrow \textcolor{red}{कर्तरि शप्‌} (पा॰सू॰~३.१.६८)~\arrow विमर्शि~शप्~इ~\arrow विमर्शि~अ~इ~\arrow \textcolor{red}{सार्वधातुकार्ध\-धातुकयोः} (पा॰सू॰~७.३.८४)~\arrow विमर्शे~अ~इ~\arrow \textcolor{red}{एचोऽयवायावः} (पा॰सू॰~६.१.७८)~\arrow विमर्शय्~अ~इ~\arrow \textcolor{red}{आद्गुणः} (पा॰सू॰~६.१.८७)~\arrow विमर्शय्~ए~\arrow विमर्शये। यद्वा \textcolor{red}{विमर्शं कुर्वे} इति विग्रहे \textcolor{red}{विमर्शये}। अत्रापि कर्त्रभिप्राये \textcolor{red}{णिचश्च} (पा॰सू॰~१.३.७४) इत्यनेनाऽत्मने\-पदम्। विमर्श~\arrow \textcolor{red}{तत्करोति तदाचष्टे} (धा॰पा॰ ग॰सू॰)~\arrow विमर्श~णिच्~\arrow विमर्श~इ~\arrow \textcolor{red}{णाविष्ठवत्प्राति\-पदिकस्य पुंवद्भाव\-रभाव\-टिलोप\-यणादि\-परार्थम्} (वा॰~६.४.४८)~\arrow विमर्श्~इ~\arrow विमर्शि~\arrow \textcolor{red}{सनाद्यन्ता धातवः} (पा॰सू॰~३.१.३२)~\arrow धातु\-सञ्ज्ञा~\arrow \textcolor{red}{णिचश्च} (पा॰सू॰~१.३.७४)~\arrow \textcolor{red}{वर्तमाने लट्} (पा॰सू॰~३.२.१२३)~\arrow विमर्शि~इट्~\arrow शेषं पूर्ववत्।} यथाबुद्धि सुधियो मर्षयन्त्वघम्॥}\nopagebreak\\
\vspace{4mm}
\fontsize{14}{21}\selectfont
\begin{sloppypar}\hyphenrules{nohyphenation}\justifying\noindent\hspace{10mm} अथ प्रकृतमनुसरामि। तत्र पूर्वं \textcolor{red}{रामायण}\-शब्द एव साधना\-प्रकारं प्रदर्शये। रामा सीता रामो रामचन्द्रो रामा च रामश्चेति विग्रहे \textcolor{red}{पुमान् स्त्रिया} (पा॰सू॰~१.२.६७) \textcolor{red}{स्त्रिया सहोक्तौ पुमाञ्छिष्यते} (वै॰सि॰कौ॰~९३३) इतिवृत्तिकेन सूत्रेणानेन स्त्री\-वाचकस्य लोपे \textcolor{red}{यः शिष्यते स लुप्यमानार्थाभिधायी}\footnote{मूलं मृग्यम्।} इति नियमाद्राम\-शब्दस्यैव द्वित्व\-बोधकतया \textcolor{red}{रामौ} इति शब्दः सीताराम\-बोधको \textcolor{red}{हंसी च हंसश्च हंसौ} इतिवत्ततो \textcolor{red}{रामयोः} सीता\-रामयोः \textcolor{red}{अयनम्‌} इति विग्रहे \textcolor{red}{इण् गतौ} (धा॰पा॰~१.१०४५) इत्यस्माद्धातोरीयते गम्यत इति कर्म\-व्युत्पत्त्या बाहुलकात्कर्मणि ल्युट्।\footnote{\textcolor{red}{कृत्यल्युटो बहुलम्‌} (पा॰सू॰~३.३.११३) इत्यनेन।}\end{sloppypar}
\begin{sloppypar}\hyphenrules{nohyphenation}\justifying\noindent\hspace{10mm} सम्प्रति \textcolor{red}{अध्यात्म\-रामायण}\-शब्दस्य सिद्धिं प्रदर्शये। अधिगतोऽन्तर्यामित्वेनाधिश्रित आत्मा मनो बुद्धिरहङ्कारश्चित्तं प्रत्यगात्मा वा याभ्यां तौ \textcolor{red}{अध्यात्मानौ} इत्यत्र \textcolor{red}{प्रादिभ्यो धातुजस्य वाच्यो वा चोत्तर\-पद\-लोपश्च} (वा॰~२.२.२२) इति वार्त्तिक\-बलेन \textcolor{red}{अनेकमन्य\-पदार्थे} (पा॰सू॰~२.२.२४) इति सूत्रेण बहुव्रीहि\-समासेऽथवाऽऽत्मानमधिगताविति \textcolor{red}{अध्यात्मानौ} एवं \textcolor{red}{अत्यादयः क्रान्ताद्यर्थे द्वितीयया} (वा॰~२.२.१८) इति वार्त्तिकेन तत्पुरुष\-समासे पुनरध्यात्म\-शब्दस्य द्विवचनान्त\-राम\-शब्देन सहाध्यात्मानौ च तौ रामौ चेति विग्रहे सति \textcolor{red}{विशेषणं विशेष्येण बहुलम्‌} (पा॰सू॰~२.१.५७) इति सूत्रेण कर्मधारय\-समासे पश्चादध्यात्म\-राम\-शब्दस्यायन\-शब्देन सह षष्ठी\-तत्पुरुषः। यद्वाऽध्यात्म\-रामाभ्यामयनमिति विग्रहे चतुर्थी\-तत्पुरुषः। \textcolor{red}{चतुर्थी तदर्थार्थ\-बलि\-हित\-सुख\-रक्षितैः} (पा॰सू॰~२.१.३६) इति सूत्रेण। उक्त\-सूत्रे वर्णित\-समस्यमान\-लक्षणतावच्छेदकत्वावच्छिन्न\-प्रतियोगितावच्छेदकताभाव\-विरहात्कथमयन\-शब्देन सह समास इति न शङ्क्यम्। पूर्वं \textcolor{red}{चतुर्थी} इति योगेन विभज्यताम्। अर्थश्च भवेत्। चतुर्थ्यन्तं सुबन्तेन समर्थेन समस्यते। इत्यर्थे समासः। पश्चात् \textcolor{red}{पूर्वपदात्सञ्ज्ञायामगः} (पा॰सू॰~८.४.३) इत्यनेन णत्वे \textcolor{red}{अध्यात्म\-रामायणम्‌}। तस्मिन् \textcolor{red}{अध्यात्म\-रामायणे}। पाणिनिना प्रोक्ताः पाणिनीयाः। \textcolor{red}{तेन प्रोक्तम्‌} (पा॰सू॰~४.३.१०१) इत्यनेन तृतीयान्त\-पाणिनि\-शब्दाच्छप्रत्यये\footnote{\textcolor{red}{पाणिनि}\-शब्दस्य \textcolor{red}{वृद्धिर्यस्याचामादिस्तद्वृद्धम्} (पा॰सू॰~१.१.७३) इत्यनेन वृद्धत्वात् \textcolor{red}{वृद्धाच्छः} (पा॰सू॰~४.२.११४) इत्यनुसारं \textcolor{red}{तेन प्रोक्तम्‌} (पा॰सू॰~४.३.१०१) इत्यनेन \textcolor{red}{छ}\-प्रत्ययः। अवृद्धात्तु \textcolor{red}{तेन प्रोक्तम्‌} (पा॰सू॰~४.३.१०१) इत्यनेन \textcolor{red}{अण्‌}\-प्रत्ययः। यथा \textcolor{red}{चन्द्रेण प्रोक्तं चान्द्रम्}।} \textcolor{red}{आयनेयीनीयियः फढखच्छघां प्रत्ययादीनाम्‌} (पा॰सू॰~७.१.२) इत्यनेन \textcolor{red}{ईय्‌}\-आदेशे भत्वात्पाणिनि\-घटकेकार\-लोपे\footnote{\textcolor{red}{यचि भम्‌} (पा॰सू॰~१.४.१८) इत्यनेन भत्वम्। \textcolor{red}{यस्येति च} (पा॰सू॰~६.४.१४८) इत्यनेनेकार\-लोपः। \textcolor{red}{सुपो धातु\-प्रातिपदिकयोः} (पा॰सू॰~२.४.७१) इत्यनेन तृतीया\-विभक्ति\-लोपः।} \textcolor{red}{जस्‌} विभक्तौ \textcolor{red}{पाणिनीयाः}। \textcolor{red}{न पाणिनीया इत्यपाणिनीया} इति नञ्समासः।\footnote{\textcolor{red}{नञ्‌} (पा॰सू॰~२.२.६) इत्यनेन समासो \textcolor{red}{नलोपो नञः} (पा॰सू॰~६.३.७३) इत्यनेन नकार\-लोपश्च।} \textcolor{red}{प्रकर्षेण युज्यन्त इति प्रयोगाः} इति प्रपूर्वो \textcolor{red}{युज्‌}\-धातोः (\textcolor{red}{युजिँर् योगे} धा॰पा॰~१४४४)
कर्मणि \textcolor{red}{घञ्‌}।\footnote{\textcolor{red}{अकर्तरि च कारके सञ्ज्ञायाम्} (पा॰सू॰~३.३.१९) इत्यनेन।} \textcolor{red}{प्रकर्षेण युज्यतेऽर्थो यैः} वेति करणे \textcolor{red}{घञ्‌}।\footnote{सोऽपि \textcolor{red}{अकर्तरि च कारके सञ्ज्ञायाम्} (पा॰सू॰~३.३.१९) इत्यनेन। \textcolor{red}{प्रयोग। पु०। प्र~{\englishfont +}~युज् भावकर्मकरणेषु यथायथं घञ्‌} इति वाचस्पत्यम्।} ततः \textcolor{red}{लशक्वतद्धिते} (पा॰सू॰~१.३.८) इति सूत्रेणेत्सञ्ज्ञायां लोपे ञकारस्य चानुबन्धकार्ये च \textcolor{red}{पुगन्तलघूपधस्य च} (पा॰सू॰~७.३.८६) इत्यनेन गुणे \textcolor{red}{चजोः कु घिण्ण्यतोः} (पा॰सू॰~७.३.५३) इत्यनेन कुत्वे जस्विभक्तौ \textcolor{red}{प्रयोगाः}। ततः \textcolor{red}{अपाणिनीयाश्चामी प्रयोगा इत्यपाणिनीयप्रयोगाः} इति कर्मधारय\-समासः। पुनः विशेषेण मृश्यत इति \textcolor{red}{विमर्शः}। वि\-पूर्वक\-\textcolor{red}{मृश्‌}\-धातोः (\textcolor{red}{मृशँ आमर्शने} धा॰पा॰~१४२५) भावे घञि\footnote{\textcolor{red}{भावे} (पा॰सू॰~३.३.१८) इत्यनेन।} पुनरनुबन्ध\-कार्ये गुणे\footnote{\textcolor{red}{पुगन्त\-लघूपधस्य च} (पा॰सू॰~७.३.८६) इत्यनेन।} रपरत्वे\footnote{\textcolor{red}{उरण् रपरः} (पा॰सू॰~१.१.५१) इत्यनेन।} विभक्ति\-कार्ये \textcolor{red}{विमर्शः}।\footnote{वि~\textcolor{red}{मृशँ आमर्शने} (धा॰पा॰~१४२५)~\arrow वि~मृश्~\arrow \textcolor{red}{भावे} (पा॰सू॰~३.३.१८)~\arrow वि~मृश्~घञ्~\arrow वि~मृश्~अ~\arrow \textcolor{red}{पुगन्त\-लघूपधस्य} (पा॰सू॰~७.३.८६)~\arrow \textcolor{red}{उरण् रपरः} (पा॰सू॰~१.१.५१)~\arrow वि~मर्श्~अ~\arrow विमर्श~\arrow विभक्तिकार्यम्~\arrow विमर्शः।} पुनः \textcolor{red}{कृदतिङ्‌} (पा॰सू॰~३.१.९३) इत्यनेन घञ्प्रत्ययत्वे कृत्सञ्ज्ञायां ततः \textcolor{red}{कर्तृ\-कर्मणोः कृति} (पा॰सू॰~२.३.६५) इत्यनेन कर्मणि षष्ठी। \textcolor{red}{अपाणिनीय\-प्रयोग}\-पदादामि नुडागमे दीर्घे णत्वे \textcolor{red}{अपाणिनीय\-प्रयोगाणाम्‌}।\footnote{अपाणिनीय\-प्रयोग~आम्~\arrow \textcolor{red}{ह्रस्वनद्यापो नुट्‌} (पा॰सू॰~७.१.५४)~\arrow \textcolor{red}{आद्यन्तौ टकितौ} (पा॰सू॰~१.१.४६)~\arrow अपाणिनीय\-प्रयोग~नुँट्~आम्~\arrow अपाणिनीय\-प्रयोग~न्~आम्~\arrow अपाणिनीय\-प्रयोग~नाम्~\arrow \textcolor{red}{नामि} (पा॰सू॰~६.४.३)~\arrow अपाणिनीय\-प्रयोगा~नाम्~\arrow \textcolor{red}{अट्कुप्वाङ्नुम्व्यवायेऽपि} (पा॰सू॰~८.४.२)~\arrow अपाणिनीय\-प्रयोगा~णाम्~\arrow अपाणिनीय\-प्रयोगाणाम्।} एवम् \textcolor{red}{अपाणिनीय\-प्रयोगाणां विमर्शः} इति। एवमध्यात्म\-रामायणोत्तर\-सप्तम्या वैषयिकतयाऽनुयोगिता\-रूपेऽर्थे स्वीकृते निष्ठा\-रूपे वाऽर्थे स्वीकृते प्रयोग\-पदोत्तर\-षष्ठ्याः कर्मता\-रूपेऽर्थेऽङ्गीकृते प्रयोग\-शब्दस्य निरूपितत्व\-सम्बन्धेनान्वयेऽध्यात्म\-रामायणानुयोगितावच्छेदक\-पाणिनीय\-प्रयोग\-भिन्न\-प्रयोग\-निष्ठ\-निरूपित\-कर्मतावच्छेदक\-विमर्शोऽथवाऽध्यात्म\-रामायण\-निष्ठतावच्छेदक\-पाणिनीय\-प्रयोग\-भिन्न\-प्रयोग\-निष्ठ\-निरूपित\-कर्मतावच्छेदक\-विमर्श इत्यर्थं प्रयच्छति \textcolor{red}{अध्यात्म\-रामायणेऽपाणिनीय\-प्रयोगाणां विमर्शः}। शोध\-प्रबन्धेऽस्मिन्नध्यात्म\-रामायणे बाल\-बुद्ध्याऽपाणिनीयताप्रतीतानां शब्दानां प्रायः सन्धि\-कारक\-समास\-लिङ्ग\-कृत्तद्धितान्त\-धातु\-सम्बन्धिनां विमर्शं कर्तुमहं निर्दिष्टः। अत्र त्रयोऽध्यायाः। प्रत्येकमध्याये द्वौ द्वौ परिच्छेदौ कल्पितौ। साम्प्रतं प्रकृते प्रथमाध्याये सन्धि\-कारक\-समास\-सम्बन्धिनोऽध्यात्म\-रामायणीया अपाणिनीयाः प्रयोगा विमृश्यन्ते।\end{sloppypar}
\vspace{4mm}
\pdfbookmark[1]{प्रथमः परिच्छेदः}{Chap1Part1}
\phantomsection
\addtocontents{toc}{\protect\setcounter{tocdepth}{1}}
\addcontentsline{toc}{section}{प्रथमः परिच्छेदः}
\addtocontents{toc}{\protect\setcounter{tocdepth}{0}}
\centering ॥ अथ प्रथमाध्याये प्रथमः परिच्छेदः ॥\nopagebreak\\
\vspace{4mm}
\pdfbookmark[2]{बालकाण्डम्‌}{Chap1Part1Kanda1}
\phantomsection
\addtocontents{toc}{\protect\setcounter{tocdepth}{2}}
\addcontentsline{toc}{subsection}{बालकाण्डीयप्रयोगाणां विमर्शः}
\addtocontents{toc}{\protect\setcounter{tocdepth}{0}}
\centering ॥ अथ बालकाण्डीयप्रयोगाणां विमर्शः ॥\nopagebreak\\
\section[जगताम्]{जगताम्‌}
\centering\textcolor{blue}{यः पृथिवीभरवारणाय दिविजैः सम्प्रार्थितश्चिन्मयः\nopagebreak\\
सञ्जातः पृथिवीतले रविकुले मायामनुष्योऽव्ययः।\nopagebreak\\
निश्चक्रं हतराक्षसः पुनरगाद्ब्रह्मत्वमाद्यं स्थिरां\nopagebreak\\
कीर्तिं पापहरां विधाय जगतां तं जानकीशं भजे॥}\nopagebreak\\
\raggedleft{–~अ॰रा॰~१.१.१}\\
\begin{sloppypar}\hyphenrules{nohyphenation}\justifying\noindent\hspace{10mm} अयं प्रयोगोऽध्यात्म\-रामायणस्य बाल\-काण्डस्य प्रथम\-सर्गस्य प्रथमे मङ्गलाचरणात्मके श्लोके \textcolor{red}{कीर्तिं पाप\-हरां विधाय जगताम्‌} इति चतुर्थ\-चरणांश उद्धृतः। निर्विघ्न\-ग्रन्थ\-समाप्तये शिवो नमस्कारात्मकं मङ्गलमाचरन् श्रीरामं स्तौति यद्भू\-भार\-हरणाय देवैः प्रार्थितो यः श्रीरामो भूतले रघु\-कुलेऽवतीर्य राक्षसान्निहत्य जगत्सु पाप\-हरां कीर्तिं व्यवस्थाप्य पुनो ब्रह्मत्वमगमत्तमेव जानकीशमहं वन्दे। अत्र \textcolor{red}{कीर्तिं पाप\-हरां विधाय} इत्यत्र प्रयुक्तो \textcolor{red}{ल्यप्‌}\-प्रत्ययान्तो वि\-पूर्वक\textcolor{red}{डुधाञ्‌}\-धातुर्धारणार्थः। यद्यपि \textcolor{red}{उपसर्गेण धात्वर्थो बलादन्यत्र नीयते। प्रहाराहार\-संहार\-विहार\-परिहारवत्॥}\footnote{मूलं मृग्यम्।} इति कारिकानुरोधेन \textcolor{red}{वि}\-उपसर्गात् \textcolor{red}{विधाय} इत्यस्य \textcolor{red}{कृत्वा} इत्यनेनार्थेन भवितव्यं किन्तु प्रकृते करण\-रूपस्यार्थस्योप\-योगो नास्त्यतोऽयं धातुरन्तर्भावित\-ण्यर्थो \textcolor{red}{विधाप्य} इत्यर्थ\-सूचकः स्वीकरणीयः। एवं \textcolor{red}{विधाय} इत्यस्य व्यवस्थाप्येत्यस्मिन्नर्थे व्यवस्थापनस्याधारे सम्भवात् \textcolor{red}{जगताम्‌} इत्यत्राधिकरण\-बोधिका सप्तम्युचिता। करणार्थेऽपि स्वीकृते करोतेश्चोत्पत्त्यर्थतयाऽत्र
सप्तम्येवोचिता। अतो \textcolor{red}{जगताम्‌} इति षष्ठ्यन्त\-प्रयोगोऽपाणिनीय इति। किन्तु विमर्शे कृत इदमपि पाणिनीय\-सिद्धान्तानुरूपम्। \textcolor{red}{विवक्षाधीनानि कारकाणि भवन्ति} इति हि भाष्य\-वचनम्।\footnote{मूलं विविध\-भाष्य\-संस्करणेषु मृग्यम्। यद्वा \textcolor{red}{कर्मादीनामविवक्षा शेषः} (भा॰पा॰सू॰~२.३.५०, २.३.५२, २.३.६७) इत्यस्य तात्पर्यमिदम्।} \textcolor{red}{विवक्षा नाम श्रोताऽर्थं बुध्येतेति वक्तुर्वक्तुमिच्छा}। अत्र 
सम्बन्ध\-विवक्षया षष्ठी। यतो हि भगवतो रामस्य कीर्तिः शाश्वत्यतस्तस्याः संसारेण सह शाश्वतः सम्बन्धः। अतः सप्तम्यपेक्षया सम्बन्ध\-विवक्षायां \textcolor{red}{षष्ठी शेषे} (पा॰सू॰~२.३.५०) इति षष्ठी विभक्तिर्वरीयसी। बहुवचनं च चतुर्दशानां भुवनानामित्यभिप्रायेण। इत्थं \textcolor{red}{जगताम्‌} इति प्रयोगः पाणिनि\-सिद्धान्तानुकूलः।\end{sloppypar}
\section[अध्यात्मरामगङ्गा]{अध्यात्मरामगङ्गा}
\centering\textcolor{blue}{पुरारिगिरिसम्भूता श्रीरामार्णवसङ्गता।\nopagebreak\\
अध्यात्मरामगङ्गेयं पुनाति भुवनत्रयम्॥}\nopagebreak\\
\raggedleft{–~अ॰रा॰~१.१.५}\\
\begin{sloppypar}\hyphenrules{nohyphenation}\justifying\noindent\hspace{10mm} अयं प्रयोगोऽध्यात्म\-रामायणस्य बाल\-काण्डस्य प्रथमे सर्गे ग्रन्थ\-प्रशंसायां कृतो वर्तते। रूपक\-विधयाऽध्यात्म\-रामायणं गङ्गात्वेन संस्मृतम्। अध्यात्म\-रामायण\-गङ्गा शङ्कर\-हिमालयात्प्रादुर्भूय श्रीराम\-रूपेण सागरेण सङ्गम्य सकल\-भुवनानि पुनातीति तात्पर्यम्। \textcolor{red}{अध्यात्म\-रामायण\-गङ्गा} इति वक्तव्ये \textcolor{red}{अध्यात्म\-राम\-गङ्गा} इत्युक्तम्। अध्यात्म\-रामायण\-गङ्गेत्यस्याध्यात्म\-राम\-गङ्गेति नैवार्थ\-बोधे समर्थः शब्दः।
पाणिनि\-मते शब्दार्थयोर्वाच्य\-वाचक\-भावः।\footnote{\textcolor{red}{तस्मात्पद\-पदार्थयोः सम्बन्धान्तरमेव शक्तिर्वाच्य\-वाचक\-भावापर\-पर्याया} (प॰ल॰म॰~१०)।} राम\-शब्दस्तद्वाच्यं दाशरथि\-नियतमर्थं बोधयिष्यति किन्त्वेकाक्षर\-कोषं विना मकारोऽकारो रकारो वा न तद्बोधयितुं समर्थः। अस्मन्मते वर्ण\-स्फोटस्य गतेवोपयोगिता। \textcolor{red}{वाक्य\-स्फोटोऽतिनिष्कर्षे तिष्ठतीति मत\-स्थितिः} (वै॰सि॰का॰~५९) इति प्राचीनोक्तेः। किं बहुना \textcolor{red}{उच्चारित एव शब्दः प्रत्यायको भवति नानुच्चारितः}\footnote{मूलं विविध\-भाष्य\-संस्करणेषु मृग्यम्। शब्द\-रत्ने च \textcolor{red}{इको यणचि} सूत्रे~– \textcolor{red}{ननु व्यक्तिपक्ष इक्पदोप\-स्थाप्योकारादिभिर्दीर्घादि\-ग्रहणं न स्यात् ‘उच्चारित एव शब्दः प्रत्यायको नानुच्चारितः’ इति ‘अणुदित्’ सूत्रे भाष्योक्तेः} (श॰र॰~४७)। यद्वा \textcolor{red}{उच्चार्यमाणः शब्दः सम्प्रत्यायको भवति न सम्प्रतीयमानः} (भा॰पा॰सू॰~१.१.६९) इत्यस्य तात्पर्यमिदम्।} इति भाष्य\-वचनादप्यध्यात्म\-राम\-गङ्गा\-शब्दोऽध्यात्म\-रामायण\-गङ्गार्थं कथं बोधयिष्यतीति चेत्। उच्यते। अत्राध्यात्म\-रामायण\-शब्देऽध्यात्म\-राम\-शब्दस्य लक्षणा। यदि चेत्सा नैव वैयाकरणैरङ्गीकृता लक्षणा\-खण्डनं विस्तरशो नागोजिभट्ट\-विरचित\-वैयाकरण\-सिद्धान्त\-लघु\-मञ्जूषायां विलसितं तदा शक्यतावच्छेदकता स्वीक्रियताम्। अथवा \textcolor{red}{विनाऽपि प्रत्ययं पूर्वोत्तर\-पद\-लोपो वक्तव्यः} (वा॰~५.३.८३)। यथा \textcolor{red}{सत्या भामा सत्यभामा भामा सत्या} इत्यादि।\footnote{\textcolor{red}{अथवा पूर्वपदलोपोऽत्र द्रष्टव्यः – अत्यन्तसिद्धः सिद्ध इति। तद्यथा देवदत्तो दत्तः सत्यभामा भामेति॥} (भा॰प॰)। \textcolor{red}{अथवा दृश्यन्ते हि वाक्येषु वाक्यैकदेशान् प्रयुञ्जानाः पदेषु च पदैकदेशान्। वाक्येषु तावद्वाक्यैकदेशान् – प्रविश पिण्डीं प्रविश तर्पर्णम्। पदेषु पदैकदेशान् – देवदत्तः दत्तः सत्यभामा भामेति} (भा॰पा॰सू॰~१.१.४५)। तत्रत्या श्रीभार्गव\-शास्त्रिणष्टिप्पणी – \textcolor{red}{अत्र वाक्यैकदेशाश्चत्वार उदाहृता इति बहुवचनमुपपद्यते। पदैकदेशाश्च द्वावेव प्रदर्शितौ। ‘देवः सत्या’ इति नोदाहृतावपि ज्ञेयाविति बहुवचनोपपत्तिः} (पाणिनीय\-व्याकरण\-महाभाष्यम्, प्रथमः खण्डः, चौखम्भा संस्कृत प्रतिष्ठान, दिल्ली, १९८७, ३९४तमे पृष्ठे)।} अनेन नियमेनायन\-शब्दस्य लोपः। तस्मादध्यात्म\-राम\-गङ्गा\-शब्दोऽध्यात्म\-रामायण\-गङ्गार्थ\-परो लुप्तेऽप्ययन\-शब्दे तदर्थ\-बोधकत्वात्।\footnote{\textcolor{red}{यः शिष्यते स लुप्यमानार्थाभिधायी} इति नियमात्।}\end{sloppypar}
\section[भक्तेषु]{भक्तेषु}
\centering\textcolor{blue}{गोप्यं यदत्यन्तमनन्यवाच्यं वदन्ति भक्तेषु महानुभावाः।\nopagebreak\\
तदप्यहोऽहं तव देव भक्ता प्रियोऽसि मे त्वं वद यत्तु पृष्टम्॥}\nopagebreak\\
\raggedleft{–~अ॰रा॰~१.१.८}\\
\begin{sloppypar}\hyphenrules{nohyphenation}\justifying\noindent\hspace{10mm}\noindent\hspace{10mm} एष प्रयोगोऽपि बाल\-काण्डस्य प्रथम\-सर्ग एव। पार्वती शिवं प्रत्यकथयद्यन्महा\-नुभावा अत्यन्त\-गोप्यमपि भक्तेषु वदन्ति। अत्र \textcolor{red}{भक्तेषु} इति सप्तमी चिन्त्या। यतो हि सप्तम्यधिकरणे भवति \textcolor{red}{सप्तम्यधिकरणे च} (पा॰सू॰~२.३.३६) इति सूत्रेण। अधिकरणं ह्याधारस्य सञ्ज्ञा \textcolor{red}{आधारोऽधिकरणम्‌} (पा॰सू॰~१.४.४५) इति सूत्रात्। अत्राऽधारस्य सम्भावनैव नास्ति।\footnote{पूर्वपक्षोऽयम्।} तथा चात्र \textcolor{red}{अकथितं च} (पा॰सू॰~१.४.५१) इत्यनेन कर्म\-सञ्ज्ञैवोचिता। एवं \textcolor{red}{भक्तेषु} इति सप्तम्या आधारः पाणिनि\-विरुद्ध इव भाति। परं विचारे कृतेऽविरुद्धमेतत्। \textcolor{red}{अकथितं च} (पा॰सू॰~१.४.५१) इत्यस्यार्थो हि सिद्धान्त\-कौमुद्यां कारक\-प्रकरणे लिखितो भट्टोजिदिक्षित\-महाभागैर्यत् \textcolor{red}{अपादानादि\-विशेषैरविवक्षितं कारकं कर्म\-सञ्ज्ञं स्यात्‌} (वै॰सि॰कौ॰~५३९)। तत्र षोडश\-धातूनां परिगणनं कारिकायामकारि~–\end{sloppypar}
\centering\textcolor{red}{दुह्याच्पच्दण्ड्रुधिप्रच्छिचिब्रूशासुजिमथ्मुषाम्।\nopagebreak\\
कर्मयुक्स्यादकथितं तथा स्यान्नीहृकृष्वहाम्॥}\nopagebreak\\
\raggedleft{–~वै॰सि॰कौ॰~५३९}\\
\begin{sloppypar}\hyphenrules{nohyphenation}\justifying\noindent इति। \textcolor{red}{अर्थ\-निबन्धनेयं सञ्ज्ञा} (वै॰सि॰कौ॰~५३९) इति नियमेनापि परिगणित\-धातु\-समानार्थकानामपि सङ्ग्रहो यथा \textcolor{red}{बलिं भिक्षते वसुधाम्‌}। अत्र हि \textcolor{red}{भिक्ष्‌}\-धातुः (\textcolor{red}{भिक्षँ भिक्षायामलाभे लाभे च} धा॰पा॰~६०६) कारिका\-परिगणित\-याच्समानार्थकः। तेनापादानेनाविवक्षितस्य बलिरित्यस्य कर्म\-सञ्ज्ञा। तथैवात्रापि धातुर्ब्रू\-समानार्थः। तस्मात् \textcolor{red}{वदन्ति} इत्यस्य योगेन \textcolor{red}{भक्तेषु} इत्यत्र द्वितीयया भवितव्यमासीत्। किन्तु यदाऽपादानादिभिरविवक्षा तदाऽयं नियम इत्येव \textcolor{red}{अकथित}\-शब्दाज्ज्ञायते। अत्र तु वैषयिकस्याधारस्य सम्भावनयाऽधिकरण\-कारकस्य विवक्षैव। अर्थाद्भक्त\-विषये वदन्ति। विषयता चात्रोपस्थिति\-रूपा। यद्वा संस्थेषु विद्यमानेषु वेत्यध्याहार्यम्। एवं च भक्तेषु विद्यमानेषु वदन्तीति निर्गलितम्। पश्चात् \textcolor{red}{गम्यमानाऽपि क्रिया कारक\-विभक्तौ प्रयोजिका} (वै॰सि॰कौ॰~५६८) इति नियमेन \textcolor{red}{गोषु दुह्यमानासु गतः} इतिवदत्रापि \textcolor{red}{यस्य च भावेन भाव\-लक्षणम्‌} (पा॰सू॰~२.३.३७) इति सूत्रेण सप्तमी। इति पाणिन्यनुकूलम्।\end{sloppypar}
\section[मे]{मे}
\centering\textcolor{blue}{अत्रोत्तरं किं विदितं भवद्भिस्तद्ब्रूत मे संशयभेदि वाक्यम्॥}\nopagebreak\\
\raggedleft{–~अ॰रा॰~१.१.१५}\\
\begin{sloppypar}\hyphenrules{nohyphenation}\justifying\noindent\hspace{10mm} अयं प्रयोगोऽपि बाल\-काण्डस्य प्रथम\-सर्गीय एव। अत्र पार्वती भगवती श्रीशिवं प्रार्थयमाना कथयति यद्यदुत्तरं ज्ञातं संशय\-भेदि वाक्यं तन्मे ब्रूतेति। \textcolor{red}{मे} इति मह्यं मम वेत्यस्याऽदेशः। अत्र द्वितीया\-स्थाने चतुर्थी\-प्रयोगः षष्ठी\-प्रयोगो वा पाणिनीयं विरुणद्धीव परं विचारे कृतेऽविरोधः। मह्यमित्यत्र \textcolor{red}{क्रियार्थोपपदस्य च कर्मणि स्थानिनः} (पा॰सू॰~२.३.१४) इत्यनेन चतुर्थी। \textcolor{red}{नमस्कुर्मो नृसिंहाय} इतिवत्। मामनुकूलयितुं मां बोधयितुं वा ब्रुवन्तु। तत्राप्रयुज्यमान\-धातु\-कर्म\-भूते \textcolor{red}{माम्‌} इत्यर्थे चतुर्थी तस्य च \textcolor{red}{मे} इत्यादेशः। यद्वा कर्मणोऽपि सम्बन्ध\-विवक्षायां षष्ठी \textcolor{red}{मातुः स्मरति} इतिवत्। अत्रापि \textcolor{red}{माम्‌} इति कर्मणः सम्बन्ध\-विवक्षा तस्मात्षष्ठी \textcolor{red}{मम} इति तस्य च \textcolor{red}{मे} इत्यादेश\footnote{\textcolor{red}{तेमयावेकवचनस्य} (पा॰सू॰~८.१.२२) इत्यनेन।} इति द्वितीयः कल्पः। अथवा \textcolor{red}{मे} इत्यस्य \textcolor{red}{संशय}\-शब्देनान्वयः। अर्थान्मत्प्रतियोगिक\-संशयस्य भेदकं वाक्यं मत्सम्बन्धि\-संशयस्येति तात्पर्यम्। सम्बन्धे षष्ठी। तत्र पार्वती प्रतियोगी संशयश्चानुयोगी विषयि\-विषय\-भाव\-सम्बन्धः। यदि चेदाशङ्का स्यात् \textcolor{red}{संशय}\-शब्दः समस्तः स च \textcolor{red}{मे}\-शब्देन सापेक्षः स च \textcolor{red}{सापेक्षः असमर्थवत्‌} इति न्यायेनासमर्थः सन् कथं समस्येदिति चेन्नित्य\-सापेक्षा\-स्थले नैवास्य नियमस्य प्रसरः \textcolor{red}{देवदत्तस्य गुरुकुलम्‌} इतिवत्। अतोऽत्र सुतरां षष्ठी।\end{sloppypar}
\section[ते कथयिष्यामि]{ते कथयिष्यामि}
\centering\textcolor{blue}{अत्र ते कथयिष्यामि रहस्यमपि दुर्लभम्।\nopagebreak\\
सीताराममरुत्सूनुसंवादं मोक्षसाधनम्॥}\nopagebreak\\
\raggedleft{–~अ॰रा॰~१.१.२५}\\
\begin{sloppypar}\hyphenrules{nohyphenation}\justifying\noindent\hspace{10mm} अत्र \textcolor{red}{त्वां कथयिष्यामि} इत्युचितम्। \textcolor{red}{कथ्‌}\-धातोः (\textcolor{red}{कथँ वाक्य\-प्रबन्धने} धा॰पा॰~१८५१) अकथित\-कर्मक\-परिगणित\-\textcolor{red}{ब्रू}\-धातोः (\textcolor{red}{ब्रूञ् व्यक्तायां वाचि} धा॰पा॰~१०४४) समानार्थकत्वात्। अतश्चतुर्थी वा षष्ठी वोभे अपि पाणिनीय\-विरुद्धे इव। \textcolor{red}{ते} इति \textcolor{red}{तुभ्यम्‌} इत्यस्य \textcolor{red}{तव} इत्यस्य वा विकरणम्।\footnote{\textcolor{red}{तेमयावेकवचनस्य} (पा॰सू॰~८.१.२२) इत्यनेन।} परमत्रोभे अपि साध्व्यौ। चतुर्थी तु \textcolor{red}{क्रियार्थोपपदस्य च कर्मणि स्थानिनः} (पा॰सू॰~२.३.१४) इत्यनेन साध्वी। अर्थात् \textcolor{red}{त्वां पार्वतीं प्रतिबोधयितुं कथयिष्यामि} अतश्चतुर्थी गम्यमान\-क्रियायाः प्रयोजकत्वात्। यद्वाऽत्र \textcolor{red}{हित}\-शब्दोऽध्याहार्यस्तथाऽत्र \textcolor{red}{ते हिताय हितं वेति कथयिष्यामि} एवं \textcolor{red}{हित\-योगे च} (वा॰~२.३.१३) इति वार्त्तिकेन चतुर्थी। यद्वा
\textcolor{red}{क्रियया यमभिप्रैति सोऽपि सम्प्रदानम्‌} (वा॰~१.४.३२) इति वार्त्तिकेनात्र चतुर्थी \textcolor{red}{पत्ये शेते} इतिवत्। अत्र शङ्करः कथन\-क्रियया पार्वतीमभि\-प्रैत्यतोऽत्र चतुर्थी। यद्वा \textcolor{red}{तादर्थ्ये चतुर्थी वाच्या} (वा॰~२.३.१३) इत्यनेन \textcolor{red}{मुक्तये हरिं भजति} इतिवदत्र चतुर्थी। अथवा \textcolor{red}{मातुः स्मरति} इतिवत्कर्मणि सम्बन्ध\-विवक्षायां षष्ठी। अथवा सम्प्रदानेन सम्बन्धेन च विवक्षितत्वाद्द्वितीयाया अवसर एव न।\footnote{\textcolor{red}{अपादानादि\-विशेषैरविवक्षितं कारकं कर्मसञ्ज्ञं स्यात्‌} (वै॰सि॰कौ॰~५३९)।}\end{sloppypar}
\section[ब्रूहि तत्त्वं हनूमते]{ब्रूहि तत्त्वं हनूमते}
\centering\textcolor{blue}{रामः सीतामुवाचेदं ब्रूहि तत्त्वं हनूमते।\nopagebreak\\
निष्कल्मषोऽयं ज्ञानस्य पात्रं नौ नित्यभक्तिमान्॥}\nopagebreak\\
\raggedleft{–~अ॰रा॰~१.१.३०}\\
\begin{sloppypar}\hyphenrules{nohyphenation}\justifying\noindent\hspace{10mm} अत्रापि स्पष्टं \textcolor{red}{ब्रू}\-धातुः (\textcolor{red}{ब्रूञ् व्यक्तायां वाचि} धा॰पा॰~१०४४)। स चाकथित\-कर्मक\-धातु\-गणना\-सूचक\-कारिकायां प्रामुख्येन गणितः। अतोऽत्र तु द्वितीया दुर्वारैव। \textcolor{red}{ब्रूहि तत्त्वं हनूमन्तम्‌} इत्युचितम्। चतुर्थ्यपाणिनीयेव। अत्र विमृश्यते। यदाऽपादानादिभिरभिधेयैरविवक्षितं सत्कारकं परिगणित\-धातुभिः सह युज्येत तदा कर्म\-सञ्ज्ञम्।\footnote{\textcolor{red}{अपादानादि\-विशेषैरविवक्षितं कारकं कर्म\-सञ्ज्ञं स्यात्‌} (वै॰सि॰कौ॰~५३९)।} इदं तु सम्प्रदानेन विवक्षितम्। सम्प्रदानादयश्च बुद्धिकृताः। अत एव \textcolor{red}{ध्रुवमपायेऽपादानम्‌} (पा॰सू॰~१.४.२४) इत्येव सूत्रं व्यवस्थाप्येतः परमपादान\-सञ्ज्ञा\-सूत्राणि सर्वाण्यपि बुद्धि\-कृतमपादानं कल्पयित्वा भाष्य\-कृता प्रत्याख्यातानि।\footnote{\textcolor{red}{अयं योगः शक्योऽवक्तुम्। ... स बुद्ध्या सम्प्राप्य निवर्तयति। तत्र “ध्रुवमपायेऽपादानम्” इत्येव सिद्धम्‌} (भा॰पा॰सू॰~१.४.२५)। \textcolor{red}{अयमपि योगः शक्योऽवक्तुम्। ... स बुद्ध्या सम्प्राप्य निवर्तते। तत्र “ध्रुवमपायेऽपादानम्” इत्येव सिद्धम्‌} (भा॰पा॰सू॰~१.४.२६)। \textcolor{red}{अयमपि योगः शक्योऽवक्तुम्। ... स बुद्ध्या सम्प्राप्य निवर्तयति। तत्र “ध्रुवमपायेऽपादानम्” इत्येव सिद्धम्‌} (भा॰पा॰सू॰~१.४.२७)। \textcolor{red}{अयमपि योगः शक्योऽवक्तुम्। ... स बुद्ध्या सम्प्राप्य निवर्तते। तत्र “ध्रुवमपायेऽपादानम्” इत्येव सिद्धम्‌} (भा॰पा॰सू॰~१.४.२८)। \textcolor{red}{अयमपि योगः शक्योऽवक्तुम्‌} (भा॰पा॰सू॰~१.४.२९)। \textcolor{red}{अयमपि योगः शक्योऽवक्तुम्‌} (भा॰पा॰सू॰~१.४.३०)। \textcolor{red}{अयमपि योगः शक्योऽवक्तुम्‌} (भा॰पा॰सू॰~१.४.३१)।} तस्मादत्रापि सम्प्रदानं विवक्षितमेव कल्प्यताम्। महा\-सञ्ज्ञानां प्रायो लक्ष्यानुरूपोऽर्थो भवत्येवान्यथा लाघव\-प्रियः पाणिनिः \textcolor{red}{घु} \textcolor{red}{टि} \textcolor{red}{घि} इत्यादि सञ्ज्ञा इवापादानाधिकरण\-सम्प्रदान\-सञ्ज्ञा अपि लघ्वीः कुर्यात्। तस्मान्महा\-सञ्ज्ञा\-करणादासां व्यवस्थितोऽर्थः। स च लक्ष्योपयोगी। प्रकृते \textcolor{red}{सम्यक्प्रकर्षेण दीयते यस्मै स सम्प्रदानम्} इति विग्रहे \textcolor{red}{सम्प्र}\-पूर्वक\-\textcolor{red}{दा}\-धातोः (\textcolor{red}{डुदाञ् दाने} धा॰पा॰~१०९१) \textcolor{red}{कृत्य\-ल्युटो बहुलम्‌} (पा॰सू॰~३.३.११३) इत्यनेन \textcolor{red}{दानीयो विप्रः} (ल॰सि॰कौ॰~७७२) इतिवत्सम्प्रदाने ल्युट्। अनुबन्ध\-कार्येऽनादेशे सम्प्रदानमिति सिद्ध्यति।\footnote{सम्~प्र~दा~\arrow \textcolor{red}{कृत्य\-ल्युटो बहुलम्‌} (पा॰सू॰~३.३.११३)~\arrow सम्~प्र~दा~ल्युँट्~\arrow सम्~प्र~दा~युँ~\arrow \textcolor{red}{युवोरनाकौ} (पा॰सू॰~७.१.१)~\arrow सम्~प्र~दा~अन~\arrow \textcolor{red}{अकः सवर्णे दीर्घः} (पा॰सू॰~६.१.१०१)~\arrow सम्~प्र~दान~\arrow सम्प्रदान~\arrow विभक्ति\-कार्यम्~\arrow सम्प्रदानम्।} हनूमांश्चात्र तत्त्व\-जिज्ञासुतया कृत\-प्रश्नः प्रबोधयितुं सीतयोपक्रम्यते। श्रीरामेण च हनूमन्तमुपदेष्टुं सीता प्रेर्यतेऽत उपदेश\-दान\-क्रियाया उद्देश्यं श्रीहनुमान्। अत्रैव \textcolor{red}{दा}\-धातोर्वाच्य\-दानस्य सम्यक्त्वं प्रकृष्टत्वञ्च सञ्जाघट्यते श्रीमहावीरे। यथाऽग्रिम\-चरणे श्रीरामः समर्थयते \textcolor{red}{निष्कल्मषोऽयं ज्ञानस्य पात्रं नौ नित्यभक्तिमान्‌}। तस्माच्चतुर्थी। अथवा श्रीरामो हनूमते हितं चिकीर्षति चिकारयिषति च। हितं च राम\-रहस्य\-तत्त्व\-ज्ञानोपदेशादेव। अतो \textcolor{red}{हिताय} इति योजनीयं ततः \textcolor{red}{हित\-योगे च} (वा॰~२.३.१३) इति वार्त्तिकेन चतुर्थी। इति नापाणिनीयता। अथवा धातूनामनेकार्थत्वात् \textcolor{red}{ब्रू}धातोर्दानार्थता। अतः \textcolor{red}{ब्रूहि तत्त्वं हनूमते} इत्यस्य \textcolor{red}{देहि तत्त्वं हनूमते} इत्यर्थः। ततः \textcolor{red}{कर्मणा यमभिप्रैति स सम्प्रदानम्‌} (पा॰सू॰~१.४.३२) इत्यनेन हनूमतः सम्प्रदान\-सञ्ज्ञा। \textcolor{red}{चतुर्थी सम्प्रदाने} (पा॰सू॰~२.३.१३) इत्यनेन \textcolor{red}{हनूमते} इत्यत्र चतुर्थी।\end{sloppypar}
\section[मया]{मया}
\centering\textcolor{blue}{मत्पाणिग्रहणं पश्चाद्भार्गवस्य मदक्षयः।\nopagebreak\\
अयोध्यानगरे वासो मया द्वादशवार्षिकः॥}\nopagebreak\\
\raggedleft{–~अ॰रा॰~१.१.३७}\\
\begin{sloppypar}\hyphenrules{nohyphenation}\justifying\noindent\hspace{10mm} अत्र \textcolor{red}{मया} इति प्रयोगो विभाव्यते। \textcolor{red}{वास}\-शब्दो भाव\-घञन्तः।\footnote{\textcolor{red}{वसँ निवासे} (धा॰पा॰~१००४) इति धातोः \textcolor{red}{भावे} (पा॰सू॰~३.३.१८) इत्यनेन घञ्। \textcolor{red}{अत उपधायाः} (पा॰सू॰~७.२.११६) इत्यनेनोपधावृद्धिः।} \textcolor{red}{घञ्‌}\-प्रत्ययश्च कृदन्तीयः।\footnote{\textcolor{red}{कृदतिङ्‌} (पा॰सू॰~३.१.९३) इत्यनेन। \textcolor{red}{कृदन्तीयः} इत्यत्र \textcolor{red}{तस्मै हितम्} (पा॰सू॰~५.१.५) इत्यनेन \textcolor{red}{छ}प्रत्ययः। \textcolor{red}{कृदन्तेभ्यो हितः} इति भावः।} तथा \textcolor{red}{कर्तृ\-कर्मणोः कृति} (पा॰सू॰~२.३.६५) इति सूत्रेण \textcolor{red}{मम} इति प्रसक्तम्। उच्यते। अत्र पूर्व\-प्रसङ्गे रामस्य लीला\-मञ्चे दृश्यमानं कर्तृत्वमौपचारिकमिति भगवती सीतोपक्रम्य निर्गुणे ब्रह्मणि रामचन्द्रे कर्तृत्वासम्भवं प्रदर्शयन्ती तत्तद्राम\-कर्तृक\-घटनासु राम\-कर्तृत्वाभासं मूल\-प्रकृतित्वात्स्वस्याः कर्तृत्वं व्यवस्थापयन्ती श्रीरघु\-नाथ\-कर्तृत्वं चाध्यारोपापवाद\-न्यायेन निराकरोति। अतोऽत्र सर्वासु घटनासु श्रीरामस्य कर्तृत्वारोपस्तस्माद्द्वादश\-वार्षिकस्यायोध्या\-वासस्य कर्ता श्रीराम एव। सीता च वास\-रूप\-क्रिया\-सिद्धौ प्रकृष्टोप\-कारकतया \textcolor{red}{साधकतमं करणम्‌} (पा॰सू॰~१.४.४२) इत्यनेन करण\-सञ्ज्ञा\-भाग्। अतः सीता\-बोधक उत्तम\-पुरुषैक\-वचन\-बोधकोऽस्मच्छब्दोऽपि \textcolor{red}{कर्तृ\-करणयोस्तृतीया} (पा॰सू॰~२.३.१८) इति सूत्रेण तृतीयामलभत। यद्वा \textcolor{red}{विनाऽपि तद्योगं तृतीया। वृद्धो यूनेत्यादिनिर्देशात्‌} (वै॰सि॰कौ॰~५६४) इति नियमेन सहशब्दाभावेऽपि तृतीया। यथा \textcolor{red}{वृद्धो यूना तल्लक्षणश्चेदेव विशेषः} (पा॰सू॰~१.२.६५) इति सूत्रे भगवान् पाणिनिः सहशब्दाभावेऽपि \textcolor{red}{यूना} इति निर्दिशति। तथैवात्रापि सह\-शब्द\-विरहेऽपि तृतीया। \textcolor{red}{मया सीतया सह रामस्य द्वादश\-वार्षिकोऽयोध्यायां वासः} इति न पाणिनि\-विरोधः।\end{sloppypar}
\section[विभीषणे राज्यदानम्]{विभीषणे राज्यदानम्‌}
\centering\textcolor{blue}{रावणस्य वधो युद्धे सपुत्रस्य दुरात्मनः।\nopagebreak\\
विभीषणे राज्यदानं पुष्पकेण मया सह॥}\nopagebreak\\
\raggedleft{–~अ॰रा॰~१.१.४१}\\
\begin{sloppypar}\hyphenrules{nohyphenation}\justifying\noindent\hspace{10mm} प्रयोगोऽयं बाल\-काण्डस्य प्रथम\-सर्गीय एव। अत्रापि \textcolor{red}{विभीषणाय राज्य\-दानम्‌} इत्यनेन भवितव्यमासीत्। यतो हि दान\-क्रियाया उद्देश्यता विभीषण एव। अस्माद्दान\-क्रियोद्देश्यस्य विभीषणस्य सम्प्रदानता निर्बाधैव। किन्त्वत्राधिकरण\-कारकम्। एवमपाणिनीयं प्रतीयते। परं विभीषण आधारत्वं परिकल्प्याधिकरण\-कारकमुचितम्। विभीषणो राम\-भक्तस्तस्य च भगवते सर्वस्व\-समर्पणम्। सम्प्रदानं हि \textcolor{red}{स्व\-स्वत्व\-निवृत्तिपूर्वकं पर\-स्वत्वोत्पादनम्‌} (त॰बो॰~५६९)। इदमत्र सम्भविष्यति नहि राम\-भक्तस्य सर्वथा स्वत्व\-हीनत्वात्। अतोऽत्र वैषयिक आधारो विभीषणः। आधारे चाधिकरण\-सञ्ज्ञा। अधिकरणे च सप्तमी निर्बाधा। \textcolor{red}{सप्तम्यधिकरणे च} (पा॰सू॰~२.३.३६) इत्यनेन। अतो \textcolor{red}{विभीषणे राज्यदानम्‌} सङ्गतम्। यद्वा \textcolor{red}{विवक्षाधीनानि कारकाणि भवन्ति}\footnote{मूलं मृग्यम्। यद्वा \textcolor{red}{कर्मादीनामविवक्षा शेषः} (भा॰पा॰सू॰~२.३.५०, २.३.५२, २.३.६७) इत्यस्य तात्पर्यमिदम्।} इति सिद्धान्तेनात्राधिकरणे विवक्षा। अथवाऽऽर्ष\-सिद्धान्त\-निरूपण\-क्रमे सूत्र\-वर्णन\-पराङ्गत्वादत्र दर्शन\-सूत्राणि वर्तन्ते। अतो यथा सूत्रे विभक्तीनां स्वातन्त्र्येण प्रयोगस्तथाऽत्रापि सीता सूत्र\-हेतुं सिद्धान्तं हनुमते कथयत्यतोऽत्र सौत्री सप्तमी। यथा पाणिनीयमधिकार\-सूत्रं \textcolor{red}{कारके} (पा॰सू॰~१.४.२३) इति। अत्र सप्तम्या उपयोगः। प्रथमार्थे सप्तमीति भाष्यकारा अपि स्वीकुर्वन्ति।\footnote{\textcolor{red}{किमिदं ‘कारके’ इति। सञ्ज्ञानिर्देशः} (भा॰पा॰सू॰~१.४.२३)। तत्र \textcolor{red}{सञ्ज्ञानिर्देश इति। सुपां सुपो भवन्तीति प्रथमायाः स्थाने सप्तमी कृतेति भावः} इति कैयटः।} तथैवात्रापि।\end{sloppypar}
\section[पूर्णेन एकत्वम्]{पूर्णेन एकत्वम्‌}
\centering\textcolor{blue}{अविच्छिन्नस्य पूर्णेन एकत्वं प्रतिपाद्यते।\nopagebreak\\
तत्त्वमस्यादिवाक्यैश्च साभासस्याहमस्तथा॥}\nopagebreak\\
\raggedleft{–~अ॰रा॰~१.१.४९}\\
\begin{sloppypar}\hyphenrules{nohyphenation}\justifying\noindent\hspace{10mm} अत्र तृतीया चिन्त्या। करणेऽपि तस्या असम्भवात्। यतो हि कारकाणि प्रायशः क्रियामेवाभ्यन्त्यनुयन्त्यत एव प्राचीनानां मते \textcolor{red}{क्रियान्वयित्वं कारकत्वम्‌} इत्येव सिद्धान्तः। यद्यपि \textcolor{red}{मातुः स्मरति} इत्यादावति\-व्याप्ति\-वारणाय \textcolor{red}{साक्षात्क्रियान्वयित्वं कारकत्वम्‌} इति प्राचीना आमनन्ति नवीनास्तु लक्षणेऽस्मिन् साक्षात्पद\-निवेशमसहमानाः \textcolor{red}{क्रिया\-जनकत्वं कारकत्वम्‌} इत्येव व्यवस्थापयन्ति। तथा च कृ\-धातुं (\textcolor{red}{डुकृञ् करणे} धा॰पा॰~१४७२) जन्यर्थकं मत्वा \textcolor{red}{करोति क्रियां निर्वर्तयतीति कारकम्‌} अस्मिन् विग्रहे कृ\-धातोः \textcolor{red}{ण्वुल्तृचौ} (पा॰सू॰~३.१.१३३) इति सूत्रेण \textcolor{red}{ण्वुल्‌}\-प्रत्ययः। \textcolor{red}{लशक्वतद्धिते} (पा॰सू॰~१.३.८) इत्यनेन लस्येत्सञ्ज्ञायां लोपे \textcolor{red}{चुटू} (पा॰सू॰~१.३.७) इत्यनेन णकारस्येत्सञ्ज्ञायां लोपे च \textcolor{red}{अचो ञ्णिति} (पा॰सू॰~७.२.११५) इत्यनेन वृद्धौ \textcolor{red}{उरण् रपरः} (पा॰सू॰~१.१.५१) इत्यनेन रपरत्वे \textcolor{red}{युवोरनाकौ} (पा॰सू॰~७.१.१) इत्यनेनाकादेशे विभक्ति\-कार्ये कारकमिति सिद्धम्। अतः पूर्वस्मिन् लक्षणे स्वीकृते क्रियान्वयित्वं कारकत्वमिति लक्षणस्य जागरूकतया \textcolor{red}{माणवकस्य पितरं पन्थानं पृच्छति} अत्र परम्परा\-सम्बन्धेन माणवकेनाऽपि क्रियान्वयः। तेनात्रापि षष्ठ्यन्ते कारकत्व\-प्रसङ्गः स्यात्। यद्यपि साक्षात्पद\-निवेशने न दोषस्तथाऽपि \textcolor{red}{मातुः स्मरति} इत्यादावपि स्थितिः सामान्या। अतो व्युत्पत्त्यनुरोधेन गौरवं चानुमाय क्रिया\-जनकत्वं कारकत्वं लक्षणं नवीनानां मते सार्वभौमतया सुस्थिरं परन्तु लक्षणेऽस्मिन् व्यवस्थितेऽपि सर्वेषां कारकाणां साक्षात्क्रिया\-जनकत्वाभावात्परम्परा\-पद\-निवेशोऽत्रापि प्राचीन\-लक्षण\-सम एव। क्रियां साक्षाद्रूपेण तु केवलं कारक\-द्वयं जनयति कर्ता कर्म च। तत्र व्यापार\-रूपां क्रियां कर्ता स्वाश्रयतयैवं फल\-रूपां क्रियां कर्म स्वानुकूलतया जनयति। शेषाणि कारकाणि यद्यप्यानुकूल्यमञ्चन्ति किन्त्वधिकरण\-कारकन्तु परम्परयैव व्यापारांशे कर्तारं सहायकं कृत्वा फलांशे च कर्म सहायकं मत्वा क्रियामुत्पादयति। अतः पक्ष\-द्वयेऽपि समान\-गौरव\-लाघवतया पूर्व\-लक्षणमेव स्फुटार्थतया श्रेयो लगति। अत एव प्रौढ\-मनोरमायामन्योऽन्याश्रय\-दोष\-परिहाराय दीक्षितेनायमेव सिद्धान्तो ध्वनितः। तत्र हीयं परिस्थितिः। अष्टाध्याय्यां प्रथमाध्याय इत्सञ्ज्ञा\-विधायक\-सूत्र\-सूत्रण\-प्रसङ्गे भगवान् पाणिनिः पपाठ \textcolor{red}{हलन्त्यम्‌} (पा॰सू॰~१.३.३) इति सूत्रम्। \textcolor{red}{वाक्यार्थ\-बोधे पदार्थ\-ज्ञानं कारणम्‌}\footnote{मूलं मृग्यम्।} अयं हि नियमः। \textcolor{red}{हलन्त्यम्‌} इति सूत्रस्य वाक्यार्थो ह्युपदेशेऽन्त्यं हलित्स्यात्। अत्र पदार्थ\-बोधोऽपि विचार्यताम्। \textcolor{red}{उपदेशे इत्‌} इति पद\-द्वितयम् \textcolor{red}{उपदेशेऽजनुनासिक इत्‌} (पा॰सू॰~१.३.२) अस्मात्सूत्रादनुवृत्तम्। \textcolor{red}{सूत्रेष्वदृष्टं पदं सूत्रान्तरादनुवर्तनीयं सर्वत्र} (ल॰सि॰कौ॰~१) इति वरदराजोक्तेः। एवं \textcolor{red}{हल्‌} \textcolor{red}{अन्त्यम्‌} इति मुख्ये सूत्रस्थे पदे। तत्र हल्शब्दस्य कोऽर्थ इति जिज्ञासायां हल्सञ्ज्ञा\-विधायकस्य \textcolor{red}{आदिरन्त्येन सहेता} (पा॰सू॰~१.१.७१) इति सूत्रस्य वाक्यार्थे विचारयिष्यमाणे तत्र हेतुत्वात्पदार्थ\-बोधस्य पूर्वं पदार्थ\-चिन्तनमनिवार्यतां गतम्। तत्र \textcolor{red}{आदिः} \textcolor{red}{अन्त्येन} \textcolor{red}{सह} \textcolor{red}{इता} इमानि चत्वारि पदानि। सूत्रेऽस्मिन्नेवं समुदाये सञ्ज्ञाया अनुपयोगादवयवेषु विश्रामः। तथा च \textcolor{red}{अन्त्येन} इत्यप्रधान\-तृतीया\-निर्देशान्मध्य\-वर्तिभिः सहाऽदि\-वाच्यस्यापि सङ्ग्रहः। तथा च \textcolor{red}{अन्त्येनेता सहित आदिर्मध्यगानां स्वस्य च सञ्ज्ञा स्यात्‌} (ल॰सि॰कौ॰~४)। अत्र \textcolor{red}{अन्त्य}\-पदस्य त्वर्थः सुस्पष्टः किन्तु \textcolor{red}{इत्‌}\-शब्दस्यार्थोऽनवगतः। स चेत्सञ्ज्ञा\-विधायकात् \textcolor{red}{हलन्त्यम्‌} (पा॰सू॰~१.३.३) इति सूत्रादवगन्तुं शक्यते। तदपि तदैवेत्पदार्थं बोधयिष्यति यदा तस्य हल्पदार्थो बोधितो भविष्यति। हल्पदार्थ\-बोधश्च तत्सञ्ज्ञा\-विधायक \textcolor{red}{आदिरन्त्येन सहेता} (पा॰सू॰~१.१.७१) इति सूत्राधीनः। स चेत्पदार्थ\-ज्ञानमन्तरेण हल्पदार्थं बोधयितुं शक्नोत्येव नहि। इत्थं हल्पदार्थ\-ज्ञानमित्पदार्थाधीनम्। इत्पदार्थ\-ज्ञानञ्च हल्पदार्थाधीनम्। अत एकैकमेकैकाधीनमित्येव परस्परापेक्षत्व\-रूपोऽन्योऽन्याश्रयः। अन्योऽन्याश्रयाणि कार्याणि न प्रकल्पन्ते। यथा नावि बद्धा नौर्नैव गतिशीला भवतीति भाष्ये\footnote{\textcolor{red}{इतरेतराश्रयाणि च कार्याणि न प्रकल्पन्ते। तद्यथा नौर्नावि बद्धा नेतरत्राणाय भवति} (भा॰पा॰सू॰~१.१.१)।} सिद्धान्तितत्वान्महान् विप्लवः समुपस्थितः। \textcolor{red}{अन्योऽन्याश्रयत्वं हि तद्ग्रह\-सापेक्ष\-ग्रह\-सापेक्ष\-ग्रह\-विषयत्वम्‌}। तद्ग्रहो हल्पदार्थ\-ग्रहस्तत्सापेक्ष\-ग्रह इद्ग्रहस्तत्सापेक्ष\-ग्रहो हल्ग्रहस्तद्विषयत्वमित्सञ्ज्ञा\-सूत्र एवमेव तद्ग्रह इद्ग्रहस्तद्विषयत्वं हल्सञ्ज्ञा\-सूत्र इत्युभयतस्पाशा रज्जुः। इत्थमन्योऽन्याश्रयमाशङ्क्य श्रीदीक्षितेन \textcolor{red}{हलन्त्यम्‌} इति सूत्रमेवावर्तितम्। विवेकोऽयं यन्मूल\-सूत्रमसमस्तम्। समासे कृते सति \textcolor{red}{अन्त्य}\-शब्दस्य विशेषणतया पूर्व\-निपातापत्तेः। किन्त्वावृत्त\-सूत्रं \textcolor{red}{हलन्त्यम्‌} इति समस्तम्। हल्शब्दस्यान्त्य\-शब्देन सह कः समासो भवेदित्येव विचारयितुमुपक्रान्तं वैयाकरण\-सिद्धान्त\-कौमुदी\-टीका\-प्रौढमनोरमा\-सञ्ज्ञा\-प्रकरणे। यथा प्रौढ\-मनोरमायाम्~–\end{sloppypar}
\centering\textcolor{red}{“हलि अन्त्यम्” इति विग्रहे “सप्तमी” (पा॰सू॰~२.१.४०) इति योग\-विभागात् “सुप्सुपा” (पा॰सू॰~२.१.४) इति वा समासः। यद्वा षष्ठी\-तत्पुरुषोऽयम्॥}\nopagebreak\\
\raggedleft{–~प्रौ॰म॰~१}\\
\begin{sloppypar}\hyphenrules{nohyphenation}\justifying\noindent इति। तत्र \textcolor{red}{सप्तमी} इति प्रतीकमादाय शब्दरत्ने व्याचक्षते श्रीहरिदीक्षित\-महाभागाः \textcolor{red}{अधिकरण\-कारकस्य कर्त्राद्यन्वय\-द्वारा क्रियान्वयादस्ति सामर्थ्यमिति भावः} (श॰र॰~१)। \textcolor{red}{सप्तमी शौण्डैः} (पा॰सू॰~२.१.४०) इति सूत्रे योग\-विभागं मत्वा सप्तमी\-समासो दर्शितः। तत्रायं पूर्वपक्षस्याक्षेपो यदधिकरण\-कारकं क्रियया नान्वेति। असति क्रियान्वये क्रियान्वय\-लक्षण\-कारकत्वस्याभावात्तस्मिन् कथं सामर्थ्यं सामर्थ्याभावे च कथं समास इति चेत्। हरिदीक्षितः कथयति यत्कर्तारं कर्म च द्वारीकृत्याधिकरण\-कारकं क्रियायामन्वेति क्रिया च ते एव माध्यमं कृत्वाऽधिकरण\-कारकेणान्वेति। यदाऽकर्मका क्रिया तदा कर्ता माध्यमो यदा च सकर्मिका तदा कर्म माध्यममित्यपि क्रियते। तद्यथा \textcolor{red}{रामः शय्यायां शेते} इत्यत्र स्व\-वृत्ति\-वृत्तित्व\-सम्बन्धेन क्रियाऽन्वेति। स्वं शय्या तद्वृत्ती रामस्तद्वृत्तिः शयनानुकूलो व्यापारः। एवमेव \textcolor{red}{स्थाल्यां तण्डुलं पचति} इत्यादौ स्वं स्थाली तत्र तद्वृत्तिस्तण्डुलस्तद्वृत्तिः पाकः। एवमेव क्रियाऽधिकरण\-कारकेन स्वाश्रयाश्रयत्व\-सम्बन्धेनान्वेति। यथा \textcolor{red}{सीता वाटिकायां वर्तते} इत्यत्र स्वं वर्तनानुकूला क्रिया तदाश्रयः सीता तदाश्रयश्च वाटिकेति।\end{sloppypar}
\begin{sloppypar}\hyphenrules{nohyphenation}\justifying\noindent\hspace{10mm} इत्थं क्रियान्वयित्व\-रूप\-कारकत्व\-सार्वजनीनत्वात् \textcolor{red}{पूर्णेन एकत्वम्‌} इत्येव साधयितुं नैव किमपि कारकं सङ्घटते। अस्मादत्रोपपद\-विभक्तिः। तस्मादुपपदं विना तृतीयाऽपाणिनीयेति चेत्। विभक्तिर्द्विधा कारक\-विभक्तिरुपपद\-विभक्तिश्च। क्रियामाश्रित्य जायमाना विभक्तिः कारक\-विभक्तिः। एवं पदमाश्रित्य जायमाना विभक्तिरुपपद\-विभक्तिः। कारक\-विभक्तिर्यथा \textcolor{red}{रामं नमति} अत्र हि नमन\-क्रियामाश्रित्यैव द्वितीया\-विभक्तिरुत्पन्ना। द्वितीया च याऽपरा सा पदमाश्रित्यैव। यथा \textcolor{red}{नमः शिवाय} अत्र क्रियान्वयाभावात्कारक\-विभक्तिर्नास्ति। एवं पद\-प्रयोगाभावादतोऽत्र तृतीया किमाधारा इति \textcolor{red}{वृद्धो यूना तल्लक्षणश्चेदेव विशेषः} (पा॰सू॰~१.२.६५) इत्यत्र सह\-प्रयोगं विनाऽपि सहोक्त्या तृतीया दरीदृश्यते।\footnote{\textcolor{red}{विनाऽपि तद्योगं तृतीया। वृद्धो यूनेत्यादिनिर्देशात्‌} (वै॰सि॰कौ॰~५६४)।} तस्मादत्र गम्यमान\-पदस्य कारक\-विभक्तीनां प्रयोजकत्वात् \textcolor{red}{पूर्णेन एकत्वम्‌} इत्यत्र पाणिनीयताऽक्षतैव।\end{sloppypar}
\section[साक्षात्कथितं तव]{साक्षात्कथितं तव}
\centering\textcolor{blue}{इदं रहस्यं हृदयं ममात्मनो मयैव साक्षात्कथितं तवानघ।\nopagebreak\\
मद्भक्तिहीनाय शठाय न त्वया दातव्यमैन्द्रादपि राज्यतोऽधिकम्॥}\nopagebreak\\
\raggedleft{–~अ॰रा॰~१.१.५२}\\
\begin{sloppypar}\hyphenrules{nohyphenation}\justifying\noindent\hspace{10mm} अत्रापि \textcolor{red}{अकथितं च} (पा॰सू॰~१.४.५१) इत्यनेन कर्म\-सञ्ज्ञा तथा च द्वितीयोचितैव। किन्तु \textcolor{red}{विवक्षाधीनानि कारकाणि भवन्ति}\footnote{मूलं मृग्यम्। यद्वा \textcolor{red}{कर्मादीनामविवक्षा शेषः} (भा॰पा॰सू॰~२.३.५०, २.३.५२, २.३.६७) इत्यस्य तात्पर्यमिदम्।} इति न्यायाङ्गीकारेण कर्मणि सम्बन्ध\-विवक्षया षष्ठी। यद्वा \textcolor{red}{समक्षम्‌} इत्यध्याहार्यम्। \textcolor{red}{तव समक्षं कथितम्‌} इति तात्पर्यम्। एवमत्र तु सुतरां सम्बन्धो निर्बाध एव। तेनात्र निसर्गतः षष्ठी।
अथवाऽत्र \textcolor{red}{पृष्ट}\-शब्दस्याध्याहारः। \textcolor{red}{तव पृष्टस्य कथितम्‌}। ततो \textcolor{red}{यस्य च भावेन भाव\-लक्षणम्‌} (पा॰सू॰~२.३.३७) इत्यनेन पृष्ट\-शब्देन कथन\-रूप\-क्रियान्तरस्य द्योतनात्षष्ठी कारकीया।\footnote{\textcolor{red}{दूरान्तिकार्थैः षष्ठ्यन्यतरस्याम्‌} (पा॰सू॰~२.३.३४) इत्यतः \textcolor{red}{षष्ठी} इत्यनुवर्त्य \textcolor{red}{षष्ठी चानादरे} (पा॰सू॰~२.३.३८) इत्यतः \textcolor{red}{षष्ठी} इत्यपकृष्य वाऽऽदरेऽपि भावलक्षणा षष्ठीति भावः।}\end{sloppypar}
\section[तेऽभिहितम्]{तेऽभिहितम्‌}
\centering\textcolor{blue}{एतत्तेऽभिहितं देवि श्रीरामहृदयं मया।\nopagebreak\\
अतिगुह्यतमं हृद्यं पवित्रं पापशोधनम्॥}\nopagebreak\\
\raggedleft{–~अ॰रा॰~१.१.५३}\\
\begin{sloppypar}\hyphenrules{nohyphenation}\justifying\noindent\hspace{10mm} अत्रापि \textcolor{red}{ब्रू}\-धातु\-समानार्थक\-\textcolor{red}{अभिधा}\-प्रकृतिकस्य\footnote{\textcolor{red}{ब्रूञ् व्यक्तायां वाचि} (धा॰पा॰~१०४४)। \textcolor{red}{अभि}\-पूर्वको \textcolor{red}{धा}\-धातुः (\textcolor{red}{डुधाञ् धारण\-पोषणयोः} धा॰पा॰~१०९२) कथनार्थः। \textcolor{red}{क्त}प्रत्यये \textcolor{red}{दधातेर्हिः} (पा॰सू॰~७.४.४२) इत्यनेन \textcolor{red}{हि}आदेशे विभक्तिकार्ये \textcolor{red}{अभिहितम्}।} \textcolor{red}{अभिहित}\-शब्दस्य प्रयोगेण \textcolor{red}{अकथितं च} (पा॰सू॰~१.४.५१) इति कर्म\-सञ्ज्ञया द्वितीया दुर्वारा किन्त्वत्र सम्प्रदान\-विवक्षया चतुर्थी। किं वा \textcolor{red}{त्वां सन्तोषयितुमभिहितम्‌} क्रियार्थोप\-पदस्याप्रयुज्यमानस्य कर्मणि चतुर्थी \textcolor{red}{क्रियार्थोपपदस्य च कर्मणि स्थानिनः} (पा॰सू॰~२.३.१४) इत्यनेन। यद्वा \textcolor{red}{ते हिताय} इति हित\-शब्दमध्याहार्यम्। ततः \textcolor{red}{हित\-योगे च} (वा॰~२.३.१३) इति वार्त्तिकेन चतुर्थी। यद्वा \textcolor{red}{सुख}पदमध्याहार्यम्। \textcolor{red}{ते तुभ्यं हनुमते सुखायाभिहितम्‌}। ततः \textcolor{red}{चतुर्थी तदर्थार्थ\-बलि\-हित\-सुख\-रक्षितैः} (पा॰सू॰~२.१.३६) इत्यत्र पठित\-सुख\-शब्दस्य चतुर्थी\-परत्व\-सूचनादत्र चतुर्थी।\footnote{यद्वा \textcolor{red}{तादर्थ्ये चतुर्थी वाच्या} (वा॰~२.३.१३) इति वार्त्तिकेनात्र चतुर्थी। त्वदर्थमभिहितम् इत्यर्थः।} यद्वा \textcolor{red}{ते} इति षष्ठ्यन्त्यम्। एवमत्र सम्बन्धे षष्ठी। यद्वा \textcolor{red}{प्रश्नोत्तर}\-शब्दोऽध्याहार्यः। \textcolor{red}{तव प्रश्नोत्तरमभिहितम्‌}।\footnote{अत्र \textcolor{red}{प्रश्न}\-शब्दस्य साकाङ्क्षतया \textcolor{red}{उत्तर}\-शब्देन कथं समास इति न शङ्क्यम्। नित्य\-सापेक्षस्थलेष्वस्य नियमस्याप्रसरात्। यद्वा \textcolor{red}{प्रश्नोत्तर}\-शब्देन पार्ष्ठिकोऽन्वयः।}\end{sloppypar}
\section[रामेणोक्तं पुरा मम]{रामेणोक्तं पुरा मम}
\centering\textcolor{blue}{शृणु देवि प्रवक्ष्यामि गुह्याद्गुह्यतरं महत्।\nopagebreak\\
अध्यात्मरामचरितं रामेणोक्तं पुरा मम॥}\nopagebreak\\
\raggedleft{–~अ॰रा॰~१.२.४}\\
\begin{sloppypar}\hyphenrules{nohyphenation}\justifying\noindent\hspace{10mm} अयं प्रयोगोऽध्यात्म\-रामायणस्य बाल\-काण्डस्य द्वितीय\-सर्गस्य चतुर्थे श्लोके शिवेन कृतो वर्तते। कृत\-राम\-विषयक\-प्रश्नां पार्वतीं सम्बोधयन् शिवः प्राह यत् \textcolor{red}{यत्कथा\-वस्तु पुरा रामेण ममोक्तम्‌}। अत्र तु \textcolor{red}{उक्तम्‌} इति शब्दोऽकथित\-गणित\-ब्रू\-धातु\-प्रकृतिक एव। यतो हि \textcolor{red}{ब्रू}\-धातोः (\textcolor{red}{ब्रूञ् व्यक्तायां वाचि} धा॰पा॰~१०४४) कर्मणि \textcolor{red}{क्त}\-प्रत्ययः। \textcolor{red}{ब्रुवो वचिः} (पा॰सू॰~२.४.५३) इत्यनेन \textcolor{red}{वच्‌}\-आदेशः। \textcolor{red}{वचि\-स्वपि\-यजादीनां किति} (पा॰सू॰~६.१.१५) इत्यनेन सम्प्रसारणम्। \textcolor{red}{सम्प्रसारणाच्च} (पा॰सू॰~६.१.१०८) इत्यनेन पूर्वरूपैकादेशः। \textcolor{red}{चोः कुः} (पा॰सू॰~८.२.३०) इत्यनेन कुत्वम्। अत्र साक्षात्कारिका\-परिगणित\-\textcolor{red}{ब्रू}\-धातोरुपस्थितौ द्वितीयाया अवश्यम्भावितया षष्ठीति पाणिनि\-विरुद्धेव किन्तु वस्तुतस्त्वनुरुद्धाऽत्र सम्बन्ध\-विवक्षया षष्ठी। रामायण\-शिवयोः प्रतिपाद्य\-प्रतिपादक\-रूप\-शाश्वत\-सम्बन्धस्य वक्तुमिष्टत्वात्।
\textcolor{red}{मम पुरः} इति वा \textcolor{red}{मम हितार्थम्‌} इति वा। \textcolor{red}{मम श्रवणे} इति वा। अत्रास्मच्छब्दस्य श्रवण\-शब्देनावयवावयवि\-भाव\-सम्बन्धस्य सिद्धत्वात्सम्बन्धे षष्ठी।\end{sloppypar}
\section[ब्रह्मणे प्राह]{ब्रह्मणे प्राह}
\centering\textcolor{blue}{भूमिर्भारेण मग्ना दशवदनमुखाशेषरक्षोगणानां\nopagebreak\\
धृत्वा गोरूपमादौ दिविजमुनिजनैः साकमब्जासनस्य।\nopagebreak\\
गत्वा लोकं रुदन्ती व्यसनमुपगतं ब्रह्मणे प्राह सर्वं\nopagebreak\\
ब्रह्मा ध्यात्वा मुहूर्तं सकलमपि हृदावेदशेषात्मकत्वात्॥}\nopagebreak\\
\raggedleft{–~अ॰रा॰~१.२.६}\\
\begin{sloppypar}\hyphenrules{nohyphenation}\justifying\noindent\hspace{10mm} अत्रापि \textcolor{red}{अकथितं च} (पा॰सू॰~१.४.५१) इत्यनेन द्वितीयैव। तत्स्थाने चतुर्थी तु \textcolor{red}{ब्रह्माणं मोदयितुं प्राह} इत्यप्रयुज्यमान\-मोदन\-क्रिया\-कर्मीभूत\-\textcolor{red}{ब्रह्म}\-शब्दात्।\footnote{\textcolor{red}{क्रियार्थोपपदस्य च कर्मणि स्थानिनः} (पा॰सू॰~२.३.१४) इत्यनेन।} यद्वा \textcolor{red}{ब्रह्मणे हिताय} इत्यध्याहारे \textcolor{red}{हित\-योगे च} (वा॰~२.३.१३) इत्यनेन चतुर्थी। यद्वा सुखमित्यध्याहृत्य \textcolor{red}{चतुर्थी तदर्थार्थबलिहितसुखरक्षितैः} (पा॰सू॰~२.१.३६) इति चतुर्थी\-समास\-सङ्केत\-सूचनाच्चतुर्थी। यद्वा \textcolor{red}{ब्रह्म परमात्मानं नयति धरा\-धाम प्रापयतीति ब्रह्मणः} इति ब्रह्मोपपदे नी\-धातोर्व्युत्पन्नम्।\footnote{ब्रह्म नयतीति ब्रह्मणः। \textcolor{red}{ब्रह्म}\-उपपदे \textcolor{red}{नी}\-धातोः (\textcolor{red}{णीञ् प्रापणे} धा.पा. ९०१) \textcolor{red}{अन्येष्वपि दृश्यते} (पा॰सू॰~३.२.१०१) इत्यनेन \textcolor{red}{ड}\-प्रत्ययः। ब्रह्मन्~अम्~नी~ड~\arrow ब्रह्मन्~अम्~नी~अ~\arrow \textcolor{red}{डित्यभस्याप्यनु\-बन्धकरण\-सामर्थ्यात्‌} (वा॰~६.४.१४३)~\arrow ब्रह्मन्~अम्~न्~अ~\arrow \textcolor{red}{सुपो धातुप्रातिपदिकयोः} (पा॰सू॰~२.४.७१)~\arrow ब्रह्मन्~न्~अ~\arrow \textcolor{red}{नलोपः प्रातिपदिकान्तस्य} (पा॰सू॰~८.२.७)~\arrow ब्रह्म~न्~अ~\arrow ब्रह्मन~\arrow \textcolor{red}{पूर्वपदात्सञ्ज्ञायामगः} (पा॰सू॰~८.४.३)~\arrow ब्रह्मण~\arrow विभक्ति\-कार्यम्~\arrow ब्रह्मणः।} ततश्च सप्तमी \textcolor{red}{उपस्थिते} इति शब्देऽध्याहृते \textcolor{red}{यस्य च भावेन भाव\-लक्षणम्‌} (पा॰सू॰~२.३.३७) इत्यनेन। यद्वा \textcolor{red}{नमस्कुर्मो नृसिंहाय} इतिवद्ब्रह्माणमनुकूलयितुं प्राह। यद्वाऽत्र \textcolor{red}{पत्ये शेते} इतिवत् \textcolor{red}{क्रियया यमभिप्रैति सोऽपि सम्प्रदानम्‌} (वा॰~१.४.३२) इति वार्त्तिकेन कथन\-क्रियया ब्रह्मणोऽभिप्रेतत्वात्सम्प्रदाने ततश्चतुर्थी।\end{sloppypar}
\section[कश्यपस्य वरो दत्तः]{कश्यपस्य वरो दत्तः}
\centering\textcolor{blue}{कश्यपस्य वरो दत्तस्तपसा तोषितेन मे॥}\nopagebreak\\
\raggedleft{–~अ॰रा॰~१.२.२५}\\
\begin{sloppypar}\hyphenrules{nohyphenation}\justifying\noindent\hspace{10mm} अत्र भाराक्रान्तया गो\-रूप\-धारिण्या पृथिव्या सह देवैः क्षीर\-सागरमभिगम्य स्तुवन्तं ब्रह्माणं प्रति स्वकीयावतरण\-प्रकारं प्रकटयन् भगवान् प्रणिगदति यन्मया पूर्वं कश्यपाय वरो दत्तो वर्तते। अतस्तस्यैव गृहे पुत्र\-रूपेणावतरिष्यामि। तत्र दत्त\-पद\-प्रयोगेण चतुर्थ्युचिता \textcolor{red}{तस्मै चपेटां ददाति} (भा॰पा॰सू॰~१.१.१) इति भाष्य\-प्रयोगात्किन्तु \textcolor{red}{कश्यपस्य} इति षष्ठी तन्त्रविरुद्धेव। परं नैतत्। दानस्य कर्मणा यमभिप्रैति स सम्प्रदानम्।\footnote{\textcolor{red}{कर्मणा यमभिप्रैति स सम्प्रदानम्} (पा॰सू॰~१.४.३२)। \textcolor{red}{दानस्य कर्मणा यमभिप्रैति स सम्प्रदान\-सञ्ज्ञः स्यात्} (वै॰सि॰कौ॰~५६९, ल॰सि॰कौ॰~८९६)।} अत्र सम्प्रदानस्य न विवक्षा। यतो हि प्रभुरात्मानं न सम्यक्प्रददाति। मङ्गलाचरण एव स्व\-धाम\-गमन\-सङ्केतात्। यथा \textcolor{red}{रजकस्य वस्त्रं ददाति} इत्यत्र क्षालयितुं वस्त्राणि दीयन्ते पुनश्च परावर्त्यन्ते तथैवात्रापि सप्तविंशति\-वर्षाणां कृते पुत्र\-रूपेणाऽगतः\footnote{कथं तर्हि \pageref{text:exileage1}\-तमे पृष्ठे ग्रन्थ\-प्रस्तावनायाम्~– \textcolor{red}{जन्मतो विवाहं यावद्द्वादशाब्दावधिस्ततो द्वादश\-वर्षं यावदयोध्यायां वास एवं पञ्चविंशे वर्षे सीता\-लक्ष्मणाभ्यां सह वन\-गमनम्‌} इति। कल्पभेदेन।} कश्यपावतारस्य दशरथस्य समक्षम्। पुनः श्रीराम\-वियोगानल\-दग्ध\-शरीरः सनयन\-नीरो धीरो दशरथ एव कश्यपतां गतः। अतोऽत्र षष्ठी दशरथस्याल्प\-कालिकत्वं सूचयति। सम्प्रदानं हि \textcolor{red}{स्व\-स्वत्व\-निवृत्तिपूर्वकं पर\-स्वत्वोत्पादनम्‌} (त॰बो॰~५६९) इति तत्त्वबोधिनी। अत्र भगवान् स्व\-स्वत्वं निवर्तयत्येव नहि स्थले स्थले लोकोत्तर\-कौतुक\-प्रदर्शनाय। यद्वा \textcolor{red}{कृते} इत्यध्याहार्यम्। \textcolor{red}{कश्यपस्य कृते वरो दत्तः} इति कृते\-योगे षष्ठी।
यद्वा \textcolor{red}{भार्यायै} इत्यध्याहार्यम्। \textcolor{red}{कश्यपस्य भार्यायै वरो दत्तः} अतो दाम्पत्य\-भाव\-रूपे सम्बन्धे षष्ठी।\end{sloppypar}
\section[दृष्टं मे]{दृष्टं मे}
\centering\textcolor{blue}{त्वं ममोदरसम्भूत इति लोकान्विडम्बसे।\nopagebreak\\
भक्तेषु पारवश्यं ते दृष्टं मेऽद्य रघूत्तम॥}\nopagebreak\\
\raggedleft{–~अ॰रा॰~१.३.२६}\\
\begin{sloppypar}\hyphenrules{nohyphenation}\justifying\noindent\hspace{10mm} अत्र परम\-पितरं परमात्मानं स्वपुर ईश्वर\-रूपेण प्रस्तुतं विलोक्य भगवती कौसल्या स्तौति यत् \textcolor{red}{हे रघूत्तम अद्य भक्तेषु ते पारवश्यं मे दृष्टम्‌}। \textcolor{red}{दृश्‌}\-धातोः (\textcolor{red}{दृशिँर् प्रेक्षणे} धा॰पा॰~९८८) कर्मणि क्त\-प्रत्यये कृते तेन च कर्मणोऽभिहितत्वात्कर्तुश्चानभिहितत्वात् \textcolor{red}{कर्तृ\-करणयोस्तृतीया} (पा॰सू॰~२.३.१८) इत्यनेनानभिहितेऽस्मत्पद\-वाच्य\-कौसल्या\-रूपिणि कर्तरि तृतीया। अत्र षष्ठी\-विचार\-विषयतामाटीकते। कर्तरि सम्बन्ध\-विवक्षायां षष्ठी। यद्वा \textcolor{red}{दृष्टम्‌} इति भावे कृत्प्रत्ययः \textcolor{red}{नपुंसके भावे क्तः} (पा॰सू॰~३.३.११४) इत्यनेन। ततश्च भावस्य विवक्षयाऽविवक्षितत्वाच्च कर्तुः \textcolor{red}{क्तस्य च वर्तमाने नपुंसके भाव उपसङ्ख्यानम्‌} (वा॰~२.३.६५) इत्यनेन \textcolor{red}{मम} इत्यत्र षष्ठी\footnote{यथा \textcolor{red}{छात्रस्य हसितम्‌। नटस्य भुक्तम्‌। मयूरस्य नृत्तम्‌। कोकिलस्य वयाहृतम्‌} (भा॰पा॰सू॰~२.३.६७) इत्यादि\-भाष्योदाहरणेषु \textcolor{red}{गतं तिरश्चीनमनूरुसारथेः} (शि॰व॰~१.२) \textcolor{red}{हसितं मधुरम् ... मधुराधिपतेः} (म॰अ॰~१) इत्यादि\-शिष्ट\-प्रयोगेषु च। \textcolor{red}{अद्य मे यद्दृष्टं दर्शनं तद्भक्तेषु ते पारवश्यम्‌} इति तात्पर्यम्।} तस्य च \textcolor{red}{मे} इत्यादेशः।\footnote{\textcolor{red}{तेमयावेकवचनस्य} (पा॰सू॰~८.१.२२) इत्यनेन।} यद्वा \textcolor{red}{मे} इत्यस्मात्परं \textcolor{red}{पुरतः} इत्यध्याहार्यम्। \textcolor{red}{मे पुरतो दृष्टम्‌} अत्र सम्बन्धे षष्ठी स्वारसिकी। यद्वा \textcolor{red}{मे} शब्दस्य \textcolor{red}{रघूत्तम}\-शब्देन अन्वयः। अर्थात् \textcolor{red}{हे मे मम रघूत्तम भक्तेषु ते पारवश्यमथ दृष्टम्‌}। अत्र पुत्र\-भावनया \textcolor{red}{मे रघूत्तम} इति व्याहरति। यथाऽयोध्या\-काण्डे स्वयमेव कौसल्या कथयति यत् \textcolor{red}{पुत्रः सभार्यो वनमेव यातः सलक्ष्मणो मे रघुरामचन्द्रः} (अ॰रा॰~२.७.८५)। \textcolor{red}{अयं मम पुत्रः} इति कौसल्या\-वचनं माधुर्यं सूचयति। अत्र च जन्य\-जनक\-भाव\-रूपे सम्बन्धे षष्ठीत्यनेन सूचितं यद्यदा जीवः श्रीरामं प्रत्येव \textcolor{red}{मम} इति सम्बोधयति तदा तस्य संसार\-ममता\-जालं नश्यति।\end{sloppypar}
\section[रामेति]{रामेति}
\centering\textcolor{blue}{यस्मिन् रमन्ते मुनयो विद्यया ज्ञानविप्लवे।\nopagebreak\\
तं गुरुः प्राह रामेति रमणाद्राम इत्यपि॥}\nopagebreak\\
\raggedleft{–~अ॰रा॰~१.३.४०}\\
\begin{sloppypar}\hyphenrules{nohyphenation}\justifying\noindent\hspace{10mm} एष प्रयोगोऽध्यात्म\-रामायण\-बाल\-काण्ड\-तृतीय\-सर्गे चत्वारिंशे श्लोके श्रीमता वसिष्ठेन कृतो भगवतो नाम\-करण\-प्रसङ्गे। अत्र राम\-शब्दस्य व्युत्पत्ति\-प्रकार\-द्वयं दर्शयति। एकोऽधिकरण\-घञन्तोऽपरः कर्त्रजन्तश्च। यस्मिन्मुनयो रमन्ते स रामो यश्च रमयति रमते वा स राम इति। अत्र प्रथमैक\-वचनान्तो \textcolor{red}{रामः} इति शब्दः। ततश्च \textcolor{red}{इति}\-शब्देन सह संहिता\-काले \textcolor{red}{राम सुँ} इति स्थिते पश्चात् \textcolor{red}{ससजुषो रुः} (पा॰सू॰~८.२.६६) इत्यनेन रुत्वे \textcolor{red}{भो\-भगो\-अघो\-अपूर्वस्य योऽशि} (पा॰सू॰~८.३.१७) इत्यनेन रोर्यत्वे \textcolor{red}{लोपः शाकल्यस्य} (पा॰सू॰~८.३.१९) इत्यनेन यकारलोपे पुना राम\-घटकाकारस्येति\-घटकेकारेण सहाऽशङ्क्यमाने गुणे \textcolor{red}{पूर्वत्रासिद्धम्‌} (पा॰सू॰~८.२.१) इत्यनेन त्रिपादीत्वाद्यलोपासिद्धौ गुणानवसरे \textcolor{red}{राम इति} इत्येव पाणिनीयम्। \textcolor{red}{पूर्वत्रासिद्धम्‌} (पा॰सू॰~८.२.१) इति सूत्रस्य जागरूकतायां गुणाभावे कथं \textcolor{red}{रामेति} इति चेत्। \textcolor{red}{मतौ च्छः सूक्तसाम्नोः} (पा॰सू॰~५.२.५९) इति सूत्र\-ज्ञापनात् \textcolor{red}{अनुकरणानु\-कार्ययोर्भेदाभेद\-विवक्षा च}\footnote{मूलं मृग्यम्। \textcolor{red}{मतौ च्छः सूक्तसाम्नोः} (पा॰सू॰~५.२.५९) इत्यस्य भाष्ये प्रदीपोद्द्योतयोश्च स्पष्टमिदम्। अभेदपक्षे \textcolor{red}{प्रकृतिवदनुकरणं भवति} (भा॰शि॰सू॰~२) इति महाभाष्ये \textcolor{red}{ऋऌक्‌} (शि॰सू॰~२) शिवसूत्र उक्तम्। \textcolor{red}{अनुकरणं ह्यनुकार्याद्भिन्नम्‌} इत्यपि महाभाष्ये \textcolor{red}{मतौ च्छः सूक्तसाम्नोः} (पा॰सू॰~५.२.४९) सूत्र उक्तमिति वैयाकरण\-भूषण\-सारस्य दर्पण\-व्याख्यायां चन्द्रिका\-प्रसाद\-द्विवेदाः। अस्माभिर्भाष्य\-संस्करणेषु \textcolor{red}{अनुकरणं ह्यनुकार्याद्भिन्नम्‌} इति नोपलब्धम्।} इति
परिभाषया तावद्भेद\-विवक्षाऽभेद\-विवक्षा च क्रियते। इति\-शब्द\-समभिव्याहरणेनात्र द्विःप्रयुक्तो राम\-शब्दोऽनुकरण\-परः। इति\-शब्दो ह्यनुकरण\-द्योतकः। तथा च पाणिनेः सूत्रम् \textcolor{red}{अव्यक्तानुकरणस्यात इतौ} (पा॰सू॰~६.१.९८)। \textcolor{red}{ध्वनेरनुकरणस्य योऽच्छब्दस्तस्मादितौ परे पररूपमेकादेशः स्यात्। पटत् इति पटिति} (वै॰सि॰कौ॰~८१)। अत्रत्या तत्त्वबोधिनी च~– \textcolor{red}{यद्यपि “अतो गुणे” इति पूर्व\-सूत्रादत इत्यनुवर्त्यातो\-ग्रहणमिह त्यक्तुं शक्यं तथाऽपि पूर्व\-सूत्रे अत इति तपर\-करणाद्ध्रस्वाकारस्य ग्रहणमिह तु शब्दाधिकार\-पक्षाश्रयणादच्छब्दस्य ग्रहणमिति व्याख्याने क्लेशः स्यादिति पुनरत्रातो\-ग्रहणं कृतम्। अव्यक्त\-शब्दं व्याचष्टे “ध्वनेरिति”। “अनुकरणस्येति”। परिस्फुटाकारादि\-वर्णस्येति भावः। तस्य चानुकरणत्वं किञ्चित्साम्येन बोध्यम्। पर\-रूपस्यास्य नित्यत्वेऽपि संहितायामविवक्षितायां तदभावादाह। “पटदितीति”} (त॰बो॰~८१)। अतोऽत्राभेद\-विवक्षायां विभक्त्यभावः।\footnote{तथा च \textcolor{red}{गवित्ययमाह। अत्रानु\-कार्यानु\-करणयोर्भेदस्याविवक्षितत्वादसत्यर्थवत्त्वे विभक्तिर्न भवति} (का॰वृ॰~१.१.१६)।} एवं च \textcolor{red}{आद्गुणः} (पा॰सू॰~६.१.८७) इत्यनेन गुणः।\footnote{अत्रोक्तम्~– \textcolor{red}{अत्र द्विःप्रयुक्तो राम\-शब्दोऽनुकरण\-परः}। यद्युभयत्रानुकरण\-परो रामशब्दः कथं तर्हि \textcolor{red}{रमणाद्राम इत्यपि} इति चेत्। संहिताया अविवक्षणात्।} ज्ञानाधिकरण\-ज्ञान\-स्वरूप इति न्याय\-वेदान्त\-बोध्य\-निर्गुण\-ब्रह्माभिन्न एव दशरथ\-पुत्रो राम इत्येव वसिष्ठ\-तात्पर्यं द्योतयितुमत्राभेद\-विवक्षायां संहिता। \textcolor{red}{संहितात्वं नामार्ध\-मात्रा\-कालातिरिक्त\-काल\-व्यवधान\-शून्यत्वम्‌}। यद्वा \textcolor{red}{अपदं न प्रयुञ्जीत} इति व्याकरण\-प्रसिद्धेर्द्विपदमनुकुर्वतो द्विपदस्य परम\-पदस्य भगवतः श्रीरामचन्द्रस्य कृते किमपदं प्रयुञ्जीत शिव इत्यपेक्षायामुच्यते। \textcolor{red}{सह सुपा} (पा॰सू॰~२.१.४) इति हि सूत्रम्। अत्र हि योग\-विभागः \textcolor{red}{सह} इति पृथक्पदं \textcolor{red}{सुपा} इति च पृथक्। सहेति समर्थेन सह समस्यते इत्यर्थक\-योग\-विभाग\-प्रथमांशेनात्र समासः \textcolor{red}{राम}\-शब्दस्य \textcolor{red}{इति}\-शब्देन। पश्चात् \textcolor{red}{सुपो धातु\-प्रातिपादिकयोः} (पा॰सू॰~२.४.७१) इत्यनेन विभक्ति\-लोपः। पश्चाद्गुणः। प्रत्यय\-लक्षणमाश्रित्य पुनर्गुण\-व्यवधानं न शक्यं यथा \textcolor{red}{गो\-हितम्‌} इत्यत्रान्तर्वर्तिनीं \textcolor{red}{ङे}\-विभक्तिमाश्रित्य न \textcolor{red}{अव्‌}\-आदेशस्तथैव \textcolor{red}{वर्णाश्रये नास्ति प्रत्यय\-लक्षणम्‌} (प॰शे॰~२०) इत्यनेनात्रापि प्रत्यय\-लक्षणं निषेध्यम्। अतो \textcolor{red}{रामेति} अयं शब्दः पाणिनीय एव।\end{sloppypar}
\section[मुनीन्द्राहम्]{मुनीन्द्राहम्‌}
\label{sec:munindraham}
\centering\textcolor{blue}{अभिवाद्य मुनिं राजा प्राञ्जलिर्भक्तिनम्रधीः।\nopagebreak\\
कृतार्थोऽस्मि मुनीन्द्राहं त्वदागमनकारणात्॥}\nopagebreak\\
\raggedleft{–~अ॰रा॰~१.४.३}\\
\begin{sloppypar}\hyphenrules{nohyphenation}\justifying\noindent\hspace{10mm} अयं च प्रयोगोऽध्यात्म\-रामायणस्य बाल\-काण्डस्य चतुर्थ\-सर्गीयः। अत्रायोध्या\-समागतं श्रीराघव\-समेतं विश्वामित्रं प्रणम्य प्राञ्जलिर्योग\-राजो दशरथः कथयति \textcolor{red}{मुनीन्द्र अहं कृतार्थः} एवम्। \textcolor{red}{दूराद्धूते च} (पा॰सू॰~८.२.८४) इत्यनेन प्लुतस्ततश्च \textcolor{red}{प्लुत\-प्रगृह्या अचि नित्यम्‌} (पा॰सू॰~६.१.१२५) इत्यनेन प्रकृति\-भावस्तस्मादतः \textcolor{red}{मुनीन्द्र३ अहम्‌} इत्येव पाणिनीयम्। \textcolor{red}{मुनीन्द्राहम्‌} इत्यार्षप्रयोगो नैव पाणिनीय इति न भ्रमितव्यम्। अत्र विश्वामित्रमभिगम्यैव दशरथः प्रणमति तेन दूर\-सम्बोधनत्वाभावान्न प्लुतावसरः। यद्वा \textcolor{red}{गुरोरनृतोऽ\-नन्त्यस्याप्येकैकस्य प्राचाम्‌} (पा॰सू॰~८.२.८६) इत्यत्र \textcolor{red}{प्राचाम्‌} इति योग\-विभागः। तथोक्तं सिद्धान्त\-कौमुद्याम् \textcolor{red}{इह प्राचामिति योगो विभज्यते। तेन सर्वः प्लुतो विकल्प्यते} (वै॰सि॰कौ॰~९७)। अनेन योग\-विभागेन सर्वेषां प्लुतानां वैकल्पिकता। प्लुताभावे \textcolor{red}{अकः सवर्णे दीर्घः} (पा॰सू॰~६.१.१०१) इत्यनेन सवर्णदीर्घः। अथवा \textcolor{red}{अकः सवर्णे दीर्घः} (पा॰सू॰~६.१.१०१) इति सूत्रं सपाद\-सप्ताध्यायीस्थं प्लुत\-विधायकञ्च त्रिपादीस्थं \textcolor{red}{दूराद्धूते च} (पा॰सू॰~८.२.८४)। ततः \textcolor{red}{पूर्वत्रासिद्धम्‌} (पा॰सू॰~८.२.१) इति सूत्र\-बलेन सपाद\-सप्ताध्याय्या दीर्घ\-शास्त्र\-कर्तव्यतायां प्लुत\-विधायकं त्रिपादी\-शास्त्रमसिद्धम्। एवं तन्निमित्तक\-प्रकृति\-भावस्य नास्ति प्रसरः। अतः \textcolor{red}{मुनीन्द्राहम्‌} इत्यत्र दीर्घः पाणिनि\-तन्त्रशोऽनुकूलः। एवमेव \textcolor{red}{तदप्यहोऽहं तव देव भक्ता} (अ॰रा॰~१.१.८) इत्यत्रापि समाधेयम्। यतो हि \textcolor{red}{अहो} इत्योदन्त\-निपातः। \textcolor{red}{ओत्‌} (पा॰सू॰~१.१.१५) इत्यनेन सूत्रेण \textcolor{red}{अहो} इत्यस्यौदन्त\-निपातत्वात्प्रगृह्य\-सञ्ज्ञा। ततोऽकारेऽचि परे \textcolor{red}{प्लुतप्रगृह्या अचि नित्यम्‌} (पा॰सू॰~६.१.१२५) इत्यनेन प्रकृतिभावे \textcolor{red}{अहो अहम्‌} इत्येव पाणिनीयम् \textcolor{red}{अहोऽहम्‌} इत्यपाणिनीयं प्रतीयत इति चेत्। अत्रापि \textcolor{red}{प्राचाम्‌} (पा॰सू॰~८.२.८६) इति योग\-विभागेन प्लुतस्य विकल्प इव प्रगृह्यस्यापि विकल्पता। उपलक्षणत्वात्। यद्वा \textcolor{red}{प्लुतप्रगृह्या अचि नित्यम्‌} (पा॰सू॰~६.१.१२५) इत्यत्र नित्य\-ग्रहणं प्रायिकार्थे। अर्थात्क्वचिदचि परे प्रगृह्यः प्रकृति\-भाव\-भाङ्न। यथा \textcolor{red}{ङमो ह्रस्वादचि ङमुण्नित्यम्‌} (पा॰सू॰~८.३.३२) इत्यत्र नित्यग्रहणस्य प्रायिकत्वात् \textcolor{red}{इको यणचि} (पा॰सू॰~६.१.७७) \textcolor{red}{सुप्तिङन्तं पदम्‌} (पा॰सू॰~१.४.१४) इत्यादौ न ङुण्मुटौ। तथैवात्र नित्य\-ग्रहणस्य प्रायिकत्वात् \textcolor{red}{अहो} इत्यस्मात्परेऽप्यकारेऽचि न प्रकृतिभावः।\end{sloppypar}
\section[मह्यम्]{मह्यम्‌}
\centering\textcolor{blue}{प्रत्याख्यातो यदि मुनिः शापं दास्यत्यसंशयः।\nopagebreak\\
कथं श्रेयो भवेन्मह्यमसत्यं चापि न स्पृशेत्॥}\nopagebreak\\
\raggedleft{–~अ॰रा॰~१.४.११}\\
\begin{sloppypar}\hyphenrules{nohyphenation}\justifying\noindent\hspace{10mm} विश्वामित्रो दशरथमभिगम्य राक्षस\-वधार्थी सलक्ष्मणं श्रीरामचन्द्रं याचते। तत्र पुत्र\-वत्सलतया रामचन्द्रं न दित्सन्महाराज\-दशरथो विकल्पते यच्छ्रीराम\-विसर्जनेऽहं जीवनमपि नैव धारयितुं शक्नोमि। एवं च प्रत्याख्यातः स मुनिः शापं दास्यति। एवमसत्य\-भाषण\-जनित\-पातकमपि लगिष्यति। तदानीमयं प्रयोगः \textcolor{red}{कथं श्रेयो भवेन्मह्यम्‌} इति।
अत्रास्मच्छब्दस्य चतुर्थ्येक\-वचनान्त\-रूपं \textcolor{red}{मह्यम्‌} इति।\footnote{अस्मद् ङे~\arrow \textcolor{red}{ङे प्रथमयोरम्‌} (पा॰सू॰~७.१.२८)~\arrow अस्मद्~अम्~\arrow \textcolor{red}{तुभ्यमह्यौ ङयि} (पा॰सू॰~७.२.९५)~\arrow मह्य~अद्~अम्~\arrow \textcolor{red}{अतो गुणे} (पा॰सू॰~६.१.९७)~\arrow मह्यद्~अम्~\arrow \textcolor{red}{शेषे लोपः} (पा॰सू॰~७.२.९०)~\arrow मह्य्~अम्~\arrow मह्यम्।} \textcolor{red}{मह्यम्‌} शब्दस्य श्रेयः\-प्रतियोगितया साकाङ्क्षत्वात्प्रतियोगिनः षष्ठ्यन्तत्वमेव। दृष्टि\-दृष्ट\-प्रायत्वादत्र सम्बन्धे षष्ठी भवेत्। किन्तु चतुर्थीयं महाराज एव श्रेयसः सम्प्रदानमिति विवक्षया। यद्वा \textcolor{red}{मह्यम्‌} इत्यस्य \textcolor{red}{शापं दास्यति} इत्यत्रान्वयः। अर्थात्प्रत्याख्यातो मुनिरसंशयं मह्यं शापं दास्यत्येवं कथं श्रेयो भवेदसत्यं च न स्पृशेदित्यत्रान्वय\-प्रकारः। इत्थं \textcolor{red}{मह्यम्‌} शब्दस्य \textcolor{red}{श्रेयः} शब्देनासत्यन्वये परिहारः। यद्वा \textcolor{red}{मह्यम्‌} न शब्द\-रूपमपि तु कृत्य\-प्रत्ययान्तम्। \textcolor{red}{महँ पूजायाम्‌} (धा॰पा॰~७३०, १८६७) इत्यस्माद्धातोः \textcolor{red}{कृत्य\-ल्युटो बहुलम्‌} (पा॰सू॰~३.३.११३) इति सूत्रानुरोधेनानुबन्ध\-लोपे सति \textcolor{red}{मह्यत इति मह्यम्‌} इत्यस्मिन् विग्रहे
कर्मणि यत्प्रत्ययः। अनुबन्ध\-लोपे विभक्ति\-कार्ये सौ सोरमादेशे\footnote{\textcolor{red}{अतोऽम्‌} (पा॰सू॰~३.३.११३) इत्यनेन} \textcolor{red}{मह्यम्‌} इति सिद्धम्। अत्र \textcolor{red}{मह्यम्‌} शब्दः \textcolor{red}{श्रेयः} शब्दस्य विशेषणम्। अर्थात् \textcolor{red}{मह्यं महत्त्व\-पूर्णम्‌}।\end{sloppypar}
\section[रामाय]{रामाय}
\centering\textcolor{blue}{विश्वामित्रोऽपि रामाय तां योजयितुमागतः।\nopagebreak\\
एतद्गुह्यतमं राजन्न वक्तव्यं कदाचन॥}\nopagebreak\\
\raggedleft{–~अ॰रा॰~१.४.१९}\\
\begin{sloppypar}\hyphenrules{nohyphenation}\justifying\noindent\hspace{10mm} वसिष्ठ\-वाक्यमिदम्। अत्र \textcolor{red}{रामेण योजयितुं} इति पाणिनीयं \textcolor{red}{रामाय} इत्यार्ष\-चतुर्थी। विचारे कृत इयमपि पाणिनीया। अत्र \textcolor{red}{हित}\-शब्दोऽध्याहार्यः। ततश्च \textcolor{red}{हित\-योगे च} (वा॰~२.३.१३) इत्यनेन चतुर्थी। यद्वा \textcolor{red}{रामं सुखयितुं तां योजयितुमागतः} इति \textcolor{red}{क्रियार्थोपपदस्य च कर्मणि स्थानिनः} (पा॰सू॰~२.३.१४) इत्यनेन चतुर्थी।\end{sloppypar}
\section[रामरामेति]{रामरामेति}
\centering\textcolor{blue}{आहूय रामरामेति लक्ष्मणेति च सादरम्।\nopagebreak\\
आलिङ्ग्य मूर्ध्न्यवघ्राय कौशिकाय समर्पयत्॥\footnote{\textcolor{red}{समार्पयत्} इति प्रयोक्तव्ये \textcolor{red}{समर्पयत्} इति प्रयोगः। अस्मिन् विषये \pageref{sec:prasarayat}तमे पृष्ठे \ref{sec:prasarayat} \nameref{sec:prasarayat} इति प्रयोगस्य विमर्शं पश्यन्तु।}}\nopagebreak\\
\raggedleft{–~अ॰रा॰~१.४.२२}\\
\begin{sloppypar}\hyphenrules{nohyphenation}\justifying\noindent\hspace{10mm} अत्र \textcolor{red}{राम}\-शब्देन सह \textcolor{red}{इति}\-शब्दस्य संहिता। तत्र \textcolor{red}{राम}\-घटकाकारस्य \textcolor{red}{इति}\-घटकेकारेण सह संहितायामुभयोः स्थाने गुणः। स एव विचार्यः। \textcolor{red}{दूराद्धूते च} (पा॰सू॰~८.२.८४) इति सूत्रेण दूर\-सम्बोधन\-वाक्यस्य \textcolor{red}{राम३ राम३} इत्यस्य टेर्मकारोत्तराकारस्य प्लुतः। एवं \textcolor{red}{प्लुत\-प्रगृह्या अचि नित्यम्‌} (पा॰सू॰~६.१.१२५) इति प्रकृति\-भावः स्यात्। अत्र च \textcolor{red}{राम३ राम३ इति} इत्येव पाणिनि\-तन्त्रीयः। अत्र सन्धिरार्षः।\footnote{पूर्वपक्षोऽयम्।} इत्थमेव प्रथम\-श्रीमद्भागवत\-टीका\-कारैः श्रीधर\-स्वामि\-पादैः~–\end{sloppypar}
\centering\textcolor{red}{यं प्रव्रजन्तमनुपेतमपेतकृत्यं द्वैपायनो विरहकातर आजुहाव।\nopagebreak\\
पुत्रेति तन्मयतया तरवोऽभिनेदुस्तं सर्वभूतहृदयं मुनिमानतोऽस्मि॥}\nopagebreak\\
\raggedleft{–~भा॰पु॰~१.२.२}\\
\begin{sloppypar}\hyphenrules{nohyphenation}\justifying\noindent इत्यत्र \textcolor{red}{पुत्रेति} अयं प्रयोगोऽप्यार्षत्वेन समाहितः।\footnote{\textcolor{red}{द्वैपायनो व्यासो विरहात्कातरो भीतः सन्पुत्र३ इति पलुतेन आजुहाव आहूतवान्। दूरादाह्वाने प्लुते सत्यपि सन्धिरार्षः} (भा॰पु॰~श्री॰टी॰~१.२.२)।} तथैवात्रापि। वस्तुतस्त्वयं पाणिनीय एव। अत्र चत्वारः पक्षाः प्रदर्श्यन्ते। प्रथमः \textcolor{red}{गुरोरनृतोऽनन्त्यस्याप्येकैकस्य प्राचाम्‌} (पा॰सू॰~८.२.८६) इत्यत्र \textcolor{red}{प्राचाम्‌} इति योग\-विभागस्तेन सर्वेषां प्लुतानां विकल्पः।\footnote{\textcolor{red}{इह प्राचामिति योगो विभज्यते। तेन सर्वः प्लुतो विकल्प्यते} (वै॰सि॰कौ॰~९७)। अत्रत्या तत्त्वबोधिनी~– \textcolor{red}{सर्वः प्लुतो विकल्प्यत इति। एतेन “द्वैपायनो विरहकातर आजुहाव पुत्रेति” इति भागवतं व्याख्यातम्। प्लुतस्य वैकल्पिकत्वात् ‘आर्षः प्रयोगः’ इति श्रीधराचार्योक्तिस्तु नादर्तव्या} (त॰बो॰~९७)।} अतोऽत्रापि प्लुताभावाद्गुणः सुकरः। द्वितीयः पक्षः \textcolor{red}{अप्लुतवदुपस्थिते} (पा॰सू॰~६.१.१२९) इत्यनेन प्लुतस्याप्लुतवद्भाव उपस्थित\-शब्देऽवैदिक\-\textcolor{red}{इति}\-परे। अर्थादवैदिक \textcolor{red}{इति}\-परे प्लुतोऽप्लुत\-वद्भवतीत्यनेनाप्लुत\-वद्भावस्ततश्च गुणोऽतो \textcolor{red}{रामरामेति} अयं प्रयोगः पाणिनीयः। पक्षान्तरेऽपि \textcolor{red}{सम्बुद्धौ शाकल्यस्येतावनार्षे} (पा॰सू॰~१.१.१६) इति सूत्रेण वैकल्पिक\-प्रकृति\-भावः। चतुर्थे पक्षे प्लुत\-प्रकरणमेवानित्यम्। अथ मया पञ्चम\-पक्षः प्रस्तूयते यद्गुण\-विधौ कर्तव्ये प्लुत\-शास्त्रमेवासिद्धमतो \textcolor{red}{रामरामेति} पाणिन्यनुकूल एव।\end{sloppypar}
\section[लक्ष्मणेति]{लक्ष्मणेति}
\centering\textcolor{blue}{आहूय रामरामेति लक्ष्मणेति च सादरम्।\nopagebreak\\
आलिङ्ग्य मूर्ध्न्यवघ्राय कौशिकाय समर्पयत्॥}\nopagebreak\\
\raggedleft{–~अ॰रा॰~१.४.२२}\\
\begin{sloppypar}\hyphenrules{nohyphenation}\justifying\noindent\hspace{10mm} अयमपि प्रयोगस्तथैव। \textcolor{red}{लक्ष्मण इति} स्थिते प्लुते जाते प्रकृति\-भावो नित्यत्वेन प्राप्तः पुनः \textcolor{red}{प्राचाम्‌} (पा॰सू॰~८.२.८६) इति योग\-विभाग\-सामर्थ्यात्प्लुत\-विकल्पे सन्धिः साधीयान्।\end{sloppypar}
\section[दीक्षां प्रविश्यताम्]{दीक्षां प्रविश्यताम्‌}
\centering\textcolor{blue}{पूजां च महतीं चक्रू रामलक्ष्मणयोर्द्रुतम्।\nopagebreak\\
श्रीरामः कौशिकं प्राह मुने दीक्षां प्रविश्यताम्॥}\nopagebreak\\
\raggedleft{–~अ॰रा॰~१.५.३}\\
\begin{sloppypar}\hyphenrules{nohyphenation}\justifying\noindent\hspace{10mm} कौशिकेन स्वाश्रमं नीतो भगवान् श्रीरामो विश्वामित्रं कथयति \textcolor{red}{हे मुने भवता दीक्षां प्रविश्यताम्‌}। अत्र \textcolor{red}{दीक्षां प्रविश} इति कर्तृ\-वाच्य\-रूपम्। कर्म\-वाच्ये च \textcolor{red}{लः कर्मणि च भावे चाकर्मकेभ्यः} (पा॰सू॰~३.४.६९) इति सूत्रानुसारं \textcolor{red}{भाव\-कर्मणोः} (पा॰सू॰~१.३.१३) इत्यनेनाऽत्मने\-पदम्। \textcolor{red}{त}\-प्रत्ययः \textcolor{red}{टित आत्मनेपदानां टेरे} (पा॰सू॰~३.४.७९) इत्यनेन \textcolor{red}{ते} इति जातः। ततश्च \textcolor{red}{आमेतः} (पा॰सू॰~३.४.९०) इत्यनेन लोड्लकारे \textcolor{red}{आम्‌} आदेशे \textcolor{red}{प्रविश्यताम्‌}।\footnote{प्र~\textcolor{red}{विशँ प्रवेशने} (धा॰पा॰~१४२४)~\arrow प्र~विश्~\arrow \textcolor{red}{भाव\-कर्मणोः} (पा॰सू॰~१.३.१३)~\arrow \textcolor{red}{लोट् च} (पा॰सू॰~३.३.१६२)~\arrow प्र~विश्~लोट्~\arrow प्र~विश्~त~\arrow \textcolor{red}{सार्वधातुके यक्‌} (पा॰सू॰~३.१.६७)~\arrow प्र~विश्~यक्~त~\arrow प्र~विश्~य~त~\arrow \textcolor{red}{टित आत्मनेपदानां टेरे} (पा॰सू॰~३.४.७९)~\arrow प्र~विश्~य~ते~\arrow \textcolor{red}{आमेतः} (पा॰सू॰~३.४.९०)~\arrow प्र~विश्~य~ताम्~\arrow प्रविश्यताम्।} एवं \textcolor{red}{यस्मिन्नर्थे प्रत्ययः स उक्तो} महा\-सञ्ज्ञा\-करणाच्च \textcolor{red}{प्रत्याययत्यर्थं यः स प्रत्यय} इत्युभयोर्नियमयोर्जागरूकत्वे कर्मणि प्रत्ययस्य विहितत्वात्प्रथमैवोचिता। एवं च \textcolor{red}{अनभिहिते} (पा॰सू॰~२.३.१) इति सूत्रं ह्यधिकारः। अधिकारो नामोत्तरोत्तर\-सम्बन्धित्वम्। ततः \textcolor{red}{कर्मणि द्वितीया} (पा॰सू॰~२.३.२) इत्यनेनाप्येतस्य सम्बन्धः। अथ च \textcolor{red}{अनभिहिते कर्मणि द्वितीया} इत्येव सूत्रार्थः। तस्मात्कर्मणोऽभिहितत्वादत्र प्रथमैव। तथा चोक्तं सिद्धान्त\-कौमुद्यां कारक\-प्रकरणे \textcolor{red}{अभिधानं च प्रायेण तिङ्कृत्तद्धित\-समासैः} (वै॰सि॰कौ॰~५३७) तेन \textcolor{red}{हरिः सेव्यते} इति प्रयोग इवात्रापि तिङ्। तिङोक्तं कर्म। अतो \textcolor{red}{दीक्षाम्‌} इत्यत्र द्वितीया पाणिनि\-विरुद्धेव। विचारे कृतेऽत्र \textcolor{red}{प्रविश्यताम्‌} इतिघटक\-धातुरकर्मकः। यद्यपि \textcolor{red}{विश्‌}\-धातुः (\textcolor{red}{विशँ प्रवेशने} धा॰पा॰~१४२४) सकर्मकः सर्व\-जन\-विदितः कथमत्राकर्मकतेत्यपेक्षायां तत्रैव बाल\-मनोरमायामात्मनेपद\-प्रक्रिया\-प्रकरणे दीक्षित\-महाभागैरेका कारिका निरटङ्कि यत्कस्यां कस्यां परिस्थितावकर्मिकाः क्रिया भवन्ति यथा~–\end{sloppypar}
\centering\textcolor{red}{धातोरर्थान्तरे वृत्तेर्धात्वर्थेनोपसङ्ग्रहात्।\nopagebreak\\
प्रसिद्धेरविवक्षातः कर्मणोऽकर्मिका क्रिया॥}\nopagebreak\\
\raggedleft{–~बा॰म॰~२६९५, वै॰सि॰कौ॰~२७०१, वा॰प॰~३.७.८८}\\
\begin{sloppypar}\hyphenrules{nohyphenation}\justifying\noindent यदा धातोरर्थान्तरं भवति तदाऽकर्मता। यथा \textcolor{red}{वहँ प्रापणे} (धा॰पा॰~१००४) इत्यस्य \textcolor{red}{वहँ स्यन्दने}। अयं \textcolor{red}{भारं वहति} इत्यत्र सकर्मकः \textcolor{red}{नदी वहति} इत्यत्र अकर्मकः। यदा धात्वर्थेनैव सङ्ग्रहो भवति तदैवाकर्मकः। यथा \textcolor{red}{जीवनं धारयति} इत्यस्य \textcolor{red}{जीवति} इत्यत्रोपसङ्ग्रहः। प्रसिद्धेरप्यकर्मकता। यथा \textcolor{red}{मेघो वर्षति} अत्र जल\-रूपस्य कर्मणः प्रसिद्धिस्तस्मादकर्मकः। एवमेव कर्मणोऽविवक्षातोऽकर्मकता। तस्मादत्र कर्मणो न विवक्षा। तस्मादकर्मक\-\textcolor{red}{विश्‌}\-धातोर्भावे लोड्लकारे त\-प्रत्ययः। एवं \textcolor{red}{दीक्षाम्‌} इत्यस्य \textcolor{red}{आश्रित्य} इत्यध्याहृता क्रिया। तस्मादाश्रयानुकूल\-व्यापारस्य कर्मणः क्त्वा\-प्रत्यय\-स्थानापन्न\-ल्यप्प्रत्यनेनानुक्तत्वम्। तस्य कर्तर्येवं विधानम्। अतोऽनुक्ते कर्मणि द्वितीया पाणिन्यनुकूला। अतो \textcolor{red}{दीक्षां प्रविश्यताम्‌} अयं प्रयोगः सुकरः। अथवा \textcolor{red}{गण\-कार्यमनित्यम्‌} (प॰शे॰~९३.३) इति नियमात् \textcolor{red}{प्र}\-पूर्वक\-\textcolor{red}{विश्‌}\-धातुर्दैवादिकः कल्प्यताम्।\footnote{\textcolor{red}{बहुलमेतन्निदर्शनम्‌} (धा॰पा॰ ग॰सू॰~१९३८) \textcolor{red}{आकृतिगणोऽयम्‌} (धा॰पा॰ ग॰सू॰~१९९२) \textcolor{red}{भूवादिष्वेतदन्तेषु दशगणीषु धातूनां पाठो निदर्शनाय तेन स्तम्भुप्रभृतयः सौत्राश्चुलुम्पादयो वाक्यकारीयाः प्रयोगसिद्धा विक्लवत्यादयश्च} (मा॰धा॰वृ॰~१०.३२८) इत्यनुसारमाकृति\-गणत्वाद्दिवादि\-गण ऊह्योऽयं धातुः।} ततश्च \textcolor{red}{दिवादिभ्यः श्यन्‌} (पा॰सू॰~३.१.६९) इत्यनेन प्र\-पूर्वक\-\textcolor{red}{विश्‌}\-धातोः \textcolor{red}{श्यन्‌} विकरणः। एवं \textcolor{red}{हे मुने तां पूर्व\-निर्दिष्टां दीक्षां त्वं प्रविश्य प्रविष्टो भव} इति नेयं कर्म\-वाच्य\-क्रियाऽपि तु \textcolor{red}{प्रविश्य} इति कर्तृ\-वाच्य\-क्रिया। इदं नव्यं समाधानम्। अथवा \textcolor{red}{दीक्षां प्रविश्य ताम्‌} इत्यत्र \textcolor{red}{प्रविश्य} इति ल्यबन्त\-प्रयोगः। तस्य च परवर्ति\-श्लोकस्थेन \textcolor{red}{दर्शयस्व} इत्यनेन क्रिया\-पदेनान्वयः। अर्थात् \textcolor{red}{हे महाभाग मुने तां पूर्व\-निर्दिष्टां दीक्षां प्रविश्य कुतस्तौ राक्षसाधमौ दर्शयस्व}।\end{sloppypar}
\section[पक्वफलादिभिः]{पक्वफलादिभिः}
\centering\textcolor{blue}{भोजयित्वा सह भ्रात्रा रामं पक्वफलादिभिः।\nopagebreak\\
पुराणवाक्यैर्मधुरैर्निनाय दिवसत्रयम्॥}\nopagebreak\\
\raggedleft{–~अ॰रा॰~१.५.११}\\
\begin{sloppypar}\hyphenrules{nohyphenation}\justifying\noindent\hspace{10mm} अत्र श्रीरामः फलं भुङ्क्ते विश्वामित्रः प्रेरयतीत्यर्थे \textcolor{red}{विश्वामित्रो रामं पक्व\-फलादीनि भोजयति}। अस्यामेवावस्थायां \textcolor{red}{क्त्वा}\-प्रत्यये \textcolor{red}{पक्वफलादीनि भोजयित्वा} इत्येव सामान्यतः पाणिनीयानुरूपम्। \textcolor{red}{पक्व\-फलादिभिः} इति कथम्। विमर्शे सति \textcolor{red}{विवक्षाधीनानि कारकाणि भवन्ति}\footnote{मूलं मृग्यम्। यद्वा \textcolor{red}{कर्मादीनामविवक्षा शेषः} (भा॰पा॰सू॰~२.३.५०, २.३.५२, २.३.६७) इत्यस्य तात्पर्यमिदम्।} इति नियमेनात्र करणत्व\-विवक्षा। करणं हि \textcolor{red}{साधकतमं करणम्‌} (पा॰सू॰~१.४.४२) इति पाणिनीय\-सूत्रानुसारं क्रिया\-सिद्धौ प्रकृष्टोप\-कारकं कारकम्। वाक्य\-पदीये च करण\-लक्षणमित्थम्~–\end{sloppypar}
\centering\textcolor{red}{क्रियायाः परिनिष्पत्तिर्यद्व्यापारादनन्तरम्।\nopagebreak\\
विवक्ष्यते यदा यत्र करणं तत्तदा स्मृतम्॥}\nopagebreak\\
\raggedleft{–~वा॰प॰~३.७.९०}\\
\begin{sloppypar}\hyphenrules{nohyphenation}\justifying\noindent अतः \textcolor{red}{कर्तृ\-करणयोस्तृतीया} (पा॰सू॰~२.३.१८) इति सूत्रेण तृतीया। यद्वा \textcolor{red}{हेतौ} (पा॰सू॰~२.३.२३) इति सूत्रेण तृतीया पक्व\-फलादौ हेतुत्व\-विवक्षणात्। सिद्धान्त\-कौमुद्यां हेतु\-करणयोरन्तर\-प्रतिपादन\-पुरः\-सरे लक्षणे व्याचष्ट दीक्षितो यत् \textcolor{red}{द्रव्यादि\-साधारणं निर्व्यापार\-साधारणं च हेतुत्वम्। करणत्वं तु क्रियामात्र\-विषयं व्यापारनियतं च} (वै॰सि॰कौ॰~५६८) इति। अतः तृतीयायां नानुपपत्तिः। यद्वा \textcolor{red}{प्रकृत्यादिभ्य उपसङ्ख्यानम्‌} (वा॰~२.३.१८) इति वार्त्तिक\-बलेन तृतीया। सा चाभेदे।\footnote{\textcolor{red}{धान्येन धनवान्‌} (भा॰पा॰सू॰~२.१.३०) इतिवत्। \textcolor{red}{मेधया तद्वान् धनेन धनवान् इत्यादि में प्रकृत्यादि होने से अभेद में तृतीया है} इति व्याकरण\-चन्द्रोदयस्य प्रथम\-खण्डे कारक\-प्रकरणे ४६तमे पृष्ठे चारुदेव\-शास्त्रिणः (मोतीलाल बनारसीदास, {\englishfont ISBN 978-81-2082-518-5})।} प्रकृत्यादिश्चाऽकृति\-गणः।\footnote{\textcolor{red}{“प्रकृत्यादिभ्य” इति। आकृतिगणोऽयम्। तेन “नाम्ना सुतीक्ष्णश्चरितेन दान्तः” (र॰वं॰~१३.४१) इत्यादि सिद्धम्‌} (त॰बो॰~४९६)।} पक्व\-फलाद्यभिन्नं भोजनं श्रीरामं भोजयित्वेति तात्पर्यम्।\end{sloppypar}
\section[यत्राहल्यास्थिता तपः]{यत्राहल्यास्थिता तपः}
\centering\textcolor{blue}{गौतमस्याश्रमं पुण्यं यत्राहल्यास्थिता तपः।\nopagebreak\\
दिव्यपुष्पफलोपेतपादपैः परिवेष्टितम्॥}\nopagebreak\\
\raggedleft{–~अ॰रा॰~१.५.१५}\\
\begin{sloppypar}\hyphenrules{nohyphenation}\justifying\noindent\hspace{10mm} अत्र \textcolor{red}{स्था}\-धातुर्गति\-निवृत्तौ (\textcolor{red}{ष्ठा गति\-निवृत्तौ} धा॰पा॰~९२८)। अस्यैव \textcolor{red}{क्त}\-प्रत्ययान्तं रूपमिदम्। अत्र तप इति वैषयिक आधारः। अत एव \textcolor{red}{आधारोऽधिकरणम्‌} (पा॰सू॰~१.४.४५) इत्यनेनाधिकरण\-सञ्ज्ञा। ततश्च \textcolor{red}{सप्तम्यधिकरणे च} (पा॰सू॰~१.४.४५) इत्यनेन सप्तमी सङ्गता। \textcolor{red}{तपसि स्थिता} इत्यनेन भवितव्यं किन्तु \textcolor{red}{यत्राऽहल्या} इत्यत्र \textcolor{red}{आकारः} प्रश्लिष्टः। स च \textcolor{red}{आश्रित्य} इत्यस्य सूचकः। अर्थात् \textcolor{red}{तप आश्रित्य स्थिता अहल्या}। यद्वाऽत्राधिरुपसर्गः स च \textcolor{red}{विनाऽपि प्रत्ययं पूर्वोत्तर\-पद\-लोपो वक्तव्यः} (वा॰~५.३.८३) इति वार्त्तिकेन लुप्तत्वात्सम्प्रति न दृश्यते किन्तु पूर्वमासीत्। तत्पूर्वकस्य स्था\-धातोर्योगे \textcolor{red}{तपः अधिष्ठिता} इति स्थिते \textcolor{red}{अधिशीङ्स्थासां कर्म} (पा॰सू॰~१.४.४६) इत्यनेन कर्म\-सञ्ज्ञायां \textcolor{red}{कर्मणि द्वितीया} (पा॰सू॰~२.३.२) इत्यनेन द्वितीया विभक्तौ पश्चादधीत्यस्य लोपे द्वितीया न निवर्तते। \textcolor{red}{जात\-संस्कारो न निवर्तते} इति परिभाषा\-बलेन। अथवा \textcolor{red}{तपःस्थिता} इत्येकं समस्तं पदम्। अत्र \textcolor{red}{सप्तमी शौण्डैः} (पा॰सू॰~२.१.४०) इति सूत्रेण \textcolor{red}{तपसि स्थिता} इति विग्रहे लौकिके \textcolor{red}{तपस् ङि स्थिता सुँ} इत्यलौकिक\-विग्रहे सप्तमी\-तत्पुरुषे \textcolor{red}{कृत्तद्धित\-समासाश्च} (पा॰सू॰~१.२.४६) इत्यनेन प्रातिपदिक\-सञ्ज्ञायां \textcolor{red}{सुपो धातु\-प्रातिपादिकयोः} (पा॰सू॰~२.४.७१) इत्यनेन विभक्ति\-लुकि विभक्त्यादि\-कार्ये \textcolor{red}{तपःस्थिता} इति।\footnote{\textcolor{red}{यत्राहल्या तपःस्थिता} इत्येव मूलपाठ इति तात्पर्यम्।} अथवा \textcolor{red}{तपसि} इति पृथक्पदं \textcolor{red}{स्थिता} इत्यपि पृथगुभे पदे च व्यस्ते तथा \textcolor{red}{सुपां सुलुक्पूर्व\-सवर्णाच्छेयाडाड्यायाजालः} (पा॰सू॰~७.१.३९) इति सूत्रेण छान्दसतया सप्तम्या लुक्। अथवाऽनेनैव सूत्रेण \textcolor{red}{सु}आदेशे पुनर्विसर्गादिः।\footnote{तपस्~ङि~\arrow \textcolor{red}{सुपां सुलुक्पूर्व\-सवर्णाच्छेयाडाड्यायाजालः} (पा॰सू॰~७.१.३९)~\arrow स्वादेशः~\arrow तपस्~सुँ~\arrow \textcolor{red}{स्वमोर्नपुंसकात्} (पा॰सू॰~७.१.२३)~\arrow तपस्~\arrow \textcolor{red}{ससजुषो रुः} (पा॰सू॰~८.२.६६)~\arrow तपरुँ~\arrow तपर्~\arrow \textcolor{red}{खरवसानयोर्विसर्जनीयः} (पा॰सू॰~८.३.१५)~\arrow तपः।} यद्वा \textcolor{red}{तप आस्थिता} इति विग्रहः। \textcolor{red}{आङ्‌}\-पूर्वस्य \textcolor{red}{स्था}\-धातोः करणार्थः। \textcolor{red}{तप आस्थिता} इत्यस्य \textcolor{red}{तपः कुर्वत्यासीत्‌} इत्यर्थः। यथा श्रीमद्भागवते \textcolor{red}{आतिष्ठ तत्तात विमत्सरस्त्वमुक्तं समात्राऽपि यदव्यलीकम्‌} (भा॰पु॰~४.८.१९) इत्यत्र टीकायां श्रीधर\-स्वामिनः \textcolor{red}{आतिष्ठ कुरु} (भा॰पु॰ श्री॰टी॰~४.८.१९) वंशीधराश्च \textcolor{red}{यत्तप उक्तं तत्कुरु} (भा॰पु॰ वं॰टी॰~४.८.१९) इति। \end{sloppypar}
\section[देवराजानम्]{देवराजानम्‌}
\label{sec:devarajanam}
\centering\textcolor{blue}{योनिलम्पट दुष्टात्मन्सहस्रभगवान्भव।\nopagebreak\\
शप्त्वा तं देवराजानं प्रविश्य स्वाश्रमं द्रुतम्॥}\nopagebreak\\
\raggedleft{–~अ॰रा॰~१.५.२६}\\
\begin{sloppypar}\hyphenrules{nohyphenation}\justifying\noindent\hspace{10mm} अत्राहल्याभिमर्श\-सञ्जात\-रोषो गौतमः कृत\-किल्बिषं पुरन्दरं क्रुद्धः शपति। अत्र \textcolor{red}{देवराजानम्‌} इति प्रयोगः कथम्। यतो हि \textcolor{red}{देवानां राजा} इति विग्रहे षष्ठी\-तत्पुरुषे \textcolor{red}{राजाऽहस्सखिभ्यष्टच्‌} (पा॰सू॰~५.४.९१) इत्यनेन टच्प्रत्यये \textcolor{red}{चुटू} (पा॰सू॰~१.३.७) इत्यनेन टकारेत्सञ्ज्ञायां लोपे चकारस्याप्यनुबन्ध\-कार्ये भत्वात् \textcolor{red}{अन्‌} इत्यस्य लोपेऽमि \textcolor{red}{देवराजम्‌}।\footnote{देवानां राजा देवराजस्तम्~\arrow देव~आम्~राजन्~अम्~\arrow \textcolor{red}{सुपो धातुप्रातिपदिकयोः} (पा॰सू॰~२.४.७१)~\arrow देव~राजन्~अम्~\arrow देवराजन्~अम्~\arrow \textcolor{red}{राजाऽहस्सखिभ्यष्टच्‌} (पा॰सू॰~५.४.९१)~\arrow देवराजन्~टच्~अम्~\arrow देवराजन्~अ~अम्~\arrow \textcolor{red}{यचि भम्‌} (पा॰सू॰~१.४.१८)~\arrow भसञ्ज्ञा~\arrow \textcolor{red}{नस्तद्धिते} (पा॰सू॰~६.४.४४)~\arrow देवराज्~अ~अम्~\arrow देवराज~अम्~\arrow \textcolor{red}{अमि पूर्वः} (पा॰सू॰~६.१.१०७)~\arrow देवराजम्।} परञ्च विचारे कृत इदमपि साधु। \textcolor{red}{साधुत्वञ्चाऽत्र वृत्त्यप्रवृत्त\-नित्य\-विध्युद्देश्यतावच्छेदकतानाक्रान्तत्वम्‌}। एवं हि \textcolor{red}{देवानां राजा} इति विग्रहेऽपि कथं न टच् प्रत्यय इत्यपेक्षायां समासान्त\-प्रत्यया अनित्या इत्येव समाधानम्।\footnote{\pageref{sec:sthapya}तमे पृष्ठे \ref{sec:sthapya} \nameref{sec:sthapya} इति प्रयोगस्य विमर्शं पश्यन्तु~– “समासान्त\-प्रत्यय\-प्रकरणं ह्यनित्यम्। प्रमाणं चात्र \textcolor{red}{यचि भम्‌} (पा॰सू॰~१.४.१८) इति सूत्रम्। अत्र \textcolor{red}{यश्चाच्च यच्‌} इति समाहार\-द्वन्द्वः। इह \textcolor{red}{द्वन्द्वाच्चु\-दषहान्तात्समाहारे} (पा॰सू॰~५.४.१०६) इत्यनेन चान्तत्वाट्टच्प्रत्ययः प्रयोक्तव्य आसीत्। तस्मिन् प्रयुक्ते \textcolor{red}{यचे भम्‌} इति स्यात्। यतो न प्रयुक्तोऽतः समासान्त\-प्रत्यस्यानित्यता ज्ञायते।”} \end{sloppypar}
\begin{sloppypar}\hyphenrules{nohyphenation}\justifying\noindent\hspace{10mm} यद्वा \textcolor{red}{राजृँ दीप्तौ} (धा॰पा॰~८२२) इत्यस्माद्धातोः \textcolor{red}{राजनं राट्‌} इति विग्रहे भावे क्विप्।\footnote{\textcolor{red}{सम्पदादिभ्‍यः क्विप्‌} (वा॰~३.३.१०८) इत्यनेन।} तस्य सर्वापहारि\-लोपः। पश्चात् \textcolor{red}{व्रश्च\-भ्रस्ज\-सृज\-मृज\-यज\-राज\-भ्राजच्छशां षः} (पा॰सू॰~८.२.३६) इति सूत्रेण मूर्धन्य\-षकारो \textcolor{red}{झलां जशोऽन्ते} (पा॰सू॰~८.२.३९) इत्यनेन जश्त्वं \textcolor{red}{वाऽवसाने} (पा॰सू॰~८.४.५६) इत्यनेन वैकल्पिक\-चर्त्वमित्थं \textcolor{red}{देवराट्‌} इति निष्पन्न\-प्रयोग\-स्थितिः। किन्तु \textcolor{red}{राजनं राट्‌} इति विग्रहे क्विप्प्रत्ययान्तस्य \textcolor{red}{राज्‌} शब्दस्य \textcolor{red}{देव} शब्देन सह समासे तेन च \textcolor{red}{देव\-राजाऽऽसमन्तादनिति} देव\-शासनेन निश्वसिति पद\-लोलुपतया सम्मानाकाङ्क्षिततया सुखं निश्वसितीति विग्रहे \textcolor{red}{आङ्‌}\-उपसर्ग\-पूर्वकात् \textcolor{red}{अन्‌}\-धातोः (\textcolor{red}{अनँ प्राणने}, धा॰पा॰~१०७०) \textcolor{red}{नन्दि\-ग्रहि\-पचादिभ्यो ल्युणिन्यचः} (पा॰सू॰~३.१.१३४) इत्यनेन \textcolor{red}{अच्‌}\-प्रत्ययः पश्चात् \textcolor{red}{तृतीया तत्कृतार्थेन गुण\-वचनेन} (पा॰सू॰~२.१.३०) इति सूत्रे \textcolor{red}{तृतीया} इति योग\-विभाग\-बलात् \textcolor{red}{देवराजा} इति शब्दस्य \textcolor{red}{आन} इति शब्देन तृतीया\-तत्पुरुषः। अथवा \textcolor{red}{देवराजे सुर\-शासनायाऽनिति} इति विग्रहे चतुर्थी\-तत्पुरुषः। अथवा \textcolor{red}{देवराज्यानिति} निर्भरतया जीवतीति विग्रहे सप्तमी\-तत्पुरुषः। अथवा \textcolor{red}{देवराज इदं देवराजार्थम्‌} इति विग्रहे \textcolor{red}{अर्थेन नित्य\-समासो विशेष्य\-लिङ्गता चेति वक्तव्यम्‌} (वा॰~२.१.३६) इति वार्त्तिक\-बलेन चतुर्थ्यन्त \textcolor{red}{देवराजे} शब्दस्य \textcolor{red}{अर्थ} शब्देन नित्य\-चतुर्थी\-समासः। पश्चात् \textcolor{red}{देवराजार्थमानिति} इति विग्रहे \textcolor{red}{देवराजार्थ} शब्दस्य \textcolor{red}{आन} शब्देन सह \textcolor{red}{सुप्सुपा} (पा॰सू॰~२.१.४) इति सूत्रेण समासः। एवं शब्दमिममाकृति\-गणत्वाच्छाक\-पार्थिवादि\-गणे मत्वा \textcolor{red}{शाकप्रियः पार्थिवः शाकपार्थिवः} इतिवत् \textcolor{red}{शाक\-पार्थिवादीनां सिद्धय उत्तर\-पद\-लोपस्योप\-सङ्ख्यानम्‌} (वा॰~२.१.६०) इति वार्त्तिकेन \textcolor{red}{अर्थ}शब्दस्य लोपे विभक्ति\-कार्ये \textcolor{red}{देवराजानम्‌} इति पूर्णतया पाणिन्यनुकूलम्। एतादृक्समास\-प्रकारस्तु पस्पशाह्निकेऽन्वमूमुदन्महा\-भाष्यकाराः सामोदम्। तत्रायं विचारः समुपस्थितो यद्यदि सिद्धः शब्दोऽर्थः सम्बन्धश्चेति लोकतो ज्ञायते। यदि लोक एषु प्रमाणं तर्हि किं शास्त्रेण क्रियत इत्यपेक्षायां वार्त्तिकमव\-तारयामासुः प्राञ्जलयः पतञ्जलयः यत् \textcolor{red}{लोकतोऽर्थप्रयुक्ते शब्द\-प्रयोगे शास्त्रेण धर्मनियमः}। धर्म\-नियम\-शब्दे बहवः समास\-प्रकाराः प्रदर्शयाम्बभूविरे। \textcolor{red}{किमिदं धर्मनियम इति। धर्माय नियमो धर्मनियमः। धर्मार्थो वा नियमो धर्मनियमः। धर्मप्रयोजनो वा नियमो धर्मनियमः} (भा॰प॰) इत्यादि। इत्थं \textcolor{red}{देवराजानम्‌} इति त्रि\-मुनि\-सम्मतम्। यद्वा \textcolor{red}{अन्‌} धातुमन्तर्भावित\-णिजन्तार्थमङ्गीकृत्य \textcolor{red}{देव\-राजं देव\-शासनमनित्यानयति} श्वासयतीति भावो निज\-प्रताप\-बलेन जीवयतीति हार्दं विग्रहेऽस्मिन्। \textcolor{red}{तत्रोपपदं सप्तमीस्थम्‌} (पा॰सू॰~३.१.९२) इत्यनेन उपपद\-सञ्ज्ञायां \textcolor{red}{कर्मण्यण्‌} (पा॰सू॰~३.२.१) इत्यनेन \textcolor{red}{अण्‌} प्रत्यये \textcolor{red}{अत उपधायाः} (पा॰सू॰~७.२.११६) इत्यनेन वृद्धौ \textcolor{red}{उपपदमतिङ्‌} (पा॰सू॰~२.२.१९) इति सूत्र\-बलेन समासे विभक्ति\-कार्ये \textcolor{red}{देवराजानम्‌} इति सम्यक्सिद्धम्।\end{sloppypar}
\section[पुलकाङ्कितसर्वाङ्गा]{पुलकाङ्कितसर्वाङ्गा}
\centering\textcolor{blue}{उत्थाय च पुनर्दृष्ट्वा रामं राजीवलोचनम्।\nopagebreak\\
पुलकाङ्कितसर्वाङ्गा गिरा गद्गदयैलत॥}\nopagebreak\\
\raggedleft{–~अ॰रा॰~१.५.४२}\\
\begin{sloppypar}\hyphenrules{nohyphenation}\justifying\noindent\hspace{10mm} श्रीराम\-चरणारविन्द\-रजः\-संस्पर्श\-लब्ध\-ललित\-ललना\-शरीरा कलित\-लोचन\-नीरा धीरा विगत\-शल्याऽहल्या कौसल्या\-सुतं श्रीरामं पुलक\-पूर्णाङ्गी स्तौति। अत्रैव \textcolor{red}{पुलकाङ्कित\-सर्वाङ्गा} इति प्रयोगोऽपाणिनीय इव। \textcolor{red}{पुलकेन अङ्कितानि सर्वाणि अङ्गानि यस्याः} इति विग्रहे \textcolor{red}{अनेकमन्य\-पदार्थे} (पा॰सू॰~२.२.२४) इत्यनेन चतुष्पदे बहुव्रीहौ
\textcolor{red}{स्वाङ्गाच्चोप\-सर्जनादसंयोगोपधात्‌} (पा॰सू॰~४.१.५४) इत्यस्योपरि \textcolor{red}{अङ्ग\-गात्र\-कण्ठेभ्य इति वक्तव्यम्} (का॰वृ॰वा॰~४.१.५४) इत्यनेन प्राप्तः \textcolor{red}{ङीष्‌} दुर्वार एव।\footnote{\textcolor{red}{अङ्ग\-गात्र\-कण्ठेभ्य इति वक्तव्यम्} इति वार्त्तिकं \textcolor{red}{स्वाङ्गाच्चोप\-सर्जनादसंयोगोपधात्‌} (पा॰सू॰~४.१.५४) इति सूत्रे काशिकायां पठितम्। \textcolor{red}{भाष्यादर्शनादप्रमाणमिदम्} इति \textcolor{red}{नासिकोदरौष्ठ\-जङ्घादन्त\-कर्णशृङ्गाच्च} (पा॰सू॰~४.१.५५) सूत्रे भट्टोजि\-दीक्षिताः~– \textcolor{red}{अत्र वृत्तिः। अङ्गगात्र\-कण्ठेभ्य इति वक्तव्यम्। स्वङ्गी स्वङ्गेत्यादि। एतच्चानुक्त\-समुच्चयार्थेन चकारेण सङ्ग्राह्यमिति केचित्। भाष्याद्यनुक्तत्वादप्रमाणमिति प्रामाणिकाः} (वै॰सि॰कौ॰~५११)। बाल\-मनोरमायामपि \textcolor{red}{एवञ्च तन्वङ्गी सुगात्री कलकण्ठी इत्यपभ्रंशा एवेति भावः} (बा॰म॰~५११)। परन्त्वेतेन \textcolor{red}{अनवद्याङ्गि} (म॰भा॰~३.६४.७२) \textcolor{red}{अवनताङ्गि} (कु॰स॰~५.८६) इत्यादयः शिष्ट\-प्रयोगा न सङ्गच्छन्ते। अत एव पदमञ्जर्यां हरदत्ताः~– \textcolor{red}{अङ्गगात्रेत्यादि भाष्येऽनुक्तमप्येतत्प्रयोग\-बाहुल्याद्वृत्तिकारेणोक्तम्} (प॰म॰~४.१.५४)।}
एवञ्च \textcolor{red}{पुलकाङ्कित\-सर्वाङ्गा} इत्यत्र \textcolor{red}{टाप्‌} कथमिति चेत्। उच्यते। अत्र हि सिद्धान्त\-कौमुदी\-वर्णनम्~– \textcolor{red}{स्वाङ्गं त्रिधा। अद्रवं मूर्तिमत्स्वाङ्गं प्राणिस्थमविकारजम्‌} (वै॰सि॰कौ॰~५१०)।\footnote{अस्य भाष्ये मूलम्~– \textcolor{red}{किं स्वाङ्गं नाम। अद्रवं मूर्तिमत्स्वाङ्गं प्राणिस्थमविकारजम्। अतत्स्थं तत्र दृष्टं च तस्य चेत्तत्तथा युतम्॥} (भा॰पा॰सू॰~४.१.५४)।} तथाऽपि \textcolor{red}{ङीष्‌}\-भावो वैकल्पिकः। \textcolor{red}{स्वाङ्गाच्चोप\-सर्जनादसंयोगोपधात्‌} (पा॰सू॰~४.१.५४) इति सूत्रं\footnote{तेन \textcolor{red}{अङ्ग\-गात्र\-कण्ठेभ्य इति वक्तव्यम्} इति वार्त्तिकं च।} हि वैकल्पिकं \textcolor{red}{ङीष्‌}\-प्रत्ययं करोति।\footnote{\textcolor{red}{वाग्रहणम् अनुवर्तते। स्वाङ्गं यदुपसर्जनमसंयोगोपधं तदन्तात्प्राति\-पदिकात्स्त्रियां वा ङीष् प्रत्ययो भवति} (का॰वृ॰~४.१.५४)। \textcolor{red}{असंयोगोपधमुपसर्जनं यत्स्वाङ्गं तदन्ताददन्तात्प्राति\-पदिकाद्वा ङीष्} (वै॰सि॰कौ॰~५११)। \textcolor{red}{असंयोगोपधमुपसर्जनं यत्स्वाङ्गं तदन्ताददन्तात् ङीष् वा स्यात्} (ल॰सि॰कौ॰~१२६८)।} यद्यपि~–\end{sloppypar}
\centering\textcolor{red}{अद्य प्रभृत्यवनताङ्गि तवास्मि दासः\nopagebreak\\
क्रीतस्तपोभिरिति वादिनि चन्द्रमौलौ।}\nopagebreak\\
\raggedleft{–~कु॰स॰~५.८६}\\
\begin{sloppypar}\hyphenrules{nohyphenation}\justifying\noindent इति कुमारसम्भवे कालिदासः \textcolor{red}{ङीष्‌}\-प्रत्ययान्तमेवाश्रयति।\footnote{एवमेव भारते वनपर्वणि दमयन्तीं प्रति महात्मनः~– \textcolor{red}{ब्रूहि सर्वानवद्याङ्गि का त्वं किं च चिकीर्षसि} (म॰भा~३.६४.७२)।} अत्र कथं नाऽश्रित इति चेत्। कवीनां कामचारः। यद्वा विकल्प\-बुद्धिमतीमहल्यां वर्णयन् विकल्प\-पक्षमेवाश्रयति।\footnote{यद्वा भाष्येऽनुक्तत्वात् \textcolor{red}{अङ्ग\-गात्र\-कण्ठेभ्य इति वक्तव्यम्} इत्यप्रमाणम्। ततः संयोगोपधात् \textcolor{red}{स्वाङ्गाच्चोप\-सर्जनादसंयोगोपधात्‌} (पा॰सू॰~४.१.५४) इत्यस्याप्रवृत्तौ \textcolor{red}{अजाद्यतष्टाप्‌} (पा॰सू॰~४.१.४) इत्यनेन टाबेव।} यद्वा नायं स्त्री\-प्रत्ययान्तोऽपि तु कर्मधारयान्मत्वर्थीयः। \textcolor{red}{पुलकेनाङ्कितानि पुलकाङ्कितानि पुलकाङ्कितानि च तानि सर्वाण्यङ्गानि} इति कर्मधारयः। \textcolor{red}{विशेषणं विशेष्येण बहुलम्‌} (पा॰सू॰~२.१.५७) इत्यनेन पुलकाङ्कित\-शब्दस्य विशेषणतया पूर्व\-निपातः। \textcolor{red}{विशेषणत्वं नाम विद्यमानत्वे सति विधेयान्वयित्वे सतीतर\-व्यावर्तकत्वम्‌}। एवं \textcolor{red}{पुलकाङ्कित\-सर्वाङ्गाणि सन्ति यस्याः}~– \textcolor{red}{प्रशस्तानि} इति शेषः~– इति विग्रहे \textcolor{red}{अर्शआदिभ्योऽच्‌} (पा॰सू॰~५.२.१२७) इत्यनेन \textcolor{red}{अच्‌}\-प्रत्यये टापि प्रत्यये\footnote{\textcolor{red}{अजाद्यतष्टाप्‌} (पा॰सू॰~४.१.४) इत्यनेन टाप्।} पुलकाङ्कितसर्वाङ्गा इति दिक्।\footnote{बहुव्रीहौ “प्रशस्तानि” इत्यार्थस्याम्भवात् \textcolor{red}{न कर्मधारयान्मत्वर्थीयो बहुव्रीहिश्चेत्तदर्थ\-प्रतिपत्तिकरः} इति नियमस्य न प्रसरः।}\end{sloppypar}
\section[रमणीयदेहिनम्]{रमणीयदेहिनम्‌}
\centering\textcolor{blue}{मर्त्यावतारे मनुजाकृतिं हरिं रामाभिधेयं रमणीयदेहिनम्।\nopagebreak\\
धनुर्धरं पद्मविशाललोचनं भजामि नित्यं न परान्भजिष्ये॥}\nopagebreak\\
\raggedleft{–~अ॰रा॰~१.५.४६}\\
\begin{sloppypar}\hyphenrules{nohyphenation}\justifying\noindent\hspace{10mm} शाप\-मुक्ता भक्ति\-युक्ता विगत\-शल्याऽहल्या कौशल्यानन्द\-वर्धनं स्तुवती\footnote{अत्र नुमागमो न। \textcolor{red}{ष्टुञ् स्तुतौ} (धा॰पा॰~१०४३)~\arrow \textcolor{red}{धात्वादेः षः सः} (पा॰सू॰~६.१.६४)~\arrow \textcolor{red}{निमित्तापाये नैमित्तिकस्याप्यपायः}~\arrow स्तु~\arrow \textcolor{red}{वर्तमाने लँट्‌} (पा॰सू॰~३.२.१२३)~\arrow स्तु~लँट्~\arrow \textcolor{red}{लटः शतृ\-शानचावप्रथमा\-समानाधिकरणे} (पा॰सू॰~३.२.१२४)~\arrow स्तु~शतृँ~\arrow \textcolor{red}{तिङ्शित्सार्वधातुकम्‌} (पा॰सू॰~३.४.११३)~\arrow \textcolor{red}{सार्वधातुकमपित्‌} (पा॰सू॰~१.२.४)~\arrow ङिद्वत्त्वम्~\arrow स्तु~अत्~\arrow \textcolor{red}{कर्तरि शप्‌} (पा॰सू॰~३.१.६८)~\arrow स्तु~शप्~अत्~\arrow \textcolor{red}{अदिप्रभृतिभ्यः शपः} (पा॰सू॰~२.४.७२)~\arrow स्तु~अत्~\arrow \textcolor{red}{ग्क्ङिति च} (पा॰सू॰~१.१.५)~\arrow गुणनिषेधः~\arrow \textcolor{red}{अचि श्नुधातुभ्रुवां य्वोरियँङुवँङौ} (पा॰सू॰~६.४.७७)~\arrow स्त्~उवँङ्~अत्~\arrow स्त्~उव्~अत्~\arrow स्तुवत्~\arrow \textcolor{red}{उगितश्च}~\arrow स्तुवत्~ङीप्~\arrow स्तुवत्~ई~\arrow स्तुवती~\arrow विभक्तिकार्यम्~\arrow स्तुवती~सुँ~\arrow \textcolor{red}{हल्ङ्याब्भ्यो दीर्घात्सुतिस्यपृक्तं हल्‌} (पा॰सू॰~६.१.६८)~\arrow स्तुवती।} श्रीराम\-विशेषणं \textcolor{red}{रमणीय\-देहिनम्‌} इति\-शब्दं प्रायुङ्क्त। अत्र \textcolor{red}{रमणीयश्चासौ देहश्चेति रमणीय\-देहः}। \textcolor{red}{विशेषणं विशेष्येण बहुलम्‌} (पा॰सू॰~२.१.५७) इति कर्मधारयः। \textcolor{red}{रमणीय\-देहो नित्यत्वेन प्राशस्त्येन वाऽस्ति यस्य स रमणीय\-देही} तं रमणीय\-देहिनम्। विचारणीयमिदं यत् \textcolor{red}{रमणीयो देहो यस्य स रमणीयदेहस्तथाभूतम्‌} इत्थं विग्रहेऽपि रमणीय\-देहवत्त्व\-रूपोऽर्थोऽवगम्येत। कर्मधारयानन्तरं मत्वर्थीयस्तु पाणिनीय\-तन्त्र\-विरुद्धः। \textcolor{red}{न कर्मधारयान्मत्वर्थीयो बहुव्रीहिश्चेत्तदर्थ\-प्रतिपत्ति\-करः} इति नियमेनात्र कर्मधारयादनन्तरं बहुव्रीहिरनुचित इति चेत्सत्यं किन्तु यदि बहुव्रीहौ विवक्षितार्थस्य प्रतीतिः स्यात्तदा कर्मधारयान्मत्वर्थीयो न। परमत्र मत्वर्थीयो नित्ययोगे प्राशस्त्ये च। तथा च कारिकां पेठुर्महाभाष्य\-कारा यत्~–\end{sloppypar}
\centering\textcolor{red}{भूमनिन्दाप्रशंसासु नित्ययोगेऽतिशायने।\nopagebreak\\
सम्बन्धेऽस्तिविवक्षायां भवन्ति मतुबादयः॥}\nopagebreak\\
\raggedleft{–~भा॰पा॰सू॰~५.२.९४}\\
\begin{sloppypar}\hyphenrules{nohyphenation}\justifying\noindent\hspace{10mm} भगवतो रमणीयो देहो नित्यः। कौशल्यायाः समक्षन्तु केवलं प्रकटस्तथा वाल्मीकीयेऽपि \textcolor{red}{कौसल्याजनयद्रामम्‌} (वा॰रा॰~१.१८.१०) इत्येव लिखितं न तु \textcolor{red}{बालम्‌} इत्यनेन पुराऽपि राम आसीदित्येव सूच्यते। अध्यात्म\-रामायणे तु शङ्ख\-चक्र\-गदा\-पद्म\-वन\-माला\-विभूषितं चतुर्भुज\-रूपमेवादर्शयत्।\footnote{\textcolor{red}{शङ्ख\-चक्र\-गदा\-पद्म\-वन\-माला\-विराजितः} (अ॰रा॰~१.३.१७)।} श्रीमानसे तु मनु\-शतरूपा\-समक्षं शर\-चाप\-युक्त\-द्वि\-भुजरूप एवाऽगच्छत्। भगवतो द्विभुजं रूपं शाश्वतं चेति श्रुति\-पुराण\-शास्त्र\-सम्मतम्। भागवतेऽपि ब्रह्मा कथयति~–\end{sloppypar}
\centering\textcolor{red}{अस्याऽपि देव वपुषो मदनुग्रहस्य स्वेच्छामयस्य न तु भूतमयस्य कोऽपि।}\nopagebreak\\
\raggedleft{–~भा॰पु॰~१०.१४.२}\\
\begin{sloppypar}\hyphenrules{nohyphenation}\justifying\noindent तुलसीदासोऽपि साटोपं कथयति~–\end{sloppypar}
\centering\textcolor{red}{चिदानन्दमय देह तुम्हारी। बिगत बिकार जान अधिकारी॥}\footnote{एतद्रूपान्तरम्–\textcolor{red}{भवदीयं चिदानन्द\-मयमस्ति कलेवरम्। तथा विकार\-रहितं विजानान्त्यधि\-कारिणः॥} (मा॰भा॰~२.१२७.५)।}\nopagebreak\\
\raggedleft{–~रा॰च॰मा॰~२.१२७.५}\\
\begin{sloppypar}\hyphenrules{nohyphenation}\justifying\noindent इति। इत्थं नित्य\-सिद्धस्य भगवतो देहस्य प्रतीतिर्बहु\-व्रीहावसम्भवान्मत्वर्थीयमन्तरेण प्राशस्त्यमपि नैवावगन्तुं शक्यते। अतोऽसत्यां तादृशार्थ\-प्रतीतौ मत्वर्थीयो वरीयान्। यद्वा \textcolor{red}{दिहँ उपचये} (धा॰पा॰~१०१५) इत्यस्माद्धातोस्ताच्छील्ये णिनिः। तथा च \textcolor{red}{रमणीयं देग्धुं तच्छीलः} इति विग्रहे कर्म\-भूत\-रमणीयोपपदे \textcolor{red}{सुप्यजातौ णिनिस्ताच्छील्ये} (पा॰सू॰~३.२.७८) इत्यनेन णिनि\-प्रत्ययेऽनुबन्ध\-कार्ये गुणेऽमि \textcolor{red}{रमणीय\-देहिनम्‌} इति।\footnote{रमणीयं देग्धुं तच्छीलो रमणीय\-देही तं रमणीय\-देहिनम्। रमणीय~अम्~दिह्~\arrow \textcolor{red}{सुप्यजातौ णिनिस्ताच्छील्ये} (पा॰सू॰~३.२.७८)~\arrow रमणीय~अम्~दिह्~णिनिँ~\arrow रमणीय~अम्~दिह्~इन्~\arrow \textcolor{red}{पुगन्त\-लघूपधस्य च} (पा॰सू॰~७.३.८६)~\arrow \textcolor{red}{रमणीय~अम्~देह्~इन्‌}~\arrow \textcolor{red}{कृत्तद्धित\-समासाश्च} (पा॰सू॰~१.२.४६)~\arrow प्रातिपदिक\-सञ्ज्ञा~\arrow \textcolor{red}{सुपो धातु\-प्रातिपदिकयोः} (पा॰सू॰~२.४.७१)~\arrow रमणीय~देह्~इन्~\arrow रमणीयदेहिन्। विभक्तिकार्ये रमणीयदेहिन् अम्~\arrow रमणीयदेहिनम्।} अथवा \textcolor{red}{रमणीया देहिनो मत्स्यादयश्चतुर्विंशतिरवतारा यस्य स रमणीय\-देही तं रमणीय\-देहिनम्‌}। श्रीरामस्यैव सर्वावतारित्वात्।\footnote{यथा \textcolor{red}{सर्वेषामवताराणामवतारी रघूत्तमः। रामपादनखज्योत्स्ना परब्रह्मेति गीयते॥} (अग॰सं॰) \textcolor{red}{तस्मिन्साकेतलोके विधिहरहरिभिः सन्ततं सेव्यमाने दिव्ये सिंहासने स्वे जनकतनयया राघवः शोभमानः। युक्तो मत्स्यैरनेकैः करिभिरपि तथा नारसिंहैरनन्तैः कूर्मैः श्रीनन्दनन्दैर्हयगलहरिभिर्नित्यमाज्ञोन्मुखैश्च॥ यज्ञः केशववामनौ नरवरो नारायणो धर्मजः श्रीकृष्णो हलधृक् तथा मधुरिपुः श्रीवासुदेवोऽपरः। एते नैकविधा महेन्द्रविधयो दुर्गादयः कोटिशः श्रीरामस्य पुरो निदेशसुमुखा नित्यास्तदीये पदे॥} (बृ॰ब्र॰सं॰) \textcolor{red}{नारायणोऽपि रामांशः शङ्खचक्रगदाधरः} (वा॰सं॰) इत्यादिषु स्पष्टम्। एते वाल्मीकीय\-रामायण\-शिरोमणि\-टीकाया मङ्गलाचरणे शिवसहाय\-महाभागैरुद्धृताः।} अथवा \textcolor{red}{कृत्य\-ल्युटो बहुलम्‌} (पा॰सू॰~३.१.११३) इति सूत्रेण \textcolor{red}{रमन्ते सर्वाकृतयो येषु ते रमणीयाः} इति विग्रहे \textcolor{red}{रमुँ क्रीडायाम्‌} (धा॰पा॰~८५३) इत्यस्माद्धातोरधिकरणे \textcolor{red}{अनीयर्‌} प्रत्ययः। एवं \textcolor{red}{रमणीया देहिनो जीवात्मानो यस्य स रमणीयदेही तम्‌}। षष्ठ्यर्थश्चांशांशि\-भाव\-रूपः। जीवात्मा परमात्मनोंऽश इति सार्वजनीनत्वात्। \textcolor{red}{ममैवांशो जीव\-लोके जीवभूतः सनातनः} (भ॰गी॰~१५.७) इति गीतोक्तेः। अथवा \textcolor{red}{रमन्ते इति रमणीयाः}। \textcolor{red}{कृत्य\-ल्युटो बहुलम्‌} (पा॰सू॰~३.१.११३) इत्यनेन कर्तरि \textcolor{red}{अनीयर्‌}। \textcolor{red}{नित्यं रमणशीला जीवात्मानो देहिनो मुक्त\-नित्या यस्मिन्स रमणीय\-देही तं रमणीय\-देहिनम्‌}।\end{sloppypar}
\centering\textcolor{blue}{यस्मिन् रमन्ते मुनयो विद्ययाऽज्ञानविप्लवे।\nopagebreak\\
तं गुरुः प्राह रामेति रमणाद्राम इत्यपि॥}\nopagebreak\\
\raggedleft{–~अ॰रा॰~१.३.४०}\\
\begin{sloppypar}\hyphenrules{nohyphenation}\justifying\noindent इत्यत्रैव ग्रन्थ उक्तत्वात्। तस्मादिदं पाणिनीयमेव।\end{sloppypar}
\section[तस्मात्ते]{तस्मात्ते}
\centering\textcolor{blue}{योषिन्मूढाहमज्ञा ते तत्त्वं जाने कथं विभो।\nopagebreak\\
तस्मात्ते शतशो राम नमस्कुर्यामनन्यधीः॥}\nopagebreak\\
\raggedleft{–~अ॰रा॰~१.५.५७}\\
\begin{sloppypar}\hyphenrules{nohyphenation}\justifying\noindent\hspace{10mm} \textcolor{red}{ते नमस्कुर्याम्‌} इत्यन्वयः। अत्र \textcolor{red}{उपपद\-विभक्तेः कारक\-विभक्तिर्बलीयसी} (भा॰पा॰सू॰~१.४.९५, २.३.४, ३.१.१९) इति परिभाषा\-बलेन द्वितीयैव सम्भवति। कदाचित्समासाभावः स्यात्तदा विसर्गस्य सत्वं न स्यात्। \textcolor{red}{नमस्पुरसोर्गत्योः} (पा॰सू॰~८.३.४०) इति सूत्रेण साम्प्रतं \textcolor{red}{साक्षात्प्रभृतीनि च} (पा॰सू॰~१.४.७४) इत्यनेन गति\-सञ्ज्ञायां विसर्गस्य सत्वेऽत्रैकार्थी\-भावः। \textcolor{red}{एकार्थी\-भावत्वञ्च पृथगर्थानामेकोपस्थित्या बोध\-जनकत्वम्‌}। यथा \textcolor{red}{राज\-पुरुषः} इत्यादौ समासात्प्राग्व्यस्तावस्थायां \textcolor{red}{राज्ञः} इत्यस्य पृथगर्थः \textcolor{red}{पुरुषः} इत्यस्य च पृथक्। सति समास उभयोरपि सत्ता समाप्ता साम्प्रतं हि \textcolor{red}{स्व\-स्वामि\-भाव\-सम्बन्धावच्छिन्न\-राज\-विशिष्ट\-पुरुषः} इत्येकोऽर्थः। तथा चोक्तं वैयाकरण\-भूषण\-सारे समास\-शक्ति\-निर्णये~–\end{sloppypar}
\centering\textcolor{red}{समासे खलु भिन्नैव शक्तिः पङ्कजशब्दवत्॥}\nopagebreak\\
\raggedleft{–~वै॰भू॰सा॰~५.३१, वै॰सि॰का॰~३१}\\
\begin{sloppypar}\hyphenrules{nohyphenation}\justifying\noindent\hspace{10mm} एवमत्रापि \textcolor{red}{कृ}धातोः (\textcolor{red}{डुकृञ् करणे} धा॰पा॰~१४७२) करणानुकूल\-व्यापारे शक्तिर्नास्ति न च \textcolor{red}{नमस्‌} इति शब्देऽवनत्यर्थ\-बोधिका सति समासे \textcolor{red}{परनिष्ठोत्कृष्टत्व\-विशिष्ट\-स्वनिष्ठापकृष्टत्वानुकूल\-व्यापारो नमस्कार\-पदार्थः}। अत्र च कारक\-विभक्तेर्बलवत्वाद्द्वितीयैव। \textcolor{red}{ते} तुभ्यं \textcolor{red}{नमस्कुर्याम्‌} इति कथम्। अत्रोच्यते। \textcolor{red}{स्वयम्भुवे नमस्कृत्य} (म॰स्मृ॰~१.१, भा॰पु॰~४.६.२) इत्यादिवत् \textcolor{red}{त्वामनुकूलयितुं प्रसादयितुं स्तोतुं वा नमस्कुर्याम्‌} इति \textcolor{red}{क्रियार्थोपपदस्य च कर्मणि स्थानिनः} (पा॰सू॰~२.३.१४) इत्यनेन चतुर्थी। यद्वा \textcolor{red}{चरणौ} इत्यध्याहृत्यावयावयवि\-भाव\-मूलक\-सम्बन्धे षष्ठी।
इति शम्।\end{sloppypar}
\section[कुटुम्बहानिः]{कुटुम्बहानिः}
\centering\textcolor{blue}{पादाम्बुजं ते विमलं हि कृत्वा पश्चात्परं तीरमहं नयामि।\nopagebreak\\
नो चेत्तरी सद्युवती मलेन स्याच्चेद्विभो विद्धि कुटुम्बहानिः॥}\nopagebreak\\
\raggedleft{–~अ॰रा॰~१.६.४}\\
\begin{sloppypar}\hyphenrules{nohyphenation}\justifying\noindent\hspace{10mm} अहल्योद्धारानन्तरं भू\-भार\-हारी हरिर्गङ्गामुत्तर्तुं निषादं नौकां याचते। कैवर्तकः केशवमप्रस्तुत\-प्रशंसाभङ्ग्याऽऽक्षिपन्नाह यत्त्वच्चरण\-कमल\-रजसा शिला नारी\-रूपेण परिणता। तत्स्पर्शेन मम नौकाऽपि नारी मा भूदिति कृत्वा चरणौ क्षालयित्वैव गङ्गा\-पारं नेष्यामि। अत्रैव कथयति \textcolor{red}{स्याच्चेद्विभो विद्धि कुटुम्ब\-हानिः}। अत्र ज्ञानार्थको \textcolor{red}{विद्‌}\-धातुः (\textcolor{red}{विदँ ज्ञाने} धा॰पा॰~१०६४)। स च सकर्मकः। ज्ञानस्य सविषयकत्वाद्विषयतया कर्मणः संस्कृते पाश्चात्य\-भाषायां सङ्कीर्तनात्। कर्मैव पाश्चात्य\-भाषायाम् \textcolor{red}{ऑब्जेक्ट्‌} शब्देन व्यवह्रियते। अतः \textcolor{red}{कुटुम्बहानिम्‌} इति द्वितीयया भवितव्यमिति चेत्सत्यम्। वैयाकरण\-सिद्धान्त\-कौमुद्याः कारके द्वितीया\-विभक्ति\-प्रकरणेऽभिधान\-चर्चायां दीक्षित\-महाभागाः संलिखन्ति यत् \textcolor{red}{अभिधानञ्च प्रायेण तिङ्कृत्तद्धित\-समासैः। तिङ् हरिः सेव्यते। कृत् लक्ष्म्या सेवितः। तद्धितः शतेन क्रीतः शत्यः। समासे प्राप्त आनन्दो यं स प्राप्तानन्दः। क्वचिन्निपातेनाप्यभिधानम्। विष\-वृक्षोऽपि संवर्ध्य स्वयं छेत्तुमसाम्प्रतम्। असाम्प्रतमित्यस्य हि न युज्यते इत्यर्थः} (वै॰सि॰कौ॰~५३७)। अत्र प्रायेणापि शब्देनैव निपातेन कर्मोक्तं मन्यन्तेऽन्यथा \textcolor{red}{संवर्ध्य} इति ल्यबन्तपद\-समभिव्याहारेण \textcolor{red}{विष\-वृक्षः} इत्यत्र द्वितीया दुर्वारेति सर्व\-विदितम्।\footnote{मल्लिनाथोऽपि~– \textcolor{red}{असाम्प्रतमित्यमेन निपातेनाभि\-हितत्वाद्‌वृक्ष इति द्वितीयान्तो न भवत्यनभिहिते कर्मणि द्वितीयाभिधानात्। यथाऽह वामनः~– “निपातेनाप्यभिहिते कर्मणि न विभक्तिः परिगणनस्य प्रायिकत्वात्” इति} (कु॰स॰ स॰व्या॰~२.५५)।} किन्त्वस्मत्सम्प्रदाये त्वस्मद्गुरु\-चरणा \textcolor{red}{असाम्प्रतम्‌}\-घटक\-\textcolor{red}{साम्प्रतम्‌}इति\-निपातेनैवोक्तं कर्म व्यवस्थापयन्ति। अत्रैवार्थ\-बोधन\-सामर्थ्यात्। सा रीतिरत्राप्यनु\-श्रियते। \textcolor{red}{स्याच्चेद्विभो विद्धि कुटुम्ब\-हानिः} इति छन्दश्चतुर्थ\-चरण\-घटक\-\textcolor{red}{चेत्‌}\-इति\-निपातेनैवोक्तं कर्म। अत उक्ते कर्मणि प्रथमा \textcolor{red}{विष\-वृक्षोऽपि संवर्ध्य} (कु॰स॰~२.५५) इतिवत् \textcolor{red}{विद्धि कुटुम्ब\-हानिः} इति।
यद्वा \textcolor{red}{कुटुम्ब\-हानिः} इति पदं \textcolor{red}{विद्धि} इति पदेन नान्वीयताम्। तथा चाविवक्षया कर्मणः \textcolor{red}{विद्धि} इति\-घटक\-\textcolor{red}{विद्‌}\-धातुम् (\textcolor{red}{विदँ ज्ञाने} धा॰पा॰~१०६४) अकर्मकं मत्वा वाक्यं विभिद्य व्याख्येयम्। अर्थात् \textcolor{red}{नौकायां गतायां मम कुटुम्ब\-हानिः स्याद्धे विभ इति त्वं विद्धि जानीहि} इति तात्पर्यम्।
इत्थं नापाणिनीयता।\footnote{\textcolor{red}{पश्य मृगो धावति} इतिवत्क्रिया कर्मेति भावः।}\end{sloppypar}
\section[मे मनःप्रीतिकरौ]{मे मनःप्रीतिकरौ}
\centering\textcolor{blue}{कस्यैतौ नरशार्दूलौ पुत्रौ देवसुतोपमौ।\nopagebreak\\
मनःप्रीतिकरौ मेऽद्य नरनारायणाविव॥}\nopagebreak\\
\raggedleft{–~अ॰रा॰~१.६.९}\\
\begin{sloppypar}\hyphenrules{nohyphenation}\justifying\noindent\hspace{10mm} अत्र लोकाभिरामं श्रीरामं सुमित्रा\-सुताभिरामं दृष्ट्वा
योगिराजो जनको विश्वामित्रं पृच्छति यत् \textcolor{red}{मे मनः\-प्रीति\-करौ एतौ कस्य सुतौ}। अत्र \textcolor{red}{मनसः प्रीति\-करौ} इति समासः। मनः\-शब्दस्य हि मम\-स्थानापन्न\-\textcolor{red}{मे}\-इति\-शब्देन साकाङ्क्षता। कस्य मन इत्यपेक्षायां मे जनकस्य यन्मनस्तस्य प्रीतिकरौ। \textcolor{red}{मे} इति मनः\-शब्दस्य विशेषणम्। इतरस्माद्व्यावर्तकत्वात्तत्र भाष्ये प्रतिपादितं \textcolor{red}{सविशेषणानां वृत्तिर्न वृत्तस्य वा विशेषण\-योगो न} (भा॰पा॰सू॰~२.१.१)। अर्थाद्विशेषणवतः शब्दस्य विशेषणं विहाय पञ्चसु वृत्तिषु कृत्तद्धित\-समासैक\-शेष\-सनाद्यन्त\-धातु\-रूपासु कतमाऽपि वृत्तिर्न भवति। यथा \textcolor{red}{ऋद्धस्य राज्ञः पुरुषः} इत्यत्र ऋद्धस्येति राज्ञ इत्यस्य विशेषणम्। एवं \textcolor{red}{ऋद्धस्य} इति पदं त्यक्त्वा \textcolor{red}{राज्ञः} इति शब्दस्य \textcolor{red}{पुरुषः} इति पदेन कथमपि न समासः।\footnote{\textcolor{red}{न च षष्ठी\-तत्पुरुषादि\-स्थलेऽपि लुप्त\-विभक्ति\-स्मरण एव चेदन्वयबोधस्तदा ‘ऋद्धस्य राजमातङ्गाः’ इत्यादि\-प्रयोगापत्तिः} (व्यु॰वा॰ का॰प्र॰)।} एवं \textcolor{red}{राज्ञः} इत्यस्य \textcolor{red}{पुरुषः} इत्यनेन समासे \textcolor{red}{ऋद्धस्य} इति विशेषणं कथमपि न योजयितुं शक्यते। तस्मात् \textcolor{red}{मे} इत्यस्य विशेषणतया जागरूकतायां \textcolor{red}{मनः} इत्यस्य \textcolor{red}{प्रीति\-करौ} इति शब्देन कथं समासः। \textcolor{red}{सापेक्षमसमर्थवत्‌} इति न्यायात्सामर्थ्य\-विरहात् \textcolor{red}{समर्थः पद\-विधिः} (पा॰सू॰~२.१.१) इति सूत्रस्य प्रसरण\-गन्धोऽपि नेत्यपेक्षायामुच्यते। नित्य\-सापेक्ष\-स्थले नियमोऽयमपोद्यते। \textcolor{red}{चैत्रस्य दास\-भार्या} इतिवत् \textcolor{red}{देवदत्तस्य गुरु\-कुलम्‌} इतिवच्च। मनसो व्यक्ति\-विशेषेण नित्य\-साकाङ्क्षतया व्यक्ताविव तन्निष्ठत्वादस्य नियमस्यापवादः। यद्वा \textcolor{red}{मनः\-प्रीति\-करौ} पृथक्समासः पश्चात् \textcolor{red}{मे} इत्यनेन सह पार्ष्ठिकोऽन्वय इति न विरोधः।\end{sloppypar}
\section[नोदितो मे]{नोदितो मे}
\centering\textcolor{blue}{मखसंरक्षणार्थाय मयाऽऽनीतौ पितुः पुरात्।\nopagebreak\\
आगच्छन् राघवो मार्गे ताटकां विश्वघातिनीम्॥\\
शरेणैकेन हतवान्नोदितो मेऽतिविक्रमः।\nopagebreak\\
ततो ममाश्रमं गत्वा मम यज्ञविहिंसकान्॥}\nopagebreak\\
\raggedleft{–~अ॰रा॰~१.६.११-१२}\\
\begin{sloppypar}\hyphenrules{nohyphenation}\justifying\noindent\hspace{10mm} विश्वामित्रो राम\-पराक्रमं प्रशंसन् कथयति \textcolor{red}{मे नोदितः शरेणैकेन श्रीरामस्ताटकां हतवान्‌}। अत्र \textcolor{red}{अहं राममनोदयम्‌} इति कर्तृ\-वाच्यार्थे \textcolor{red}{मया रामोऽनोद्यत} इति कर्म\-वाच्ये तस्मिन्नेव \textcolor{red}{तयोरेव कृत्यक्तखलर्थाः} (पा॰सू॰~३.४.७०) इति सूत्रानुसारं \textcolor{red}{निष्ठा} (पा॰सू॰~२.२.३६) इत्यनेन कर्मणि \textcolor{red}{क्त}\-प्रत्ययेन कर्मण उक्तत्वादनुक्ते कर्तरि तृतीया\-सम्भवात्\footnote{\textcolor{red}{कर्तृ\-करणयोस्तृतीया} (पा॰सू॰~२.३.१८) इत्यनेन।} \textcolor{red}{मया नोदितः} इत्येव वाच्यम्।
\textcolor{red}{मे} इति तु सम्बन्ध\-विवक्षायां कर्तरि षष्ठी।
\textcolor{red}{मत्सम्बन्धि\-नोदनाश्रयो रामः} इति तात्पर्यम्। यद्वा \textcolor{red}{नोदनं नोदः}। भावघञन्तः।\footnote{\textcolor{red}{भावे} (पा॰सू॰~३.३.१८) इत्यनेन।} स सञ्जातोऽस्य स नोदितः। \textcolor{red}{मे} इत्यस्य \textcolor{red}{राघवः} (अ॰रा॰~१.६.११) इत्यनेन सम्बन्धः। स च गुरु\-शिष्य\-भाव\-मूलको भावना\-पक्षे सेवक\-सेव्य\-भाव\-मूलकश्चेति मीमांसायां \textcolor{red}{तदस्य सञ्जातं तारकादिभ्य इतच्‌} (पा॰सू॰~५.२.३६) इति सूत्रेणाऽकृति\-गणतया तारकादि\-गणे मत्वा \textcolor{red}{नोद}\-शब्दात् \textcolor{red}{इतच्‌}\-प्रत्यय एवं भत्वाट्टिलोपे\footnote{\textcolor{red}{यचि भम्} (पा॰सू॰~१.४.१८) इत्यनेन भत्वम्। \textcolor{red}{यस्येति च} (पा॰सू॰~६.४.१४८) इत्यनेनाकार\-लोपः।} \textcolor{red}{नोदितः}। प्रेरणावाल्लीँला\-शक्त्या मे मम विश्वामित्रस्य शिष्य आराध्य इष्ट\-देवः श्रीराम\-चन्द्रस्ताटकामेकेनैव बाणेन हतवानिति पाणिनीयतयाऽर्थ\-माधुर्यमपि समायातम्।\end{sloppypar}
\section[रामाय दर्शय]{रामाय दर्शय}
\centering\textcolor{blue}{ततः सम्प्रेषयामास मन्त्रिणं बुद्धिमत्तरम्।\nopagebreak\\
जनक उवाच\nopagebreak\\
शीघ्रमानय विश्वेशचापं रामाय दर्शय॥}\nopagebreak\\
\raggedleft{–~अ॰रा॰~१.६.१८}\\
\begin{sloppypar}\hyphenrules{nohyphenation}\justifying\noindent\hspace{10mm} अत्र महर्षि\-विश्वामित्रः श्रीराम\-पराक्रम\-परिचय\-प्रस्तावनया जनकस्य हृदि श्रीरामस्य शैव\-कोदण्ड\-खण्डन\-सामर्थ्य\-विषयक\-विश्वासं सबलमुत्पाद्य \textcolor{red}{रामाय धनुर्दर्शय} इत्यादिशति। अत्र \textcolor{red}{रामाय} इति चतुर्थी विचार\-विषयः। वस्तुतस्तु रामश्चापं पश्यतु त्वं च प्रेरय इत्यर्थे \textcolor{red}{हेतुमति च} (पा॰सू॰~३.१.२६) इत्यनेन ण्यन्त\-\textcolor{red}{दृश्‌}\-धातुर्ज्ञान\-सामान्यार्थकः। एवं च \textcolor{red}{गति\-बुद्धि\-प्रत्यवसानार्थ\-शब्द\-कर्माकर्मकाणामणि कर्ता स णौ} (पा॰सू॰~१.४.५२) इति सूत्रेणाण्यन्तः कर्तृ\-भूतो रामोऽधुना ण्यन्तावस्थायां कर्म स्यात्। \textcolor{red}{रामं दर्शय} इत्येवाऽपातत उचितम्। पुनर्ण्यन्ते सति \textcolor{red}{रामम्‌} इत्येव भवितव्यं \textcolor{red}{रामाय} इति कथं तदुपर्युच्यते। \textcolor{red}{तादर्थ्ये चतुर्थी वाच्या} (वा॰~२.३.१३) इति वार्त्तिकेनात्र चतुर्थी। \textcolor{red}{रामार्थं धनुर्दर्शय} इति तात्पर्यम्। तवानेन किमपि प्रयोजनं न सेत्स्यतीति व्यज्यते। यद्वाऽत्र सम्प्रदानस्य विवक्षा। अर्थादिदं रामाय सम्प्रदेहि स्व\-स्वत्व\-निवर्तनं कुरु। रामः सज्जीकृत्य त्रोटयेद्वा रक्षेद्वा। तव कोऽप्यधिकारो नापेक्षते। एतावत्कालं यावद्धनुः पूजितवानधुना धनुर्धरं पूजय समुद्रं प्राप्य नदीमिव धनुर्धरे ममतां कुरु धनुर्ममतां च त्यजेत्येव विश्वामित्र\-तात्पर्यं प्रतिभाति। यद्वा \textcolor{red}{रामाय दातुं धनुर्दर्शय} इति गम्यमान\-दान\-क्रिया। तद्बलेन सुतरां चतुर्थी। यद्वा \textcolor{red}{रामं परीक्षितुं धनुर्दर्शय} इति \textcolor{red}{क्रियार्थोपपदस्य च कर्मणि स्थानिनः} (पा॰सू॰~२.३.१४) इत्यनेन चतुर्थी। यद्वा \textcolor{red}{रामाय} इति न चतुर्थ्यन्तमपि तु जनकस्य सम्बोधनमिदम्। विश्वामित्राद्रामं सगुणं ब्रह्म श्रुत्वा तमेवात्मना प्रविशन्तं समाधि\-मग्नं श्रीराम\-रूप\-माधुरी\-चोरित\-हृदयं पावके द्रवीभूतं कनकमिव जनकं विलोक्य विश्वामित्रो \textcolor{red}{रामाय} इति शब्देन जनकं सम्बोधयति। \textcolor{red}{अयँ गतौ} (धा॰पा॰~४७४) इति धातुः। गत्यर्थस्य ज्ञानार्थतया प्राप्त्यर्थतया च प्रसिद्धत्वात्\footnote{\textcolor{red}{गत्यर्थका ज्ञानार्थकाः प्राप्त्यर्थका अपि भवन्ति}।} \textcolor{red}{रामं श्रीरामचन्द्रमयते गच्छति प्रविशति जानाति वा स रामायः} इति विग्रहे \textcolor{red}{कर्मण्यण्‌} (पा॰सू॰~३.२.१) इत्यनेन \textcolor{red}{अण्‌}\-प्रत्यये विभक्तिकार्ये \textcolor{red}{रामायः}। तत्सम्बुद्धौ \textcolor{red}{हे रामाय}। \textcolor{red}{हे राम\-तत्त्वज्ञ}। \textcolor{red}{अधुना विश्वस्तः सन्नहङ्कार\-प्रतीकं तमःप्रतीकं वा शिव\-धनुः श्रीरामं दर्शय}। तस्मिन् सज्जीकृते सति यथा श्रीसीता\-राम\-विवाहो विलोक्येत।\end{sloppypar}
\section[दर्शयामास रामाय]{दर्शयामास रामाय}
\centering\textcolor{blue}{दर्शयामास रामाय मन्त्री मन्त्रयतां वरः।\nopagebreak\\
दृष्ट्वा रामः प्रहृष्टात्मा बद्ध्वा परिकरं दृढम्॥}\nopagebreak\\
\raggedleft{–~अ॰रा॰~१.६.२३}\\
\begin{sloppypar}\hyphenrules{nohyphenation}\justifying\noindent\hspace{10mm} अत्रापि \textcolor{red}{रामं दर्शयामास} इति प्रयोक्तव्ये \textcolor{red}{रामाय} इति प्रायुङ्क्त। यतो ह्यत्र सम्प्रदानस्य विवक्षा। \textcolor{red}{रामाय दातुं दर्शयामास} इति तात्पर्यम्।\footnote{\textcolor{red}{दातुम्} इत्यध्याहार्यमिति भावः।} यद्वा तादर्थ्ये चतुर्थी।\footnote{\textcolor{red}{तादर्थ्ये चतुर्थी वाच्या} (वा॰~२.३.१३) इत्यनेन।} \textcolor{red}{रामार्थं दर्शयामास} इति।\footnote{यद्वा पूर्ववत् \textcolor{red}{रामं परीक्षितुं धनुर्दर्शयामास} इति \textcolor{red}{क्रियार्थोपपदस्य च कर्मणि स्थानिनः} (पा॰सू॰~२.३.१४) इत्यनेन चतुर्थी।}\end{sloppypar}
\section[दीयते मे सुता]{दीयते मे सुता}
\centering\textcolor{blue}{दीयते मे सुता तुभ्यं प्रीतो भव रघूत्तम।\nopagebreak\\
इति प्रीतेन मनसा सीतां रामकरेऽर्पयन्॥}\nopagebreak\\
\raggedleft{–~अ॰रा॰~१.६.५४}\\
\begin{sloppypar}\hyphenrules{nohyphenation}\justifying\noindent\hspace{10mm} अत्र \textcolor{red}{मया दीयते} इति प्रयोक्तव्ये \textcolor{red}{मे दीयते} इति प्रयुक्तम्। यतो ह्यत्र सम्बन्ध\-विवक्षायां षष्ठी। स च सम्बन्धः पालक\-पाल्य\-रूपः। यद्वा \textcolor{red}{मे} इत्यस्य सुतया साकमन्वयः। मत्सम्बन्धिनी सुता तुभ्यं दीयते।
इत्थमन्वय\-विपरिणाम आपत्ति\-निरासः।\end{sloppypar}
\section[वसिष्ठाय विश्वामित्राय]{वसिष्ठाय विश्वामित्राय}
\centering\textcolor{blue}{ततोऽब्रवीद्वसिष्ठाय विश्वामित्राय मैथिलः।\nopagebreak\\
जनकः स्वसुतोदन्तं नारदेनाभिभाषितम्॥}\nopagebreak\\
\raggedleft{–~अ॰रा॰~१.६.५८}\\
\begin{sloppypar}\hyphenrules{nohyphenation}\justifying\noindent\hspace{10mm} अत्र \textcolor{red}{वसिष्ठाय विश्वामित्राय} चेति प्रयोगद्वयमपि द्वितीयायां प्रयोक्तव्यमासीत्किन्तु \textcolor{red}{विवक्षाधीनानि कारकाणि भवन्ति}\footnote{मूलं मृग्यम्। यद्वा \textcolor{red}{कर्मादीनामविवक्षा शेषः} (भा॰पा॰सू॰~२.३.५०, २.३.५२, २.३.६७) इत्यस्य तात्पर्यमिदम्।} इति नियममनुसृत्य चतुर्थी प्रयुक्ता। यद्वा \textcolor{red}{वसिष्ठं विश्वामित्रञ्च तोषयितुम्‌} इति \textcolor{red}{क्रियार्थोपपदस्य च कर्मणि स्थानिनः} (पा॰सू॰~२.३.१४) इत्यनेन चतुर्थी। यद्वा \textcolor{red}{स्व\-सुतोदन्तम्‌} इत्यत्र \textcolor{red}{रुचितम्‌} इत्यध्याहार्यम्। सीता\-कथायाः सर्वेभ्योऽपि रुचितत्वौचित्यात्। अतः \textcolor{red}{रुचितं स्व\-सुतोदन्तम्‌}। काभ्याम्। वसिष्ठाय विश्वामित्राय। द्वयोरेवर्षिवर्ययोर्नाम।\footnote{“द्वयोः एव ऋर्षिवर्ययोः” इत्यत्र सम्बन्ध\-विवक्षायां षष्ठी।} कथम्। यतो हि द्वावपि वैदिकावृषी। वसिष्ठो राम\-मन्त्रार्थ\-तत्त्वज्ञो विश्वामित्रश्च सीता\-मन्त्रार्थ\-तत्त्ववित्। इत्थं \textcolor{red}{राममन्त्रे स्थिता सीता सीतामन्त्रे च राघवः}\footnote{मूलं मृग्यम्।} इति प्राचीनोक्तेरुभयोरपि वेद\-मन्त्र\-साक्षात्\-कारित्वम्। विश्वामित्रस्तु साक्षाद्ब्रह्म\-गायत्र्या ऋषिः सीतैव गायत्री\-मन्त्रार्थः। वसिष्ठश्च महा\-मृत्युञ्जय\-मन्त्र\-द्रष्टर्षिर्महा\-मृत्युञ्जयश्च सीताराम\-गुण\-गान\-परः। तथा हि~–\end{sloppypar}
\centering\textcolor{red}{त्र्य॑म्बकं यजामहे सु॒गन्धिं॑ पुष्टि॒वर्ध॑नम्।\nopagebreak\\
उ॒र्वा॒रु॒कमि॑व॒ बन्ध॑नान्मृ॒त्योर्मु॑क्षीय॒ मामृता॑त्॥}\nopagebreak\\
\raggedleft{–~ऋ॰वे॰सं॰~७.५९.१२}\\
\begin{sloppypar}\hyphenrules{nohyphenation}\justifying\noindent अस्यार्थः। तिस्रः कौसल्या\-कैकेयी\-सुमित्रा अरुन्धत्यहल्यानसूयाः कौसल्या\-सुनयना\-शबर्योऽम्बा यस्य स त्र्यम्बको रामचन्द्रः। तं सुगन्धिं सुरभि\-युक्तं पुष्टिवर्धनमनुग्रह\-वर्धन\-कर्तारमुर्वारुकं दूर्वादलमिव श्यामलं श्रीरामचन्द्रं यजामहे यथा मृत्योर्मुक्षीय मुक्तो भवेयं माऽमृतादमृतान्मुक्तो न भवेयम्। यद्वा त्रीण्यम्बकानि सीता\-लक्ष्मण\-भरताख्यानि यस्य स त्र्यम्बको रामस्तं त्र्यम्बकं शेषं समानम्। अत एव द्वाभ्यां सीता\-वृत्तं रोचेतैव। तस्मात् \textcolor{red}{वसिष्ठाय विश्वामित्राय रुचितम्‌} इति वाक्ये \textcolor{red}{रुच्यर्थानां प्रीयमाणः} (पा॰सू॰~१.४.३३) इत्यनेन चतुर्थी।\end{sloppypar}
\vspace{2mm}
\centering ॥ इति बालकाण्डीयप्रयोगाणां विमर्शः ॥\nopagebreak\\
\vspace{4mm}
\pdfbookmark[2]{अयोध्याकाण्डम्‌}{Chap1Part1Kanda2}
\phantomsection
\addtocontents{toc}{\protect\setcounter{tocdepth}{2}}
\addcontentsline{toc}{subsection}{अयोध्याकाण्डीयप्रयोगाणां विमर्शः}
\addtocontents{toc}{\protect\setcounter{tocdepth}{0}}
\centering ॥ अथायोध्याकाण्डीयप्रयोगाणां विमर्शः ॥\nopagebreak\\
\section[प्रतिज्ञा ते कृता]{प्रतिज्ञा ते कृता}
\centering\textcolor{blue}{यदि राज्याभिसंसक्तो रावणं न हनिष्यसि।\nopagebreak\\
प्रतिज्ञा ते कृता राम भूभारहरणाय वै॥}\nopagebreak\\
\raggedleft{–~अ॰रा॰~२.१.३४}\\
\begin{sloppypar}\hyphenrules{nohyphenation}\justifying\noindent\hspace{10mm} एष प्रयोगोऽयोध्या\-काण्डस्य प्रथम\-सर्गीयः। अत्र विविध\-विद्या\-विशारदो भगवान्नारदः श्रीराममभिगम्य कथयति यद्राज्य\-लोभासक्तो रावणं न हनिष्यसि तदा ते कृता प्रतिज्ञा व्यर्था भविष्यति। \textcolor{red}{त्वया कृता} इति प्रयोक्तव्ये \textcolor{red}{ते कृता} इति प्रयुक्तम्। अत्र सम्बन्ध\-विवक्षायां षष्ठी। यद्वा \textcolor{red}{ते प्रतिज्ञा} इत्यन्वये सम्बन्धे षष्ठी।\end{sloppypar}
\section[शृणु मे]{शृणु मे}
\centering\textcolor{blue}{शृणु नारद मे किञ्चिद्विद्यतेऽविदितं क्वचित्।\nopagebreak\\
प्रतिज्ञातं च यत्पूर्वं करिष्ये तन्न संशयः॥}\nopagebreak\\
\raggedleft{–~अ॰रा॰~२.१.३६}\\
\begin{sloppypar}\hyphenrules{nohyphenation}\justifying\noindent\hspace{10mm} अत्र \textcolor{red}{शृणु मत्‌} इति प्रयोक्तव्ये \textcolor{red}{शृणु मे} इति सम्बन्ध\-विवक्षायां षष्ठी। यद्वा \textcolor{red}{मे वाक्यं शृणु} इत्यन्वये षष्ठी साधु। सा च सम्बन्ध\-सामान्ये। सम्बन्ध\-मीमांसायां न्याय\-व्याकरणयोरीषदन्तरम्। नैयायिकाः सम्बन्धमेक\-निष्ठं मन्यन्ते तदपि प्रतियोगित्व\-समानाधिकरणेन व्यवस्थापयन्ति किन्तु वयं सम्बन्धं द्विष्ठं मन्यामहे। एवं \textcolor{red}{सम्यग्बध्नाति प्रतियोग्यनुयोगिनौ यः स सम्बन्धः}। अत एव \textcolor{red}{सम्बन्धत्वं नाम सम्बन्धि\-भिन्नत्वे सति द्विष्ठत्वे सति विशिष्ट\-बुद्धि\-नियामकतावच्छेदकत्वम्}।\footnote{\textcolor{red}{“सम्बन्धो हि सम्बन्धि\-द्वय\-भिन्नत्वे सति द्विष्ठत्वे च सत्याश्रयतया विशिष्ट\-बुद्धि\-नियामकः” इत्यभियुक्त\-व्यवहारात्‌} (प॰ल॰म॰~११)।} यथा \textcolor{red}{राज\-पुरुषः} इत्यत्र स्व\-स्वामि\-भाव\-रूपः सम्बन्धः स च राजनि पुरुषे चेति द्वयोस्तिष्ठति राज\-पुरुषाभ्यां भिन्न एवं \textcolor{red}{स्वस्वामि\-भाव\-सम्बन्धावच्छिन्न\-राज\-विशिष्टः पुरुषः} इति विशिष्ट\-बुद्धि\-नियामकश्च। तस्माच्छक्ति\-निर्णय\-प्रसङ्गे \textcolor{red}{को नाम शक्ति\-पदार्थः} इति जिज्ञासायां \textcolor{red}{बोध\-जनकता शक्तिः}\footnote{\textcolor{red}{इन्द्रियाणां चक्षुरादीनां स्वविषयेषु चाक्षुषेषु घटादिषु यथाऽनादिर्योग्यता तदीय\-चाक्षुषादि\-कारणता तथा शब्दानामप्यर्थैः सह तद्बोध\-कारणतैव योग्यता सैव शक्तिरित्यर्थः} (वै॰भू॰सा॰~६.३७)।} इति यत्प्राचीनानां मतं तदप्यनयैवापत्त्या खण्डित\-प्रायम्। शक्तिर्हि शब्दार्थयोः सम्बन्धः। स च बोध\-जनकत्व\-रूपश्चेदसङ्गतः। यतो हि बोध\-जनकता समवायतया शब्दमधितिष्ठति नार्थम्। सम्बन्धस्य द्विष्ठत्वं सकल\-वैयाकरण\-सम्मतं नोपपद्येत। तस्माद्वाच्य\-वाचक\-भावापर\-पर्याया शक्तिः।\footnote{\textcolor{red}{तस्मात्पद\-पदार्थयोः सम्बन्धान्तरमेव शक्तिर्वाच्य\-वाचक\-भावापर\-पर्याया} (प॰ल॰म॰~११)।} स एव सम्बन्धो द्वावपि शब्दार्थावधितिष्ठति। सम्बन्धोऽयं तादात्म्य\-मूलकः शब्दार्थयोः। तत्रापि तादात्म्ये मूलमितरेतराध्यासः। तादात्म्यं नामाद्वैत\-वादि\-वेदान्ति\-दृष्ट्या तु तदभिन्नत्वे सति तद्भेदेन प्रतीयमानतावच्छेदकत्वम्। अस्मन्मते \textcolor{red}{तद्भिन्नत्वे सति तदभेदेन प्रतीयमानतावच्छेदकत्वम्}।\footnote{\textcolor{red}{तादात्म्यञ्च तद्भिन्नत्वे सति तदभेदेन प्रतीयमानत्वम्} (ल॰म॰, प॰ल॰म॰~१६)।} अद्वैत\-वेदान्तिनोऽभेदं पारमार्थिकं मन्यन्ते वयञ्च तमौपचारिकं स्वीकुर्महे। इत्थं वैयाकरण\-सिद्धान्तितः सम्बन्ध एव सूत्रकारैः शेष\-शब्देन व्यवह्रियते। तथा च सूत्रं \textcolor{red}{षष्ठी शेषे} (पा॰सू॰~२.३.५०)। अत्र कौमुदी\-कारो लिखति \textcolor{red}{कारक\-प्रातिपदिकार्थादि\-व्यतिरिक्तः स्व\-स्वामि\-भावादि\-सम्बन्धः शेषस्तत्र षष्ठी स्यात्‌} (वै॰सि॰कौ॰~६०६)। भाष्य\-कारास्तु कथयन्ति यत् \textcolor{red}{एकशतं हि षष्ठ्यर्थाः} (भा॰पा॰सू॰~१.१.४९)। तथाऽपि सम्बन्ध\-सामान्ये षष्ठी प्रसिद्धा। कुत्रचित्तत्तत्कारकेष्वपि षष्ठी भवति तत्तत्सूत्र\-विहित\-षष्ठी। सामान्य\-सम्बन्ध\-षष्ठ्योर्बाह्यतस्तु न कोऽपि भेदः किन्तु शाब्द\-बोधे वैलक्षण्यम्। तत्तल्लक्षणेष्वपि समासाद्यभावरूपं वैलक्षण्यमवगम्यत एव। यथा सम्बन्ध\-षष्ठ्यां \textcolor{red}{षष्ठी} (पा॰सू॰~२.२.८) इति सूत्रेण तत्पुरुष\-समासो भवति \textcolor{red}{राज\-पुरुषः} इतिवत्। कृत्षष्ठ्यां च \textcolor{red}{कृद्योगा च षष्ठी समस्यत इति वक्तव्यम्‌} (वा॰~२.२.८) इति वार्त्तिकेन तत्पुरुष\-समासो भवति \textcolor{red}{इध्मप्रव्रश्चनः} इतिवत्। परञ्च प्रतिपद\-विहित\-षष्ठ्यां समासं नैवेच्छन्ति भाष्यकारा यथा \textcolor{red}{प्रतिपद\-विधाना षष्ठी च न समस्यते} (भा॰पा॰सू॰~२.२.८)।\footnote{यथा~– \textcolor{red}{सर्पिषो ज्ञानम्। मधुनो ज्ञानम्} (का॰वृ॰~२.२.१०)। अत्र \textcolor{red}{ज्ञोऽविदर्थस्य करणे} (पा॰सू॰~२.३.५१) इत्यनेन प्रतिपद\-विधाना\-षष्ठी। ततः समासाभावः।} तस्मादत्रापादाने सम्बन्ध\-विवक्षया षष्ठी। यद्वा \textcolor{red}{मे वाक्यं श्रृणु} इति योजनया वाच्य\-वाचक\-भाव\-मूलिका षष्ठी।\end{sloppypar}
\section[तवाहितं कर्ता]{तवाहितं कर्ता}
\centering\textcolor{blue}{को वा तवाहितं कर्ता नारी वा पुरुषोऽपि वा।\nopagebreak\\
स मे दण्ड्यश्च वध्यश्च भविष्यति न संशयः॥}\nopagebreak\\
\raggedleft{–~अ॰रा॰~२.३.९}\\
\begin{sloppypar}\hyphenrules{nohyphenation}\justifying\noindent\hspace{10mm} अत्र श्वोभावि\-राम\-राज्याभिषेक\-सूचनादित्सया चक्रवर्ती दशरथः कोप\-भवने शयानां कैकेयीं श्रुत्वा तामनुनयन् कथयति यद्भवत्याः \textcolor{red}{कः अहितं कर्ता}। अत्र कृ\-धातोः (\textcolor{red}{डुकृञ् करणे} धा॰पा॰~१४७२) \textcolor{red}{ण्वुल्तृचौ} (पा॰सू॰~३.१.१३३) इत्यनेन विहिते तृचि ततः \textcolor{red}{कर्तृकर्मणोः कृति} (पा॰सू॰~२.३.६५) इत्यनेन कृतायां षष्ठ्याम् \textcolor{red}{अहितस्य कर्ता} इत्यनेन भवितव्यमासीत्। \textcolor{red}{अहितं कर्ता} इत्यापाततोऽपाणिनीयं लगति। परं तृन्नन्तमिदम्। अर्थात् \textcolor{red}{तृन्‌} (पा॰सू॰~३.२.१३५) इति सूत्रेण कृ\-धातोस्तृन्। रपरत्वेन गुणे विभक्ति\-कार्ये सौ \textcolor{red}{ऋदुशनस्पुरुदंसोऽनेहसां च} (पा॰सू॰~७.१.९४) इत्यनेनानङि \textcolor{red}{अप्तृन्तृच्स्वसृ\-नप्तृ\-नेष्टृ\-त्वष्टृ\-क्षत्तृ\-होतृ\-पोतृ\-प्रशास्तॄणाम्‌} (पा॰सू॰~६.४.११) इत्यनेन दीर्घे सुलोपे नलोपे चेति \textcolor{red}{कर्ता}।\footnote{\textcolor{red}{डुकृञ् करणे} (धा॰पा॰~१४७२)~\arrow कृ~\arrow \textcolor{red}{तृन्‌} (पा॰सू॰~३.२.१३५)~\arrow कृ~तृन्~\arrow कृ~तृ~\arrow \textcolor{red}{सार्वधातुकार्ध\-धातुकयोः} (पा॰सू॰~७.३.८४)~\arrow \textcolor{red}{उरण् रपरः} (पा॰सू॰~१.१.५१)~\arrow कर्~तृ~\arrow कर्तृ~\arrow विभक्तिकार्यम्~\arrow कर्तृ~सुँ~\arrow कर्तृ~स्~\arrow \textcolor{red}{ऋदुशनस्पुरुदंसोऽनेहसां च} (पा॰सू॰~७.१.९४)~\arrow कर्त्~अनँङ्~स्~\arrow कर्त्~अन्~स्~\arrow कर्तन्~स्~\arrow \textcolor{red}{अप्तृन्तृच्स्वसृ\-नप्तृ\-नेष्टृ\-त्वष्टृ\-क्षत्तृ\-होतृ\-पोतृ\-प्रशास्तॄणाम्} (पा॰सू॰~६.४.११)~\arrow कर्तान्~स्~\arrow \textcolor{red}{हल्ङ्याब्भ्यो दीर्घात्सुतिस्यपृक्तं हल्} (पा॰सू॰~६.१.६८)~\arrow कर्तान्~\arrow \textcolor{red}{न लोपः प्रातिपदिकान्तस्य} (पा॰सू॰~८.२.७)~\arrow कर्ता।} एवं प्राप्तायां षष्ठ्यां \textcolor{red}{न लोकाव्यय\-निष्ठा\-खलर्थ\-तृनाम्‌} (पा॰सू॰~२.३.६९) इत्यनेन निषेधेऽनुक्तत्वाच्च कर्मणस्तत्रैव \textcolor{red}{कर्मणि द्वितीया} (पा॰सू॰~२.३.२) इत्यनेन द्वितीया विभक्तिः। यद्वा \textcolor{red}{कर्ता} इति तिङन्त\-रूपं न कृदन्तम्। एवं हि कृ\-धातोः \textcolor{red}{अनद्यतने लुट्‌} (पा॰सू॰~३.३.१५) इत्यनेन लुड्\-लकारे \textcolor{red}{तिप्तस्झि\-सिप्थस्थ\-मिब्वस्मस्ताताञ्झ\-थासाथान्ध्वमिड्वहिमहिङ्‌} (पा॰सू॰~३.४.७८) इत्यनेन प्रथम\-पुरुषैकवचने \textcolor{red}{तिप्‌}\-प्रत्यये \textcolor{red}{स्यतासी लृलुटोः} (पा॰सू॰~३.१.३३) इत्यनेन \textcolor{red}{तासि}\-प्रत्यये गुणे रपरत्वे \textcolor{red}{लुटः प्रथमस्य डारौरसः} (पा॰सू॰~२.४.८५) इत्यनेन \textcolor{red}{डा} आदेशे \textcolor{red}{चुटू} (पा॰सू॰~१.३.७) इत्यनेन डकारेत्सञ्ज्ञायां \textcolor{red}{डित्त्वसामर्थ्यादभस्यापि टेर्लोपः} (ल॰सि॰कौ॰~३४३) इति नियमेन\footnote{\textcolor{red}{डित्यभस्याप्यनु\-बन्धकरण\-सामर्थ्यात्‌} (वा॰~६.४.१४३)।} टि\-लोपे \textcolor{red}{कर्ता} इति सिद्धम्।\footnote{\textcolor{red}{डुकृञ् करणे} (धा॰पा॰~१४७२)~\arrow कृ~\arrow \textcolor{red}{शेषात्कर्तरि परस्मैपदम्} (पा॰सू॰~१.३.७८)~\arrow \textcolor{red}{अनद्यतने लुट्‌} (पा॰सू॰~३.३.१५)~\arrow कृ~लुट्~\arrow कृ~तिप्~\arrow कृ~ति~\arrow \textcolor{red}{स्यतासी लृलुटोः} (पा॰सू॰~३.१.३३)~\arrow कृ~तासिँ~ति~\arrow कृ~तास्~ति~\arrow \textcolor{red}{एकाच उपदेशेऽनुदात्तात्‌} (पा॰सू॰~७.२.१०)~\arrow इडागम\-निषेधः~\arrow \textcolor{red}{सार्वधातुकार्ध\-धातुकयोः} (पा॰सू॰~७.३.८४)~\arrow \textcolor{red}{उरण् रपरः} (पा॰सू॰~१.१.५१)~\arrow कर्~तास्~ति~\arrow \textcolor{red}{लुटः प्रथमस्य डारौरसः} (पा॰सू॰~२.४.८५)~\arrow कर्~तास्~डा~\arrow कर्~तास्~आ~\arrow \textcolor{red}{डित्यभस्याप्यनु\-बन्धकरण\-सामर्थ्यात्‌} (वा॰~६.४.१४३)~\arrow कर्~त्~आ~\arrow कर्ता।} अतः कर्मणि द्वितीया निरुपद्रवा निर्भ्रान्ता।\end{sloppypar}
\section[मम]{मम}
\centering\textcolor{blue}{रामः प्राह न मे मातर्भोजनावसरः कृतः।\nopagebreak\\
दण्डकागमने शीघ्रं मम कालोऽद्य निश्चितः॥}\nopagebreak\\
\raggedleft{–~अ॰रा॰~२.४.४}\\
\begin{sloppypar}\hyphenrules{nohyphenation}\justifying\noindent\hspace{10mm} अत्र \textcolor{red}{नि}उपसर्ग\-पूर्वकाच्चयनार्थक\textcolor{red}{चिञ्‌}\-धातोः (धा॰पा॰~१२५१) भूते कर्मणि \textcolor{red}{क्तः}। कर्मण्युक्ते सति तत्र प्रथमा किन्त्वनुक्ते कर्तरि तृतीया। अतः \textcolor{red}{मया कालः निश्चितः} इत्येव। \textcolor{red}{मम कालः निश्चितः} इति कथम्। कर्तरि सम्बन्ध\-विवक्षया षष्ठी। यद्वा \textcolor{red}{मम} इत्यस्य \textcolor{red}{कालः} इत्यनेनान्वयः। तेन सम्बन्ध\-सामान्ये षष्ठी। अर्थात्स्वयं कालातीतः सन् भक्त\-वत्सलो राघवः कालमेव नियन्तारं मत्वा मातरं प्राह यदद्य मन्निरूपकः कालो दण्डकारण्य\-गमनाय पित्रा निश्चितः। यद्वा निरुपसर्ग\-चिञ्‌\-धातोः \textcolor{red}{नपुंसके भावे क्तः} (पा॰सू॰~३.३.११४) इत्यनेन भावार्थे क्त\-प्रत्ययः। ततश्च \textcolor{red}{निश्चितमस्त्यस्मिन्‌} इति विग्रहे \textcolor{red}{अर्शआदिभ्योऽच्‌} (पा॰सू॰~५.२.१२७) इत्यनेन \textcolor{red}{अच्‌}\-प्रत्ययः। अर्थाद्दण्डकारण्य\-गमनाय कालोऽयं निश्चयवानिति राघवेन्द्रस्य तात्पर्यं प्रतिभाति। यद्वा \textcolor{red}{निश्चिनोतीति निश्चितः} इति विग्रहे कर्तर्येव \textcolor{red}{क्त}\-प्रत्ययः।\footnote{\textcolor{red}{गत्यर्थाकर्मक\-श्लिष\-शीङ्स्थाऽऽस\-वस\-जन\-रुह\-जीर्यतिभ्यश्च} (पा॰सू॰~३.४.७२) इत्यनेन कर्तरि क्तः। \textcolor{red}{धातोरर्थान्तरे वृत्तेर्धात्वर्थेनोपसङ्ग्रहात्। प्रसिद्धेरविवक्षातः कर्मणोऽकर्मिका क्रिया॥} (वा॰प॰~३.७.८८)। कर्मणोऽविवक्षणादकर्मकत्वम्। यथा रघुवंशे कालिदासोऽपि प्रायुङ्क्त~– \textcolor{red}{निर्ययावथ पौलस्त्यः पुनर्युद्धाय मन्दिरात्। अरावणमरामं वा जगदद्येति निश्चितः॥} (र॰वं॰~१२.८३)। अत्र मल्लिनाथः~– \textcolor{red}{अद्य जगदरावणं रावणशून्यमरामं रामशून्यं वा भवेदेति निश्चितो निश्चितवान्। कर्तरि क्तः} (र॰वं॰ स॰व्या॰~१२.८३)।} अर्थात्साम्प्रतमयं कालो मम विश्रामं न ह्यनुमन्यते। तस्मादाज्ञां देहि। यद्वा \textcolor{red}{मम} शब्दस्य \textcolor{red}{दण्डकागमने} इति शब्देनान्वयः।\end{sloppypar}
\section[शरीरम्]{शरीरम्‌}
\centering\textcolor{blue}{प्रतिक्षणं क्षरत्येतदायुरामघटाम्बुवत्।\nopagebreak\\
सपत्ना इव रोगौघाः शरीरं प्रहरन्त्यहो॥}\nopagebreak\\
\raggedleft{–~अ॰रा॰~२.४.२८}\\
\begin{sloppypar}\hyphenrules{nohyphenation}\justifying\noindent\hspace{10mm} अत्र श्रीरामचन्द्रः क्रुद्धं लक्ष्मणं सान्त्वयन्नुपदिशति यत्काम\-क्रोधादयः षट्सपत्नाः शरीरं प्रहरन्ति। \textcolor{red}{शरीरे प्रहरन्ति} इति प्रयोक्तव्ये \textcolor{red}{शरीरम्‌} इति प्रयुक्तम्। \textcolor{red}{शरीरेऽरिः प्रहरति हृदये स्वजनस्तथा} (प्र॰ना॰~१.१२) इति प्रतिमानाटके प्रयुक्तत्वात्प्रहारस्याधार\-तयाऽधिकरणत्वं परित्यज्य कर्मत्वमुक्तमिति चेत्। कथ्यते। \textcolor{red}{शरीरमवलोक्य प्रहरन्ति} इति व्याख्यायताम्।\footnote{पूर्वपक्षोऽयम्।} न चास्मिन् व्याख्याने \textcolor{red}{ल्यब्लोपे कर्मण्यधिकरणे च} (वा॰~२.३.२८) इत्यनेन पञ्चम्याशङ्क्या। तदा \textcolor{red}{शरीरं लक्षयन्तः शरीरं घातयन्तो वा प्रहरन्ति}।\footnote{उत्तरपक्षोऽयम्।} यद्वा \textcolor{red}{परौ भुवोऽवज्ञाने} (पा॰सू॰~३.३.५५) इत्यत्रावज्ञान\-ग्रहणेन धातोरनेकार्थत्वं सूच्यते। यतो हि \textcolor{red}{परि}\-पूर्वकस्य \textcolor{red}{भू}\-धातोः (\textcolor{red}{भू सत्तायाम्} धा॰पा॰~१) निष्पन्नः \textcolor{red}{परिभव}\-शब्दोऽपमान\-सूचकः। अपमानं ह्यवज्ञानम्। \textcolor{red}{अनादरः परिभवः परीभावस्तिरस्क्रिया} (अ॰को॰~१.७.२२) इति कोष\-प्रामाण्यात्। \textcolor{red}{परि}\-पूर्वक\-\textcolor{red}{भू}\-धातोर्निसर्गतोऽप\-मानार्थे सिद्धेऽवज्ञानार्थ एवं प्रत्यय\-विधानेनावज्ञान\-ग्रहणं व्यर्थम्। तदेव व्यर्थं सज्ज्ञापयति यत् \textcolor{red}{अनेकार्था हि धातवः}। तेनात्रावज्ञान\-ग्रहणं चरितार्थं ज्ञापकञ्च। तदेव परं यत् \textcolor{red}{स्वांशे चरितार्थं वचन\-सिद्धिः फलमन्यत्र} इति नियमेनावज्ञान\-ग्रहणं स्वांशे चरितार्थमन्यस्मिन्नर्थे प्रत्यय\-व्यावर्तकत्वात् \textcolor{red}{अनेकार्था हि धातवः} इति वचन\-सिद्धिः \textcolor{red}{परिभाव्य} इत्यादौ
सत्क्रिया\-रूपं फलम्।\footnote{यथा~– \textcolor{red}{अस्यां वृणीष्व मनसा परिभाव्य कञ्चित्} (नै॰च॰~११.८) \textcolor{red}{मुनित्रयं नमस्कृत्य तदुक्तीः परिभाव्य च} (वै॰सि॰कौ॰ मङ्गलाचरणे~१) \textcolor{red}{एवं स परिभाव्य बिलद्वारं गत्वा तमाहूतवान्} (प॰त॰~४) इत्यादि\-शिष्ट\-प्रयोगेषु \textcolor{red}{परिभाव्य} इत्यस्य \textcolor{red}{विचार्य पर्यालोच्य} वेत्यर्थः। एवमेव ग्रन्थकारस्य श्रीभार्गव\-राघवीय\-महाकाव्ये \textcolor{red}{मनसा परिभाव्य भामिनीभयमाशङ्क्य भवाय भावुकः} (भा॰रा॰~५.६१) इति प्रयोगे।} तथैवात्रापि \textcolor{red}{प्र}\-पूर्वस्य \textcolor{red}{हृ}\-धातोः (\textcolor{red}{हृञ् हरणे} धा॰पा॰~८९९) हनन\-रूपोऽर्थः। यद्यपि प्रहार\-शब्दस्य सामान्यतो हिंसार्थ\-शस्त्रादि\-प्रक्षेपक\-रूपोऽर्थः स्वीक्रियते। प्रक्षेपणञ्च कस्मिंश्चिदाधारे सम्भवात्प्रक्षिप्त\-वस्तु\-संयोगाश्रयतयौपश्लेषिक आधारो जागरूकतामापद्यते। अतो मनन्ति प्राचीनाः~–\end{sloppypar}
\centering\textcolor{red}{उपसर्गेण धात्वर्थो बलादन्यत्र नीयते।\nopagebreak\\
प्रहाराहारसंहारविहारपरिहारवत्॥}\\
\begin{sloppypar}\hyphenrules{nohyphenation}\justifying\noindent\hspace{10mm} तथाऽप्यत्र 
लक्ष्यानुरोधेन प्रहार\-शब्दस्य हननार्थ\-स्वीकारे तत्कर्मतया \textcolor{red}{शरीरम्‌} इत्यत्र द्वितीया। यद्वाऽधिकरणेऽविवक्षिते सति \textcolor{red}{अकथितं च} (पा॰सू॰~१.४.५१) इत्यनेन कर्म। न च परिगणित\-षोडश\-धातुभ्योऽतिरिक्तस्य कथं कर्मतेति चेत् \textcolor{red}{तथा स्यान्नीहृकृष्वहाम्‌} (वै॰सि॰कौ॰~५३९) इत्युत्तरार्धे \textcolor{red}{हृ}\-धातोरपि गणनात्। ननु \textcolor{red}{प्र}\-पूर्वस्य \textcolor{red}{हृ}\-धातोर्धात्वन्तरतया न सङ्ग्रहो भविष्यति। उपसर्ग\-संयोजनेन धातुर्न परिवर्तते। यथा नील\-वस्त्र\-धरस्य बालकस्य कदाचिच्छ्वेत\-वस्त्र\-धारणेऽपि नैव परिवर्तनं लोके। \textcolor{red}{न हि शास्त्रं लोकाद्भिद्यते} (भा॰पा॰सू॰~१.१.३, ४.१.९३) इति महा\-भाष्य\-वचनाल्लोक\-मतमप्यादर्तव्यम्। न च \textcolor{red}{प्र}\-उपसर्ग\-संयोजनेन धातोरर्थान्तरं तथा च \textcolor{red}{अर्थ\-निबन्धनेयं सञ्ज्ञा} (वै॰सि॰कौ॰~५३९) इति वचनेन हृधात्वर्थाभावान्न कर्म\-संज्ञेति वाच्यम्। प्रापण\-रूपस्यैवार्थस्य \textcolor{red}{हृ}\-धातोर्वास्तविकत्वमिति नैव राजाज्ञा। धातोरनेकार्थकत्वस्यानुपदमेव निरूपितत्वात्। प्रायो धातूनामर्था भवन्त्युपसर्गास्तु केवलं प्रस्फोटयन्ति। निपाता द्योतका वाचका वेति पश्चाद्वक्ष्यते। यथा \textcolor{red}{भू}\-धातोः (\textcolor{red}{भू सत्तायाम्} धा॰पा॰~१) अनुभव\-रूपोऽर्थः शाश्वतः। अनुरुपसर्गो द्योतयति तद्यथा \textcolor{red}{अनुभवति} इत्यादि।
अत एवात्र हननार्थतया शरीरं कर्म। ततो द्वितीया।\end{sloppypar}
\section[तव]{तव}
\centering\textcolor{blue}{अहमग्रे गमिष्यामि वनं पश्चात्त्वमेष्यसि।\nopagebreak\\
इत्याह मां विना गन्तुं तव राघव नोचितम्॥}\nopagebreak\\
\raggedleft{–~अ॰रा॰~२.४.६३}\\
\begin{sloppypar}\hyphenrules{nohyphenation}\justifying\noindent\hspace{10mm} अत्र भगवती सीता श्रीरामं प्रार्थयमानाऽनुनयन्ती ब्रवीति हे राघव \textcolor{red}{मां विना तव गन्तुं न उचितम्‌}। अत्र \textcolor{red}{त्वया} इति प्रयोक्तव्ये \textcolor{red}{तव} इति प्रयुक्तम्। अत्र कर्तृ\-शेषत्व\-विवक्षायां षष्ठी। यद्वा \textcolor{red}{उचित}\-शब्देन सहान्वयात्सम्बन्ध\-षष्ठी। यद्वा \textcolor{red}{भावे तुमुन्‌} (भा॰पा॰सू॰~३.३.१०, ३.४.९)। ततः कर्तुरनुक्तत्वादनुक्ते कर्तरि \textcolor{red}{कर्तृ\-कर्मणोः कृति} (पा॰सू॰~२.३.६५) इति षष्ठी।
यद्वा \textcolor{red}{तव कृते} इत्यध्याहार्यम्। अतः षष्ठ्युचिता।\end{sloppypar}
\section[श्रुतानि बहुभिर्द्विजैः]{श्रुतानि बहुभिर्द्विजैः}
\centering\textcolor{blue}{अन्यत्किञ्चित्प्रवक्ष्यामि श्रुत्वा मां नय काननम्।\nopagebreak\\
रामायणानि बहुशः श्रुतानि बहुभिर्द्विजैः॥}\nopagebreak\\
\raggedleft{–~अ॰रा॰~२.४.७७}\\
\begin{sloppypar}\hyphenrules{nohyphenation}\justifying\noindent\hspace{10mm} वन\-गमनायानुरोधं कुर्वती भगवती सीता कथयति यत् \textcolor{red}{द्विजैर्बहूनि रामायणानि श्रुतानि}। अत्र शब्दानां पृथग्भवनतः \textcolor{red}{ध्रुवमपायेऽपादानम्‌} (पा॰सू॰~१.४.२४) इत्यनेनापादान\-सञ्ज्ञा स्यात्। यतो हि \textcolor{red}{अपायो विश्लेषस्तस्मिन्साध्ये यद्ध्रुवमवधि\-भूतं कारकं तदपादान\-सञ्ज्ञं स्यात्‌} (ल॰सि॰कौ॰~८९९) इत्यर्थः। तदनु विश्लिष्ट\-रामायणात्मक\-शब्द\-समूहस्यावधिभूत\-द्विजानामपादान\-सञ्ज्ञा। ततश्च \textcolor{red}{अपादाने पञ्चमी} (पा॰सू॰~२.३.२८) इत्यनेन पञ्चमी विभक्तिः। एवं \textcolor{red}{द्विजेभ्यः} इत्युचितं कथं \textcolor{red}{द्विजैः} इति। अत्रोच्यते। अत्र करणत्व\-विवक्षा। एवं करण\-तृतीया। यद्वा हेतुत्व\-विवक्षायां \textcolor{red}{हेतौ} (पा॰सू॰~२.३.२३) इत्यनेन तृतीया। यद्वा \textcolor{red}{कथितानि निगदितानि} इत्यध्याहार्यम्। एवं \textcolor{red}{द्विजैः कथितानि निगदितानि वा रामायणानि श्रुतानि} इत्यनुक्ते कर्तरि तृतीया। यद्वा भवता तु श्रुतान्येव किन्तु द्विजैरपि श्रुतानि। अतो रामायणस्य श्रवणस्य द्विजेषु कर्तृत्वम्। तात्पर्यमिदं यद्भवतश्चरित्रं श्लाघा\-विपर्ययतः कदाचिद्भवान्न शृणोति किन्तु संस्कार\-शीला द्विजा बहुशः शृण्वन्त्यश्रौषुश्च। शुद्ध\-संस्कारतया ते न विस्मरन्ति। तेषां वाक्यस्य प्रामाण्यं भवानपि मन्यतेऽतस्तान् पृच्छतु यत्कस्मिंश्चिद्रामायणे कस्मिंश्चिद्वा कल्पे भवान्मां विना वनमगच्छत्। श्रीसीताया हार्दमिदं यद्यद्यपि प्रतिकल्पं भवानवतरति मया सह तेषु तेषु कल्पेषु भिन्नानि भिन्नानि चरित्राणि समाचरति भवानतश्च तच्चारु\-चारु\-चरित्र\-प्रतिपादन\-परतया सहस्त्रशो रामायणानि व्याचक्षत मुनीन्द्राः। सत्सु प्रति\-रामायणे भिन्नेषु भवच्चरित्रेषु जन्म\-लीला विवाह\-लीला वन\-लीला रण\-लीला राज्य\-लीलाश्च सर्वत्र समाना एव। तत्र वन\-लीलायां मया सह भवद्वन\-गमनं सर्वैरपि रामायणैरनु\-मोदितं श्रुतवन्तो द्विजास्तत्र प्रमाणम्। अतो \textcolor{red}{द्विजैः} इत्यनुक्त\-कर्तरि तृतीया।\end{sloppypar}
\section[सजानकिम्]{सजानकिम्‌}
\centering\textcolor{blue}{आयान्तं नागरा दृष्ट्वा मार्गे रामं सजानकिम्।\nopagebreak\\
लक्ष्मणेन समं वीक्ष्य ऊचुः सर्वे परस्परम्॥}\nopagebreak\\
\raggedleft{–~अ॰रा॰~२.५.१}\\
\begin{sloppypar}\hyphenrules{nohyphenation}\justifying\noindent\hspace{10mm} \textcolor{red}{जानक्या सह वर्तमानं रामम्‌} इति विग्रहे \textcolor{red}{तेन सहेति तुल्य\-योगे} (पा॰सू॰~२.२.२८) इत्यनेन समासे सहस्य सादेशे\footnote{\textcolor{red}{वोपसर्जनस्य} (पा॰सू॰~६.३.८२) इत्यनेन।} अमि च \textcolor{red}{सजानकीम्‌} इत्येव।\footnote{पूर्वपक्षोऽयम्।} ह्रस्वः कथम्।
\textcolor{red}{जानकीवाऽचरति जानकयति}।\footnote{जानकी~\arrow \textcolor{red}{सर्वप्राति\-पदिकेभ्य आचारे क्विब्वा वक्तव्यः} (वा॰~३.१.११)~\arrow जानकी~क्विँप्~\arrow जानकी~व्~\arrow \textcolor{red}{वेरपृक्तस्य} (पा॰सू॰~६.१.६७)~\arrow जानकी~\arrow \textcolor{red}{सनाद्यन्ता धातवः} (पा॰सू॰~३.१.३२)~\arrow धातु\-सञ्ज्ञा~\arrow \textcolor{red}{शेषात्कर्तरि परस्मैपदम्} (पा॰सू॰~१.३.७८)~\arrow \textcolor{red}{वर्तमाने लट्} (पा॰सू॰~३.२.१२३)~\arrow जानकी~लट्~\arrow जानकी~तिप्~\arrow जानकी~ति~\arrow \textcolor{red}{कर्तरि शप्} (पा॰सू॰~३.१.६८)~\arrow जानकी~शप्~ति~\arrow जानकी~अ~ति~\arrow \textcolor{red}{सार्वधातुकार्ध\-धातुकयोः} (पा॰सू॰~७.३.८४)~\arrow जानके~अ~ति~\arrow \textcolor{red}{एचोऽयवायावः} (पा॰सू॰~६.१.७८)~\arrow जानकय्~अ~ति~\arrow जानकयति।} \textcolor{red}{जानकयतीति जानकिः} इति विग्रहे कर्तरि क्विप्।
पृषोदरादित्वाद्ध्रस्वः।
ततः समासे \textcolor{red}{सजानकिम्‌} इति। यद्वा \textcolor{red}{जानक्या सह वर्तमानम्‌} इति विग्रहेऽपि \textcolor{red}{गोस्त्रियोरुप\-सर्जनस्य} (पा॰सू॰~१.२.४८) इत्यनेन ह्रस्वः।\footnote{अत्र \textcolor{red}{तेन सहेति तुल्ययोगे} (पा॰सू॰~२.२.२८) इत्यनेन बहुव्रीहि\-समासः। \textcolor{red}{अनेकमन्यपदार्थे} (पा॰सू॰~२.२.२४) इति निर्देशनेन \textcolor{red}{सर्वोपसर्जनो बहुव्रीहिः}। तस्मात् \textcolor{red}{सह} इत्यस्य \textcolor{red}{जानकी} इति स्त्रीप्रत्ययान्तस्य चोपसर्जन\-सञ्ज्ञा। \textcolor{red}{वोपसर्जनस्य} (पा॰सू॰~६.३.८२) इत्यनेन \textcolor{red}{सह} इत्यस्य \textcolor{red}{स} इत्यादेशः। \textcolor{red}{गोस्त्रियोरुप\-सर्जनस्य} (पा॰सू॰~१.२.४८) इत्यनेन \textcolor{red}{जानकी} इत्यस्य \textcolor{red}{जानकि} इति ह्रस्वः। यथा \textcolor{red}{गोभिः सह वर्तमानः} इति विग्रहे \textcolor{red}{सगुः} इत्यत्र। यथा तैत्तिरीयारण्यके~– \textcolor{red}{इन्द्र॑स्य गृ॒हो॑सि॒ तं त्वा॒ प्रप॑द्ये॒ सगु॒ साश्व॑ स॒ह यन्मे॒ अस्ति॒ तेन॑} (कृ॰य॰ तै॰आ॰~४.४२.९)। अत्र सायणाचार्याः~– \textcolor{red}{सगुर्गोसहितः}। एवमेवापस्तम्ब\-श्रौत\-सूत्रे \textcolor{red}{ब्रह्म वर्म ममान्तरं तं त्वेन्द्रग्रह प्रविशानि सगुः साश्वः सपूरुषः} (आ॰श्रौ॰सू॰~१४.२६.१) इत्यत्रापि \textcolor{red}{सगुः}।} अनित्यत्वाच्च न कप्प्रत्ययः।\footnote{\textcolor{red}{नद्यृतश्च} (पा॰सू॰~५.४.१५३) इत्यनेन बहुव्रीहि\-समासान्ते \textcolor{red}{कप्‌}\-प्रत्ययः प्राप्तः। यथा \textcolor{red}{तेन सहेति तुल्ययोगे} (पा॰सू॰~२.२.२८) इत्यनेन जातेषु \textcolor{red}{सदेवीकः} (क॰स॰सा॰~७.१.९६) \textcolor{red}{सपत्नीकः} (र॰वं॰~१.८१, म॰पु॰~५८.२०, ७१.२) \textcolor{red}{सश्रीकः} (अ॰शा॰~५.९, ह॰व॰~२२.२९) \textcolor{red}{सश्रीकम्} (भा॰पु॰~९.६.१९) \textcolor{red}{ससाध्वीकाः} (बृ॰सं॰~१३.४) \textcolor{red}{ससुन्दरीकः} (क॰स॰सा॰~८.६.२५०) \textcolor{red}{सस्त्रीकाः} (भा॰पु॰~१०.३३.५) \textcolor{red}{सहपत्नीकाः} (आ॰श्रौ॰सू॰~७.२१.६, ७.२७.१६, १३.२०.५, १५.१३.१०) \textcolor{red}{सारुन्धतीकाः} (कु॰स॰~६.४) इत्यादि\-समासेषु। अत्र कबभावः। समासान्त\-प्रत्यय\-प्रकरणस्यानित्यत्वात्। \pageref{sec:sthapya}तमे पृष्ठे \ref{sec:sthapya} \nameref{sec:sthapya} इति प्रयोगस्य विमर्शं पश्यन्तु~– “समासान्त\-प्रत्यय\-प्रकरणं ह्यनित्यम्। प्रमाणं चात्र \textcolor{red}{यचि भम्‌} (पा॰सू॰~१.४.१८) इति सूत्रम्। अत्र \textcolor{red}{यश्चाच्च यच्‌} इति समाहार\-द्वन्द्वः। इह \textcolor{red}{द्वन्द्वाच्चु\-दषहान्तात्समाहारे} (पा॰सू॰~५.४.१०६) इत्यनेन चान्तत्वाट्टच्प्रत्ययः प्रयोक्तव्य आसीत्। तस्मिन् प्रयुक्ते \textcolor{red}{यचे भम्‌} इति स्यात्। यतो न प्रयुक्तोऽतः समासान्त\-प्रत्यस्यानित्यता ज्ञायते।”}\end{sloppypar}
\section[वनम्]{वनम्‌}
\centering\textcolor{blue}{गन्ताऽद्यैव वनं रामो लक्ष्मणेन सहायवान्।\nopagebreak\\
एषा सीता हरेर्माया सृष्टिस्थित्यन्तकारिणी॥}\nopagebreak\\
\raggedleft{–~अ॰रा॰~२.५.२३}\\
\begin{sloppypar}\hyphenrules{nohyphenation}\justifying\noindent\hspace{10mm} एष प्रयोगोऽध्यात्म\-रामायणस्यायोध्या\-काण्डे पञ्चमसर्गीयः। अत्र सीता\-लक्ष्मणाभ्यां सह दण्डकावनं गच्छन्तं श्रीरामं विलोक्य शोक\-सन्तप्त\-हृदयान् विषाद\-जलधौ निमज्जतः कोशल\-पुर\-वासिनो निरीक्ष्य भगवान् वामदेव आध्यात्मिक\-श्रीराम\-तत्त्व\-वर्णन\-माध्यमेन सर्वेषां शोकापनोदाय यतमानो ब्रवीति यत् \textcolor{red}{लक्ष्मणेन सह श्रीरामोऽद्यैव वनं गन्ता}। अत्र \textcolor{red}{गन्ता} इति तृजन्त\-प्रयोगः। \textcolor{red}{गम्‌}धातोः (\textcolor{red}{गमॢँ गतौ} धा.पा. ९८२) \textcolor{red}{गच्छतीति गन्ता} इति विग्रहे \textcolor{red}{तृच्‌}प्रत्यये\footnote{\textcolor{red}{ण्वुल्तृचौ} (पा॰सू॰~३.१.१३३) \textcolor{red}{कर्तरि कृत्‌} (पा॰सू॰~३.४.६७) इत्याभ्याम्।} \textcolor{red}{नश्चापदान्तस्य झलि} (पा॰सू॰~८.३.२४) इत्यनेनानुस्वारे \textcolor{red}{अनुस्वारस्य ययि परसवर्णः} (पा॰सू॰~८.४.५८) इत्यनेन परसवर्णे नकारे विभक्ति\-कार्ये सौ प्रत्यये \textcolor{red}{ऋदुशनस्पुरुदंसोऽनेहसां च} (पा॰सू॰~७.१.९४) इत्यनेन अनङि \textcolor{red}{अप्तृन्तृच्स्वसृ\-नप्तृ\-नेष्टृ\-त्वष्टृ\-क्षत्तृ\-होतृ\-पोतृ\-प्रशास्तॄणाम्‌} (पा॰सू॰~६.४.११) इत्यनेन दीर्घे सुलोप\-नलोपयोः\footnote{\textcolor{red}{हल्ङ्याब्भ्यो दीर्घात्सुतिस्यपृक्तं हल्‌} (पा॰सू॰~६.१.६८) \textcolor{red}{न लोपः प्रातिपदिकान्तस्य} (पा॰सू॰~८.२.७) इत्याभ्याम्।} \textcolor{red}{गन्ता} इति सिध्यति। \textcolor{red}{वनम्‌} इत्यत्र \textcolor{red}{कर्तृ\-कर्मणोः कृति} (पा॰सू॰~२.३.६५) इत्यनेन षष्ठ्युचिता। परञ्चात्र न \textcolor{red}{तृच्‌} अपि तु \textcolor{red}{तृन्‌}।\footnote{\textcolor{red}{तृन्‌} (पा॰सू॰~३.२.१३५) इत्यनेन तद्धर्मकर्तरि।} तृन्तृचोः स्वरभेदः।\footnote{शेषप्रक्रिया तु पूर्ववदिति भावः।} तृन्प्रत्ययेऽपि \textcolor{red}{कर्तृ\-कर्मणोः कृति} (पा॰सू॰~२.३.६५) इत्यनेन षष्ठी प्राप्ता किन्तु \textcolor{red}{न लोकाव्यय\-निष्ठा\-खलर्थ\-तृनाम्‌} (पा॰सू॰~२.३.६९) इत्यनेन षष्ठीनिषेधः। यद्वा \textcolor{red}{गन्ता} इति लुट्प्रथम\-पुरुषैक\-वचनान्तं तिङन्त\-रूपम्। अतस्तद्योगे \textcolor{red}{कर्तुरीप्सिततमं कर्म} (पा॰सू॰~१.४.४९) इत्यनेन कर्मसञ्ज्ञा \textcolor{red}{कर्मणि द्वितीया} (पा॰सू॰~२.३.२) इत्यनेन द्वितीया। अनद्यतनत्वस्य चाविवक्षा।\footnote{\textcolor{red}{अनद्यतने लुँट्‌} (पा॰सू॰~३.३.१५) इत्यनेनानद्यतने भविष्यति लुँट् विहितः। कथं तर्हि \textcolor{red}{अद्यैव} इत्यस्य योगेऽपि लुँट्। इत्यपेक्षायामुक्तम्~– \textcolor{red}{अनद्यतनत्वस्य चाविवक्षा}। यथा भारते \textcolor{red}{युवा षोडशवर्षो हि यदद्य भविता भवान्‌} (म॰भा॰~१४.५६.२२) इत्यत्रापि \textcolor{red}{अद्य}योगेऽपि लुँट्। यद्वाऽत्र परिदेवने भविष्यति लुँट्। \textcolor{red}{परिदेवने श्वस्तनीभविष्यन्त्यर्थे} (वा॰~३.३.१५) इत्यनेन। शोक\-सन्तप्तान् कोशल\-पुर\-वासिनो निरीक्ष्य वामदेवोऽपि शोकसन्तप्तः सन् निजपरिदेवनं द्योतयति।}\end{sloppypar}
\section[रामसीतयोः]{रामसीतयोः}
\centering\textcolor{blue}{य इदं चिन्तयेन्नित्यं रहस्यं रामसीतयोः।\nopagebreak\\
तस्य रामे दृढा भक्तिर्भवेद्विज्ञानपूर्विका॥}\nopagebreak\\
\raggedleft{–~अ॰रा॰~२.५.३१}\\
\begin{sloppypar}\hyphenrules{nohyphenation}\justifying\noindent\hspace{10mm} भगवान् वामदेवः श्रीरामचन्द्रं वल्कल\-धारिणं वनं प्रयान्तं दृष्ट्वा शोक\-कलित\-मनस्कानयोध्या\-वासिनः श्रीरामस्याऽध्यात्मिक\-तत्त्वं बोधयित्वा फल\-श्रुतिं श्रावयन्नाह \textcolor{red}{रहस्यं राम\-सीतयोः} इति। यो मानवो मयोदितं राम\-सीतयो रहस्यं चिन्तयिष्यति स निष्पापो भविष्यति। \textcolor{red}{रामश्च सीतेति रामसीते तयोः} इति द्वन्द्वः। विप्रतिपत्तिर्यत् \textcolor{red}{सीता\-रामयोः} इत्यनेन भवितव्यं यथा तुलसीदासोऽपि प्रयुङ्क्ते \textcolor{red}{सीता\-राम\-गुण\-ग्राम\-पुण्यारण्य\-विहारिणौ} (रा॰च॰मा॰~१/म॰~१.४)। सीता ह्यस्माकं जनन्यतः श्रीरामापेक्षयाऽभ्यर्हितैवं तस्या एव पूर्व\-प्रयोगो भवेत्।\footnote{\textcolor{red}{अभ्यर्हितम्} (वा॰~२.२.३४)। \textcolor{red}{अभ्यर्हितं पूर्वं निपततीति वक्तव्यम्। मातापितरौ श्रद्धामेधे} (भा॰पा॰सू॰~२.२.३४)।} उक्तञ्च स्मृतौ~–\end{sloppypar}
\centering\textcolor{red}{पितुर्दशगुणा माता गौरवेणातिरिच्यते।\nopagebreak\\
मातुर्दशगुणा मान्या विमाता धर्मभीरुणा॥}\footnote{मूलं स्मृति\-ग्रन्थेषु मृग्यम्। महाभारतस्य दाक्षिणात्य\-पाठे \textcolor{red}{पितुः शतगुणं माता गौरवेणातिरिच्यते} (म॰भा॰~१४.११०.६०) इति वर्तते। \textcolor{red}{“पितुः शतगुणं माता” इति स्मृतेश्च} (वा॰रा॰ भू॰टी॰~१.२३.२) इति \textcolor{red}{कौसल्या सुप्रजा राम} (वा॰रा॰~१.२३.२) इति श्लोके भूषण\-टीकायां गोविन्दराजाः।}\\
\begin{sloppypar}\hyphenrules{nohyphenation}\justifying\noindent वाल्मीकीय\-रामायणे बङ्ग\-संस्करणे चापि \textcolor{red}{जननी जन्मभूमिश्च स्वर्गादपि गरीयसी}।\footnote{{\englishfont Julius J. Lipner (Translator) (2005). Chatterji, Bankimcandra. \textit{Anandamath, or The Sacred Brotherhood}. Oxford University Press: Oxford, UK. ISBN 978-01-95346-33-6. p. 241: \textit{VMBS} (Sabyasachi Bhattacharya's \textit{Vande Mataram: The Biography of a Song}) says that this eulogy of the mother and the birthland ocurred in the version of Valmiki's Ramayana current in Bengal}.} प्रसिद्धं चापि \textcolor{red}{गौरी\-शङ्करौ लक्ष्मी\-नारायणौ शची\-पुरन्दरौ सीता\-रामौ राधा\-कृष्णौ} इत्याद्यत्रेयमेवापाणिनीयता प्रतिभाति। \textcolor{red}{आर्षत्वात्‌} इत्यपि न विचार्यताम्। ऐश्वर्य\-दृष्ट्या सीता सर्वेषां जननी। मुनिरत्र वामदेवो दशरथ\-पुरोहितो गुरु\-पुत्रो मन्त्री च। अयोध्यायाः सीता पुत्र\-वधूरिति। श्वशुरालये स्त्रियः प्राधान्यं वा श्रैष्ठ्यत्वं वा नाङ्गीक्रियते शास्त्रीय\-संस्कृतौ पत्न्याः पतिं प्रति दासीभाव\-व्यवस्थापनात्। अतो वामदेव\-दृष्टावयोध्या\-निवास\-कारणाच्छ्रीरामोऽभ्यर्हितः। अतः \textcolor{red}{रामश्च सीता चेति तयो रामसीतयोः} इति प्रयुक्तम्। यद्वा \textcolor{red}{सिनोति बध्नाति}। \textcolor{red}{प्रेम्णा सिनोति सीतां यः स सीतः}।\footnote{\textcolor{red}{षिञ् बन्धने} (धा॰पा॰~१४७७) इति धातोः \textcolor{red}{गत्यर्थाकर्मक\-श्लिष\-शीङ्स्थाऽऽस\-वस\-जन\-रुह\-जीर्यतिभ्यश्च} (पा॰सू॰~३.४.७२) इत्यनेन कर्तरि क्तः। पृषोदरादित्वाद्दीर्घः। अविवक्षित\-कर्मकत्वादकर्मकत्वम्। एवमेव ग्रन्थ\-कारैः श्रीसीता\-सुधा\-निधौ \textcolor{red}{सीता}\-शब्दस्य व्युत्पत्तिः प्रदर्शिता~– \textcolor{red}{यत्पाद\-पङ्कज\-पराग\-रसानुराग\-खड्गात् खडन्ति मुनयो भवबन्धनानि। रामं सिनोषि तमहो निजभाव\-रज्ज्वा सीतेति नाम समगास्त्वमतो जगत्याम्॥} (सी॰सु॰नि॰~१.१७)। अत्र भाषायां भक्तिटीका च~– \textcolor{red}{सिनोति इति सीता इस व्युत्पत्ति से षिञ् बन्धने धातु से कर्ता में क्त प्रत्यय और पृषोदरादित्वात् दीर्घ ईकार करके सीता शब्द की निष्पत्ति की जाती है} (सी॰सु॰नि॰ भ॰टी॰~१.१७)।} \textcolor{red}{सीता च सीतश्च} इति विग्रहे \textcolor{red}{पुमान् स्त्रिया} (पा॰सू॰~१.२.६७) इत्यनेनैक\-शेषः \textcolor{red}{सीतौ}। \textcolor{red}{रमयत इति रामौ}।\footnote{\textcolor{red}{रमयति सर्वान् गुणैरिति रामः। “रामो रमयतां वरः” (वा॰रा॰~७.४२.२१, ७.१०८.२५) इत्यार्ष\-निर्वचन\-बलात्कर्तर्यपि कारके घञ् वर्ण्यते} (वा॰रा॰ भू॰टी॰~१.१.८)। एवमेव रमयतः सर्वान् गुणैरिति रामौ।} \textcolor{red}{रामौ च तौ सीतौ चेति राम\-सीतौ तयो राम\-सीतयोः} इति। यद्वा पूर्व\-निपात\-प्रकरणमनित्यम्। \textcolor{red}{लक्षण\-हेत्वोः क्रियायाः} (पा॰सू॰~३.२.१२६) इति दर्शनात्।\footnote{\textcolor{red}{लक्षण\-हेत्वोः} इत्यत्र द्वन्द्व\-समासे \textcolor{red}{अल्पाच्तरम्} (पा॰सू॰~२.२.३४) इत्यनेन \textcolor{red}{हेतु}\-शब्दस्य पूर्व\-निपाते \textcolor{red}{हेतु\-लक्षणयोः} इत्यनेन भवितव्यमासीत्। \textcolor{red}{लक्षण\-हेत्वोः} इति पाणिनि\-प्रयोगः पूर्व\-निपात\-प्रकरणस्यानित्यत्वं ज्ञापयति। \textcolor{red}{लक्षण\-हेत्वोरिति निर्देशः पूर्वनिपात\-व्यभिचार\-लिङ्गम्} (का॰वृ॰~३.२.१२६)।}\end{sloppypar}
\section[मे]{मे}
\centering\textcolor{blue}{गृहाण फलमूलानि त्वदर्थं सञ्चितानि मे।\nopagebreak\\
अनुगृह्णीष्व भगवन् दासस्तेऽहं सुरोत्तम॥}\nopagebreak\\
\raggedleft{–~अ॰रा॰~२.५.६७}\\
\begin{sloppypar}\hyphenrules{nohyphenation}\justifying\noindent\hspace{10mm} अयोध्यातः प्रतिष्ठमानः सीता\-लक्ष्मण\-समेतः श्रीरामो निषादेन सह गतः। तदनु फल\-मूलानि श्रीरामाय निषादो न्यवेदयत यत् \textcolor{red}{मे सञ्चितानि फलमूलानि गृहाण}। अत्र सञ्चयनानुकूल\-व्यापारस्याऽश्रयः कर्ता निषादः प्रत्ययश्चात्र कर्मणि। अतः कर्तुरनुक्तत्वादत्र तृतीयया भवितव्यं\footnote{\textcolor{red}{कर्तृ\-करणयोस्तृतीया} (पा॰सू॰~२.३.१८) इत्यनेन।} किन्तु कर्तुः सम्बन्ध\-विवक्षायां षष्ठी। यद्वा \textcolor{red}{क्तस्य च वर्तमाने} (पा॰सू॰~२.३.६८) इति सूत्रेण क्तयोगा षष्ठी।\footnote{\textcolor{red}{मति\-बुद्धि\-पूजार्थेभ्यश्च} (पा॰सू॰~३.२.१८८) इत्यनेन वर्तमाने क्तः। \textcolor{red}{च}\-कारेणान्यत्रापि। \textcolor{red}{शीलितो रक्षितः क्षान्त आक्रुष्टो जुष्ट इत्यपि। रुष्टश्च रुषितश्चोभावभिव्याहृत इत्यपि॥ हृष्टतुष्टौ तथा कान्तस्तथोभौ संयतोद्यतौ। कष्टं भविष्यतीत्याहुरमृताः पूर्ववत्समृताः॥ न म्रियन्तेऽमृताः} (भा॰पा॰सू॰~३.२.१८८)। \textcolor{red}{अनुक्तसमुच्चयार्थश्चकारः। ... तथा सुप्तः शयित आशितो लिप्तस्तृप्त इत्येवमादयोऽपि वर्तमाने द्रष्टव्याः} (का॰वृ॰~३.२.१८८)। \textcolor{red}{चकारोऽनुक्त\-समुच्चयार्थः} (वै॰सि॰कौ॰~३०८९)।} यद्वा \textcolor{red}{मे} इति फलमूलैः सहान्वितम्। तेन \textcolor{red}{मत्सम्बन्धीनि फल\-मूलानि गृहाण} इति सामान्य\-सम्बन्धेन षष्ठी। यद्वा \textcolor{red}{मे} इति \textcolor{red}{अनुगृह्णीष्व} इत्यनेनान्वितम्। \textcolor{red}{मे ममोपर्यनुगृह्णीष्वानुग्रहं कुरुष्व} इति तात्पर्यम्।\end{sloppypar}
\section[आरुह्यतां नौकाम्]{आरुह्यतां नौकाम्‌}
\centering\textcolor{blue}{स्वयमेव दृढं नावमानिनाय सुलक्षणाम्।\nopagebreak\\
स्वामिन्नारुह्यतां नौकां सीतया लक्ष्मणेन च॥}\nopagebreak\\
\raggedleft{–~अ॰रा॰~२.६.१८}\\
\begin{sloppypar}\hyphenrules{nohyphenation}\justifying\noindent\hspace{10mm} गङ्गा\-पारं गन्तुकामं रामं राजीव\-लोचनं स्वामिनम् \textcolor{red}{आरुह्यतां नौकाम्‌} इत्युवाच निनाय नावं निषादः। अत्र \textcolor{red}{आरुह्यताम्‌} इति कर्म\-वाच्य\-प्रयोगः।\end{sloppypar}
\centering\textcolor{red}{कर्मवाच्यप्रयोगेषु प्रथमा कर्मकारके।\nopagebreak\\
तृतीयान्तो भवेत्कर्ता कर्माधीनं क्रियापदम्॥}\nopagebreak\\
\raggedleft{–~अस्मद्गुरुचरणाः}\\
\begin{sloppypar}\hyphenrules{nohyphenation}\justifying\noindent इति नियमानुसारमत्र \textcolor{red}{आरुह्यताम्‌} इति कर्म\-वाच्य\-प्रयोगः। अत्र \textcolor{red}{नौका} इति कर्म। एवमत्र प्रथमा\-विभक्त्या भवितव्यम् \textcolor{red}{आरुह्यतां नौका} इत्येव \textcolor{red}{आरुह्यतां नौकाम्‌} इत्यार्ष\-प्रयोगः प्रतिभाति। कारणमिदं यद्यस्मिन्नर्थे प्रत्ययो भवति स उक्तः। लकाराः प्रायस्त्रिषु कर्तरि कर्मणि भावे च भवन्ति। तत्र सूत्रम् \textcolor{red}{लः कर्मणि च भावे चाकर्मकेभ्यः} (पा॰सू॰~३.४.६९) इति। अकर्मकेभ्यो धातुभ्यः कर्तरि भावे चैवं सकर्मकेभ्यो धातुभ्यः कर्मणि कर्तरि च लकारा भवन्त्विति। अत्राऽशङ्क्यते यत् \textcolor{red}{भाव\-कर्मणोः} (पा॰सू॰~१.३.१३) इत्यस्मात्सूत्राद्भाव\-कर्मणोर्ज्ञाने \textcolor{red}{कर्तरि कर्मव्यतिहारे} (पा॰सू॰~१.३.१४) इत्यस्मात् \textcolor{red}{कर्तृ}\-पदस्यावगतौ भावे कर्तरि कर्मणि लकाराः स्युरिति द्वाभ्यां सूत्राभ्यामुपात्तं\footnote{एताभ्यां सूत्राभ्यां कर्मणि भावे कर्म\-व्यतिहारे कर्तरि चात्मने\-पद\-प्रत्ययाः स्युरित्युपात्तम्। \textcolor{red}{तिप्तस्झि\-सिप्थस्थमिब्वस्मस्ताताञ्झ\-थासाथान्ध्वमिड्वहिमहिङ्} (पा॰सू॰~३.४.७८) \textcolor{red}{लः परस्मैपदम्} (पा॰सू॰~१.४.९९) \textcolor{red}{तङानावात्मनेपदम्} (पा॰सू॰~१.४.१००) इत्येभिर्लादेशाः परमैपद\-सञ्ज्ञका आत्मने\-पद\-सञ्ज्ञका वा स्युरित्युपात्तम्। ततः कर्मणि भावे कर्तरि च लादेशाः स्युरित्युपात्तम्। तस्माद्वह्नि\-धूम\-न्यायेन कर्मणि भावे कर्तरि च लकाराः स्युरित्यप्युपात्तम्। तथा च \textcolor{red}{शेषात्कर्तरि परस्मैपदम्‌} (पा॰सू॰~१.३.७८) इत्यस्मात्कर्तरि परस्मैपद\-सञ्ज्ञका लादेशाः स्युरित्युपात्तम्। अपि च \textcolor{red}{कर्तरि कृत्‌} (पा॰सू॰~३.४.६७) इत्यस्माल्लकाराः कर्तरि स्युरित्युपात्तम्। तेषां कृत्त्वात्।} किमनेन सूत्रारम्भेण इति चेन्न। सूत्राभावे सकर्मकेभ्यो धातुभ्यो लकारा भावेऽपि भविष्यन्ति प्रतिरोधकाभावात्। एवं \textcolor{red}{घटं क्रियते} इत्यनिष्ट\-प्रयोगः स्यात्। अकर्मकेभ्यश्च धातुभ्यः कर्मणि लकारा भविष्यन्ति। तदर्थं सूत्रारम्भ आवश्यकः। न च \textcolor{red}{अकर्मकेभ्यो भावे लः} इत्येव न्यासोऽस्तु। एतन्न्यासे \textcolor{red}{असति बाधके सर्वं वाक्यं सावधारणं भवति}\footnote{मूलं मृग्यम्।} इति न्यायेन \textcolor{red}{अकर्मकेभ्यो भाव एव लः} इति नियमे जातेऽकर्मकेभ्यो भाव एव लकारा भविष्यन्ति न कर्तरि। तदा \textcolor{red}{रामः शेते} इत्यादि\-प्रयोगा न भविष्यन्ति \textcolor{red}{रामेण क्रीड्यते} इत्यादयो भाव\-प्रयोगा एव सेत्स्यन्ति। एव\-कारस्य त्रिधाऽर्थोऽस्मत्सम्प्रदाये प्रसिद्धोऽन्ययोग\-व्यवच्छेदोऽयोग\-व्यवच्छेदोऽत्यन्तायोग\-व्यवच्छेदश्च। 
अन्य\-योग\-व्यवच्छेदो नामान्य\-सम्बन्धि\-योग\-निवर्तकत्वम्। अयोग\-व्यवच्छेदो नाम योगाभाव\-निवर्तनम्। अत्यन्तायोग\-व्यवच्छेदो नाम योगत्वावच्छिन्ने प्रतियोगितावच्छेदकाभाव\-व्यावर्तकत्वम्। यत्र विशेषण\-सङ्गत एवकारो भवति तत्रान्य\-योगो व्यवच्छिद्यते यथा \textcolor{red}{नीलमेवोत्पलम्‌} अत्र नीलत्व\-प्रतिष्ठापनेन श्वेतत्वादीनां व्यवच्छेदः। विशेष्य\-सङ्गत एव\-कारे सत्ययोगो व्यवच्छिद्यते यथा \textcolor{red}{दाशरथी राम एव सर्वावतारि\-ब्रह्म}। अत्यन्तायोगो व्यावर्त्यते क्रिया\-सङ्गतेनैव\-कारेण यथा \textcolor{red}{दाशरथी रामः शराणागतान् रक्षत्येव}। इत्थम् \textcolor{red}{अकर्मकेभ्यो भाव एव लकारः} इत्येव\-कारेण कर्ता व्यवच्छिद्यते तथा \textcolor{red}{रामो राजते मुकुन्दः} इत्यादयः प्रयोगा न भविष्यन्ति। \textcolor{red}{लो भावे चाकर्मकेभ्यो} इति सूत्र्यतां चकारेण कर्ताऽऽगमिष्यति पुनर्द्वितीय\-चकारस्य किं प्रयोजनमिति चेत्सत्यम्। तथा सत्यकर्मकेभ्य एव कर्तरि प्रयोगो भविष्यति। एवम् \textcolor{red}{आरामे रमते रामः} इत्यादय एव प्रयोगाः सेत्स्यन्ति। एवं सकर्मक\-धातोः कर्तरि प्रयोगो नैव सम्पत्स्यते। तथा सति \textcolor{red}{लश्च भावे चाकर्मकेभ्यः} न्यासो भवतु। द्वितीय\-चकारेण सकर्मकेभ्योऽपि कर्तरि लकारो भविष्यति पुनः \textcolor{red}{कर्मणि} इति पदस्य काऽऽवश्यकता इति चेत्। \textcolor{red}{कर्मणि} पदाभावे \textcolor{red}{रामो रावणं हन्ति} इत्यादय एव प्रयोगाः सम्पत्स्यन्ते तथा च \textcolor{red}{रामेण रावणो हन्यते} इत्यादयः प्रयोगा न निष्पत्स्यन्ते। न च यथा द्वितीय\-चकारेण \textcolor{red}{कर्तरि} इत्यस्यानुकर्षणं\footnote{\textcolor{red}{कर्तरि कृत्‌} (पा॰सू॰~३.४.६७) इत्यस्मात्।} तथैव \textcolor{red}{कर्मणि} इत्यस्य\footnote{\textcolor{red}{उपमाने कर्मणि च} (पा॰सू॰~३.४.४५) इत्यस्मान्मण्डूकप्लुत्या।} कथं नेति चेत् \textcolor{red}{कर्मणि} इत्यस्य पुनर्वैयर्थ्यं सत्यम्। उत्तर\-सूत्रेऽनुवृत्त्यर्थमिदम्। न च \textcolor{red}{सकर्मकेभ्यः कर्तरि कर्मणि च लकाराः} इति कुत आयातं सूत्रे। अकर्मकेभ्यो भावे लकारविधानात् \textcolor{red}{कर्मणि} इति पद\-ग्रहणेन तत्र लकारस्य प्राप्तत्वात्कर्मणश्चाकर्मक\-धातुष्वसम्भवात् \textcolor{red}{सकर्मकेभ्यः} इत्यस्य सुतरां लाभः। \textcolor{red}{नास्ति कर्म येषां त अकर्मकाः कर्मणा सहिताः सकर्मकाः}। यत्र फलव्यापारौ द्वावपि धातु\-वाच्यावेकस्मिन्नधिकरणे तिष्ठतः स एवाकर्मको यत्र च फल\-व्यापारौ धात्वर्थौ विरुद्धमधिकरणमधि\-तिष्ठतस्तत्रैव सकर्मकत्वम्। यद्यपि प्राचीना बालोपलालनार्थमिमां कारिकामपाठिषुर्यत्~–\end{sloppypar}
\centering\textcolor{red}{लज्जासत्तास्थितिजागरणं वृद्धिक्षयभयजीवितमरणम्।\nopagebreak\\
शयनक्रीडारुचिदीप्त्यर्थं धातुगणं तमकर्मकमाहुः॥}\footnote{मूलं मृग्यम्।}\\
\begin{sloppypar}\hyphenrules{nohyphenation}\justifying\noindent इति। किन्त्विदं न्यूनम्। \textcolor{red}{रामः पश्यति}, \textcolor{red}{रावणो माद्यति}, \textcolor{red}{लक्ष्मणः परिश्राम्यति}, \textcolor{red}{रामो हसति} इत्यादीनामसङ्ग्रहात्। अतो व्यवस्थितं लक्षणम् \textcolor{red}{स्वार्थ\-व्यापार\-समानाधिकरण\-फल\-वाचकत्वमकर्मकत्वम्‌} इति। कर्म\-वाच्ये कर्मणि लकारे \textcolor{red}{नौका आरुह्यताम्‌} इत्येव वरमुक्ते कर्मणि प्रथमानियमादिति चेदुच्यते। \textcolor{red}{आरुह्यताम्‌} 
इत्यविवक्षित\-कर्माकर्मक\-धातु\-प्रयोगः। \textcolor{red}{नौकां दृष्ट्वा आरुह्यताम्‌} यद्वा \textcolor{red}{नौकां सनाथयिष्यताऽऽरुह्यतां भवता} इति तात्पर्यम्। यद्वा \textcolor{red}{आरुह्य} इति ल्यबन्तः प्रयोगः \textcolor{red}{तां नौकां आरुह्य स्वामिन् पारं व्रज} इति निदर्शयत्यन्यथा \textcolor{red}{स्वामिन्‌} इति कथयित्वा कर्तृ\-सम्बोधनं पुनः \textcolor{red}{आरुह्यताम्‌} इति कर्म\-वाच्यं कथं प्रयुञ्जीत। न च \textcolor{red}{स्वामिन्। भवता आरुह्यताम्} इति चेत्। \textcolor{red}{सम्भवत्येक\-वाक्यत्वे वाक्य\-भेदो न युज्यते} (श्लो॰वा॰~१.९) इति वचन\-बलेन निरर्थक\-वाक्य\-भेदस्यानौचित्यात् \textcolor{red}{स्वामिन्‌} इति शब्देन सह ल्यबन्त\-प्रयोगमेव रोचयामहे। तद्योगे च \textcolor{red}{तां नौकाम्‌} इति कर्मणि द्वितीया साध्वी। यद्वा \textcolor{red}{आरोहणमित्यारुट्‌} इति विग्रहे \textcolor{red}{आङ्‌}\-पूर्वक\-\textcolor{red}{रुह्‌}\-धातोः (\textcolor{red}{रुहँ बीजजन्मनि प्रादुर्भावे च} धा॰पा॰~८५९) क्विप्‌।\footnote{\textcolor{red}{आङ्‌}\-पूर्वकात् \textcolor{red}{रुह्‌}\-धातोः \textcolor{red}{सम्पदादिभ्‍यः क्विप्‌} (वा॰~३.३.१०८) इत्यनेन भावे क्विप्। सर्वापहारि\-लोपे कित्त्वाल्लघूपध\-गुण\-निषेधे \textcolor{red}{आरुह्} इति प्रातिपदिके जाते विभक्ति\-कार्ये सौ प्रत्यये \textcolor{red}{हल्ङ्याब्भ्यो दीर्घात्सुतिस्यपृक्तं हल्} (पा॰सू॰~६.१.६८) इत्यनेन सोर्लोपे \textcolor{red}{हो ढः} (पा॰सू॰~८.२.३१) इत्यनेन ढत्वे \textcolor{red}{झलां जशोऽन्ते} (पा॰सू॰~८.२.३९) इत्यनेन जश्त्वे \textcolor{red}{वाऽवसाने} (पा॰सू॰~८.४.५६) इत्यनेन वैकल्पिक\-चर्त्वे \textcolor{red}{आरुट्} इति सिध्यति।} ततः \textcolor{red}{सुप आत्मनः क्यच्‌} (पा॰सू॰~३.१.८) इत्यनेन \textcolor{red}{आत्मन आरुहमिच्छति} इति विग्रहे \textcolor{red}{आरुह्यति}। \textcolor{red}{सनाद्यन्ता धातवः} (पा॰सू॰~३.१.३२) इत्यनेन धातु\-सञ्ज्ञायां तस्यैव लोड्लकार\-प्रथम\-पुरुषैक\-वचनान्तं रूपम् \textcolor{red}{आरुह्य} अतो द्वितीया। यद्वाऽऽकृति\-गणत्वात् \textcolor{red}{आङ्‌}\-पूर्वको \textcolor{red}{रुह्‌}\-धातुर्दिवादि\-गणे पठ्यताम्।\footnote{\textcolor{red}{बहुलमेतन्निदर्शनम्‌} (धा॰पा॰ ग॰सू॰~१९३८) \textcolor{red}{आकृतिगणोऽयम्‌} (धा॰पा॰ ग॰सू॰~१९९२) \textcolor{red}{भूवादिष्वेतदन्तेषु दशगणीषु धातूनां पाठो निदर्शनाय तेन स्तम्भुप्रभृतयः सौत्राश्चुलुम्पादयो वाक्यकारीयाः प्रयोगसिद्धा विक्लवत्यादयश्च} (मा॰धा॰वृ॰~१०.३२८) इत्यनुसारमाकृति\-गणत्वाद्दिवादि\-गणेऽप्यूह्योऽयं धातुः।} तथा च धातु\-सञ्ज्ञायां लड्लकारे तिप्प्रत्यये \textcolor{red}{दिवादिभ्यः श्यन्‌} (पा॰सू॰~३.१.६९) इत्यनेन \textcolor{red}{श्यन्‌} विकरणेऽनुबन्ध\-कार्ये \textcolor{red}{आरुह्यति} इति रूपं भवति\footnote{आ~रुह्~\arrow \textcolor{red}{शेषात्कर्तरि परस्मैपदम्} (पा॰सू॰~१.३.७८)~\arrow \textcolor{red}{वर्तमाने लट्} (पा॰सू॰~३.२.१२३)~\arrow आ~रुह्~लट्~\arrow आ~रुह्~तिप्~\arrow आ~रुह्~ति~\arrow \textcolor{red}{दिवादिभ्यः श्यन्} (पा॰सू॰~३.१.६९)~\arrow आ~रुह्~श्यन्~ति~\arrow \textcolor{red}{सार्वधातुकमपित्} (पा॰सू॰~१.२.४)~\arrow ङित्त्वम्~\arrow \textcolor{red}{ग्क्ङिति च} (पा॰सू॰~१.१.५)~\arrow लघूपधगुणनिषेधः~\arrow आ~रुह्~य~ति~\arrow आरुह्यति।} तस्यैव लोड्लकारे प्रथम\-पुरुषैक\-वचने रूपम् \textcolor{red}{आरुह्य} इति।\footnote{आ~रुह्~\arrow \textcolor{red}{शेषात्कर्तरि परस्मैपदम्} (पा॰सू॰~१.३.७८)~\arrow \textcolor{red}{लोट् च} (पा॰सू॰~३.३.१६२)~\arrow आ~रुह्~लोट्~\arrow आ~रुह्~सिप्~\arrow आ~रुह्~सि~\arrow \textcolor{red}{दिवादिभ्यः श्यन्} (पा॰सू॰~३.१.६९)~\arrow आ~रुह्~श्यन्~सि~\arrow \textcolor{red}{सार्वधातुकमपित्} (पा॰सू॰~१.२.४)~\arrow ङित्त्वम्~\arrow \textcolor{red}{ग्क्ङिति च} (पा॰सू॰~१.१.५)~\arrow लघूपधगुणनिषेधः~\arrow आ~रुह्~य~सि~\arrow \textcolor{red}{सेर्ह्यपिच्च} (पा॰सू॰~३.४.८७)~\arrow आ~रुह्~य~हि~\arrow \textcolor{red}{अतो हेः} (पा॰सू॰~६.४.१०५)~\arrow आ~रुह्~य~\arrow आरुह्य।} यद्वाऽत्र \textcolor{red}{कर्तरि कर्म\-व्यतिहारे} (पा॰सू॰~१.३.१४) इत्यनेनात्मने\-पदम्।\footnote{अस्यार्थः। निषादो विवक्षति यत्स्वामिन् भवान् मां सुदृढनावमयाचत (\textcolor{red}{उवाच शीघ्रं सुदृढां नावमानय मे सखे} अ॰रा॰~२.६.१७)। सैवानीता मया (\textcolor{red}{स्वयमेव दृढं नावमानिनाय सुलक्षणाम्‌} अ॰रा॰~२.६.१८)। परन्त्वियं न राजनौः। भवान् हि ज्येष्ठ\-राजपुत्रः। अतो भवतः कृते राजनौरेवोचिता। ज्येष्ठ\-राजपुत्रो भूत्वाऽपि भवान् सुदृढनावमेवारोहति न राजनावम्। अयमेव कर्मव्यतिहारः। तं ध्वनयितुं निषाद आत्मनेपदं प्रयुङ्क्ते। न च निषादस्य राजनौर्न वर्तत इति चेत्। भरत\-शत्रुघ्न\-कौसल्या\-वसिष्ठानां कृते स एव निषादो राजनावमानेष्यति। यथाऽष्टमे सर्गे~– \textcolor{red}{इत्युक्त्वा त्वरितं गत्वा नावः पञ्चशतानि ह॥ समानयत्ससैन्यस्य तर्तुं गङ्गां महानदीम्। स्वयमेवानिनायैकां राजनावं गुहस्तदा॥ आरोप्य भरतं तत्र शत्रुघ्नं राममातरम्। वसिष्ठं च तथाऽन्यत्र कैकेयीं चान्ययोषितः॥} (अ॰रा॰~२.८.३८.४०) इत्यत्र तस्य राजनौकानयनं वर्णितमस्ति।} एवं लोड्लकारे प्रथम\-पुरुषैक\-वचने \textcolor{red}{त}\-प्रत्यये \textcolor{red}{टित आत्मने\-पदानां टेरे} (पा॰सू॰~३.४.७९) इत्यनेनैत्वे \textcolor{red}{आमेतः} (पा॰सू॰~३.४.९०) इत्यनेन \textcolor{red}{आम्‌} आदेशे \textcolor{red}{आरुह्यताम्‌}।\footnote{आ~रुह्~\arrow \textcolor{red}{कर्तरि कर्म\-व्यतिहारे} (पा॰सू॰~१.३.१४)~\arrow \textcolor{red}{लोट् च} (पा॰सू॰~३.३.१६२)~\arrow आ~रुह्~लोट्~\arrow आ~रुह्~त~\arrow \textcolor{red}{दिवादिभ्यः श्यन्} (पा॰सू॰~३.१.६९)~\arrow आ~रुह्~श्यन्~त~\arrow \textcolor{red}{सार्वधातुकमपित्} (पा॰सू॰~१.२.४)~\arrow ङित्त्वम्~\arrow \textcolor{red}{ग्क्ङिति च} (पा॰सू॰~१.१.५)~\arrow लघूपधगुणनिषेधः~\arrow आ~रुह्~य~त~\arrow \textcolor{red}{टित आत्मने\-पदानां टेरे} (पा॰सू॰~३.४.७९)~\arrow आ~रुह्~य~ते~\arrow \textcolor{red}{आमेतः} (पा॰सू॰~३.४.९०)~\arrow आ~रुह्~य~ताम्~\arrow आरुह्यताम्।} \textcolor{red}{स्वामिन् भवान् नौकाम् आरुह्यताम्‌} इत्यनुक्ते कर्मणि द्वितीया।\end{sloppypar}
\section[बहिर्वनस्य]{बहिर्वनस्य}
\label{sec:bahirvanasya}
\centering\textcolor{blue}{रामो दाशरथिः सीतालक्ष्मणाभ्यां समन्वितः।\nopagebreak\\
आस्ते बहिर्वनस्येति ह्युच्यतां मुनिसन्निधौ॥}\nopagebreak\\
\raggedleft{–~अ॰रा॰~२.६.३०}\\
\begin{sloppypar}\hyphenrules{nohyphenation}\justifying\noindent\hspace{10mm} नौकया जाह्नवीमवतीर्य सौमित्रि\-सीतानुचरः पद\-चरः श्रीरामो भरद्वाज\-चरण\-कमलं दिदृक्षुर्दृष्टिं पातयन् भरद्वाज\-शिष्यं प्रत्यात्मनो दर्शनेच्छां भारद्वाजाय ज्ञापयन् ब्रूते यत्सीता\-लक्ष्मण\-सहायो रामो भवद्दर्शनं चिकीर्षन् \textcolor{red}{बहिर्वनस्य} तिष्ठन् प्रतीक्षत इति। तत्रैव \textcolor{red}{बहिर्वनस्य} इति प्रयोगं करोति। बहिर्योगे पञ्चमीति \textcolor{red}{अपपरिबहिरञ्चवः पञ्चम्या} (पा॰सू॰~२.१.१२) इति सूत्रेण ज्ञापिता। अत्र पञ्चमी प्रयोक्तव्याऽऽसीत्किन्तु षष्ठी प्रयुक्ता। किमिदमार्षम्। वस्तुतस्त्विदमपि पाणिनीयम्। \textcolor{red}{आस्ते बहिर्वनस्य} इत्यत्र \textcolor{red}{वनस्य बहिःप्रदेश आस्ते} इति योजनीयम्। इत्थं चावयवावयवि\-भाव\-सम्बन्धमूलिका षष्ठी। यद्वा \textcolor{red}{बहिर्योगे पञ्चमी} इति वचनं ज्ञापक\-मूलकम्। यतो हि \textcolor{red}{अपपरिबहिरञ्चवः पञ्चम्या} (पा॰सू॰~२.१.१२) इति समास\-विधायकेन सूत्रेण \textcolor{red}{बहिः} शब्दस्य पञ्चम्यन्त\-प्रादिपदिकेन समासो ज्ञाप्यते। \textcolor{red}{वनाद्बहिरिति बहिर्वनम्‌}। तेनैव बहिर्योगे पञ्चमी ज्ञापिता। यदि बहिर्योगे पञ्चमी न स्यात्तर्हि तत्र योगे समास\-विधानमपि न स्याद्यतः समास\-विधानं ततः पञ्चमीत्यन्वय\-व्यतिरेकाभ्यां बहिर्योगे पञ्चमी साधिता। \textcolor{red}{ज्ञापक\-सिद्धं न सर्वत्र}\footnote{मूलं मृग्यम्।} इति नियमेन नियमोऽयं ज्ञापक\-सिद्धतया न सार्वत्रिकः।\footnote{\textcolor{red}{अपपरिबहिः। अपपरियोगे ‘पञ्चम्यपाङ्परिभिः’ (पा॰सू॰~२.३.१०) इति पञ्चमी विहिता। अञ्चूत्तरपदयोगेऽपि ‘अन्यारात्’ (पा॰सू॰~२.३.२९) इत्यादिना विहितैव। तेनाऽत्र ‘पञ्चम्या’ इति ग्रहणं ‘बहिर्योगे पञ्चमी भवति’ इति ज्ञापनार्थम्। ‘ज्ञापकसिद्धं न सर्वत्र’ इति ‘करस्य करभो बहिः’ (अ॰को॰~२.६.८०क) इत्यपि सिद्धम्‌} (त॰बो॰~६६६)।} अतः षष्ठी। यद्वा \textcolor{red}{विवक्षाधीनानि कारकाणि भवन्ति}\footnote{मूलं मृग्यम्। यद्वा \textcolor{red}{कर्मादीनामविवक्षा शेषः} (भा॰पा॰सू॰~२.३.५०, २.३.५२, २.३.६७) इत्यस्य तात्पर्यमिदम्।} इति नियमेन षष्ठी। यद्वा पञ्चम्या सह बहिःशब्दस्य समासो ज्ञापितः। किन्तु \textcolor{red}{बहिर्योगे पञ्चम्येव भवतु} इति नेतरा विभक्तयो निषिद्धाः। केवलं बहिः\-शब्दस्येतर\-विभक्त्यन्त\-योग\-समास एव नियन्त्रितस्तस्मादर्थानुरोधेनात्र षष्ठी साधीयसी। पञ्चम्यां सत्यां समासाग्रहो गले पतितः षष्ठ्यां तु समास आग्रहो नास्ति। इयान् विशेषः।\end{sloppypar}
\section[भरद्वाजाय मुनये]{भरद्वाजाय मुनये}
\centering\textcolor{blue}{सभार्यः सानुजः श्रीमानाह मां देवसन्निभः।\nopagebreak\\
भरद्वाजाय मुनये ज्ञापयस्व यथोचितम्॥}\nopagebreak\\
\raggedleft{–~अ॰रा॰~२.६.३२}\\
\begin{sloppypar}\hyphenrules{nohyphenation}\justifying\noindent\hspace{10mm} अत्र स्वकीयं सन्देश\-प्रकारं तन्वन् श्रीरामो ब्रवीति \textcolor{red}{भरद्वाजाय मुनये} ज्ञापयस्व। अत्र \textcolor{red}{भरद्वाजो जानातु त्वं प्रेरय} इत्यर्थे \textcolor{red}{भरद्वाजं ज्ञापयस्व}। यतो ह्यण्यन्तावस्थायाः कर्ता ण्यन्ते कर्म भवति। तथा च सूत्रं \textcolor{red}{गति\-बुद्धि\-प्रत्यवसानार्थ\-शब्दकर्माकर्मकाणामणि कर्ता स णौ} (पा॰सू॰~१.४.५२)। अत्र \textcolor{red}{ज्ञा}\-धातुः (\textcolor{red}{ज्ञा अवबोधने} धा॰पा॰~१५०७) बुद्ध्यर्थस्तस्याण्यन्तावस्थायां कर्ता भरद्वाज इति सम्प्रति ण्यन्तावस्थायां कर्म भवत्वित्येव पाणिनीयम्। परमत्र चतुर्थ्यपि पाणिनीयैव। \textcolor{red}{भरद्वाजाय मुनय आनन्दं दातुं ज्ञापयस्व} इति दा\-धातु\-योगे (\textcolor{red}{डुदाञ् दाने} धा॰पा॰~१०९१) चतुर्थी।\footnote{\textcolor{red}{आनन्दं दातुम्} इत्यध्याहार्यमिति भावः।} यद्वा \textcolor{red}{भरद्वाजाय मुनये रुचितं ज्ञापयस्व} इति रुचित\-पदाध्याहारे \textcolor{red}{रुच्यर्थानां प्रीयमाणः} (पा॰सू॰~१.४.३३) इत्यनेन सम्प्रदान\-सञ्ज्ञा ततश्चतुर्थी। यद्वा \textcolor{red}{भरद्वाजं मुनिं हर्षयितुं ज्ञापयस्व} इत्यप्रयुज्यमान\-तुमुन्\-कर्मणि चतुर्थी \textcolor{red}{क्रियार्थोपपदस्य च कर्मणि स्थानिनः} (पा॰सू॰~२.३.१४) इत्यनेन।\end{sloppypar}
\section[सीतया लक्ष्मणेन च]{सीतया लक्ष्मणेन च}
\label{sec:sitaya_lakmanena_ca}
\centering\textcolor{blue}{ततो राजा नमन्तं तं सुमन्त्रं विह्वलोऽब्रवीत्।\nopagebreak\\
सुमन्त्र रामः कुत्रास्ते सीतया लक्ष्मणेन च॥}\nopagebreak\\
\raggedleft{–~अ॰रा॰~२.७.३}\\
\begin{sloppypar}\hyphenrules{nohyphenation}\justifying\noindent\hspace{10mm} श्रीरामं रथेन प्रयागं यावत्प्रेष्य प्रत्यागतमयोध्यां सुमन्त्रं प्रति राजा पप्रच्छ यत्सीतया लक्ष्मणेन च रामः कुत्रास्ते। अत्र तृतीया
विमर्शास्पदम्। यतो हि नहि काचिदत्र क्रिया यां प्रत्यनयोः कर्तृत्वं संसाध्य तयोस्तृतीयोपकल्प्यताम्।\footnote{\textcolor{red}{कर्तृ\-करणयोस्तृतीया} (पा॰सू॰~२.३.१८) इत्यनेन।} न वा कश्चन क्रिया\-सिद्धौ व्यापारोऽवशिष्यते यद्व्यापारादनन्तरं क्रिया परिनिष्पद्येत येन करणत्वं साध्यताम्। न वा किमपि 
व्यापार\-गीतं विलोक्यते येन हेतुत्वं साधयित्वा ततश्च तृतीया क्रियताम्।\footnote{\textcolor{red}{हेतौ} (पा॰सू॰~२.३.२३) इत्यनेन।} अत्रोच्यते। विनाऽपि सहार्थ\-वाचक\-शब्द\-योगेन तृतीया भवति \textcolor{red}{वृद्धो यूना तल्लक्षणश्चेदेव विशेषः} (पा॰सू॰~१.२.६५) इत्यादि\-निर्देशात्। यद्वा प्रयुक्त\-सह\-शब्दस्य \textcolor{red}{विनाऽपि प्रत्ययं पूर्वोत्तर\-पद\-लोपो वक्तव्यः} (वा॰~५.३.८३) इति वार्त्तिकेन \textcolor{red}{सह} इत्यस्य लोपे \textcolor{red}{यः शिष्यते स लुप्यमानार्थामिधायी} इति नियमेन लुप्यमान\-सहार्थावगतेस्तृतीया। यद्वा \textcolor{red}{प्रकृत्यादिभ्य उप\-सङ्ख्यानम्‌} (वा॰~२.३.१८) इति वार्त्तिकेन प्रकृत्यादित्वादभेदार्थे तृतीया। \textcolor{red}{प्रकृत्या चारु} इतिवत्। सीतालक्ष्मणाभिन्नो राम इति भावः। सीतया लक्ष्मणेन च श्रीरामस्याभेद\-प्रसिद्धेः। वाल्मीकीय\-रामायणे द्वाभ्यामपि श्रीसीता\-रामाभ्यामभेदस्य प्रतिपादितत्वात्~–\end{sloppypar}
\centering\textcolor{red}{अनन्या हि मया सीता भास्करेण यथा प्रभा॥}\nopagebreak\\
\raggedleft{–~वा॰रा॰~६.११८.१९}\\
\begin{sloppypar}\hyphenrules{nohyphenation}\justifying\noindent इति रामवचनम्।\end{sloppypar}
\centering\textcolor{red}{अनन्या राघवेणाहं भास्करेण यथा प्रभा॥}\nopagebreak\\
\raggedleft{–~वा॰रा॰~५.२१.१५}\\
\begin{sloppypar}\hyphenrules{nohyphenation}\justifying\noindent इति सीतावचनम्। उभयत्र प्रयुक्तोऽनन्य\-शब्दोऽभावाभाव\-प्रतिपादन\-परः। अभावाभावो हि प्रतियोगि\-ज्ञानम्। यथा घटाभावाभावो घट एव। तथैव \textcolor{red}{न अन्या अनन्या}।\footnote{\textcolor{red}{नञ्‌} (पा॰सू॰~२.२.६) इत्यनेन समासे \textcolor{red}{नलोपो नञः} (पा॰सू॰~६.३.७३) इत्यनेन नकार\-लोपे \textcolor{red}{तस्मान्नुडचि} (पा॰सू॰~६.३.७४) इत्यनेन नुँट्।} \textcolor{red}{अन्या रामाद्भिन्ना}। \textcolor{red}{न अन्या अन्यभिन्ना अर्थादभिन्नैव}~–\end{sloppypar}
\centering\textcolor{red}{प्रभा जाइ कहँ भानु बिहाई। कहँ चन्द्रिका चन्द्र तजि जाई॥}\footnote{एतद्रूपान्तरम्–\textcolor{red}{कुत्र प्रगन्तुं शक्नोति प्रभा त्यक्त्वा प्रभाकरम्। त्यक्त्वा चन्द्रमसं कुत्र गन्तुं शक्नोति चन्द्रिका॥} (मा॰भा॰~२.९७.६)।}\nopagebreak\\
\raggedleft{–~रा॰च॰मा॰~२.९७.६}\\
\begin{sloppypar}\hyphenrules{nohyphenation}\justifying\noindent अन्यत्वं हि भेदे सिध्यत्यनन्यत्वमभेदे।\end{sloppypar}
\centering\textcolor{red}{गिरा अरथ जल बीचि सम देखियत भिन्न न भिन्न।}\footnote{एतद्रूपान्तरम्–\textcolor{red}{वागर्थतुल्यौ जलवीचितुल्यौ वाच्यौ पृथक्किन्तु विभेदशून्यौ} (मा॰भा॰~१.१८)।}\nopagebreak\\
\raggedleft{–~रा॰च॰मा॰~१.१८}\\
\begin{sloppypar}\hyphenrules{nohyphenation}\justifying\noindent इति गोस्वामिनाऽपि तत्र तत्र मानसेऽपि प्रतिपादितत्वात्। अतः प्रकृत्यादित्वादभेदे तृतीया स्वर्ण\-सौरभ\-संयोग इवोत्कर्षमाटीकते। करपात्र\-स्वामि\-चरणैरपि रामायण\-मीमांसा\-मङ्गलाचरणे लिखितं यत्~–\end{sloppypar}
\centering\textcolor{red}{सौन्दर्यसारसर्वस्वं माधुर्यगुणबृंहितम्।\nopagebreak\\
ब्रह्मैकमद्वितीयं तत्तत्त्वमेकं द्विधा कृतम्॥}\nopagebreak\\
\raggedleft{–~रा॰मी॰~मङ्गलाचरणे}\\
\begin{sloppypar}\hyphenrules{nohyphenation}\justifying\noindent लक्ष्मणस्याभेदत्वं तु शेषत्वात्। \textcolor{red}{शेषस्तु लक्ष्मणो राजन्‌} (अ॰रा॰~१.४.१७) इत्यस्मिन् ग्रन्थ एवोक्तत्वाच्छेषोंऽशः श्रीरामोंऽश्यंशांशिनोर्बिन्दु\-सिन्धोरिवाभेद\-प्रसिद्धेः। सीता\-लक्ष्मणयोः प्रकृत्यादि\-गणे पठितत्वौचित्यं हि सीताया मूल\-प्रकृतित्वाल्लक्ष्मणस्य च जीवाचार्यत्वात्। यद्वा \textcolor{red}{इत्थं\-भूत\-लक्षणे} (पा॰सू॰~२.३.२१) इदं हि सूत्रम्। अस्यार्थः किञ्चित्प्रकारस्य लक्षणे तृतीया। अर्थात् \textcolor{red}{इत्थं भूतं लक्षयतीति इत्थं\-भूत\-लक्षणं तस्मिन्‌}। कर्तरि बाहुलकाल्ल्युँट्।\footnote{\textcolor{red}{कृत्यल्युटो बहुलम्} (पा॰सू॰~३.३.११३) इत्यनेन।} ततश्चानुबन्ध\-कार्ये सति \textcolor{red}{युवोरनाकौ} (पा॰सू॰~७.१.१) इत्यनेनानादेशे \textcolor{red}{अट्कुप्वाङ्नुम्व्यवायेऽपि} (पा॰सू॰~८.४.२) इत्यनेन णत्वम्। यद्वा \textcolor{red}{लक्ष्यतेऽनेनेति लक्षणम्‌}। \textcolor{red}{करणाधिकरणयोश्च} (पा॰सू॰~३.३.११७) इति करणे ल्युट्। \textcolor{red}{इत्थं\-भूतस्य लक्षणमित्थं\-भूत\-लक्षणम्} इति षष्ठी\-तत्पुरुषः।\footnote{\textcolor{red}{कर्तृकर्मणोः कृति} (पा॰सू॰~२.३.६५) इत्यनेन कृद्योगा षष्ठी \textcolor{red}{कृद्योगा च षष्ठी समस्यत इति वक्तव्यम्‌} (वा॰~२.२.८) इत्यनेन समासः।} अर्थाद्येन व्यापक\-चिह्न\-विशेषेण व्यक्तेः कश्चित्प्रकारो द्योत्यते तत्र तृतीया स्यात्। एवं हि \textcolor{red}{इत्थं\-भूत\-लक्षणे} इत्यनेनैव तृतीया। रामचन्द्रस्य रामत्वं हि सीता\-लक्ष्मणाभ्यामेव सूच्यते। यथा जटाभिस्तापसः। मौञ्जी\-मेखलया ब्रह्मचारी। अत्र जटा\-ज्ञाप्य\-तापसत्व\-विशिष्टस्तापस एवं मौञ्जी\-मेखला\-ज्ञाप्य\-ब्रह्मचारित्व\-विशिष्टो ब्रह्मचारी। तथैवात्र सीता\-लक्ष्मण\-ज्ञाप्य\-रामत्व\-विशिष्टः श्रीरामः। रामत्वं हि सीता\-लक्ष्मणाभ्यामेव ज्ञाप्यते। रामेति नामन्यपि वर्ण\-त्रय\-विवरणं कृत्वा ब्रह्म\-माया\-जीवानिव राम\-सीता\-लक्ष्मणानवगच्छन्ति भावुक\-भक्ताः। एवं रकारे श्रीराम आकारे श्रीसीता मकारे च श्रीलक्ष्मणः। यथा~–\end{sloppypar}
\centering\textcolor{red}{रकारे रामचन्द्रः स्यान्मकारे लक्ष्मणः स्वराट्।\nopagebreak\\
तयोः संयोजनार्थाय सीता आकार उच्यते॥}\nopagebreak\\
\raggedleft{–~इति साम्प्रदायिकाः}\\
\begin{sloppypar}\hyphenrules{nohyphenation}\justifying\noindent इत्थं \textcolor{red}{सीतया लक्ष्मणेन च} इति द्वावपि पाणिनीयौ।\end{sloppypar}
\section[तयोः]{तयोः}
\centering\textcolor{blue}{तयोस्त्वमुदकं देहि शीघ्रमेवाविचारयन्।\nopagebreak\\
न चेत्त्वां भस्मसात्कुर्यात्पिता मे यदि कुप्यति॥}\nopagebreak\\
\raggedleft{–~अ॰रा॰~२.७.२८}\\
\begin{sloppypar}\hyphenrules{nohyphenation}\justifying\noindent\hspace{10mm} म्रियमाणो दशरथः शरीर\-रक्षायै राम\-मात्रा प्रार्थितः। मरणमपरि\-वर्तनीयमिति साधयन् प्रसङ्गोपात्त\-श्रमण\-कुमार\-कथां प्रस्तुवन् शब्द\-वेधि\-बाण\-विक्रियमाण\-महा\-प्रयाण\-श्रमण\-वचनमानुपूर्वीतयाऽनुवदति यत्प्राणांस्त्यजञ्छ्रमणो दशरथं कथयति \textcolor{red}{विचारं न कृत्वा तयोरुदकं देहि}।
अर्थान्मम पितृभ्यां जलमर्पयेति। अभीप्सिते \textcolor{red}{ताभ्याम्‌} इति चतुर्थ्यन्ते प्रयोक्तव्ये \textcolor{red}{तयोः} इति षष्ठ्यन्तं प्रयुक्तम्। अत्र \textcolor{red}{विवक्षाधीनानि कारकाणि भवन्ति}\footnote{मूलं मृग्यम्। यद्वा \textcolor{red}{कर्मादीनामविवक्षा शेषः} (भा॰पा॰सू॰~२.३.५०, २.३.५२, २.३.६७) इत्यस्य तात्पर्यमिदम्।} इति नियमेन सम्बन्ध\-विवक्षणात्षष्ठी। अथवा \textcolor{red}{तयोः} इत्यस्य \textcolor{red}{त्वम्‌} शब्देन सहान्वयः। अत्र च रक्ष्य\-रक्षक\-भाव\-सम्बन्ध\-मूलिका षष्ठी।\footnote{षड्विंशतितमे श्लोके दशरथं सम्बोधयन् श्रमणः कथयति~– \textcolor{red}{मा भैषीर्नृपसत्तम} (अ॰रा॰~२.७.२६)। नॄन् नरान् पातीति नृपः। यथाह मनुः~– \textcolor{red}{ब्राह्मं प्राप्तेन संस्कारं क्षत्त्रियेण यथाविधि। सर्वस्यास्य यथान्यायं कर्तव्यं परिरक्षणम्॥ अराजके हि लोकेऽस्मिन्सर्वतो विद्रुतो भयात्। रक्षार्थमस्य सर्वस्य राजानमसृजत्प्रभुः॥} (म॰स्मृ॰~७.२.३)। श्रमणमते दशरथस्तु नृपसत्तमः। अर्थान्नररक्षितृश्रेष्ठः। अत एव तेन \textcolor{red}{तयोस्त्वमुदकं देहि} इत्यत्र रक्ष्य\-रक्षक\-भाव\-सम्बन्ध\-मूलिका षष्ठी प्रयुक्ता।} यद्वा पित्रोर्म्रियमाणत्वात्तत्र सम्प्रदान\-विवक्षैव न। यतो हि पुत्र\-प्रदत्तोदकमेव पितरौ सम्यग्गृह्णीतः। अतः \textcolor{red}{तयोः} इति षष्ठी न पाणिनीय\-विरुद्धा।\end{sloppypar}
\section[देह्यावयोः]{देह्यावयोः}
\centering\textcolor{blue}{देह्यावयोः सुपानीयं पिब त्वमपि पुत्रक।\nopagebreak\\
इत्येवं लपतोर्भीत्या सकाशमगमं शनैः॥}\nopagebreak\\
\raggedleft{–~अ॰रा॰~२.७.३४}\\
\begin{sloppypar}\hyphenrules{nohyphenation}\justifying\noindent\hspace{10mm} अत्र श्रमण\-कुमारे मृते हस्ते कलशं गृहीत्वा निकटं गच्छतो दशरथस्य चरण\-सञ्चार\-ध्वनिं श्रुत्वा श्रमण\-कुमार\-बुद्ध्याऽन्धौ पितरौ प्राहतुर्यत् \textcolor{red}{आवयोः सुपानीयं देहि}। अत्र \textcolor{red}{दा}\-धातु\-प्रयोगे (\textcolor{red}{डुदाञ् दाने} धा॰पा॰~१०९१) \textcolor{red}{कर्मणा यमभिप्रैति स सम्प्रदानम्‌} (पा॰सू॰~१.४.३२) इत्यनेन सम्प्रदान\-सञ्ज्ञा तदनुगामिनी चतुर्थी\-विभक्तिश्च प्रयोक्तव्या। \textcolor{red}{आवाभ्यां सुपानीयं देहि} इत्येव वक्तव्यम्। परमत्र षष्ठी पाणिनीय\-विरुद्धेव। अथोच्यते। वात्सल्य\-धिया पितरौ सम्प्रदानं न विवक्षतः। अर्थाज्जलदानं तु तव कर्तव्यं नैव ते कृपा न वा नैमित्तिकं कर्म। अत आवयोरुदकं देहीति भावं प्रकटयितुं सम्बन्ध\-विवक्षा। अथवा \textcolor{red}{आवयोर्मुख उदकं देहि} इति मुख\-शब्दस्याध्याहारेणावयवावयवि\-भावे सम्बन्ध\-षष्ठी। यद्वा \textcolor{red}{आवयोः} शब्दस्य \textcolor{red}{पुत्रक} शब्देन अन्वयः। अर्थात् \textcolor{red}{हे आवयोः पुत्रक सुपानीयं देहि त्वमपि 
पिब} इत्यन्वये \textcolor{red}{आवयोः} इत्यत्र जन्य\-जनक\-सम्बन्धे षष्ठी।\end{sloppypar}
\section[मे वद]{मे वद}
\centering\textcolor{blue}{त्वया विना न मे तातः कदाचिद्रहसि स्थितः।\nopagebreak\\
इदानीं दृश्यते नैव कुत्र तिष्ठति मे वद॥}\nopagebreak\\
\raggedleft{–~अ॰रा॰~२.७.६३}\\
\begin{sloppypar}\hyphenrules{nohyphenation}\justifying\noindent\hspace{10mm} अत्र मृते दशरथे दूत\-द्वारा गुरु\-सन्देशं प्राप्यायोध्यां प्रत्यागतो भरतः कैकेयी\-सकाशं गत्वाऽदृष्ट्वा पितरं साश्चर्यं पृच्छति यत् \textcolor{red}{कुत्र तिष्ठति मे वद}। अत्र \textcolor{red}{ब्रू}\-धातु\-समानार्थकतया (\textcolor{red}{ब्रूञ् व्यक्तायां वाचि} धा॰पा॰~१०४४) \textcolor{red}{अकथितं च} (पा॰सू॰~१.४.५१) इत्यनेन कर्म\-सञ्ज्ञा प्राप्ता \textcolor{red}{कर्मणि द्वितीया} (पा॰सू॰~२.३.२) इत्यनेन च द्वितीया प्रयोक्तव्या। एवं च \textcolor{red}{मां वद} इत्यनेनैव भवितव्यम्। तदुपर्युच्यते। अत्र सम्प्रदान\-विवक्षया चतुर्थी। यद्वा \textcolor{red}{मामनुग्रहीतुं वद} इति गम्यमान\-तुमुन्प्रयोगेण तस्य कर्मणि चतुर्थी \textcolor{red}{मह्यम्‌} इति \textcolor{red}{क्रियार्थोपपदस्य च कर्मणि स्थानिनः} (पा॰सू॰~२.३.१४) इत्यनेन तस्य च \textcolor{red}{मे} इत्यादेशः।\footnote{\textcolor{red}{तेमयावेकवचनस्य} (पा॰सू॰~८.१.२२) इत्यनेन।} यद्वा \textcolor{red}{मातः} इत्यध्याहृत्य तेन सहान्वये \textcolor{red}{हे मे मातः वद} इत्यर्थे सम्बन्ध\-सामान्ये षष्ठी।
\end{sloppypar}
\section[हेऽम्ब]{हेऽम्ब}
\centering\textcolor{blue}{तामाह भरतो हेऽम्ब रामः सन्निहितो न किम्।\nopagebreak\\
तदानीं लक्ष्मणो वाऽपि सीता वा कुत्र ते गताः॥}\nopagebreak\\
\raggedleft{–~अ॰रा॰~२.७.७१}\\
\begin{sloppypar}\hyphenrules{nohyphenation}\justifying\noindent\hspace{10mm} अत्र शोकाकुलित\-हृदयो भरतो मातरं पृच्छति यत् \textcolor{red}{हेऽम्ब राम\-लक्ष्मण\-सीताः कुत्र गताः}। अत्र \textcolor{red}{हैहेप्रयोगे हैहयोः} (पा॰सू॰~८.२.८५) इति सूत्रेण \textcolor{red}{हे}\-घटकैकारस्य प्लुते \textcolor{red}{प्लुतप्रगृह्या अचि नित्यम्‌} (पा॰सू॰~६.१.१२५) इत्यनेन प्रकृति\-भावे \textcolor{red}{हे३ अम्ब} इत्येव पाणिनीयं \textcolor{red}{हेऽम्ब} इति कथम्। उच्यते। \textcolor{red}{गुरोरनृतोऽनन्त्यस्याप्येकैकस्य प्राचाम्‌} (पा॰सू॰~८.२.८६) इत्यत्र \textcolor{red}{प्राचाम्‌} इत्यस्य ग्रहणं व्यर्थं सज्ज्ञापयति यत्सर्वोऽपि प्लुतो विकल्प्यते।\footnote{\textcolor{red}{इह प्राचामिति योगो विभज्यते। तेन सर्वः प्लुतो विकल्प्यते} (वै॰सि॰कौ॰~९७)।} तस्मादत्रापि प्लुत\-विकल्पः। एवं \textcolor{red}{हे} इत्यव्ययं मत्वा \textcolor{red}{अव्ययादाप्सुपः} (पा॰सू॰~२.४.८२) इत्यनेन विभक्ति\-लोपे पदान्तत्वादेकाराकारयोः \textcolor{red}{एङः पदान्तादति} (पा॰सू॰~६.१.१०९) इत्यनेन पररूपे \textcolor{red}{हेऽम्ब} इति सिद्धम्। यद्वा त्रिपादीस्थत्वात्सपाद\-सप्ताध्यायी\-पर\-रूप\-विधायक\-शास्त्र\-दृष्ट्येदमसिद्धं ततः पर\-रूपे \textcolor{red}{हेऽम्ब} इति सिद्धम्। न च प्लुतारम्भ\-सामर्थ्यादसिद्धत्वं लोपयितुं न शक्यते। \textcolor{red}{हे३ राम राम है३} इत्यादौ प्लुत\-स्वरे चारितार्थ्यात्। न च \textcolor{red}{प्रतिलक्ष्यं लक्षणोपप्लवः}\footnote{मूलं मृग्यम्।}
इति नियमेन प्रकृति\-भाव\-रूपे लक्ष्येऽचारितार्थ्यात्तद्वैयर्थ्ये नैवासिद्धत्वं रोधयिष्यते। अन्यत्र \textcolor{red}{हे३ अरि\-सूदन} \textcolor{red}{हे३ अम्ब} इत्यादि\-प्रयोगेषु च चारितार्थ्यात्।\footnote{संहिताया अविवक्षायामिति शेषः।}
यद्वा \textcolor{red}{प्लुत\-प्रगृह्या अचि नित्यम्‌} (पा॰सू॰~६.१.१२५) इत्यत्र नित्य\-ग्रहणस्य प्रायिकत्वात्प्रकृति\-भावाभावः।\footnote{\pageref{sec:munindraham}तमे पृष्ठे \ref{sec:munindraham} \nameref{sec:munindraham} इति प्रयोगस्य विमर्शं पश्यन्तु~– “यथा \textcolor{red}{ङमो ह्रस्वादचि ङमुण्नित्यम्‌} (पा॰सू॰~८.३.३२) इत्यत्र नित्यग्रहणस्य प्रायिकत्वात् \textcolor{red}{इको यणचि} (पा॰सू॰~६.१.७७) \textcolor{red}{सुप्तिङन्तं पदम्‌} (पा॰सू॰~१.४.१४) इत्यादौ न ङुण्मुटौ।”}\end{sloppypar}
\section[तव राज्यप्रदानाय]{तव राज्यप्रदानाय}
\centering\textcolor{blue}{रामस्य यौवराज्यार्थं पित्रा ते सम्भ्रमः कृतः।\nopagebreak\\
तव राज्यप्रदानाय तदाहं विघ्नमाचरम्॥}\nopagebreak\\
\raggedleft{–~अ॰रा॰~२.७.७२}\\
\begin{sloppypar}\hyphenrules{nohyphenation}\justifying\noindent\hspace{10mm} कैकयी भरतं प्रबोधयन्ती प्राह यत् \textcolor{red}{तव राज्य\-प्रदानायाहमेव राम\-राज्ये विघ्नमाचरम्‌}। अत्र दानयोगे चतुर्थी वक्तव्याऽऽसीत् \textcolor{red}{तुभ्यं राज्य\-प्रदानाय} इति। \textcolor{red}{तव} इति सम्बन्ध\-विवक्षायां षष्ठी। अर्थात् \textcolor{red}{न च राज्यमिदं परकीयं मम विवाहे पितुः पण आसीद्यन्मम दौहित्र एव राज्याधिकारी भवेत्तत इदं राज्यमधिकारतस्तव}। अस्मादधि\-कार्याधि\-कारक\-भावे\footnote{अधिक्रियते यत्तत् \textcolor{red}{अधिकार्यम्‌}। \textcolor{red}{तयोरेव कृत्यक्तखलर्थाः} (पा॰सू॰~३.४.७०), \textcolor{red}{ऋहलोर्ण्यत्‌} (पा॰सू॰~३.१.१२४) इत्याभ्याम् \textcolor{red}{अधि}पूर्वक\textcolor{red}{कृ}धातोः (\textcolor{red}{डुकृञ् करणे} धा॰पा॰~१४७२) कर्मणि कृत्यसञ्ज्ञको ण्यत्। अधिकरोतीति \textcolor{red}{अधिकारकः}। \textcolor{red}{अधि}पूर्वक\textcolor{red}{कृ}धातोः \textcolor{red}{ण्वुल्तृचौ} (पा॰सू॰~३.१.१३३) इत्यनेन कर्तरि ण्वुल्।} सम्बन्धे षष्ठी। न च \textcolor{red}{सापेक्षमसमर्थवत्‌} इति नियमेन \textcolor{red}{राज्य}\-शब्दस्य \textcolor{red}{तव}\-शब्देन सह साकाङ्क्षतया \textcolor{red}{तव} इत्यस्य च विशेषणत्वात् \textcolor{red}{राज्य\-प्रदानाय} इति कथं तत्पुरुष इति वाच्यम्। नित्य\-साकाङ्क्षा\-स्थले तस्य नियमस्यानङ्गीकारात्। \textcolor{red}{देवदत्तस्य गुरु\-कुलम्‌} (भा॰पा॰सू॰~२.१.१) इत्यादि\-भाष्य\-प्रयोगाच्च। यद्वा \textcolor{red}{कृते} इत्यध्याहार्यम्।
अर्थात् \textcolor{red}{तव कृते राज्य\-प्रदानाय} इति। सम्बन्धे षष्ठी।\end{sloppypar}
\section[तवैव]{तवैव}
\centering\textcolor{blue}{राज्यं रामस्य चैकेन वनवासो मुनिव्रतम्।\nopagebreak\\
ततः सत्यपरो राजा राज्यं दत्त्वा तवैव हि॥}\nopagebreak\\
\raggedleft{–~अ॰रा॰~२.७.७४}\\
\begin{sloppypar}\hyphenrules{nohyphenation}\justifying\noindent\hspace{10mm} अत्र कैकेयी राम\-वन\-वास\-घटनां वर्णयति। \textcolor{red}{तवैव राज्यं दत्त्वा पिता दिवं गतः}। \textcolor{red}{दत्त्वा} इति स्पष्टो \textcolor{red}{दा}\-धातोः (\textcolor{red}{डुदाञ् दाने} धा॰पा॰~१०९१) क्त्वान्त\-प्रयोगः। अत्र \textcolor{red}{तुभ्यम्‌} इति सम्प्रदानाच्चतुर्थीप्रयोग एव सम्यग्भाति। \textcolor{red}{तव} इति षष्ठी तु सम्बन्ध\-सामान्ये। यद्वा \textcolor{red}{तव कृते राज्यं दत्त्वा}।
यद्वा \textcolor{red}{तव शासनाय राज्यं दत्त्वा} इति।\footnote{\textcolor{red}{शासनाय} इत्यध्याहार्यमिति भावः। एवं \textcolor{red}{तव} इत्यत्र \textcolor{red}{कर्तृकर्मणोः कृति} (पा॰सू॰~२.३.६५) इत्यनेन षष्ठी।} यद्वा कैकेयी\-मुखात्सरस्वत्येव भरतं सङ्केतयति यदिदं राज्यं तुभ्यं महाराजेन न दत्तम्। यावद्रामागमनं शासनाय दत्तम्। अतस्तव तत्र स्वत्वं नास्ति। स्वत्वं तु राघवस्यैव। तस्मात् \textcolor{red}{रजकस्य वस्त्रं ददाति} इतिवदत्र राज्य\-परावर्तन\-सङ्केते षष्ठी।\end{sloppypar}
\section[मे]{मे}
\centering\textcolor{blue}{असम्भाष्याऽसि पापे मे घोरे त्वं भर्तृघातिनी।\nopagebreak\\
पापे त्वद्गर्भजातोऽहं पापवानस्मि साम्प्रतम्।\nopagebreak\\
अहमग्निं प्रवेक्ष्यामि विषं वा भक्षयाम्यहम्॥}\nopagebreak\\
\raggedleft{–~अ॰रा॰~२.७.८०}\\
\begin{sloppypar}\hyphenrules{nohyphenation}\justifying\noindent\hspace{10mm} कैकेयी\-मुखाद्राम\-वन\-वास\-कथा\-श्रवणेन जात\-कारुण्य\-कोपो भरतः कथयति \textcolor{red}{हे पापे त्वं मे असम्भाष्या}। अथ \textcolor{red}{ऋहलोर्ण्यत्‌} (पा॰सू॰~३.१.१२४) इत्यनेन \textcolor{red}{तयोरेव कृत्य\-क्त\-खलर्थाः} (पा॰सू॰~३.४.७०) इत्यनेन च सम्पूर्वक\-\textcolor{red}{भाष्‌}\-धातोः (\textcolor{red}{भाषँ व्यक्तायां वाचि} धा॰पा॰~६१२) कर्मणि ण्यत्प्रत्यये \textcolor{red}{चुटू} (पा॰सू॰~१.३.७) इत्यनेन णकारेत्सञ्ज्ञायां \textcolor{red}{तस्य लोपः} (पा॰सू॰~१.३.९) इत्यनेन लोपे \textcolor{red}{हलन्त्यम्‌} (पा॰सू॰~१.३.३) चेत्यनेन तकारेत्सञ्ज्ञायां तेनैव सूत्रेण लोपे विभक्ति\-कार्ये \textcolor{red}{सम्भाष्या}। \textcolor{red}{न सम्भाष्येत्यसम्भाष्या}।\footnote{\textcolor{red}{नञ्‌} (पा॰सू॰~२.२.६) इत्यनेन नञ्तत्पुरुषसमासः। \textcolor{red}{नलोपो नञः} (पा॰सू॰~६.३.७३) इत्यनेन नञो नलोपे \textcolor{red}{असम्भाष्या}।} \textcolor{red}{असम्भाष्या} इत्यत्र कर्मणि प्रत्यय\-विधानादनुक्तत्वाच्च कर्तुः \textcolor{red}{कर्तृकरणयोस्तृतीया} (पा॰सू॰~२.३.१८) इत्यनेन तृतीयया भवितव्यम्। \textcolor{red}{मया असम्भाष्यासि}। \textcolor{red}{मे} इति \textcolor{red}{कृत्यानां कर्तरि वा} (पा॰सू॰~२.३.७१) इत्यनेन वैकल्पकी षष्ठी। यद्वा \textcolor{red}{मे} इत्यस्मादग्रे \textcolor{red}{दृष्टौ} इत्यध्याहार्यम्। \textcolor{red}{पापे मे दृष्टावसम्भाष्या}। अत्राप्यवयवावयवि\-भाव\-सम्बन्धे षष्ठी।\end{sloppypar}
\section[मे]{मे}
\centering\textcolor{blue}{हा राम हा मे रघुवंशनाथ जातोऽसि मे त्वं परतः परात्मा।\nopagebreak\\
तथाऽपि दुःखं न जहाति मां वै विधिर्बलीयानिति मे मनीषा॥}\nopagebreak\\
\raggedleft{–~अ॰रा॰~२.७.८६}\\
\begin{sloppypar}\hyphenrules{nohyphenation}\justifying\noindent\hspace{10mm} अत्र शोक\-सन्तप्त\-हृदया भगवती कौसल्याम्बा श्रीभरत\-समक्षं श्रीराममुद्दिश्य विलपति। \textcolor{red}{परतः परात्मा त्वं मे जातोऽसि}। अत्र \textcolor{red}{मे} इति षष्ठ्यन्तं विचाराय। यतो हि जायमानस्याऽधारो माता। एवं पुत्र\-जन्मनि पिता निमित्त\-कारणं रजः\-शुक्र\-संयोगोऽसमवायि\-कारणमेवं माता समवायि\-कारणम्।\footnote{\textcolor{red}{कारणं त्रिविधं समवाय्यसमवायि\-निमित्त\-भेदात्। यत्समवेतं कार्यमुत्पद्यते तत्समवायि\-कारणम्। यथा तन्तवः पटस्य पटश्च स्वगतरूपादेः। कार्येण कारणेन वा सहैकस्मिन्नर्थे समवेतं सत्कारणमसमवायिकारणम्। यथा तन्तुसंयोगः पटस्य। तन्तुरूपं पटरूपस्य। तदुभयभिन्नं कारणं निमित्तकारणम्। यथा तुरीवेमादिकं पटस्य} (त॰स॰~४०)।} तत्र मातरि समवाय\-सम्बन्धेन सन्ततिरुत्पद्यते।\footnote{\textcolor{red}{यस्मिन्समवेतं सत्समवायेन सम्बद्धं सत्कार्यमुत्पद्यते तत्समवायि\-कारणमित्यर्थः ... समवाय\-सम्बन्धावच्छिन्न\-कार्यता\-निरूपित\-तादात्म्य\-सम्बन्धावच्छिन्न\-कारणत्वं समवायि\-कारणतत्वमिति} (त॰स॰ न्या॰बो॰व्या॰~४०)।} तथा च तस्या एवाऽधारत्वात्सप्तम्युचिता। आधारो ह्यत्रौपश्लेषिको बोध्यः संयोग\-रूपो न तु समवायः।\footnote{अनयोरयं भेदः~– \textcolor{red}{अयुत\-सिद्धयोस्सम्बन्धः समवायः। अन्ययोस्तु संयोगः} (बा॰म॰~६३३)।} अतः \textcolor{red}{मयि जातः} इत्युचितं यथा भागवते~–\end{sloppypar}
\centering\textcolor{red}{निशीथे तमउद्भूते जायमाने जनार्दने।\nopagebreak\\
देवक्यां देवरूपिण्यां विष्णुः सर्वगुहाशयः॥\nopagebreak\\
आविरासीद्यथा प्राच्यां दिशीन्दुरिव पुष्कलः॥}\nopagebreak\\
\raggedleft{–~भा॰पु॰~१०.३.८}\\
\begin{sloppypar}\hyphenrules{nohyphenation}\justifying\noindent इति। अत्र \textcolor{red}{देवक्याम्‌} इति सप्तम्यन्त\-प्रयोगः। तस्मादत्रापि सप्तम्युचितेति चेत्। उच्यते। \textcolor{red}{मे कुक्षौ} इत्यध्याहारे \textcolor{red}{सम्बन्धे षष्ठी}। यद्वा \textcolor{red}{जातः} इत्यस्य पुत्रोऽर्थः।\footnote{यथा \textcolor{red}{अयि जात कथयितव्यं कथय} (उ॰रा॰च॰~४.२३) इति भवभूति\-प्रयोगे।} अर्थात् \textcolor{red}{त्वं मे जातः पुत्रोऽसि}। इत्यर्थान्तर\-सम्बन्धे षष्ठी। अथवा \textcolor{red}{मे} इति चतुर्थ्यन्तं \textcolor{red}{तादर्थ्ये चतुर्थी वाच्या} (वा॰~२.३.१३) इति वार्त्तिकेन चतुर्थी। \textcolor{red}{कौसल्या\-हित\-कारी}\footnote{एतद्रूपान्तरम्–\textcolor{red}{कोसलात्मजाहितावहो} (मा॰भा॰~१.१९२.१)।} (रा॰च॰मा॰~१.१९२.१) इति मानसेऽपि समर्थितत्वात्। अथवा भगवतः कौसल्यां निमित्ती\-कृत्य प्राकट्यान्नैव सप्तमी। ततः \textcolor{red}{मे जातः} इत्यस्य \textcolor{red}{मे पुरतः प्रकटः} इति नव्यं समाधानम्।\end{sloppypar}
\section[मे]{मे}
\centering\textcolor{blue}{यत्र रामस्त्वया दृष्टस्तत्र मां नय सुव्रत।\nopagebreak\\
सीतया सहितो यत्र सुप्तस्तद्दर्शयस्व मे॥}\nopagebreak\\
\raggedleft{–~अ॰रा॰~२.८.२५}\\
\begin{sloppypar}\hyphenrules{nohyphenation}\justifying\noindent\hspace{10mm} अत्र चित्रकूटं गच्छन् भरतो निषादेन सङ्गम्य भगवच्छयन\-स्थानं पृच्छति \textcolor{red}{यत्र भगवान् सुप्तस्तन्मे दर्शयस्व}। अत्र \textcolor{red}{अहं पश्यानि त्वं प्रेरय} इत्यर्थे \textcolor{red}{त्वं मां दर्शयस्व} इत्येव द्वितीयोचिता। चतुर्थी\-प्रयोगस्तु सम्प्रदान\-विवक्षया। भरतो दान\-रूपेण स्थान\-दर्शनं याचते। यद्वा \textcolor{red}{मां सुखयितुं दर्शयस्व} इति \textcolor{red}{क्रियार्थोपपदस्य च कर्मणि स्थानिनः} (पा॰सू॰~२.३.१४) इत्यनेन चतुर्थी।\end{sloppypar}
\vspace{2mm}
\centering ॥ इत्ययोध्याकाण्डीयप्रयोगाणां विमर्शः ॥\nopagebreak\\
\vspace{4mm}
\centering\textcolor{blue}{\fontsize{16}{24}\selectfont अध्यात्मरामायणसंसृतानां बहिःस्थितानामिव पाणिनीयात्।\nopagebreak\\
मया पदानां प्रथमो विमर्शे शोधे परिच्छेद इति व्यधायि॥}\nopagebreak\\
\vspace{4mm}
\centering इत्यध्यात्म\-रामायणेऽपाणिनीय\-प्रयोगाणां\-विमर्श\-नामके शोध\-प्रबन्धे प्रथमाध्याये प्रथम\-परिच्छेदः।\\
\pagebreak
\pdfbookmark[1]{द्वितीयः परिच्छेदः}{Chap1Part2}
\phantomsection
\addtocontents{toc}{\protect\setcounter{tocdepth}{1}}
\addcontentsline{toc}{section}{द्वितीयः परिच्छेदः}
\addtocontents{toc}{\protect\setcounter{tocdepth}{0}}
\centering ॥ अथ प्रथमाध्याये द्वितीयः परिच्छेदः ॥\nopagebreak\\
\vspace{4mm}
\pdfbookmark[2]{अरण्यकाण्डम्‌}{Chap1Part2Kanda3}
\phantomsection
\addtocontents{toc}{\protect\setcounter{tocdepth}{2}}
\addcontentsline{toc}{subsection}{अरण्यकाण्डीयप्रयोगाणां विमर्शः}
\addtocontents{toc}{\protect\setcounter{tocdepth}{0}}
\centering ॥ अथारण्यकाण्डीयप्रयोगाणां विमर्शः ॥\nopagebreak\\
\section[सहितेन मे]{सहितेन मे}
\centering\textcolor{blue}{इतः परं प्रयत्नेन गन्तव्यं सहितेन मे।\nopagebreak\\
धनुर्गुणेन संयोज्य शरानपि करे दधत्॥}\nopagebreak\\
\raggedleft{–~अ॰रा॰~३.१.१२}\\
\begin{sloppypar}\hyphenrules{nohyphenation}\justifying\noindent\hspace{10mm} प्रयोगोऽयमध्यात्म\-रामायणेऽरण्य\-काण्डे प्रथमे सर्गे श्रीरामेण विहितः। चित्रकूटं त्यक्त्वा पश्चात्सीता\-लक्ष्मणाभ्यां सह कियद्दूरं गत्वा घोर\-काननं निरीक्ष्य लक्ष्मणं सतर्कयति यत् \textcolor{red}{इतः परं मे सहितेन त्वया गन्तव्यम्‌}। \textcolor{red}{सहितेन} इति लक्ष्मणस्य विशेषणम्। तथा च \textcolor{red}{सह\-युक्तेऽप्रधाने} (पा॰सू॰~२.३.१९) इत्यनेनात्र तृतीयायां \textcolor{red}{मया सहितेन} त्वया गन्तव्यं \textcolor{red}{मे} इति कथमिति चेत्। सम्बन्ध\-विवक्षायां षष्ठी। यद्वा \textcolor{red}{हितेन सह वर्तमानः सहितः}। \textcolor{red}{तेन सहेति तुल्य\-योगे} (पा॰सू॰~२.२.२८) इत्यनेन समासः।\footnote{\textcolor{red}{वोपसर्जनस्य} (पा॰सू॰~६.३.८२) इत्यनेन \textcolor{red}{सह} इत्यस्य \textcolor{red}{स} इत्यादेशः।} तेनात्र \textcolor{red}{मे} इत्यस्य \textcolor{red}{हित}\-शब्देनान्वयः। तत्र \textcolor{red}{मे हित\-सहितेन त्वया गन्तव्यम्‌} इति श्रीरामस्य तात्पर्यम्। न च \textcolor{red}{सापेक्षमसमर्थवत्‌} इति वचनेन \textcolor{red}{हित}\-शब्दस्य \textcolor{red}{मे} इत्यनेन सापेक्षतया कथं समासः। नित्य\-सापेक्ष\-स्थल उक्त\-नियमस्यानादरात्। यद्वा गमन\-क्रियायाः कर्ता श्रीरामः। एवं \textcolor{red}{हितेन सह वर्तमानेनेति सहितेन त्वया लक्ष्मणेन सह इतो मे गन्तव्यम्‌}। अत्र तव्यत्प्रत्ययो भावे।\footnote{\textcolor{red}{तयोरेव कृत्य\-क्तखलर्थाः} (पा॰सू॰~३.४.७०) इत्यनेन।} तस्मादनुक्त\-कर्तरि यद्यपि तृतीया प्राप्नोति तथाऽपि \textcolor{red}{कृत्यानां कर्तरि वा} (पा॰सू॰~२.३.७१) इति सूत्रेण \textcolor{red}{कृत्याः} (पा॰सू॰~३.१.९५) इत्यधिकारे विहितस्य \textcolor{red}{तव्यत्‌}\-प्रत्ययस्य कर्तरि राम\-वाच्येऽस्मच्छब्दे तृतीया\-स्थाने षष्ठी \textcolor{red}{मम} इति तस्य \textcolor{red}{मे} इत्यादेशः।\footnote{\textcolor{red}{तेमयावेकवचनस्य} (पा॰सू॰~८.१.२२) इत्यनेन।} यद्वा \textcolor{red}{बन्धुना} इत्यध्याहार्यम्। \textcolor{red}{मे बन्धुना त्वया गन्तव्यम्‌} इति विग्रहे पाल्य\-पालक\-भावे स्व\-स्वामि\-भावे वा षष्ठी। यद्वा \textcolor{red}{कृते} इत्यध्याहार्यम्।
\textcolor{red}{मे मम कृते सहितेन त्वया गन्तव्यम्‌}। यद्वा \textcolor{red}{मे} इति चतुर्थ्यन्तं \textcolor{red}{मे मह्यं गन्तव्यम्‌}। \textcolor{red}{तादर्थ्ये चतुर्थी वाच्या} (वा॰~२.३.१३) इति चतुर्थी। अर्थात् \textcolor{red}{इतः पूर्वं मया साकं त्वं स्व\-सुखानुभूत्याऽयासीः किन्त्वधुनेतः परं राक्षस\-सङ्ग्रामे महत्त्व\-पूर्ण\-योग\-दानाय मदर्थमेव गन्तव्यम्‌} इति श्रीरामस्य तात्पर्यं प्रतिभाति। राम\-रावण\-सङ्ग्रामे मेघनादादि\-वधे लक्ष्मणस्य महत्त्व\-पूर्णा भूमिका सर्वैरपि ज्ञाता। 
अथवा \textcolor{red}{मया} इत्यर्थे \textcolor{red}{मे} इत्यव्ययम्।\end{sloppypar}
\section[तव दास्यामि]{तव दास्यामि}
\centering\textcolor{blue}{तव सन्दर्शनाकाङ्क्षी राम त्वं परमेश्वरः।\nopagebreak\\
अद्य मत्तपसा सिद्धं यत्पुण्यं बहु विद्यते।\nopagebreak\\
तत्सर्वं तव दास्यामि ततो मुक्तिं व्रजाम्यहम्॥}\nopagebreak\\
\raggedleft{–~अ॰रा॰~३.२.५}\\
\begin{sloppypar}\hyphenrules{nohyphenation}\justifying\noindent\hspace{10mm} श्रीसीता\-लक्ष्मण\-समेत\-श्रीरामेण शरभङ्गश्चितायां शरीरं दिधक्षुः प्राप्यते। अथात्र \textcolor{red}{सम्पूर्णस्य तपसः पुण्यं तुभ्यं दास्यामि}। श्लोकेऽस्मिन् \textcolor{red}{तुभ्यं दास्यामि} इति प्रयोक्तव्ये \textcolor{red}{तव दास्यामि} इति प्रयुक्तम्। अत्र सम्प्रदाने सम्बन्ध\-विवक्षायां षष्ठी। यतो हि पुण्यस्य त्वयैव सह शाश्वतः सम्बन्धः। त्वमेव पुण्य\-निधानं त्वत्तः किमपि परं नास्ति। अतस्तवैव वस्तु तवैव दास्यामि मम किमपि स्वामित्वं नास्ति। यथा \textcolor{red}{रजकस्य वस्त्रं ददाति} इत्यत्र दाता स्व\-स्वत्वं न त्यजति।\footnote{\textcolor{red}{दानं चापुनर्ग्रहणाय स्व\-स्वत्व\-निवृत्तिपूर्वकं पर\-स्वत्वोत्पादनम्‌। अत एव “रजकस्य वस्त्रं ददाति” इत्यादौ न भवति} (त॰बो॰~५६९)। रजक\-वस्त्र\-दाने वस्त्र\-परावर्तनात्स्व\-स्वत्व\-निवृत्तिर्न पर\-स्वत्वोत्पादनञ्च न। तस्मान्न सम्प्रदानत्वम्। एवमेवात्र भगवत्पुण्यदानेऽपि भगवत एव पुण्य\-स्वामित्वात्स्व\-स्वत्व\-निवृत्तिर्न पर\-स्वत्वोत्पादनञ्च न। तस्मान्न सम्प्रदानत्वम्। अयं भावः।} पुनर्न सम्प्रदानं विनियमः। विनिमयो नाम प्रतिदानम्। \textcolor{red}{तव प्रतिदास्यामि} इदमेव वाक्यम्।\footnote{अत्र शरभङ्गः पुण्यं ददाति मुक्तिं च प्राप्नोति। भगवांश्च पुण्यं प्राप्नोति मुक्तिं च ददाति। इदमेव प्रतिदानम्। भगवान् पुण्यात्प्रतियच्छति मुक्तिमिति भावः।} \textcolor{red}{प्रति} इत्यस्य \textcolor{red}{विनाऽपि प्रत्ययं पूर्वोत्तर\-पद\-लोपो वक्तव्यः} (वा॰~५.३.८३) इत्यनेन लोपः। अथवा \textcolor{red}{तव दास्यामि} येन त्वं राक्षसानां
संहारं कुरु। उपासकानामयं नियमो यत्सत्कर्माणि समाचरन्ति तेषां फलञ्च स्वस्मायिष्टदेवाय प्रयच्छन्ति यथा भागवते~–\end{sloppypar}
\centering\textcolor{red}{कायेन वाचा मनसेन्द्रियैर्वा बुद्ध्याऽऽत्मना वाऽनुसृतस्वभावात्।\nopagebreak\\
करोति यद्यत्सकलं परस्मै नारायणायेति समर्पयेत्तत्॥}\nopagebreak\\
\raggedleft{–~भा॰पु॰~११.२.३६}\\
\begin{sloppypar}\hyphenrules{nohyphenation}\justifying\noindent अत एव \textcolor{red}{जय जय} इति कथयन्ति जनाः। वाल्मीकिरप्यन्तिम\-वाक्यं लिखति \textcolor{red}{बलं विष्णोश्च वर्धताम्‌} (वा॰रा॰~६.१२८.१२१)।\end{sloppypar}
\section[समर्प्य रामस्य]{समर्प्य रामस्य}
\centering\textcolor{blue}{समर्प्य रामस्य महत्सुपुण्यफलं विरक्तः शरभङ्गयोगी।\nopagebreak\\
चितिं समारोहयदप्रमेयं रामं ससीतं सहसा प्रणम्य॥}\nopagebreak\\
\raggedleft{–~अ॰रा॰~३.२.६}\\
\begin{sloppypar}\hyphenrules{nohyphenation}\justifying\noindent\hspace{10mm} अत्र भगवाञ्छरभङ्गः समस्तं पुण्य\-जातं श्रीरामाय समर्पयति। अत्र \textcolor{red}{रामस्य समर्प्य} इति लिखितम्।\footnote{णिजन्तात् \textcolor{red}{सम्‌}\-पूर्वकात् \textcolor{red}{ऋ}\-धातोः (\textcolor{red}{ऋ गतिप्रापणयोः} धा॰पा॰~९३६) क्त्वा प्रत्यये \textcolor{red}{समर्प्य} इति रूपं सिध्यति। ऋ~\arrow \textcolor{red}{हेतुमति च} (पा॰सू॰~३.१.२६)~\arrow ऋ~णिच्~\arrow ऋ~इ~\arrow \textcolor{red}{अर्ति\-ह्री\-व्ली\-री\-क्नूयी\-क्ष्माय्यातां पुङ्णौ} (पा॰सू॰~७.३.३६)~\arrow \textcolor{red}{आद्यन्तौ टकितौ} (पा॰सू॰~१.१.४६)~\arrow ऋ~पुँक्~इ~\arrow ऋ~प्~इ~\arrow \textcolor{red}{पुगन्त\-लघूपधस्य च} (पा॰सू॰~७.३.८६)~\arrow \textcolor{red}{उरण् रपरः} (पा॰सू॰~१.१.५१)~\arrow अर्प्~इ~\arrow अर्पि~\arrow \textcolor{red}{सनाद्यन्ता धातवः} (पा॰सू॰~३.१.३२)~\arrow धातु\-सञ्ज्ञा। सम्~अर्पि~\arrow \textcolor{red}{समानकर्तृकयोः पूर्वकाले} (पा॰सू॰~३.४.२१)~\arrow समर्पि~क्त्वा~\arrow \textcolor{red}{कुगतिप्रादयः}~\arrow \textcolor{red}{समासेऽनञ्पूर्वे क्त्वो ल्यप्‌} (पा॰सू॰~७.१.३७)~\arrow समर्पि~ल्यप्~\arrow समर्पि~य~\arrow \textcolor{red}{णेरनिटि} (पा॰सू॰~६.४.५१)~\arrow समर्प्~य~\arrow समर्प्य।} \textcolor{red}{समर्प्य} इति पदस्य योगेन चतुर्थी प्रयोक्तव्या \textcolor{red}{दा}\-धातु\-समानार्थकत्वात्। किन्त्वत्र षष्ठी किमपि विशेषं वक्ति। \textcolor{red}{रामस्यैव पुण्य\-जातं न्यास\-रूपेण मम पार्श्व आसीत्साम्प्रतं परावर्तयामि} इमं भावं ध्वनयितुं \textcolor{red}{रामस्य} इति। यद्वा \textcolor{red}{रामस्य करे समर्प्य} इत्यध्याहृते सम्बन्ध\-सामान्य\-विवक्षायां षष्ठी।\footnote{अवयवावयवि\-भाव\-मूलकश्चात्र सम्बन्धः।}\end{sloppypar}
\section[शरभङ्गयोगी]{शरभङ्गयोगी}
\centering\textcolor{blue}{समर्प्य रामस्य महत्सुपुण्यफलं विरक्तः शरभङ्गयोगी।\nopagebreak\\
चितिं समारोहयदप्रमेयं रामं ससीतं सहसा प्रणम्य॥}\nopagebreak\\
\raggedleft{–~अ॰रा॰~३.२.६}\\
\begin{sloppypar}\hyphenrules{nohyphenation}\justifying\noindent\hspace{10mm} अत्र राघवेन्द्रं प्रणम्य पुण्य\-जातं समर्प्य महर्षिः शरभङ्गश्चितां समारोहत्। \textcolor{red}{शरभङ्ग एव योगीति शरभङ्ग\-योगी} इति विग्रहे कर्मधारयः। अत्र \textcolor{red}{योगि}\-शब्दः \textcolor{red}{शरभङ्ग}\-शब्दस्य विशेषणम्। अन्यः कोऽपि शरभङ्गो न योग्येवायं हि शरभङ्ग\-योगी संसार\-जनस्तु रस\-भङ्ग\-योगीति। \textcolor{red}{विशेषणत्वं नाम विद्यमानत्वे सति विधेयान्वयित्वे
सतीतर\-व्यावर्तकत्वम्}। अत्रेदं ध्येयं विशेषणमुप\-लक्षणमुपाधिरिमे त्रयः प्रायः समाना अतिसूक्ष्ममन्तरमेतेषु। विशेषण\-लक्षणमुक्तम्। \textcolor{red}{उपाधिरविद्यमानत्वे सति विधेयान्वयित्वे सतीतर\-व्यावर्तकत्वम्}। \textcolor{red}{उपलक्षणं ह्यविद्यमानत्वे सति विधेयानन्वयित्वे सतीतर\-व्यावर्तकत्वम्}। यथा \textcolor{red}{काकवन्तो देवदत्तस्य गृहाः} इति। एवमेवेतर\-व्यावर्तकतया योगीति विशेषणं शरभङ्गश्च विशेष्यम्। तथा \textcolor{red}{विशेषणं विशेष्येण बहुलम्‌} (पा॰सू॰~२.१.५७) इत्यनेन समासे \textcolor{red}{प्रथमा\-निर्दिष्टं समास उपसर्जनम्‌} (पा॰सू॰~१.२.४३) समास\-विधायके शास्त्रे प्रथमया निर्दिष्टं पदं समास उपसर्जन\-सञ्ज्ञं स्यात्। अत्र समास\-विधायकं शास्त्रं \textcolor{red}{विशेषणं विशेष्येण बहुलम्‌} इति। प्रथमया निर्दिष्टं \textcolor{red}{विशेषणम्‌} इति। अत्रानुपूर्व्यां न तात्पर्यं तस्या अनुपयोगात्। \textcolor{red}{विशेषणम्‌} इत्यानुपूर्व्याः \textcolor{red}{विशेष्येण} इत्यानुपूर्व्या सह समासाभावादनुप\-योगाच्च। तस्मादानुपूर्व्या आनुपूर्व्यभिधेये तात्पर्यमत्र। विशेषण\-वाचक\-पदमित्यर्थः। एवं \textcolor{red}{नीलम् उत्पलम्‌} इत्यत्र \textcolor{red}{नीलम्‌} इत्यस्योपसर्जन\-सञ्ज्ञा। न च समास\-विधायके शास्त्रे त्रीणि पदानि \textcolor{red}{विशेषणम् विशेष्येण बहुलम्‌} इत्येतेष्वेकं तृतीयान्तं द्वे च प्रथमान्ते। यथा \textcolor{red}{विशेषणम्‌} इति प्रथमा\-निर्दिष्टं तथैव \textcolor{red}{बहुलम्‌} इत्यपि। अस्य कथं नोपसर्जन\-सञ्ज्ञेति चेत्सत्यम्। उपसर्जन\-सञ्ज्ञा\-विधायक\-सूत्रं \textcolor{red}{प्रथमा\-निर्दिष्टं समास उपसर्जनम्‌} इति। अत्र \textcolor{red}{समासे} इति सप्तमी च वैषयिकी। विषयता च घटकत्व\-रूपा। अर्थात्समासे घटकतया सत्प्रविष्टं सत्समास\-विधायक\-सूत्रे प्रथमया निर्दिष्टमेवोपसर्जन\-सञ्ज्ञम्। \textcolor{red}{बहुलम्‌} इति प्रथमया निर्दिष्टं किन्तु समासे घटकतया न प्रविष्टम्। अतो न दोषः। तस्मादुपसर्जन\-सञ्ज्ञायाम् \textcolor{red}{उपसर्जनं पूर्वम्‌} (पा॰सू॰~२.२.३०) इत्यनेन पूर्व\-प्रयोगे सति \textcolor{red}{योगि\-शरभङ्गः} न तु \textcolor{red}{शरभङ्ग\-योगी} इति। अत्रोच्यते। पूर्व\-निपात\-प्रकरणमनित्यमिति पूर्वमेवोप\-पादितम्। तेन शरभङ्ग\-शब्दस्य पूर्व\-प्रयोगः। अथवा विशेषण\-विशेष्य\-भावे काम\-चारः। प्रत्येकं पदं प्रत्येकस्य विशेषणं विशेष्यञ्च भवति। को योगी योगिनस्तु बहवस्तदा शरभङ्ग इति। अत्रापि त्वितर\-व्यावर्तकता। तस्माच्छरभङ्ग\-शब्दस्येतर\-व्यावर्तकतया विशेषणता। ततश्च पूर्व\-प्रयोगः। यद्वाऽनेन सूत्रेण समासस्त्यज्यताम्। \textcolor{red}{मयूर\-व्यंसकादयश्च} (पा॰सू॰~२.१.७२) इत्यनेन समासः। \textcolor{red}{भाष्याब्धिः} (भा॰प्र॰~मङ्गलाचरणे ६) इति कैयटेनापि सरणि\-स्वीकारात् \textcolor{red}{शरभङ्ग एव योगी} इति विग्रहः। यद्वा \textcolor{red}{शरं चिता तस्मिन् शरीरभङ्ग इति शरभङ्गस्तमेव योजयितुं शीलमस्य} इति विग्रहे \textcolor{red}{सुप्यजातौ णिनिस्ताच्छील्ये} (पा॰सू॰~३.२.७८) इत्यनेन णिनि\-प्रत्ययः। अनुबन्ध\-लोपे गुणे\footnote{\textcolor{red}{पुगन्त\-लघूपधस्य च} (पा॰सू॰~७.३.८६) इत्यनेन।} \textcolor{red}{चजोः कु घिण्ण्यतोः} (पा॰सू॰~७.३.५२) इत्यनेन जकारस्य कुत्वे\footnote{कुत्वे सति \textcolor{red}{स्थानेऽन्तरतमः} (पा॰सू॰~१.१.५०) इत्यनेन गत्वम्।} विभक्ति\-कार्ये \textcolor{red}{शरभङ्ग\-योगी} इति।\end{sloppypar}
\section[आदौ ऋषीणाम्]{आदौ ऋषीणाम्‌}
\centering\textcolor{blue}{शेषांशं शङ्खचक्रे द्वे भरतं सानुजं तथा।\nopagebreak\\
अतश्चादौ ऋषीणां त्वं दुःखं मोक्तुमिहार्हसि॥}\nopagebreak\\
\raggedleft{–~अ॰रा॰~३.२.१६}\\
\begin{sloppypar}\hyphenrules{nohyphenation}\justifying\noindent\hspace{10mm} अत्र सीता\-लक्ष्मण\-समन्वितं श्रीरामं दृष्ट्वा तं ब्रह्म ज्ञात्वा कथयन्ति मुनीश्वरा यत् \textcolor{red}{ऋषीणामादौ त्वं दुःखं हर्तुमर्हसि}। अत्र \textcolor{red}{एचोऽयवायावः} (पा॰सू॰~६.१.७८) इत्यनेनाऽवादेशोऽनिवार्यः। \textcolor{red}{आदावृषीणाम्‌} इत्येव भवेत् \textcolor{red}{आदौ ऋषीणाम्‌} इति त्वपाणिनीयमिव। अत्रोच्यते। सन्धेरविवक्षया। साम्प्रतं भगवान् राक्षसैः सह विग्रहं चिकीर्षत्यतो राक्षसैः सह सन्धिरनुचित इति भावं ध्वनयितुमत्र सन्धिर्न। यथा पत्न्या सह विजिघृक्षुर्भर्तृहरिर्नीति\-शतकस्य द्वितीये श्लोके कथयति~–\end{sloppypar}
\centering\textcolor{red}{यां चिन्तयामि सततं मयि सा विरक्ता\nopagebreak\\
साऽप्यन्यमिच्छति जनं स जनोऽन्यसक्तः।\\
अस्मत्कृते च परिशुष्यति काचिदन्या\\
धिक्ताञ्च तञ्च मदनञ्च इमाञ्च माञ्च॥}\nopagebreak\\
\raggedleft{–~भ॰नी॰~२}\\
\begin{sloppypar}\hyphenrules{nohyphenation}\justifying\noindent अत्र सन्धिरिष्टो नाऽसीदतः श्लोकेऽपि \textcolor{red}{मदनञ्च इमाञ्च} इत्यत्र न सन्धिस्तथैवात्रापि राक्षसैः सह विग्रह इष्टोऽतोऽसन्धि\-प्रयोगः। सन्धिर्वाक्ये विवक्षाधीनो भवतीति व्याकरण\-सिद्धान्तः। एकपदे सन्धिर्नित्यो यथा \textcolor{red}{रामेण}। धातूपसर्गयोस्तथा यथा \textcolor{red}{प्रार्च्छति}। समासेऽपि सन्धिर्नित्यो यथा \textcolor{red}{रामानुजः}। किन्तु वाक्ये तु विवक्षामपेक्षते तथा चोक्तम्~–\end{sloppypar}
\centering\textcolor{red}{संहितैकपदे नित्या नित्या धातूपसर्गयोः।\nopagebreak\\
नित्या समासे वाक्ये तु सा विवक्षामपेक्षते॥}\nopagebreak\\
\raggedleft{–~वै॰सि॰कौ॰~२२३२}\\
\begin{sloppypar}\hyphenrules{nohyphenation}\justifying\noindent अतोऽत्र न विवक्षा। अतो न सन्धिः। \end{sloppypar}
\section[त्वन्मन्त्र\-साधन\-परेषु]{त्वन्मन्त्र\-साधन\-परेषु}
\centering\textcolor{blue}{त्वं सर्वभूतहृदयेषु कृतालयोऽपि\nopagebreak\\
त्वन्मन्त्रजाप्यविमुखेषु तनोषि मायाम्।\nopagebreak\\
त्वन्मन्त्रसाधनपरेष्वपयाति माया\nopagebreak\\
सेवानुरूपफलदोऽसि यथा महीपः॥}\nopagebreak\\
\raggedleft{–~अ॰रा॰~३.२.२९}\\
\begin{sloppypar}\hyphenrules{nohyphenation}\justifying\noindent\hspace{10mm} नीलोत्पल\-दल\-श्यामं रामं राजीव\-लोचनं जानकी\-लक्ष्मण\-युतं जटा\-वल्कल\-धारिणं सुतीक्ष्णस्तीक्ष्णया भक्त्या तुष्टाव रघु\-नन्दनम्। \textcolor{red}{त्वन्मन्त्रसाधन\-परेषु} इति प्रायुङ्क्त सुतीक्ष्णः। तत्र हि \textcolor{red}{अपयाति माया} इति पद\-द्वय\-समभिव्याहारेण \textcolor{red}{त्वन्मन्त्रसाधनपरेषु} इत्यत्र पञ्चम्या भवितव्यम्। \textcolor{red}{अपयाति} इत्यस्य हि पूर्व\-देश\-त्यागानुकूलो व्यापारोऽर्थः। तत्र च कस्मादपयातीत्यपेक्षायां विश्लेषे साध्यमाने विश्लिष्ट\-मायाया अवधि\-भूततया \textcolor{red}{त्वन्मन्त्र\-साधन\-परेभ्यः} इत्यपादान\-सञ्ज्ञा\-फल\-भूत\-पञ्चमी। तथा च सूत्रम् \textcolor{red}{ध्रुवमपायेऽपादानम्‌} (पा॰सू॰~१.४.२४)। \textcolor{red}{अपायो विश्लेषस्तस्मिन् साध्ये ध्रुवमवधि\-भूतं कारकमपादानं स्यात्‌} (वै॰सि॰कौ॰~५८६) इति। सप्तम्यत्र विमर्श\-विषयः। अत्र समाधानम्। \textcolor{red}{सत्सु} इत्यध्याहार्यं तथा च \textcolor{red}{त्वन्मन्त्र\-साधन\-परेषु सत्सु पश्चान्मायाऽपयाति} इति। \textcolor{red}{यस्य च भावेन भाव\-लक्षणम्‌} (पा॰सू॰~२.३.३७) इत्यनेन सप्तमी। यद्वा \textcolor{red}{सप्तम्यधिकरणे च} (पा॰सू॰~२.३.३६) इत्यत्र चकारात्सप्तमी।\footnote{\textcolor{red}{चकाराद्दूरान्तिकार्थेभ्यः} (का॰वृ॰~२.३.३५, वै॰सि॰कौ॰~६३३, ल॰सि॰कौ॰~९०६)।} यद्वा \textcolor{red}{त्वन्मन्त्र\-साधन\-परानपेक्ष्य मायाऽपयाति} इति ल्यबन्त\-लोपे सति तत्रानादरार्थे सप्तमी।\footnote{\textcolor{red}{षष्ठी चानादरे} (पा॰सू॰~२.३.३८) इत्यनेन।} यद्वा \textcolor{red}{असाधु}\-पदस्याध्याहारे \textcolor{red}{साध्व\-साधु\-प्रयोगे च} (वा॰~२.३.३६) इत्यनेन \textcolor{red}{त्वन्मन्त्र\-साधन\-परेष्वसाधु\-मायाऽप्यपयाति} इति सप्तमी।
\end{sloppypar}
\section[मनो मे त्वरयति]{मनो मे त्वरयति}
\centering\textcolor{blue}{देहान्ते मम सायुज्यं लप्स्यसे नात्र संशयः।\nopagebreak\\
गुरुं ते द्रष्टुमिच्छामि ह्यगस्त्यं मुनिनायकम्।\nopagebreak\\
किञ्चित्कालं तत्र वस्तुं मनो मे त्वरयत्यलम्॥}\nopagebreak\\
\raggedleft{–~अ॰रा॰~३.२.३९}\\
\begin{sloppypar}\hyphenrules{nohyphenation}\justifying\noindent\hspace{10mm} अत्र सुतीक्ष्णमभिगम्य श्रीरामोऽगस्त्यं द्रष्टुमिच्छुः कथयति यत् \textcolor{red}{मे मनस्त्वरयति}। अत्र णिजन्तात् \textcolor{red}{त्वर्‌}\-धातोः\footnote{अस्य धातोर्घटाद्यन्तर्गणे पाठात् \textcolor{red}{घटादयो मितः} (धा॰पा॰ ग॰सू॰) इत्यनेन मित्त्वाण्णिचि \textcolor{red}{अत उपधायाः} (पा॰सू॰~७.२.११६) इत्यनेनोपधा\-वृद्धौ \textcolor{red}{मितां ह्रस्वः} (पा॰सू॰~६.४.९२) इत्यनेन ह्रस्वे \textcolor{red}{त्वरि} इत्यस्य धातु\-सञ्ज्ञायां लटि तिपि शपि गुणेऽयादेशे \textcolor{red}{त्वरयति}।} (\textcolor{red}{ञित्वराँ सम्भ्रमे} धा॰पा॰~७७५) कर्मतया \textcolor{red}{माम्‌} इति प्रयोक्तव्ये \textcolor{red}{मे} इति प्रयुक्तम्। अत्र \textcolor{red}{कर्तुरीप्सित\-तमं कर्म} (पा॰सू॰~१.४.४९) इत्यनेन कर्म\-सञ्ज्ञा। ततः \textcolor{red}{कर्मणि द्वितीया} (पा॰सू॰~२.३.२) इत्यनेन द्वितीया प्राप्ता किन्त्वत्र षष्ठी। \textcolor{red}{कर्मादीनामपि सम्बन्ध\-मात्र\-विवक्षायां षष्ठ्येव} (वै॰सि॰कौ॰~६०६) तथैवात्राप्येवम्। \textcolor{red}{मनः\-कर्तृक\-वर्तमान\-कालिक\-श्रीराम\-कर्मक\-मुनि\-दर्शन\-विषयकोत्कण्ठानुकूल\-व्यापारः} इति शाब्दबोधः। यद्वा \textcolor{red}{मे मनः} इत्यन्वये \textcolor{red}{मत्सम्बन्धि मनः}।
सम्बन्ध\-सामान्ये षष्ठी। तथा च \textcolor{red}{मे मनः मां त्वरयति} इत्यध्याहार्यम्।\end{sloppypar}
\section[सुतीक्ष्णेन]{सुतीक्ष्णेन}
\centering\textcolor{blue}{अथ रामः सुतीक्ष्णेन जानक्या लक्ष्मणेन च।\nopagebreak\\
अगस्त्यस्यानुजस्थानं मध्याह्ने समपद्यत॥}\nopagebreak\\
\raggedleft{–~अ॰रा॰~३.३.१}\\
\begin{sloppypar}\hyphenrules{nohyphenation}\justifying\noindent\hspace{10mm} अत्र सह\-शब्दं विनाऽपि तृतीया। विनाऽपि सह\-शब्दं तृतीया\-विधानात्।\footnote{\textcolor{red}{विनाऽपि तद्योगं तृतीया। वृद्धो यूनेत्यादिनिर्देशात्‌} (वै॰सि॰कौ॰~५६४)।} यद्वा \textcolor{red}{इत्थं\-भूत\-लक्षणे} (पा॰सू॰~२.३.२१) इति सूत्रेण तृतीया।\footnote{सुतीक्ष्ण\-ज्ञाप्य\-रामत्व\-विशिष्टः श्रीरामोऽग्निजिह्व\-स्थानं समपद्यतेति भावः।} यद्वा \textcolor{red}{हेतौ} (पा॰सू॰~२.३.२३) इत्यनेन तृतीया।\footnote{\textcolor{red}{पुण्येन दृष्टो हरिः} (वै॰सि॰कौ॰~५६८) इतिवदत्राग्निजिह्व\-मुनि\-स्थान\-गमने सुतीक्ष्णो हेतुरिति भावः।} यद्वा \textcolor{red}{प्रकृत्यादिभ्य उप\-सङ्ख्यानम्‌} (पा॰सू॰~२.३.१८) इत्यनेनाभेदे तृतीया।\footnote{एतत्पूर्वसर्गे \textcolor{red}{निरपेक्षा नान्यगतास्तेषां दृश्योऽहमन्वहम्} (अ॰रा॰~३.२.३७) इत्यत्र श्रीरामेण सुतीक्ष्णस्य कृते स्वस्य नित्यदृश्यत्वमुक्तम्। \textcolor{red}{देहान्ते मां सायुज्यं लप्स्यसे नात्र संशयः} (अ॰रा॰~३.२.३९) इत्यत्राभेदश्चोक्तः। तस्मादभेदे तृतीयेति भावः।}\end{sloppypar}
\section[अगस्त्यमुनिवर्याय]{अगस्त्यमुनिवर्याय}
\centering\textcolor{blue}{सुतीक्ष्ण गच्छ त्वं शीघ्रमागतं मां निवेदय॥\\
अगस्त्यमुनिवर्याय सीतया लक्ष्मणेन च।\nopagebreak\\
महाप्रसाद इत्युक्त्वा सुतीक्ष्णः प्रययौ गुरोः॥}\nopagebreak\\
\raggedleft{–~अ॰रा॰~३.३.५-६}\\
\begin{sloppypar}\hyphenrules{nohyphenation}\justifying\noindent\hspace{10mm} अत्र श्रीरामः सुतीक्ष्णं प्रति कथयति \textcolor{red}{त्वमागतं मामगस्त्य\-मुनि\-वर्याय निवेदय}। \textcolor{red}{गति\-बुद्धि\-प्रत्यवसानार्थ\-शब्द\-कर्माकर्मकाणामणि कर्ता स णौ} (पा॰सू॰~१.४.५२) इत्यनेन नि\-पूर्वक\-\textcolor{red}{विद्‌}\-धातोः (\textcolor{red}{विदँ ज्ञाने} धा॰पा॰~१०६४) बुद्ध्यर्थतया शब्द\-कर्मतया च \textcolor{red}{अगस्त्यमुनिवर्य} इत्यस्य कर्म\-सञ्ज्ञा प्राप्ता। एवं तदनुगामिनी द्वितीया\-विभक्तिः प्राप्ता। किन्त्वत्र चतुर्थ्यपाणिनीयेव। परं विचारे कृत इयमपि पाणिन्यनुकूला। अत्र \textcolor{red}{अगस्त्य\-मुनि\-वर्यं तोषयितुं मां निवेदय} इति \textcolor{red}{क्रियार्थोपपदस्य च कर्मणि स्थानिनः} (पा॰सू॰~२.३.१४) इत्यनेन चतुर्थी। यद्वा \textcolor{red}{तादर्थ्ये चतुर्थी वाच्या} (वा॰~२.३.१३) इत्यनेन \textcolor{red}{मुक्तये हरिं भजति} इत्यादिवच्चतुर्थी। यद्वा \textcolor{red}{अगस्त्य\-मुनि\-वर्याय} इति सुतीक्ष्ण\-सम्बोधनम्। एवम् \textcolor{red}{अगस्त्य\-मुनि\-वर्यमयतेऽन्तेवासित्वेन सेवार्थं गच्छतीत्यगस्त्य\-मुनि\-वर्यायः} तत्सम्बुद्धौ \textcolor{red}{अगस्त्य\-मुनि\-वर्याय}। अत्र \textcolor{red}{कर्मण्यण्‌} (पा॰सू॰~३.२.१) इत्यनेन \textcolor{red}{अण्‌} प्रत्ययः।\footnote{\textcolor{red}{अयँ गतौ} (धा॰पा॰~४७४) इति धातोः \textcolor{red}{अगस्त्य\-मुनि\-वर्यम्‌} इति कर्मोपपदे।} ततश्च \textcolor{red}{उपपदमतिङ्‌} (पा॰सू॰~२.२.१९) इत्यनेन समासे प्राप्ते \textcolor{red}{कृत्तद्धित\-समासाश्च} (पा॰सू॰~१.२.४६) इत्यनेन विभक्तिः प्राप्ता। द्वयोर्मध्ये कतरेण भाव्यमिति सामञ्जस्ये \textcolor{red}{गति\-कारकोपपदानां कृद्भिः सह समास\-वचनं प्राक्सुबुत्पत्तेः} (भा॰पा॰सू॰~८.४.११) इत्यनेन समास उपक्रान्ते पूर्वं \textcolor{red}{कर्तृ\-कर्मणोः कृति} (पा॰सू॰~२.३.६५) इत्यनेन कृद्योगे षष्ठी पश्चात्समासे पश्चाद्विभक्ति\-लोपे दीर्घे पुनर्विभक्ति\-कार्ये \textcolor{red}{अगस्त्य\-मुनि\-वर्यायः} तत्सम्बुद्धौ \textcolor{red}{हे अगस्त्य\-मुनि\-वर्याय} इति न दोषः। अर्थात्त्वं निरन्तरमगस्त्य\-मुनि\-वर्यमुपगच्छसि सेवार्थमतस्त्वं तस्य स्वभावेन परिचितोऽतो मामप्यागतं निवेदय।\footnote{यद्वा \textcolor{red}{अयत इत्ययः}। \textcolor{red}{अयः अयते अच् गन्तरि} इति वाचस्पत्यकारः। \textcolor{red}{नन्दि\-ग्रहि\-पचादिभ्यो ल्युणिन्यचः} (पा॰सू॰~३.१.१३४) इत्यनेन कर्तर्यच्। अगस्त्य\-मुनि\-वर्यस्यायः \textcolor{red}{अगस्त्यमुनिवर्यायः} तत्सम्बुद्धौ \textcolor{red}{अगस्त्य\-मुनि\-वर्याय}। \textcolor{red}{कर्तृकर्मणोः कृति} (पा॰सू॰~२.३.६५) इत्यनेन कृत्षष्ठ्यां \textcolor{red}{कृद्योगा च षष्ठी समस्यत इति वक्तव्यम्‌} (वा॰~२.२.८) इति वार्त्तिकेन तत्पुरुष\-समासः \textcolor{red}{इध्मप्रव्रश्चनः} इतिवत्।}\end{sloppypar}
\section[गुरोः]{गुरोः}
\centering\textcolor{blue}{अगस्त्यमुनिवर्याय सीतया लक्ष्मणेन च।\nopagebreak\\
महाप्रसाद इत्युक्त्वा सुतीक्ष्णः प्रययौ गुरोः॥}\nopagebreak\\
\raggedleft{–~अ॰रा॰~३.३.६}\\
\begin{sloppypar}\hyphenrules{nohyphenation}\justifying\noindent\hspace{10mm} अत्र श्रीराम\-वचनं श्रुत्वा सुतीक्ष्णः सहर्षमगस्त्यं प्रति गतः। अत्र \textcolor{red}{प्रययौ गुरोः} इति षष्ठ्यन्तं प्रयुक्तम्। तदपाणिनीय\-भ्रमावहम्। यतो हि \textcolor{red}{कर्तुरीप्सिततमं कर्म} (पा॰सू॰~१.४.४९) इति सूत्रम्। अस्य सामान्योऽर्थः \textcolor{red}{कर्तुः क्रिययाऽऽप्तुमिष्टतमं कारकं कर्म\-सञ्ज्ञं स्यात्‌} (वै॰सि॰कौ॰~५३५)। विशेषस्तु \textcolor{red}{कारके} (पा॰सू॰~१.४.२३) इत्यधिकार\-सूत्रेण क्रिया\-पदस्य लाभः। एवं या या क्रिया सा सा कर्तारं विनाऽनुपपन्नेति क्रिययैव कर्तुराक्षेपे सिद्धे पुनः कर्तुर्ग्रहणं व्यर्थं सज्ज्ञापयति यत् 
\textcolor{red}{प्रकृति\-धातूपात्त\-कर्तृ\-वृत्ति\-व्यापार\-प्रयोज्य\-फलाश्रयः कर्म}। तेन \textcolor{red}{भक्तो रामं भजति} इत्यत्र
प्रकृति\-धातूपात्त\-कर्ता भक्तस्तद्वृत्ति\-व्यापारो भजनानुकूलस्तत्प्रयोज्य\-फलं भजनं तदाश्रयो राम इत्यत्र कर्म\-सञ्ज्ञा। असति कर्तृ\-ग्रहणे क्रियाया लब्धस्य कर्तृ\-पदस्य प्रकृति\-धातु\-कर्तेत्यर्थो नावगन्तुं शक्येत। एवं \textcolor{red}{माषेष्वश्वं बध्नाति} इत्यत्र माषाणामपि कर्म\-सञ्ज्ञा स्यात्तथा च कर्ताऽश्वस्तद्वृत्ति\-व्यापारो गल\-विलाधः\-संयोगरूपस्तत्प्रयोज्य\-फलं भक्षणानुकूलं तदाश्रयः कर्म माषा एवेति माषाणामेव कर्म\-सञ्ज्ञा स्यात्। सति \textcolor{red}{कर्तुः} इति पद\-ग्रहणे प्रकृति\-धातूपात्तो यो व्यापारस्तदाश्रयो यः कर्तेत्यर्थे सम्पन्ने प्रकृति\-धातुः \textcolor{red}{बन्ध्‌}\-धातुः (\textcolor{red}{बन्धँ बन्धने} धा॰पा॰~१५०८) तद्वृत्ति\-व्यापारः शङ्कु\-संयोगानुकूलस्तदाश्रयो देवदत्तस्तादृक्कर्तृ\-वृत्ति\-व्यापार\-प्रयोज्यं फलं बन्धन\-रूपं तदाश्रयः कर्माश्व एव। एवं \textcolor{red}{कर्तुः} इत्यत्र षष्ठी \textcolor{red}{क्तस्य च वर्तमाने} (पा॰सू॰~२.३.६७) इत्यनेन। साऽपि कर्तर्येव। अतः कर्तृ\-पदमपि कर्तारं बोधयति षष्ठ्यपि। \textcolor{red}{कर्तैव यः कर्ता} इत्यर्थे सति \textcolor{red}{प्रधान\-भूत\-व्यापाराश्रय\-कर्ता} इत्यर्थोऽवगम्यते। तथा च \textcolor{red}{ईप्सिततमम्‌} इति \textcolor{red}{आप्‌}\-धातोः (\textcolor{red}{आपॢँ लम्भने} धा॰पा॰~१८३९) सन्नन्त\-रूपम्।
\textcolor{red}{आप्तुमिष्टमीप्सितम्‌}।\footnote{\textcolor{red}{आप्तुमिष्यमाणमीप्सितम्} (बा॰म॰~५३५) इति बाल\-मनोरमा। \textcolor{red}{इष्टम्} इत्यत्रापि \textcolor{red}{मतिबुद्धि\-पूजार्थेभ्यश्च} (पा॰सू॰~३.२.१८८) इत्यनेनेच्छार्थे वर्तमान\-काल एव \textcolor{red}{क्त}\-प्रत्ययः। तस्मादुभे अप्युक्ती समानार्थिके। \textcolor{red}{आपॢँ लम्भने} (धा॰पा॰~१८३९)~\arrow आप्‌~\arrow \textcolor{red}{धातोः कर्मणः समानकर्तृकादिच्छायां वा} (पा॰सू॰~३.१.७)~\arrow आप्~सन्~\arrow आप्~स~\arrow \textcolor{red}{एकाच उपदेशेऽनुदात्तात्‌} (पा॰सू॰~७.२.१०)~\arrow इडागम\-निषेधः~\arrow \textcolor{red}{आप्ज्ञप्यृधामीत्‌} (पा॰सू॰~७.४.५५)~\arrow ईप्~स~\arrow \textcolor{red}{सन्यङोः} (पा॰सू॰~६.१.९)~\arrow ईप्~ईप्~स~\arrow \textcolor{red}{पूर्वोऽभ्यासः} (पा॰सू॰~६.१.४)~\arrow \textcolor{red}{अत्र लोपोऽभ्यासस्य} (पा॰सू॰~७.४.५८)~\arrow ईप्~स~\arrow \textcolor{red}{सनाद्यन्ता धातवः} (पा॰सू॰~३.१.३२)~\arrow धातु\-सञ्ज्ञा~\arrow \textcolor{red}{तयोरेव कृत्य\-क्तखलर्थाः} (पा॰सू॰~३.४.७०)~\arrow \textcolor{red}{मतिबुद्धि\-पूजार्थेभ्यश्च} (पा॰सू॰~३.२.१८८)~\arrow ईप्स~क्त~\arrow ईप्स~त~\arrow \textcolor{red}{आर्धधातुकस्येड्वलादेः} (पा॰सू॰~७.२.३५)~\arrow ईप्स~इट्~त~\arrow ईप्स~इ~त~\arrow ईप्सित~\arrow विभक्ति\-कार्यम्~\arrow ईप्सित~सुँ~\arrow \textcolor{red}{अतोऽम्} (पा॰सू॰~७.१.२४)~\arrow ईप्सित~अम्~\arrow \textcolor{red}{अमि पूर्वः} (पा॰सू॰~६.१.१०७)~\arrow ईप्सितम्।} \textcolor{red}{अतिशयेनेप्सितमितीप्सित\-तमम्‌}\footnote{\textcolor{red}{अतिशयेनेप्सितमीप्सित\-तमम्‌} (बा॰म॰~५३५)।} इति विग्रहे \textcolor{red}{अतिशायने तमबिष्ठनौ} (पा॰सू॰~५.३.५५) इत्यनेन \textcolor{red}{तमप्‌}\-प्रत्ययः। तथा च प्रकृति\-धातूपात्त\-प्रधान\-भूत\-व्यापाराश्रय\-कर्तृ\-वृत्ति\-व्यापार\-प्रयोज्य\-फलाश्रयता\-प्रकारिकेच्छा\-निरूपितोद्देश्यता\-फलाश्रयत्वावच्छेदकं कर्म। एवं च \textcolor{red}{भक्तो रामं भजति} इत्यत्र प्रकृति\-धातुः \textcolor{red}{भज्‌}\-धातुः (\textcolor{red}{भजँ सेवायाम्‌} धा॰पा॰~९९८) तत्प्रधान\-भूत\-व्यापारो भजनानुकूल\-व्यापारस्तदाश्रयः कर्ता भक्तस्तद्वृत्तिर्भजनानुकूल\-व्यापार\-प्रकारिकेच्छा यद्वृत्ति\-व्यापारेण रामः सन्तुष्टो भवतीत्याकारिका तन्निरूपितोद्देश्यता\-फलाश्रयत्वावच्छेदकं राम एव तस्यैव कर्म\-सञ्ज्ञा।
इत्थं \textcolor{red}{कर्तुरीप्सिततमं कर्म} (पा॰सू॰~१.४.४९) इत्यत्रेप्सिततम\-ग्रहणाभावे \textcolor{red}{कर्तुः कर्म} इति सूत्रे सम्पन्ने \textcolor{red}{कर्तुरुद्देश्यं कर्म} इत्यर्थे \textcolor{red}{पयसौदनं भुङ्क्ते} इति प्रत्युदाहरणम्। अर्थात्पयसा मिश्रमोदनं भुङ्क्ते। कृत\-भोजनं प्रति यदि कश्चिद्ब्रवीति यत् \textcolor{red}{ओदनं भुङ्क्ष्व तुभ्यं पयो दास्यामि} ततश्च स भोजने प्रवर्तते तर्ह्येव \textcolor{red}{पय ओदनं भुङ्क्ते} इति प्रयुज्यते। \textcolor{red}{पयसा मिश्रमोदनं भुङ्क्ते} इत्यर्थे \textcolor{red}{पयसौदनं भुङ्क्ते} इति प्रयुज्यते। अत्र तृतीया। \textcolor{red}{कर्तुरुद्देश्यं कर्म} इत्यर्थे कृतेऽत्रापि द्वितीया स्यात्। पयसोऽपि कर्तुरुद्देश्यत्वात्। पय उद्देश्यं किन्त्वीप्सिततमं नास्ति। न च \textcolor{red}{तमप्‌}\-ग्रहणं 
मा भूत् \textcolor{red}{ईप्सित}\-ग्रहणेनैव कार्ये सिद्धे तथा च \textcolor{red}{कर्तुरीप्सितं कर्म} इत्येव सूत्रं स्यात्। तथा सति \textcolor{red}{अग्नेर्माणवकं वारयति} इत्यत्र \textcolor{red}{अग्नेः} इत्यत्रापि कर्म\-सञ्ज्ञा स्यात्। \textcolor{red}{वारयति} इत्यस्यार्थः \textcolor{red}{अग्नि\-संयोगानुकूल\-व्यापाराभावानुकूल\-व्यापारः}। मद्वृत्ति\-व्यापारेणाग्निर्बालक\-संयोगाभाववान् भवत्वितीच्छाश्रयः। \textcolor{red}{अग्नेः} इत्यत्रापि कर्म\-सञ्ज्ञा मा भूत्तस्मात्तमब्ग्रहणम्।
न च \textcolor{red}{वारणार्थानामीप्सितः} (पा॰सू॰~१.४.२७) इत्यपादान\-सञ्ज्ञा कर्म\-सञ्ज्ञां बाधिष्यते सा च \textcolor{red}{माणवकम्} इत्यत्र स्यादियं कर्म\-सञ्ज्ञा च। परत्वात्कर्म\-सञ्ज्ञैव।\footnote{\textcolor{red}{“गाम्” इत्यत्रेप्सितत्व\-प्रयुक्ताऽपादानसञ्ज्ञा न भवति। ईप्सिततमत्व\-विवक्षायां परत्वात्कर्म\-संज्ञाप्रवृत्तेः} (त॰बो॰~५९०)।} सूत्राभावे \textcolor{red}{अग्नेः} इत्यत्र कर्म\-सञ्ज्ञां च को वारयिष्यति।\footnote{\textcolor{red}{कर्तुरीप्सिततमं कर्म} (पा॰सू॰~१.४.४९) इति सूत्रस्याभावे \textcolor{red}{अकथितं च} (पा॰सू॰~१.४.५१) इत्यनेनापि \textcolor{red}{माणवकम्} इत्यत्रेप्सित\-तमस्य कर्मसञ्ज्ञा प्राप्नोति परन्त्वेतेन सूत्रेणेप्सितस्याप्यकथितस्य कर्म\-सञ्ज्ञा\-प्राप्तौ \textcolor{red}{*अग्निं माणवकं वारयति} इत्यनिष्ट\-प्रयोगः स्यात्तस्मात्सूत्रारम्भ आवश्यक इति भावः।}
न च \textcolor{red}{अधि\-शीङ्\-स्थासां कर्म} (पा॰सू॰~१.४.४६) इत्यतः \textcolor{red}{कर्म} इति पदमनुवर्त्यतां पुनरत्र \textcolor{red}{कर्म}\-ग्रहणं किमर्थमिति चेत्। यदि ततः \textcolor{red}{कर्म} इति पदं तर्हि \textcolor{red}{एक\-योग\-निर्दिष्टानां सह वा प्रवृत्तिः सह वा निवृत्तिः} (प॰शे॰~१७) इति नियमेनाधारस्यैव कर्म\-सञ्ज्ञा स्यात् \textcolor{red}{गेहं प्रविशति} इत्यादावेव। तदर्थं कर्म\-ग्रहणम्। \textcolor{red}{क्वचिदेक\-देशोऽप्यनुवर्तते} (प॰शे॰~१८) इति नियमेनाधार\-निवृत्तौ कर्म\-ग्रहणमदृष्टार्थम्। इत्यर्थं \textcolor{red}{कर्तुरीप्सिततमं कर्म} इत्यनेन सञ्ज्ञाऽनिवार्या। तथा च \textcolor{red}{प्रययौ} इति \textcolor{red}{या}\-धातोः (\textcolor{red}{या प्रापणे} धा॰पा॰~१०४९) कर्ता सुतीक्ष्णस्तद्\-वृत्ति\-व्यापार उत्तर\-देश\-संयोगानुकूल\-व्यापारस्तत्प्रयोज्य\-फलाश्रयत्व\-प्रकारिकेच्छा मद्वृत्तिव्यापारेण गुरुर्मत्संयोगवान् भवतु तन्निरूपितोद्देश्यता गुरावेव ततः स एव कर्म। अतोऽत्र कथं न कर्म\-सञ्ज्ञेति चेत्। उच्यते। कर्मणः शेषत्व\-विवक्षायां षष्ठी। कर्म\-सञ्ज्ञा तु जातैव केवलं तत्फल\-भूता द्वितीया न।
ननु \textcolor{red}{या या सञ्ज्ञा सा सा फलवती भवति} इति न्यायेन फलाभावेन कर्म\-सञ्ज्ञायाः किं प्रयोजनम्।
असत्याञ्च कर्म\-सञ्ज्ञायां सूत्राप्रवृत्तौ न साधुताऽभावे साधुताया न पुण्य\-जनकतावच्छेदकतेति चेत्। कर्म\-सञ्ज्ञायाः फलं न केवलं द्वितीया। अन्यान्य\-फलानि सन्ति। अत्र किं फलमिति चेत्। \textcolor{red}{गुरोः प्रययौ} इत्यत्र हि \textcolor{red}{गुरु\-कर्मक\-भूतकालावच्छिन्नोत्तर\-देश\-संयोगानुकूल\-व्यापारः} इत्यत्र शाब्द\-बोधे कर्म\-मूलक\-सम्बन्ध\-बोधयोः फलम्। यद्वा \textcolor{red}{गुरोः सकाशं प्रययौ} इति पदाध्याहारे सम्बन्धे षष्ठी।\end{sloppypar}
\section[शिष्येभ्यः]{शिष्येभ्यः}
\centering\textcolor{blue}{व्याख्यातराममन्त्रार्थं शिष्येभ्यश्चातिभक्तितः।\nopagebreak\\
दृष्ट्वाऽगस्त्यं मुनिश्रेष्ठं सुतीक्ष्णः प्रययौ मुनेः॥}\nopagebreak\\
\raggedleft{–~अ॰रा॰~३.३.८}\\
\begin{sloppypar}\hyphenrules{nohyphenation}\justifying\noindent\hspace{10mm} अत्र शिष्यान् प्रति राम\-मन्त्रार्थं व्याचक्षाणस्यागस्त्यस्य स्थितिं वर्णयति। अत्र \textcolor{red}{शिष्यान् प्रति} इति प्रयोक्तव्यमासीत्। \textcolor{red}{शिष्येभ्यः} इति चतुर्थी\-प्रयोगस्तु \textcolor{red}{शिष्यान् बोधयितुम्‌} इत्यप्रयुज्यमान\-तुमुनः कर्मणि चतुर्थी।\footnote{\textcolor{red}{क्रियार्थोप\-पदस्य च कर्मणि स्थानिनः} (पा॰सू॰~२.३.१४) इत्यनेन।} यद्वा सुख\-योगे सति चतुर्थी।\footnote{\textcolor{red}{शिष्येभ्यः सुखाय} इत्यध्याहार्यमिति भावः।} यद्वा \textcolor{red}{शिष्येभ्यः हिताय} इत्यध्याहारे \textcolor{red}{हित\-योगे च} (वा॰~२.३.१३) इत्यनेन चतुर्थी। यद्वा \textcolor{red}{तादर्थ्ये चतुर्थी वाच्या} (वा॰~२.३.१३) इति वार्त्तिकेन चतुर्थी। न च केभ्यो राम\-मन्त्रार्थो व्याख्यात इत्यपेक्षायां \textcolor{red}{शिष्येभ्यः} इत्यस्य व्याख्यात\-घटक\-व्याख्यान\-क्रियाया अपेक्षितत्वात्कथं बहुव्रीहिरिति चेत्। नित्य\-साकाङ्क्ष\-स्थल उक्त\-नियमस्य प्रसक्त्यभावात्। यद्वा \textcolor{red}{शिष्यान् आहूय} वा \textcolor{red}{शिष्यान् अपेक्ष्य व्याख्यात\-राम\-मन्त्रार्थः} इति योजनया \textcolor{red}{ल्यब्लोपे कर्मण्यधिकरणे च} (वा॰~२.३.२८) इति वार्त्तिकेनात्र पञ्चमी। अस्य वार्त्तिकस्यार्थः \textcolor{red}{ल्यबन्तस्य लोपे सति तस्य कर्मण्यधिकरणे च पञ्चमी भवति}। अत्र हि ल्यबन्तम् \textcolor{red}{आहूय} अपेक्षतयाऽस्य कर्म \textcolor{red}{शिष्यान्‌} इति ततः पञ्चमी \textcolor{red}{श्वशुराज्जिह्रेति} (वै॰सि॰कौ॰~५९४) इतिवत्। इत्थं \textcolor{red}{शिष्येभ्यः} इति चतुर्थ्यन्तः पञ्चम्यन्तो वा पाणिनीय एव। \end{sloppypar}
\section[ब्रवीमि ते]{ब्रवीमि ते}
\centering\textcolor{blue}{किं राम बहुनोक्तेन सारं किञ्चिद्ब्रवीमि ते।\nopagebreak\\
साधुसङ्गतिरेवात्र मोक्षहेतुरुदाहृता॥}\nopagebreak\\
\raggedleft{–~अ॰रा॰~३.३.३६}\\
\begin{sloppypar}\hyphenrules{nohyphenation}\justifying\noindent\hspace{10mm} अत्रागस्त्यः श्रीरामं प्रति गोपनीय\-विषयं वर्णयति यत् \textcolor{red}{ते किञ्चित्सारं ब्रवीमि}। \textcolor{red}{ब्रू}\-धातुः (\textcolor{red}{ब्रूञ् व्यक्तायां वाचि} धा॰पा॰~१०४४) अत्राकथित\-परिगणित\-धात्वन्तर्गतः~– \textcolor{red}{चिब्रूशासुजिमथ्मुषाम्‌} (वै॰सि॰कौ॰~५३९)। अत्र \textcolor{red}{त्वां ब्रवीमि} इति पाणिनीयम्। \textcolor{red}{ते ब्रवीमि} इति कथम्। कर्मणः शेषत्व\-विवक्षायां षष्ठी। यद्वा \textcolor{red}{समक्षम्‌} इत्यध्याहार्यम्। एवं \textcolor{red}{ते} इति\-शब्दे सम्बन्ध\-विवक्षायां षष्ठी। अथवा \textcolor{red}{त्वां स्तोतुं ब्रवीमि} इति तुमुन्यप्रयुक्ते तत्कर्मणि \textcolor{red}{क्रियार्थोपपदस्य च कर्मणि स्थानिनः} (पा॰सू॰~२.३.१४) इति चतुर्थी। अथवा \textcolor{red}{ते} इत्यस्य \textcolor{red}{सारम्‌} इत्यनेनान्वयस्तेन सम्बन्धे षष्ठी। \textcolor{red}{त्वत्सम्बन्धि\-सारं ब्रवीमि} इत्यर्थः। अथवा \textcolor{red}{शृण्वतः} इति शत्रन्तमध्याहार्यम्। \textcolor{red}{ते शृण्वतः सारं ब्रवीमि} इत्यर्थे \textcolor{red}{यस्य च भावेन भाव\-लक्षणम्‌} (पा॰सू॰~२.३.३७) इति षष्ठी।\footnote{\textcolor{red}{दूरान्तिकार्थैः षष्ठ्यन्यतरस्याम्‌} (पा॰सू॰~२.३.३४) इत्यतः \textcolor{red}{षष्ठी} इत्यनुवर्त्य \textcolor{red}{षष्ठी चानादरे} (पा॰सू॰~२.३.३८) इत्यतः \textcolor{red}{षष्ठी} इत्यपकृष्य वाऽऽदरेऽपि भावलक्षणा षष्ठीति भावः।}\end{sloppypar}
\section[नेतव्यस्तत्र ते कालः]{नेतव्यस्तत्र ते कालः}
\centering\textcolor{blue}{अस्ति पञ्चवटीनाम्ना आश्रमो गौतमीतटे।\nopagebreak\\
नेतव्यस्तत्र ते कालः शेषो रघुकुलोद्वह॥}\nopagebreak\\
\raggedleft{–~अ॰रा॰~३.३.४८}\\
\begin{sloppypar}\hyphenrules{nohyphenation}\justifying\noindent\hspace{10mm} अत्र अगस्त्यः श्रीरामचन्द्रं दण्डक\-वासाय प्रेरयति यत् \textcolor{red}{रामभद्र तत्रैव ते भवतः कालो नेतव्यः}। आशङ्क्योऽयम्।\footnote{पूर्वपक्षोऽयम्।} \textcolor{red}{णीञ् प्रापणे} (धा॰पा॰~९०१) इत्यस्माद्धातोः \textcolor{red}{तयोरेव कृत्य\-क्त\-खलर्थाः} (पा॰सू॰~३.४.७०) इति सूत्र\-सहकारेण \textcolor{red}{तव्यत्तव्यानीयरः} (पा॰सू॰~३.१.९६) इत्यनेन तव्यत्प्रत्ययः स च कर्मणि। तेन कर्ताऽनुक्तः। तत्रैव तृतीया भाव्या।\footnote{\textcolor{red}{कर्तृ\-करणयोस्तृतीया} (पा॰सू॰~२.३.१८) इत्यनेन।} अत्र \textcolor{red}{कृत्यानां कर्तरि वा} (पा॰सू॰~२.३.७१) इत्यनेन वैकल्पिकी षष्ठी।\footnote{पक्षे \textcolor{red}{कर्तृ\-करणयोस्तृतीया} (पा॰सू॰~२.३.१८) इत्यनेन तृतीयाऽपि।} अथवा \textcolor{red}{ते} इत्यस्य \textcolor{red}{कालः} इत्यनेनान्वयः। अर्थात्त्वत्सम्बन्धी कालस्तत्रैव नेतव्यः।
\textcolor{red}{अकथितं च} (पा॰सू॰~१.४.५१) इत्यत्र परिगणित\-षोडश\-धातूनां मध्य उत्तरार्धे \textcolor{red}{तथा स्यान्नीहृकृष्वहाम्‌} (वै॰सि॰कौ॰~५३९) इत्यत्र \textcolor{red}{नी} इत्यस्य ग्रहणात्तथा च \textcolor{red}{ग्राममजां नयति} इतिवदत्रापि \textcolor{red}{तत्र} इत्यस्य कर्म\-सञ्ज्ञा भवेत्।\footnote{अयमपि पूर्वपक्षः।} न च कर्म\-सञ्ज्ञायामेवं त्रल्। \textcolor{red}{सप्तमी शौण्डैः} (पा॰सू॰~२.१.४०) इत्यत्र सप्तमी\-पदस्य ग्रहणात् \textcolor{red}{सप्तम्यास्त्रल्‌} (पा॰सू॰~५.३.१०) इत्यनेन \textcolor{red}{तत्र} इति सप्तम्यन्तस्य प्रयोगः।\footnote{तद्~ङि~\arrow \textcolor{red}{सप्तम्यास्त्रल्‌} (पा॰सू॰~५.३.१०)~\arrow तद्~ङि~त्रल्~\arrow \textcolor{red}{कृत्तद्धित\-समासाश्च} (पा॰सू॰~१.२.४६)~\arrow प्रातिपदिक\-सञ्ज्ञा~\arrow \textcolor{red}{सुपो धातु\-प्रातिपदिकयोः} (पा॰सू॰~२.४.७१)~\arrow तद्~त्र~\arrow \textcolor{red}{प्राग्दिशो विभक्तिः} (पा॰सू॰~५.३.१)~\arrow \textcolor{red}{त्यदादीनामः} (पा॰सू॰~७.२.१०२)~\arrow त~अ~त्र~\arrow \textcolor{red}{अतो गुणे} (पा॰सू॰~६.१.९७)~\arrow त~त्र~\arrow तत्र~\arrow \textcolor{red}{तद्धितश्चासर्व\-विभक्तिः} (पा॰सू॰~१.१.३८)~\arrow अव्यय\-सञ्ज्ञा~\arrow विभक्ति\-कार्यम्~\arrow तत्र~सुँ~\arrow \textcolor{red}{अव्ययादाप्सुपः} (पा॰सू॰~२.४.८२)~\arrow तत्र। पक्षे \textcolor{red}{तस्मिन्} इति। तद्~ङि~\arrow \textcolor{red}{ङसिङ्योः स्मात्स्मिनौ} (पा॰सू॰~७.१.१५)~\arrow तद्~स्मिन्~\arrow \textcolor{red}{त्यदादीनामः} (पा॰सू॰~७.२.१०२)~\arrow त~अ~स्मिन्~\arrow \textcolor{red}{अतो गुणे} (पा॰सू॰~६.१.९७)~\arrow त~स्मिन्~\arrow तस्मिन्।} सप्तमीं विना द्वितीयातस्त्रल् कथं सम्भवेत्। उच्यते। यदाऽपादानादिभिर्न विवक्षा स्यात्तदा कर्म\-सञ्ज्ञा भवति।\footnote{\textcolor{red}{अपादानादि\-विशेषैरविवक्षितं कारकं कर्म\-सञ्ज्ञं स्यात्‌} (वै॰सि॰कौ॰~५३९)।} अधुना त्वधिकरण\-विवक्षैव। आत्म\-समानाधिकरणस्य सकल\-जगदधिकरणस्य विविध\-दिव्याभरणस्य जगदाधारस्यापि रामस्याऽधार\-भूतत्वात् \textcolor{red}{नेतव्यस्तत्र ते कालः} इति पञ्चवटीं जगदाधारस्याऽधारभूतामिति साम्प्रदायिका ध्वनयितुं षष्ठीं युक्तिभिः साधयन्ति।\end{sloppypar}
\section[तयोः]{तयोः}
\centering\textcolor{blue}{अध्युवास सुखं रामो देवलोक इवापरः।\nopagebreak\\
कन्दमूलफलादीनि लक्ष्मणोऽनुदिनं तयोः॥\\
आनीय प्रददौ रामसेवातत्परमानसः।\nopagebreak\\
धनुर्बाणधरो नित्यं रात्रौ जागर्ति सर्वतः॥}\nopagebreak\\
\raggedleft{–~अ॰रा॰~३.४.१२-१३}\\
\begin{sloppypar}\hyphenrules{nohyphenation}\justifying\noindent\hspace{10mm} दण्डकारण्ये निवसतोः \textcolor{red}{तयोर्लक्ष्मणः कन्द\-मूल\-फलानि प्रददौ}। अत्र \textcolor{red}{दा}\-धातु\-योगे (\textcolor{red}{डुदाञ् दाने} धा॰पा॰~१०९१) चतुर्थ्यां \textcolor{red}{ताभ्याम्‌} इति भवेत्।\footnote{\textcolor{red}{चतुर्थी सम्प्रदाने} (पा॰सू॰~२.३.१३) इत्यनेन।} \textcolor{red}{तयोः} इति कथनं तु सम्प्रदाने सम्बन्ध\-विवक्षायाम्। यद्वा \textcolor{red}{तयोः पुरतः} इत्यध्याहारे सामान्य\-सम्बन्धे षष्ठी। यद्वा \textcolor{red}{निवसतोः तयोः} इति शत्रन्ते \textcolor{red}{यस्य च भावेन भाव\-लक्षणम्‌} (पा॰सू॰~२.३.३७) इति षष्ठीसप्तम्यौ।\footnote{\textcolor{red}{दूरान्तिकार्थैः षष्ठ्यन्यतरस्याम्‌} (पा॰सू॰~२.३.३४) इत्यतः \textcolor{red}{षष्ठी} इत्यनुवर्त्य \textcolor{red}{षष्ठी चानादरे} (पा॰सू॰~२.३.३८) इत्यतः \textcolor{red}{षष्ठी} इत्यपकृष्य वाऽऽदरेऽपि \textcolor{red}{यस्य च भावेन भाव\-लक्षणम्‌} (पा॰सू॰~२.३.३७) इत्यनेन भावलक्षणा षष्ठीति भावः।} अथवा \textcolor{red}{कन्द\-मूल\-फलानि} इत्यनेन साकमन्वित्य \textcolor{red}{तयोः एव कन्द\-मूल\-फलानि} इत्यर्थे सम्बन्धे षष्ठी। सीता च प्रकृतिः श्रीरामः पुरुषस्तयोः संयोगात्सृष्टिरिति सार्वभौमत्वादध्यात्म\-रामायणे च तत्र तत्र प्रतिपादितत्वाज्जन्य\-जनक\-भाव\-सम्बन्धे षष्ठ्यैश्वर्यं द्रढयितुम्।\end{sloppypar}
\section[मे]{मे}
\centering\textcolor{blue}{ज्ञानं विज्ञानसहितं भक्तिवैराग्यबृंहितम्।\nopagebreak\\
आचक्ष्व मे रघुश्रेष्ठ वक्ता नान्योऽस्ति भूतले॥}\nopagebreak\\
\raggedleft{–~अ॰रा॰~३.४.१८}\\
\begin{sloppypar}\hyphenrules{nohyphenation}\justifying\noindent\hspace{10mm} अत्र लक्ष्मणः श्रीरामं प्रार्थयते यत् \textcolor{red}{ज्ञानं विज्ञानञ्च मे आचक्ष्व}। अत्र \textcolor{red}{आचक्ष्व} इति प्रयुक्तम्। \textcolor{red}{चक्ष्‌}\-धातोः (\textcolor{red}{चक्षिङ् व्यक्तायां वाचि} धा॰पा॰~१०१७) लोड्\-लकारस्य मध्यम\-पुरुषस्यैक\-वचनान्तं रूपमिदम्। अयं हि धातुः कथनार्थः। कथनार्थक\-धातूनां योगे द्वितीया प्रसिद्धा। अत्र चतुर्थी तु \textcolor{red}{मे हिताय आचक्ष्व} इति तात्पर्य एवं \textcolor{red}{हित\-योगे च} (वा॰~२.३.१३) इति चतुर्थी। अथवा \textcolor{red}{मे मम} इति षष्ठ्यन्त\-प्रयोगः पश्चात् \textcolor{red}{परतः} इत्यध्याहार्यम्। अथवा \textcolor{red}{पृच्छतः} इति शत्रन्त\-प्रयोग एवं \textcolor{red}{यस्य च भावेन भाव\-लक्षणम्‌} (पा॰सू॰~२.३.३७) इत्यनेन षष्ठी।\footnote{\textcolor{red}{दूरान्तिकार्थैः षष्ठ्यन्यतरस्याम्‌} (पा॰सू॰~२.३.३४) इत्यतः \textcolor{red}{षष्ठी} इत्यनुवर्त्य \textcolor{red}{षष्ठी चानादरे} (पा॰सू॰~२.३.३८) इत्यतः \textcolor{red}{षष्ठी} इत्यपकृष्य वाऽऽदरेऽपि भावलक्षणा षष्ठीति भावः।}\end{sloppypar}
\section[ते]{ते}
\centering\textcolor{blue}{शृणु वक्ष्यामि ते वत्स गुह्याद्गुह्यतरं परम्।\nopagebreak\\
यद्विज्ञाय नरो जह्यात्सद्यो वैकल्पकं भ्रमम्॥}\nopagebreak\\
\raggedleft{–~अ॰रा॰~३.४.१९}\\
\begin{sloppypar}\hyphenrules{nohyphenation}\justifying\noindent\hspace{10mm} श्रीरामो लक्ष्मणं प्रति वक्ति यत् \textcolor{red}{ते गृह्याद्गुह्यतरं वक्ष्यामि}। \textcolor{red}{वक्ष्यामि} इति \textcolor{red}{ब्रू}\-धातु\-निष्पन्नो (\textcolor{red}{ब्रूञ् व्यक्तायां वाचि} धा॰पा॰~१०४४) लृडुत्तम\-पुरुषैक\-वचनान्तः।\footnote{\textcolor{red}{ब्रूञ् व्यक्तायां वाचि} (धा॰पा॰~१०४४)~\arrow ब्रू~\arrow \textcolor{red}{शेषात्कर्तरि परस्मैपदम्} (पा॰सू॰~१.३.७८)~\arrow \textcolor{red}{लृट् शेषे च} (पा॰सू॰~३.३.१३)~\arrow ब्रू~लृट्~\arrow ब्रू~मिप्~\arrow ब्रू~मि~\arrow \textcolor{red}{ब्रुवो वचिः} (पा॰सू॰~२.४.५३)~\arrow वच्~मि~\arrow \textcolor{red}{स्यतासी लृलुटोः} (पा॰सू॰~३.१.३३)~\arrow वच्~स्य~मि~\arrow \textcolor{red}{एकाच उपदेशेऽनुदात्तात्‌} (पा॰सू॰~७.२.१०)~\arrow इडागम\-निषेधः~\arrow वच्~स्य~मि~\arrow \textcolor{red}{चोः कुः} (पा॰सू॰~८.२.३०)~\arrow वक्~स्य~मि~\arrow \textcolor{red}{अतो दीर्घो यञि} (पा॰सू॰~७.३.१०१)~\arrow वक्~स्या~मि~\textcolor{red}{आदेश\-प्रत्यययोः} (पा॰सू॰~८.३.५९)~\arrow वक्~ष्या~मि~\arrow वक्ष्यामि।} अत्र \textcolor{red}{ब्रू}\-धातु\-योगे द्वितीया प्राप्तैव। चतुर्थी तु \textcolor{red}{त्वां बोधयितुं वक्ष्यामि} इत्यप्रयुज्यमान\-तुमुन्कर्मणि चतुर्थी।\footnote{\textcolor{red}{क्रियार्थोपपदस्य च कर्मणि स्थानिनः} (पा॰सू॰~२.३.१४) इत्यनेन।} \textcolor{red}{ते हितं वक्ष्यामि} इत्यध्याहारे \textcolor{red}{हित\-योगे च} (वा॰~२.३.१३) इत्यनेन वा। \textcolor{red}{पृष्टवतः ते वक्ष्यामि} इत्यध्याहारे भाव\-लक्षणा षष्ठी वा।\footnote{\textcolor{red}{दूरान्तिकार्थैः षष्ठ्यन्यतरस्याम्‌} (पा॰सू॰~२.३.३४) इत्यतः \textcolor{red}{षष्ठी} इत्यनुवर्त्य \textcolor{red}{षष्ठी चानादरे} (पा॰सू॰~२.३.३८) इत्यतः \textcolor{red}{षष्ठी} इत्यपकृष्य वाऽऽदरेऽपि \textcolor{red}{यस्य च भावेन भाव\-लक्षणम्‌} (पा॰सू॰~२.३.३७) इत्यनेन भावलक्षणा षष्ठीति भावः।} \textcolor{red}{ते समक्षं वक्ष्यामि} इत्यध्याहार\-बलेन षष्ठी वा।\end{sloppypar}
\section[मे]{मे}
\centering\textcolor{blue}{एतैर्विलक्षणो जीवः परमात्मा निरामयः।\nopagebreak\\
तस्य जीवस्य विज्ञाने साधनान्यपि मे शृणु॥}\nopagebreak\\
\raggedleft{–~अ॰रा॰~३.४.३०}\\
\begin{sloppypar}\hyphenrules{nohyphenation}\justifying\noindent\hspace{10mm} श्रीरामः श्रीलक्ष्मणमामन्त्रयति \textcolor{red}{मे शृणु} इति। अत्र विषयः श्रीराम\-मुखान्निर्गत्य लक्ष्मण\-कर्णौ प्रविशतीति शब्द\-विश्लेषे श्रीरामस्य ध्रुवत्वेनावधि\-भूतत्वात्तत्रापादान\-सञ्ज्ञा तन्मूलिका च पञ्चम्याशङ्क्यते। किन्त्वत्र शब्दानां विश्लेषोऽपेक्ष्यत एव नहि। वैयाकरणानां नये शब्दो ब्रह्म। \textcolor{red}{वाग्वै ब्रह्म} (बृ॰उ॰~१.३.२१) इति श्रुतेः। श्रीरामस्तु पर\-ब्रह्म शब्द\-ब्रह्म यस्य निःश्वास\-भूतम्। \textcolor{red}{अस्य महतो भूतस्य निःश्वसितमेतद्यदृग्वेदो यजुर्वेदः सामवेदोऽथर्वाङ्गिरस} (बृ॰उ॰~२.४.१०) इति श्रुतेः। \textcolor{red}{यस्य निश्श्वसितं वेदाः} (ऋ॰वे॰सं॰ सा॰भा॰ उ॰प्र॰~२) इति वचनाच्च। तर्हि तस्य शब्द\-ब्रह्मणः श्रीरामाद्विश्लेषः सम्भव एव नहि। ब्रह्मणो रामचन्द्रस्य सर्व\-व्यापित्वात्। लक्ष्मणश्च भगवदंशः। \textcolor{red}{शेषांशो लक्ष्मणः साक्षात्‌} इति वचनात्।\footnote{मूलं मृग्यम्।} भगवान् श्रीरामोंऽशी। अंशांशिनोरप्यभेदात्तेन सम्बन्ध\-विवक्षायामेव षष्ठ्यत्रत्य\-भाव\-गरिमाणं गरयति। यद्वा \textcolor{red}{मे सकाशात् शृणु} इत्यध्याहारेण सम्बन्ध\-षष्ठी। यद्वा \textcolor{red}{मे मह्यम्‌} इति चतुर्थी सा च \textcolor{red}{माम् तोषयितुम् माम् अनुकूलयितुम्‌} वेति तुमुन्कर्मणि चतुर्थी।\footnote{\textcolor{red}{क्रियार्थोपपदस्य च कर्मणि स्थानिनः} (पा॰सू॰~२.३.१४) इत्यनेन।} सावधान\-श्रोतुरुपरि वक्तुः प्रीतिर्वर्धत इति न केषामप्यविदितम्।\end{sloppypar}
\section[चक्षुष्मताम्]{चक्षुष्मताम्‌}
\centering\textcolor{blue}{चक्षुष्मतामपि तथा रात्रौ सम्यङ्न दृश्यते।\nopagebreak\\
पदं दीपसमेतानां दृश्यते सम्यगेव हि॥}\nopagebreak\\
\raggedleft{–~अ॰रा॰~३.४.४६}\\
\begin{sloppypar}\hyphenrules{nohyphenation}\justifying\noindent\hspace{10mm} अत्र \textcolor{red}{दृश्यते} इति हि भाव\-वाच्य\-प्रयोगः। अत्र कर्तुरनुक्तत्वात्तत्र तृतीयया भवितव्यम्।\footnote{\textcolor{red}{कर्तृ\-करणयोस्तृतीया} (पा॰सू॰~२.३.१८) इत्यनेन।} \textcolor{red}{चक्षुष्मद्भिर्न दृश्यते} इत्येव। षष्ठी तु \textcolor{red}{विवक्षाधीनानि कारकाणि भवन्ति}\footnote{मूलं मृग्यम्। यद्वा \textcolor{red}{कर्मादीनामविवक्षा शेषः} (भा॰पा॰सू॰~२.३.५०, २.३.५२, २.३.६७) इत्यस्य तात्पर्यमिदम्।} इत्यनेन षष्ठी सम्बन्ध\-विवक्षायाम्। यद्वा \textcolor{red}{दृष्ट्या} इत्यध्याहार्यम्। \textcolor{red}{चक्षुष्मतां दृष्ट्या न दृश्यते} इत्यवयवावयवि\-भावे सा।\end{sloppypar}
\section[राघवे]{राघवे}
\label{sec:raaghave_3_5_36}
\centering\textcolor{blue}{लक्ष्मणोऽपि गुहामध्यात्सीतामादाय राघवे।\nopagebreak\\
समर्प्य राक्षसान्दृष्ट्वा हतान्विस्मयमाययौ॥}\nopagebreak\\
\raggedleft{–~अ॰रा॰~३.५.३६}\\
\begin{sloppypar}\hyphenrules{nohyphenation}\justifying\noindent\hspace{10mm} खर\-दूषण\-वधानन्तरं लक्ष्मणो गुहातो निर्गत्य सीतां राघवाय समर्पयत्। तत्र \textcolor{red}{राघवे} इति सप्तमी\-प्रयोगस्तु शङ्कावहः। अत्र सम्प्रदानत्वस्यैव प्राप्तत्वात्। अत्रोच्यते। यत्र स्वकीयं वस्तु कस्मैचिद्दीयते तत्रैव सम्प्रदानम्।\footnote{\textcolor{red}{दानं चापुनर्ग्रहणाय स्व\-स्वत्व\-निवृत्तिपूर्वकं पर\-स्वत्वोत्पादनम्‌। अत एव “रजकस्य वस्त्रं ददाति” इत्यादौ न भवति} (त॰बो॰~५६९)।} अत्र तथाविध\-प्रसङ्गो नास्ति। यतो भगवाञ्छ्रीरामः सीतायाः प्राणाधारः। लौकिक\-व्यवहारेऽपि सीताया लता\-स्थानित्वाच्छ्रीरामस्य च तरु\-स्थानित्वात्। इमं पक्षमेव स्पष्टयितुमेकं राम\-कथा\-सम्बन्ध\-दृष्टान्तं प्रस्तूयते। चित्रकूटे निवासं कुर्वन् रममाणो रामचन्द्रः सीतया लक्ष्मणेन सह विनोद\-भङ्ग्या सीता\-त्याग\-गौरवं व्यञ्जनया वर्णयल्लँतामेकां सङ्केत्य समवोचत्प्रणतां वनितां यत् \textcolor{red}{प्रिये। इयं लता सम्मान\-योग्यतामर्हति या खलु नीरस\-तरुमपि विभूषयति सौन्दर्य\-लक्ष्म्या स्वकीय\-समालिङ्गन\-द्वारा}। अर्थाल्लतास्थानापन्ना त्वं तरु\-स्थानीयस्य मे श्रियं वर्धयसि। तत्क्षणं सीतोवाच \textcolor{red}{देव। नेत्थम्। लताश्रयस्तरुरेव धन्यो यः खलु निराधारां लतां सबहुमानं शाखा\-बाहुभिरालम्बते}। अर्थाल्लतास्थानीयाया म आधार\-भूतस्तरु\-स्थानीयो भवान्। उभयोर्वार्तां निशम्य निरीक्ष्य पावन\-प्रेम\-मण्डित\-परस्पर\-प्रशंसा\-निःस्पृहां सुमित्रा\-नन्दनो लक्ष्मणः प्राह यत् \textcolor{red}{प्रभो लता तरुर्वा नोभौ धन्यौ लक्ष्मण\-रूपः पथिक एव धन्यो यः सीता\-लता\-श्रीराम\-कल्प\-वृक्ष\-स्निग्ध\-छायामाश्रित्य समेधते}। अत्रत्य\-भाव\-सङ्ग्रह\-रूपमेकं शार्दूल\-विक्रीडितं प्रस्तुवन् धृष्टतामाधरामि यत्~–\end{sloppypar}
\centering\textcolor{red}{रामः प्राह लता प्रिये बहुमता सम्भूषयन्ती तरुं\nopagebreak\\
सीतोवाच लताश्रयस्तरुरयं धन्यो न चैषा लता।\nopagebreak\\
सौमित्रिर्निजगाद नाथ विटपो धन्यो न वल्ली तथा\nopagebreak\\
श्लाघ्योऽयं पथिको नितान्तमुभयोश्छायां श्रयन्मोदते॥}\\
\begin{sloppypar}\hyphenrules{nohyphenation}\justifying\noindent इत्थमनेन सीता\-राम\-लक्ष्मण\-वार्तालाप\-दृष्टान्त\-प्रस्तावेन सीताया आधारो रामः। अतः \textcolor{red}{आधारोऽधिकरणम्‌} (पा॰सू॰~१.४.४५) इत्यनेनाधिकरण\-सञ्ज्ञा। आधारश्च त्रिधौपश्लेषिको वैषयिकोऽभिव्यापकश्च। औपश्लेषिको नाम सामीप्य\-संयोगान्तरः सम्बन्धो यथा \textcolor{red}{कटे आस्ते} \textcolor{red}{गुरौ वसति}।\footnote{गुरोः समीपे वसतीत्यर्थ औपश्लेषिक आधारः।} वैषयिकः संयोग\-समवाय\-सम्बन्ध\-भिन्नो यथा \textcolor{red}{मोक्षे इच्छा}। अभिव्यापकः सर्वावयव\-व्यापको यथा \textcolor{red}{तिलेषु तैलम्‌} \textcolor{red}{सर्वस्मिन् आत्मा} \textcolor{red}{दधिनि सर्पिः} इत्यादि। इमे त्रयोऽप्याधार\-गुणाः श्रीरामेऽत्र। सीता\-विषयकमाधार\-त्रितयत्वं श्रीराम एव सङ्घटते। यथा \textcolor{red}{कटे आस्ते} इतिवत्सीता रामोपश्लिष्टा। सामीप्येनापि संयोगेनापि निरन्तरं सा भगवतोऽत्यन्त\-निकटा। एवमाह्लादिनी\-शक्ति\-रूपेण कृपा\-शक्ति\-रूपेण वा संयुक्ता। भावुकानां मते तु स्वयं मैथिली पीताम्बर\-रूपेण भगवतः श्रीविग्रहं निरन्तरमुपश्लिष्यति। सामीप्यञ्चात्राव्यवहितमेव यथा \textcolor{red}{इको यणचि} (पा॰सू॰~६.१.७७) इत्यत्र \textcolor{red}{अचि} औपश्लेषिक आधारः संसारे संसाधितः \textcolor{red}{संहितायाम्‌} (पा॰सू॰~६.१.७२) इति सूत्रे भगवता भाष्य\-कारेण।\footnote{\textcolor{red}{अयं योगः शक्योऽवक्तुम्। कथम्। अधिकरणं नाम त्रिःप्रकारम्। व्यापकम् औपश्लेषिकम् वैषयिकम् इति। शब्दस्य च शब्देन कोऽन्योऽभिसम्बन्धो भवितुमर्हत्यन्यदत उपश्लेषात्। इको यणचि। अच्युपश्लिष्टस्येति। तत्रान्तरेण संहिताग्रहणं संहितायामेव भविष्यति} (भा॰पा॰सू॰~६.१.७२)।} तथैव भगवती सीता शब्दार्थयोरिवाव्यवहित\-सामीप्यमञ्चति। वैषयिकश्चाप्याधारो राम एव। \textcolor{red}{आकाशे पक्षी} इतिवत्सीता रामे। अत्र 
संयोग\-समवायावान्तरेण निरूप्य\-निरूपक\-सम्बन्धतया श्रीसीताया वैषयिक आधारः श्रीराम एव। \textcolor{red}{मोक्षे इच्छा} इतिवत्। अभिव्यापकश्चाधारः श्रीराम एव कार्त्स्न्येन। \textcolor{red}{सर्वस्मिन् आत्मा} इत्यादिवत्।
सीता श्रीरामस्य प्रतिरामं रमते। एषु त्रिष्वप्याधारेषूपश्लेष एव मुख्यो भेद\-कल्पनया त्रयः। तथा कारक\-प्रकरणे भर्तृहरिः प्राह~–\end{sloppypar}
\centering\textcolor{red}{उपश्लेषस्य चाभेदस्तिलाकाशकटादिषु।}\nopagebreak\\
\raggedleft{–~वा॰प॰~३.७.१४९}\\
\begin{sloppypar}\hyphenrules{nohyphenation}\justifying\noindent हेलाराजोऽपि समर्थयति यद्वैषयिकाभि\-व्यापकयोरप्युप\-श्लेषस्यानुस्यूतत्वात्।\footnote{\textcolor{red}{उपश्लेष आधारस्याधेयेन सम्बन्धः। यद्वशावसावाधारः। तस्य त्रिष्वप्यधिकरणेष्वभेदः} (वा॰प॰ हे॰टी॰~३.७.१४९)।} तथैव सीतोपश्लेषस्य पक्षत्रयेऽपि सम्भवः। \textcolor{red}{अनन्या राघवेणाहं भास्करेण यथा प्रभा} (वा॰रा॰~५.२१.१५) इति प्रभा\-भास्करयोः समवाय\-सम्बन्धो यथा तथैव सीता\-रामयोः किन्तु \textcolor{red}{सीताऽप्यनुगता रामं शशिनं रोहिणी यथा} (वा॰रा॰~१.१.२८) इति वचनेनोपश्लेष एव सार्वत्रिकः। अतो \textcolor{red}{राघवे समर्प्य} इत्यत्र न्यास\-भूतां सीतां लक्ष्मणः पुना राघवमाधारं मत्वा समर्पयतीत्यर्थ\-बुबोधयिषया सप्तमीऽयं मधुर\-भाव\-बोधन\-पुरः\-सरं सर्वतो\-भावेन पाणिनीया।\end{sloppypar}
\section[निशायाम्]{निशायाम्‌}
\centering\textcolor{blue}{विचिन्त्यैवं निशायां स प्रभाते रथमास्थितः।\nopagebreak\\
रावणो मनसा कार्यमेकं निश्चित्य बुद्धिमान्॥}\nopagebreak\\
\raggedleft{–~अ॰रा॰~३.६.१}\\
\begin{sloppypar}\hyphenrules{nohyphenation}\justifying\noindent\hspace{10mm} \textcolor{red}{निशायाम्‌} इत्यत्राजन्त\-प्रयोगोऽपाणिनीय इव। प्रायशोऽयं हलन्त\-प्रयोगः। \textcolor{red}{निशायाम्‌} इति कथम्। यतो हि \textcolor{red}{अजाद्यतष्टाप्‌} (पा॰सू॰~४.१.४) इति सूत्रं ह्यजादीनामकारान्तस्य च टाप्प्रत्ययं करोत्ययं नाजादिर्न वाऽकारान्त इति चेत्। भागुरिर्हलन्त\-स्त्रीलिङ्गे पठितानां \textcolor{red}{टाप्‌}\-प्रत्ययं वदति। तथा च कारिका~–\end{sloppypar}
\centering\textcolor{red}{वष्टि भागुरिरल्लोपमवाप्योरुपसर्गयोः।\nopagebreak\\
आपं चैव हलन्तानां यथा वाचा दिशा निशा॥}\nopagebreak\\
\raggedleft{–~वै॰सि॰कौ॰ अव्ययप्रकरणान्ते कारिका २}\\
\begin{sloppypar}\hyphenrules{nohyphenation}\justifying\noindent न च भागुरि\-मतमपि त्वपाणिनीयमिति वाच्यम्। निशा\-शब्दस्य सूत्र उच्चारणात्पाणिनिनाऽपि भागुरि\-मतस्य कृत\-समादर\-दर्शनात्। यथा \textcolor{red}{विभाषा सेना\-सुराच्छाया\-शाला\-निशानाम्‌} (पा॰सू॰~२.४.२५) इत्यत्र \textcolor{red}{निशानाम्‌} षष्ठ्यन्त\-प्रयोगेणायं पाणिनीय एव। एवमेव \textcolor{red}{दिवा\-विभा\-निशा\-प्रभा\-भास्करान्तानन्तादि\-बहु\-नान्दीकिंलिपि\-लिबि\-बलि\-भक्ति\-कर्तृ\-चित्र\-क्षेत्र\-सङ्ख्या\-जङ्घा\-बाह्वहर्यत्तद्धनुररुष्षु} (पा॰सू॰~३.२.२१) \textcolor{red}{निशा\-प्रदोषाभ्यां च} (पा॰सू॰~४.३.१४) इत्यनयोरपि।\end{sloppypar}
\section[ब्रूहि मे]{ब्रूहि मे}
\centering\textcolor{blue}{ब्रूहि मे न हि गोप्यं चेत्करवाणि तव प्रियम्।\nopagebreak\\
न्याय्यं चेद्ब्रूहि राजेन्द्र वृजिनं मां स्पृशेन्नहि॥}\nopagebreak\\
\raggedleft{–~अ॰रा॰~३.६.६}\\
\begin{sloppypar}\hyphenrules{nohyphenation}\justifying\noindent\hspace{10mm} \textcolor{red}{माम्‌} इति वक्तव्ये \textcolor{red}{मे} इत्युक्तम्। कर्मणि सम्बन्ध\-विवक्षया षष्ठी। यद्वा \textcolor{red}{मे पुरतः} इत्यध्याहार\-बलेन षष्ठी। यद्वा \textcolor{red}{मे मह्यम्‌}। \textcolor{red}{मां ज्ञापयितुं ब्रूहि} इति तुमुन्कर्मणि चतुर्थी।\footnote{\textcolor{red}{क्रियार्थोपपदस्य च कर्मणि स्थानिनः} (पा॰सू॰~२.३.१४) इत्यनेन।}\end{sloppypar}
\section[आश्रमादपनेष्यसि]{आश्रमादपनेष्यसि}
\centering\textcolor{blue}{त्वं तु मायामृगो भूत्वा ह्याश्रमादपनेष्यसि।\nopagebreak\\
रामं च लक्ष्मणं चैव तदा सीतां हराम्यहम्॥}\nopagebreak\\
\raggedleft{–~अ॰रा॰~३.६.१३}\\
\begin{sloppypar}\hyphenrules{nohyphenation}\justifying\noindent\hspace{10mm} अत्र रावणो मारीचं प्रत्युपायं निर्दिशति। \textcolor{red}{अकथितं च} (पा॰सू॰~१.४.५१) इत्यनेनात्र कर्म\-सञ्ज्ञा। अकथित\-कर्मणां धातूनां क्रमे \textcolor{red}{नी}\-धातोरपि (\textcolor{red}{णीञ् प्रापणे} धा॰पा॰~९०१) पठितत्वात्।\footnote{\textcolor{red}{दुह्याच्पच्दण्ड्रुधिप्रच्छिचिब्रूशासुजिमथ्मुषाम्। कर्मयुक्स्यादकथितं तथा स्यान्नीहृकृष्वहाम्॥} (वै॰सि॰कौ॰~५३९)।} किन्त्वपादानादिभिरिविवक्षिते सतीयं व्यवस्था। अत्र त्वपादानस्य विवक्षाऽऽवश्यकी विश्लेषावधि\-भूतस्याश्रमस्य ध्रुवत्वं स्पष्टयितुम्।\end{sloppypar}
\section[देवायम्]{देवायम्‌}
\centering\textcolor{blue}{लक्ष्मणो राममाहेदं देवायं मृगरूपधृक्।\nopagebreak\\
मारीचोऽत्र न सन्देह एवंभूतो मृगः कुतः॥}\nopagebreak\\
\raggedleft{–~अ॰रा॰~३.७.९}\\
\begin{sloppypar}\hyphenrules{nohyphenation}\justifying\noindent\hspace{10mm} अत्र कपट\-कुरङ्ग\-वेष\-धारिणं मारीचमभिगन्तुकामं श्रीरामं लक्ष्मणो निर्दिशति \textcolor{red}{देव। अयं मृग\-रूप\-धृङ्मारीचः}। अत्र \textcolor{red}{देव} इति सम्बोधनम्। अत्र च \textcolor{red}{गुरोरनृतोऽनन्त्यस्याप्येकैकस्य प्राचाम्‌} (पा॰सू॰~८.२.८६) इत्यनेन प्लुतः। ततश्च \textcolor{red}{प्लुत\-प्रगृह्या अचि नित्यम्‌} (पा॰सू॰~६.१.१२५) इत्यनेन प्रकृति\-भावे कथं दीर्घ इति चेत्। \textcolor{red}{प्राचाम्‌} इति योग\-विभागेन \textcolor{red}{सर्वः प्लुतो विकल्प्यते} (वै॰सि॰कौ॰~९७) इत्यनेन प्लुतस्य वैकल्पिकत्वात्प्लुताभाव\-पक्षे दीर्घः।\footnote{\textcolor{red}{अकः सवर्णे दीर्घः} (पा॰सू॰~६.१.१०१) इत्यनेन।}\end{sloppypar}
\section[कालमेघसमद्युतिम्]{कालमेघसमद्युतिम्‌}
\centering\textcolor{blue}{स्वरूपं दर्शयामास महापर्वतसन्निभम्।\nopagebreak\\
दशास्यं विंशतिभुजं कालमेघसमद्युतिम्॥}\nopagebreak\\
\raggedleft{–~अ॰रा॰~३.७.५०}\\
\begin{sloppypar}\hyphenrules{nohyphenation}\justifying\noindent\hspace{10mm} सीता\-हरणाय यति\-वेषेणागतो
रावणः सीतां स्वरूपं दर्शयामास। अत्र \textcolor{red}{स्वरूपम्‌} इति नपुंसकलिङ्गं तस्य विशेषणं \textcolor{red}{काल\-मेघ\-सम\-द्युतिम्‌} इति नपुंसक\-लिङ्गस्य विशेषणं कथं यतो ह्यत्र \textcolor{red}{काल\-मेघेन समाना द्युतिर्यस्य तत्‌} इति विग्रहेऽन्यपदार्थे बहुव्रीहिः। अन्य\-पदार्थो हि \textcolor{red}{स्वरूपम्‌} इति। तस्य नपुंसक\-लिङ्गता सर्व\-विदिताऽतस्तद्विशेषणेनाप्यनिवार्या सा। व्युत्पत्ति\-वादेऽपि समान\-लिङ्गता समान\-वचनकत्वञ्च बहुशः प्रतिपादितम्। कथयन्ति च बुधाः~–\end{sloppypar}
\centering\textcolor{red}{या विशेष्येषु दृश्यन्ते लिङ्गसङ्ख्याविभक्तयः।\nopagebreak\\
प्रायस्ता एव कर्तव्याः समानार्थे विशेषणे॥}\footnote{मूलं मृग्यम्।}\\
\begin{sloppypar}\hyphenrules{nohyphenation}\justifying\noindent अतः सति नपुंसक\-लिङ्गे \textcolor{red}{कालमेघ\-समद्युतिम्‌} इत्यपाणिनीयतामावहतीव। यतो हि नपुंसक\-लिङ्गेऽमि विभक्तौ \textcolor{red}{स्वमोर्नपुंसकात्‌} (पा॰सू॰~७.१.२३) इति लुग्विधानात् \textcolor{red}{काल\-मेघ\-सम\-द्युति} इत्येव पाणिनीयम्। परं विमर्शे कृते विरोधः परिहर्तुं शक्यते। \textcolor{red}{अर्धर्चाः पुंसि च} (पा॰सू॰~२.४.३१) इति सूत्रेऽर्धर्चादि\-गणस्य पुल्लिँङ्गे नपुंसक\-लिङ्गे च विधानादाकृति\-गणत्वात् \textcolor{red}{स्वरूप}\-शब्दमर्धर्चादि\-गणे पठित्वा तत्र पुल्लिँङ्गता\-स्वीकारे तद्विशेषणेऽपि। तेन \textcolor{red}{हरिम्‌} इत्यादिवदत्रापि \textcolor{red}{अमि पूर्वः} (पा॰सू॰~६.१.१०७) इत्यनेन पूर्व\-रूपम्। यद्वा स्वरूपं वर्णयित्वा द्युतिं पृथग्वर्णयति। एवं चात्र कर्मधारयः \textcolor{red}{काल\-मेघ\-समा चासौ द्युतिश्च तां काल\-मेघ\-सम\-द्युतिम्‌} इति \textcolor{red}{मतिम्‌} इत्यादिवत्।\end{sloppypar}
\section[त्वरा]{त्वरा}
\centering\textcolor{blue}{त्वद्वाक्यसदृशं श्रुत्वा मां गच्छेति त्वराऽब्रवीत्।\nopagebreak\\
रुदन्ती सा मया प्रोक्ता देवि राक्षसभाषितम्।\nopagebreak\\
नेदं रामस्य वचनं स्वस्था भव शुचिस्मिते॥}\nopagebreak\\
\raggedleft{–~अ॰रा॰~३.८.११}\\
\begin{sloppypar}\hyphenrules{nohyphenation}\justifying\noindent\hspace{10mm} अत्र \textcolor{red}{त्वरया अब्रवीत्‌} इति वक्तव्ये \textcolor{red}{त्वरा अब्रवीत्‌} इति प्रयुक्तम्। यतो हि हलन्त\-स्त्री\-लिङ्गानां वैकल्पिक\-टाब्विधाने पाक्षिक\-हलन्तत्वे तृतीयायां \textcolor{red}{त्वर् टा} इति स्थिते टकारानुबन्ध\-लोपे \textcolor{red}{त्वरा} इति \textcolor{red}{गिरा} इव।\footnote{\textcolor{red}{ञित्वराँ सम्भ्रमे} (धा॰पा॰~७७५) इति धातोः \textcolor{red}{सम्पदादिभ्‍यः क्विप्} (वा॰~३.३.१०८) इत्यनेन स्त्रियां भावे क्विप्। तस्मात्तृतीयायां विभक्तौ \textcolor{red}{टा}\-प्रत्यये \textcolor{red}{त्वरा} इति भावः।} यद्वा \textcolor{red}{सह सुपा} (पा॰सू॰~२.१.४) इत्यनेन सुबन्तस्य धातुना सह समासे \textcolor{red}{त्वराब्रवीत्‌} इति।\footnote{अस्मिन् पक्षे \textcolor{red}{त्वर्‌}\-धातोः \textcolor{red}{घटादयः षितः} (धा॰पा॰ ग॰सू॰) इत्यनेन षित्त्वात् \textcolor{red}{षिद्भिदादिभ्योऽङ्} (पा॰सू॰~३.३.१०४) इत्यनेनाङि \textcolor{red}{अजाद्यतष्टाप्‌} (पा॰सू॰~४.१.४) इत्यनेन टापि \textcolor{red}{त्वरा} प्रातिपदिकम्। \textcolor{red}{सम्भ्रमस्त्वरा} (अ॰को॰~३.२.२६) इत्यमरः। तस्मात्तृतीयायां विभक्तौ~– त्वरा~टा~\arrow त्वरा~आ~\arrow \textcolor{red}{आङि चापः} (पा॰सू॰~७.३.१०५)~\arrow तवरे~आ~\arrow \textcolor{red}{एचोऽयवायावः} (पा॰सू॰~६.१.७८)~\arrow त्वरय्~आ~\arrow त्वरया। ततः \textcolor{red}{त्वरया अब्रवीत्‌} इति स्थिते सुपस्तिङा समासे \textcolor{red}{सुपो धातु\-प्रातिपदिकयोः} (पा॰सू॰~२.४.७१) इत्यनेन विभक्तेर्लुकि \textcolor{red}{त्वरा अब्रवीत्‌} इति जाते \textcolor{red}{अकः सवर्णे दीर्घः} (पा॰सू॰~६.१.१०१) इत्यनेन सवर्ण\-दीर्घे \textcolor{red}{त्वराऽब्रवीत्‌} इति।} समासस्य षड्विधत्वं प्रसिद्धमेव। यथा सिद्धान्त\-कौमुद्याम्~– \textcolor{red}{किञ्च}\end{sloppypar}
\centering\textcolor{red}{सुपां सुपा तिङा नाम्ना धातुनाऽथ तिङां तिङा।\nopagebreak\\
सुबन्तेनेति विज्ञेयः समासः षड्विधो बुधैः॥}\\
\begin{sloppypar}\hyphenrules{nohyphenation}\justifying\noindent \textcolor{red}{सुपां सुपा राजपुरुषः। तिङा पर्यभूषत्। नाम्ना कुम्भकारः। धातुना कटप्रूः। अजस्रम्। तिङां तिङा पिबतखादता। खादतमोदता। तिङां सुपा कृन्त विचक्षणेति यस्यां क्रियायां सा कृन्त\-विचक्षणा।} (वै॰सि॰कौ॰ सर्वसमासशेष\-प्रकरणे)। तस्मात्तृतीयान्त\-\textcolor{red}{त्वरया}\-इति\-शब्देन समासो विभक्ति\-लोपश्च।
\end{sloppypar}
\section[जायेति सीतेति]{जायेति सीतेति}
\label{sec:jaayeti_siiteti}
\centering\textcolor{blue}{निर्ममो निरहङ्कारोऽप्यखण्डानन्दरूपवान्।\nopagebreak\\
मम जायेति सीतेति विललापातिदुःखितः॥}\nopagebreak\\
\raggedleft{–~अ॰रा॰~३.८.२०}\\
\begin{sloppypar}\hyphenrules{nohyphenation}\justifying\noindent\hspace{10mm} अत्र सीता\-हरण\-सञ्जात\-शोकः श्रीरामोऽखण्डानन्द\-रूपवान् सन् \textcolor{red}{हे मम जाये इति हे सीते इति} व्यलपत्। अत्र \textcolor{red}{जायेति सीतेति} च प्रयोग\-द्वयं विभाव्यम्। \textcolor{red}{जाये सीते} च द्वावपि शब्दौ सम्बोधनान्तौ। एवं प्रातिपदिकात्सौ विभक्तौ सम्बोधनतया \textcolor{red}{संबुद्धौ च} (पा॰सू॰~७.३.१०६) इत्येकार्ये \textcolor{red}{एङ्ह्रस्वात्सम्बुद्धेः} (पा॰सू॰~६.१.६९) इत्यनेन सुलोपे \textcolor{red}{जाये सीते} इति सिध्यतः। तत \textcolor{red}{इति}\-घटकेकारेण सह सन्धावयादेशे\footnote{\textcolor{red}{एचोऽयवायावः} (पा॰सू॰~६.१.७८) इत्यनेन।} यलोपे\footnote{\textcolor{red}{लोपः शाकल्यस्य} (पा॰सू॰~८.३.१९) इत्यनेन।} \textcolor{red}{जाय इति सीत इति} च। \textcolor{red}{जायेति सीतेति} अपाणिनीयमिव। उच्यते। अत्र च \textcolor{red}{इति}\-शब्दार्थकः \textcolor{red}{ति}\-शब्दः।\footnote{\setcounter{dummy}{\value{footnote}}\addtocounter{dummy}{-1}\refstepcounter{dummy}\label{fn:ti}\textcolor{red}{ति}\-शब्स्यार्थो वाचस्पत्ये~– \textcolor{red}{इति + वेदे पृषो॰। इतिशब्दार्थे। “सहोवाचास्तीह प्रायश्चित्तिरित्यस्तीति का ति पिता ते वेदेति शत॰~ब्रा॰~११.६.१.३। का प्रायश्चित्तिस्ति इति प्रश्नः}। शतपथब्राह्मणे त्रिषु मन्त्रेषु \textcolor{red}{ति}\-शब्द \textcolor{red}{इति}\-अर्थे प्रयुक्तः – \textcolor{red}{का ति पिता ते वेदेति} (श॰ब्रा॰~११.६.१.३) \textcolor{red}{का ति पितैव ते वेदेति} (श॰ब्रा॰~११.६.१.४) \textcolor{red}{का ति पितैव ते वेदेति} (श॰ब्रा॰~११.६.१.५)।} \textcolor{red}{जाये ति सीते ति}। यथा \textcolor{red}{ईदूदेद्द्विवचनं प्रगृह्यम्‌} (पा॰सू॰~१.१.११) इत्यनेनेदन्तोदन्तैदन्तानां द्विवचनानां प्रगृह्य\-सञ्ज्ञा क्रियते। सत्यां च तस्यां प्रकृति\-भावो भवति। \textcolor{red}{हरी एतौ विष्णू इमौ गङ्गे अमू} (वै॰सि॰कौ॰~१००) इत्यादिवत्। तर्हि \textcolor{red}{मणीवोष्ट्रस्य लम्बेते प्रियौ वत्सतरौ मम} (म॰भा॰~१२.१७६.११) इति महाभारत\-श्लोक\-वाक्ये \textcolor{red}{मणी इव उष्ट्रस्य} इति स्थितेऽत्रापीदन्त\-द्विवचनतया प्रगृह्यत्वात्प्रकृति\-भावे कथं \textcolor{red}{मणीव} इत्यत्र दीर्घः। अतः कौमुद्यामाक्षिप्य समाहितं \textcolor{red}{‘मणी वोष्ट्रस्य लम्बेते प्रियौ वत्सतरौ मम’ इत्यत्र त्विवार्थे वशब्दो वाशब्दो वा बोद्ध्यः} (वै॰सि॰कौ॰~१००)। तत्रत्या तत्त्वबोधिनी च – \textcolor{red}{वशब्द इत्यादि। ‘वं प्रचेतसि जानीयादिवार्थे च तदव्ययम्’ (मे॰को॰~व॰~१) इति मेदिनी। ‘व वा यथा तथैवैवं साम्ये’ (अ॰को॰~३.४.९) इत्यमरः। ‘कादम्ब\-खण्डित\-दलानि व पङ्कजानि’ इत्यादि\-प्रयोग\-दर्शनाच्चेति भावः} (त॰बो॰~१००)। स एव पन्था अत्राप्यनुयातव्यः। \textcolor{red}{इति}\-अर्थे \textcolor{red}{ति}\-शब्दो बोद्धव्यः। एतेन \textcolor{red}{हे कृष्ण हे यादव हे सखे ति} (भ॰गी॰~११.४१) इति भगवद्गीतोक्तमप्यपाणिनीयं समाहितम्।\footnote{इदं प्रथमं समाधानम्। द्वितीयं पश्चाद्वक्ष्यन्ति।} यद्वा \textcolor{red}{जाया इति सीता इति} शुद्ध\-प्रथमान्तमेव रूप\-द्वयं ततो गुणे \textcolor{red}{जायेति सीतेति}। अथवा \textcolor{red}{न मु ने} (पा॰सू॰~८.२.३) इत्यत्र \textcolor{red}{न}\-ग्रहणमेव \textcolor{red}{पूर्वत्रासिद्धम्‌} (पा॰सू॰~८.२.१) इति सूत्रस्यासिद्ध\-करणानित्यत्वं ज्ञापयति क्वाचित्कम्। अन्यथा \textcolor{red}{अमुना} इत्येव ब्रूयात्। तत्र हि \textcolor{red}{न। मु ने} इति योग\-विभागः। \textcolor{red}{पूर्वत्रासिद्धं न} इति प्रथमोऽर्थः। \textcolor{red}{ने मु न असिद्धः} इति द्वितीयोऽर्थः। योग\-विभाग\-फलं क्वचित्क्वचित् \textcolor{red}{पूर्वत्रासिद्धम्‌} (पा॰सू॰~८.२.१) इति सूत्रस्य प्रसराभावः। तेन \textcolor{red}{न चीरमस्याः प्रविधीयतेति} (वा॰रा॰~२.३७.३४) इह \textcolor{red}{प्रविधीयते इति} इत्यवस्थायां \textcolor{red}{एचोऽयवायावः} (पा॰सू॰~६.१.७८) इत्यनेनायादेशे \textcolor{red}{प्रविधीयतय् इति} इति जाते \textcolor{red}{लोपः शाकल्यस्य} (पा॰सू॰~८.३.१९) इत्यनेन यकार\-लोपे \textcolor{red}{आद्गुणः} (पा॰सू॰~६.१.८७) इत्यनेन गुणे प्राप्ते यलोपस्य त्रिपादीत्वाद्गुण\-शास्त्रस्य सपाद\-सप्ताध्यायीत्वात् \textcolor{red}{पूर्वत्रासिद्धम्‌} (पा॰सू॰~८.२.१) इत्यनेन लोप\-कार्यस्यासिद्धत्वे गुणाभावे \textcolor{red}{प्रविधीयतेति} इति कथम्। ततो \textcolor{red}{न मु ने} इत्यनेनैव \textcolor{red}{पूर्वत्रासिद्धम्‌} (पा॰सू॰~८.२.१) इत्यस्याप्रवृत्तौ लोप\-कार्ये सिद्धे गुणे \textcolor{red}{प्रविधीयतेति}।\footnote{\textcolor{red}{प्रविधीयतेतीत्यत्र सन्धिस्तु ‘न मु ने’ इत्यत्र योगविभागेन क्वचित्त्रिपाद्या असिद्धत्वाभावज्ञापनादार्षत्वाद्वा} (वा॰रा॰ ति॰टी॰~२.३७.३४)।} एवमेव \textcolor{red}{हे कृष्ण हे यादव हे सखेति} (भ॰गी॰~११.४१) इत्यत्रापि कर्तव्ये लोपकार्ये सिद्धे गुणः।\footnote{इदं द्वितीयं समाधानम्।} एवमेव \textcolor{red}{प्रियः प्रियायार्हसि देव सोढुम्‌} (भ॰गी॰~११.४४) इत्यत्रापि \textcolor{red}{प्रियायास् अर्हसि} इति स्थिते \textcolor{red}{ससजुषो रुः} (पा॰सू॰~८.२.६६) इत्यनेन सकारस्य रुत्वे \textcolor{red}{भोभगोअघोअपूर्वस्य योऽशि} (पा॰सू॰~८.३.१७) इत्यनेन यत्वे \textcolor{red}{लोपः शाकल्यस्य} (पा॰सू॰~८.३.१९) इत्यनेन लोपे दीर्घे प्राप्ते पुनस्त्रि\-पादित्वाल्लोप\-कार्यस्यासिद्धत्वे प्राप्ते दीर्घे च सङ्कटापन्ने \textcolor{red}{न मु ने} (पा॰सू॰~८.२.३) इत्यनेन \textcolor{red}{पूर्वत्रासिद्धम्‌} (पा॰सू॰~८.२.१) इति सूत्रे निराकृते लोप\-कार्ये सिद्धे दीर्घे \textcolor{red}{प्रियायार्हसि} इति साधु। अनया मीमांसया \textcolor{red}{प्रियाय अर्हसि} इति चतुर्थ्यन्तं श्रीरामानुज\-कथनमपास्तम्।\footnote{\textcolor{red}{यस्मात्त्वं सर्वस्य पिता पूज्यतमो गुरुश्च कारुण्यादिगुणैश्च सर्वाधिकोऽसि तस्मात्त्वामीशमीड्यं प्रणम्य प्रणिधाय च कायं प्रसादये। यथा कृतापराधस्य अपि पुत्रस्य यथा च सख्युः प्रणामपूर्वकं प्रार्थितः पिता सखा वा प्रसीदति तथा त्वं परमकारुणिकः प्रियः प्रियाय मे सर्वं सोढुमर्हसि} (भ॰गी॰ रा॰भा॰~११.४४)।} एवमेव \textcolor{red}{मम जाये इति सीते इति} इत्युभयत्राप्येकारस्यायादेशे यकारस्य शाकल\-लोपे\footnote{शाकलश्चासौ लोपश्चेति शाकललोपः। शकलस्य गोत्रापत्यं पुमान् शाकल्यः। गर्गादित्वात् \textcolor{red}{यञ्‌}\-प्रत्ययः। शाकल्यस्यायं शाकलः। \textcolor{red}{लोपः शाकल्यस्य} (पा॰सू॰~८.३.१९) इत्यनेन विहितो लोपः शाकलो लोपः शाकललोपो वा। शकल~\arrow \textcolor{red}{गर्गादिभ्यो यञ्‌} (पा॰सू॰~४.१.१०५)~\arrow शकल~यञ्~\arrow शकल~य~\arrow \textcolor{red}{यचि भम्‌} (पा॰सू॰~१.४.१८)~\arrow भसञ्ज्ञा~\arrow \textcolor{red}{तद्धितेष्वचामादेः} (पा॰सू॰~७.२.११७)~\arrow शाकल~य~\arrow\textcolor{red}{यस्येति च} (पा॰सू॰~६.४.१४८)~\arrow शाकल्~य~\arrow शाकल्य~\arrow \textcolor{red}{तस्येदम्‌} (पा॰सू॰~४.३.१२०)~\arrow शाकल्य~अण्~\arrow शाकल्य~अ~\arrow \textcolor{red}{यचि भम्‌} (पा॰सू॰~१.४.१८)~\arrow भसञ्ज्ञा~\arrow \textcolor{red}{यस्येति च} (पा॰सू॰~६.४.१४८)~\arrow शाकल्य्~अ~\arrow \textcolor{red}{आपत्यस्य च तद्धितेऽनाति} (पा॰सू॰~६.४.१५१)~\arrow शाकल्~अ~\arrow शाकल~\arrow विभक्तिकार्यम्~\arrow शाकलः। ष्फाणौ प्रत्ययौ \textcolor{red}{शाकल्य}\-शब्दाद्भवत इति महाभाष्ये स्पष्टम्। यथा~– \textcolor{red}{कण्वात्तु शकलः पूर्वः कतादुत्तर इष्यते। पूर्वोत्तरौ तदन्तादी ष्फाणौ तत्र प्रयोजनम्॥} (भा॰पा॰सू॰~४.१.१८)। तत्रैव \textcolor{red}{शाकल्यायनी} शाकल्यस्य च्छात्राः \textcolor{red}{शाकलाः} इत्युदाहृतौ।} गुणे प्राप्ते \textcolor{red}{पूर्वत्रासिद्धम्‌} (पा॰सू॰~८.२.१) इति सूत्रेण यकार\-लोपस्यासिद्धत्वे प्राप्ते \textcolor{red}{न मु ने} (पा॰सू॰~८.२.३) इत्यनेन सूत्रेऽस्मिन्निराकृते गुण उभावपि प्रयोगौ निरवद्यौ। \textcolor{red}{न मु ने} इत्यत्र नकारयोगविभागं समर्थयन्ते वैयाकरण\-सिद्धान्त\-कौमुदी\-तत्त्व\-बोधिनी\-टीकाकारा ज्ञानेन्द्र\-सरस्वती\-महाभागाः। तथा च तत्रत्या तत्त्वबोधिनी~– \textcolor{red}{ननु “अधुना” इतिवत् “अमुना” इत्योवोच्यतां किमनेनासिद्धत्वनिषेधेनेति चेत्। अत्राहुः। “न मु ने” (पा॰सू॰~८.२.३) इत्युक्तिः “न” इति योगविभागार्था। तेन रामः रामेभ्य इत्यादि सिध्यति। अन्यथा हि रोरसिद्ध\-तयोकारस्येत्सञ्ज्ञा\-लोपौ कथं स्याताम्। न चानुनासिक\-निर्देश\-सामर्थ्यादित्सञ्ज्ञा\-लोपौ प्रति रुत्वं नासिद्धमिति वाच्यम्। तरुमूलं देवरुहीत्यादौ “हशि च” (पा॰सू॰~६.१.११४) इत्यस्य व्यावृत्तये “अतो रोरप्लुतादप्लुते” (पा॰सू॰~६.१.११३) इत्यत्रानु\-नासिकस्यैव निर्देशेन तत्रैव चरितार्थत्वात्। एवं च स्थानिवत्सूत्रस्यापि प्रवृतौ पदत्वाद्विसर्गो लभ्यते। “प्रत्ययः” (पा॰सू॰~३.१.१) “परश्च” (पा॰सू॰~३.१.२) इत्यादि\-निर्देशाश्चेह लिङ्गमिति दिक्} (त॰बो॰~४३९)।\end{sloppypar}
\section[जटायो]{जटायो}
\centering\textcolor{blue}{जटायो ब्रूहि मे भार्या केन नीता शुभानना।\nopagebreak\\
मत्कार्यार्थं हतोऽसि त्वमतो मे प्रियबान्धवः॥}\nopagebreak\\
\raggedleft{–~अ॰रा॰~३.८.३१}\\
\centering\textcolor{blue}{इत्युक्त्वा राघवः प्राह जटायो गच्छ मत्पदम्।\nopagebreak\\
मत्सारूप्यं भजस्वाद्य सर्वलोकस्य पश्यतः॥}\nopagebreak\\
\raggedleft{–~अ॰रा॰~३.८.४०}\\
\begin{sloppypar}\hyphenrules{nohyphenation}\justifying\noindent\hspace{10mm} \textcolor{red}{जटायामायुर्यस्य} इति विग्रहे व्यधिकरण\-बहुव्रीहौ विभक्ति\-कार्ये सम्बोधने \textcolor{red}{हे जटायुः} इति चेत्।\footnote{\textcolor{red}{जटायुस्} प्राति\-पदिकात्सम्बुद्धौ सौ विभक्तौ जटायुस्~सुँ~\arrow जटायुस्~स्~\arrow \textcolor{red}{हल्ङ्याब्भ्यो दीर्घात्सुतिस्यपृक्तं हल्} (पा॰सू॰~६.१.६८)~\arrow जटायुस्~\arrow \textcolor{red}{ससजुषो रुः} (पा॰सू॰~८.२.६६)~\arrow जटायुरुँ~\arrow जटायुर्~\arrow \textcolor{red}{खरवसानयोर्विसर्जनीयः} (पा॰सू॰~८.३.१५)~\arrow जटायुः।} \textcolor{red}{आयुस्‌} \textcolor{red}{आयु} शब्दश्च द्वावप्यायुष्य\-वाचकौ।\footnote{उकारान्त\-\textcolor{red}{आयु}\-शब्दो यथा~– \textcolor{red}{यथा श्रुतं मया पूर्वं वायुना जगदायुना} (वायु॰पु॰~१.५४.२) \textcolor{red}{स्थाणोः पश्चिमदिग्भागे वायुना जगदायुना} (वाम॰पु॰ स॰मा॰~२५.३९)। तत्त्व\-बोधिन्यामुणादि\-प्रकरणे \textcolor{red}{छन्दसीणः} (प॰उ॰~१.२) इति सूत्रे ज्ञानेन्द्र\-सरस्वती\-महाभागास्तु~– \textcolor{red}{उणनुवर्तते। एति गच्छतीत्यायुः। “मा न आयौ” इति। आयुशब्दो मनुष्यपर्यायेषु वैदिकनिघण्टौ पठितः। अत एव “त्वाम॑ग्ने प्रथ॒ममा॒युमा॒यवे॑” (ऋ॰वे॰सं॰~१.३१.११) “मा न॑स्तो॒के तन॑ये॒ मा न॑ आ॒यौ” (ऋ॰वे॰सं॰~१.११४.८) इत्यादिमन्त्रेषु वेदभाष्ये तथैव व्याख्यातम्। अर्वाचीनास्तु “छन्दसीणः” इति सूत्रं बहुल\-वचनाद्भाषायामपि प्रवर्तत इति स्वीकृत्य “आयुर्जीवितकालो ना” (अ॰को॰~२.८.१२०) इत्यमरग्रन्थे ‘आयु’\-शब्दमुकारान्तं व्याचख्युः। ननु “एतेर्णिच्च” (प॰उ॰~२.११८) इत्युस्प्रत्यये सकारान्तो वक्ष्यमाण ‘आयुः’\-शब्दस्तु लोक\-वेदयोर्नर्विवाद एव। अत एव ‘जटा आयुरस्य’ इति विग्रहे “गृध्रं हत्वा जटायुषम्” (वा॰रा॰~१.१.५२) इति रामायणप्रयोगः “यदि त्रिलोकी गणनापरा स्यात्तस्याः समाप्तिर्यदि नायुषः स्यात्” (नै॰च॰~३.४०) इति श्रीहर्ष\-प्रयोगश्च सङ्गच्छते। तथा च “आयुर्जीवितकालो ना” (अ॰को॰~२.८.१२०) इत्यत्रायुःशब्दः सकारान्त इत्येव व्याख्यायतां किमुकारान्ताभ्युप\-गमेनेति चेत्। अत्राहुः~– सकारान्त आयुःशब्दो नपुसंक इति तस्य पुँल्लिङ्गता नेत्याशयेन तथोक्तमिति। अन्ये तु “छन्दसीणः” (प॰उ॰~१.२) इति सूत्रस्य भाषायां प्रवृत्त्यभावे “मा वधीष्ट जटायुं माम्” (भ॰का॰~६.४१) इति भट्टिप्रयोगः “तटीं विन्ध्यस्याऽद्रेरभजत जटायोः प्रथमजः” इति विन्ध्यवर्णने अभिनन्दोक्त\-प्रयोगश्च न सङ्गच्छेतेत्याहुः। वस्तुतस्तु “जटां याति प्राप्नोतीति जटायुः”। मृगय्वादित्वात्कुः। आयातीत्यायुः। एवं च “जटायुषा जटायुं च विद्यादायुं तथायुषा” इति द्विरूपकोशः “वायुना जगदायुना” इति वर्णविवेकश्च सुसाध इति दिक्}। ऋग्वेद\-संहितायाम् \textcolor{red}{अग्ने॒ जर॑स्व स्वप॒त्य आयु॑न्यू॒र्जा पि॑न्वस्व॒ समिषो॑ दिदीहि नः} (ऋ॰वे॰सं॰~३.३.७) \textcolor{red}{व॒त्सं न पूर्व॒ आयु॑नि जा॒तं रि॑हन्ति मा॒तरः॑} (ऋ॰वे॰सं॰~सा॰भा॰~९.१००.१) इति मन्त्रयोरुकारान्तः \textcolor{red}{आयु}\-शब्द इति केचित्। अत्र सायणाः \textcolor{red}{आयुनि आयुषि} (ऋ॰वे॰सं॰~३.३.७) \textcolor{red}{आयुनि वयसि} (ऋ॰वे॰सं॰~सा॰भा॰~९.१००.१)।} अत्र \textcolor{red}{आयु}\-शब्द एव जटा\-शब्देन सह समस्तः। ततो प्रथमा\-विभक्तौ \textcolor{red}{एक\-वचनं सम्बुद्धिः} (पा॰सू॰~२.३.४९) इत्यनेन सम्बुद्धि\-सञ्ज्ञायां \textcolor{red}{ह्रस्वस्य गुणः} (पा॰सू॰~७.३.१०८) इत्यनेन गुणे सति \textcolor{red}{एङ्ह्रस्वात्सम्बुद्धेः} (पा॰सू॰~६.१.६९) इत्यनेन सोर्लोपे \textcolor{red}{जटायो} इति।\footnote{सकारान्त\-\textcolor{red}{जटायुस्‌}\-शब्दः सर्वविदितः। उकारान्तो \textcolor{red}{जटायु}\-शब्दोऽपि दृश्यते। यथा \textcolor{red}{जटायोस्तु वधं श्रुत्वा दुःखितः सोऽरुणात्मजः} (वा॰रा॰~५.३५.६५) इति वाल्मीकि\-रामायणे सुन्दर\-काण्डे पाठभेदः। अत्र \textcolor{red}{जटायुषो वधं श्रुत्वा} इति गोविन्द\-राज\-सम्मत\-पाठः। अरण्यकाण्डे तु \textcolor{red}{तलेनाभिजघानार्तो जटायुं क्रोधमूर्छितः} (वा॰रा॰~३.५१.३७) इत्यत्र गोविन्द\-राजा अपि~– \textcolor{red}{जटायुं जटायुरित्युकारान्तोऽप्यस्ति} (वा॰रा॰ भू॰टी॰~३.५१.३७)। अन्यच्च~– \textcolor{red}{जटायोः कीर्तनं चक्रू रामकार्ये मृतं पुरा} (आ॰रा॰~१.८.१११) \textcolor{red}{जघान तेन दुष्टात्मा जटायुं धर्मचारिणम्} (नर॰पु॰~४९.९६) \textcolor{red}{माऽऽवधिष्ठा जटायुं माम्} (भ॰का॰~६.४१) \textcolor{red}{जटायोः मोक्षप्राप्तिः} (सा॰द॰~६.६९) \textcolor{red}{ब्रह्मास्त्रे चापि दत्ते पथि पितृसुहृदं वीक्ष्य भूयो जटायुम्} (ना॰~९.३४.७) इत्यादिषु।}\end{sloppypar}
\section[वन्दितो मे]{वन्दितो मे}
\centering\textcolor{blue}{अष्टावक्रः पुनः प्राह वन्दितो मे दयापरः।\nopagebreak\\
शापस्यान्तं च मे प्राह तपसा द्योतितप्रभः॥}\nopagebreak\\
\raggedleft{–~अ॰रा॰~३.९.१८}\\
\begin{sloppypar}\hyphenrules{nohyphenation}\justifying\noindent\hspace{10mm} अत्र \textcolor{red}{मया वन्दितः} इत्येवोचितं यतो हि कर्मणि \textcolor{red}{क्त}\-प्रत्ययः कर्ता चानुक्तोऽतोऽनुक्ते कर्तरि तृतीया \textcolor{red}{मया} इति।\footnote{\textcolor{red}{कर्तृ\-करणयोस्तृतीया} (पा॰सू॰~२.३.१८) इत्यनेन।}
अत्र कर्मणि सम्बन्ध\-विवक्षायां षष्ठी। अथवा \textcolor{red}{क्तस्य च वर्तमाने} (पा॰सू॰~२.३.६७) इत्यनेन षष्ठी।\footnote{\textcolor{red}{मति\-बुद्धि\-पूजार्थेभ्यश्च} (पा॰सू॰~३.२.१८८) इत्यनेन पूजार्थे वर्तमाने क्तः। मे वन्दितः सन् वन्द्यमानो वा प्राह इत्यर्थो न त्वहं तमवन्दे पश्चात्स प्राहेति। एवमेव \textcolor{red}{त्वमव्ययः शाश्वतधर्मगोप्ता सनातनस्त्वं पुरुषो मतो मे} (भ॰गी॰~११.१८) इति गीतावचने \textcolor{red}{मतः} इत्यत्र \textcolor{red}{मति\-बुद्धि\-पूजार्थेभ्यश्च} (पा॰सू॰~३.२.१८८) इत्यनेन पूजार्थे वर्तमाने क्तः \textcolor{red}{क्तस्य च वर्तमाने} (पा॰सू॰~२.३.६७) इत्यनेन च षष्ठी।} यद्वा \textcolor{red}{मे} इत्यस्य \textcolor{red}{प्राह} इत्यनेनान्वयः। तुमुन्कर्मणि चतुर्थी।\footnote{\textcolor{red}{क्रियार्थोपपदस्य च कर्मणि स्थानिनः} (पा॰सू॰~२.३.१४) इत्यनेन। \textcolor{red}{मां बोधयितुमुद्धर्तुं वा प्राह} इत्यर्थः। एवमेवोत्तरार्धे \textcolor{red}{शापस्यान्तं च मे प्राह} इत्यत्र बोध्यम्।}\end{sloppypar}
\section[देवराजानम्]{देवराजानम्‌}
\centering\textcolor{blue}{कदाचिद्देवराजानमभ्याद्रवमहं रुषा।\nopagebreak\\
सोऽपि वज्रेण मां राम शिरोदेशेऽभ्यताडयत्॥}\nopagebreak\\
\raggedleft{–~अ॰रा॰~३.९.२१}\\
\begin{sloppypar}\hyphenrules{nohyphenation}\justifying\noindent\hspace{10mm} \textcolor{red}{देवानां राजेति देवराजः} इति विग्रहे षष्ठी\-तत्पुरुषे \textcolor{red}{राजाऽहस्सखिभ्यष्टच्‌} (पा॰सू॰~५.४.९१) इत्यनेन टच्प्रत्यये \textcolor{red}{देवराजम्‌} इत्युचितम्। \textcolor{red}{देव\-राजानम्‌} इति च समासान्त\-प्रत्ययानामनित्यत्व\-स्वीकारे सति। अथवा \textcolor{red}{राजनं राट्‌}। भावे क्विप्।\footnote{\textcolor{red}{सम्पदादिभ्‍यः क्विप्‌} (वा॰~३.३.१०८) इत्यनेन।} \textcolor{red}{देवानां राडिति देवराट्‌}। \textcolor{red}{देवराजाऽऽसमन्तादनिति} इति विग्रहे \textcolor{red}{देवराजानः} तं \textcolor{red}{देवराजानम्‌} इति पाणिनीयमेव।\footnote{विस्तृत\-व्याख्यानं \pageref{sec:devarajanam}तमे पृष्ठे \ref{sec:devarajanam} \nameref{sec:devarajanam} इति प्रयोगस्य विमर्शे पश्यन्तु।}\end{sloppypar}
\section[ते]{ते}
\centering\textcolor{blue}{भक्त्या त्वत्पादकमले भक्तिमार्गविशारदा।\nopagebreak\\
तां प्रयाहि महाभाग सर्वं ते कथयिष्यति॥}\nopagebreak\\
\raggedleft{–~अ॰रा॰~३.१०.२}\\
\begin{sloppypar}\hyphenrules{nohyphenation}\justifying\noindent\hspace{10mm} अत्र \textcolor{red}{त्वां बोधयितुं कथयिष्यति} इति तुमुन्कर्मणि चतुर्थी।\footnote{\textcolor{red}{क्रियार्थोपपदस्य च कर्मणि स्थानिनः} (वा॰~२.३.१४) इत्यनेन।} यद्वा \textcolor{red}{ते पुरतः कथयिष्यति} इति सम्बन्ध\-षष्ठी।\end{sloppypar}
\vspace{2mm}
\centering ॥ इत्यरण्यकाण्डीयप्रयोगाणां विमर्शः ॥\nopagebreak\\
\vspace{4mm}
\pdfbookmark[2]{किष्किन्धाकाण्डम्‌}{Chap1Part2Kanda4}
\phantomsection
\addtocontents{toc}{\protect\setcounter{tocdepth}{2}}\addtocontents{toc}{\protect\setcounter{tocdepth}{2}}
\addcontentsline{toc}{subsection}{किष्किन्धाकाण्डीयप्रयोगाणां विमर्शः}
\addtocontents{toc}{\protect\setcounter{tocdepth}{0}}
\centering ॥ अथ किष्किन्धाकाण्डीयप्रयोगाणां विमर्शः ॥\nopagebreak\\
\section[भाति मनो मम]{भाति मनो मम}
\centering\textcolor{blue}{युवां त्रैलोक्यकर्ताराविति भाति मनो मम।\nopagebreak\\
युवां प्रधानपुरुषौ जगद्धेतू जगन्मयौ॥}\nopagebreak\\
\raggedleft{–~अ॰रा॰~४.१.१३}\\
\begin{sloppypar}\hyphenrules{nohyphenation}\justifying\noindent\hspace{10mm} अत्र पम्पातीर ऋष्यमूक\-गिरेः पार्श्वे गच्छन्तौ राम\-लक्ष्मणौ विलोक्य भीतेन सुग्रीवेण प्रेषितो विप्र\-वेष\-धारी हनुमान् तौ पृच्छन् कथयति यत् \textcolor{red}{युवाम् ईश्वरौ इति मे मनो भाति}। \textcolor{red}{मनः} इति द्वितीयान्तम्। अर्थात् \textcolor{red}{मे मनः प्रतीत्थं प्रतीयते}।\footnote{\textcolor{red}{प्रति} इत्यध्याहार्यमिति भावः। ततः \textcolor{red}{अभितः\-परितः\-समया\-निकषा\-हा\-प्रति\-योगेऽपि} (वा॰~२.३.२) इत्यनेन द्वितीया।} यद्वा \textcolor{red}{मनः} इति प्रथमान्तमेव। \textcolor{red}{भाति} इत्यस्य प्रत्यायनमर्थो न तु भानम्। प्रतीत्युपसर्ग\-संयोजनेन धातोः प्रत्यायन\-रूपस्यार्थस्य स्वीकारात्।\footnote{यथा भारते विराट\-पर्वणि विराटोऽर्जुनं प्रति~– \textcolor{red}{नैवंविधाः क्लीबरूपा भवन्ति कथञ्चनेति प्रतिभाति मे मनः} (म॰भा॰~४.११.७)।} न च प्रतीत्यत्र न विलोक्यत इति चेत्। \textcolor{red}{विनाऽपि प्रत्ययं पूर्वोत्तर\-पद\-लोपो वक्तव्यः} (वा॰~५.३.८३) इति लुप्तत्वात्।\end{sloppypar}
\section[समागम्य रघूत्तमम्]{समागम्य रघूत्तमम्‌}
\centering\textcolor{blue}{ततोऽतिहर्षात्सुग्रीवः समागम्य रघूत्तमम्।\nopagebreak\\
वृक्षशाखां स्वयं छित्त्वा विष्टराय ददौ मुदा॥}\nopagebreak\\
\raggedleft{–~अ॰रा॰~४.१.३२}\\
\begin{sloppypar}\hyphenrules{nohyphenation}\justifying\noindent\hspace{10mm} \textcolor{red}{समागम्य} इत्यस्य मित्रता\-रूपेऽर्थे स्वीकृतेऽलिङ्गने वा तृतीयया भवितव्यं\footnote{यथा \textcolor{red}{राजभिस्तत्र वार्ष्णेयः समागच्छद्यथावयः} (म॰भा॰~५.९१.७) \textcolor{red}{त्वमात्मनस्तुल्यममुं वृणीष्व रत्नं समागच्छतु काञ्चनेन} (र॰वं॰~६.७९) \textcolor{red}{प्रहृष्टो भव विप्रर्षे समागच्छ मया सह} (म॰भा॰~१३.५०.८५) \textcolor{red}{सा त्वं मया समागच्छ} (म॰भा॰~३.३०७.२८) इत्यादिषु। अन्यान्युदाहरणानि चारुदेव\-शास्त्रि\-कृतायाम् \textcolor{red}{उपसर्गार्थ\-चन्द्रिकायाम्} द्वितीय\-खण्डे \textcolor{red}{समा–गम्} इत्यत्र पश्यन्तु।} किन्त्वागमन\-रूपार्थ\-स्वीकारे नापत्तिः।\footnote{यथा भारते \textcolor{red}{क्व नु नाम वयं सर्वाः कालेनाल्पेन तं नरम्। समागच्छेम यो नस्तद्रूपमापादयेत्पुनः॥} (म॰भा॰~१.२१७.१३) इत्यत्र।} यद्वा \textcolor{red}{रघूत्तमं प्रति समागम्य पश्चात्तेन मैत्रीं चकार} इति द्वितीया सङ्गता।\footnote{\textcolor{red}{प्रति} इत्यध्याहार्यमिति भावः। ततः \textcolor{red}{अभितः\-परितः\-समया\-निकषा\-हा\-प्रति\-योगेऽपि} (वा॰~२.३.२) इत्यनेन द्वितीया।}\end{sloppypar}
\section[हा सीतेति]{हा सीतेति (४.१.४१)}
\centering\textcolor{blue}{विमुच्य रामस्तद्दृष्ट्वा हा सीतेति मुहुर्मुहुः।\nopagebreak\\
हृदि निक्षिप्य तत्सर्वं रुरोद प्राकृतो यथा॥}\nopagebreak\\
\raggedleft{–~अ॰रा॰~४.१.४१}\\
\begin{sloppypar}\hyphenrules{nohyphenation}\justifying\noindent\hspace{10mm} अत्रापि \textcolor{red}{हा सीते इति} अस्यामवस्थायां गुणोऽसङ्गतः।\footnote{\textcolor{red}{हा सीते इति} इति स्थिते \textcolor{red}{एचोऽयवायावः} (पा॰सू॰~६.१.७८) इत्यनेन \textcolor{red}{हा सीतय् इति} इति जाते \textcolor{red}{लोपः शाकल्यस्य} (पा॰सू॰~८.३.१९) इत्यनेन यकारलोपे \textcolor{red}{हा सीत इति} जाते यकार\-लोप\-विधायि\-सूत्रस्य त्रिपादीस्थत्वात् \textcolor{red}{आद्गुणः} (पा॰सू॰~६.१.८७) इति गुण\-विधायि\-सूत्रस्य सपाद\-सप्ताध्यायी\-स्थत्वाद्यलोपेऽसिद्धे गुणोऽसङ्गत इति भावः।} किन्तु \textcolor{red}{इति}\-शब्द\-ग्राहक\-\textcolor{red}{ति}\-शब्द\-स्वीकारे न दोषः।\footnote{\pageref{fn:ti}तमे पृष्ठे \ref{fn:ti}तमीं पादटिप्पणीं पश्यन्तु।
} अथवा शुद्ध\-प्रथमान्त\-स्वीकारे गुणः सङ्गत एव। अथवा \textcolor{red}{अवङ् स्फोटायनस्य} (पा॰सू॰~६.१.१२३) इति व्यवस्थित\-विभाषया \textcolor{red}{क्वचिदन्यदेव} इति नियमेनात्राचि परेऽपि पूर्व\-रूपम्। अथवा \textcolor{red}{पृषोदरादीनि यथोपदिष्टम्‌} (पा॰सू॰~६.३.१०९) इति सूत्रेण तस्मिन् गणे पठित्वाऽत्र पूर्व\-रूपं \textcolor{red}{हा सीतेति}। अथवा पूर्वोक्त\-रीत्या \textcolor{red}{न मु ने} (पा॰सू॰~८.२.३) इत्यस्य नकारस्य मुभावादतिरिक्ते स्थलेऽपि प्रवृत्तत्वाद्यकारलोपे सिद्धे गुणे च \textcolor{red}{हा सीतेति} इति पाणिनीयम्।\footnote{\pageref{sec:jaayeti_siiteti}तमे पृष्ठे \ref{sec:jaayeti_siiteti} \nameref{sec:jaayeti_siiteti} इति प्रयोगस्य विमर्शमपि पश्यन्तु।}\end{sloppypar}
\section[ते]{ते}
\centering\textcolor{blue}{आश्वास्य राघवं भ्राता लक्ष्मणो वाक्यमब्रवीत्।\nopagebreak\\
अचिरेणैव ते राम प्राप्यते जानकी शुभा।\nopagebreak\\
वानरेन्द्रसहायेन हत्वा रावणमाहवे॥}\nopagebreak\\
\raggedleft{–~अ॰रा॰~४.१.४२}\\
\begin{sloppypar}\hyphenrules{nohyphenation}\justifying\noindent\hspace{10mm} अत्र \textcolor{red}{प्राप्यते} इति कर्म\-वाच्ये प्रयोगेऽनुक्तत्वात्कर्तरि तृतीया प्रयोक्तव्या।\footnote{\textcolor{red}{कर्तृ\-करणयोस्तृतीया} (पा॰सू॰~२.३.१८) इत्यनेन।} षष्ठी\-प्रयोगस्तु कर्मणि सम्बन्ध\-विवक्षया। यद्वा \textcolor{red}{ते} इत्यस्य \textcolor{red}{जानकी} इति शब्देन सह अन्वयः। अर्थात् \textcolor{red}{ते तव जानकी} इति सामान्य\-सम्बन्धे षष्ठी।\end{sloppypar}
\section[तव]{तव}
\centering\textcolor{blue}{सुग्रीवोऽप्याह हे राम प्रतिज्ञां करवाणि ते।\nopagebreak\\
समरे रावणं हत्वा तव दास्यामि जानकीम्॥}\nopagebreak\\
\raggedleft{–~अ॰रा॰~४.१.४३}\\
\begin{sloppypar}\hyphenrules{nohyphenation}\justifying\noindent\hspace{10mm} अत्र \textcolor{red}{दा}\-धातु\-प्रयोगे (\textcolor{red}{डुदाञ् दाने} धा॰पा॰~१०९१) \textcolor{red}{तुभ्यम्‌} इति चतुर्थी प्रयोक्तव्या।\footnote{\textcolor{red}{चतुर्थी सम्प्रदाने} (पा॰सू॰~२.३.१३) इत्यनेन।} षष्ठी\-प्रयोगस्तु सम्प्रदानेऽपि सम्बन्ध\-विवक्षायाम्। अथवा \textcolor{red}{तव सहचारिणीं जानकीं दास्यामि} इति दाम्पत्य\-भावे सम्बन्धे षष्ठी।\footnote{\textcolor{red}{सहचारिणीम्} इत्यध्याहार्यमिति भावः।}\end{sloppypar}
\section[रघुनायके]{रघुनायके}
\centering\textcolor{blue}{स्वोदन्तं कथयामास प्रणयाद्रघुनायके।\nopagebreak\\
सखे श्रृणु ममोदन्तं वालिना यत्कृतं पुरा॥}\nopagebreak\\
\raggedleft{–~अ॰रा॰~४.१.४६}\\
\begin{sloppypar}\hyphenrules{nohyphenation}\justifying\noindent\hspace{10mm} अत्र \textcolor{red}{कथयामास} इति \textcolor{red}{कथ}\-धातु\-प्रयोगे (\textcolor{red}{कथँ वाक्य\-प्रबन्धने} धा॰पा॰~१८५१) \textcolor{red}{रघुनायकम्‌} इति द्वितीयया भवितव्यमासीत्।\footnote{\textcolor{red}{अकथितं च} (पा॰सू॰~१.४.५१) इत्यनेन कर्म\-सञ्ज्ञायां \textcolor{red}{कर्मणि द्वितीया} (पा॰सू॰~२.३.२) इत्यनेन।} \textcolor{red}{रघुनायके} इति सप्तमी त्वपाणिनीयेव। अत्र वैषयिक आधारे सप्तमी। विषयता च प्रतिपाल्यता\-रूपा। यद्वा \textcolor{red}{रघुनायके शृण्वति कथयामास} इत्यध्याहारे \textcolor{red}{यस्य च भावेन भाव\-लक्षणम्‌} (पा॰सू॰~२.३.३७) इत्यनेन सप्तमी। यद्वा \textcolor{red}{रघुनायके साधुः सुग्रीवः कथयामास} इत्यध्याहारे \textcolor{red}{साध्व\-साधु\-प्रयोगे च} (वा॰~२.३.३६) इत्यनेन सप्तमी।\end{sloppypar}
\section[तदादि]{तदादि}
\centering\textcolor{blue}{तदादि मम भार्यां स स्वयं भुङ्क्ते विमूढधीः।\nopagebreak\\
अतो दुःखेन सन्तप्तो हृतदारो हृताश्रयः॥}\nopagebreak\\
\raggedleft{–~अ॰रा॰~४.१.५७}\\
\begin{sloppypar}\hyphenrules{nohyphenation}\justifying\noindent\hspace{10mm} अत्र प्रभृत्यर्थतया पञ्चमी युक्ता\footnote{\textcolor{red}{‘अपादाने पञ्चमी’ इति सूत्रे ‘कार्तिक्याः प्रभृति’ इति भाष्य\-प्रयोगात् प्रभृत्यर्थयोगे पञ्चमी} (वै॰सि॰कौ॰~५९५)।} किन्तु \textcolor{red}{आरभ्य\-योगे द्वितीया च}\footnote{मूलं मृग्यम्।} इति द्वितीयैवात्र।\footnote{\textcolor{red}{अत्राऽरभ्येति क्रियापेक्षया कर्मत्वविवक्षायां द्वितीयैव। ‘उपपदविभक्तेः कारकविभक्तिर्बलीयसी’ इत्युक्तेः। यथा ‘सूर्योदयमारभ्य आऽस्तमयाज्जपति’ इत्यादौ} (बा॰म॰~५९५)।}\end{sloppypar}
\section[कपीन्द्राय]{कपीन्द्राय}
\centering\textcolor{blue}{किन्तु लोका वदिष्यन्ति मामेवं रघुनन्दनः।\nopagebreak\\
कृतवान्किं कपीन्द्राय सख्यं कृत्वाऽग्निसाक्षिकम्॥}\nopagebreak\\
\raggedleft{–~अ॰रा॰~४.२.३}\\
\begin{sloppypar}\hyphenrules{nohyphenation}\justifying\noindent\hspace{10mm}
अत्र \textcolor{red}{कपीन्द्रस्य} इति षष्ठ्यापाततः किन्तु विचारे कृते \textcolor{red}{हिताय} इत्यध्याहारे \textcolor{red}{हित\-योगे च} (वा॰~२.३.१३) इत्यनेन चतुर्थी। यद्वा \textcolor{red}{तादर्थ्ये चतुर्थी वाच्या} (वा॰~२.३.१३) इति वार्त्तिकेन चतुर्थी।\end{sloppypar}
\section[मे]{मे}
\centering\textcolor{blue}{एवं मे प्रत्ययं कृत्वा सत्यवादिन् रघूत्तम।\nopagebreak\\
उपेक्षसे किमर्थं मां शरणागतवत्सल॥}\nopagebreak\\
\raggedleft{–~अ॰रा॰~४.२.१२}\\
\begin{sloppypar}\hyphenrules{nohyphenation}\justifying\noindent\hspace{10mm} \textcolor{red}{मे} इति षष्ठी\-प्रयोगोऽसङ्गतः। यतो हि रामः सुग्रीवे विश्वासमुत्पादितवान् वालि\-हनन\-प्रत्ययतः। वैषयिकाधारतया \textcolor{red}{मयि} इति प्रयोक्तव्यम्। परं \textcolor{red}{मे उपरि मे हृदये} वेत्यध्याहारे न दोषः। अथवा सुग्रीव आत्मीयतया कथयति \textcolor{red}{हे मे रघूत्तम} इति मत्सम्बन्धि\-रघूत्तम। सम्बन्धश्च रक्ष्य\-रक्षक\-भाव\-रूपः। तत्र षष्ठी।\end{sloppypar}
\section[त्वां शपे]{त्वां शपे}
\centering\textcolor{blue}{गत्वाऽह्वय पुनः शत्रुं हतं द्रक्ष्यसि वालिनम्।\nopagebreak\\
रामोऽहं त्वां शपे भ्रातर्हनिष्यामि रिपुं क्षणात्॥}\nopagebreak\\
\raggedleft{–~अ॰रा॰~४.२.१५}\\
\begin{sloppypar}\hyphenrules{nohyphenation}\justifying\noindent\hspace{10mm} अत्र \textcolor{red}{त्वाम्‌} इति प्रयोगः सन्दिग्धः। शापस्थ\-विषयतया सुग्रीवो वैषयिक आधारः। तद्वाचक\-युष्मच्छब्द एक\-वचने सप्तम्युचिता।\footnote{\textcolor{red}{सप्तम्यधिकरणे च} (पा॰सू॰~२.३.३६) इत्यनेन।} किन्तु मीमांस्यमाने प्रतिरध्याहार्यः। \textcolor{red}{त्वां प्रति शपे} इति योजने \textcolor{red}{अभितः\-परितः\-समया\-निकषा\-हा\-प्रति\-योगेऽपि} (वा॰~२.३.२) इत्यनेन द्वितीया।\end{sloppypar}
\section[ते]{ते}
\centering\textcolor{blue}{इदानीमेव ते भग्नः पुनरायाति सत्वरः।\nopagebreak\\
सहायो बलवांस्तस्य कश्चिन्नूनं समागतः॥}\nopagebreak\\
\raggedleft{–~अ॰रा॰~४.२.२१}\\
\centering\textcolor{blue}{इति निश्चित्य तौ यातौ निश्चितं श्रृणु मद्वचः।\nopagebreak\\
इदानीमेव ते भग्नः कथं पुनरुपागतः॥}\nopagebreak\\
\raggedleft{–~अ॰रा॰~४.२.३०}\\
\begin{sloppypar}\hyphenrules{nohyphenation}\justifying\noindent\hspace{10mm} \textcolor{red}{ते} इत्यनुचितो यतो हि \textcolor{red}{भग्नः} इति कर्मणि क्तान्तस्ततः कर्तर्यनुक्ते तृतीया\footnote{\textcolor{red}{कर्तृ\-करणयोस्तृतीया} (पा॰सू॰~२.३.१८) इत्यनेन।} किन्तु \textcolor{red}{ते मुष्टिना भग्नः} इत्यध्याहारेऽङ्गाङ्गि\-भाव\-सम्बन्धे षष्ठी।\end{sloppypar}
\section[एकामपि]{एकामपि}
\centering\textcolor{blue}{अधर्मकारिणं हत्वा सद्धर्मं पालयाम्यहम्।\nopagebreak\\
दुहिता भगिनी भ्रातुर्भार्या चैव तथा स्नुषा॥\\
समा यो रमते तासामेकामपि विमूढधीः।\nopagebreak\\
पातकी स तु विज्ञेयः स वध्यो राजभिः सदा॥}\nopagebreak\\
\raggedleft{–~अ॰रा॰~४.२.६०.६१}\\
\begin{sloppypar}\hyphenrules{nohyphenation}\justifying\noindent\hspace{10mm} रमणं सार्धं वाऽऽधारे वा भवति। यदि चेत्सार्धं तदा \textcolor{red}{एकया सह} इति \textcolor{red}{सह\-युक्तेऽप्रधाने} (पा॰सू॰~२.३.१९) इत्यनेन तृतीया। यदि चेदौपश्लेषिक आधारे रमणं चेत् \textcolor{red}{एकस्याम्‌} इति सप्तमी। किन्त्वत्र \textcolor{red}{एकाम्‌} इति द्वितीया तु \textcolor{red}{एकां गृहीत्वा} इत्यध्याहारेण ग्रहण\-कर्मणि।\end{sloppypar}
\section[चतुर्द्वारकपाटादीन्]{चतुर्द्वारकपाटादीन्‌}
\centering\textcolor{blue}{चतुर्द्वारकपाटादीन् बद्ध्वा रक्षामहे पुरीम्।\nopagebreak\\
वानराणां तु राजानमङ्गदं कुरु भामिनि॥}\nopagebreak\\
\raggedleft{–~अ॰रा॰~४.३.३}\\
\begin{sloppypar}\hyphenrules{nohyphenation}\justifying\noindent\hspace{10mm} अत्र श्रीरामभद्रेण वालिनि हते तारां प्रति भीत\-वानराः कथयन्ति यत् \textcolor{red}{चतुर्द्वारकपाटादीन् रुन्त्स्व}। अत्र बहुव्रीहिः। \textcolor{red}{चतुर्द्वार\-कपाटान्यादौ येषां तान्‌} इति विग्रहेऽत्र पदार्थस्य सामान्यतया \textcolor{red}{सामान्ये नपुंसकम्‌} (वा॰~२.४.३०) इत्यनेन नपुंसके \textcolor{red}{चतुर्द्वार\-कपाटादीनि} इत्येव पाणिनीयम्।\footnote{चतुर्द्वार\-कपाटादि~शस्~\arrow \textcolor{red}{जश्शसोः शिः} (पा॰सू॰~७.१.२०)~\arrow चतुर्द्वार\-कपाटादि~शि~\arrow \textcolor{red}{शि सर्वनामस्थानम्} (पा॰सू॰~१.१.४२)~\arrow सर्वनाम\-स्थान\-सञ्ज्ञा~\arrow चतुर्द्वार\-कपाटादि~इ~\arrow \textcolor{red}{इकोऽचि विभक्तौ} (पा॰सू॰~७.१.७३)~\arrow \textcolor{red}{आद्यन्तौ टकितौ} (पा॰सू॰~१.१.४६)~\arrow चतुर्द्वार\-कपाटादि~नुँम्~इ~\arrow चतुर्द्वार\-कपाटादि~न्~इ~\arrow \textcolor{red}{सर्वनामस्थाने चासम्बुद्धौ} (पा॰सू॰~६.४.८)~\arrow चतुर्द्वार\-कपाटादी~न्~इ~\arrow चतुर्द्वार\-कपाटादीनि।} किन्तु विशेषण\-वस्तु\-योजनान्न दोषः। यथा \textcolor{red}{चतुर्द्वार\-कपाटादीन् उपकरण\-विशेषान्‌} इत्यध्याहारे न दोषः।\footnote{चतुर्द्वार\-कपाटादि~शस्~\arrow \textcolor{red}{प्रथमयोः पूर्वसवर्णः} (पा॰सू॰~६.१.१०२)~\arrow चतुर्द्वार\-कपाटादी~शस्~\arrow \textcolor{red}{तस्माच्छसो नः पुंसि} (पा॰सू॰~६.१.१०३)~\arrow चतुर्द्वार\-कपाटादी~न्~\arrow चतुर्द्वार\-कपाटादीन्।}\end{sloppypar}
\section[पतिना]{पतिना}
\label{sec:patina}
\centering\textcolor{blue}{किमङ्गदेन राज्येन नगरेण धनेन वा।\nopagebreak\\
इदानीमेव निधनं यास्यामि पतिना सह॥}\nopagebreak\\
\raggedleft{–~अ॰रा॰~४.३.५}\\
\begin{sloppypar}\hyphenrules{nohyphenation}\justifying\noindent\hspace{10mm} अत्र रामेण वालिनि हते तारा विलपन्ती प्राह यत् \textcolor{red}{पतिना सह निधनं यास्यामि}। अत्र \textcolor{red}{पतिना} इति प्रयोगोऽपाणिनीय इव। यतो हि \textcolor{red}{पति}\-शब्दस्य तृतीयैक\-वचने \textcolor{red}{घि}\-सञ्ज्ञायाम् \textcolor{red}{आङो नाऽस्त्रियाम्‌} (पा॰सू॰~७.३.१२०) इत्यनेनाङो नादेशे \textcolor{red}{पतिना} इति सम्भवति किन्तु नादेशः सत्यां घिसञ्ज्ञायां 
सम्भवः।
सा च \textcolor{red}{घि}\-सञ्ज्ञा पति\-शब्दस्य समास एव 
सम्भवा।
\textcolor{red}{पतिः समास एव} (पा॰सू॰~१.४.८) इति सूत्रात्। अतः समासाभावे घि\-सञ्ज्ञाया असम्भवात्तदभावे च नादेशस्यासम्भवे \textcolor{red}{पतिना} इत्यनुपपन्न एवेति चेत्। अत्र नास्ति शुद्ध\-पति\-शब्दोऽपि तु \textcolor{red}{पतिरिवाऽचरति पतयति} इति विग्रह आचारे \textcolor{red}{क्विप्}।\footnote{पति~\arrow \textcolor{red}{सर्वप्राति\-पदिकेभ्य आचारे क्विब्वा वक्तव्यः} (वा॰~३.१.११)~\arrow पति~क्विँप्~\arrow पति~व्~\arrow \textcolor{red}{वेरपृक्तस्य} (पा॰सू॰~६.१.६७)~\arrow पति~\arrow \textcolor{red}{सनाद्यन्ता धातवः} (पा॰सू॰~३.१.३२)~\arrow धातुसञ्ज्ञा~\arrow \textcolor{red}{शेषात्कर्तरि परस्मैपदम्} (पा॰सू॰~१.३.७८)~\arrow \textcolor{red}{वर्तमाने लट्} (पा॰सू॰~३.२.१२३)~\arrow पति~लट्~\arrow पति~तिप्~\arrow पति~ति~\arrow \textcolor{red}{कर्तरि शप्‌} (पा॰सू॰~३.१.६८)~\arrow पति~शप्~ति~\arrow पति~अ~ति~\arrow \textcolor{red}{सार्वधातुकार्ध\-धातुकयोः} (पा॰सू॰~७.३.८४)~\arrow पते~अ~ति~\arrow \textcolor{red}{एचोऽयवायावः} (पा॰सू॰~६.१.७८)~\arrow पतय्~अ~ति~\arrow पतयति।} पुनः \textcolor{red}{पतयतीति पतिः}। कर्तरि \textcolor{red}{क्विप्}।\footnote{पति~\arrow पूर्ववद्धातु\-सञ्ज्ञा~\arrow \textcolor{red}{क्विप् च} (पा॰सू॰~३.२.७६)~\arrow पति~क्विँप्~\arrow पति~व्~\arrow \textcolor{red}{वेरपृक्तस्य} (पा॰सू॰~६.१.६७)~\arrow पति~\arrow विभक्तिकार्यम्~\arrow पतिः।} सर्वापहारि\-लोपे \textcolor{red}{गौण\-मुख्ययोर्मुख्ये कार्य\-सम्प्रत्ययः} (प॰शे॰~१५) इति परिभाषा\-बलेनात्र गौणे पति\-शब्दे \textcolor{red}{पतिः समास एव} (पा॰सू॰~१.४.८) इति सूत्रस्याप्रवृत्तौ \textcolor{red}{शेषो घ्यसखि} (पा॰सू॰~१.४.७) इति सूत्रेण घि\-सञ्ज्ञायाम् \textcolor{red}{आङो नाऽस्त्रियाम्‌} (पा॰सू॰~७.३.१२०) इत्यनेन नादेशे \textcolor{red}{पतिना} इति सिद्धं पाणिनीयमेव।\footnote{अपि च तत्त्वबोधिनीकाराः \textcolor{red}{पतिः समास एव} (पा॰सू॰~१.४.८) इति सूत्रे – \textcolor{red}{अथ कथं ‘सीतायाः पतये नमः’ (रा॰र॰स्तो॰~२७) इति ‘नष्टे मृते प्रव्रजिते क्लीबे च पतिते पतौ’ (प॰स्मृ॰~४.३०) इति पराशरश्च। अत्राहुः। पतिरित्याख्यातः पतिः ‘तत्करोति तदाचष्टे’ (धा॰पा॰ ग॰सू॰) इति णिचि टिलोपे ‘अच इः’ (प॰उ॰~४.१४८) इत्यौणादिक\-प्रत्यये ‘णेरनिटि’ (पा॰सू॰~६.४.५१) इति णिलोपे च निष्पन्नोऽयं पतिशब्दः ‘पतिः समास एव’ (पा॰सू॰~१.४.८) इत्यत्र न गृह्यते लाक्षणिकत्वादिति। एतेन ‘कृष्णस्य सखिरर्जुनः’ इति भारतम् ‘सखिना वानरेन्द्रेण’ इति रामायणं च व्याख्यातम्‌} (त॰बो॰~२५७)। अत्रत्या प्रक्रिया~– पति~\arrow \textcolor{red}{तत्करोति तदाचष्टे} (धा॰पा॰ ग॰सू॰~१८७)~\arrow पति~णिच्~\arrow पति~इ~\arrow \textcolor{red}{णाविष्ठवत्प्राति\-पदिकस्य पुंवद्भाव\-रभाव\-टिलोप\-यणादि\-परार्थम्} (वा॰~६.४.४८)~\arrow पत्~इ~\arrow पति~\arrow \textcolor{red}{सनाद्यन्ता धातवः} (पा॰सू॰~३.१.३२)~\arrow धातुसञ्ज्ञा~\arrow \textcolor{red}{शेषात्कर्तरि परस्मैपदम्} (पा॰सू॰~१.३.७८)~\arrow \textcolor{red}{वर्तमाने लट्} (पा॰सू॰~३.२.१२३)~\arrow पति~लट्~\arrow पति~तिप्~\arrow पति~ति~\arrow \textcolor{red}{कर्तरि शप्‌} (पा॰सू॰~३.१.६८)~\arrow पति~शप्~ति~\arrow पति~अ~ति~\arrow \textcolor{red}{सार्वधातुकार्ध\-धातुकयोः} (पा॰सू॰~७.३.८४)~\arrow पते~अ~ति~\arrow \textcolor{red}{एचोऽयवायावः} (पा॰सू॰~६.१.७८)~\arrow पतय्~अ~ति~\arrow पतयति। पति~\arrow पूर्ववद्धातु\-सञ्ज्ञा~\arrow \textcolor{red}{अच इः} (प॰उ॰~४.१४८)~\arrow पति~इ~\arrow \textcolor{red}{णेरनिटि} (पा॰सू॰~६.४.५१)~\arrow पत्~इ~\arrow पति~\arrow विभक्तिकार्यम्~\arrow पतिः।} यद्वा \textcolor{red}{पतिं स्वर्गं नयति} इति विग्रहे \textcolor{red}{उणादयो बहुलम्‌} (पा॰सू॰~३.३.१) इत्यनेन \textcolor{red}{डन्‌}\-प्रत्यये\footnote{\textcolor{red}{कार्याद्विद्यादनूबन्धम्} (भा॰पा॰सू॰~३.३.१) \textcolor{red}{केचिदविहिता अप्यूह्याः} (वै॰सि॰कौ॰~३१६९) इत्यनुसारमूह्योऽ\-त्राविहितो \textcolor{red}{डन्‌}\-प्रत्ययः।} \textcolor{red}{डित्यभस्याप्यनु\-बन्धकरण\-सामर्थ्यात्‌} (वा॰~६.४.१४३) इत्यनेन नी\-घटकेकारस्य लोपे \textcolor{red}{पतिनन्‌} इति जाते ततः प्रातिपदिक\-सञ्ज्ञायां सौ विभक्तौ \textcolor{red}{अलोऽन्त्यात्पूर्व उपधा} (पा॰सू॰~१.१.६५) इत्यनेनोपधा\-सञ्ज्ञायां \textcolor{red}{सर्वनामस्थाने चासम्बुद्धौ} (पा॰सू॰~६.४.८) इत्यनेन दीर्घे \textcolor{red}{हल्ङ्याब्भ्यो दीर्घात्सुतिस्यपृक्तं हल्‌} (पा॰सू॰~६.१.६८) इत्यनेन सुलोपे \textcolor{red}{नलोपः प्रातिपदिकान्तस्य} (पा॰सू॰~८.२.७) इत्यनेन नकार\-लोपे \textcolor{red}{पतिना} इति प्रथमान्त एव न तृतीयान्तः। अर्थात् \textcolor{red}{पतिना भवान्‌} अर्थात् \textcolor{red}{रामभद्र भवान् पतिं स्वर्गं नीतवानतोऽहमपि निधनं यास्यामि}।\footnote{\textcolor{red}{सह}\-शब्दान्वये \textcolor{red}{अहमपि सह युगपदेव निधनं यास्यामीत्यर्थः}। \textcolor{red}{सह}\-शब्दो यौगपद्ये इति वाचस्पत्य\-काराः। यथा \textcolor{red}{अस्तोदयौ सहैवासौ कुरुते नृपतिर्द्विषाम्} इत्यत्र।} निःशेषं धनं निधनं स्वकीयं सर्वस्वं पतिमनु\-यास्यामि। अथवा निचितं निखिलं भोगादि\-धनं मोक्ष\-रूपं वा धनं यस्मिन् तादृशं निधनं साकेतमहमपि यास्यामि। तारा प्रार्थयते यत् \textcolor{red}{भवान् पतिनाऽर्थात्पतितमपि मे पतिं यया कृपया साकेतं नीतवान् तयैवाहमपि निधनं साकेतं यास्यामि} इति पाणिनीय एव \textcolor{red}{पतिना} शब्दस्तेन नात्रासिद्धिः। अथवा \textcolor{red}{पतिं न असहते} अर्थात्सहत एवेति \textcolor{red}{षहँ मर्षणे} (धा॰पा॰~८५२, १८०९) इत्यस्मात् \textcolor{red}{अच्‌} प्रत्ययः पचादित्वात्।\footnote{\textcolor{red}{नन्दि\-ग्रहि\-पचादिभ्यो ल्युणिन्यचः} (पा॰सू॰~३.१.१३४) इत्यनेन।} अत्र \textcolor{red}{न सहः} इति \textcolor{red}{असहः} इति नञ्तत्पुरुषः।
\textcolor{red}{न असहः} इति \textcolor{red}{नासहः} इति सुप्सुपा\-समासः।\footnote{\textcolor{red}{नलोपो नञः} (पा॰सू॰~६.३.७३) इत्यनेन नलोपो नञ्तत्पुरुष\-समासे भवति परन्तु सुप्सुपा\-समासे न भवतीति व्याकरण\-चन्द्रोदये प्रथम\-खण्डे चारुदेव\-शास्त्रिणः। यथा नचिर (\textcolor{red}{यथा नचिरकालं नो निष्कृतिः स्यात्त्रिलोकगे} म॰भा॰~१.९६.१९, \textcolor{red}{मुनिर्ब्रह्म नचिरेणाधिगच्छति} भ॰गी॰~५.६, \textcolor{red}{भवामि नचिरात्पार्थ मय्यावेशितचेतसाम्‌} भ॰गी॰~१२.७), नान्तरीयक (\textcolor{red}{यत्र नान्तरीयकोऽलाश्रीयते नासावल्विधिः} भा॰पा॰सू॰~१.१.५६ \textcolor{red}{नान्तरीयकत्वात्} भा॰पा॰सू॰~१.२.३९, ३.३.१८, ३.४.२१, ४.१.९२) नाणक (\textcolor{red}{तुलाशासन\-मानानां कूटकृन्नाणकस्य च} या॰स्मृ॰~२.२०.२४०), नास्ति (\textcolor{red}{अस्ति नास्ति न जानन्ति देहि देहि पुनः पुनः} म॰सु॰स॰~५५४), नेष्ट (\textcolor{red}{अतिककुदाः कृशदेहा नेष्टा हीनाधिकाङ्ग्यश्च॥} बृ॰सं॰~६१.४), नसंहत (\textcolor{red}{नसंहतास्तस्य नभिन्नवृत्तयः} कि॰~१.१९), नभिन्न (\textcolor{red}{नसंहतास्तस्य नभिन्नवृत्तयः} कि॰~१.१९), नसुकर (\textcolor{red}{कृत्वा नसुकरं कर्म गता वैवस्वतक्षयम्} म॰भा॰~८.३.२१), नैकभेद (\textcolor{red}{उच्चावचं नैकभेदम्} अ॰को॰~३.१.८३) नैक (\textcolor{red}{सा ददर्श नगान्नैकान् नैकाश्च सरितस्तथा} म॰भा॰~३.६१.१०४, \textcolor{red}{एको नैकः सवः कः किं यत्तत्पदमनुत्तमम्‌} वि॰स॰ना॰~६१) इत्यादिषु। यथा वाचस्पतये – \textcolor{red}{नैक त्रि० न एकः नञर्थनशब्देन “सह सुपा” पा॰स॰~अनेके १ एकभिन्ने २ विष्णौ पु॰~“एको नैकः सवः कः किम्” विष्णुस॰~मायया बहुरूपत्वान्नैकः “इन्द्रो मायाभिः पुरुरूप ईयते इति श्रुतेः” भा॰}।} ततः \textcolor{red}{पत्युर्नासहः} इति \textcolor{red}{पति\-नासहः} इति षष्ठी\-तत्पुरुष\-समासस्तत्सम्बुद्धौ \textcolor{red}{हे पति\-नासह} अर्थात् \textcolor{red}{हे पति\-पाप\-सहन\-कर्तः}। अर्थात्प्रभो त्वमति\-करुणोऽसि कारुणिकोत्तमोऽसि यतो हि पर\-कलत्र\-गामिनं नितान्त\-कामिनं पापिनं मम स्वामिनमपि त्वं नासहसे। \textcolor{red}{अभावाभावः प्रतियोगि\-ज्ञानस्य कारणं भवति} इति न्यायेन तमपि क्षमसे स्मेदृशी ते क्षमा तयैव प्रभो मेऽपराधमपि क्षान्त्वा मामपि निधनं साकेतं नय। \textcolor{red}{ह} इति पाद\-पूर्तौ \textcolor{red}{पतिमपि नाऽसिनोषि} न कर्म\-बन्धनेन बध्नास्यपि तु तमपि कर्म\-बन्धनान्मोचयित्वा साकेतं नयसीति \textcolor{red}{पतिनासः} इति विग्रहे \textcolor{red}{न आसिनोतीति नासः} इति विग्रहे \textcolor{red}{आ}पूर्वकात् \textcolor{red}{षिञ् बन्धने} (धा॰पा॰~१२४८, १४७७) इत्यस्मात् \textcolor{red}{अन्येष्वपि दृश्यते} (पा॰सू॰~३.२.१०१) इत्यनेन \textcolor{red}{ड}\-प्रत्यये डित्त्वाट्टिलोपे\footnote{\textcolor{red}{डित्यभस्याप्यनु\-बन्धकरण\-सामर्थ्यात्‌} (वा॰~६.४.१४३) इत्यनेन।} \textcolor{red}{पतिनासः} तत्सम्बुद्धौ \textcolor{red}{हे पतिनास हे वालि\-मोक्ष\-दायिन्‌} मामपि साकेतं नय। अथवा \textcolor{red}{पतिश्चासाविनश्चेति पतिनः}।\footnote{\textcolor{red}{शकन्ध्वादिषु पर\-रूपं वाच्यम्‌} (वा॰~६.१.९४) इत्यनेन पररूपः।} \textcolor{red}{न सहते भक्ताभक्त\-पीडामित्यसहः}।\footnote{अत्रापि \textcolor{red}{नन्दि\-ग्रहि\-पचादिभ्यो ल्युणिन्यचः} (पा॰सू॰~३.१.१३४) इत्यनेनाच्।} \textcolor{red}{पतिनश्चासावसहश्चेति पतिनासहस्तत्सम्बुद्धौ हे पतिनासह} इति विग्रहेऽपि प्रयोगसिद्धिः। अत्र पूर्वं स्वामि\-वाचकेन \textcolor{red}{इन}\-शब्देन \textcolor{red}{पति}\-शब्दस्य कर्मधारयः \textcolor{red}{पतिरेवेनः} इति विग्रहः। अत्र पति\-शब्दः पालकार्थ\-वाचकः। सति कर्मधारये विभक्ति\-लोपे पति\-घटकेकारस्येन\-घटकेकारेण सह \textcolor{red}{अकः सवर्णे दीर्घः} (पा॰सू॰~६.१.१०१) इति दीर्घः स्यादिति वाच्यम्। \textcolor{red}{शकन्ध्वादिषु पर\-रूपं वाच्यम्‌} (वा॰~६.१.९४) इत्यनेन पर\-रूपे सति न दोषः। न च \textcolor{red}{पतिन}\-शब्दस्य शकन्ध्वादि\-गणे न पाठः। तस्याप्याकृति\-गणतया\footnote{\textcolor{red}{आकृतिगणोऽयम्} (वै॰सि॰कौ॰~७९, ल॰सि॰कौ॰~३९)।} पाठ\-स्वीकारे क्षति\-विरहः। एवं \textcolor{red}{न सहते} इति पचाद्यञ्ञिष्पन्नस्य\footnote{\textcolor{red}{नन्दि\-ग्रहि\-पचादिभ्यो ल्युणिन्यचः} (पा॰सू॰~३.१.१३४)इत्यनेन।} \textcolor{red}{असह}\-शब्दस्य \textcolor{red}{पतिन एवासहः} इति विग्रहे कर्मधारये दीर्घे सम्बोधने च \textcolor{red}{हे पतिनासह} इति सिद्धम्। यद्वा \textcolor{red}{पतिं नाशयतीति पति\-नाशः} इति विग्रहे \textcolor{red}{कर्मण्यण्‌} (पा॰सू॰~३.२.१) इत्यनेन \textcolor{red}{अण्‌}प्रत्यये \textcolor{red}{नश्‌}\-धातोः (\textcolor{red}{णशँ अदर्शने} धा॰पा॰~११९४) उपधा\-वृद्धौ\footnote{\textcolor{red}{अत उपधायाः} (पा॰सू॰~७.२.११६) इत्यनेन।} समासे पृषोदरादित्वाद्दन्त्य\-सकारे सति सम्बोधने \textcolor{red}{हे पति\-नास}। \textcolor{red}{ह} इति पाद\-पूर्तौ। अथवा \textcolor{red}{अनः शकटे जले} इति नानार्थ\-कोषात्।\footnote{मूलं मृग्यम्। \textcolor{red}{क्लीबेऽनः शकटोऽस्त्री स्यात्} (अ॰को॰~२.८.५२) इत्यमरः। \textcolor{red}{आ ते॑ कारो शृणवामा॒ वचां॑सि य॒याथ॑ दू॒रादन॑सा॒ रथे॑न} (ऋ॰वे॰सं॰~३.३३.१०) इति मन्त्रे \textcolor{red}{अनसा शकटेन} इति सायणाः।} अत्र \textcolor{red}{पत्युरनः शरीर\-रूपं शकटं हन्तीति पतिनासहः} तत्सम्बुद्धौ \textcolor{red}{हे पतिनासह} इति विग्रहे \textcolor{red}{पति}\-शब्दस्य \textcolor{red}{अनस्‌}\-शब्देन तत्पुरुषे पृषोदरादित्वादकार\-लोपे \textcolor{red}{डस}\-प्रत्यये\footnote{\textcolor{red}{कार्याद्विद्यादनूबन्धम्} (भा॰पा॰सू॰~३.३.१) \textcolor{red}{केचिदविहिता अप्यूह्याः} (वै॰सि॰कौ॰~३१६९) इत्यनुसारमूह्योऽ\-त्राविहितः समासान्तो \textcolor{red}{डस}\-प्रत्ययः।} डित्त्वसामर्थ्याट्टिलोप\footnote{\textcolor{red}{डित्यभस्याप्यनु\-बन्धकरण\-सामर्थ्यात्‌} (वा॰~६.४.१४३) इत्यनेन।} आकारादेशे\footnote{पृषोदरादित्वादाकारादेशः।} पुनः \textcolor{red}{हन्‌} धातोः (\textcolor{red}{हनँ हिंसा\-गतयोः} धा॰पा॰~१०१२)
\textcolor{red}{अन्येष्वपि दृश्यते} (पा॰सू॰~३.२.१०१) इत्यनेन \textcolor{red}{ड}\-प्रत्यये टिलोपे\footnote{\textcolor{red}{डित्यभस्याप्यनु\-बन्धकरण\-सामर्थ्यात्‌} (वा॰~६.४.१४३) इत्यनेन।} \textcolor{red}{पतिनासहः}\footnote{पति~अनस्‌~\arrow \textcolor{red}{पृषोदरादीनि यथोपदिष्टम्} (पा॰सू॰~६.३.१०९)~\arrow अलोपः~\arrow पति~नस्~\arrow पतिनस्~\arrow \textcolor{red}{उणादयो बहुलम्} (पा॰सू॰~३.३.१)~\arrow पतिनस्~डस~\arrow पतिनस्~अस~\arrow \textcolor{red}{डित्यभस्याप्यनु\-बन्धकरण\-सामर्थ्यात्‌} (वा॰~६.४.१४३)~\arrow पतिन्~अस~\arrow पतिनस~\arrow \textcolor{red}{पृषोदरादीनि यथोपदिष्टम्} (पा॰सू॰~६.३.१०९)~\arrow आकारादेशः~\arrow पतिनास। \textcolor{red}{पतिनासं हन्ति} इति विग्रहे \textcolor{red}{पतिनासम्} इत्युपपदे \textcolor{red}{अन्येष्वपि दृश्यते} (पा॰सू॰~३.२.१०१) इत्यनेन \textcolor{red}{ड}\-प्रत्ययः। पतिनास~अम्~हन्~ड~\arrow पतिनास~अम्~हन्~अ~\arrow \textcolor{red}{डित्यभस्याप्यनु\-बन्धकरण\-सामर्थ्यात्‌} (वा॰~६.४.१४३)~\arrow पतिनास~अम्~ह्~अ~\arrow पतिनास~अम्~ह~\arrow \textcolor{red}{सुपो धातु\-प्रातिपदिकयोः} (पा॰सू॰~२.४.७१)~\arrow पतिनास~ह~\arrow पतिनासह~\arrow विभक्ति\-कार्यम्~\arrow पतिनासहः।} तत्सम्बुद्धौ \textcolor{red}{हे पतिनासह} अर्थाद्धे मत्पति\-शरीर\-शकट\-हन्तः प्रभोऽहमपि त्वल्लोकं जिगमिषामि। अथवा \textcolor{red}{अपतत्त्वच्चरणारविन्दे यः स पतिर्वाली} इति \textcolor{red}{पत्‌}\-धातोः (\textcolor{red}{पतॢँ गतौ} धा॰पा॰~८४५) भूतकाल औणादिके \textcolor{red}{इच्‌}प्रत्यये\footnote{\textcolor{red}{कार्याद्विद्यादनूबन्धम्} (भा॰पा॰सू॰~३.३.१) \textcolor{red}{केचिदविहिता अप्यूह्याः} (वै॰सि॰कौ॰~३१६९) इत्यनुसारमूह्योऽ\-त्राविहितः \textcolor{red}{इच्}\-प्रत्ययः।} चकारानुबन्ध\-कार्ये पुनस्तस्यैव घि\-सञ्ज्ञायां तृतीया\-\textcolor{red}{टा}\-विभक्तौ \textcolor{red}{आङो नाऽस्त्रियाम्‌} (पा॰सू॰~७.३.१२०) इत्यनेन नादेशे\footnote{पूर्ववत् \textcolor{red}{गौण\-मुख्ययोर्मुख्ये कार्य\-सम्प्रत्ययः} (प॰शे॰~१५) इति परिभाषया गौणे पति\-शब्दे \textcolor{red}{पतिः समास एव} (पा॰सू॰~१.४.८) इति सूत्रस्याप्रवृत्तौ \textcolor{red}{शेषो घ्यसखि} (पा॰सू॰~१.४.७) इति सूत्रेण घि\-सञ्ज्ञा।} योगे \textcolor{red}{सह\-युक्तेऽप्रधाने} (पा॰सू॰~२.३.१९) इत्यनेन तृतीयाप्राप्तौ \textcolor{red}{पतिना सह} इति पाणिनीयपरम्परया सम्यक्साधु। अर्थात् \textcolor{red}{त्वच्चरणारविन्द\-पतितेन पतिना वालिना सहाहमपि निधनं त्वत्सालोक्यं यास्यामि}। अथवा \textcolor{red}{पत्युर्वालिनो नाशः कुकर्म\-कार्यतया नरक\-गमनमिति पतिनासः} पृषोदरादित्वाद्दन्त्य\-सकारः \textcolor{red}{तमेव हन्तीति पतिनासहस्तत्सम्बुद्धौ हे पतिनासह} इति विग्रहे 
\textcolor{red}{हन्‌} धातोः (\textcolor{red}{हनँ हिंसा\-गतयोः} धा॰पा॰~१०१२) \textcolor{red}{अन्येष्वपि दृश्यते} (पा॰सू॰~३.२.१०१) इत्यनेन
\textcolor{red}{ड}\-प्रत्ययेऽभत्वेऽपि
डित्त्व\-सामर्थ्याट्टिलोपे\footnote{\textcolor{red}{डित्यभस्याप्यनु\-बन्धकरण\-सामर्थ्यात्‌} (वा॰~६.४.१४३) इत्यनेन।} सम्बोधन एकवचने \textcolor{red}{पतिनासह}। अर्थान्निज\-कृत\-कुकर्म\-परिणामेन तु मम पत्युर्नरक\-गमनमनिवार्यमासीत्किन्तु रामभद्र त्वमेव दीन\-वत्सलतया निज\-कृपा\-मन्दाकिनी\-शीकर\-सम्पात\-सेचनेन तादृशं नरक\-रूपं पतिनासमपि निहत्य तस्मै परां गतिं दत्तवानतोऽहमपि तामेव गन्तुमीह इति तारा\-तात्पर्यम्।\end{sloppypar}
\section[नाथनाथेति]{नाथनाथेति}
\centering\textcolor{blue}{पतितं वालिनं दृष्ट्वा रक्तैः पांसुभिरावृतम्।\nopagebreak\\
रुदती नाथनाथेति पतिता तस्य पादयोः॥}\nopagebreak\\
\raggedleft{–~अ॰रा॰~४.३.७}\\
\begin{sloppypar}\hyphenrules{nohyphenation}\justifying\noindent\hspace{10mm} अत्र रामेण निहतं वालिनं \textcolor{red}{नाथ\-नाथेति} तारा रुदती व्याहरत्। \textcolor{red}{नाथ\-नाथ इति} अयं सम्बोधने प्रथमैक\-वचनान्तः।
अतोऽत्र प्लुतः।\footnote{\textcolor{red}{दूराद्धूते च} (पा॰सू॰~८.२.८४) इत्यनेन।} एवं च \textcolor{red}{प्लुत\-प्रगृह्या अचि नित्यम्‌} (पा॰सू॰~६.१.१२५) इत्यनेन प्रकृति\-भावे गुणोऽसङ्गत इव\footnote{\textcolor{red}{नाथ\-नाथ३ इति} इत्यनेन भवितव्यमिति भावः।} किन्तु \textcolor{red}{प्राचाम्‌} (पा॰सू॰~८.२.८६) इति योग\-विभागे प्लुतानां वैकल्पिकत्वाद्गुण\-सम्भवः।\footnote{\textcolor{red}{इह प्राचामिति योगो विभज्यते। तेन सर्वः प्लुतो विकल्प्यते} (वै॰सि॰कौ॰~९७)।} यद्वा \textcolor{red}{अनुकरणानु\-कार्ययोर्भेदाभेद\-विवक्षा च}\footnote{मूलं मृग्यम्। \textcolor{red}{मतौ च्छः सूक्तसाम्नोः} (पा॰सू॰~५.२.५९) इत्यस्य भाष्ये प्रदीपोद्द्योतयोश्च स्पष्टमिदम्। अभेदपक्षे \textcolor{red}{प्रकृतिवदनुकरणं भवति} (भा॰शि॰सू॰~२) इति महाभाष्ये \textcolor{red}{ऋऌक्‌} (शि॰सू॰~२) शिवसूत्र उक्तम्। \textcolor{red}{अनुकरणं ह्यनुकार्याद्भिन्नम्‌} इत्यपि महाभाष्ये \textcolor{red}{मतौ च्छः सूक्तसाम्नोः} (पा॰सू॰~५.२.४९) सूत्र उक्तमिति वैयाकरण\-भूषण\-सारस्य दर्पण\-व्याख्यायां चन्द्रिका\-प्रसाद\-द्विवेदाः। अस्माभिर्भाष्य\-संस्करणेषु \textcolor{red}{अनुकरणं ह्यनुकार्याद्भिन्नम्‌} इति नोपलब्धम्।} इति परिभाषयाऽस्य प्रयोगस्यानु\-करणतयाऽभेद\-विवक्षायां विभक्त्यभावे सति गुणः सङ्गत एव। यद्वा \textcolor{red}{नाथ एव नाथो यस्याः सा नाथ\-नाथा} अर्थान्नाथो वाल्येव नाथः पतिर्यस्याः सा नाथ\-नाथा तारा। वालि\-परिचयार्थं सम्बोधयति यत् \textcolor{red}{पते त्वमेव यस्याः स्वामी सा एति तव सम्मुखमागच्छत्यतो मां पश्य} इत्यभिप्राये नाथ\-नाथा\-शब्दे बहुव्रीहिरन्य\-पदार्थश्च तारा। एवं गच्छत्यर्थक\-\textcolor{red}{एति}\-शब्दो \textcolor{red}{नाथ\-नाथा} इत्यनेन सह सन्धीयतां ततः \textcolor{red}{एङि पर\-रूपम्‌} (पा॰सू॰~६.१.९४) इत्यनेन पररूपे \textcolor{red}{नाथ\-नाथेति}। न चोक्त\-सूत्रमुपसर्ग\-धातु\-सन्धि\-विषयकमिति वाच्यम्। 
लक्ष्यानुरोधेनोप\-सर्गांशानुवृत्ति\-मोषे न दोषः।\footnote{\textcolor{red}{उपसर्गादृति धातौ} (पा॰सू॰~६.१.९१) इत्यतोऽनुवृत्तम् \textcolor{red}{उपसर्गात्} इति पदं लक्ष्यानुरोधेनात्र मोषणीयमिति भावः।} यथा \textcolor{red}{कर्तुरीप्सिततमं कर्म} (पा॰सू॰~१.४.४९) इत्यत्र क्रियाक्षिप्त\-कर्त्रंशे जागरूके व्यर्थीभूते \textcolor{red}{कर्तुः} इति पदं
\textcolor{red}{प्रकृतिधातूपात्त\-प्रधान\-भूत\-व्यापाराश्रयो यः कर्ता} इति विशिष्टार्थं बोधयति। तत्र पुनः \textcolor{red}{ईप्सित} इति वर्तमाने क्तप्रयोगात् \textcolor{red}{क्तस्य च वर्तमाने} (पा॰सू॰~२.३.६७) इति षष्ठ्यपि कर्तरि।\footnote{\textcolor{red}{मति\-बुद्धि\-पूजार्थेभ्यश्च} (पा॰सू॰~३.२.१८८) इत्यनेन \textcolor{red}{ईप्सित} इत्यत्र मत्यर्थे पूजार्थे वा वर्तमाने क्तः। \textcolor{red}{मतिरिच्छा} (का॰वृ॰~३.२.१८८) इति काशिका। \textcolor{red}{मतिरिहेच्छा} (वै॰सि॰कौ॰~३०८९) इति सिद्धान्त\-कौमुदी।} तत इयमप्येवं प्रकृतिरपीत्युभे
कर्त्रर्थं बोधयतः। अतः
षष्ठी\-वाच्य\-कर्त्रर्थस्य मोषं कुर्वन्ति गुरवः शेमुषीजुषः। तथैवात्राप्युपसर्गांश\-मोषेऽपि न रोषः कर्तव्यः। इत्थं \textcolor{red}{नाथनाथेति} इत्यत्र \textcolor{red}{एङि पररूपम्‌} (पा॰सू॰~६.१.९४) इत्यनेन पररूपम्।\end{sloppypar}
\section[मे]{मे}
\centering\textcolor{blue}{पूर्वजन्मनि ते सुभ्रु कृता मद्भक्तिरुत्तमा।\nopagebreak\\
अतस्तव विमोक्षाय रूपं मे दर्शितं शुभे॥}\nopagebreak\\
\raggedleft{–~अ॰रा॰~४.३.३४}\\
\begin{sloppypar}\hyphenrules{nohyphenation}\justifying\noindent\hspace{10mm} अत्र \textcolor{red}{क्तस्य च वर्तमाने} (पा॰सू॰~२.३.६७) इत्यनेन षष्ठी तृतीयां प्रबाध्य।\footnote{\textcolor{red}{मति\-बुद्धि\-पूजार्थेभ्यश्च} (पा॰सू॰~३.२.१८८) इत्यनेन बुद्ध्यर्थे वर्तमाने क्तः। सूत्रेऽस्मिन् बुद्धेर्ज्ञानमर्थः। यथा काशिकाकाराः~– \textcolor{red}{बुद्धिर्ज्ञानम्‌} (का॰वृ॰~३.२.१८८)। भगवतो रूपदर्शनं च प्रत्यक्ष\-ज्ञापनम्। यतो हि \textcolor{red}{इन्द्रियार्थ\-सन्निकर्ष\-जन्यं ज्ञानं प्रत्यक्षम्‌} (त॰स॰~४२)। यद्वा \textcolor{red}{मति\-बुद्धि\-पूजार्थेभ्यश्च} (पा॰सू॰~३.२.१८८) इत्यत्र \textcolor{red}{च}कारेणान्यत्रापि। \textcolor{red}{चकारोऽनुक्त\-समुच्चयार्थः। “शीलितो रक्षितः क्षान्त आक्रुष्टो जुष्ट इत्यपि” इत्यादि} (वै॰सि॰कौ॰~३०८९) इति सिद्धान्त\-कौमुद्यां भट्टोजि\-दीक्षित\-महाभागाः।} यद्वा \textcolor{red}{मे} इत्यस्य \textcolor{red}{रूपम्‌} इत्यनेनान्वयः।\footnote{तर्हि सम्बन्ध\-सामान्ये षष्ठी। सम्बन्धश्च धर्मि\-धर्म\-भाव\-रूपः। पूर्वार्धे च \textcolor{red}{त्वया कृता} इति प्रयोक्तव्ये \textcolor{red}{ते कृता} इति प्रयुक्तम्। \textcolor{red}{ते} इत्यस्य \textcolor{red}{पूर्वजन्मनि} इत्यनेनान्वये शङ्कापरिहारः। \textcolor{red}{सुभ्रु तव पूर्वजन्मनि उत्तमा मद्भक्तिः कृता} इति भावः।}\end{sloppypar}
\section[कुण्डेन]{कुण्डेन}
\centering\textcolor{blue}{अगस्त्येनोक्तमार्गेण कुण्डेनाऽगमवित्तमः।\nopagebreak\\
जुहुयान्मूलमन्त्रेण पुंसूक्तेनाथवा बुधः॥}\nopagebreak\\
\raggedleft{–~अ॰रा॰~४.४.३१}\\
\begin{sloppypar}\hyphenrules{nohyphenation}\justifying\noindent\hspace{10mm} अत्र सप्तम्यां तृतीया प्रोक्ता। करणत्व\-विवक्षायां\footnote{\textcolor{red}{कर्तृकरणयोस्तृतीया} (पा॰सू॰~२.३.१८) इत्यनेन।} हेतुत्व\-विवक्षायां\footnote{\textcolor{red}{हेतौ} (पा॰सू॰~२.३.२३) इत्यनेन।} प्रकृत्यादित्वाद्वा\footnote{\textcolor{red}{प्रकृत्यादिभ्य उपसङ्ख्यानम्‌} (वा॰~२.३.१८) इत्यनेन।} इयं साध्वी।\footnote{यद्वा \textcolor{red}{कुण्डे नाऽऽगमवित्तमः} इति पदच्छेदे \textcolor{red}{आगमवित्तमो ना मनुष्यः कुण्डे जुहुयात्‌} इत्यपि समाधानम्।}\end{sloppypar}
\section[माम्]{माम्‌}
\centering\textcolor{blue}{मद्भक्तो यदि मामेवं पूजां चैव दिने दिने।\nopagebreak\\
करोति मम सारूप्यं प्राप्नोत्येव न संशयः॥}\nopagebreak\\
\raggedleft{–~अ॰रा॰~४.४.३९}\\
\begin{sloppypar}\hyphenrules{nohyphenation}\justifying\noindent\hspace{10mm} अत्र \textcolor{red}{पूजाम्‌} इति कृदन्त\-प्रयोगेण \textcolor{red}{कर्तृ\-कर्मणोः कृति} (पा॰सू॰~२.३.६५) इत्यनेन षष्ठ्युचिता किन्तु \textcolor{red}{प्रति} इत्यध्याहारेण \textcolor{red}{मां प्रति पूजां करोति} इति \textcolor{red}{अभितः\-परितः\-समया\-निकषा\-हाप्रति\-योगेऽपि} (वा॰~२.३.२) इति \textcolor{red}{प्रति}\-योगे द्वितीयेति सिद्धम्।\end{sloppypar}
\section[शेषांशाय]{शेषांशाय}
\centering\textcolor{blue}{एवं परात्मा श्रीरामः क्रियायोगमनुत्तमम्।\nopagebreak\\
पृष्टः प्राह स्वभक्ताय शेषांशाय महात्मने॥}\nopagebreak\\
\raggedleft{–~अ॰रा॰~४.४.४१}\\
\begin{sloppypar}\hyphenrules{nohyphenation}\justifying\noindent\hspace{10mm} अत्रापि पूर्व\-प्रकारेण चतुर्थी \textcolor{red}{हित\-योगे च} (वा॰~२.३.१३) इत्यनेन।\footnote{\textcolor{red}{हितम्} इत्यध्याहार्यमिति भावः। यद्वा \textcolor{red}{तादर्थ्ये चतुर्थी वाच्या} (वा॰~२.३.१३) इत्यनेन चतुर्थी।}\end{sloppypar}
\section[हा सीतेति]{हा सीतेति}
\centering\textcolor{blue}{पुनः प्राकृतवद्रामो मायामालम्ब्य दुःखितः।\nopagebreak\\
हा सीतेति वदन्नैव निद्रां लेभे कथञ्चन॥}\nopagebreak\\
\raggedleft{–~अ॰रा॰~४.४.४२}\\
\begin{sloppypar}\hyphenrules{nohyphenation}\justifying\noindent\hspace{10mm} \textcolor{red}{हा सीतेति} इत्यत्र \textcolor{red}{इति}\-शब्दार्थकं \textcolor{red}{ति}\-शब्दं स्वीकृत्य समाधेयम्।\footnote{विस्तराय \pageref{sec:jaayeti_siiteti}तमे पृष्ठे \ref{sec:jaayeti_siiteti} \nameref{sec:jaayeti_siiteti} इति प्रयोगस्य विमर्शं पश्यन्तु।}\end{sloppypar}
\section[मणिसानौ]{मणिसानौ}
\centering\textcolor{blue}{रामस्तु पर्वतस्याग्रे मणिसानौ निशामुखे।\nopagebreak\\
सीताविरहजं शोकमसहन्निदमब्रवीत्॥}\nopagebreak\\
\raggedleft{–~अ॰रा॰~४.५.१}\\
\begin{sloppypar}\hyphenrules{nohyphenation}\justifying\noindent\hspace{10mm} अत्र \textcolor{red}{सानु}\-शब्दस्य प्रायो नपुंसक\-लिङ्गे प्रयोगात्\footnote{यथा \textcolor{red}{या सानु॑नि॒ पर्व॑ताना॒मदा॑भ्या म॒हस्त॒स्थतु॒रर्व॑तेव सा॒धुना॑} (ऋ॰वे॰सं॰~१.१५५.१) \textcolor{red}{बृ॑ह॒तः सानु॑न॒स्परि॑} (ऋ॰वे॰सं॰~५.५९.७) \textcolor{red}{सानू॑नि दि॒वो अ॒मृत॑स्य के॒तुना॑} (ऋ॰वे॰सं॰~६.७.६) \textcolor{red}{सानूनि मृगपक्षिणः} (वा॰रा॰~२.३३.२३) \textcolor{red}{गिरेः सानूनि रम्याणि} (वा॰रा॰~२.९३.९) \textcolor{red}{तस्य शैलस्य सानूनि} (वा॰रा॰~३.६१.२१) \textcolor{red}{दक्षिणे गिरिसानुनि} (वा॰रा॰~४.१.७३) \textcolor{red}{अस्मिन्सानुनि} (वा॰रा॰~४.१.१०३) \textcolor{red}{सानूनि सुमहान्ति च} (वा॰रा॰~६.६७.४) \textcolor{red}{तस्मिन्मन्दरसानुनि} (ग॰सं॰~७.४३.१४, भा॰पु॰~४.२३.२४) \textcolor{red}{यत्र स्रुतक्षीरतया प्रसूतः सानूनि गन्धः सुरभीकरोति} (कु॰स॰~१.९) इत्यादिषु। \textcolor{red}{सानुगतां} (कु॰स॰~१.५) इत्यत्र मल्लिनाथाः~– \textcolor{red}{सानूनि मेघमण्डलादधस्तटानि गतां प्राप्ताम्} (कु॰स॰ स॰व्या॰~१.९)।} सप्तम्येकवचने \textcolor{red}{इकोऽचि विभक्तौ} (पा॰सू॰~७.१.७३) इत्यनेन नुम्यनुबन्ध\-कार्ये \textcolor{red}{मणि\-सानुनि} इत्येव पाणिनीयम्।\footnote{पूर्वपक्षोऽयम्।} परं \textcolor{red}{मणि\-सानु}\-शब्दम् \textcolor{red}{अर्धर्चादि}\-गणे (\textcolor{red}{अर्धर्चाः पुंसि च} पा॰सू॰~२.४.३१) आकृति\-गणतया मत्वा पुंस्त्वम्। यद्वा \textcolor{red}{लिङ्गमशिष्यं लोकाश्रयत्वाल्लिङ्गस्य} (भा॰पा॰सू॰~२.१.३६) इति भाष्य\-वचनेन लिङ्गानामनियमनतया पुल्लिँङ्गे प्रयोगोऽपि पाणिनीयः।\footnote{यद्वा लिङ्गानुशासने \textcolor{red}{मद्गु\-मधु\-सीधु\-शीधु\-सानु\-कमण्डलूनि नपुंसके च} (लि॰~५६) इत्यत्र चकारात्पुंस्यपि \textcolor{red}{सानु}\-शब्दः।  यथा \textcolor{red}{अ॒पाद॑ह॒स्तो अ॑पृतन्य॒दिन्द्र॒मास्य॒ वज्र॒मधि॒ सानौ॑ जघान} (ऋ॰वे॰सं॰~१.३२.७) \textcolor{red}{उ॒र्व्याः प॒दो नि द॑धाति॒ सानौ॑} (ऋ॰वे॰सं॰~१.१४६.२) \textcolor{red}{पृथि॒व्याः सानौ॒ जङ्घ॑नन्त पा॒णिभि॑} (ऋ॰वे॰सं॰~२.३१.२) \textcolor{red}{म॒ना॒नग्रेतो॑ जहतुर्वि॒यन्ता॒ सानौ॒} (ऋ॰वे॰सं॰~१०.६१.६) इत्यादिषु। \textcolor{red}{स्नुः प्रस्थः सानुरस्त्रियौ} (अ॰को॰~२.३.५) इत्यमरः। \textcolor{red}{‘सानु’आदि\-शब्दानां स्वत एव द्विलिङ्गता~– ‘सानुने’ ‘सानवे’। स्नुः प्रस्थः सानुरस्त्रियामित्यमरः} (ह॰ना॰व्या॰~२.९३) इति हरिनामामृत\-व्याकरणे जीवगोस्वामिनः।}\end{sloppypar}
\section[मम]{मम}
\centering\textcolor{blue}{जीवतीति मम ब्रूयात्कश्चिद्वा प्रियकृत्स मे।\nopagebreak\\
यदि जानामि तां साध्वीं जीवन्तीं यत्र कुत्र वा॥}\nopagebreak\\
\raggedleft{–~अ॰रा॰~४.५.३}\\
\begin{sloppypar}\hyphenrules{nohyphenation}\justifying\noindent\hspace{10mm} अत्र \textcolor{red}{ब्रू}\-धातु\-योगे (\textcolor{red}{ब्रूञ् व्यक्तायां वाचि} धा॰पा॰~१०४४) द्वितीया तूचितैव\footnote{\textcolor{red}{अकथितं च} (पा॰सू॰~१.४.५१) इत्यनेन कर्म\-सञ्ज्ञायां \textcolor{red}{कर्मणि द्वितीया} (पा॰सू॰~२.३.२) इत्यनेन।} किन्तु कर्मणि सम्बन्ध\-विवक्षायां षष्ठ्यपि साध्वी।\end{sloppypar}
\section[बहुऋक्षवानरैः]{बहुऋक्षवानरैः}
\centering\textcolor{blue}{भेरीमृदङ्गैर्बहुऋक्षवानरैः श्वेतातपत्रैर्व्यजनैश्च शोभितः।\nopagebreak\\
नीलाङ्गदाद्यैर्हनुमत्प्रधानैः समावृतो राघवमभ्यगाद्धरिः॥}\nopagebreak\\
\raggedleft{–~अ॰रा॰~४.५.६३}\\
\begin{sloppypar}\hyphenrules{nohyphenation}\justifying\noindent\hspace{10mm} अत्र 
लक्ष्मणेन मृत्योर्भीयमानः प्रमादी सुग्रीवः श्रीराममुपगच्छति। अत्रैव प्रयोगः \textcolor{red}{बहुऋक्षवानरैः}। अत्र \textcolor{red}{बहु}\-शब्द\-घटकोकारस्य \textcolor{red}{ऋक्ष}\-घटकर्कारेण सन्धौ \textcolor{red}{बह्वृक्ष\-वानरैः} इत्येव पाणिनीयम्। \textcolor{red}{बहु}\-शब्दस्य \textcolor{red}{ऋक्ष\-वानर}\-शब्देन तत्पुरुष\-कर्मधारये समासे संहिताया नित्यत्वात् \textcolor{red}{इको यणचि} (पा॰सू॰~६.१.७७) इत्यनेन \textcolor{red}{यण्‌} अनिवार्यः। किन्तु व्यस्तावस्थायां सन्धिरनिवार्यो नास्ति। अतः \textcolor{red}{बहु ऋक्षवानरैः} इति पाणिनीयमेव। न च समासं विना \textcolor{red}{बहु}\-शब्दस्य \textcolor{red}{ऋक्ष\-वानर}\-शब्द\-विशेषणतया \textcolor{red}{बहुभिः ऋक्ष\-वानरैः} इति सविभक्तिकः प्रयोगः स्यादेवं \textcolor{red}{बहु ऋक्ष\-वानरैः} इत्यसङ्गतमिति वाच्यम्। \textcolor{red}{सुपां सुलुक्पूर्व\-सवर्णाच्छेयाडाड्यायाजालः} (पा॰सू॰~७.१.३९) इत्यनेन भिस्विभक्तेर्लुकि सङ्गतम्। यद्वा \textcolor{red}{बहु} इति क्रियाविशेषणम्। क्रिया\-विशेषणानां द्वितीयात्वं नपुंसकत्वमेक\-वचनत्वमौत्पत्तिकं सर्व\-विदितमेव। तेन नात्र सन्धिः। यद्वा \textcolor{red}{ऋत्यकः} (पा॰सू॰~६.१.१२४) इत्यनेन शाकल्यमते वैकल्पिक\-प्रकृतिभावः \textcolor{red}{तस्यां वै भार्गवऋषेः} (भा॰पु॰~९.१५.१३) इतिवत्।\footnote{\textcolor{red}{समासेऽप्ययं प्रकृतिभावः। सप्तऋषीणाम्। सप्तर्षीणाम्} (वै॰सि॰कौ॰~९२)। अन्यच्च \textcolor{red}{तत्रासीनं सुरऋषिम्} (भा॰पु॰~७.१.१४) \textcolor{red}{सम्पूज्य देवऋषिवर्यम्} (भा॰पु॰~१०.६९.१६) \textcolor{red}{ब्रह्मऋषीनेतान्} (भा॰पु॰~१०.८६.५७) इत्यादिष्वपि।}\end{sloppypar}
\section[बहिर्गुहाम्]{बहिर्गुहाम्‌}
\centering\textcolor{blue}{यूयं पिदध्वमक्षीणि गमिष्यथ बहिर्गुहाम्।\nopagebreak\\
तथैव चक्रुस्ते वेगाद्गताः पूर्वस्थितं वनम्॥}\nopagebreak\\
\raggedleft{–~अ॰रा॰~४.६.५८}\\
\begin{sloppypar}\hyphenrules{nohyphenation}\justifying\noindent\hspace{10mm} अत्र बहिर्योगे पञ्चमी पाणिनीया\footnote{\textcolor{red}{अपपरिबहिरञ्चवः पञ्चम्या} (पा॰सू॰~२.१.१२) इति ज्ञापनेन।} किन्तु \textcolor{red}{गुहां प्रति बहिः} इति प्रति\-योगे द्वितीयाऽपि पाणिनि\-सम्मता।\footnote{\textcolor{red}{प्रति} इत्यध्याहार्यमिति भावः। ततः \textcolor{red}{अभितः\-परितः\-समया\-निकषा\-हाप्रति\-योगेऽपि} (वा॰~२.३.२) इत्यनेन द्वितीया।} यद्वा \textcolor{red}{बहिर्देशे स्थितां गुहाम्‌} इत्यध्याहारेण द्वितीया।\footnote{यद्वा \textcolor{red}{ज्ञापकसिद्धं न सर्वत्र} इत्यनेन \textcolor{red}{करस्य करभो बहिः} (अ॰को॰~२.६.८०क) इतिवत्पञ्चमीतर\-विभक्तिः। विशेषं \pageref{sec:bahirvanasya}तमे पृष्ठे \ref{sec:bahirvanasya} \nameref{sec:bahirvanasya} इति प्रयोगस्य विमर्शे पश्यन्तु।}\end{sloppypar}
\section[मे]{मे}
\centering\textcolor{blue}{दासी तवाहं राजेन्द्र दर्शनार्थमिहागता।\nopagebreak\\
बहुवर्षसहस्त्राणि तप्तं मे दुश्चरं तपः॥}\nopagebreak\\
\raggedleft{–~अ॰रा॰~४.६.६१}\\
\begin{sloppypar}\hyphenrules{nohyphenation}\justifying\noindent\hspace{10mm} अत्र \textcolor{red}{मया तप्तम्‌} इति कर्मणि \textcolor{red}{क्त}\-प्रत्यय\-विधानादनुक्त\-कर्तरि तृतीया पाणिनीया\footnote{\textcolor{red}{कर्तृ\-करणयोस्तृतीया} (पा॰सू॰~२.३.१८) इत्यनेन।} किन्तु \textcolor{red}{मे} इति कृद्योगा षष्ठी \textcolor{red}{क्तस्य च वर्तमाने} (पा॰सू॰~२.३.६७) इति विहितत्वात्पाणिनि\-सम्मता।\footnote{\textcolor{red}{मति\-बुद्धि\-पूजार्थेभ्यश्च} (पा॰सू॰~३.२.१८८) इत्यनेन वर्तमान\-विवक्षायां क्तः। \textcolor{red}{च}कारेणान्यत्रापि। \textcolor{red}{चकारोऽनुक्त\-समुच्चयार्थः। “शीलितो रक्षितः क्षान्त आक्रुष्टो जुष्ट इत्यपि” इत्यादि} (वै॰सि॰कौ॰~३०८९) इति सिद्धान्त\-कौमुद्यां भट्टोजि\-दीक्षित\-महाभागाः। यद्वा \textcolor{red}{मे} इत्यस्य \textcolor{red}{दुश्चरं तपः} इत्यनेनान्वये कर्मणि सम्बन्ध\-विवक्षायां षष्ठी। \textcolor{red}{मे दुश्चरं तपो बहुवर्षसहस्त्राणि तप्तम्‌} इति भावः।}\end{sloppypar}
\section[रक्षोगणविनाशने]{रक्षोगणविनाशने}
\centering\textcolor{blue}{ब्रह्मणा प्रार्थिताः सर्वे रक्षोगणविनाशने।\nopagebreak\\
मायामानुषभावेन जाता लोकैकरक्षकाः॥}\nopagebreak\\
\raggedleft{–~अ॰रा॰~४.७.१८}\\
\begin{sloppypar}\hyphenrules{nohyphenation}\justifying\noindent\hspace{10mm} अत्र \textcolor{red}{रक्षोगण\-विनाशनाय} इति तादर्थ्ये चतुर्थी सामान्यतः पाणिनीया\footnote{\textcolor{red}{तादर्थ्ये चतुर्थी वाच्या} (वा॰~२.३.१३) इत्यनेन।} किन्तु \textcolor{red}{निमित्तात्कर्म\-संयोगे} (वा॰~२.३.३६) इत्यनेन सप्तमी। निमित्तमिह फलं तस्य कर्मणा सह योगे तद्वाचकात्सप्तमी यथा भाष्य\-वचनम्~–\end{sloppypar}
\centering\textcolor{red}{चर्मणि द्वीपिनं हन्ति दन्तयोर्हन्ति कुञ्जरम्।\nopagebreak\\
केशेषु चमरीं हन्ति सीम्नि पुष्कलको हतः॥}\nopagebreak\\
\raggedleft{–~भा॰पा॰सू॰~२.३.३६}\\
\begin{sloppypar}\hyphenrules{nohyphenation}\justifying\noindent\hspace{10mm} तथा \textcolor{red}{रक्षोगण\-विनाशनं} हि फलम्। कर्म \textcolor{red}{लोके जन्म\-ग्रहणम्‌}। अतोऽत्र सप्तमी।\end{sloppypar}
\section[वानरवृन्दान्]{वानरवृन्दान्‌}
\centering\textcolor{blue}{सुग्रीवः प्रेषयामास सीतायाः परिमार्गणे।\nopagebreak\\
अस्मान्वानरवृन्दान्वै महासत्त्वान्महाबलः॥}\nopagebreak\\
\raggedleft{–~अ॰रा॰~४.७.४३}\\
\begin{sloppypar}\hyphenrules{nohyphenation}\justifying\noindent \textcolor{red}{वृन्द}\-शब्दस्य नपुंसकतया \textcolor{red}{वानर\-वृन्दानि} इति पाणिनि\-सम्मतं किन्तु \textcolor{red}{लिङ्गमशिष्यं लोकाश्रयत्वाल्लिङ्गस्य} (भा॰पा॰सू॰~२.१.३६) इति वचनेन पुँल्लिङ्गेऽप्यनुकूलता।\end{sloppypar}
\vspace{2mm}
\centering ॥ इति किष्किन्धाकाण्डीयप्रयोगाणां विमर्शः ॥\nopagebreak\\
\vspace{4mm}
\pdfbookmark[2]{सुन्दरकाण्डम्‌}{Chap1Part2Kanda5}
\phantomsection
\addtocontents{toc}{\protect\setcounter{tocdepth}{2}}
\addcontentsline{toc}{subsection}{सुन्दरकाण्डीयप्रयोगाणां विमर्शः}
\addtocontents{toc}{\protect\setcounter{tocdepth}{0}}
\centering ॥ अथ सुन्दरकाण्डीयप्रयोगाणां विमर्शः ॥\nopagebreak\\
\section[देवतावृन्दः]{देवतावृन्दः}
\centering\textcolor{blue}{अब्रवीद्देवतावृन्दः कौतूहलसमन्वितः।\nopagebreak\\
गच्छ त्वं वानरेन्द्रस्य किञ्चिद्विघ्नं समाचर॥}\nopagebreak\\
\raggedleft{–~अ॰रा॰~५.१.११}\\
\begin{sloppypar}\hyphenrules{nohyphenation}\justifying\noindent\hspace{10mm} अत्रापि नपुंसक\-लिङ्गे \textcolor{red}{वृन्द}\-शब्दस्य प्रचलित\-प्रयोगात्\footnote{यथा \textcolor{red}{वृन्दान्युत्सार्यमाणानि दूरमुत्ससृजुस्तदा} (वा॰रा॰~६.११७.२१)  \textcolor{red}{वृन्दानि ददृशे तदा} (म॰भा॰~३.१५०.१९)  \textcolor{red}{अभ्यकीर्यन्त वृन्दानि} (म॰भा॰~४.१८.३२)  \textcolor{red}{मासादयैतद्रथसिंहवृन्दम्} (म॰भा॰~४.४९.४)  \textcolor{red}{संशप्तानि च वृन्दानि} (म॰भा॰~५.५४.५५)  \textcolor{red}{मेघवृन्दानि} (म॰भा॰~५.१७८.८४)  \textcolor{red}{सदश्ववृन्दानि} (म॰भा॰~६.५६.१६)  \textcolor{red}{अश्ववृन्दानि} (म॰भा॰~७.१७२.२६)  \textcolor{red}{रथवृन्दानि} (म॰भा॰~६.५९.११, ७.१६१.४९, ७.१७२.२६)  \textcolor{red}{यो वृन्‍दानि त्‍वरयति पथि श्राम्‍यतां प्रोषितानां} (मे॰दू॰~२.३६)  इत्यादिषु।} पुँल्लिङ्ग\-प्रयोगस्तु \textcolor{red}{लिङ्गमशिष्यं लोकाश्रयत्वाल्लिङ्गस्य} (भा॰पा॰सू॰~२.१.३६) इति भाष्य\-नियमेन पाणिनीय एव।\footnote{यद्वा शब्दमिममर्धर्चादिगणे पठित्वा \textcolor{red}{अर्धर्चाः पुंसि च} (पा॰सू॰~२.४.३१) इत्यनेन पुँल्लिङ्ग\-प्रयोगः समर्थनीयः। एवमेव \textcolor{red}{देववृन्दः सदा त्वां तु स्मृत्वा विजयतेऽसुरान्} (म॰भा॰~१.२११.१९)  इत्यादिषु बोध्यम्।} \textcolor{red}{यथोत्तरं मुनीनां प्रामाण्यम्‌} इति वचनात्।\end{sloppypar}
\section[मे]{मे}
\centering\textcolor{blue}{सर्वं कथय रामाय यथा मे जायते दया।\nopagebreak\\
मासद्वयावधि प्राणाः स्थास्यन्ति मम सत्तम॥}\nopagebreak\\
\raggedleft{–~अ॰रा॰~५.३.४०}\\
\begin{sloppypar}\hyphenrules{nohyphenation}\justifying\noindent\hspace{10mm} अत्र \textcolor{red}{मयि} इति प्रयोगस्त्वर्थानुकूलः किन्तु \textcolor{red}{उपरि} इत्यध्याहारे सम्बन्ध\-विवक्षायां \textcolor{red}{मे} इत्यपि। यद्वा \textcolor{red}{मे} इति \textcolor{red}{मह्यम्‌} इति चतुर्थी।\footnote{\textcolor{red}{तेमयावेकवचनस्य} (पा॰सू॰~८.१.२२) इत्यनेन।} सा च \textcolor{red}{मामुद्धर्तुं दया जायते मे हिताय वा दया जायताम्‌} इत्युभयथाऽपि तुमुन्कर्मणि\footnote{\textcolor{red}{क्रियार्थोपपदस्य च कर्मणि स्थानिनः} (वा॰~२.३.१४) इत्यनेन।} हित\-योगे\footnote{\textcolor{red}{हित\-योगे च} (वा॰~२.३.१३) इत्यनेन।} वा चतुर्थी।\end{sloppypar}
\section[देव्यै]{देव्यै}
\centering\textcolor{blue}{श्रुत्वा तद्वचनं देव्यै पूर्वरूपमदर्शयत्।\nopagebreak\\
मेरुमन्दरसङ्काशं रक्षोगणविभीषणम्॥}\nopagebreak\\
\raggedleft{–~अ॰रा॰~५.३.६४}\\
\begin{sloppypar}\hyphenrules{nohyphenation}\justifying\noindent\hspace{10mm} अत्र \textcolor{red}{देवीं विश्वासयितुं पूर्व\-रूपमदर्शयत्‌} इति \textcolor{red}{क्रियार्थोपपदस्य च कर्मणि स्थानिनः} (पा॰सू॰~२.३.१४) इत्यनेनाप्रयुज्यमान\-तुमुन्कर्मणि चतुर्थी। \textcolor{red}{हित}शब्दस्याध्याहारे हित\-योगा वा।\footnote{\textcolor{red}{हित\-योगे च} (वा॰~२.३.१३) इत्यनेन।}\end{sloppypar}
\section[महाप्रियम्]{महाप्रियम्‌}
\centering\textcolor{blue}{प्रासादरक्षिणः सर्वान्हत्वा तत्रैव तस्थिवान्।\nopagebreak\\
तच्छ्रुत्वा तूर्णमुत्थाय वनभङ्गं महाप्रियम्॥}\nopagebreak\\
\raggedleft{–~अ॰रा॰~५.३.७७}\\
\begin{sloppypar}\hyphenrules{nohyphenation}\justifying\noindent\hspace{10mm} अत्र \textcolor{red}{महत् अप्रियम् इति महाप्रियम्‌} इति विग्रहे समासः। \textcolor{red}{सन्महत्परमोत्तमोत्कृष्टाः पूज्यमानैः} (पा॰सू॰~२.१.६१) इति सूत्रेण कर्मधारयः। \textcolor{red}{आन्महतः समानाधिकरणजातीययोः} (पा॰सू॰~६.३.४६) इत्यनेनाऽकारादेशः। \textcolor{red}{महाप्रियम्‌} इति \textcolor{red}{महाखलः} इतिवत्।\end{sloppypar}
\section[राघवे]{राघवे}
\centering\textcolor{blue}{राक्षसीनां तर्जनैस्तत्सर्वं कथय राघवे।\nopagebreak\\
मयोक्तं देवि रामोऽपि त्वच्चिन्तापरिनिष्ठितः॥}\nopagebreak\\
\raggedleft{–~अ॰रा॰~५.५.४९}\\
\begin{sloppypar}\hyphenrules{nohyphenation}\justifying\noindent\hspace{10mm} अत्र \textcolor{red}{कथ्‌}\-धातु\-प्रयोगात् (\textcolor{red}{कथँ वाक्य\-प्रबन्धने} धा॰पा॰~१८५१) द्वितीयोचितैव किन्तु \textcolor{red}{राघवे शृण्वति कथय} इत्यध्याहारे सति सप्तम्यपि सिद्धान्तानुकूला।\footnote{\textcolor{red}{यस्य च भावेन भाव\-लक्षणम्‌} (पा॰सू॰~२.३.३७) इत्यनेन।}\end{sloppypar}
\vspace{2mm}
\centering ॥ इति सुन्दरकाण्डीयप्रयोगाणां विमर्शः ॥\nopagebreak\\
\vspace{4mm}
\pdfbookmark[2]{युद्धकाण्डम्‌}{Chap1Part2Kanda6}
\phantomsection
\addtocontents{toc}{\protect\setcounter{tocdepth}{2}}
\addcontentsline{toc}{subsection}{युद्धकाण्डीयप्रयोगाणां विमर्शः}
\addtocontents{toc}{\protect\setcounter{tocdepth}{0}}
\centering ॥ अथ युद्धकाण्डीयप्रयोगाणां विमर्शः ॥\nopagebreak\\
\section[हनूमन्तम्]{हनूमन्तम्‌}
\centering\textcolor{blue}{आययुश्चानुपूर्व्येण समुद्रं भीमनिःस्वनम्।\nopagebreak\\
अवतीर्य हनूमन्तं रामः सुग्रीवसंयुतः॥}\nopagebreak\\
\raggedleft{–~अ॰रा॰~६.१.४२}\\
\begin{sloppypar}\hyphenrules{nohyphenation}\justifying\noindent\hspace{10mm} अत्र लङ्कां प्रतिष्ठमानो भगवाञ्छ्रीरामो हनूमतोऽवतीर्य समुद्र\-वेलायां सैन्यं निवेशयति। राम\-विश्लेषस्य ध्रुवत्वेनावधि\-भूतत्वात् \textcolor{red}{हनूमन्तम्‌} इति द्वितीया पाणिनीय\-विरुद्धेव। \textcolor{red}{ध्रुवमपायेऽपादानम्‌} (पा॰सू॰~१.४.२४) इत्यनेनापादान\-सञ्ज्ञायां पञ्चम्येव। किन्तु \textcolor{red}{हनूमन्तं त्यक्त्वा अवतीर्य} इत्यध्याहारेण \textcolor{red}{त्यज्‌}\-कर्मतया (\textcolor{red}{त्यजँ हानौ} धा॰पा॰~९८६) द्वितीया। यद्वा \textcolor{red}{दुह्याच्‌} (वै॰सि॰कौ॰~५३९) इति परिगणनस्य शब्दरत्नादौ
खण्डनेनापादानस्याविवक्षायां \textcolor{red}{वृक्षं पुष्पं चिनोति}\footnote{\textcolor{red}{वृक्षमवचिनोति फलानि} (भा॰पा॰सू॰~१.४.५१) इति महाभाष्य उदाहृतः।} 
इत्यादिवद्द्वितीया।
\textcolor{red}{अकथितं च} (पा॰सू॰~१.४.५१) इत्यनेन कर्म\-सञ्ज्ञा\-बलाद्द्वितीया सङ्गतेति पाणिनीयता।\end{sloppypar}
\section[रघुनायकस्य]{रघुनायकस्य}
\centering\textcolor{blue}{यावन्न रामस्य शिताः शिलीमुखा लङ्कामभिव्याप्य शिरांसि रक्षसाम्।\nopagebreak\\
छिन्दन्ति तावद्रघुनायकस्य भोस्तां जानकीं त्वं प्रतिदातुमर्हसि॥}\nopagebreak\\
\raggedleft{–~अ॰रा॰~६.२.२४}\\
\begin{sloppypar}\hyphenrules{nohyphenation}\justifying\noindent\hspace{10mm} अत्र यद्यपि \textcolor{red}{दा}\-धातु\-योगे (\textcolor{red}{डुदाञ् दाने} धा॰पा॰~१०९१) चतुर्थी पाणिनीया किन्तु प्रतीत्युपसर्ग\-योजनतया विनिमय\-रूपेऽर्थे जाते सम्प्रदानताभावे षष्ठी\-साहित्यम्। \textcolor{red}{प्रतिदातुम्‌} इत्यनेन विनिमय\-द्योतनात्। यद्वा \textcolor{red}{रघुनायकस्य} इत्यस्य \textcolor{red}{जानकीम्‌} इत्यनेनान्वये दाम्पत्य\-भाव\-सम्बन्धे षष्ठी। \end{sloppypar}
\section[मे]{मे}
\centering\textcolor{blue}{मन्त्रिभिः सायुधैरस्मान् विवरे निहनिष्यति।\nopagebreak\\
तदाज्ञापय मे देव वानरैर्हन्यतामयम्॥}\nopagebreak\\
\raggedleft{–~अ॰रा॰~६.३.८}\\
\begin{sloppypar}\hyphenrules{nohyphenation}\justifying\noindent\hspace{10mm} अत्र सुग्रीवो विभीषणं प्रत्याशङ्कमानः प्राह \textcolor{red}{मे आज्ञापय}। \textcolor{red}{मां पोषयितुमाज्ञापय} इति चतुर्थी।\footnote{\textcolor{red}{क्रियार्थोपपदस्य च कर्मणि स्थानिनः} (पा॰सू॰~२.३.१४) इत्यनेन।} यद्वा कारकाणां बुद्धि\-कल्पितत्वात्\footnote{यथा \textcolor{red}{अयं योगः शक्योऽवक्तुम्। ... स बुद्ध्या सम्प्राप्य निवर्तयति। तत्र “ध्रुवमपायेऽपादानम्” इत्येव सिद्धम्‌} (भा॰पा॰सू॰~१.४.२५)। \textcolor{red}{अयमपि योगः शक्योऽवक्तुम्। ... स बुद्ध्या सम्प्राप्य निवर्तते। तत्र “ध्रुवमपायेऽपादानम्” इत्येव सिद्धम्‌} (भा॰पा॰सू॰~१.४.२६)। \textcolor{red}{अयमपि योगः शक्योऽवक्तुम्। ... स बुद्ध्या सम्प्राप्य निवर्तयति। तत्र “ध्रुवमपायेऽपादानम्” इत्येव सिद्धम्‌} (भा॰पा॰सू॰~१.४.२७)। \textcolor{red}{अयमपि योगः शक्योऽवक्तुम्। ... स बुद्ध्या सम्प्राप्य निवर्तते। तत्र “ध्रुवमपायेऽपादानम्” इत्येव सिद्धम्‌} (भा॰पा॰सू॰~१.४.२८)। \textcolor{red}{अयमपि योगः शक्योऽवक्तुम्‌} (भा॰पा॰सू॰~१.४.२९)। \textcolor{red}{अयमपि योगः शक्योऽवक्तुम्‌} (भा॰पा॰सू॰~१.४.३०)। \textcolor{red}{अयमपि योगः शक्योऽवक्तुम्‌} (भा॰पा॰सू॰~१.४.३१) इत्यादिषु स्पष्टम्।} \textcolor{red}{दा}\-धातुं (\textcolor{red}{डुदाञ् दाने} धा॰पा॰~१०९१) विनाऽपि सम्प्रदाने चतुर्थी।\footnote{\textcolor{red}{चतुर्थी सम्प्रदाने} (पा॰सू॰~२.३.१३) इत्यनेन।}\end{sloppypar}
\section[रघूणां पतये]{रघूणां पतये}
\centering\textcolor{blue}{नमोऽनन्ताय शान्ताय रामायामिततेजसे।\nopagebreak\\
सुग्रीवमित्राय च ते रघूणां पतये नमः॥}\nopagebreak\\
\raggedleft{–~अ॰रा॰~६.३.१८}\\
\begin{sloppypar}\hyphenrules{nohyphenation}\justifying\noindent\hspace{10mm} अत्र शरणागत\-विभीषणो रघु\-कुल\-भूषणं सुग्रीव\-लक्ष्मणाभिरामं रामं समीडानो \textcolor{red}{रघूणां पतये नमः} इति प्रयुङ्क्ते। अत्र समासाभावे \textcolor{red}{रघूणां पत्ये} इत्येव पाणिनीयं \textcolor{red}{पतिः समास एव} (पा॰सू॰~१.४.८) इत्यनेनासमास\-सञ्ज्ञा\-निषेधात् \textcolor{red}{घेर्ङिति} (पा॰सू॰~७.३.१११) इति गुणाभावेऽयादेशाभावे च \textcolor{red}{रघूणां पतये} इति कथम्। परं \textcolor{red}{पतिरिवाऽचरतीति पतिः}। आचारे \textcolor{red}{क्विप्}।\footnote{पति~\arrow \textcolor{red}{सर्वप्राति\-पदिकेभ्य आचारे क्विब्वा वक्तव्यः} (वा॰~३.१.११)~\arrow पति~क्विँप्~\arrow पति~व्~\arrow \textcolor{red}{वेरपृक्तस्य} (पा॰सू॰~६.१.६७)~\arrow पति~\arrow \textcolor{red}{सनाद्यन्ता धातवः} (पा॰सू॰~३.१.३२)~\arrow धातुसञ्ज्ञा~\arrow \textcolor{red}{क्विप् च} (पा॰सू॰~३.२.७६)~\arrow पति~क्विँप्~\arrow पति~व्~\arrow \textcolor{red}{वेरपृक्तस्य} (पा॰सू॰~६.१.६७)~\arrow पति~\arrow विभक्ति\-कार्यम्~\arrow पतिः।} ततः सर्वापहारि\-लोपे \textcolor{red}{लक्षण\-प्रतिपदोक्तयोः प्रतिपदोक्तस्यैव ग्रहणम्‌}\footnote{\textcolor{red}{सिद्धं तु लक्षण\-प्रतिपदोक्तयोः प्रतिपदोक्तस्यैव ग्रहणात्} (वा॰~६.२.२)।} इति परिभाषया लाक्षणिक\-पति\-शब्दे \textcolor{red}{पतिः समास एव} (पा॰सू॰~१.४.८) इति सूत्रं न प्रवर्तते। अतो घि\-सञ्ज्ञायां गुणेऽयादेशे \textcolor{red}{रघूणां पतये} इति।\footnote{\pageref{sec:patina}तमे पृष्ठे \ref{sec:patina} \nameref{sec:patina} इति प्रयोगस्य विमर्शमपि पश्यन्तु।} यद्वा \textcolor{red}{अपां पतये नमः}, \textcolor{red}{तस्क॑राणां॒ पत॑ये॒ नमः॒} (कृ॰य॰ तै॰सं॰~४.५.३.१) इति छान्दस\-प्रयोग इव विभीषणेनाप्यत्र
ब्राह्मणत्व\-प्रदिदर्शयिषया भगवतो रामचन्द्रस्य छन्दोमयत्वाच्छान्दसः प्रयोगः कृतः। \textcolor{red}{षष्ठीयुक्तश्छन्दसि वा} (पा॰सू॰~१.४.९) इति सूत्रेण। \textcolor{red}{षष्ठीयुक्तश्छन्दसि वा} इत्यत्र \textcolor{red}{वा} ग्रहणेन लोकेऽपि। अथवा षष्ठ्याऽलुक्समासे चोक्त\-प्रयोगः पाणिनीयः \textcolor{red}{बहुलं छन्दसि} (पा॰सू॰~२.४.७३) इति सूत्रेण \textcolor{red}{सुपो धातु\-प्रातिपदिकयोः} (पा॰सू॰~२.४.७१) इति सूत्राप्रवृत्तेः। अथवा \textcolor{red}{तत्पुरुषे कृति बहुलम्‌} (पा॰सू॰~६.३.१४) इति सूत्रेण पूर्व\-साधित\-कृदन्त\-\textcolor{red}{पति}\-शब्दे परेऽलुकि \textcolor{red}{रघूणां पतये} इत्यपि पाणिनीयम्।\footnote{मण्डूक\-प्लुत्या \textcolor{red}{हलदन्तात्सप्तम्याः संज्ञायाम्‌} (पा॰सू॰~६.३.९) इत्यस्मादनुवृत्तस्य \textcolor{red}{सप्तम्याः} इति पदस्य \textcolor{red}{तत्पुरुषे कृति बहुलम्‌} (पा॰सू॰~६.३.१४) इत्यत्र निवृत्तौ षष्ठ्या अप्यलुक्। यद्वा \textcolor{red}{बहुलम्‌} इत्यस्य ग्रहणेन षष्ठ्या अप्यलुक्।}\end{sloppypar}
\section[विभीषणे]{विभीषणे}
\centering\textcolor{blue}{सीतां प्रयच्छ रामाय राज्यं देहि विभीषणे।\nopagebreak\\
वनं याहि महाबाहो रम्यं मुनिगणाश्रयम्॥}\nopagebreak\\
\raggedleft{–~अ॰रा॰~६.६.४६}\\
\begin{sloppypar}\hyphenrules{nohyphenation}\justifying\noindent\hspace{10mm} अत्र चिर\-जीवित्वादाधार\-विवक्षयाऽधिकरण\-सप्तम्यपि पाणिनीया।\footnote{\textcolor{red}{अश्वत्थामा बलिर्व्यासो हनूमांश्च विभीषणः। कृपः परशुरामश्च सप्तैते चिरजीविनः॥} (आ॰रा॰~९.७.११८)।}\end{sloppypar}
\section[परैः]{परैः}
\centering\textcolor{blue}{निहन्मि त्वां दुरात्मानं मच्छासनपराङ्मुखम्।\nopagebreak\\
परैः किञ्चिद्गृहीत्वा त्वं भाषसे रामकिंकरः॥}\nopagebreak\\
\raggedleft{–~अ॰रा॰~६.७.२}\\
\begin{sloppypar}\hyphenrules{nohyphenation}\justifying\noindent\hspace{10mm} अत्र ग्रहणस्यावधि\-भूतत्वात् \textcolor{red}{परेभ्यः किञ्चिद्गृहीत्वा} इति वक्तव्यं तथाऽपि \textcolor{red}{परैः दीयमानम्‌} इत्यध्याहारे कर्तरि तृतीया। उताहो \textcolor{red}{परैः} इति करणे तृतीया। दान\-क्रियायां पर\-दानस्याऽपि करणत्वात्।\end{sloppypar}
\section[वायुसूनोः]{वायुसूनोः}
\centering\textcolor{blue}{हतस्यापि शरैस्तीक्ष्णैर्वायुसूनोः स्वतेजसा।\nopagebreak\\
व्यवर्धत पुनस्तेजो ननर्द च महाकपिः॥}\nopagebreak\\
\raggedleft{–~अ॰रा॰~६.६.२४}\\
\begin{sloppypar}\hyphenrules{nohyphenation}\justifying\noindent\hspace{10mm} अत्र कर्मणि सम्बन्ध\-विवक्षायां षष्ठी।\footnote{\textcolor{red}{षष्ठी शेषे} (पा॰सू॰~२.३.५०) इत्यनेन।}\end{sloppypar}
\section[लक्ष्मणाय]{लक्ष्मणाय}
\centering\textcolor{blue}{चिकित्सां कारयामास लक्ष्मणाय महात्मने।\nopagebreak\\
ततः सुप्तोत्थित इव बुद्ध्वा प्रोवाच लक्ष्मण॥}\nopagebreak\\
\raggedleft{–~अ॰रा॰~६.७.३७}\\
\begin{sloppypar}\hyphenrules{nohyphenation}\justifying\noindent\hspace{10mm} \textcolor{red}{लक्ष्मणस्य महात्मनः} इति सम्बन्ध\-विवक्षया षष्ठ्युचिता किन्तु \textcolor{red}{लक्ष्मणं महात्मानं जीवयितुम्‌} इत्यप्रयुज्यमान\-तुमुन्कर्मणि चतुर्थी।\footnote{\textcolor{red}{क्रियार्थोपपदस्य च कर्मणि स्थानिनः} (पा॰सू॰~२.३.१४) इत्यनेन।}\end{sloppypar}
\section[रघूत्तमे]{रघूत्तमे}
\centering\textcolor{blue}{इतः परं वा वैदेहीं प्रेषयस्व रघूत्तमे।\nopagebreak\\
विभीषणाय राज्यं तु दत्त्वा गच्छामहे वनम्॥}\nopagebreak\\
\raggedleft{–~अ॰रा॰~६.१०.५४}\\
\begin{sloppypar}\hyphenrules{nohyphenation}\justifying\noindent\hspace{10mm} अत्राप्याधार\-विवक्षायां सप्तमी।\footnote{\textcolor{red}{आधारोऽधिकरणम्} (पा॰सू॰~१.४.४५) \textcolor{red}{सप्तम्यधिकरणे च} (पा॰सू॰~२.३.३६) इत्याभ्याम्।} रामस्याधारत्वं पूर्वं निरूपितम्।\footnote{\pageref{sec:raaghave_3_5_36}तमे पृष्ठे \ref{sec:raaghave_3_5_36} \nameref{sec:raaghave_3_5_36} इति प्रयोगस्य विमर्शं पश्यन्तु।}\end{sloppypar}
\section[कारुण्यभाजनाः]{कारुण्यभाजनाः}
\centering\textcolor{blue}{वयं तु सात्त्विका देवा विष्णोः कारुण्यभाजनाः।\nopagebreak\\
भयदुःखादिभिर्व्याप्ताः संसारे परिवर्तिनः॥}\nopagebreak\\
\raggedleft{–~अ॰रा॰~६.११.८०}\\
\begin{sloppypar}\hyphenrules{nohyphenation}\justifying\noindent\hspace{10mm} \textcolor{red}{कारुण्य\-भाजनमेषाम्‌} इत्यर्शआदित्वादच्।\footnote{\textcolor{red}{अर्शआदिभ्योऽच्‌} (पा॰सू॰~५.२.१२७) इत्यनेन।} अथवा \textcolor{red}{भजन्त इति भजनाः} कर्तरि ल्युट्।\footnote{\textcolor{red}{कृत्यल्युटो बहुलम्‌} (पा॰सू॰~३.३.११३) इत्यनेन।} \textcolor{red}{भजना एव भाजनाः} इति प्रज्ञादित्वादण्।\footnote{\textcolor{red}{प्रज्ञादिभ्यश्च} (पा॰सू॰~५.४.३८) इत्यनेन।} \textcolor{red}{कारुण्यस्य भाजनाः} इति षष्ठी\-समासः।\footnote{\textcolor{red}{कृद्योगा च षष्ठी समस्यत इति वक्तव्यम्‌} (वा॰~२.२.८) इत्यनेन। यद्वा \textcolor{red}{कारुण्यस्य भाजनं येषाम्‌} इति बहुव्रीहिः।} अतो न लिङ्ग\-दोषः।\end{sloppypar}
\section[शिबिकोत्तमे]{शिबिकोत्तमे}
\centering\textcolor{blue}{सर्वाभरणसम्पन्नामारोप्य शिबिकोत्तमे।\nopagebreak\\
याष्टीकैर्बहुभिर्गुप्तां कञ्चुकोष्णीषिभिः शुभाम्॥}\nopagebreak\\
\raggedleft{–~अ॰रा॰~६.१२.७०}\\
\begin{sloppypar}\hyphenrules{nohyphenation}\justifying\noindent\hspace{10mm} अस्याप्यत्र नपुंसक\-लिङ्गे पाठात् \textcolor{red}{शिबिकोत्तमायाम्‌} इत्यनुक्त्वेदं कथमुक्तमिति न भ्रमितव्यम्।\footnote{\textcolor{red}{लिङ्गमशिष्यं लोकाश्रयत्वाल्लिङ्गस्य} (भा॰पा॰सू॰~२.१.३६) इत्यनेन लिङ्गं शिष्ट\-प्रयोगाधीनमिति भावः। वाल्मीकीय\-रामायणे भारते तु \textcolor{red}{शिबिका} इति स्त्रीलिङ्ग एव प्रयोगः।}\end{sloppypar}
\section[देवताभ्यः]{देवताभ्यः}
\centering\textcolor{blue}{पश्यतां सर्वलोकानां देवराक्षसयोषिताम्।\nopagebreak\\
प्रणम्य देवताभ्यश्च ब्राह्मणेभ्यश्च मैथिली॥}\nopagebreak\\
\raggedleft{–~अ॰रा॰~६.१२.८०}\\
\begin{sloppypar}\hyphenrules{nohyphenation}\justifying\noindent\hspace{10mm} \textcolor{red}{देवता अनुकूलयितुं प्रसादयितुं वा प्रणम्य} इति तुमुन्कर्मणि चतुर्थी।\footnote{\textcolor{red}{क्रियार्थोपपदस्य च कर्मणि स्थानिनः} (पा॰सू॰~२.३.१४) इत्यनेन। एवमेव \textcolor{red}{ ब्राह्मणेभ्यः} इत्यत्रापि बोध्यम्।}\end{sloppypar}
\section[गन्धर्वाप्सरसोरगाः]{गन्धर्वाप्सरसोरगाः}
\centering\textcolor{blue}{ततः शक्रः सहस्राक्षो यमश्च वरुणस्तथा।\nopagebreak\\
कुबेरश्च महातेजाः पिनाकी वृषवाहनः॥\\
ब्रह्मा ब्रह्मविदां श्रेष्ठो मुनिभिः सिद्धचारणैः।\nopagebreak\\
ऋषयः पितरः साध्या गन्धर्वाप्सरसोरगाः॥\\
एते चान्ये विमानाग्र्यैराजग्मुर्यत्र राघवः।\nopagebreak\\
अब्रुवन् परमात्मानं रामं प्राञ्जलयश्च ते॥}\nopagebreak\\
\raggedleft{–~अ॰रा॰~६.१३.१-३}\\
\begin{sloppypar}\hyphenrules{nohyphenation}\justifying\noindent\hspace{10mm} हलन्त\-\textcolor{red}{अप्सरस्‌} इति पाठे तु \textcolor{red}{गन्धर्वाप्सरउरगाः} इत्येव।\footnote{गन्धर्वाश्चाप्सरसश्चोरगाश्चेति गन्धर्वाप्सरउरगाः। गन्धर्व~जस् अप्सरस्~जस् उरग~जस्~\arrow \textcolor{red}{चार्थे द्वन्द्वः} (पा॰सू॰~२.२.२९)~\arrow \textcolor{red}{सुपो धातु\-प्रातिपदिकयोः} (पा॰सू॰~२.४.७१)~\arrow गन्धर्व~अप्सरस्~उरग~\arrow \textcolor{red}{अकः सवर्णे दीर्घः} (पा॰सू॰~६.१.१०१)~\arrow गन्धर्वाप्सरस्~उरग~\arrow \textcolor{red}{ससजुषो रुः} (पा॰सू॰~८.२.६६)~\arrow गन्धर्वाप्सररुँ~उरग~\arrow \textcolor{red}{भो भोभगोअघोअपूर्वस्य योऽशि} (पा॰सू॰~८.३.१७)~\arrow गन्धर्वाप्सरय्~उरग~\arrow \textcolor{red}{लोपः शाकल्यस्य} (पा॰सू॰~८.३.१९)~\arrow गन्धर्वाप्सर~उरग~\arrow गन्धर्वाप्सरउरग~\arrow विभक्तिकार्यम्~\arrow गन्धर्वाप्सरउरग~जस्~\arrow गन्धर्वाप्सरउरग~अस्~\arrow \textcolor{red}{प्रथमयोः पूर्वसवर्णः} (पा॰सू॰~६.१.१०२)~\arrow गन्धर्वाप्सरउरगास्~\arrow \textcolor{red}{ससजुषो रुः} (पा॰सू॰~८.२.६६)~\arrow \textcolor{red}{खरवसानयोर्विसर्जनीयः} (पा॰सू॰~८.३.१५)~\arrow गन्धर्वाप्सरउरगाः।} अजन्ते \textcolor{red}{गन्धर्वाप्सरसोरगाः}।\footnote{\textcolor{red}{सर्वे सान्ता अदन्ताः स्युः} इत्युक्तेः \textcolor{red}{अप्सर} इति शब्दोऽपि स्वीकरणीय इति भावः। \textcolor{red}{अप्सर}\-शब्दस्य प्रयोगः स्कन्द\-पुराणेऽवन्ती\-खण्डे रेवाखण्डे कुसुमेश्वर\-तीर्थ\-माहात्म्य\-वर्णने कृतोऽस्ति~– \textcolor{red}{वसन्तमासे कुसुमाकराकुले मयूरदात्यूह\-सुकोकिलाकुले। प्रनृत्य\-देवाप्सरगीत\-सङ्कुले प्रवाति वाते यमनैरृताकुले॥} (स्क॰पु॰~रे॰ख॰~१५०.१४)। अत्र \textcolor{red}{प्रनृत्य\-देवाप्सरगीत\-सङ्कुले} इत्यत्र \textcolor{red}{अप्सर} इत्येव शब्दः। एवमेवात्र \textcolor{red}{अप्सर}\-शब्द\-स्वीकारे \textcolor{red}{स}\-शब्दः सर्पे (\textcolor{red}{“सः स्याद्विष्णौ हरे सर्पे” इति भरतैकार्थ\-सङ्ग्रहः} इति शब्द\-कल्प\-द्रुमः) \textcolor{red}{उरग}\-शब्दश्च नागे स्वीकरणीयः। \textcolor{red}{देवगन्धर्व\-मानुषोरग\-राक्षसान्} (न॰उ॰~१.२९) \textcolor{red}{गन्धर्वोरग\-रक्षसाम्} (म॰स्मृ॰~३.१९६) इत्यादिषु \textcolor{red}{उरग}\-शब्दो नागार्थ इत्याप्टे\-कोशः। अत्र पर्याययोः कथं पृथगुपादानमिति न भ्रमितव्यम्। सर्प\-नागयोरीषदन्तरम्। अत एव गीतायां \textcolor{red}{सर्पाणामस्मि वासुकिः} (भ॰गी॰~१०.२८) इत्युक्त्वाऽपि \textcolor{red}{अनन्तश्चास्मि नागानाम्} (भ॰गी॰~१०.२९) इत्युक्तं भगवता। नागा नागलोक\-वासिनो मानव\-मुखाः सर्पजाति\-विशेषाः। \textcolor{red}{सर्पा एकशिरसः} (भ॰गी॰ रा॰भा॰~१०.२९) \textcolor{red}{नागा बहुशिरसः} (भ॰गी॰ रा॰भा॰~१०.२९) इति भगवन्तो रामानुजाचार्या गीताभाष्ये। एवं तर्हि गन्धर्व~जस् अप्सर~जस् स~जस् उरग~जस्~\arrow \textcolor{red}{चार्थे द्वन्द्वः} (पा॰सू॰~२.२.२९)~\arrow \textcolor{red}{सुपो धातु\-प्रातिपदिकयोः} (पा॰सू॰~२.४.७१)~\arrow गन्धर्व~अप्सर~स~उरग~\arrow \textcolor{red}{अकः सवर्णे दीर्घः} (पा॰सू॰~६.१.१०१)~\arrow गन्धर्वाप्सर~स~उरग~\arrow \textcolor{red}{आद्गुणः} (पा॰सू॰~६.१.८७)~\arrow गन्धर्वाप्सर~सोरग~\arrow गन्धर्वाप्सरसोरग~\arrow विभक्तिकार्यम्~\arrow गन्धर्वाप्सरसोरग~जस्~\arrow गन्धर्वाप्सरसोरग~अस्~\arrow \textcolor{red}{प्रथमयोः पूर्वसवर्णः} (पा॰सू॰~६.१.१०२)~\arrow गन्धर्वाप्सरसोरगास्~\arrow \textcolor{red}{ससजुषो रुः} (पा॰सू॰~८.२.६६)~\arrow \textcolor{red}{खरवसानयोर्विसर्जनीयः} (पा॰सू॰~८.३.१५)~\arrow गन्धर्वाप्सरसोरगाः।}
अथवा \textcolor{red}{द्वन्द्वाच्चुदषहान्तात्समाहारे} (पा॰सू॰~५.४.१०६) इत्यत्र \textcolor{red}{द्वन्द्वात्} इति योग\-विभागेन \textcolor{red}{टच्‌}\-प्रत्ययः।\footnote{योगविभागे \textcolor{red}{चुदषहान्तात् समाहारे} इत्यनयोर्मोषे सान्तादितरेतर\-द्वन्द्वादपि कुत्रचिदिति भावः।}
न चायमन्ते करोति। तर्हि पूर्वं \textcolor{red}{गन्धर्वाप्सरसाः} इति खण्डवाक्यमेकं पश्चात् \textcolor{red}{उरग}\-शब्देन समासः।\footnote{यद्वाऽत्र न समासोऽपि तु \textcolor{red}{गन्धर्वाप्सरसः} इति पृथक् \textcolor{red}{उरगाः} इति च पृथक्। ततः संहितायां \textcolor{red}{ससजुषो रुः} (पा॰सू॰~८.२.६६) इत्यनेन रुत्वे \textcolor{red}{भो भोभगोअघोअपूर्वस्य योऽशि} (पा॰सू॰~८.३.१७) इत्यनेन यत्वे \textcolor{red}{लोपः शाकल्यस्य} (पा॰सू॰~८.३.१९) इत्यनेन यलोपे गुणे प्राप्ते त्रिपादीत्वाल्लोप\-कार्यस्यासिद्धत्वे सामान्यतः \textcolor{red}{गन्धर्वाप्सरस उरगाः} इत्येव। परन्त्वत्र \textcolor{red}{न मु ने} (पा॰सू॰~८.२.३) इत्यनेन \textcolor{red}{पूर्वत्रासिद्धम्‌} (पा॰सू॰~८.२.१) इति सूत्रे निराकृते सिद्धे यलोपे \textcolor{red}{आद्गुणः} (पा॰सू॰~६.१.८७) इत्यनेन गुणे \textcolor{red}{गन्धर्वाप्सरसोरगाः} इति सिद्धम्। विस्तराय \pageref{sec:jaayeti_siiteti}तमे पृष्ठे \ref{sec:jaayeti_siiteti} \nameref{sec:jaayeti_siiteti} इति प्रयोगस्य विमर्शे पश्यन्तु।}\end{sloppypar}
\vspace{2mm}
\centering ॥ इति युद्धकाण्डीयप्रयोगाणां विमर्शः ॥\nopagebreak\\
\vspace{4mm}
\pdfbookmark[2]{उत्तरकाण्डम्‌}{Chap1Part2Kanda7}
\phantomsection
\addtocontents{toc}{\protect\setcounter{tocdepth}{2}}
\addcontentsline{toc}{subsection}{उत्तरकाण्डीयप्रयोगाणां विमर्शः}
\addtocontents{toc}{\protect\setcounter{tocdepth}{0}}
\centering ॥ अथोत्तरकाण्डीयप्रयोगाणां विमर्शः ॥\nopagebreak\\
\section[मम]{मम}
\centering\textcolor{blue}{श्रृणु राम यथा वृत्तं रावणे रावणस्य च।\nopagebreak\\
जन्म कर्म वरादानं सङ्क्षेपाद्वदतो मम॥}\nopagebreak\\
\raggedleft{–~अ॰रा॰~७.१.२४}\\
\begin{sloppypar}\hyphenrules{nohyphenation}\justifying\noindent\hspace{10mm} अत्र पञ्चम्या भवितव्यमासीत्\footnote{\textcolor{red}{अपादाने पञ्चमी} (पा॰सू॰~२.३.२८) इत्यनेन। अत्र शब्दविश्लेषादपादानत्वम्। यद्वा \textcolor{red}{आख्यातोपयोगे} (पा॰सू॰~१.४.२९) इत्यनेनापादानत्वम्।} किन्तु \textcolor{red}{वदतो मम} इति भाव\-लक्षणा षष्ठी \textcolor{red}{यस्य च भावेन भाव\-लक्षणम्‌} (पा॰सू॰~२.३.३७) इति सूत्रेण।\footnote{\textcolor{red}{दूरान्तिकार्थैः षष्ठ्यन्यतरस्याम्‌} (पा॰सू॰~२.३.३४) इत्यतः \textcolor{red}{षष्ठी} इत्यनुवर्त्य \textcolor{red}{षष्ठी चानादरे} (पा॰सू॰~२.३.३८) इत्यतः \textcolor{red}{षष्ठी} इत्यपकृष्य वाऽऽदरेऽपि \textcolor{red}{यस्य च भावेन भाव\-लक्षणम्‌} (पा॰सू॰~२.३.३७) इत्यनेन भावलक्षणा षष्ठीति भावः।}\end{sloppypar}
\section[कालस्य]{कालस्य}
\centering\textcolor{blue}{स तत्र सुचिरं कालमुवास पितृसम्मतः।\nopagebreak\\
कस्यचित्त्वथ कालस्य सुमाली नाम राक्षसः॥}\nopagebreak\\
\raggedleft{–~अ॰रा॰~७.१.४५}\\
\begin{sloppypar}\hyphenrules{nohyphenation}\justifying\noindent\hspace{10mm} \textcolor{red}{आगतस्य} इति \textcolor{red}{व्यतीतस्य} वेत्यध्याहारे भाव\-लक्षणा षष्ठी।\footnote{\textcolor{red}{दूरान्तिकार्थैः षष्ठ्यन्यतरस्याम्‌} (पा॰सू॰~२.३.३४) इत्यतः \textcolor{red}{षष्ठी} इत्यनुवर्त्य \textcolor{red}{षष्ठी चानादरे} (पा॰सू॰~२.३.३८) इत्यतः \textcolor{red}{षष्ठी} इत्यपकृष्य वाऽऽदरेऽपि \textcolor{red}{यस्य च भावेन भाव\-लक्षणम्‌} (पा॰सू॰~२.३.३७) इत्यनेन भावलक्षणा षष्ठीति भावः।}\end{sloppypar}
\section[जगत्त्रयम्]{जगत्त्रयम्‌}
\centering\textcolor{blue}{भगवन्ब्रूहि मे योद्धुं कुत्र सन्ति महाबलाः।\nopagebreak\\
योद्धुमिच्छामि बलिभिस्त्वं ज्ञाताऽसि जगत्त्रयम्॥}\nopagebreak\\
\raggedleft{–~अ॰रा॰~७.४.२}\\
\begin{sloppypar}\hyphenrules{nohyphenation}\justifying\noindent\hspace{10mm} अत्र \textcolor{red}{ज्ञाता} इति तृन्प्रत्ययान्तः।\footnote{\textcolor{red}{तृन्} (पा॰सू॰~३.२.१३५) इत्यनेन ताच्छील्ये ताद्धर्म्ये वा \textcolor{red}{तृन्‌}\-प्रत्ययः।} तृन्प्रत्ययान्त\-प्रयोगात् \textcolor{red}{जगत्त्रयम्‌} इत्यत्र द्वितीया पाणिनीया।\footnote{\textcolor{red}{कर्ता कटान्। वदिता जनापवादान्} (का॰वृ॰~२.३.६९) \textcolor{red}{कर्ता लोकान्} (वै॰सि॰कौ॰~६२७) इतिवत्।} \textcolor{red}{न लोकाव्यय\-निष्ठा\-खलर्थ\-तृनाम्‌} (पा॰सू॰~२.३.६९) इत्यनेन षष्ठी\-निषेधात्।\end{sloppypar}
\section[जगाम ऋषिवाटस्य]{जगाम ऋषिवाटस्य}
\centering\textcolor{blue}{वाल्मीकिरपि सङ्गृह्य गायन्तौ तौ कुशीलवौ।\nopagebreak\\
जगाम ऋषिवाटस्य समीपं मुनिपुङ्गवः॥}\nopagebreak\\
\raggedleft{–~अ॰रा॰~७.६.३६}\\
\begin{sloppypar}\hyphenrules{nohyphenation}\justifying\noindent\hspace{10mm} अत्र संहिताया अविवक्षणान्न सन्धिः।\footnote{यद्वा \textcolor{red}{ऋत्यकः} (पा॰सू॰~६.१.१२८) इत्यनेन शाकल\-प्रकृति\-भावः।}\end{sloppypar}
\section[मह्यम्]{मह्यम्‌}
\centering\textcolor{blue}{एवमेतन्महाप्राज्ञ यथा वदसि सुव्रत।\nopagebreak\\
प्रत्ययो जनितो मह्यं तव वाक्यैरकिल्बिषैः॥}\nopagebreak\\
\raggedleft{–~अ॰रा॰~७.७.३४}\\
\begin{sloppypar}\hyphenrules{nohyphenation}\justifying\noindent\hspace{10mm} \textcolor{red}{मां विश्वासयितुं} प्रत्ययो जनितो \textcolor{red}{मां तोषयितुं} वेति तुमुन्कर्मणि चतुर्थी।\footnote{\textcolor{red}{क्रियार्थोपपदस्य च कर्मणि स्थानिनः} (पा॰सू॰~२.३.१४) इत्यनेन।}\end{sloppypar}
\section[मुनये]{मुनये}
\centering\textcolor{blue}{तस्मै स मुनये रामः पूजां कृत्वा यथाविधि।\nopagebreak\\
पृष्ट्वाऽनामयव्यग्रो रामः पृष्टोऽथ तेन सः॥}\nopagebreak\\
\raggedleft{–~अ॰रा॰~७.८.१५}\\
\begin{sloppypar}\hyphenrules{nohyphenation}\justifying\noindent\hspace{10mm} \textcolor{red}{मुनिमनुकूलयितुम्‌} इति तुमुन्कर्मणि चतुर्थी \textcolor{red}{क्रियार्थोपपदस्य च कर्मणि स्थानिनः} (पा॰सू॰~२.३.१४) इत्यनेन।\end{sloppypar}
\vspace{2mm}
\centering ॥ इत्युत्तरकाण्डीयप्रयोगाणां विमर्शः ॥\nopagebreak\\
\vspace{4mm}
\centering इत्यध्यात्म\-रामायणेऽपाणिनीय\-प्रयोगाणां\-विमर्श\-नामके शोध\-प्रबन्धे प्रथमाध्याये द्वितीय\-परिच्छेदः।\nopagebreak\\
\vspace{4mm}
\centering\textcolor{blue}{\fontsize{16}{24}\selectfont इत्थं महीजारघुवीरलेखकः अध्यात्मरामायणमध्यवर्तिनः।\nopagebreak\\
अपाणिनीयान् स्वधिया विमृश्य वै अध्यायमेतं प्रथमं व्यमर्शयम्॥}\nopagebreak\\
\vspace{4mm}
\centering इत्यध्यात्म\-रामायणेऽपाणिनीय\-प्रयोगाणां\-विमर्श\-नामके शोध\-प्रबन्धे प्रथमोऽध्यायः।


% Nityanand Misra: LaTeX code to typeset a book in Sanskrit
% Copyright (C) 2016 Nityanand Misra
%
% This program is free software: you can redistribute it and/or modify it under
% the terms of the GNU General Public License as published by the Free Software
% Foundation, either version 3 of the License, or (at your option) any later
% version.
%
% This program is distributed in the hope that it will be useful, but WITHOUT
% ANY WARRANTY; without even the implied warranty of  MERCHANTABILITY or FITNESS
% FOR A PARTICULAR PURPOSE. See the GNU General Public License for more details.
%
% You should have received a copy of the GNU General Public License along with
% this program.  If not, see <http://www.gnu.org/licenses/>.

\renewcommand\chaptername{अथ द्वितीयोऽध्यायः}
\chapter[\texorpdfstring{कृत्तद्धितप्रकरणम्}{द्वितीयोऽध्यायः}]{कृत्तद्धितप्रकरणम्}
\vspace{-5mm}
\fontsize{16}{24}\selectfont\centering\textcolor{blue}{नत्वा नीलाम्बुदश्यामं रामं तामरसाननम्।\nopagebreak\\
शोधे गिरिधरः प्रेम्णा द्वितीयाध्यायमारभे॥}\nopagebreak\\
\vspace{4mm}
\fontsize{14}{21}\selectfont\begin{sloppypar}\hyphenrules{nohyphenation}\justifying\noindent\hspace{10mm} अथाध्यात्म\-रामायणे समागतान् कृत्तद्धित\-सम्बन्धिनोऽपाणिनीयान् प्रयोगाननु\-सन्दधे।\end{sloppypar}
\vspace{4mm}
\pdfbookmark[1]{प्रथमः परिच्छेदः}{Chap2Part1}
\phantomsection
\addtocontents{toc}{\protect\setcounter{tocdepth}{1}}
\addcontentsline{toc}{section}{प्रथमः परिच्छेदः}
\addtocontents{toc}{\protect\setcounter{tocdepth}{0}}
\centering ॥ अथ द्वितीयाध्याये प्रथमः परिच्छेदः ॥\nopagebreak\\
\vspace{4mm}
\pdfbookmark[2]{बालकाण्डम्}{Chap2Part1Kanda1}
\phantomsection
\addtocontents{toc}{\protect\setcounter{tocdepth}{2}}
\addcontentsline{toc}{subsection}{बालकाण्डीयप्रयोगाणां विमर्शः}
\addtocontents{toc}{\protect\setcounter{tocdepth}{0}}
\centering ॥ अथ बालकाण्डीयप्रयोगाणां विमर्शः ॥\nopagebreak\\
\section[संविष्टम्]{संविष्टम्}
\centering\textcolor{blue}{कैलासाग्रे कदाचिद्रविशतविमले मन्दिरे रत्नपीठे\nopagebreak\\
संविष्टं ध्याननिष्ठं त्रिनयनमभयं सेवितं सिद्धसङ्घैः।\nopagebreak\\
देवी वामाङ्कसंस्था गिरिवरतनया पार्वती भक्तिनम्रा\nopagebreak\\
प्राहेदं देवमीशं सकलमलहरं वाक्यमानन्दकन्दम्॥}\nopagebreak\\
\raggedleft{–~अ॰रा॰~१.१.६}\\
\begin{sloppypar}\hyphenrules{nohyphenation}\justifying\noindent\hspace{10mm} अत्राध्यात्म\-रामायणस्य प्रारम्भ\-स्थितिं प्रस्तौति।\footnote{‘व्यासः’ इति शेषः।} कैलास\-गिरौ संविष्टं भगवन्तं शिवं पार्वती पृच्छति। अत्र सम्पूर्वकात् \textcolor{red}{विश्‌}\-धातोः (\textcolor{red}{विशँ प्रवेशने} धा॰पा॰~१६३९) कर्तरि \textcolor{red}{गत्यर्थाकर्मक\-श्लिष\-शीङ्स्थास\-वस\-जन\-रुह\-जी\-र्यतिभ्यश्च} (पा॰सू॰~३.४.७२) इत्यनेन \textcolor{red}{क्त}\-प्रत्ययः। \textcolor{red}{लशक्वतद्धिते} (पा॰सू॰~१.३.८) इत्यनेनेत्सञ्ज्ञायां लोपे\footnote{\textcolor{red}{तस्य लोपः} (पा॰सू॰~१.३.९) इत्यनेन} \textcolor{red}{व्रश्च\-भ्रस्ज\-सृज\-मृज\-यज\-राज\-भ्राजच्छशां षः} (पा॰सू॰~८.२.३६) इत्यनेन षत्वे \textcolor{red}{ष्टुना ष्टुः} (पा॰सू॰~८.४.४१) इत्यनेन ष्टुत्वे विभक्ति\-कार्ये \textcolor{red}{संविष्टम्} इति। सम्पूर्वकस्य \textcolor{red}{विश्‌}\-धातोः शयनमर्थः।\footnote{यथा~– \textcolor{red}{पश्चादग्नेरुदगग्रेषु दर्भेषु प्राक्शिराः संविशति} (गो॰गृ॰सू॰~२.६.१०) \textcolor{red}{क्रमेण सुप्तामनु संविवेश} (र॰वं॰~२.२४) \textcolor{red}{चरमं संविशति या प्रथमं प्रतिबुध्यते} (म॰भा॰~२.८८.३६) \textcolor{red}{आन्याय्यादुत्थानादान्याय्याच्च संवेशनादेषोऽद्यतनः कालः} (का॰वृ॰~१.२.५७) इत्यादिषु। अवलम्बः~– चारुदेवशास्त्रिकृता \textcolor{red}{उपसर्गार्थचन्द्रिका}।} तर्ह्युपवेशन\-रूपोऽर्थः कथमिति चेत्। \textcolor{red}{समुपविष्टम्} इत्येवात्र।
\textcolor{red}{उप}\-उपसर्गस्य लोपः।\footnote{\textcolor{red}{विनाऽपि प्रत्ययं पूर्वोत्तर\-पद\-लोपो वक्तव्यः} (वा॰~५.३.८३) इत्यनेन।} अत एव \textcolor{red}{संविष्टम्} इत्यस्य हि \textcolor{red}{समुपविष्टम्} इत्यर्थः।\footnote{अन्यत्रापि दृश्यते। \textcolor{red}{जघनार्धेन च पशुरुच्च तिष्ठति सं च विशति} (श॰ब्रा॰~८.२.४.२०, संविशति = निषीदति)। \textcolor{red}{पयटेत्कीटवद्भूमिं वर्षास्वेकत्र संविशेत्} (ल॰वि॰स्मृ॰~४.५, संविशेत् = तिष्ठेत् = वसेत्)। \textcolor{red}{अन्येनोत्थाप्यतेऽन्येन तथा संवेश्यते जरी} (वि॰पु॰~६.५.३३, संवेश्यते = उपवेश्यते)। अवलम्बः~– चारुदेवशास्त्रिकृता \textcolor{red}{उपसर्गार्थचन्द्रिका}।}\end{sloppypar}
\section[पुरा रामायणे रामः]{पुरा रामायणे रामः}
\centering\textcolor{blue}{पुरा रामायणे रामो रावणं देवकण्टकम्।\nopagebreak\\
हत्वा रणे रणश्लाघी सपुत्रबलवाहनम्॥\\
सीतया सह सुग्रीवलक्ष्मणाभ्यां समन्वितः।\nopagebreak\\
अयोध्यामगमद्रामो हनूमत्प्रमुखैर्वृतः॥}\nopagebreak\\
\raggedleft{–~अ॰रा॰~१.१.२६-२७}\\
\begin{sloppypar}\hyphenrules{nohyphenation}\justifying\noindent\hspace{10mm} अत्र भूतभावनो भगवान् शिवोऽध्यात्म\-रामायण\-कथायाः प्रस्तावं करोति यत् \textcolor{red}{पुरा रामायणे} श्रीरामो रावणं हत्वाऽयोध्यामगमत्। अत्र \textcolor{red}{रामायणे} इति हि कस्य विशेषणं किमभिप्रायकं वाऽत्र सप्तमी वा किन्निमित्तिका। यदि चेदाधारे सप्तमी तदा रामायणं पुस्तकमत्र राम\-निरूपिताऽऽधारता कथं सम्भवा। यदि चेल्लक्षणया रामायण\-लक्षितस्तस्या मूलमन्वयानुप\-पत्तिस्तात्पर्यानुप\-पत्तिश्च।\footnote{\textcolor{red}{अन्वयाद्यनुपपत्ति\-प्रतिसन्धानञ्च लक्षणाबीजम्। वस्तुतस्तु तात्पर्यानुपपत्ति\-सन्धानमेव तद्बीजम्} (प॰ल॰म॰~२३–२४)।} यथा \textcolor{red}{गङ्गायां घोषः} इत्यत्र तात्पर्यानुपपत्तिः। यतो हि घोष आभीरपल्ली। सा च भगीरथ\-रथ\-खातावच्छिन्न\-जल\-प्रवाहे सम्भवा नहि। अतोऽन्वयानुप\-पत्तिरपि तात्पर्यानुपपत्तिश्चेति चेत्सामीप्य\-सम्बन्धेन गङ्गा\-पदस्य गङ्गा\-तीरे लक्षणा। तथैवेत्यत्रापि \textcolor{red}{रामायणे} तात्पर्यानुपपत्तेः \textcolor{red}{रामायण}\-पदस्य \textcolor{red}{रामायणोपलक्षिते काले} लक्षणा। इयं च जघन्या वृत्तिर्वैयाकरण\-मते। एतस्या अस्तित्वमपि नास्तीति चेच्छक्यतावच्छेदकारोप इति चेत्।\footnote{\textcolor{red}{तन्न। सति तात्पर्ये सर्वे सर्वार्थवाचका इति भाष्याल्लक्षणाया अभावाद्वृत्ति\-द्वयावच्छेदक\-द्वय\-कल्पने गौरवात्। जघन्य\-वृत्ति\-कल्पनाया अन्याय्यत्वाच्च} (प॰ल॰म॰~२७)।} 
अलं गुरु\-गुरु\-कल्पनया। \textcolor{red}{रामायणमस्त्यस्मिन् स रामायणः} इति विग्रहे प्रथमान्ताद्रामायण\-शब्दात् \textcolor{red}{अर्श\-आदिभ्योऽच्} (पा॰सू॰~५.२.१२७) इत्यनेन \textcolor{red}{अच्} प्रत्यये \textcolor{red}{यचि भम्} (पा॰सू॰~१.४.१८) इत्यनेन भ\-सञ्ज्ञायां \textcolor{red}{यस्येति च} (पा॰सू॰~६.४.१४८) इत्यनेनाकार\-लोपे \textcolor{red}{रामायणः} तस्मिन् \textcolor{red}{रामायणे} इति साधु। अर्थाद्रामायणे काले।\end{sloppypar}
\section[हनूमन्तम्]{हनूमन्तम्}
\centering\textcolor{blue}{दृष्ट्वा तदा हनूमन्तं प्राञ्जलिं पुरतः स्थितम्।\nopagebreak\\
कृतकार्यं निराकाङ्क्षं ज्ञानापेक्षं महामतिम्॥}\nopagebreak\\
\raggedleft{–~अ॰रा॰~१.१.२९}\\
\centering\textcolor{blue}{ततो रामः स्वयं प्राह हनूमन्तमुपस्थितम्।\nopagebreak\\
शृणु तत्त्वं प्रवक्ष्यामि ह्यात्मानात्मपरात्मनाम्॥}\nopagebreak\\
\raggedleft{–~अ॰रा॰~१.१.४४}\\
\begin{sloppypar}\hyphenrules{nohyphenation}\justifying\noindent\hspace{10mm} \textcolor{red}{प्रशस्तो हनुर्यस्य स हनुमान्} इति विग्रहे \textcolor{red}{तदस्यास्त्यस्मिन्निति मतुप्} (पा॰सू॰~५.२.९४) इत्यनेन मतुप्प्रत्यये विभक्ति\-कार्ये \textcolor{red}{हनुमन्तम्} इति पाणिनीयम्। \textcolor{red}{हनूमन्तम्} इति हि कथम्। \textcolor{red}{ऊङुतः} (पा॰सू॰~४.१.६६) इत्यनेनोङि कृते दीर्घे \textcolor{red}{हनू} इति।\footnote{छान्दसत्वाद्बाहुलकाद्वा नोपधादमनुष्य\-जातेरप्यूङित्यर्थः। अत्र वाचस्पत्य\-काराः – \textcolor{red}{हनु(नू) पुंस्त्री॰~हन-उन् स्त्रीत्वे वा ऊङ्}। शब्दकल्पद्रुम\-काराश्च – \textcolor{red}{हनूः, स्त्री, हनु + पक्षे ऊञ्। हनुः। इत्यमरटीकायां भरतः}। भाष्ये तु \textcolor{red}{ऊङ्प्रकरणेऽप्राणिजातेश्चारज्ज्वादीनाम्} (वा॰~४.१.६६) इति वार्त्तिकानन्तरं \textcolor{red}{हनु}शब्दो रज्ज्वादिगणे पठितः। परन्तु दृश्यते हि \textcolor{red}{हनू}\-रूपमार्षग्रन्थेषु। यथा मूल\-रामायणे \textcolor{red}{भरतस्यान्तिकं रामो हनूमन्तं व्यसर्जयत्} (वा॰रा॰~१.१.८५)। अत्र गोविन्दराजाश्च~– \textcolor{red}{हनूशब्द ऊकारान्तोऽप्यस्ति} (वा॰रा॰ भू॰टी॰~१.१.८५)।} ततो मतुप्प्रत्यये \textcolor{red}{हनूमान्} इति। विभक्ति\-कार्ये \textcolor{red}{हनूमन्तम्}। यद्वा \textcolor{red}{अन्येषामपि दृश्यते} (पा॰सू॰~६.३.१३८) इत्यनेन दीर्घे \textcolor{red}{हनूमन्तम्} इति।\end{sloppypar}
\section[आनन्दम्]{आनन्दम्}
\centering\textcolor{blue}{आनन्दं निर्मलं शान्तं निर्विकारं निरञ्जनम्।\nopagebreak\\
सर्वव्यापिनमात्मानं स्वप्रकाशमकल्मषम्॥}\nopagebreak\\
\raggedleft{–~अ॰रा॰~१.१.३३}\\
\begin{sloppypar}\hyphenrules{nohyphenation}\justifying\noindent\hspace{10mm} अत्र \textcolor{red}{आनन्द}\-शब्दस्य \textcolor{red}{आत्मानम्} इत्यनेन सामानाधिकरण्यं कथम्। यतो ह्यात्माऽऽनन्दस्याधिकरणम्। तथा चात्र कथं न षष्ठीति चेत्। अद्वैत\-वेदान्त\-मत आत्मन आनन्द\-रूपत्वात्। न च \textcolor{red}{आनन्दमयोऽभ्यासात्} (ब्र॰सू॰~१.१.१३) इति ब्रह्मसूत्र आनन्द\-स्वरूपत्वं न प्रत्यपादि। अथ कस्मिन्नर्थे \textcolor{red}{मयट्}। किं \textcolor{red}{तस्य विकारः} (पा॰सू॰~४.३.१३४) इत्यनेन विकारार्थे। नहि तावदात्मा निर्विकारः \textcolor{red}{आनन्दं निर्मलं शान्तमविकारमकल्मषम्} इत्यत्रैवोक्तत्वात् \textcolor{red}{अविकार्योऽयमुच्यते} (भ॰गी॰~२.२५) इति गीतायामप्युक्तत्वादिति चेत्। \textcolor{red}{तत्प्रकृत\-वचने मयट्} (पा॰सू॰~५.४.२१) इत्यनेन प्राचुर्यार्थे। अद्वैतवादिनां मते स्वरूपे। अतः सामानाधिकरण्यं प्राचुर्यार्थे मयटि। परमात्माऽऽनन्दस्य निलयमित्यपेक्षायाम् \textcolor{red}{आनन्दोऽस्त्यस्मिन्} इत्यर्श\-आद्यच्।\footnote{\textcolor{red}{अर्शआदिभ्योऽच्} (पा॰सू॰~५.२.१२७) इत्यनेन।} यद्वा \textcolor{red}{आनन्द इवाऽचरतीत्यानन्दति}।\footnote{आनन्द~\arrow \textcolor{red}{सर्वप्राति\-पदिकेभ्य आचारे क्विब्वा वक्तव्यः} (वा॰~३.१.११)~\arrow आनन्द~क्विँप्~\arrow आनन्द~व्~\arrow \textcolor{red}{वेरपृक्तस्य} (पा॰सू॰~६.१.६७)~\arrow आनन्द~\arrow \textcolor{red}{सनाद्यन्ता धातवः} (पा॰सू॰~३.१.३२)~\arrow धातुसञ्ज्ञा~\arrow \textcolor{red}{शेषात्कर्तरि परस्मैपदम्} (पा॰सू॰~१.३.७८)~\arrow \textcolor{red}{वर्तमाने लट्} (पा॰सू॰~३.२.१२३)~\arrow आनन्द~लट्~\arrow आनन्द~तिप्~\arrow आनन्द~ति~\arrow \textcolor{red}{कर्तरि शप्‌} (पा॰सू॰~३.१.६८)~\arrow आनन्द~शप्~ति~\arrow आनन्द~अ~ति~\arrow \textcolor{red}{अतो गुणे} (पा॰सू॰~६.१.९७)~\arrow आनन्द~ति~\arrow आनन्दति।} \textcolor{red}{आनन्दतीत्यानन्दः}।\footnote{आनन्द~\arrow धातुसञ्ज्ञा (पूर्ववत्)~\arrow \textcolor{red}{क्विप् च} (पा॰सू॰~३.२.७६)~\arrow आनन्द~क्विँप्~\arrow आनन्द~व्~\arrow \textcolor{red}{वेरपृक्तस्य} (पा॰सू॰~६.१.६७)~\arrow आनन्द~\arrow विभक्तिकार्यम्~\arrow आनन्दः। यद्वा \textcolor{red}{नन्दि\-ग्रहि\-पचादिभ्यो ल्युणिन्यचः} (पा॰सू॰~३.१.१३४) इत्यनेन कर्तर्यचि। आनन्द~\arrow \textcolor{red}{धातुसञ्ज्ञा} (पूर्ववत्)~\arrow \textcolor{red}{नन्दि\-ग्रहि\-पचादिभ्यो ल्युणिन्यचः} (पा॰सू॰~३.१.१३४)~\arrow आनन्द~अच्~\arrow आनन्द~अ~\arrow \textcolor{red}{अतो गुणे} (पा॰सू॰~६.१.९७)~\arrow आनन्द~\arrow विभक्तिकार्यम्~\arrow आनन्दः।} कर्तरि क्विपि सर्वापहारि\-लोपे \textcolor{red}{कृत्तद्धित\-समासाश्च} (पा॰सू॰~१.२.४६) इत्यनेन प्रातिपदिक\-सञ्ज्ञायां \textcolor{red}{स्वौ\-जसमौट्छष्टा\-भ्याम्भिस्ङे\-भ्याम्भ्यस्ङसि\-भ्याम्भ्यस्ङसोसाम्ङ्योस्सुप्} (पा॰सू॰~४.१.२) इत्यनेन \textcolor{red}{अमि} विभक्तौ \textcolor{red}{अमि पूर्वः} (पा॰सू॰~६.१.१०७) इत्यनेन पूर्व\-रूपे \textcolor{red}{आनन्दम्} इति पाणिनीयमेव।\end{sloppypar}
\section[अविकारिणि]{अविकारिणि}
\centering\textcolor{blue}{साभासबुद्धेः कर्तृत्वमविच्छिन्नेऽविकारिणि।\nopagebreak\\
साक्षिण्यारोप्यते भ्रान्त्या जीवत्वं च तथा बुधैः॥}\nopagebreak\\
\raggedleft{–~अ॰रा॰~१.१.४७}\\
\begin{sloppypar}\hyphenrules{nohyphenation}\justifying\noindent\hspace{10mm} अत्र \textcolor{red}{न विकार इत्यविकारः}। \textcolor{red}{अविकारोऽस्त्यस्मिन्निति अविकारी} इति विग्रहे \textcolor{red}{अत इनिठनौ} (पा॰सू॰~५.२.११५) इत्यनेन \textcolor{red}{इनि}\-प्रत्यये विभक्ति\-लोपे भ\-सञ्ज्ञायां \textcolor{red}{यस्येति च} (पा॰सू॰~६.४.१४८) इत्यनेनाकार\-लोपे पुनः सप्तमी\-\textcolor{red}{ङि}\-विभक्तौ \textcolor{red}{लशक्वतद्धिते} (पा॰सू॰~१.३.८) इत्यनेन ङकारेत्सञ्ज्ञायामनुबन्ध\-लोपे \textcolor{red}{अट्कुप्वाङ्नुम्व्यवायेऽपि} (पा॰सू॰~८.४.२) इत्यनेन णत्वे \textcolor{red}{अविकारिणि}। अत्र \textcolor{red}{न कर्मधारयान्मत्वर्थीयो बहुव्रीहिश्चेत्तदर्थ\-प्रतिपत्ति\-करः}\footnote{मूलं मृग्यम्।} इति वचनेन हि मत्वर्थीय\-निषेधात् \textcolor{red}{इनिः} अपाणिनीय इति चेत्। अत्र कर्मधारयो नास्ति तदा कथमुक्त\-नियमस्य प्रसरः। अत्र कर्मधारयः सकल\-समासोपलक्षणमिति चेत्। अत्र न \textcolor{red}{इनिः} किन्तु \textcolor{red}{न विकर्तुं तच्छीलः} इति विग्रहे \textcolor{red}{वि}\-पूर्वकात् \textcolor{red}{कृ}\-धातोः (\textcolor{red}{डुकृञ् करणे} धा॰पा॰~१४७२) \textcolor{red}{सुप्यजातौ णिनिस्ताच्छील्ये} (पा॰सू॰~३.२.७८) इत्यनेन णिनिः। णकारानुबन्धे कार्ये \textcolor{red}{अचो ञ्णिति} (पा॰सू॰~७.२.११५) इत्यनेन वृद्धौ रपरत्वे\footnote{\textcolor{red}{उरण् रपरः} (पा॰सू॰~१.१.५१) इत्यनेन।} सप्तम्येकवचने णत्वे\footnote{\textcolor{red}{अट्कुप्वाङ्नुम्व्यवायेऽपि} (पा॰सू॰~८.४.२) इत्यनेन।} \textcolor{red}{अविकारिणि} इति पाणिनीयमेव।\end{sloppypar}
\section[जगत्येन]{जगत्येन}
\centering\textcolor{blue}{मायया गुणमय्या त्वं सृजस्यवसि लुम्पसि।\nopagebreak\\
जगत्येन न ते लेप आनन्दानुभवात्मनः॥}\nopagebreak\\
\raggedleft{–~अ॰रा॰~१.२.१५}\\
\begin{sloppypar}\hyphenrules{nohyphenation}\justifying\noindent\hspace{10mm} अत्र रावण\-कुकृत्य\-जन्य\-पाद\-भार\-पीडितया वसुमत्या सह सकल\-देव\-पुरःसरं क्षीर\-सागरं गतो ब्रह्मा भगवन्तं स्तुवन्नाह यत्त्रिगुणमय्या मायया हेतु\-भूतयोपलक्षितस्त्वं संसारं सृजसि पालयसि नाशयसि किन्तु त्वं जागतिक\-पदार्थेन न लिप्यसे। अत्र \textcolor{red}{जागतिकेन} इत्येव प्रयोक्तव्यं यतो हि \textcolor{red}{जगति भवं जागतिकम्} इति विग्रहे \textcolor{red}{तत्र भवः} (पा॰सू॰~४.३.५३) इत्यनेन \textcolor{red}{ठक्} प्रत्यये \textcolor{red}{तद्धितेष्वचामादेः} (पा॰सू॰~७.२.११७) इति वृद्धौ 
\textcolor{red}{इसुसुक्तान्तात्कः} (पा॰सू॰~७.३.५१) इत्यनेन कादेशे विभक्ति\-कार्ये \textcolor{red}{जागत्केन} इत्येव पाणिनीयम्। एवं \textcolor{red}{जगत्येन} इति विमृश्यते। तथा च \textcolor{red}{आत्मनो जगदिच्छतीति जगत्यति} इति विग्रहे \textcolor{red}{सुप आत्मनः क्यच्} (पा॰सू॰~३.१.८) इत्यनेन \textcolor{red}{क्यच्} प्रत्यये \textcolor{red}{लशक्वतद्धिते} (पा॰सू॰~१.३.८) इत्यनेनेत्सञ्ज्ञायामनु\-बन्ध\-लोपे \textcolor{red}{वर्तमाने लट्} (पा॰सू॰~३.२.१२३) इति लट्। ततः तिपि शपि \textcolor{red}{जगत्यति}।\footnote{\textcolor{red}{जगत्यति} इति भाष्ये \textcolor{red}{तुग्यणेकादेश\-गुण\-वृद्ध्यौत्त्व\-दीर्घत्वेत्वमुमेत्त्त्वरी\-विधिभ्यः} (वा॰~१.४.२) इति वार्तिक उदाहृतम्।} ततः \textcolor{red}{जगत्यतीति जगत्यः} इति विग्रहे \textcolor{red}{नन्दिग्रहिपचादिभ्यो ल्युणिन्यचः} (पा॰सू॰~३.१.१३४) इत्यनेन \textcolor{red}{अच्} प्रत्यये विभक्ति\-कार्ये तृतीयैक\-वचने \textcolor{red}{टा}\-विभक्तौ \textcolor{red}{टाङसिङसामिनात्स्याः} (पा॰सू॰~७.१.१२) इत्यनेनेनादेशे गुणे\footnote{\textcolor{red}{आद्गुणः} (पा॰सू॰~६.१.८७) इत्यनेन।} \textcolor{red}{जगत्येन} इति पाणिनीयमेव। यद्वा \textcolor{red}{जगत्संसारं याति गच्छति} इति विग्रहे जगदुपपदे \textcolor{red}{या}\-धातोः (\textcolor{red}{या प्रापणे} धा॰पा॰~१०४९) \textcolor{red}{आतोऽनुपसर्गे कः} (पा॰सू॰~३.२.३) इत्यनेन \textcolor{red}{क}\-प्रत्यये ककारस्यानुबन्ध\-कार्ये \textcolor{red}{आतो धातोः} (पा॰सू॰~६.४.१४०) इत्यनेनाऽकार\-लोपेऽपदत्वाज्जश्त्वाभावे\footnote{\textcolor{red}{पृषोदरादीनि यथोपदिष्टम्} (पा॰सू॰~६.३.१०९) इति सूत्रेणापदत्वम्। यद्वा \textcolor{red}{अयस्मयादीनि च्छन्दसि} (पा॰सू॰~१.४.२०) इति सूत्रेण छान्दसभत्वम्।} \textcolor{red}{जगत्यः} तेन \textcolor{red}{जगत्येन}।
यद्वा \textcolor{red}{जगदाचष्ट इति जगतयति}।\footnote{जगत्~\arrow \textcolor{red}{तत्करोति तदाचष्टे} (धा॰पा॰ ग॰सू॰)~\arrow जगत्~णिच्~\arrow जगत्~इ~\arrow जगति~\arrow \textcolor{red}{सनाद्यन्ता धातवः} (पा॰सू॰~३.१.३२)~\arrow \textcolor{red}{शेषात्कर्तरि परस्मैपदम्} (पा॰सू॰~१.३.७८)~\arrow \textcolor{red}{वर्तमाने लँट्} (पा॰सू॰~३.२.१२३)~\arrow जगति~लट्~\arrow जगति~तिप्~\arrow \textcolor{red}{कर्तरि शप्‌} (पा॰सू॰~३.१.६८)~\arrow जगति~शप्~तिप्~\arrow जगति~अ~ति~\arrow \textcolor{red}{सार्वधातुकार्धधातुकयोः} (पा॰सू॰~७.३.८४)~\arrow जगते~अ~ति~\arrow \textcolor{red}{एचोऽयवायावः} (पा॰सू॰~६.१.७८)~\arrow जगतय्~अ~ति~\arrow जगतयति।} \textcolor{red}{जगतयतीति जगत्} इत्याचक्षाणाण्णिजन्तात्क्विपि ककारस्य \textcolor{red}{लशक्वतद्धिते} (पा॰सू॰~१.३.८) इत्यनेनेत्सञ्ज्ञायां \textcolor{red}{तस्य लोपः} (पा॰सू॰~१.३.९) इत्यनेन लोप इकारस्य \textcolor{red}{उपदेशेऽजनुनासिक इत्} (पा॰सू॰~१.३.२) इत्यनेनेत्सञ्ज्ञायां लोपे पकारस्य \textcolor{red}{हलन्त्यम्} (पा॰सू॰~१.३.३) इत्यनेनेत्सञ्ज्ञायां लोपे \textcolor{red}{वेरपृक्तस्य} (पा॰सू॰~६.१.६७) इत्यनेन वकार\-लोपे \textcolor{red}{णेरनिटि} (पा॰सू॰~६.४.५१) इत्यनेन णिलोपे।\footnote{जगति~\arrow धातुसञ्ज्ञा (पूर्ववत्)~\arrow \textcolor{red}{क्विप् च} (पा॰सू॰~३.२.७६)~\arrow जगति~क्विँप्~\arrow जगति~व्~\arrow \textcolor{red}{वेरपृक्तस्य} (पा॰सू॰~६.१.६७)~\arrow जगति~\arrow \textcolor{red}{णेरनिटि} (पा॰सू॰~६.४.५१)~\arrow जगत्~\arrow विभक्तिकार्यम्~\arrow जगत्।} पुनः \textcolor{red}{यातीति यः}।\footnote{\textcolor{red}{या प्रापणे} (धा॰पा॰~१०४९)~\arrow या~\arrow \textcolor{red}{अन्येष्वपि दृश्यते} (पा॰सू॰~३.२.१०१)~\arrow या~ड~\arrow या~अ~\arrow \textcolor{red}{डित्यभस्याप्यनु\-बन्धकरण\-सामर्थ्यात्} (वा॰~६.४.१४३)~\arrow य्~अ~\arrow य~\arrow विभक्तिकार्यम्~\arrow यः।} \textcolor{red}{जगदेव य इति जगत्यः} इति विग्रहे कर्मधारय\-समासे \textcolor{red}{जगत्यः}। न च \textcolor{red}{झलां जशोऽन्ते} (पा॰सू॰~८.२.३९) इत्यनेन कथं न जश्त्वम्। पृषोदरादित्वात्\footnote{\textcolor{red}{पृषोदरादीनि यथोपदिष्टम्} (पा॰सू॰~६.३.१०९) इत्यनेन। यद्वा \textcolor{red}{अयस्मयादीनि च्छन्दसि} (पा॰सू॰~१.४.२०) इति सूत्रेण छान्दसभत्वम्।} तद्भाव\-कल्पनेनादोषात्। पुनः तेन \textcolor{red}{जगत्येन}।\end{sloppypar}
\section[सपत्निवत्]{सपत्निवत्}
\centering\textcolor{blue}{तवाङ्घ्रिपूजानिर्माल्यतुलसीमालया विभो।\nopagebreak\\
स्पर्धते वक्षसि पदं लब्ध्वाऽपि श्रीः सपत्निवत्॥}\nopagebreak\\
\raggedleft{–~अ॰रा॰~१.२.१९}\\
\begin{sloppypar}\hyphenrules{nohyphenation}\justifying\noindent\hspace{10mm} अत्र \textcolor{red}{सपत्न्या तुल्यम्} इति विग्रहे \textcolor{red}{तेन तुल्यं क्रिया चेद्वतिः} (पा॰सू॰~५.१.११५) इत्यनेन \textcolor{red}{वति} प्रत्यये \textcolor{red}{सपत्नीवत्}।\footnote{\textcolor{red}{समानः पतिरस्याः} इति विग्रहे \textcolor{red}{नित्यं सपत्न्यादिषु} (पा॰सू॰~४.१.३५) इत्यनेन निपातनात्समानस्य सादेशे \textcolor{red}{सपत्नी}\-शब्दो व्युत्पन्नः।} \textcolor{red}{सपत्निवत्} इति कथम्।
अत्र हि \textcolor{red}{ङ्यापोः सञ्ज्ञा\-छन्दसोर्बहुलम्} (पा॰सू॰~६.३.६३) इत्यनेन ह्रस्वः।\footnote{अपि च वाल्मीकीय\-रामायणेऽयोध्या\-काण्डे~– \textcolor{red}{साहं त्वदर्थे सम्प्राप्ता त्वं तु मां नावबुध्यसे। सपत्निवृद्धौ या मे त्वं प्रदेयं दातुमिच्छसि॥} (वा॰रा॰~२.८.२६)। अत्र टीकाकाराः – \textcolor{red}{ङ्यापोः सञ्ज्ञा\-छन्दसोर्बहुलम्} (पा॰सू॰~६.३.६३) इत्यनेनार्षत्वेन वा ह्रस्वत्वम्। यथा~– \textcolor{red}{सपत्निवृद्धाविति ‘ङ्यापोः’ इति ह्रस्वः} (वा॰रा॰ भू॰टी॰)। \textcolor{red}{सपत्न्याः वृद्धिः – सपत्निवृद्धिः ‘ङ्यापोः’ इति ह्रस्वः} (वा॰रा॰ क॰टी॰)। \textcolor{red}{सपत्निवृद्धावित्यत्र ह्रस्व आर्षः} (वा॰रा॰ शि॰टी॰)। \textcolor{red}{सपत्निवृद्धावित्यार्षो ह्रस्वः} (वा॰रा॰ ति॰टी॰)।}\end{sloppypar}
\section[अतिहर्षितः]{अतिहर्षितः}
\centering\textcolor{blue}{इति ब्रुवन्तं ब्रह्माणं बभाषे भगवान् हरिः।\nopagebreak\\
किं करोमीति तं वेधाः प्रत्युवाचातिहर्षितः॥}\nopagebreak\\
\raggedleft{–~अ॰रा॰~१.२.२२}\\
\begin{sloppypar}\hyphenrules{nohyphenation}\justifying\noindent\hspace{10mm} ब्रह्मणः स्तुतिं श्रुत्वाऽऽत्मानं प्रदर्श्य भगवान् बभाषे यत् \textcolor{red}{किं करोमि} इति। समाकर्ण्याति\-हृष्टो ब्रह्मा प्रत्युवाच। \textcolor{red}{अतिहृष्टः} इति प्रयोगतन्त्रे \textcolor{red}{अतिहर्षितः} इत्यपाणिनीय इव प्रयुक्तः। यतो हि \textcolor{red}{हृष्‌}\-धातोः (\textcolor{red}{हृषँ तुष्टौ} धा॰पा॰~१२२९) \textcolor{red}{गत्यर्थाकर्मक\-श्लिष\-शीङ्स्थास\-वस\-जन\-रुह\-जीर्यतिभ्यश्च} (पा॰सू॰~३.४.७२) इत्यनेनाकर्मक\-धातोः कर्तरि \textcolor{red}{क्त}\-प्रत्यये ककारानुबन्ध\-लोपे\footnote{\textcolor{red}{लशक्वतद्धिते} (पा॰सू॰~१.३.८) \textcolor{red}{तस्य लोपः} (पा॰सू॰~१.३.९) इत्याभ्याम्।} ष्टुत्वे\footnote{\textcolor{red}{ष्टुना ष्टुः} (पा॰सू॰~८.४.४१) इत्यनेन।} विभक्तिकार्ये \textcolor{red}{हृष्टः} इत्येव हि पाणिनीयम्। \textcolor{red}{हर्षितः} इति प्रयोगोऽपि पाणिनीयः। यतो हि \textcolor{red}{हर्षः सञ्जातोऽस्य} इति विग्रहे \textcolor{red}{तदस्य सञ्जातं तारकादिभ्य इतच्} (पा॰सू॰~५.२.३६) इत्यनेन \textcolor{red}{इतच्‌}\-प्रत्ययेऽनुबन्ध\-लोपे \textcolor{red}{यचि भम्} (पा॰सू॰~१.४.१८) इत्यनेन भसञ्ज्ञायां \textcolor{red}{यस्येति च} (पा॰सू॰~६.४.१४८) इत्यनेनाकार\-लोपे विभक्ति\-कार्ये \textcolor{red}{हर्षितः}। यद्वा \textcolor{red}{हर्षमितः} इति विग्रहे \textcolor{red}{द्वितीया श्रितातीत\-पतित\-गतात्यस्त\-प्राप्तापन्नैः} (पा॰सू॰~२.१.२४) इत्यनेन द्वितीया\-तत्पुरुष\-समासः। न च \textcolor{red}{इत}\-शब्दस्य श्रितादि\-बहिर्भूतत्वात्कथं समास इति वाच्यम्। \textcolor{red}{द्वितीया} इति योग\-विभागेन द्वितीया समर्थेन सुबन्तेन समस्यत इत्यर्थकरणे द्वितीया\-तत्पुरुषो नैव दोषावहः। तेन \textcolor{red}{गृहं यातो गृहयातः गृहमागतो गृहागतः} इत्यादि\-प्रयोगा अपि सङ्गच्छन्ते। कालिदासोऽपि प्रयुङ्क्ते यथा~–\end{sloppypar}
\centering\textcolor{red}{सुरेन्द्रमात्राश्रितगर्भगौरवात्प्रयत्नमुक्तासनया गृहागतः।\nopagebreak\\
तयोपचाराञ्जलिखिन्नहस्तया ननन्द पारिप्लवनेत्रया नृपः॥}\nopagebreak\\
\raggedleft{–~र॰वं॰~३.११}\\
\begin{sloppypar}\hyphenrules{nohyphenation}\justifying\noindent अत्र \textcolor{red}{गृहमागतो गृहागतः} इत्यसति योग\-विभागे कथं समासः सम्भवः। तस्मात् \textcolor{red}{हर्षमितः} इति विग्रहे द्वितीया\-तत्पुरुषे \textcolor{red}{कृत्तद्धित\-समासाश्च} (पा॰सू॰~१.२.४६) इत्यनेन प्रातिपदिक\-सञ्ज्ञायां \textcolor{red}{सुपो धातु\-प्रातिपदिकयोः} (पा॰सू॰~२.४.७१) इत्यनेन विभक्ति\-लुकि पुनस्तामेव प्रातिपदिक\-सञ्ज्ञामाश्रित्य \textcolor{red}{सौ} विभक्तौ \textcolor{red}{हर्षितः}। न च सति विभक्ति\-लोपे \textcolor{red}{हर्ष इत} इति स्थिते गुणः स्यादिति चेत्। अत्राऽकृति\-गणत्वात् \textcolor{red}{शकन्ध्वादिषु पर\-रूपं वाच्यम्} (वा॰~६.१.९४) इत्यनेन पर\-रूपम्। कुत्र स्यात्कस्य वेति चेत्। \textcolor{red}{तच्च टेः} (वै॰सि॰कौ॰~७९, ल॰सि॰कौ॰~३९)। अर्थात्तत्पर\-रूपं भसञ्ज्ञकस्य टेः स्थाने भवतु।\footnote{पूर्वपक्षोऽयम्।} असति च तस्मिन् \textcolor{red}{शक}\-घटकाकारस्य पर\-रूपतया \textcolor{red}{शक अन्धु} इति स्थिते तयोर्दीर्घो दुर्वार एव। एवं \textcolor{red}{मनस् ईषा} इति स्थिते \textcolor{red}{शकन्ध्वादिषु पररूपं वाच्यम्} (वा॰~६.१.९४) इत्यनेनाऽकृति\-गणत्वात् \textcolor{red}{मनीषा} इत्यत्र \textcolor{red}{टि} इत्यस्य पररूपे \textcolor{red}{मन्~ई~ईषा} इति स्थिते ततो \textcolor{red}{अज्झीनं वर्णपरेण संयोज्यम्} (स्व॰शि॰~८.१५) इति वचनात् \textcolor{red}{मनीषा} इत्यत्र दीर्घः। अनिष्टः प्रयोगः स्यात्किन्तु \textcolor{red}{पतत् अञ्जलौ} इति। \textcolor{red}{टि} इत्यस्य पररूपे दीर्घे \textcolor{red}{पताञ्जलिः} इत्यसङ्गतं स्यात्।\footnote{\textcolor{red}{यदि त्वादित्यधि\-कारादस्यैवेष्येत तर्हि मनीषा पतञ्जलिरिति न सिध्येत्। केचित्तु मनः\-पतच्छब्दयोः पृषोदरादित्वादन्त्य\-लोप अकारस्यैव पररूपमाहुः} (त॰बो॰~७९)।} अतः \textcolor{red}{‘टेः’ इति पञ्चम्यन्तम्। तदन्वचि परे पूर्व\-परयोः पर\-रूपम्} इत्यस्मद्गुरु\-चरणाः।\footnote{बालमनोरमायामपि~– \textcolor{red}{ततश्च शकादि\-शब्दानां टेरचि परे टेश्च परस्याचस्स्थाने पर\-रूपमेकादेश इत्यर्थाल्लभ्यते। आदित्यनु\-वृत्तौ शकन्ध्वादिगणे ‘सीमन्त’ इति कतिपय\-रूपाणामसिद्धेः} (बा॰म॰~७९)।} तथैवात्रापि पर\-रूपं करणीयम्।\end{sloppypar}
\section[मद्दत्तवरदर्पितः]{मद्दत्तवरदर्पितः}
\centering\textcolor{blue}{भगवन् रावणो नाम पौलस्त्यतनयो महान्।\nopagebreak\\
राक्षसानामधिपतिर्मद्दत्तवरदर्पितः॥}\nopagebreak\\
\raggedleft{–~अ॰रा॰~१.२.२३}\\
\begin{sloppypar}\hyphenrules{nohyphenation}\justifying\noindent\hspace{10mm} अत्र ब्रह्मा रावणस्योच्छृङ्खलतां वर्णयति। \textcolor{red}{मद्दत्तवर\-दर्पितः} इति। अत्र \textcolor{red}{मया दत्तो वर इति मद्दत्तवरस्तेन दर्पितः} इति विग्रहे \textcolor{red}{दृप्‌}\-धातोः (\textcolor{red}{दृपँ हर्षमोहनयोः} धा॰पा॰~११९६) अनिट्कत्वात्\footnote{\textcolor{red}{एकाच उपदेशेऽनुदात्तात्‌} (पा॰सू॰~७.२.१०) इति सूत्रेण धातोरनिट्कत्वम्। तद्बाधित्वा \textcolor{red}{रधादिभ्यश्च} (पा॰सू॰~७.२.४५) इत्यनेन वैकल्पिकेट्प्राप्तिः। परन्तु निष्ठायां \textcolor{red}{यस्य विभाषा} (पा॰सू॰~७.२.१५) इत्यनेनेडभावः।} \textcolor{red}{दृप्तः} इत्येव पाणिनीयं \textcolor{red}{दर्पितः} इति कथम्। \textcolor{red}{दर्पः सञ्जातोऽस्य} इति विग्रहे तारकादिगणे \textcolor{red}{दर्पित}\-शब्दस्य पाठात् \textcolor{red}{इतच्} प्रत्यये\footnote{\textcolor{red}{तदस्य सञ्जातं तारकादिभ्य इतच्} (पा॰सू॰~५.२.३६) इत्यनेन।} भत्वादकार\-लोपे\footnote{\textcolor{red}{यचि भम्} (पा॰सू॰~१.४.१८) इत्यनेन भत्वम्। \textcolor{red}{यस्येति च} (पा॰सू॰~६.४.१४८) इत्यनेनाकार\-लोपः।} \textcolor{red}{दर्पितः}। यद्वा \textcolor{red}{दर्पमितः} इति द्वितीया\-समासे पर\-रूपे\footnote{\textcolor{red}{शकन्ध्वादिषु पररूपं वाच्यम्} (वा॰~६.१.९४) इत्यनेन।} \textcolor{red}{दर्पितः} इति।\end{sloppypar}
\section[सहायम्]{सहायम्}
\label{sec:sahayam}
\centering\textcolor{blue}{यूयं सृजध्वं सर्वेऽपि वानरेष्वंशसम्भवान्।\nopagebreak\\
विष्णोः सहायं कुरुत यावत्स्थास्यति भूतले॥}\nopagebreak\\
\raggedleft{–~अ॰रा॰~१.२.३०}\\
\begin{sloppypar}\hyphenrules{nohyphenation}\justifying\noindent\hspace{10mm} अत्र भगवान् ब्रह्मा सर्वान् देवान् वानर\-शरीराणि स्रष्टुं प्रेरयति। \textcolor{red}{विष्णोः सहायं कुरुत} अत्र \textcolor{red}{सहायम्} इत्यपाणिनीयमिव। यतो हि \textcolor{red}{अयँ गतौ} (धा.पा. ४७४) इत्यस्मात्कर्तरि \textcolor{red}{अच्‌}\-प्रत्यये \textcolor{red}{सहायः} इति भवति\footnote{\textcolor{red}{सहायः, पुं॰, (सह अयते इति। अय + अच्)} इति शब्द\-कल्प\-द्रुमः।} किन्त्वत्र तु \textcolor{red}{सर्वे सहायं कुरुत} इति विवक्षायां सहाय\-शब्दो भाव\-साधनः प्रतीयते। भावे च \textcolor{red}{सहायस्य भावः साहाय्यम्} इति विग्रहे \textcolor{red}{गुण\-वचन\-ब्राह्मणादिभ्यः कर्मणि च} (पा.सू. ५.१.१२४) इत्यनेन \textcolor{red}{ष्यञ्‌}\-प्रत्यये वृद्धौ भत्वादकार\-लोपे\footnote{\textcolor{red}{तद्धितेष्वचामादेः} (पा॰सू॰~७.२.११७) इत्यनेनादिवृद्धिः। \textcolor{red}{यचि भम्} (पा॰सू॰~१.४.१८) इत्यनेन भत्वम्। \textcolor{red}{यस्येति च} (पा॰सू॰~६.४.१४८) इत्यनेनाकार\-लोपः।} \textcolor{red}{साहाय्यम्}। \textcolor{red}{अत्र सहायम्} इति हि। पञ्चवट्यां निवासे \textcolor{red}{भगवतः साह्यं कुर्वन्तु}\footnote{\textcolor{red}{साह्यम्, क्ली॰, सहस्य भावः (सह + ष्यञ्)} इति शब्दकल्पद्रुमः।} इति विवक्षायां \textcolor{red}{सहायम्} इति कथमिति चेत्। \textcolor{red}{सहायम्} इत्यत्र पृषोदरादित्वाद्वृद्ध्यभावे यकार\-लोपे \textcolor{red}{सहायम्}। यद्वा \textcolor{red}{सर्वे सहायं यथा स्यात्तथा कुर्वन्तु} इति क्रिया\-विशेषणत्वाद्द्वितीया। यद्वा \textcolor{red}{आत्मानं सहायं कुर्वन्तु} इत्यध्याहारेऽप्यपाणिनीयता\-परिहारः। यद्वा \textcolor{red}{अयनम् अयः} इति भावे \textcolor{red}{एरच्} (पा॰सू॰~३.३.५६) इत्यनेन \textcolor{red}{इ}\-धातोः (\textcolor{red}{इण् गतौ} धा॰पा॰~१०४५) अचि गुणेऽयादेशे विभक्ति\-कार्ये \textcolor{red}{अयः}। \textcolor{red}{सह अयः सहायस्तं सहायम्} इति साधनीयम्।\footnote{\textcolor{red}{सह अयः सहायः} इत्यत्र \textcolor{red}{सह सुपा} (पा॰सू॰~२.१.४) इत्यनेन सुप्सुपा\-समासः। \pageref{sec:sahayam_me}तमे पृष्ठे \ref{sec:sahayam_me} \nameref{sec:sahayam_me} इति प्रयोगस्य विमर्शमपि पश्यन्तु। अनेन \textcolor{red}{जानकिनाथ सहाय करैं जब कौन बिगाड़ करे नर तेरो} इति मुक्तके \textcolor{red}{सहाय करैं} इति गोस्वामि\-तुलसी\-दासकृतोऽवधी\-भाषा\-प्रयोगोऽपि व्याख्यातः।}\end{sloppypar}
\section[सहायार्थम्]{सहायार्थम्}
\centering\textcolor{blue}{देवाश्च सर्वे हरिरूपधारिणः स्थिताः सहायार्थमितस्ततो हरेः।\nopagebreak\\
महाबलाः पर्वतवृक्षयोधिनः प्रतीक्षमाणा भगवन्तमीश्वरम्॥}\nopagebreak\\
\raggedleft{–~अ॰रा॰~१.२.३२}\\
\begin{sloppypar}\hyphenrules{nohyphenation}\justifying\noindent\hspace{10mm} अत्र \textcolor{red}{सहायार्थम्} इति प्रयोगोऽपि तथैव। अत्र \textcolor{red}{अयनमयः}। भावे \textcolor{red}{अच्}।\footnote{\textcolor{red}{इण् गतौ} (धा॰पा॰~१.१०४५) इति धातोः \textcolor{red}{एरच्} (पा॰सू॰~३.३.५६) इत्यनेन।} \textcolor{red}{सह अयः सहायः}।\footnote{सुप्सुपासमासः।} तस्मायिदम् \textcolor{red}{सहायार्थम्}।\footnote{\textcolor{red}{अर्थेन नित्य\-समासो विशेष्य\-लिङ्गता चेति वक्तव्यम्‌} (वा॰~२.१.३६) इत्यनेन नित्यसमासः।} यद्वा भावेऽयं घञन्तः \textcolor{red}{आयः}।\footnote{\textcolor{red}{अयँ गतौ} (धा॰पा॰~४७४) इति धातोः \textcolor{red}{भावे} (पा॰सू॰~३.३.१८) इत्यनेन।} तेन सन्धौ कृते न दोषः। \textcolor{red}{सह आयः सहायः}। \textcolor{red}{तस्मा इदं सहायार्थम्}।
एवं निष्पन्न\-\textcolor{red}{सहाय}\-शब्दस्य \textcolor{red}{चतुर्थी तदर्थार्थ\-बलि\-हित\-सुख\-रक्षितैः} (पा॰सू॰~२.१.३६) इत्यनेन \textcolor{red}{अर्थेन नित्य\-समासो विशेष्य\-लिङ्गता चेति वक्तव्यम्} (वा॰~२.१.३६) इति\-वार्त्तिक\-बलेन नित्य\-समासे सिद्धमिदम्। न च \textcolor{red}{सहायाय इदम्} इति लौकिक\-विग्रहे श्रूयते पश्चात् \textcolor{red}{अर्थ}\-शब्देन सह समास इति चेत्। विग्रहो द्विधा स्वपद\-विग्रहोऽस्वपदविग्रहश्च। स्वपद\-विग्रहो नाम यत्र विग्रहे श्रुतानां पदानां समासः। यत्र विग्रहेऽश्रुतानामपि श्रुतार्थ\-बोध\-समर्थानां समासस्तत्रास्वपद\-विग्रहः। यथा \textcolor{red}{प्रातिपदिकार्थ\-लिङ्ग\-परिमाण\-वचन\-मात्रे प्रथमा} (पा॰सू॰~२.३.४६) इति सूत्रे। अत्र हि \textcolor{red}{प्रातिपदिकार्थश्च लिङ्गञ्च परिमाणञ्च वचनञ्चेति प्रातिपदिकार्थ\-लिङ्ग\-परिमाण\-वचनानि तान्येवेति प्रातिपदिकार्थ\-लिङ्ग\-परिमाण\-वचन\-मात्रं तस्मिन् प्रातिपदिकार्थ\-लिङ्ग\-परिमाण\-वचनमात्रे}। अत्र विग्रहे श्रुतस्तु \textcolor{red}{एव}\-शब्दः किन्त्वस्व\-पद\-विग्रहतया तदर्थ\-वाचक आगतो \textcolor{red}{मात्र}\-शब्दः। स च प्रत्येकमन्वितः। यतो हि \textcolor{red}{द्वन्द्वान्ते श्रूयमाणं पदं प्रत्येकमभिसम्बध्यते}। न च द्वन्द्व\-घटकोऽन्त्य\-शब्दस्तु वचनमिति तदेव प्रत्येकमभिसम्बध्यतामिति चेत्। अन्त्य\-शब्दस्य समीपार्थे लक्षणा। न च सा जघन्या वृत्तिर्वैयाकरणैरस्वीकृताऽपि।\footnote{\textcolor{red}{तन्न। सति तात्पर्ये सर्वे सर्वार्थवाचका इति भाष्याल्लक्षणाया अभावाद्वृत्ति\-द्वयावच्छेदक\-द्वय\-कल्पने गौरवात्। जघन्य\-वृत्ति\-कल्पनाया अन्याय्यत्वाच्च} (प॰ल॰म॰~२७)।} तथा च \textcolor{red}{गङ्गायां घोषः} इत्यत्राऽभीर\-पल्ल्यर्थ\-वाचकस्य घोष\-शब्दस्य गङ्गा\-पदाभिधेय\-भगीरथ\-रथ\-खातावच्छिन्न\-जल\-प्रवाह\-रूपे शक्यार्थेऽन्वयासम्भवात्तात्पर्यानुपपतेश्च। अत्र सामीप्य\-सम्बन्धेन गङ्गा\-पदस्य गङ्गा\-समीपवर्ति\-तटे लक्षणा। न च \textcolor{red}{गङ्गायाम्} इत्यत्र सप्तमी कथमिति चेत्। औपश्लेषिकस्याऽधारस्यात्र विवक्षा। औपश्लेषिको हि सम्बन्धः।
\textcolor{red}{उपश्लेषे भव औपश्लेषिकः} इति \textcolor{red}{तत्र जातः} (पा॰सू॰~४.३.२५) इत्यनेन \textcolor{red}{ठक्}।\footnote{उपश्लेष~\arrow \textcolor{red}{तत्र जातः} (पा॰सू॰~४.३.२५)~\arrow उपश्लेष~ठक्~\arrow उपश्लेष~ठ~\arrow \textcolor{red}{ठस्येकः} (पा॰सू॰~७.३.५०)~\arrow उपश्लेष~इक~\arrow \textcolor{red}{यचि भम्} (पा॰सू॰~१.४.१८)~\arrow भसञ्ज्ञा~\arrow \textcolor{red}{किति च} (पा॰सू॰~७.२.११८)~\arrow औपश्लेष~इक~\arrow \textcolor{red}{यस्येति च} (पा॰सू॰~६.४.१४८)~\arrow औपश्लेष्~इक~\arrow औपश्लेषिक~\arrow विभक्ति\-कार्यम्~\arrow औपश्लेषिकः।} उपश्लेषश्च सामीप्येन संयोगेन वा सम्बन्धः। संयोगे यथा \textcolor{red}{कटे शेते}। सामीप्ये यथा \textcolor{red}{गुरौ वसति}। न तु \textcolor{red}{गुरुमभिसंयुज्य वसति} अपि तु \textcolor{red}{गुरोः समीपे वसति}। तथैव \textcolor{red}{गङ्गायां घोषः}। न तु \textcolor{red}{गङ्गामभिसंयुज्य घोषः} अपि तु \textcolor{red}{गङ्गासमीपे घोषः}। ध्येयमेतत्~– शक्यार्थे स्वीकृते \textcolor{red}{गङ्गायां मीनः} इत्यादौ \textcolor{red}{कटे शेते} इत्यादिवत् संयोग\-सम्बन्ध\-रूप औपश्लेषिक आधारे सप्तमी किन्तु लक्ष्यार्थे \textcolor{red}{गङ्गायां घोषः} इत्यादौ सामीप्य\-सम्बन्ध औपश्लेषिक आधारे सप्तमी \textcolor{red}{गुरौ वसति} इत्यादिवत्। परञ्च \textcolor{red}{गङ्गायां मीन\-घोषौ} इत्यादौ लक्षणा\-स्वीकारे दोषः। यतो हि \textcolor{red}{मीनश्च घोषश्च मीन\-घोषौ} इति द्वन्द्वः। स च साहित्यमपेक्षते। \textcolor{red}{साहित्यं नामैक\-धर्मावच्छिन्नस्यैक\-धर्मावच्छिन्न\-संसर्गेणैक\-कर्मावच्छिन्नेऽन्वयः}। यथा \textcolor{red}{राम\-लक्ष्मणौ राक्षसान् हतः} इत्यत्र \textcolor{red}{राम\-लक्ष्मणौ} इत्यस्य कर्तृत्व\-रूप\-सम्बन्धावच्छिन्न\-कर्तृत्व\-रूप\-धर्मेण हन्तृत्व\-रूप\-धर्मावच्छिन्ने हनन\-कर्मण्यन्वयः। तथात्रासम्भवः। एकधर्माव\-च्छिन्नस्य \textcolor{red}{मीन\-घोषौ} इत्यस्यैक\-धर्मावच्छिन्न\-संसर्गेण स्वनिष्ठत्वावच्छिन्न\-स्वनिष्ठत्व\-संसर्गेण सत्तात्वावच्छिन्न\-सत्ता\-पदार्थेऽन्वयः। परमत्र \textcolor{red}{गङ्गायां मीन\-घोषौ} इत्यत्र मीन\-पदार्थाय गङ्गा\-पदस्य प्रवाहार्थो घोष\-कृते तीर\-रूपो लक्ष्यार्थस्तर्हि कथमन्वय इति चेत्। अत्र लक्षणामनङ्गीकृत्य शक्यतावच्छेदक\-धर्मारोपः।
तथा कृत उभयत्र प्रवाह\-रूपोऽर्थः समान एव। तस्मादत्र न लक्षणा। सामीप्ये षष्ठी च चेत् \textcolor{red}{द्वन्द्वस्यान्तं द्वन्द्वान्तं द्वन्द्व\-समीपान्तम्} इत्यर्थः। सामीप्य\-षष्ठ्यर्थस्यास्मन्नयेऽस्वीकारात्प्रौढ\-मनोरमा\-बृहच्छब्दरत्न\-लघुशब्दरत्नादौ साटोपं खण्डितत्वाच्च। तत्रेयं परिस्थितिर्यदावृत्त\-हलन्तमित्यत्रैव प्रौढमनोरमायां दीक्षित\-महाभागैः पूर्वपक्षः प्रदर्शितो यत् \textcolor{red}{हलित्येकदेशस्यैव तन्त्रावृत्त्येकशेषान्यतममस्तु} (प्रौ॰म॰~१)। \textcolor{red}{हस्य ल्} इति समासः करिष्यत एवं च षष्ठ्यर्थः सामीप्यं भविष्यति। तथा च \textcolor{red}{अन्त्य}\-शब्दो \textcolor{red}{ल्}शब्देन सहान्वेष्यते। तथा च \textcolor{red}{ह\-समीप\-वर्त्यन्त्यो लकार इत्} इत्यर्थ\-स्वीकारेऽन्योऽन्याश्रय\-दोष\-परिहारः किन्तु स्वीकृतेऽस्मिन् विकल्पे षष्ठ्यर्थो हि सामीप्यम्। तस्य च सर्वथाऽसिद्धत्वम्। यतो हि सामीप्य\-रूपस्य षष्ठ्यर्थस्य लोकेऽप्रसिद्धिः। नहि \textcolor{red}{नीलमम्बरं यस्य} इत्युक्ते नीलाम्बर\-समीप\-वर्ति\-भवनादिः प्रतीयते न वा \textcolor{red}{चित्रा गावो यस्य} इत्युक्ते चित्रगवीनां समीप\-वर्ति\-वृक्षादिः प्रतीयते। किं वा \textcolor{red}{चित्रा गावोऽन्तरा समीपे वा यस्य} इत्यर्थे बहुव्रीहि\-मत्वर्थीयमपि न भवति \textcolor{red}{अन्तरा}\-शब्दस्य सामीप्यस्य च षष्ठ्यर्थ\-स्वीकारे मानाभावात्। अत एव \textcolor{red}{अन्तरेण न समासः} इति भाष्योक्तिः।\footnote{\textcolor{red}{स्त्र्यन्तस्य प्रातिपदिकस्यो\-पसर्जनस्य ह्रस्वो भवतीत्युच्यते न चान्तरेण समासं स्त्र्यन्तं प्रातिपदिकमुप\-सर्जनमस्ति} (भा॰पा॰सू॰~१.२.४८)।} वस्तुतस्तु षष्ठ्यर्था यद्यप्येकशतम् \textcolor{red}{एक\-शतं हि षष्ठ्यर्थाः} (भा॰पा॰सू॰~१.१.४९) इति भाष्योक्तेः किन्तु सम्बन्ध\-भेदेनार्थात्सम्बन्ध\-वैचित्र्येण तदर्थ\-वैविध्यम्। यतो हि सूत्र\-कारोऽपि षष्ठ्याः सम्बन्धमेवार्थं कण्ठ\-रवेण कथयति। \textcolor{red}{षष्ठी शेषे} (पा॰सू॰~२.३.५०)। \textcolor{red}{शेष}\-शब्दो हि \textcolor{red}{शिष्‌}\-धातोः (\textcolor{red}{शिषॢँ विशेषणे} धा॰पा॰~१४५१) कर्तरि \textcolor{red}{अच्}प्रत्ययेन। \textcolor{red}{शिनष्टि कारकादीनि प्रातिपदिकार्थञ्च विशिनष्टि स्वस्माद्व्यावर्तयतीति शेषः}। पचादित्वादचि प्रत्यये\footnote{\textcolor{red}{नन्दि\-ग्रहि\-पचादिभ्यो ल्युणिन्यचः} (पा॰सू॰~३.१.१३४) इत्यनेन।} \textcolor{red}{पुगन्त\-लघूपधस्य च} (पा॰सू॰~७.३.८६) इत्यनेन गुणे विभक्तिकार्ये सिद्धः \textcolor{red}{शेषः} इति।
यद्वा \textcolor{red}{शिष्यते सर्वेभ्यः कारक\-प्रातिपदिकार्थेभ्यो व्यावर्तत इति शेषः} इति \textcolor{red}{अकर्तरि च कारके सञ्ज्ञायाम्} (पा॰सू॰~३.३.१९) इत्यनेन कर्तृ\-भिन्ने कर्मणि कारके घञि गुणे सिद्धम्। यद्वा \textcolor{red}{शेषणं शेषः} इति विग्रहे भाव\-घञन्तः।\footnote{\textcolor{red}{भावे} (पा॰सू॰~३.३.१८) इत्यनेन।} सोऽस्त्यस्मिन्निति मत्वर्थीयोऽर्शआद्यच्।\footnote{\textcolor{red}{अर्शआदिभ्योऽच्} पा॰सू॰~५.२.१२७) इत्यनेन।} न च करणे \textcolor{red}{घञ्} इति चेत्। \textcolor{red}{करणाधिकरणयोश्च} (पा॰सू॰~३.३.११७) इत्यनेन विधीयमान\-ल्युट्\-प्रत्यनेन बाधोऽत एव। \textcolor{red}{पुंसि सञ्ज्ञायां घः प्रायेण} (पा॰सू॰~३.३.११८) इत्यनेन \textcolor{red}{घः} प्रत्ययः क्रियताम्। सञ्ज्ञा\-भावे कथमत्र। \textcolor{red}{घः प्रायेण} इति हि \textcolor{red}{सञ्ज्ञायाम्} इत्यस्य विशेषणमिति चेत्। \textcolor{red}{प्रायेण} इति विशेषणं न करण\-घञ्व्यवस्थापकम्। \textcolor{red}{न ह्युपाधेरुपाधिर्भवति} (भा॰पा॰सू॰~१.३.२, ५.१.१६) इति भाष्यकारेण निषिद्धत्वात्। अत एव \textcolor{red}{हलश्च} (पा॰सू॰~३.३.१२१) इति घञपि न यतो ह्ययं घविषयेऽपवादत्वात्प्रवर्तते। \textcolor{red}{उत्सर्गापवादौ समान\-देशौ भवतः}\footnote{मूलं मृग्यम्।} इति नियमस्य सार्वत्रिकत्वात् \textcolor{red}{घ} एव बाध्यते तर्हि घञः का गतिः।
अगतिक\-गतिस्तस्माद्भाव\-घञन्त एव। पश्चात् \textcolor{red}{अच्} प्रत्यये \textcolor{red}{शेष}\-शब्दो निष्पाद्यते। इमा युक्तयः प्रौढमनोरमायाम् \textcolor{red}{उपदेश}\-शब्द\-निष्पत्तौ स्पष्टं निरूपिता दीक्षित\-महाभागैर्यथा~– \textcolor{red}{यद्यपि उपदिश्यतेऽनेनेति करण\-व्युत्पत्त्या शास्त्रमुपदेश इति भाष्य\-वृत्त्यादिषु व्याख्यातं तथाऽपि तत्प्रौढिवादमात्रं करणे घञो दुर्लभत्वाल्ल्युटा बाधात्। न च “घः” असञ्ज्ञत्वात्। प्रायेण सञ्ज्ञायामिति व्याख्यानस्य क्लिष्टत्वात्। न ह्युपाधेरुपाधिर्भवतीत्यादिना भाष्य\-कृतावहेलनाच्च। अत एव घापवादः “हलश्च” (पा॰सू॰~३.३.१२१) इति घञपीह न। बाहुलकं त्वगतिक\-गतिः। अत एव प्रक्रियन्ते शब्दा याभिरिति करण\-व्युत्पत्तिरपि परास्ता। तथा च वार्त्तिकं “अजब्भ्यां स्त्रीखलनाः” (वा॰~३.३.१२६) “स्त्रियाः खलनौ विप्रतिषेधेन” (वा॰~३.३.१२६) इति। अतो भाव एव प्रत्ययो न्याय्य इति भावः} इति (प्रौ॰म॰~३)।
किन्तु यन्निमित्तोपदेश\-प्रवृत्तावित्यादिषु स्थलेषु भाष्ये करण\-घञन्तस्य प्रतिपादितत्वात्सा सरणिरत्राप्यनु\-सर्तव्या। तथा च \textcolor{red}{शिष्यन्ते कारक\-प्रातिपदिकार्था अनेनेति शेषः}। स सम्बन्धः। तथा च सिद्धान्तकौमुद्यां \textcolor{red}{कारक\-प्रातिपदिकार्थ\-व्यतिरिक्तः स्व\-स्वामि\-भावादि\-सम्बन्धः शेषस्तत्र षष्ठी स्यात्} (वै॰सि॰कौ॰~६०६)। एवं हि सम्बन्धः षष्ठ्यर्थ इति राद्धान्तः। सम्पूर्वकात् \textcolor{red}{बन्ध्‌}\-धातोः (\textcolor{red}{बन्धँ बन्धने} धा॰पा॰~१५०८) \textcolor{red}{सम्यग्बध्नाति प्रतियोग्यनुयोगिनौ यः स सम्बन्धः} इति विग्रहे कर्तरि पचाद्यचि।\footnote{\textcolor{red}{नन्दि\-ग्रहि\-पचादिभ्यो ल्युणिन्यचः} (पा॰सू॰~३.१.१३४) इत्यनेन।} तथा च सम्बन्धो हि द्विष्ठः सम्बन्धि\-भिन्न आश्रयतया विशिष्ट\-बुद्धि\-नियामकश्चेति वैयाकरण\-सिद्धान्तः।\footnote{\textcolor{red}{“सम्बन्धो हि सम्बन्धि\-द्वय\-भिन्नत्वे सति द्विष्ठत्वे च सत्याश्रयतया विशिष्टबुद्धिनियामकः” इत्यभियुक्त\-व्यवहारात्} (प॰ल॰म॰~११)।} सामीप्यं हि नैव द्विष्ठम्। अतः कथं सम्बन्धो भवेदिति ममापि मनीषा। एवं षष्ठ्यर्थो न सामीप्यम्। अतः षष्ठी\-बलेन सामीप्यार्थो नावगन्तुं शक्यते। तेन
पद\-समीपेऽन्त\-पदस्य लक्षणा। साऽस्वीकृतेति चेच्छक्यतावच्छेदकता\-रूपः। इत्थमुपबृंहणेन सिद्धमिदं यदस्वपद\-विग्रह एव \textcolor{red}{प्रातिपदिकार्थ\-लिङ्ग\-परिमाण\-वचन\-मात्रे प्रथमा} (पा॰सू॰~२.३.४६) इति सूत्रे। \textcolor{red}{मात्र}\-शब्दस्य मयूर\-व्यंसकादित्वात्समासः।\footnote{\textcolor{red}{मयूरव्यंसकादयश्च} (पा॰सू॰~२.१.७२) इत्यनेन।} तथैव \textcolor{red}{सहायायेदं सहायार्थम्} इत्यत्राप्यस्व\-पद\-विग्रहे \textcolor{red}{अर्थ}\-शब्देन नित्य\-समासः\footnote{\textcolor{red}{अर्थेन नित्य\-समासो विशेष्य\-लिङ्गता चेति वक्तव्यम्‌} (वा॰~२.१.३६) इत्यनेन।} क्रिया\-विशेषणत्वाच्च द्वितीया। अथवा \textcolor{red}{सहायार्थम्} इति पदं \textcolor{red}{भगवन्तम्} इत्यस्य विशेषणम्। एवं हि \textcolor{red}{हरि\-रूप\-धारिणः सर्वे देवा हरेरितस्ततः स्थिताः सहायार्थमीश्वरं भगवन्तं प्रतीक्षमाणाः} इति योजनीयम्। एवं \textcolor{red}{सहायानामर्थः सहायार्थः}। \textcolor{red}{सोऽस्त्यस्मिन्निति सहायार्थः}। अर्शआद्यच्।\footnote{\textcolor{red}{अर्शआदिभ्योऽच्} पा॰सू॰~५.२.१२७) इत्यनेन।} तं \textcolor{red}{सहायार्थम्}। सहाय\-प्रयोजनवन्तमिति भावः। सेवकेभ्यो गौरवं दातुं भगवाञ्छ्रीरामो रावणेन पराजितानपि देवान् स्वकीय\-लीलोपकरणानि मत्वा तत्साहाय्यमभिलषति यथा श्रीमद्भागवते~–\end{sloppypar}
\centering\textcolor{red}{नेदं यशो रघुपतेः सुरयाच्ञयाऽऽत्तलीलातनोरधिकसाम्यविमुक्तधाम्नः।\nopagebreak\\
रक्षोवधो जलधिबन्धनमस्त्रपूगैः किं तस्य शत्रुहनने कपयः सहायाः॥}\nopagebreak\\
\raggedleft{–~भा॰पु॰~९.११.२०}\\
\begin{sloppypar}\hyphenrules{nohyphenation}\justifying\noindent यद्वा \textcolor{red}{सहायान् कपीश्वरानर्थयतेऽभि\-लषत्यन्वेष्टि वेति सहायार्थस्तं सहायार्थम्} इति विग्रहे णिजन्त\-\textcolor{red}{अर्थ्‌}\-धातोः (\textcolor{red}{अर्थ उपयाच्ञायाम्} धा॰पा॰~१९०६) कर्मण्यणि\footnote{\textcolor{red}{कर्मण्यण्} (पा॰सू॰~३.२.१) इत्यनेन।} दीर्घे विभक्ति\-कार्ये सिद्धमिदम्।\footnote{\textcolor{red}{अर्थ उपयाच्ञायाम्} (धा॰पा॰~१९०६)~\arrow \textcolor{red}{सत्याप\-पाश\-रूप\-वीणा\-तूल\-श्लोक\-सेना\-लोम\-त्वच\-वर्म\-वर्ण\-चूर्ण\-चुरादिभ्यो णिच्} (पा॰सू॰~३.१.२५)~\arrow अर्थ~णिच्~\arrow अर्थ~इ~\arrow \textcolor{red}{अतो लोपः} (पा॰सू॰~६.४.४८)~\arrow अर्थ्~इ~\arrow अर्थि~\arrow \textcolor{red}{सनाद्यन्ता धातवः} (पा॰सू॰~३.१.३२)~\arrow धातुसञ्ज्ञा। सहाय~शस्~अर्थि~\arrow \textcolor{red}{कर्मण्यण्} (पा॰सू॰~३.२.१)~\arrow सहाय~शस्~अर्थि~अण्~\arrow सहाय~शस्~अर्थि~अ~\arrow \textcolor{red}{णेरनिटि} (पा॰सू॰~६.४.५१)~\arrow सहाय~शस्~अर्थ्~अ~\arrow सहाय~शस्~अर्थ~\arrow \textcolor{red}{उपपदमतिङ्} (पा॰सू॰~२.२.१९)~\arrow \textcolor{red}{कृत्तद्धित\-समासाश्च} (पा॰सू॰~१.२.४६)~\arrow प्रातिपदिकसञ्ज्ञा~\arrow \textcolor{red}{सुपो धातु\-प्रातिपदिकयोः} (पा॰सू॰~२.४.७१)~\arrow सहाय~अर्थ~\arrow \textcolor{red}{अकः सवर्णे दीर्घः} (पा॰सू॰~६.१.१०१)~\arrow सहायार्थ~\arrow विभक्तिकार्यम्~\arrow सहायार्थ~अम्~\arrow \textcolor{red}{अमि पूर्वः} (पा॰सू॰~६.१.१०७)~\arrow सहायार्थम्।}\end{sloppypar}
\begin{sloppypar}\hyphenrules{nohyphenation}\justifying\noindent\hspace{10mm} \end{sloppypar}
\section[पुत्रीयम्]{पुत्रीयम्}
\centering\textcolor{blue}{गृहाण पायसं दिव्यं पुत्रीयं देवनिर्मितम्।\nopagebreak\\
लप्स्यसे परमात्मानं पुत्रत्वेन न संशयः॥}\nopagebreak\\
\raggedleft{–~अ॰रा॰~१.३.८}\\
\begin{sloppypar}\hyphenrules{nohyphenation}\justifying\noindent\hspace{10mm} अत्र \textcolor{red}{पुत्रे भवं पुत्रीयम्} इति भवार्थे \textcolor{red}{छ}प्रत्यये\footnote{\textcolor{red}{तत्र भवः} (पा॰सू॰~४.३.५३) इत्यनेन।} तस्य चेयादेशे\footnote{\textcolor{red}{आयनेयीनीयियः फढखच्छघां प्रत्ययादीनाम्‌} (पा॰सू॰~७.१.२) इत्यनेन।} विभक्ति\-कार्ये \textcolor{red}{पुत्रीयम्} इति। किन्तु भवार्थस्यानुपयोगादत्र \textcolor{red}{पुत्राय हितम्} इति विग्रहे \textcolor{red}{छ}\-प्रत्यये\footnote{\textcolor{red}{तस्मै हितम्} (पा॰सू॰~५.१.५) इत्यनेन।} शब्द\-सिद्धिः। यद्वा \textcolor{red}{आत्मनः पुत्रमिच्छति} इति विग्रहे \textcolor{red}{सुप आत्मनः क्यच्} (पा॰सू॰~३.१.८) इत्यनेन \textcolor{red}{क्यच्}प्रत्यये \textcolor{red}{क्यचि च} (पा॰सू॰~७.४.३३) इत्यनेन दीर्घ ईकारे धातु\-सञ्ज्ञायां\footnote{\textcolor{red}{सनाद्यन्ता धातवः} (पा॰सू॰~३.१.३२) इत्यनेन।} लटि तिपि शपि \textcolor{red}{पुत्रीयति}।\footnote{यथा भट्टिकाव्ये~– \textcolor{red}{पुत्रीयता तेन वराङ्गनाभिरानायि विद्वान् क्रतुषु क्रियावान्} (भ॰का॰~१.१०) इत्यत्र शत्रन्त\-प्रयोगे।} \textcolor{red}{पुत्रीयतीति पुत्रीयः} इति विग्रहे पचादित्वादच्।\footnote{\textcolor{red}{नन्दि\-ग्रहि\-पचादिभ्यो ल्युणिन्यचः} (पा॰सू॰~३.१.१३४) इत्यनेन।} तच्च भावे।
एवं तदस्त्यस्मिन्निति विग्रहेऽच्प्रत्यये\footnote{\textcolor{red}{अर्शआदिभ्योऽच्} (पा॰सू॰~५.२.१२७) इत्यनेन।} \textcolor{red}{पुत्रीयम्}। अथवा \textcolor{red}{पुत्री} इति पृथक्पदम्।\footnote{यथा रामरक्षास्तोत्रे~– \textcolor{red}{स चिरायुः सुखी पुत्री विजयी विनयी भवेत्} (रा॰र॰स्तो॰~१०)। पुत्र~\arrow \textcolor{red}{अत इनिठनौ} (पा॰सू॰~५.२.११५)~\arrow पुत्र~इनिँ~\arrow पुत्र~इन्~\arrow \textcolor{red}{यचि भम्} (पा॰सू॰~१.४.१८)~\arrow भसञ्ज्ञा~\arrow \textcolor{red}{यस्येति च} (पा॰सू॰~६.४.१४८)~\arrow पुत्र्~इन्~\arrow पुत्रिन्~\arrow विभक्ति\-कार्यम्~\arrow पुत्रिन्~सुँ~\arrow पुत्रिन्~स्~\arrow \textcolor{red}{सौ च} (पा॰सू॰~६.४.१३)~\arrow पुत्रीन्~स्~\arrow \textcolor{red}{हल्ङ्याब्भ्यो दीर्घात्सुतिस्यपृक्तं हल्} (पा॰सू॰~६.१.६८)~\arrow पुत्रीन्~\arrow \textcolor{red}{नलोपः प्रातिपदिकान्तस्य} (पा॰सू॰~८.२.७)~\arrow पुत्री।} अग्निर्यज्ञ\-स्थले प्रादुर्भूय महाराजं दशरथं कथयति यत् \textcolor{red}{त्वं पुत्री दिव्यं पायसं गृहाण यं गृहीत्वा पुत्रत्वोपलक्षितं परमात्मानं लप्स्यसे}।\end{sloppypar}
\section[इत्युक्ते]{इत्युक्ते}
\centering\textcolor{blue}{वृणीष्व वरमित्युक्ते त्वं मे पुत्रो भवामल।\nopagebreak\\
इति त्वया याचितोऽसौ भगवान्भूतभावनः॥}\nopagebreak\\
\raggedleft{–~अ॰रा॰~१.४.१६}\\
\begin{sloppypar}\hyphenrules{nohyphenation}\justifying\noindent\hspace{10mm} अत्राध्यात्म\-रामायणे बाल\-काण्डे चतुर्थ\-सर्गे विश्वामित्रो यज्ञ\-रक्षार्थं श्रीरामं दशरथमयाचत्। तत्र पुत्र\-वात्सल्य\-धिया मही\-पतिना महा\-मुनिः विश्वामित्रः प्रत्याख्यातः। पश्चाद्वसिष्ठो महा\-राजं प्रत्यबोधयद्यत्पुरा त्वया बहु तप्तं\footnote{\textcolor{red}{भवन्तौ तप उग्रं वै तेपाथे बहुवत्सरम्} (अ॰रा॰~१.४.१४)।} तदा \textcolor{red}{वृणीष्व वरम्} इति भगवत्युक्तवति त्वया भगवान् पुत्र\-रूपेण याचितः स एव परमात्मा श्रीराम\-रूपेण तव गृहेऽवातरत्। इह \textcolor{red}{उक्ते} इति भगवद्विशेषणम् \textcolor{red}{उक्तवति} इत्यभिप्रायकम्। प्रायो निष्ठा\-सञ्ज्ञकौ द्वौ प्रत्ययौ क्तः क्तवतुश्च।\footnote{\textcolor{red}{क्तक्तवतू निष्ठा} (पा॰सू॰~१.१.२६)।} क्त\-प्रत्ययः प्रायशः कर्मणि भावे च भवति।\footnote{\textcolor{red}{अदिकर्मणि क्तः कर्तरि च} (पा॰सू॰~३.४.७१) \textcolor{red}{गत्यर्थाकर्मक\-श्लिष\-शीङ्स्थास\-वस\-जन\-रुह\-जी\-र्यतिभ्यश्च} (पा॰सू॰~३.४.७२) इत्याभ्यां कर्तर्यपि भवति।} तथा च सूत्रम् \textcolor{red}{तयोरेव कृत्य\-क्त\-खलर्थाः} (पा॰सू॰~३.४.७०)। क्तवत्तु\-प्रत्ययः कर्तर्येव भवति।\footnote{\textcolor{red}{कर्तरि कृत्} (पा॰सू॰~३.४.६७) इत्यनेन।} किन्त्वत्र \textcolor{red}{उक्ते} इति कथमिति चेत्। \textcolor{red}{ब्रू}\-धातोः (\textcolor{red}{ब्रूञ् व्यक्तायां वाचि} धा॰पा॰~१०४४) भावे \textcolor{red}{क्त}\-प्रत्ययः।\footnote{\textcolor{red}{नपुंसके भावे क्तः} (पा॰सू॰~३.३.११४) इत्यनेन।} \textcolor{red}{ब्रुवो वचिः} (पा॰सू॰~२.४.५३) इत्यनेन \textcolor{red}{वच्} आदेशे \textcolor{red}{वचि\-स्वपि\-यजादीनां किति} (पा॰सू॰~६.१.१५) इत्यनेन सम्प्रसारणे \textcolor{red}{चोः कुः} (पा॰सू॰~८.२.३०) इत्यनेन ककारे विभक्ति\-कार्ये \textcolor{red}{उक्तम्}। तदस्त्यस्मिन्निति \textcolor{red}{उक्तम्}। अर्शआदित्वादच्।\footnote{\textcolor{red}{अर्शआदिभ्योऽच्} पा॰सू॰~५.२.१२७) इत्यनेन।} तस्मिन्नुक्ते। इत्थं मत्वर्थीयेनाच्प्रत्ययेन \textcolor{red}{उक्तवति} इत्यभिप्रायकः \textcolor{red}{उक्ते} इति शब्दोऽपि पाणिनीय एव। यद्वा \textcolor{red}{उक्तमाचष्टे करोति वा} इति विग्रहे \textcolor{red}{तत्करोति तदाचष्टे} (धा॰पा॰ ग॰सू॰~१८७) इति णिचि तिपि शपि \textcolor{red}{उक्तयति}।\footnote{उक्त~\arrow \textcolor{red}{तत्करोति तदाचष्टे} (धा॰पा॰ ग॰सू॰~१८७)~\arrow उक्त~णिच्~\arrow उक्त~इ~\arrow \textcolor{red}{णाविष्ठवत्प्राति\-पदिकस्य पुंवद्भाव\-रभाव\-टिलोप\-यणादि\-परार्थम्} (वा॰~६.४.४८)~\arrow उक्त्~इ~\arrow उक्ति~\arrow \textcolor{red}{सनाद्यन्ता धातवः} (पा॰सू॰~३.१.३२)~\arrow धातुसञ्ज्ञा~\arrow \textcolor{red}{शेषात्कर्तरि परस्मैपदम्} (पा॰सू॰~१.३.७८)~\arrow \textcolor{red}{वर्तमाने लट्} (पा॰सू॰~३.२.१२३)~\arrow उक्ति~लट्~\arrow उक्ति~तिप्~\arrow उक्ति~ति~\arrow \textcolor{red}{कर्तरि शप्‌} (पा॰सू॰~३.१.६८)~\arrow उक्ति~शप्~ति~\arrow उक्ति~अ~ति~\arrow \textcolor{red}{सार्वधातुकार्ध\-धातुकयोः} (पा॰सू॰~७.३.८४)~\arrow उक्ते~अ~ति~\arrow \textcolor{red}{एचोऽयवायावः} (पा॰सू॰~६.१.७८)~\arrow उक्तय्~अ~ति~\arrow उक्तयति।} \textcolor{red}{उक्तयतीत्युक्तः}। कर्तर्यच्।\footnote{उक्ति~\arrow पूर्ववद्धातु\-सञ्ज्ञा~\arrow \textcolor{red}{नन्दि\-ग्रहि\-पचादिभ्यो ल्युणिन्यचः} (पा॰सू॰~३.१.१३४)~\arrow उक्ति~अच्~\arrow अनुबन्धलोपः~\arrow उक्ति~अ~\arrow \textcolor{red}{णेरनिटि} (पा॰सू॰~६.४.५१)~\arrow णिलोपः~\arrow उक्त्~अ~\arrow उक्त~\arrow विभक्तिकार्यम्~\arrow उक्तः।} तस्मिन् \textcolor{red}{उक्ते}। यद्वा \textcolor{red}{उक्ते} इति कर्मण्येव प्रत्ययः। \textcolor{red}{वृणीष्व वरमिति भगवतोक्ते त्वयि} इति न दोषः।\footnote{\textcolor{red}{त्वयि} इत्यध्याहार्यमिति भावः।} अथवा \textcolor{red}{वाक्य उक्ते}।\footnote{\textcolor{red}{वाक्ये} इत्यध्याहार्यमिति भावः।}\end{sloppypar}
\section[प्रमुदितान्तरः]{प्रमुदितान्तरः}
\centering\textcolor{blue}{वसिष्ठेनैवमुक्तस्तु राजा दशरथस्तदा।\nopagebreak\\
कृतकृत्यमिवात्मानं मेने प्रमुदितान्तरः॥}\nopagebreak\\
\raggedleft{–~अ॰रा॰~१.४.२१}\\
\begin{sloppypar}\hyphenrules{nohyphenation}\justifying\noindent\hspace{10mm} श्रीरघुनाथस्याऽध्यात्मिक\-रहस्यं वसिष्ठ\-मुखान्निशम्य महा\-राजो मुमुदे। अत्र प्रयोगो वर्तते \textcolor{red}{प्रमुदितान्तरः}। \textcolor{red}{प्रमुदितमन्तरं यस्य स प्रमुदितान्तरः} इति।
\textcolor{red}{प्रामोद्यत} इति \textcolor{red}{प्रमोदितम्}। अयं भूत\-कालानुसारेण विग्रहः। णिचि \textcolor{red}{पुगन्त\-लघूपधस्य च} (पा॰सू॰~७.३.८६) इत्यनेन गुणे धातुसञ्ज्ञायां क्त\-प्रत्यये सति \textcolor{red}{आर्धधातुकस्येड्वलादेः} (पा॰सू॰~७.२.३५) इत्यनेनेटि कृते णिलोपे विभक्ति\-कार्ये \textcolor{red}{प्रमोदितम्} इत्येव पाणिनीयम्।\footnote{यथा \textcolor{red}{अनुमोदितम् आमोदितम्} इत्यादिषु। वाचस्पत्येऽपि~– \textcolor{red}{अनुमोदित। त्रि॰ अनु मुद-णिच्-कर्म्मणि क्तः}। \textcolor{red}{आमोदित। त्रि॰ आ मुद-णिच्-क्तः}। \textcolor{red}{मुदँ हर्षे} (धा॰पा॰~१६)~\arrow मुद्~\arrow \textcolor{red}{हेतुमति च} (पा॰सू॰~३.१.२६)~\arrow मुद्~णिच्~\arrow मुद्~इ~\arrow \textcolor{red}{पुगन्त\-लघूपधस्य च} (पा॰सू॰~७.३.८६)~\arrow मोद्~इ~\arrow मोदि~\arrow \textcolor{red}{सनाद्यन्ता धातवः} (पा॰सू॰~३.१.३२)~\arrow धातु\-सञ्ज्ञा। प्र~मोदि~\arrow \textcolor{red}{तयोरेव कृत्य\-क्तखलर्थाः} (पा॰सू॰~३.४.७०)~\arrow \textcolor{red}{निष्ठा} (पा॰सू॰~३.२.१०२)~\arrow प्र~मोदि~क्त~\arrow प्र~मोदि~त~\arrow \textcolor{red}{आर्धधातुकस्येड्वलादेः} (पा॰सू॰~७.२.३५)~\arrow \textcolor{red}{आद्यन्तौ टकितौ} (पा॰सू॰~१.१.४६)~\arrow प्र~मोदि~इट्~त~\arrow प्र~मोदि~इ~त~\arrow \textcolor{red}{निष्ठायां सेटि} (पा॰सू॰~६.४.५२)~\arrow प्र~मोद्~इ~त~\arrow प्रमोदित~\arrow विभक्तिकार्यम्~\arrow प्रमोदित~सुँप्~\arrow \textcolor{red}{अतोऽम्} (पा॰सू॰~७.१.२४)~\arrow प्रमोदित~अम्~\arrow \textcolor{red}{अमि पूर्वः} (पा॰सू॰~६.१.१०७)~\arrow प्रमोदितम्।}
न च \textcolor{red}{पुगन्तलघूपधस्य च} इति गुणोऽत्र न
प्रवर्त्स्यत्यतो नापाणिनीयता।\footnote{\textcolor{red}{प्रवर्त्स्यति} इत्यत्र \textcolor{red}{वृद्भ्यः स्यसनोः} (पा॰सू॰~१.३.९२) इत्यनेन परस्मैपदम्।} यतो हि \textcolor{red}{सार्वधातुकार्ध\-धातुकयोः} (पा॰सू॰~७.३.८४) इति सम्पूर्णस्य सूत्रस्यात्रानुवृत्तिस्तत्र च सप्तमी। एवं \textcolor{red}{मिदेर्गुणः} (पा॰सू॰~७.३.८२) इत्यतो \textcolor{red}{गुण}\-पदमनुवर्त्यते। तथा च \textcolor{red}{इको गुण\-वृद्धी} (पा॰सू॰~१.१.३) इति सूत्रेण षष्ठ्यन्तम् \textcolor{red}{इकः} इति पदमप्युपतिष्ठते। यतो हि षड्विध\-सूत्रेषु \textcolor{red}{इको गुण\-वृद्धी} इति सूत्रं परिभाषा\-सूत्रम्। \textcolor{red}{परिभाषा नामानियमे नियमकारिणी}। अर्थाद्गुणः कुत्र भवेदित्यनियमे सतीयं नियमयति यदिक एव गुणः स्यात्। तथा हि सूत्रार्थः \textcolor{red}{गुण\-वृद्धि\-शब्दाभ्यां यत्र गुण\-वृद्धी विधीयेते तत्र ‘इक’ इति षष्ठ्यन्तं पदमुपतिष्ठते} (वै॰सि॰कौ॰~३४)। \textcolor{red}{गुण\-वृद्धि\-शब्दाभ्याम्} इत्यत्र हि पञ्चमी। सा च \textcolor{red}{गुण\-वृद्धि\-शब्दावुच्चार्य} इत्यर्थे। उच्चार्य\-रूप\-ल्यबन्त\-कर्मणि \textcolor{red}{गुण\-वृद्धि\-शब्द}\-इत्यत्र \textcolor{red}{ल्यब्लोपे कर्मण्यधिकरणे च} (पा॰सू॰~२.३.२८) इति वार्त्तिकेन \textcolor{red}{श्वशुराज्जिह्रेति} (वै॰सि॰कौ॰~५९४) इतिवत्पञ्चमी। अर्थात् \textcolor{red}{गुण\-वृद्धी} उच्चार्य यत्र गुण\-वृद्धी विधीयेते तत्र \textcolor{red}{इकः} इति षष्ठ्यन्तमुपतिष्ठते। एवं गुण\-पद\-प्रयोज्य\-विधेयताश्रय\-भूते गुणे विधीयमाने तथा वृद्धि\-पद\-प्रयोज्य\-विधेयताश्रय\-भूतायां वृद्धौ विधीयमानायाम् \textcolor{red}{इकः} इति षष्ठ्यन्तं पदमुपतिष्ठत इति। अर्थात् \textcolor{red}{मिदेर्गुणः} (पा॰सू॰~७.३.८२) इति\-सूत्रादनुवृत्त\-लघूपध\-गुण\-विधायक\-सूत्रस्थ\-गुण\-पद\-प्रयोज्य\-विधेयताश्रयो गुणो विधीयतेऽतः षष्ठ्यन्तम् \textcolor{red}{इकः} इति पदमुपतिष्ठते। तस्य च \textcolor{red}{लघूपधस्य} इत्यनेनान्वये षष्ठी\-समभिव्याहारादवयवावयवि\-भाव\-सम्बन्धः। एवं \textcolor{red}{सार्वधातुकार्धधातुकाव्यवहित\-पूर्वत्व\-विशिष्टस्य पुगन्त\-लघूपधाङ्गावयवस्येकः स्थाने गुणः} अयं हि परिष्कृत\-सूत्रार्थः। तथा चात्र \textcolor{red}{णिच्} इत्यार्धधातुक\-सञ्ज्ञः। \textcolor{red}{आर्धधातुकं शेषः} (पा॰सू॰~३.४.११४) इति सूत्रविहिता सा सञ्ज्ञा। तथा च \textcolor{red}{प्रमुद्~इ~त}\footnote{\textcolor{red}{प्रमुद्~णिच्~क्त} इति भावः।} इत्यत्र \textcolor{red}{इक्} मकारानन्तरमुकारः। किन्तु तदार्धधातुकाव्यवहितं नास्ति दकारस्य व्यवधानात्। अत एव गुणाप्राप्तौ \textcolor{red}{प्रमुदित} इति कामं पाणिनीय इति चेत्। \textcolor{red}{येन नाव्यवधानं तद्व्यवहितेऽपि वचन\-प्रामाण्यात्} (भा॰पा॰सू॰~७.२.३, ७.३.४४, ७.३.५४, ७.४.१, ७.४.९३)। \textcolor{red}{येन} इति कर्तरि तृतीया। अर्थात् \textcolor{red}{यत्कर्तृकमव्यवधानं न तेन व्यवहितेऽपि वचन\-प्रामाण्याद्गुणो भवेत्}। यतो ह्ययं लघूपध\-गुणः।
\textcolor{red}{उपधात्वं नाम समुदाय\-विशिष्टवर्णत्वम्}।\footnote{\textcolor{red}{समुदाय\-विशिष्ट\-वर्णत्वमुपधात्वम्} (ल॰शे॰)।}
\textcolor{red}{वैशिष्ट्यञ्च स्वघटकत्व\-स्वघटकान्त्यालवधिक\-पूर्वत्व\-विशिष्टत्वम्} इति।\footnote{\textcolor{red}{वैशिष्ट्यञ्च स्वघटकत्व\-स्वघटकान्त्यालवधिक\-पूर्वत्वोभयसम्बन्धेन} (ल॰शे॰)।} एतदुभय\-सम्बन्धेन। तथा च सूत्रम् \textcolor{red}{अलोऽन्त्यात्पूर्व उपधा} (पा॰सू॰~१.१.६५)। \textcolor{red}{अन्त्यादलः पूर्वो वर्ण उपधा\-सञ्ज्ञः स्यात्} (वै॰सि॰कौ॰~२४९)। एवं \textcolor{red}{लघ्व्युपधा लघूपधा पुगन्तञ्च लघूपधा चेति} समाहार\-द्वन्द्वे तस्य च \textcolor{red}{अङ्गस्य} इत्यनेनावयवावयवि\-भाव\-सम्बन्धेनान्वय एवम् \textcolor{red}{अङ्गावयवं सार्वधातुकार्धधातुकाव्यवहित\-पूर्वत्व\-विशिष्टं यत्पुगन्तं लघूपधञ्च तस्येको गुणः} इति सुपरिष्कृतः सूत्रार्थः। तथा च लघूपध\-शब्द\-योगेन वर्ण\-मात्र\-व्यवधानं त्वनिवार्यमेव। तदभाव उपधा कुतः। यतो ह्यन्त्यादलः पूर्वस्थैव सा। अतः \textcolor{red}{येन नाव्यवधानं तेन व्यवहितेऽपि वचन\-प्रामाण्यात्} (वै॰सि॰कौ॰~२१८९) इति प्रवर्तते। एवं च \textcolor{red}{प्रमुद्~इ~त}\footnote{\textcolor{red}{प्रमुद्~णिच्~क्त} इति भावः।} इत्यत्र दकार\-कर्तृकं व्यवधानं त्वनिवार्यम्। अत एक\-वर्ण\-कर्तृक\-व्यवधानमत्र सोढव्यं \textcolor{red}{येन} इत्येक\-वचन\-प्रयोगात्। तेन \textcolor{red}{भिनत्ति} इत्यत्रानेक\-व्यवहित इकि न गुणः।\footnote{\textcolor{red}{अनेक\-व्यवहित इकि} इत्यत्र \textcolor{red}{यस्य च भावेन भावलक्षणम्‌} (पा॰सू॰~२.३.३७) इत्यनेन भावलक्षणा सप्तमी। अनेक\-व्यवहित इकि तस्येको न गुण इति भावः। \textcolor{red}{भिदिँर् विदारणे} (धा॰पा॰~१४३९)~\arrow भिद्~\arrow \textcolor{red}{शेषात्कर्तरि परस्मैपदम्} (पा॰सू॰~१.३.७८)~\arrow \textcolor{red}{वर्तमाने लट्} (पा॰सू॰~३.२.१२३)~\arrow भिद्~तिप्~\arrow भिद्~ति~\arrow \textcolor{red}{रुधादिभ्यः श्नम्} (पा॰सू॰~३.१.७८)~\arrow \textcolor{red}{मिदचोऽन्त्यात्परः} (पा॰सू॰~१.१.४७)~\arrow भि~श्नम्~द्~ति~\arrow भि~न~द्~ति~\arrow \textcolor{red}{खरि च} (पा॰सू॰~८.४.५५)~\arrow भि~न~त्~ति~\arrow भिनत्ति।} \textcolor{red}{भेत्ता} इत्यादौ गुणः।\footnote{\textcolor{red}{भिदिँर् विदारणे} (धा॰पा॰~१४३९)~\arrow भिद्~\arrow \textcolor{red}{ण्वुल्तृचौ} (पा॰सू॰~३.१.१३३)~\arrow भिद्~तृच्~\arrow भिद्~तृ~\arrow \textcolor{red}{पुगन्त\-लघूपधस्य च} (पा॰सू॰~७.३.८६)~\arrow भेद्~तृ~\arrow \textcolor{red}{खरि च} (पा॰सू॰~८.४.५५)~\arrow भेत्~तृ~\arrow भेत्तृ~\arrow विभक्तिकार्यम्~\arrow भेत्तृ~सुँ~\arrow भेत्तृ~स्~\arrow \textcolor{red}{ऋदुशनस्पुरुदंसोऽनेहसां च} (पा॰सू॰~७.१.९४)~\arrow भेत्त्~अनँङ्~स्~\arrow भेत्त्~अन्~स्~\arrow \textcolor{red}{अप्तृन्तृच्स्वसृ\-नप्तृ\-नेष्टृ\-त्वष्टृ\-क्षत्तृ\-होतृ\-पोतृ\-प्रशास्तॄणाम्} (पा॰सू॰~६.४.११)~\arrow भेत्त्~आन्~स्~\arrow \textcolor{red}{हल्ङ्याब्भ्यो दीर्घात्सुतिस्यपृक्तं हल्} (पा॰सू॰~६.१.६८)~\arrow भेत्त्~आन्~\arrow \textcolor{red}{नलोपः प्रातिपदिकान्तस्य} (पा॰सू॰~८.२.७)~\arrow भेत्त्~आ~\arrow भेत्ता।} इदं सर्वं सिद्धान्त\-कौमुद्यां भ्वादिप्रकरणे प्रपञ्चितम्।\footnote{\textcolor{red}{येन नाव्यवधानं तेन व्यवहितेऽपि। वचन\-प्रामाण्यात्। तेन भिनत्तीत्यादावनेक\-व्यवहितस्येको न गुणः} (वै॰सि॰कौ॰~२१८९)।} अतोऽत्र गुण\-प्राप्तिर्दुर्वारेति। अत्रोच्यते। वस्तुतोऽयं न णिजन्तोऽपि तु शुद्धात् \textcolor{red}{मुदँ हर्षे} (धा॰पा॰~१६) इति धातोः कर्तरि क्तान्तप्रयोगः।\footnote{\textcolor{red}{गत्यर्थाकर्मक\-श्लिष\-शीङ्स्थास\-वस\-जन\-रुह\-जीर्यतिभ्यश्च} (पा॰सू॰~३.४.७२) इत्यनेन कर्तरि क्तः। \textcolor{red}{हृष्टो मत्तस्तृप्तः प्रह्लन्नः प्रमुदितः प्रीतः} (अ॰को॰~३.१.१०३) इत्यमरः। तत्रैव सुधाटीकायां \textcolor{red}{प्रमुदितम्। प्रमोदते स्म} (अ॰को॰ व्या॰सु॰~३.१.१०३) इति भानुजि\-दीक्षिताः। कर्तरि \textcolor{red}{पुगन्त\-लघूपधस्य च} (पा॰सू॰~७.३.८६) इत्यनेन गुणे प्राप्ते \textcolor{red}{ग्क्ङिति च} (पा॰सू॰~१.१.५) इत्यनेन गुणनिषेधे \textcolor{red}{प्रमुदितम्} इत्येव। \textcolor{red}{उदुपधाद्भावादि\-कर्मणोरन्य\-तरस्याम्} (पा॰सू॰~१.२.२१) इत्यनेनाऽदिकर्मणि भावे च निष्ठाया वैकल्पिक\-कित्त्वात् \textcolor{red}{ग्क्ङिति च} (पा॰सू॰~१.१.५) इत्यनेन प्राप्तस्य गुणनिषेधस्यापि वैकल्पिकत्वात् \textcolor{red}{प्रमुदितम् प्रमोदितम्} इति रूपद्वयम्।} यद्वा नायं क्तान्तः प्रयोगोऽपि तु \textcolor{red}{प्रमुद्यत इति प्रमुत्}\footnote{यद्वा \textcolor{red}{मुत्प्रीतिः प्रमदो हर्षः प्रमोदामोदसम्मदाः} (अ॰को॰~१.४.२४) इत्यमरकोशानुशासनात् \textcolor{red}{प्रकृष्टा मुत् प्रमुत्}। यथा भागवते~– \textcolor{red}{तदङ्ग\-सङ्ग\-प्रमुदाकुलेन्द्रियाः} (भा॰पु॰~१०.३३.१८)। अत्रान्वितार्थ\-प्रकाशिका~–\textcolor{red}{तस्य भगवतः अङ्गसङ्गेन प्रकृष्टा या मुत् हर्षस्तया आकुलानि अवशानीन्द्रियाणी यासां ताः} (भा॰पु॰ अ॰प्र॰टी॰~१०.३३.१८)। एवमेव श्रीभार्गव\-राघवीये~– \textcolor{red}{वक्राणि मूर्ध्ना प्रमुदा वहन्ती} (भा॰रा॰~१३.९) \textcolor{red}{निशम्य पौराः प्रमुदा समाययुः} (भा॰रा॰~१७.११) इत्यादौ।} इति विग्रहे भावे \textcolor{red}{क्विप्‌}\-प्रत्यये\footnote{\textcolor{red}{सम्पदादिभ्‍यः क्विप्} (वा॰~३.३.१०८) इत्यनेन।} सर्वापहारि\-लोपे \textcolor{red}{सा सञ्जाताऽस्य} इति सञ्जातार्थ आकृति\-गणतया तारकादित्वात् \textcolor{red}{इतच्‌}\-प्रत्यये\footnote{\textcolor{red}{तदस्य सञ्जातं तारकादिभ्य इतच्} (पा॰सू॰~५.२.३६) इत्यनेन।} चकारानुबन्ध\-कार्ये विभक्ति\-कार्ये \textcolor{red}{प्रमुदितम्} इति। यद्वा \textcolor{red}{प्रमुदमितं प्रमुदितम्} इति विग्रहे \textcolor{red}{द्वितीया} इति योग\-विभागेन \textcolor{red}{द्वितीया श्रितातीत\-पतित\-गतात्यस्त\-प्राप्तापन्नैः} (पा॰सू॰~२.१.२४) इत्यनेन समासे \textcolor{red}{प्रमुदितम्} इति पाणिनीयमेव। \textcolor{red}{प्रमुदितमन्तरं यस्य प्रमुदितान्तरः} इति साधनिका\-प्रकारः।\end{sloppypar}
\section[क्षुत्क्षामादि]{क्षुत्क्षामादि}
\centering\textcolor{blue}{ददौ बलां चातिबलां विद्ये द्वे देवनिर्मिते।\nopagebreak\\
ययोर्ग्रहणमात्रेण क्षुत्क्षामादि न जायते॥}\nopagebreak\\
\raggedleft{–~अ॰रा॰~१.४.२५}\\
\begin{sloppypar}\hyphenrules{nohyphenation}\justifying\noindent\hspace{10mm} अत्र विश्वामित्रो यज्ञ\-रक्षार्थं सौमित्रि\-सहितं रण\-धीरं श्रीरघु\-वीरं सिद्धाश्रमं समानयन् सरयू\-तटे श्रीरामभद्राय बलामतिबलां चैव द्वे विद्ये प्रयच्छति स्म ययोर्ग्रहणमात्रेण क्षुत्पिपासे न लगतः।\footnote{\textcolor{red}{लगतः} इत्यत्र \textcolor{red}{लगेँ सङ्गे} (धा॰पा॰~७८६) इति धातुः।} तत्र \textcolor{red}{क्षुत्क्षामादि} इति प्रयोगो वर्तते। \textcolor{red}{क्षुच्च क्षामं चेति क्षुत्क्षामौ ते आदौ यस्य तत्} इति। एवं क्षाम\-शब्दो हि प्रायः कर्तरि\footnote{\textcolor{red}{क्षै क्षये} (धा॰पा॰~९१३) इत्यकर्मक\-धातोः \textcolor{red}{गत्यर्थाकर्मक\-श्लिष\-शीङ्स्थास\-वस\-जन\-रुह\-जीर्यतिभ्यश्च} (पा॰सू॰~३.४.७२) इत्यनेन कर्तरि \textcolor{red}{क्त}\-प्रत्यये \textcolor{red}{क्षाम}\-शब्दस्त्रिलिङ्गे। \textcolor{red}{क्षाम इति। ‘आदेचः’ इत्यात्वम्। ‘गत्यर्थाकर्मके’ति कर्तरि क्तः। क्षीण इत्यर्थः। अन्तर्भावितण्यर्थत्वे क्षपित इत्यर्थः} (बा॰म॰~३०३२) इति बालमनोरमा। यथा वाल्मीकीय\-रामायणे~– \textcolor{red}{तां क्षामां सुविभक्ताङ्गीं विनाभरणशोभिनीम्। प्रहर्षमतुलं लेभे मारुतिः प्रेक्ष्य मैथिलीम्॥} (वा॰रा॰~५.१५.३०) इति पाठे। अत्रत्या तिलक\-टीका~– \textcolor{red}{क्षामां कृशाम्} (वा॰रा॰ ति॰टी॰~५.१५.३०)। \textcolor{red}{क्षमाम्} इति गोविन्दराज\-सम्मतः पाठः। एवमेव भागवते \textcolor{red}{नातिक्षामं भगवतः स्निग्धापाङ्गावलोकनात्} (भा॰पु॰~३.२१.४६) \textcolor{red}{कालेन भूयसा क्षामां कर्शितां व्रतचर्यया} (भा॰पु॰~३.२३.५१) इत्यनयोः। तथा च मेघदूते \textcolor{red}{क्षामच्छायम्} (मे॰दू॰~२.१७) \textcolor{red}{मध्ये क्षामा} (मे॰दू॰~२.१९) \textcolor{red}{आधिक्षामाम्} (मे॰दू॰~२.२६) इत्यादिषु। प्रणेतॄणां भृङ्गदूताभिधे दूतकाव्येऽपि \textcolor{red}{श्यामां क्षामां क्षपितहृदिभूकुङ्कुमामश्रुधारा\-सारैर्नित्यं नमितवदनाम्भोरुहां रुग्णचित्ताम्। सूर्येन्दुभ्यामिव विरहितां कौहवीं सान्ध्यवेलां सीतां भीतामिव हरिणिकां द्रक्ष्यसि त्वं शुनीषु॥} (भृ॰दू॰~२.६१) \textcolor{red}{याऽयोध्यायां जननिसविधे वाग्भिरत्युज्ज्वलाभिर्नैच्छद्वस्तुं क्षणमपि तदा साधु सीताऽनुनीता। दूरीभूता मृगमृगयुतः साम्प्रतं सा मृगाक्षी क्षामा श्यामा श्वसिति किमहो कोटिकूटे त्रिकूटे॥} (भृ॰दू॰~२.१३७) इत्यनयोर्वृत्तयोः कर्तर्येव \textcolor{red}{क्षाम}\-शब्दः।} तर्हि कथं पिपासा\-वाचकः। अत्र \textcolor{red}{क्षै क्षये} (धा॰पा॰~९१३) इत्यस्माद्धातोर्भावे \textcolor{red}{क्त}\-प्रत्ययः।\footnote{\textcolor{red}{नपुंसके भावे क्तः} (पा॰सू॰~३.३.११४) इत्यनेन।} \textcolor{red}{आदेच उपदेशेऽशिति} (पा॰सू॰~६.१.४५) इत्यनेनाऽकारादेशे \textcolor{red}{क्षायो मः} (पा॰सू॰~८.२.५३) इत्यनेन मादेशे विभक्ति\-कार्ये \textcolor{red}{क्षामम्}।\end{sloppypar}
\section[क्रोधसम्मूर्च्छिता]{क्रोधसम्मूर्च्छिता}
\centering\textcolor{blue}{तच्छ्रुत्वाऽसहमाना सा ताटका घोररूपिणी।\nopagebreak\\
क्रोधसम्मूर्च्छिता राममभिदुद्राव मेघवत्॥}\nopagebreak\\
\raggedleft{–~अ॰रा॰~१.४.२९}\\
\begin{sloppypar}\hyphenrules{nohyphenation}\justifying\noindent\hspace{10mm} ताटकामवलोक्य श्रीरामभद्राय विश्वामित्र आदेशं ददौ। अग्नौ शलभ इव सा स्वयमेव क्रोध\-सम्मूर्च्छिता रघुनन्दनमभिदुद्राव। अत्र \textcolor{red}{क्रोधेन सम्मूर्च्छिता} इति प्रयोगोऽपाणिनीयो लगति। यतो हि \textcolor{red}{मुर्छाँ मोहस\-मुच्छ्राययोः} (धा॰पा॰~२१२) इति धातोः कर्तरि \textcolor{red}{क्त}\-प्रत्यये \textcolor{red}{राल्लोपः} (पा॰सू॰~६.४.२१) इत्यनेन रेफादङ्गस्य च्छस्य लोपे टापि \textcolor{red}{मूर्ता} इति पाणिनीयम्।\footnote{\textcolor{red}{मुर्छाँ मोह\-समुच्छ्राययोः} (धा॰पा॰~२१२)~\arrow मुर्छ्~\arrow \textcolor{red}{गत्यर्थाकर्मक\-श्लिष\-शीङ्स्थास\-वस\-जन\-रुह\-जी\-र्यतिभ्यश्च} (पा॰सू॰~३.४.७२)~\arrow मुर्छ्~क्त~\arrow मुर्छ्~त~\arrow \textcolor{red}{राल्लोपः} (पा॰सू॰~६.४.२१)~\arrow मुर्~त~\arrow \textcolor{red}{न ध्याख्यापॄमूर्छिमदाम्} (पा॰सू॰~८.२.५७)~\arrow नकारादेश\-निषेधः~\arrow \textcolor{red}{आदितश्च} (पा॰सू॰~७.२.१६)~\arrow इडागम\-निषेधः~\arrow \textcolor{red}{हलि च} (पा॰सू॰~८.२.७७)~\arrow मूर्~त~\arrow मूर्त~\arrow \textcolor{red}{अजाद्यतष्टाप्} (पा॰सू॰~४.१.४)~\arrow मूर्त~टाप्~\arrow मूर्त~आ~\arrow \textcolor{red}{अकः सवर्णे दीर्घः} (पा॰सू॰~६.१.१०१)~\arrow मूर्ता~\arrow विभक्तिकार्यम्~\arrow मूर्ता~सुँ~\arrow \textcolor{red}{हल्ङ्याब्भ्यो दीर्घात्सुतिस्यपृक्तं हल्} (पा॰सू॰~६.१.६८)~\arrow मूर्ता।} \textcolor{red}{मूर्च्छिता} इति कथम्। \textcolor{red}{मूर्च्छनं मूर्च्छा}।\footnote{\textcolor{red}{मुर्छाँ मोह\-समुच्छ्राययोः} (धा॰पा॰~२१२)~\arrow मुर्छ्~\arrow \textcolor{red}{षिद्भिदादिभ्योऽङ्} (पा॰सू॰~३.३.१०४)~\arrow मुर्छ्~अङ्~\arrow मुर्छ्~अ~\arrow \textcolor{red}{उपधायां च} (पा॰सू॰~८.२.७८)~\arrow मूर्छ्~अ~\arrow \textcolor{red}{अजाद्यतष्टाप्‌} (पा॰सू॰~४.१.४)~\arrow मूर्छ्~अ~आ~\arrow \textcolor{red}{अकः सवर्णे दीर्घः} (पा॰सू॰~६.१.१०१)~\arrow मूर्छ्~आ~\arrow मूर्छा~\arrow \textcolor{red}{अचो रहाभ्यां द्वे} (पा॰सू॰~८.४.४६)~\arrow मूर्~छ्~छा~\arrow \textcolor{red}{खरि च} (पा॰सू॰~८.४.५५)~\arrow मूर्~च्~छा~\arrow मूर्च्छा।} \textcolor{red}{मूर्च्छिता} इति तु \textcolor{red}{मूर्च्छाऽस्याः सञ्जाता सा मूर्च्छिता} इति विग्रहे \textcolor{red}{तदस्य सञ्जातं तारकादिभ्य इतच्} (पा॰सू॰~५.२.३६) इत्यनेन \textcolor{red}{इतच्} प्रत्यये भत्वाट्टिलोपे\footnote{\textcolor{red}{यचि भम्} (पा॰सू॰~१.४.१८) इत्यनेन भत्वे \textcolor{red}{यस्येति च} (पा॰सू॰~६.४.१४८) इत्यनेनाऽलोपः।} टापि \textcolor{red}{मूर्च्छिता}। यद्वा \textcolor{red}{मूर्च्छामिता मूर्च्छिता} इति विग्रहे द्वितीया\-तत्पुरुषे शकन्ध्वादित्वात्पर\-रूपे \textcolor{red}{मूर्च्छिता} इति।\footnote{यद्वाऽत्र \textcolor{red}{आदिकर्मणि क्तः कर्तरि च} (पा॰सू॰~३.४.७१) इत्यनेनाऽदिकर्मणि कर्तरि क्तः। ततः \textcolor{red}{विभाषा भावादिकर्मणोः} (पा॰सू॰~७.२.१७) इत्यनेन पाक्षिकेण्निषेध इट्पक्षे टापि विभक्तिकार्ये \textcolor{red}{मूर्च्छिता}। अपि च \textcolor{red}{घोररूपिणी} इत्यत्र \textcolor{red}{न कर्मधारयान्मत्वर्थीयो बहुव्रीहिश्चेत्तदर्थ\-प्रतिपत्ति\-करः} इत्यनेन न पाणिनीयतेति न भ्रमितव्यम्। \pageref{sec:ghorarupinah}तमे पृष्ठे \ref{sec:ghorarupinah} \nameref{sec:ghorarupinah} इति प्रयोगस्य विमर्शं पश्यन्तु।}\end{sloppypar}
\section[प्रस्थिता]{प्रस्थिताः}
\centering\textcolor{blue}{तत्र कामाश्रमे रम्ये कानने मुनिसङ्कुले।\nopagebreak\\
उषित्वा रजनीमेकां प्रभाते प्रस्थिताः शनैः॥}\nopagebreak\\
\raggedleft{–~अ॰रा॰~१.५.१}\\
\begin{sloppypar}\hyphenrules{nohyphenation}\justifying\noindent\hspace{10mm} कानने महर्षि\-विश्वामित्र\-महाभागैरेकां
रजनीमुषित्वा
प्रभाते प्रास्थायि। अत्र प्रोपसर्ग\-संयोजनेन गति\-निवृत्त्यर्थक\-\textcolor{red}{स्था}\-धातोः (\textcolor{red}{ष्ठा गतिनिवृत्तौ} धा॰पा॰~९२८) गत्यर्थकतयाऽकर्मकत्वाभावात्कर्तरि कथं \textcolor{red}{क्त}\-प्रत्ययः। एवं हि \textcolor{red}{प्रस्थिताः} इति स्था\-धातोः कर्तरि \textcolor{red}{क्त}\-प्रत्यये \textcolor{red}{घु\-मा\-स्था\-गा\-पा\-जहाति\-सां हलि} (पा॰सू॰~६.४.६६) इत्यनेनाऽकारस्येकारे विभक्तिकार्ये \textcolor{red}{प्रस्थिताः} इति चेत्। कर्मणोऽविवक्षायां \textcolor{red}{क्त}\-प्रत्ययः।\footnote{\textcolor{red}{धातोरर्थान्तरे वृत्तेर्धात्वर्थेनोपसङ्ग्रहात्। प्रसिद्धेरविवक्षातः कर्मणोऽकर्मिका क्रिया॥} (वा॰प॰~३.७.८८)। अकर्मकत्वात् \textcolor{red}{गत्यर्थाकर्मक\-श्लिष\-शीङ्स्थाऽऽस\-वस\-जन\-रुह\-जीर्यतिभ्यश्च} (पा॰सू॰~३.४.७२) इत्यनेन कर्तरि क्तः।} यद्वा गत्यर्थकतयाऽपि \textcolor{red}{क्त}\-प्रत्यय आपत्त्यभावः।\footnote{सोऽपि \textcolor{red}{गत्यर्थाकर्मक\-श्लिष\-शीङ्स्थाऽऽस\-वस\-जन\-रुह\-जीर्यतिभ्यश्च} (पा॰सू॰~३.४.७२) इत्यनेन।}\end{sloppypar}
\section[कुतः]{कुतः}
\centering\textcolor{blue}{दर्शयस्व महाभाग कुतस्तौ राक्षसाधमौ।\nopagebreak\\
तथेत्युक्त्वा मुनिर्यष्टुमारेभे मुनिभिः सह॥}\nopagebreak\\
\raggedleft{–~अ॰रा॰~१.५.४}\\
\begin{sloppypar}\hyphenrules{nohyphenation}\justifying\noindent\hspace{10mm} अत्र श्रीरामभद्रो राक्षसयोः सुबाहु\-मारीचयोः स्थितिं जिज्ञासते। \textcolor{red}{कुत्र तौ राक्षसाधमौ} इति प्रष्टव्ये \textcolor{red}{कुतः} इति पृच्छति। \textcolor{red}{कुतः} इत्यत्र \textcolor{red}{कस्मात्} इति विग्रहे \textcolor{red}{पञ्चम्यास्तसिल्} (पा॰सू॰~५.३.७) इति तसिलन्त\-प्रयोगः। सा च पञ्चमी विश्लेष\-मूलिका। अत्र विश्लेषाभावेऽपि कथं पञ्चमीति चेत्। अत्र \textcolor{red}{विवक्षाधीनानि कारकाणि भवन्ति}\footnote{मूलं मृग्यम्। यद्वा \textcolor{red}{कर्मादीनामविवक्षा शेषः} (भा॰पा॰सू॰~२.३.५०, २.३.५२, २.३.६७) इत्यस्य तात्पर्यमिदम्।} इति वचनेन पञ्चम्यन्त\-विवक्षया न दोषः। \textcolor{red}{विवक्षाधीनानि कारकाणि भवन्ति} इति वचने किं मानमिति चेत् \textcolor{red}{कारके} (पा॰सू॰~१.४.२३) इति हि सूत्रम्। अत्र प्रथमार्थे सप्तमीति भाष्य\-निर्देशात्\footnote{\textcolor{red}{किमिदं ‘कारके’ इति। सञ्ज्ञानिर्देशः} (भा॰पा॰सू॰~१.४.२३)। अत्र कैयटः~– \textcolor{red}{सञ्ज्ञानिर्देश इति। सुपां सुपो भवन्तीति प्रथमायाः स्थाने सप्तमी कृतेति भावः} (भा॰प्र॰ पा॰सू॰~१.४.२३)।} सूत्रकार\-प्रयोग एव मानम्। अथवाऽवध्यवधिमतोः\footnote{\textcolor{red}{उत्तरादिभ्य एनब्वा स्यादवध्यवधिमतोः सामीप्ये पञ्चमीं विना} (वै॰सि॰कौ॰~१९८४, ५.३.३५)।} \textcolor{red}{कुतः स्थानात्समीपं राक्षसाधमौ} इत्यर्थेऽन्तिक\-शब्द\-योगे पञ्चमी।\footnote{ \textcolor{red}{दूरान्तिकार्थैः षष्ठ्यन्यतरस्याम्} (पा॰सू॰~२.३.३४) इत्यनेन पञ्चमी।}
अथवा \textcolor{red}{कुं पृथ्वीं तनुतः कृशां कुरुत इति कुतः} इति विग्रहः। \textcolor{red}{कु}\-उपपदे \textcolor{red}{तन्‌}\-धातोः (\textcolor{red}{तनुँ विस्तारे} धा॰पा॰~१४६३) औणादिके \textcolor{red}{ड्विन्} प्रत्यये\footnote{\textcolor{red}{कार्याद्विद्यादनूबन्धम्} (भा॰पा॰सू॰~३.३.१) \textcolor{red}{केचिदविहिता अप्यूह्याः} (वै॰सि॰कौ॰~३१६९) इत्यनुसारमूह्योऽ\-त्राविहितो \textcolor{red}{ड्विन्‌}\-प्रत्ययः।} डित्त्व\-सामर्थ्यादभस्यापि टेर्लोपे\footnote{\textcolor{red}{डित्यभस्याप्यनु\-बन्धकरण\-सामर्थ्यात्} (वा॰~६.४.१४३)।} सर्वापहारि\-लोपे च \textcolor{red}{कुत्}। तस्य द्विवचनान्तं रूपं \textcolor{red}{कुतौ}। \textcolor{red}{व्यत्ययो बहुलम्} (पा॰सू॰~३.१.८५) इत्यनेन \textcolor{red}{औ}\-विभक्ति\-स्थाने \textcolor{red}{ङस्‌}\-विभक्तौ \textcolor{red}{कुतः} इति छान्दस\-रूपम्।\footnote{\textcolor{red}{सुप्तिङुपग्रह\-लिङ्गनराणां कालहलच्स्वर\-कर्तृयङां च। व्यत्ययमिच्छति शास्त्रकृदेषां सोऽपि च सिध्यति बाहुलकेन॥} (भा॰पा॰सू॰~३.१.८५)। \textcolor{red}{बहुलग्रहणं सर्वविधि\-व्यभिचारार्थम्} (का॰वृ॰~३.१.८५)।} अर्थात् \textcolor{red}{वसुमतीं कृशयतो राक्षसाधमौ दर्शयस्व}।
यद्वाऽत्र नास्ति \textcolor{red}{तसिल्} अपि तु \textcolor{red}{तसि\-प्रकरण आद्यादिभ्य उपसङ्ख्यानम्} (वा॰~५.४.४४) इत्यनेन सप्तमीतस्तसि प्रत्ययः।\footnote{तसेः सार्व\-विभक्तिकत्वं तदन्तानामाकृति\-गणत्वं च \pageref{fn:yatah}तमे पृष्ठे \ref{fn:yatah}तम्यां पादटिप्पण्यां स्पष्टीकृतम्।
}\end{sloppypar}
\section[त्वरितम्]{त्वरितम्}
\centering\textcolor{blue}{कदाचिन्मुनिवेषेण गौतमे निर्गते गृहात्।\nopagebreak\\
धर्षयित्वाऽथ निरगात्त्वरितं मुनिरप्यगात्॥}\nopagebreak\\
\raggedleft{–~अ॰रा॰~१.५.२२}\\
\begin{sloppypar}\hyphenrules{nohyphenation}\justifying\noindent\hspace{10mm} न चात्र \textcolor{red}{तूर्णम्} इति प्रयोक्तव्ये \textcolor{red}{त्वरितम्} इति प्रयुक्तम्।\footnote{पूर्वपक्षोऽयम्।}
अत्र \textcolor{red}{आदितश्च} (पा॰सू॰~७.२.१६) इत्यनेनेण्निषेधे प्राप्ते \textcolor{red}{रुष्यमत्वर\-सङ्घुषास्वनाम्} (पा॰सू॰~७.२.२८) इत्यनेन निषेधविकल्पः। तस्मात् \textcolor{red}{तूर्णम्} \textcolor{red}{त्वरितम्} इति रूपद्वयम्।\footnote{\textcolor{red}{ञित्वराँ सम्भ्रमे} (धा॰पा॰~७७५)~\arrow तवर्~\arrow \textcolor{red}{नपुंसके भावे क्तः} (पा॰सू॰~३.३.११४)~\arrow त्वर्~क्त~\arrow त्वर्~त~\arrow \textcolor{red}{आर्धधातुकस्येड्वलादेः} (पा॰सू॰~७.२.३५)~\arrow इट्प्राप्तिः~\arrow \textcolor{red}{आदितश्च} (पा॰सू॰~७.२.१६)~\arrow इण्निषेधः~\arrow \textcolor{red}{रुष्यमत्वर\-सङ्घुषास्वनाम्} (पा॰सू॰~७.२.२८)~\arrow वैकल्पिकेण्निषेधः। निषेधपक्षे~– त्वर्~त~\arrow \textcolor{red}{ज्वरत्वर\-श्रिव्यविम\-वामुपधायाश्च} (पा॰सू॰~६.४.२०)~\arrow त्~ऊठ्~र्~त~\arrow त्~ऊ~र्~त~\arrow \textcolor{red}{रदाभ्यां निष्ठातो नः पूर्वस्य च दः} (पा॰सू॰~८.२.४२)~\arrow त्~ऊ~र्~न~\arrow \textcolor{red}{रषाभ्यां नो णः समानपदे} (पा॰सू॰~८.४.१)~\arrow त्~ऊ~र्~ण~\arrow तूर्ण~\arrow विभक्तिकार्यम्~\arrow तूर्णम्। इट्पक्षे~– त्वर्~इट्~त~\arrow तवर्~इ~त~\arrow त्वरित~\arrow विभक्तिकार्यम्~\arrow त्वरितम्।} यद्वा
\textcolor{red}{त्वरामाचष्टे त्वरयति}\footnote{\textcolor{red}{घटादयः षितः} (धा॰पा॰ ग॰सू॰) इत्यनेन \textcolor{red}{त्वर्‌}\-धातोः षित्त्वम्। त्वर्~\arrow \textcolor{red}{षिद्भिदादिभ्योऽङ्} (पा॰सू॰~३.३.१०४)~\arrow त्वर्~अङ्~\arrow तवर्~अ~\arrow \textcolor{red}{अजाद्यतष्टाप्‌} (पा॰सू॰~४.१.४)~\arrow त्वर्~अ~आ~\arrow \textcolor{red}{अकः सवर्णे दीर्घः} (पा॰सू॰~६.१.१०१)~\arrow त्वर्~आ~\arrow त्वरा। त्वरा~\arrow \textcolor{red}{तत्करोति तदाचष्टे} (धा॰पा॰ ग॰सू॰)~\arrow त्वरा~णिच्~\arrow त्वरा~इ~\arrow \textcolor{red}{णाविष्ठवत्प्राति\-पदिकस्य पुंवद्भाव\-रभाव\-टिलोप\-यणादि\-परार्थम्} (वा॰~६.४.४८)~\arrow त्वर्~इ~\arrow त्वरि~\arrow \textcolor{red}{सनाद्यन्ता धातवः} (पा॰सू॰~३.१.३२)~\arrow धातु\-सञ्ज्ञा~\arrow \textcolor{red}{शेषात्कर्तरि परस्मैपदम्} (पा॰सू॰~१.३.७८)~\arrow \textcolor{red}{वर्तमाने लट्} (पा॰सू॰~३.२.१२३)~\arrow तवरि~तिप्~\arrow त्वरि~ति~\arrow \textcolor{red}{कर्तरि शप्‌} (पा॰सू॰~३.१.६८)~\arrow त्वरि~शप्~ति~\arrow त्वरि~अ~ति~\arrow \textcolor{red}{सार्वधातुकार्ध\-धातुकयोः} (पा॰सू॰~७.३.८४)~\arrow त्वरे~अ~ति~\arrow \textcolor{red}{एचोऽयवायावः} (पा॰सू॰~६.१.७८)~\arrow त्वरय्~अ~ति~\arrow त्वरयति।} इत्यस्मात् \textcolor{red}{त्वरितम्}।\footnote{त्वरि~\arrow धातु\-सञ्ज्ञा (पूर्ववत्)~\arrow \textcolor{red}{नपुंसके भावे क्तः} (पा॰सू॰~३.३.११४)~\arrow त्वरि~क्त~\arrow त्वरि~त~\arrow \textcolor{red}{आर्धधातुकस्येड्वलादेः} (पा॰सू॰~७.२.३५)~\arrow त्वरि~इट्~त~\arrow त्वरि~इ~त~\arrow \textcolor{red}{निष्ठायां सेटि} (पा॰सू॰~६.४.५२)~\arrow त्वर्~इ~त~\arrow त्वरित~\arrow विभक्तिकार्यम्~\arrow त्वरितम्।} भावे क्तः।\footnote{\textcolor{red}{नपुंसके भावे क्तः} (पा॰सू॰~३.३.११४) इत्यनेन।} क्रिया\-विशेषणत्वाद्द्वितीया। यद्वा \textcolor{red}{त्वरयेतं यथा स्यात्तथा} इति विग्रहे \textcolor{red}{तृतीया तत्कृतार्थेन गुण\-वचनेन} (पा॰सू॰~२.१.३०) इत्यत्र \textcolor{red}{तृतीया} इति योग\-विभागेन समासः।\footnote{\textcolor{red}{त्वरा~इत} इति स्थिते \textcolor{red}{शकन्ध्वादिषु पर\-रूपं वाच्यम्} (वा॰~६.१.९१) इत्यनेन शकन्ध्वादित्वात्पर\-रूपे \textcolor{red}{त्वरित} इति शेषः।
}\end{sloppypar}
\vspace{2mm}
\centering ॥ इति बालकाण्डीयप्रयोगाणां विमर्शः ॥\nopagebreak\\
\vspace{4mm}
\pdfbookmark[2]{अयोध्याकाण्डम्}{Chap2Part1Kanda2}
\phantomsection
\addtocontents{toc}{\protect\setcounter{tocdepth}{2}}
\addcontentsline{toc}{subsection}{अयोध्याकाण्डीयप्रयोगाणां विमर्शः}
\addtocontents{toc}{\protect\setcounter{tocdepth}{0}}
\centering ॥ अथायोध्याकाण्डीयप्रयोगाणां विमर्शः ॥\nopagebreak\\
\section[अतिविदूयता]{अतिविदूयता}
\centering\textcolor{blue}{हसन्ती मामुपायाति सा किं नैवाद्य दृश्यते।\nopagebreak\\
इत्यात्मन्येव सञ्चिन्त्य मनसाऽतिविदूयता॥}\nopagebreak\\
\raggedleft{–~अ॰रा॰~२.३.३}\\
\begin{sloppypar}\hyphenrules{nohyphenation}\justifying\noindent\hspace{10mm} अत्राध्यात्म\-रामायणेऽयोध्या\-काण्डे तृतीय\-सर्गे श्रीरामराज्याभिषेक\-समाचारं श्रावयितुमिच्छुश्चक्रवर्ती दशरथः
कैकेयि\-भवनं प्रविश्य तां दृष्ट्वा खिन्नेन मनसा पप्रच्छ। अत्र \textcolor{red}{अतिविदूयता} इति प्रयुक्तम्। अयं ह्यात्मने\-पदीयो धातुः। तथा च \textcolor{red}{दूङ् परितापे} (धा॰पा॰~११३३) इत्यस्मात्।\footnote{\textcolor{red}{अनुदात्तङित आत्मने\-पदम्} (पा॰सू॰~१.३.१२) इत्यनेन धातोर्ङित्त्वादात्मने\-पदमिति भावः।} यथा कालिदासोऽपि प्रयुङ्क्ते~–\end{sloppypar}
\centering\textcolor{red}{तया हीनं विधातर्मां कथं पश्यन्न दूयसे।\nopagebreak\\
सिक्तं स्वयमिव स्नेहाद्वन्ध्यमाश्रमवृक्षकम्॥}\nopagebreak\\
\raggedleft{–~र॰वं॰~१.७०}\\
\begin{sloppypar}\hyphenrules{nohyphenation}\justifying\noindent एवञ्च \textcolor{red}{दूयत इति दूयमानं तेन दूयमानेन}। आत्मनेपदीयत्वात् \textcolor{red}{शानच्}प्रत्यय एव सम्भवः। तथा च सूत्रम्~– \textcolor{red}{लटः शतृ\-शानचावप्रथमा\-समानाधिकरणे} (पा॰सू॰~३.२.१२४)। अनेन लड्लकारस्य शतृ\-शानचौ प्राप्तौ तदाऽग्रिम\-सूत्रेण \textcolor{red}{शानच्‌}\-प्रत्यय आत्मने\-पद एव विधीयते \textcolor{red}{तङानावात्मनेपदम्} (पा॰सू॰~१.४.१००) इत्यनेन। तर्हि \textcolor{red}{विदूयता} इति कथमिति चेत्।
\textcolor{red}{विदूयत इति विदूयः} इति विग्रहे पचाद्यच्।\footnote{\textcolor{red}{नन्दि\-ग्रहि\-पचादिभ्यो ल्युणिन्यचः} (पा॰सू॰~३.१.१३४) इत्यनेन।} \textcolor{red}{विदूयस्य भावो विदूयता} इति विग्रहे \textcolor{red}{तस्य भावस्त्वतलौ} (पा॰सू॰~५.१.११९) इत्यनेन \textcolor{red}{तल्‌}\-प्रत्यये टापि \textcolor{red}{विदूयता}। न च \textcolor{red}{मनसाऽति\-विदूयता} इत्यनेन न सामानाधिकरण्यमेकत्र तृतीयान्ताऽपरत्र प्रथमान्तेति। सामानाधिकरण्ये नैव राजाज्ञा। \textcolor{red}{मनसा} इति तृतीया तु \textcolor{red}{इत्थं\-भूत\-लक्षणे} (पा॰सू॰~२.३.२१) इत्यनेन। \textcolor{red}{मनसोपलक्षिताऽतिविदूयता}।\end{sloppypar}
\section[निवारयित्वा]{निवारयित्वा}
\centering\textcolor{blue}{निवारयित्वा तान्सर्वान्कैकेयी रोषमास्थिता।\nopagebreak\\
ततः प्रभातसमये मध्यकक्षमुपस्थिताः॥}\nopagebreak\\
\raggedleft{–~अ॰रा॰~२.३.३५}\\
\begin{sloppypar}\hyphenrules{nohyphenation}\justifying\noindent\hspace{10mm} अत्र श्रीराम\-वन\-वास\-षड्यन्त्रं कार्यान्वितं चिकीर्षुः कैकेयी सर्वानपि मङ्गलोत्सवान्निवारयति। अत्र \textcolor{red}{वारि}\-धातोः\footnote{\textcolor{red}{वृञ् आवरणे} (धा॰पा॰~१८१३)~\arrow वृ~\arrow \textcolor{red}{सत्याप\-पाश\-रूप\-वीणा\-तूल\-श्लोक\-सेना\-लोम\-त्वच\-वर्म\-वर्ण\-चूर्ण\-चुरादिभ्यो णिच्} (पा॰सू॰~३.१.२५)~\arrow वृ~णिच्~\arrow वृ~इ~\arrow \textcolor{red}{अचो ञ्णिति} (पा॰सू॰~७.२.११५)~\arrow \textcolor{red}{उरण् रपरः} (पा॰सू॰~१.१.५१)~\arrow वार्~इ~\arrow वारि~\arrow \textcolor{red}{सनाद्यन्ता धातवः} (पा॰सू॰~३.१.३२)~\arrow धातु\-सञ्ज्ञा।} \textcolor{red}{समान\-कर्तृकयोः पूर्व\-काले} (पा॰सू॰~३.४.२१) इत्यनेन \textcolor{red}{क्त्वा} प्रत्यये सतीटि\footnote{\textcolor{red}{आर्धधातुकस्येड्वलादेः} (पा॰सू॰~७.२.३५) इत्यनेनेडागमः।} \textcolor{red}{सार्वधातुकार्ध\-धातुकयोः} (पा॰सू॰~७.३.८४) इत्यनेन \textcolor{red}{वारि} इत्यस्य गुणेऽयादेशे \textcolor{red}{नि}\-उपसर्गे \textcolor{red}{निवारयित्वा} इत्याशङ्क्य \textcolor{red}{नि}\-पूर्वकात् \textcolor{red}{वारि}\-धातोर्निष्पन्नोऽयं शब्दः। तथा \textcolor{red}{समासेऽनञ्पूर्वे क्त्वो ल्यप्} (पा॰सू॰~७.१.३७) इत्यनेन \textcolor{red}{ल्यप्} प्रत्यये तथा च \textcolor{red}{निवार्य} इति पाणिनीयम्।\footnote{ल्यपो वलादित्वाभावादिण्न। नि~वारि~ल्यप्~\arrow नि~वारि~य~\arrow \textcolor{red}{णेरनिटि} (पा॰सू॰~६.४.५१)~\arrow नि~वार्~य~\arrow निवार्य।} \textcolor{red}{निवारयित्वा} इति कथम्।
उच्यते।
न चात्र समासो विकल्पो विवक्षाधीनो वाऽतः समासाभाव उक्त\-दोष एव नास्ति। \textcolor{red}{कु\-गति\-प्रादयः} (पा॰सू॰~२.२.१८) इत्यनेन हि नित्यसमासः। तर्हि 
\textcolor{red}{नयतीति नीः} इति विग्रहे क्विप्प्रत्यये\footnote{\textcolor{red}{क्विप् च} (पा॰सू॰~३.२.७६) इत्यनेन।} सर्वापहारि\-लोपेऽर्थात्
\textcolor{red}{सर्वान् दुखं नयन्ती वारयित्वा}।\footnote{\textcolor{red}{सुपां सुलुक्पूर्व\-सवर्णाच्छेयाडाड्यायाजालः} (पा॰सू॰~७.१.३९) इत्यनेन \textcolor{red}{नी}\-प्रातिपदिकात्सोश्छान्दसो लुक्। पृषोदरादित्वाद्ध्रस्वः। \textcolor{red}{नि} इति पृथक्पदमिति भावः।} यद्वा \textcolor{red}{निषिद्धं वारं निवारं करोतीति निवारयति}\footnote{निवार~\arrow \textcolor{red}{तत्करोति तदाचष्टे} (धा॰पा॰ ग॰सू॰)~\arrow निवार~णिच्~\arrow निवार~इ~\arrow \textcolor{red}{णाविष्ठवत्प्राति\-पदिकस्य पुंवद्भाव\-रभाव\-टिलोप\-यणादि\-परार्थम्} (वा॰~६.४.४८)~\arrow निवार्~इ~\arrow निवारि~\arrow \textcolor{red}{सनाद्यन्ता धातवः} (पा॰सू॰~३.१.३२)~\arrow धातु\-सञ्ज्ञा~\arrow \textcolor{red}{शेषात्कर्तरि परस्मैपदम्} (पा॰सू॰~१.३.७८)~\arrow \textcolor{red}{वर्तमाने लट्} (पा॰सू॰~३.२.१२३)~\arrow निवारि~तिप्~\arrow निवारि~ति~\arrow \textcolor{red}{कर्तरि शप्‌} (पा॰सू॰~३.१.६८)~\arrow निवारि~शप्~ति~\arrow निवारि~अ~ति~\arrow \textcolor{red}{सार्वधातुकार्ध\-धातुकयोः} (पा॰सू॰~७.३.८४)~\arrow निवारे~अ~ति~\arrow \textcolor{red}{एचोऽयवायावः} (पा॰सू॰~६.१.७८)~\arrow निवारय्~अ~ति~\arrow निवारयति।} इति विग्रह आचार\-णिजन्तात् \textcolor{red}{निवारि}\-धातोः \textcolor{red}{क्त्वा}\-प्रत्ययः।\footnote{निवारि~\arrow धातु\-सञ्ज्ञा (पूर्ववत्)~\arrow निवारि~\arrow \textcolor{red}{समान\-कर्तृकयोः पूर्व\-काले} (पा॰सू॰~३.४.२१)~\arrow निवारि~क्त्वा~\arrow निवारि~त्वा~\arrow \textcolor{red}{आर्धधातुकस्येड्वलादेः} (पा॰सू॰~७.२.३५)~\arrow निवारि~इट्~त्वा~\arrow निवारि~इ~त्वा~\arrow \textcolor{red}{सार्वधातुकार्ध\-धातुकयोः} (पा॰सू॰~७.३.८४)~\arrow नीवारे~इ~त्वा~\arrow \textcolor{red}{एचोऽयवायावः} (पा॰सू॰~६.१.७८)~\arrow निवारय्~इ~त्वा~\arrow निवारयित्वा।} एवं च कैकय्यभिन्नैक\-कर्तृक\-पूर्व\-कालावच्छिन्न\-निषिद्ध\-भूतक\-वारणानुकूलो
व्यापारः।\footnote{\textcolor{red}{निषिद्ध\-भूतक} इत्यत्र \textcolor{red}{यज्ञेभ्य इतीयत्युच्यमाने य एव/एते सञ्ज्ञीभूतका यज्ञास्तत उत्पत्तिः स्यात्} (भा॰पा॰सू॰~४.३.६८, ५.१.९५) इत्यत्र \textcolor{red}{सञ्ज्ञीभूतकाः} इतिवत्समासान्ते \textcolor{red}{कप्‌}\-प्रत्ययः।} इत्थं \textcolor{red}{निवारि}\-धातोः क्त्वा\-प्रत्यये नापाणिनीयता।\footnote{अन्यत्र च~– \textcolor{red}{निवारयित्वा सर्वान्स राजानमिदमब्रवीत्} (ग॰सं॰~१०.१६.५३) \textcolor{red}{निवारयित्वा कृतवाल्लोँक\-नाथो मरुद्गणान्} (वाम॰पु॰~७२.६९)। अयं प्रयोगः \textcolor{red}{प्रार्थयित्वा} इतिवत्। यथा~– \textcolor{red}{तं प्रार्थयित्वा विधिवत्प्रसाद्य च विशेषतः} (ब्रह्मा॰पु॰~२.५४.२८) \textcolor{red}{इति नीराजनं कृत्वा प्रार्थयित्वा निजेश्वरम्} (ना॰पु॰~६६.१४) \textcolor{red}{ततः कृष्णेन भीमेन प्रार्थयित्वा द्विजान्नृपान्} (ग॰सं॰~१०.५७.१) \textcolor{red}{प्रार्थयित्वा द्विजान् भोज्य भुक्ति\-मुक्तिमवाप्नुयात्} (अ॰पु॰~१८४.८) \textcolor{red}{उपेक्षितमशक्त्या वा प्रार्थयित्वा विरोधितम्} (अ॰शा॰~८.५.२८) इत्यादिषु।} अथवा \textcolor{red}{निवारयतीति निवारयित्वा} इति विग्रहे नि\-पूर्वक\-वारि\-धातोः \textcolor{red}{अन्येभ्योऽपि दृश्यन्ते} (पा॰सू॰~३.२.७५) इत्यनेन \textcolor{red}{वनिप्‌}\-प्रत्ययः। पकारस्येत्सञ्ज्ञायां लोप इटि गुणेऽयादेशे \textcolor{red}{ह्रस्वस्य पिति कृति तुक्} (पा॰सू॰~६.१.७१) इत्यनेन तुगागमे विभक्ति\-कार्ये \textcolor{red}{सर्वनामस्थाने चासम्बुद्धौ} (पा॰सू॰~६.४.८) इत्यनेन दीर्घे सुलोपनलोपयोः \textcolor{red}{निवारयित्वा} इति सम्यक्पाणिनीयम्।\footnote{स्त्रीत्वस्याविवक्षेति शेषः। अन्यथा \textcolor{red}{वनो र च} (पा॰सू॰~४.१.७) इत्यनेन \textcolor{red}{निवारयित्वरी} इति स्यात्। निवारि~\arrow धातु\-सञ्ज्ञा पूर्ववत्~\arrow \textcolor{red}{अन्येभ्योऽपि दृश्यन्ते} (पा॰सू॰~३.२.७५)~\arrow निवारि~वनिँप्~\arrow निवारि~वन्~\arrow \textcolor{red}{आर्धधातुकस्येड्वलादेः} (पा॰सू॰~७.२.३५)~\arrow निवारि~इट्~वन्~\arrow निवारि~इ~वन्~\arrow \textcolor{red}{सार्वधातुकार्ध\-धातुकयोः} (पा॰सू॰~७.३.८४)~\arrow निवारे~इ~वन्~\arrow \textcolor{red}{एचोऽयवायावः} (पा॰सू॰~६.१.७८)~\arrow निवारय्~इ~वन्~\arrow \textcolor{red}{ह्रस्वस्य पिति कृति तुक्} (पा॰सू॰~६.१.७१)~\arrow निवारय्~इ~तुँक्~वन्~\arrow निवारय्~इ~त्~वन्~\arrow निवारयित्वन्~\arrow विभक्ति\-कार्यम्~\arrow निवारयित्वन्~सुँ~\arrow निवारयित्वन्~स्~\arrow \textcolor{red}{सर्वनामस्थाने चासम्बुद्धौ} (पा॰सू॰~६.४.८)~\arrow निवारयित्वान्~स्~\arrow \textcolor{red}{हल्ङ्याब्भ्यो दीर्घात्सुतिस्यपृक्तं हल्} (पा॰सू॰~६.१.६८)~\arrow निवारयित्वान्~\arrow \textcolor{red}{नलोपः प्रातिपदिकान्तस्य} (पा॰सू॰~८.२.७)~\arrow निवारयित्वा।} \textcolor{red}{निवारण\-शीला कैकेयी सर्वान्प्रति रोषं प्रकटयन्ती मध्य\-कक्षमाश्रिता} इति ग्रन्थान्वय\-प्रकारः।\end{sloppypar}
\section[भुञ्जन्]{भुञ्जन्}
\centering\textcolor{blue}{बुद्ध्यादिभ्यो बहिः सर्वमनुवर्तस्व मा खिदः।\nopagebreak\\
भुञ्जन्प्रारब्धमखिलं सुखं वा दुःखमेव वा॥}\nopagebreak\\
\raggedleft{–~अ॰रा॰~२.४.४१}\\
\begin{sloppypar}\hyphenrules{nohyphenation}\justifying\noindent\hspace{10mm} अत्र श्रीरामो लक्ष्मणमुपदिशन् \textcolor{red}{भुञ्जन्} इति प्रयुङ्क्ते। \textcolor{red}{भुजँ पालनाभ्यवहारयोः} (धा॰पा॰~१४५४) इत्यस्माद्धातोः \textcolor{red}{भुङ्क्त इति भुञ्जानः} इत्येव रूपं सामान्यतः पाणिनीय\-मते धातोरस्याभ्यवहारार्थ
आत्मनेपदीयत्वात्।\footnote{\textcolor{red}{भुजोऽनवने} (पा॰सू॰~१.३.६६) इत्यनेन।} \textcolor{red}{भुङ्क्ते} इति विग्रहे लड्लकारे कर्तरि \textcolor{red}{लटः शतृ\-शानचावप्रथमा\-समानाधिकरणे} (पा॰सू॰~३.२.१२४) इत्यनेन शतृ\-शानचौ प्राप्तौ। \textcolor{red}{तङानावात्मनेपदम्} (पा॰सू॰~१.४.१००) इत्यनेन \textcolor{red}{शानच्‌}\-प्रत्यये कृते शकारस्यानुबन्ध\-कार्ये चकारानुबन्ध\-लोपे च \textcolor{red}{भुञ्जानः} इत्येव।\footnote{\textcolor{red}{भुजँ पालनाभ्यवहारयोः} (धा॰पा॰~१४५४)~\arrow भुज्~\arrow \textcolor{red}{भुजोऽनवने} (पा॰सू॰~१.३.६६)~\arrow \textcolor{red}{वर्तमाने लट्} (पा॰सू॰~३.२.१२३)~\arrow भुज्~लट्~\arrow \textcolor{red}{लटः शतृ\-शानचावप्रथमा\-समानाधिकरणे} (पा॰सू॰~३.२.१२४)~\arrow भुज्~शानच्~\arrow \textcolor{red}{सार्वधातुकमपित्} (पा॰सू॰~१.२.४)~\arrow शानचो ङित्त्वम्~\arrow भुज्~आन~\arrow \textcolor{red}{रुधादिभ्यः श्नम्}~\arrow \textcolor{red}{मिदचोऽन्त्यात्परः} (पा॰सू॰~१.१.४७)~\arrow भु~श्नम्~ज्~आन~\arrow भु~न~ज्~आन~\arrow \textcolor{red}{श्नसोरल्लोपः} (पा॰सू॰~६.४.१११)~\arrow भु~न्~ज्~आन~\arrow \textcolor{red}{नश्चापदान्तस्य झलि} (पा॰सू॰~८.३.२४)~\arrow भुं~ज्~आन~\arrow \textcolor{red}{अनुस्वारस्य ययि परसवर्णः} (पा॰सू॰~८.४.५८)~\arrow भुञ्~ज्~आन~\arrow भुञ्जान~\arrow \textcolor{red}{कृत्तद्धित\-समासाश्च} (पा॰सू॰~१.२.४६)~\arrow प्रातिपदिक\-सञ्ज्ञा~\arrow विभक्ति\-कार्यम्~\arrow भुञ्जानः।} \textcolor{red}{भुञ्जन्} इति कथमिति चेत् \textcolor{red}{भुङ्क्ते} इति विग्रहे। \textcolor{red}{भुञ्जानं वाऽचष्टे भुञ्जति}\footnote{अत्र \textcolor{red}{इव} इत्यर्थे \textcolor{red}{वा}। \textcolor{red}{उपमायां विकल्पे वा} (अ॰को॰~३.३.२४९) \textcolor{red}{व वा यथा तथैवेवं साम्ये} (अ॰को॰~३.४.९) इति कोषात्।
भुञ्जान~\arrow \textcolor{red}{तत्करोति तदाचष्टे } (धा॰पा॰ ग॰सू॰~१८७)~\arrow भुञ्जान~णिच्~\arrow भुञ्जान~इ~\arrow \textcolor{red}{णाविष्ठवत्प्राति\-पदिकस्य पुंवद्भाव\-रभाव\-टिलोप\-यणादि\-परार्थम् } (वा॰~६.४.४८)~\arrow भुञ्जान्~इ~\arrow भुञ्जानि~\arrow \textcolor{red}{सनाद्यन्ता धातवः} (पा॰सू॰~३.१.३२)~\arrow धातु\-सञ्ज्ञा~\arrow \textcolor{red}{क्विप् च } (पा॰सू॰~३.२.७६)~\arrow भुञ्जानि~क्विँप्~\arrow भुञ्जानि~व्~\arrow \textcolor{red}{वेरपृक्तस्य} (पा॰सू॰~६.१.६७)~\arrow भुञ्जानि~\arrow \textcolor{red}{णेरनिटि } (पा॰सू॰~६.४.५१)~\arrow भुञ्जान्~\arrow \textcolor{red}{सर्वप्राति\-पदिकेभ्य आचारे क्विब्वा वक्तव्यः } (वा॰~३.१.११)~\arrow भुञ्जान्~क्विँप्~\arrow भुञ्जान्~व्~\arrow \textcolor{red}{वेरपृक्तस्य} (पा॰सू॰~६.१.६७)~\arrow भुञ्जान्~\arrow \textcolor{red}{सनाद्यन्ता धातवः} (पा॰सू॰~३.१.३२)~\arrow धातुसञ्ज्ञा~\arrow \textcolor{red}{अन्येष्वपि दृश्यते } (पा॰सू॰~३.२.१०१)~\arrow भुञ्जान्~ड~\arrow भुञ्जान्~अ~\arrow \textcolor{red}{डित्यभस्याप्यनु\-बन्धकरण\-सामर्थ्यात्} (वा॰~६.४.१४३)~\arrow भुञ्ज्~अ~\arrow भुञ्ज~\arrow \textcolor{red}{सर्वप्राति\-पदिकेभ्य आचारे क्विब्वा वक्तव्यः } (वा॰~३.१.११) भुञ्ज~क्विँप्~\arrow भुञ्ज~व्~\arrow \textcolor{red}{वेरपृक्तस्य} (पा॰सू॰~६.१.६७)~\arrow भुञ्ज~\arrow \textcolor{red}{सनाद्यन्ता धातवः} (पा॰सू॰~३.१.३२)~\arrow धातु\-सञ्ज्ञा~\arrow \textcolor{red}{शेषात्कर्तरि परस्मैपदम्} (पा॰सू॰~१.३.७८)~\arrow \textcolor{red}{वर्तमाने लट्} (पा॰सू॰~३.२.१२३)~\arrow भुञ्ज~लट्~\arrow भुञ्ज~तिप्~\arrow भुञ्ज~ति~\arrow \textcolor{red}{कर्तरि शप्‌} (पा॰सू॰~३.१.६८)~\arrow भुञ्ज~शप्~ति~\arrow भुञ्ज~अ~ति~\arrow \textcolor{red}{अतो गुणे} (पा॰सू॰~६.१.९७)~\arrow भुञ्ज~ति~\arrow भुञ्जति।} इति विग्रहे पुनः शतरि \textcolor{red}{भुञ्जन्}।\footnote{भुञ्ज~\arrow पूर्ववद्धातु\-सञ्ज्ञा~\arrow \textcolor{red}{शेषात्कर्तरि परस्मैपदम्} (पा॰सू॰~१.३.७८)~\arrow \textcolor{red}{वर्तमाने लट्} (पा॰सू॰~३.२.१२३)~\arrow भुञ्ज~लट्~\arrow \textcolor{red}{लटः शतृ\-शानचावप्रथमा\-समानाधिकरणे} (पा॰सू॰~३.२.१२४)~\arrow भुञ्ज~शतृँ~\arrow भुञ्ज~अत्~\arrow \textcolor{red}{कर्तरि शप्‌} (पा॰सू॰~३.१.६८)~\arrow भुञ्ज~शप्~अत्~\arrow भुञ्ज~अ~अत्~\arrow \textcolor{red}{अतो गुणे} (पा॰सू॰~६.१.९७)~\arrow भुञ्ज~अत्~\arrow \textcolor{red}{अतो गुणे} (पा॰सू॰~६.१.९७)~\arrow भुञ्जत्~\arrow \textcolor{red}{कृत्तद्धित\-समासाश्च} (पा॰सू॰~१.२.४६)~\arrow प्रातिपदिक\-सञ्ज्ञा~\arrow विभक्ति\-कार्यम्~\arrow भुञ्जत्~सुँ~\arrow \textcolor{red}{उगिदचां सर्वनामस्थानेऽधातोः} (पा॰सू॰~७.१.७०)~\arrow \textcolor{red}{मिदचोऽन्त्यात्परः} (पा॰सू॰~१.१.४७)~\arrow भुञ्ज~नुँम्~त्~सुँ~\arrow भुञ्ज~न्~त्~सुँ~\arrow \textcolor{red}{हल्ङ्याब्भ्यो दीर्घात्सुतिस्यपृक्तं हल्} (पा॰सू॰~६.१.६८)~\arrow भुञ्जत्~न्~\arrow \textcolor{red}{संयोगान्तस्य लोपः} (पा॰सू॰~८.२.२३)~\arrow भुञ्जन्।} यद्वा \textcolor{red}{भुजोऽनवने} (पा॰सू॰~१.३.६६) इति सूत्रेणात्मनेपदं तच्चावन\-भिन्नेऽर्थे \textcolor{red}{अनवने} इति पर्युदासात्। ध्येयं यत्पाणिनि\-सूत्रेषु निषेधस्य द्वे क्रिये प्रसिद्धे पर्युदासः प्रसज्यश्च। पर्युदासस्तदा भवति प्रायो यदा निषेधः समास\-गर्भे यथा \textcolor{red}{स्थानिवदादेशोऽनल्विधौ} (पा॰सू॰~१.१.५६)। अत्र \textcolor{red}{अनल्विधौ} अर्थादल्विधि\-भिन्ने तत्सदृशे तथैवात्रापि। प्रसज्यस्तु यदा स्वतन्त्रो नञ्वाचको नकारो यथा \textcolor{red}{न विभक्तौ तुस्माः} (पा॰सू॰~१.३.४)। अत्र नकारः क्रियान्वयी। लक्षणमित्थं करणीयम्~– \textcolor{red}{पाणिन्युच्चरित\-निषेधार्थ\-बोधकत्वे सति कारकान्वयित्वे सति समास\-गर्भ\-नञ्धर्म\-वत्त्वं पर्युदासत्वम्}। प्रसज्यत्वं तु \textcolor{red}{पाणिन्युच्चरित\-निषेधार्थ\-बोधकत्वे सति क्रियान्वयित्वे सति नञ्धर्मतावच्छेदकतावत्त्वम्}। तथा चोच्यते~–\end{sloppypar}
\centering\textcolor{red}{द्वौ नञर्थौ समाख्यातौ पर्युदासप्रसज्यकौ।\nopagebreak\\
पर्युदासः सदृग्ग्राही प्रसज्यः प्रतिषेधकृत्॥}\nopagebreak\\
\raggedleft{–~इति गुरवः}\\
\begin{sloppypar}\hyphenrules{nohyphenation}\justifying\noindent अतोऽत्रापि \textcolor{red}{अनवने} इति पर्युदासः। अवन\-भिन्नेऽवन\-सदृशे। सादृश्यं चाऽत्र धातु\-वाच्यत्वेन। सदृशत्वं नाम \textcolor{red}{तद्भिन्नत्वे सति तद्गत\-भूयोधर्मवत्त्वम्}। यथा \textcolor{red}{चन्द्रमिव मुखं पश्यति} (का॰सू॰वृ॰~४.२.१३) इत्यत्र चन्द्र\-भिन्नत्वे सति चन्द्र\-गताह्लादकत्वमिति सदृशत्वम्। तथैव \textcolor{red}{अवन\-भिन्नत्वे सत्यवन\-गत\-धातु\-वाचकत्वम्}। एवमत्रावनार्थं प्रसृज्य भोजनार्थ आत्मनेपदम्। यतो हि \textcolor{red}{भुज्} धातोर्द्वयोरर्थयोः शक्तिर्भोजने पालने चैव। \textcolor{red}{भुजँ पालनाभ्यवहारयोः} (धा॰पा॰~१४५४) इति पठितत्वात्। अतोऽवन\-भिन्नेऽर्थ आत्मनेपदं भवत्यवने तु परस्मैपदम्। तत्रैव शतृ\-प्रत्ययः। अर्थात् \textcolor{red}{भुनक्तीति भुञ्जन्} इति विग्रहे \textcolor{red}{भुज्} धातोः वर्तमाने लड्लकारे \textcolor{red}{लटः शतृ\-शानचावप्रथमा\-समानाधिकरणे} (पा॰सू॰~३.२.१२४) इत्यनेन \textcolor{red}{शतृँ}\-प्रत्यये \textcolor{red}{लशक्वतद्धिते} (पा॰सू॰~१.३.८) इत्यनेन शकारस्येत्सञ्ज्ञायां \textcolor{red}{तस्य लोपः} (पा॰सू॰~१.३.९) इत्यनेन लोप ऋकारस्य च \textcolor{red}{उपदेशेऽजनुनासिक इत्} (पा॰सू॰~१.३.२) इत्यनेनेत्सञ्ज्ञायां तेनैव सूत्रेण लोपे प्रातिपदिक\-सञ्ज्ञायां विभक्ति\-कार्ये सौ \textcolor{red}{भुज्~अत्~सु} इति स्थिते \textcolor{red}{रुधादिभ्यः श्नम्} (पा॰सू॰~३.१.७८) इत्यनेन श्नमि प्रत्यये \textcolor{red}{मिदचोऽन्त्यात्परः} (पा॰सू॰~१.१.४७) इत्यनेन जकारात्पूर्वं कृते शकार\-मकारयोरनुबन्ध\-कार्ये \textcolor{red}{श्नसोरल्लोपः} (पा॰सू॰~६.४.१११) इत्यनेनालोपे \textcolor{red}{उगिदचां सर्वनामस्थानेऽधातोः} (पा॰सू॰~७.१.७०) इत्यनेन नुमि सोर्लोपे तकारलोपे च \textcolor{red}{भुञ्जन्}।\footnote{\textcolor{red}{भुजँ पालनाभ्यवहारयोः} (धा॰पा॰~१४५४)~\arrow भुज्~\arrow \textcolor{red}{शेषात्कर्तरि परस्मैपदम्} (पा॰सू॰~१.३.७८)~\arrow \textcolor{red}{वर्तमाने लट्} (पा॰सू॰~३.२.१२३)~\arrow भुज्~लट्~\arrow \textcolor{red}{लटः शतृ\-शानचावप्रथमा\-समानाधिकरणे} (पा॰सू॰~३.२.१२४)~\arrow भुज्~शतृँ~\arrow भुज्~अत्~\arrow \textcolor{red}{रुधादिभ्यः श्नम्}~\arrow \textcolor{red}{मिदचोऽन्त्यात्परः} (पा॰सू॰~१.१.४७)~\arrow भु~श्नम्~ज्~अत्~\arrow भु~न~ज्~अत्~\arrow \textcolor{red}{श्नसोरल्लोपः} (पा॰सू॰~६.४.१११)~\arrow भु~न्~ज्~अत्~\arrow \textcolor{red}{नश्चापदान्तस्य झलि} (पा॰सू॰~८.३.२४)~\arrow भुं~ज्~अत्~\arrow \textcolor{red}{अनुस्वारस्य ययि परसवर्णः} (पा॰सू॰~८.४.५८)~\arrow भुञ्~ज्~अत्~\arrow भुञ्जत्~\arrow \textcolor{red}{कृत्तद्धित\-समासाश्च} (पा॰सू॰~१.२.४६)~\arrow प्रातिपदिक\-सञ्ज्ञा~\arrow विभक्ति\-कार्यम्~\arrow भुञ्जत्~सुँ~\arrow \textcolor{red}{उगिदचां सर्वनामस्थानेऽधातोः} (पा॰सू॰~७.१.७०)~\arrow \textcolor{red}{मिदचोऽन्त्यात्परः} (पा॰सू॰~१.१.४७)~\arrow भुञ्ज~नुँम्~त्~सुँ~\arrow भुञ्ज~न्~त्~सुँ~\arrow \textcolor{red}{हल्ङ्याब्भ्यो दीर्घात्सुतिस्यपृक्तं हल्} (पा॰सू॰~६.१.६८)~\arrow भुञ्जत्~न्~\arrow \textcolor{red}{संयोगान्तस्य लोपः} (पा॰सू॰~८.२.२३)~\arrow भुञ्जन्।} भगवतोऽयमभिप्रायो यत्त्वं प्रारब्धं मा \textcolor{red}{भुङ्क्ष्व} अपि तु \textcolor{red}{भुङ्ग्धि}। कारणमिदं यत्त्वं योगेश्वरोऽसि। योगेश्वरो भोगं न बुभुक्षतेऽपि तु योगमेव युयुक्षते। अतस्त्वं \textcolor{red}{भुङ्ग्धि} मर्यादानुसारं रक्ष पालय। यतो हि लक्ष्मण ईश्वररूपः। ईश्वरश्चक्रवर्ति\-दशरथस्य गृह आत्मानं चतुर्धा कृत्वा प्रकटयाम्बभूवेति रामायण\-पुराणादौ प्रसिद्धम्। यथा वाल्मीकीय\-रामायणे~–\end{sloppypar}
\centering\textcolor{red}{आदिदेवो महाबाहुर्हरिर्नारायणः प्रभुः।\nopagebreak\\
साक्षाद्रामो रघुश्रेष्ठः शेषो लक्ष्मण उच्यते॥}\nopagebreak\\
\raggedleft{–~वा॰रा॰~६.१२८.१२०}\\
\begin{sloppypar}\hyphenrules{nohyphenation}\justifying\noindent एवं श्रीमद्भागवतेऽपि~–\end{sloppypar}
\centering\textcolor{red}{तस्यापि भगवानेष साक्षाद्ब्रह्ममयो हरिः।\nopagebreak\\
अंशांशेन चतुर्धाऽगात्पुत्रत्वं प्रार्थितः सुरैः।\nopagebreak\\
रामलक्ष्मणभरतशत्रुघ्ना इति सञ्ज्ञया॥}\nopagebreak\\
\raggedleft{–~भा॰पु॰~९.१०.२}\\
\begin{sloppypar}\hyphenrules{nohyphenation}\justifying\noindent एवमत्राप्यध्यात्म\-रामायणे \textcolor{red}{शेषस्तु लक्ष्मणो राजन्} (अ॰रा॰~१.४.१७) इति बहुत्र सङ्कीर्तनात्। ईश्वरस्य कर्म\-विपाको न भवति। सूत्रेऽपि \textcolor{red}{क्लेश\-कर्म\-विपाकाशयैरपरामृष्टः पुरुषविशेष ईश्वरः} (यो॰सू॰~१.२४) इति। अत एव~–\end{sloppypar}
\centering\textcolor{red}{न मां कर्माणि लिम्पन्ति न मे कर्मफले स्पृहा।\nopagebreak\\
इति मां योऽभिजानाति कर्मभिर्न स बध्यते॥}\nopagebreak\\
\raggedleft{–~भ॰गी॰~४.१४}\\
\begin{sloppypar}\hyphenrules{nohyphenation}\justifying\noindent इति भगवद्गीतोक्तमपि सङ्गच्छेत। अतो लक्ष्मण ईश्वरः। न वा तस्य प्रारब्धो न वा कर्मबन्धनं न वा बुभुक्षा\-मुमुक्षे। अत एवाऽचतुर्दशाब्दमरण्ये त्यक्त\-निद्रा\-नारी\-भोजनत्वमपि सङ्घटते। अतो भगवाञ्छ्रीरामः कथयति यत् \textcolor{red}{त्वं प्रारब्ध\-भोगाय न विवशोऽपि तु लोक\-सङ्ग्रहार्थमखिलं प्रारब्धं भुञ्जन्नवन् रक्षन्नित्यर्थस्तटस्थो भूत्वा वर्तस्व}। इत्थं लक्ष्मणाभिन्नैक\-कर्तृक\-प्रारब्ध\-कर्मक\-वर्तमान\-कालावच्छिन्न\-प्रारब्ध\-पालनानुकूल\-व्यापाराश्रयो लक्ष्मण इति शाब्द\-बोधः।\end{sloppypar}
\section[उद्वीक्षयन्]{उद्वीक्षयन्}
\label{sec:udviksayan}
\centering\textcolor{blue}{श्रीरामः सह सीतया नृपपथे गच्छन् शनैः सानुजः\nopagebreak\\
पौरान् जानपदान् कुतूहलदृशः सानन्दमुद्वीक्षयन्।\nopagebreak\\
श्यामः कामसहस्रसुन्दरवपुः कान्त्या दिशो भासयन्\nopagebreak\\
पादन्यासपवित्रिताखिलजगत्प्रापालयं तत्पितुः॥}\nopagebreak\\
\raggedleft{–~अ॰रा॰~२.४.८७}\\
\begin{sloppypar}\hyphenrules{nohyphenation}\justifying\noindent\hspace{10mm} अत्र \textcolor{red}{ईक्ष्‌}\-धातुर्दर्शने (\textcolor{red}{ईक्षँ दर्शने} धा॰पा॰~६१०) आत्मनेपदी। स च \textcolor{red}{उद्वि}\-इत्युपसर्ग\-द्वय\-पूर्वकः। तथा चाऽत्मनेपदित्वादत्र शानज्रूपम् \textcolor{red}{उद्वीक्षमाणः} इति पाणिनि\-तन्त्रानुरूपम्। \textcolor{red}{उद्वीक्षयन्} अपि तथैव। अत्र स्वार्थे णिच्। \textcolor{red}{उद्वीक्षते} इत्यर्थे \textcolor{red}{उद्वीक्षयति}।\footnote{स्वार्थे णिचि रामस्य सर्वद्रष्टृत्वात्क्रियाफलस्य परगामित्वात् \textcolor{red}{णिचश्च} (पा॰सू॰~१.३.७४) इत्यस्याप्रवृत्तौ \textcolor{red}{शेषात्कर्तरि परस्मैपदम्} (पा॰सू॰~१.३.७८) इत्यनेन परस्मैपदम्।} \textcolor{red}{उद्वीक्षयतीत्युद्वीक्षयन्}। यद्वा \textcolor{red}{उद्वीक्ष्यत इत्युद्वीक्षा} उद्विपूर्वक\-धातोर्भावे \textcolor{red}{अः} प्रत्ययः स्त्रियाम्।\footnote{\textcolor{red}{गुरोश्च हलः} (पा॰सू॰~३.३.१०३) इत्यनेन। उद्~वि~ईक्ष्~\arrow \textcolor{red}{गुरोश्च हलः} (पा॰सू॰~३.३.१०३)~\arrow उद्~वि~ईक्ष्~अ~\arrow \textcolor{red}{अकः सवर्णे दीर्घः} (पा॰सू॰~६.१.१०१)~\arrow उद्~वीक्ष्~अ~\arrow उद्वीक्ष~\arrow \textcolor{red}{अजाद्यतष्टाप्‌} (पा॰सू॰~४.१.४)~\arrow उद्वीक्ष~टाप्~\arrow उद्वीक्ष~आ~\arrow \textcolor{red}{अकः सवर्णे दीर्घः} (पा॰सू॰~६.१.१०१)~\arrow उद्वीक्षा।} \textcolor{red}{उद्वीक्षां करोत्युद्वीक्षयति} इति विग्रहे \textcolor{red}{तत्करोति तदाचष्टे} (धा॰पा॰ ग॰सू॰~१८७) इत्यनेन णिचि टिलोपादौ \textcolor{red}{उद्वीक्षयति}।\footnote{उद्वीक्षा~\arrow \textcolor{red}{तत्करोति तदाचष्टे } (धा॰पा॰ ग॰सू॰~१८७)~\arrow उद्वीक्षा~णिच्~\arrow उद्वीक्षा~इ~\arrow \textcolor{red}{णाविष्ठवत्प्राति\-पदिकस्य पुंवद्भाव\-रभाव\-टिलोप\-यणादि\-परार्थम् } (वा॰~६.४.४८)~\arrow उद्वीक्ष्~इ~\arrow उद्वीक्षि~\arrow \textcolor{red}{सनाद्यन्ता धातवः} (पा॰सू॰~३.१.३२)~\arrow धातु\-सञ्ज्ञा~\arrow \textcolor{red}{शेषात्कर्तरि परस्मैपदम्} (पा॰सू॰~१.३.७८)~\arrow \textcolor{red}{वर्तमाने लट्} (पा॰सू॰~३.२.१२३)~\arrow उद्वीक्षि~लट्~\arrow उद्वीक्षि~तिप्~\arrow उद्वीक्षि~ति~\arrow \textcolor{red}{कर्तरि शप्‌} (पा॰सू॰~३.१.६८)~\arrow उद्वीक्षि~शप्~ति~\arrow उद्वीक्षि~अ~ति~\arrow \textcolor{red}{सार्वधातुकार्ध\-धातुकयोः} (पा॰सू॰~७.३.८४)~\arrow उद्वीक्षे~अ~ति~\arrow \textcolor{red}{एचोऽयवायावः} (पा॰सू॰~६.१.७८)~\arrow उद्वीक्षय्~ति~\arrow उद्वीक्षयति।} ततः शतरि विभक्ति\-कार्ये \textcolor{red}{उद्वीक्षयन्}। यद्वा \textcolor{red}{पौराः श्रीराममुद्वीक्षन्ते रामस्तान् प्रेरयति} इत्यर्थे \textcolor{red}{रामः पौरानुद्वीक्षयति} इति विग्रहे \textcolor{red}{तत्प्रयोजको हेतुश्च} (पा॰सू॰~१.४.५५) इत्यनेन हेतु\-सञ्ज्ञायां \textcolor{red}{हेतुमति च} (पा॰सू॰~३.१.२६) इत्यनेन णिचि \textcolor{red}{सनाद्यन्ता धातवः} (पा॰सू॰~३.१.३२) इत्यनेन धातु\-सञ्ज्ञायां लटि तिपि शपि गुणेऽयादेशे \textcolor{red}{उद्वीक्षयति}। \textcolor{red}{उद्वीक्षयतीत्युद्वीक्षयन्} इति विग्रहे शतृ\-प्रत्यये पूर्वोक्त\-साधन\-प्रक्रियातः \textcolor{red}{उद्वीक्षयन्} इति पाणिनीयमेव। प्रथम\-पक्षे \textcolor{red}{पौर\-जन\-कर्मक\-दर्शनानुकूल\-व्यापाराश्रयो रामः}। द्वितीय\-कल्पे \textcolor{red}{वर्तमान\-कालावच्छिन्न\-पौर\-जन\-कर्तृक\-राम\-कर्मक\-वीक्षणानुकूल\-व्यापारानुकूल\-व्यापाराश्रयो रामः} इति शाब्द\-बोधः।\end{sloppypar}
\section[गच्छतीम्]{गच्छतीम्}
\label{sec:gacchatim}
\centering\textcolor{blue}{यत्र रामः सभार्यश्च सानुजो गन्तुमिच्छति।\nopagebreak\\
पश्यन्तु जानकीं सर्वे पादचारेण गच्छतीम्॥}\nopagebreak\\
\raggedleft{–~अ॰रा॰~२.५.५}\\
\begin{sloppypar}\hyphenrules{nohyphenation}\justifying\noindent\hspace{10mm} अत्र भगवतीं जनक\-नन्दिनीं सीतां वल्कल\-वस्त्र\-धारिणं पाद\-चारिणं सौमित्रि\-सुख\-कारिणं भक्त\-भय\-हारिणमाप्त\-कामं श्रीराममनुगच्छन्तीं विलोक्य सशोकाः पौर\-लोकाः समालोचमाना निगदन्ति \textcolor{red}{पश्यन्तु जानकीं सर्वे पाद\-चारेण गच्छतीम्}। अत्र \textcolor{red}{गच्छन्तीम्} इति हि पाणिनीयम्। यतो हि \textcolor{red}{गच्छतीति गच्छन्ती ताम्} इति विग्रहे \textcolor{red}{गम्‌}\-धातोः (\textcolor{red}{गमॢँ गतौ} धा॰पा॰~९८२) वर्तमान\-काले लड्लकारे शतृ\-प्रत्ययेऽनुबन्ध\-कार्ये \textcolor{red}{इषुगमियमां छः} (पा॰सू॰~७.३.७७) इत्यनेन छान्तादेशे \textcolor{red}{छे च} (पा॰सू॰~६.१.७३) इत्यनेन तुगागमे श्चुत्वे \textcolor{red}{उगितश्च} (पा॰सू॰~४.१.६) इत्यनेन ङीपि नुम्यमि \textcolor{red}{गच्छन्तीम्}।\footnote{\textcolor{red}{गमॢँ गतौ} (धा॰पा॰~९८२)~\arrow गम्~\arrow \textcolor{red}{शेषात्कर्तरि परस्मैपदम्} (पा॰सू॰~१.३.७८)~\arrow \textcolor{red}{वर्तमाने लट्} (पा॰सू॰~३.२.१२३)~\arrow गम्~लट्~\arrow \textcolor{red}{लटः शतृशानचावप्रथमा\-समानाधिकरणे} (पा॰सू॰~३.२.१२४)~\arrow गम्~शतृँ~\arrow गम्~अत्~\arrow \textcolor{red}{इषुगमियमां छः} (पा॰सू॰~७.३.७७)~\arrow गछ्~अत्~\arrow \textcolor{red}{छे च} (पा॰सू॰~६.१.७३)~\arrow \textcolor{red}{आद्यन्तौ टकितौ} (पा॰सू॰~१.१.४६)~\arrow ग~तुँक्~छ्~अत्~\arrow ग~त्~छ्~अत्~\arrow \textcolor{red}{स्तोः श्चुना श्चुः} (पा॰सू॰~८.४.४०)~\arrow ग~च्~छ्~अत्~\arrow गच्छ्~अत्~\arrow \textcolor{red}{कर्तरि शप्‌} (पा॰सू॰~३.१.६८)~\arrow गच्छ्~शप्~अत्~\arrow गच्छ्~अ~अत्~\arrow \textcolor{red}{अतो गुणे} (पा॰सू॰~६.१.९७)~\arrow गच्छ्~अत्~\arrow \textcolor{red}{उगितश्च} (पा॰सू॰~४.१.६)~\arrow गच्छ्~अत्~ङीप्~\arrow गच्छ्~अत्~ई~\arrow \textcolor{red}{शप्श्यनोर्नित्यम्} (पा॰सू॰~७.१.८१)~\arrow \textcolor{red}{आद्यन्तौ टकितौ} (पा॰सू॰~१.१.४६)~\arrow गच्छ्~अ~नुँम्~त्~ई~\arrow गच्छ्~अ~न्~त्~ई~\arrow गच्छन्ती~\arrow विभक्ति\-कार्यम्~\arrow गच्छन्ती~अम्~\arrow \textcolor{red}{अमि पूर्वः} (पा॰सू॰~६.१.१०७)~\arrow गच्छन्तीम्।} \textcolor{red}{गच्छतीम्} इति कथम्। अत्र हि पृषोदरादित्वान्नुमभावः। यद्वा \textcolor{red}{आगम\-शास्त्रमनित्यम्} (प॰शे॰~९३.३)। अतो नुमभावः कल्प्यताम्। यद्वाऽत्रौणादिकः \textcolor{red}{तृँच्‌}\-प्रत्ययः।\footnote{नायं \textcolor{red}{बहुलमन्यत्रापि} (प॰उ॰~२.९५) इति तृच्। स नोगित्। \textcolor{red}{कार्याद्विद्यादनूबन्धम्} (भा॰पा॰सू॰~३.३.१) \textcolor{red}{केचिदविहिता अप्यूह्याः} (वै॰सि॰कौ॰~३१६९) इत्यनुसारमूह्योऽ\-यमविहित उगित्प्रत्ययः। \textcolor{red}{तृँच्} प्रत्यये चात्र शबागमोऽप्यूह्यः। \textcolor{red}{नयतेः षुगागमः} (प॰उ॰ श्वे॰वृ॰~२.९६) इतिवत्।} ततश्च \textcolor{red}{उगितश्च} (पा॰सू॰~४.१.६) इत्यनेन ङीप्। इत्थम् \textcolor{red}{गच्छतीम्} इति पाणिनीयम्। उणादयः पाणिनि\-सम्मता न वेति चैत्तत्रैव पाणिनि\-सूत्रम् \textcolor{red}{उणादयो बहुलम्} (पा॰सू॰~३.३.१)। तत्रैव कारिका~–\end{sloppypar}
\centering\textcolor{red}{सञ्ज्ञासु धातुरूपाणि प्रत्ययाश्च ततः परे।\nopagebreak\\
कार्याद्विद्यादनूबन्धमेतच्छाशास्त्रमुणादिषु॥}\nopagebreak\\
\raggedleft{–~भा॰पा॰सू॰~३.३.१} \\
\section[विजानती]{विजानती}
\centering\textcolor{blue}{रामस्तु वस्त्राण्युत्सृज्य वन्यचीराणि पर्यधात्।\nopagebreak\\
लक्ष्मणोऽपि तथा चक्रे सीता तन्न विजानती॥}\nopagebreak\\
\raggedleft{–~अ॰रा॰~२.५.३६}\\
\begin{sloppypar}\hyphenrules{nohyphenation}\justifying\noindent\hspace{10mm} अत्रापि \textcolor{red}{विजानाति} इति विग्रहे \textcolor{red}{शतृ}\-प्रत्यये ङीपि नुमि \textcolor{red}{विजानन्ती} इति।\footnote{\setcounter{dummy}{\value{footnote}}\addtocounter{dummy}{-1}\refstepcounter{dummy}\label{fn:jananti}पूर्वपक्षोऽयम्। यथा \textcolor{red}{बलमात्मनि जानन्ती न मां शङ्कितुमर्हसि} (वा॰रा॰~२.१०.३५) \textcolor{red}{जानन्ती बत दिष्ट्या मां वैदेहि परिपृच्छसि} (वा॰रा॰~५.३५.६) \textcolor{red}{साऽहमेतद्विजानन्ती तोषयिष्ये द्विजोत्तमम्} (म॰भा॰~३.२८८.८) \textcolor{red}{एतत्सर्वं विजानन्ती सा क्षमामन्वपद्यत} (म॰भा॰~४.१९.६९) \textcolor{red}{साऽहं धर्मं विजानन्ती धर्मनित्ये त्वयि स्थिते} (म॰भा॰~१२.३४७.१२) \textcolor{red}{अजानन्त्या परं भावं तथाप्यस्त्वाभयाय मे} (भा॰पु॰~३.२३.५४) \textcolor{red}{कृतद्युतिरजानन्ती सपत्‍नीनामघं महत्} (भा॰पु॰~६.१४.४४) \textcolor{red}{द्वे ज्योतिषी अजानन्त्या निर्भिन्ने कण्टकेन वै} (भा॰पु॰~९.३.७) \textcolor{red}{वन इव पुरेऽपि विचरति पुरुषं त्वामेव जानन्ती} (आ॰स॰श॰~४६०) इत्यादिषु।} परमत्र नुमभावः। \textcolor{red}{श्नाभ्यस्तयोरातः} (पा॰सू॰~६.४.११२) इति लोपेन।\footnote{नित्यत्वादन्त\-रङ्गत्वाच्च \textcolor{red}{श्नाभ्यस्तयोरातः} (पा॰सू॰~६.४.११२) इति लोपविधिः \textcolor{red}{आच्छीनद्योर्नुम्} (पा॰सू॰~७.१.८०) इत्यागम\-विधेर्बलीयानिति भावः। वि~\textcolor{red}{ज्ञा अवबोधने} (धा॰पा॰~१५०७)~\arrow वि~ज्ञा~\arrow \textcolor{red}{शेषात्कर्तरि परस्मैपदम्} (पा॰सू॰~१.३.७८)~\arrow \textcolor{red}{वर्तमाने लट्} (पा॰सू॰~३.२.१२३)~\arrow वि~ज्ञा~लट्~\arrow \textcolor{red}{लटः शतृशानचावप्रथमा\-समानाधिकरणे} (पा॰सू॰~३.२.१२४)~\arrow वि~ज्ञा~शतृँ~\arrow वि~ज्ञा~अत्~\arrow \textcolor{red}{क्र्यादिभ्यः श्ना} (पा॰सू॰~३.१.८१)~\arrow वि~ज्ञा~श्ना~अत्~\arrow वि~ज्ञा~ना~अत्~\arrow \textcolor{red}{ज्ञाजनोर्जा} (पा॰सू॰~७.३.७९)~\arrow वि~जा~ना~अत्~\arrow \textcolor{red}{श्नाभ्यस्तयोरातः} (पा॰सू॰~६.४.११२)~\arrow वि~जा~न्~अत्~\arrow \textcolor{red}{उगितश्च} (पा॰सू॰~४.१.६)~\arrow वि~जा~न्~अत्~ङीप्‌~\arrow वि~जा~न्~अत्~ई~\arrow विजानती~\arrow विभक्तिकार्यम्~\arrow विजानती~सुँप्~\arrow \textcolor{red}{हल्ङ्याब्भ्यो दीर्घात्सुतिस्यपृक्तं हल्} (पा॰सू॰~६.१.६८)~\arrow विजानती। कथं तर्हि \ref{fn:jananti}तम्यां टिप्पण्यामुद्धृतेषूदाहरणेषु नुम्भावः। \textcolor{red}{आगम\-शास्त्रमनित्यम्} (प॰शे॰~९३.२)। यद्वा धात्वन्तरः कल्प्यताम्।}
\end{sloppypar}
\section[भाषतोः]{भाषतोः}
\centering\textcolor{blue}{गुहलक्ष्मणयोरेवं भाषतोर्विमलं नभः।\nopagebreak\\
बभूव रामः सलिलं स्पृष्ट्वा प्रातः समाहितः॥}\nopagebreak\\
\raggedleft{–~अ॰रा॰~२.६.१६}\\
\begin{sloppypar}\hyphenrules{nohyphenation}\justifying\noindent\hspace{10mm} अत्र गुह\-लक्ष्मणयोर्भाषतोर्भाष\-माणयो रात्रिर्व्यतीता। \textcolor{red}{भाषतोः} इत्यपाणिनीयमिव। यतो हि भाषणार्थको \textcolor{red}{भाष्‌}\-धातुः (\textcolor{red}{भाषँ व्यक्तायां वाचि} धा॰पा॰~६१२) आत्मनेपदी। ततश्च \textcolor{red}{भाषेते इति भाषमाणौ तयोर्भाषमाणयोः} इति पाणिनीयम्। किन्तु \textcolor{red}{भाषेते इति भाषौ} पचादित्वादच्।\footnote{\textcolor{red}{नन्दि\-ग्रहि\-पचादिभ्यो ल्युणिन्यचः} (पा॰सू॰~३.१.१३४) इत्यनेन।} पुनः \textcolor{red}{भाषाविवाऽचरतो भाषतः}।\footnote{भाष~\arrow \textcolor{red}{सर्वप्राति\-पदिकेभ्य आचारे क्विब्वा वक्तव्यः} (वा॰~३.१.११)~\arrow भाष~क्विँप्~\arrow भाष~व्~\arrow \textcolor{red}{वेरपृक्तस्य} (पा॰सू॰~६.१.६७)~\arrow भाष~\arrow \textcolor{red}{सनाद्यन्ता धातवः} (पा॰सू॰~३.१.३२)~\arrow धातुसञ्ज्ञा~\arrow \textcolor{red}{शेषात्कर्तरि परस्मैपदम्} (पा॰सू॰~१.३.७८)~\arrow \textcolor{red}{वर्तमाने लट्} (पा॰सू॰~३.२.१२३)~\arrow भाष~लट्~\arrow भाष~तस्~\arrow \textcolor{red}{कर्तरि शप्‌} (पा॰सू॰~३.१.६८)~\arrow भाष~शप्~तस्~\arrow भाष~अ~तस्~\arrow \textcolor{red}{अतो गुणे} (पा॰सू॰~६.१.९७)~\arrow भाष~तस्~\arrow \textcolor{red}{ससजुषो रुः} (पा॰सू॰~८.२.६६)~\arrow भाषतरुँ~\arrow \textcolor{red}{खरवसानयोर्विसर्जनीयः} (पा॰सू॰~८.३.१५)~\arrow भाषतः।} आचारार्थे क्विप्। ततश्च धातुत्वाल्लटि तसि शपि। ततः \textcolor{red}{भाषत इति भाषन्तौ}\footnote{भाष~\arrow धातुसञ्ज्ञा (पूर्ववत्)~\arrow \textcolor{red}{शेषात्कर्तरि परस्मैपदम्} (पा॰सू॰~१.३.७८)~\arrow \textcolor{red}{वर्तमाने लट्} (पा॰सू॰~३.२.१२३)~\arrow भाष~लट्~\arrow \textcolor{red}{लटः शतृशानचावप्रथमा\-समानाधिकरणे} (पा॰सू॰~३.२.१२४)~\arrow भाष~शतृँ~\arrow भाष~अत्~\arrow \textcolor{red}{अतो गुणे} (पा॰सू॰~६.१.९७)~\arrow भाषत्~\arrow \textcolor{red}{कृत्तद्धित\-समासाश्च} (पा॰सू॰~१.२.४६)~\arrow प्रातिपदिक\-सञ्ज्ञा~\arrow विभक्तिकार्यम्~\arrow भाषत्~औ~\arrow \textcolor{red}{उगिदचां सर्वनामस्थानेऽधातोः} (पा॰सू॰~७.१.७०)~\arrow \textcolor{red}{मिदचोऽन्त्यात्परः} (पा॰सू॰~१.१.४७)~\arrow भाष~नुँम्~त्~औ~\arrow भाष~न्~त्~औ~\arrow भाषन्तौ। } इति विग्रहे परस्मैपद\-धातोः \textcolor{red}{शतृ}\-प्रत्यये \textcolor{red}{भाषतोः} षष्ठ्यन्तं सप्तम्यन्तं वा।\footnote{भाषत्~\arrow प्रातिपदिक\-सञ्ज्ञा (पूर्ववत्)~\arrow विभक्तिकार्यम्~\arrow भाषत्~ओस्~\arrow भाषतोस्~\arrow \textcolor{red}{ससजुषो रुः} (पा॰सू॰~८.२.६६)~\arrow भाषतोरुँ~\arrow \textcolor{red}{खरवसानयोर्विसर्जनीयः} (पा॰सू॰~८.३.१५)~\arrow भाषतोः।} ज्ञात्वाऽपि श्रीराम\-तत्त्वं कथनोपकथन\-व्याजेन लोके श्रीराम\-तत्त्वाविश्चिकीर्षया भाषणमाचरतोरिवेति ग्रन्थ\-तात्पर्यं प्रतिभाति। इदं ह्यध्यात्म\-रामायणम्। अत्र प्रत्यक्षरं निगूढ\-दर्शन\-पीयूष\-निर्भरम्। अतो व्याकरण\-प्रयोगा अपि विशेषमेव निगूढ\-रहस्यात्मकं तत्त्वं वितन्वन्तो लक्ष्यन्ते। यतो हि लक्ष्मणः साक्षान्नारायणस्य भगवतः श्रीरामस्यांशः। दार्शनिक\-दृष्ट्या च चतस्रोऽवस्था जाग्रत्स्वप्नसुषुप्तितुरीयाः। चतसृणामपि चत्वारो विभवो विराड्ढिरण्यगर्भ\-सर्वज्ञ\-ब्रह्माख्याः। तत्राशेष\-विशेषातीतं
ज्ञान\-गीर्गोऽतीतं 
निर्गुणं ब्रह्म तुरीयं श्रीरामः। तदवस्था तुरीयावस्था श्रीसीता। तुरीयावस्था\-ब्रह्मणोरिव सीता\-रामयोरप्यभेदः। सुषुप्त्यवस्था माण्डवी। तच्चैतन्याधिष्ठान\-देवता सर्वज्ञ ईश्वरो भरतः। स्वप्नावस्था श्रुतकीर्तिः। तच्चैतन्याधिष्ठानं हिरण्यगर्भः शत्रुघ्नः। जागृतावस्थोर्मिला। तद्विभुर्विराट् श्रीलक्ष्मणः। अतस्तुलसीदासो गायति~–\end{sloppypar}
\centering\textcolor{red}{सुन्दरी सुन्दर बरनि सह सब एक मंडप राजहीं।\nopagebreak\\
जनु जीव उर चारिउ अवस्था बिभुन सहित बिराजहीं॥}\footnote{एतद्रूपान्तरम्–\textcolor{red}{सुरम्याः सुरम्यैर्वरैः सर्ववध्वो व्यराजन्नभिन्ने शुभे मण्डपे च। ध्रुवं जीवचित्तेऽब्धिसङ्ख्या अवस्थाः स्वकीयैरधीशैर्युताः संव्यराजन्॥} (मा॰भा॰~१.३२५.१४)।}\nopagebreak\\
\raggedleft{–~रा॰च॰मा॰~१.३२५.१४}\\
\begin{sloppypar}\hyphenrules{nohyphenation}\justifying\noindent एवमेवेमे चत्वारो राम\-भरत\-लक्ष्मण\-शत्रुघ्नाः क्रमशो मोक्ष\-काम\-धर्मार्थ\-रूपाः। \textcolor{red}{धर्मादिष्वनियमः} (वा॰~२.२.३४) इति वार्त्तिकेनात्राभि\-प्रायानुसारं क्रम\-व्यत्यासः। एवम् \textcolor{red}{ओम्} (ॐ) इत्यत्र हि चत्वारि पदानि~– अ उ म् अर्धमात्रा च। तत्रार्ध\-मात्रात्मको रामो मकारो भरतो वासुदेव उकारः शत्रुघ्नो विधि\-रूपोऽकारो लक्ष्मणः शिव\-रूपश्चेति कथ्यते\footnote{\textcolor{red}{अकाराक्षर\-सम्भूतः सौमित्रिर्विश्व\-भावनः। उकाराक्षर\-सम्भूतः शत्रुघ्नस्तैजसात्मकः॥ प्राज्ञात्मकस्तु भरतो मकाराक्षर\-सम्भवः। अर्ध\-मात्रात्मको रामो ब्रह्मानन्दैक\-विग्रहः॥} (रा॰उ॰ता॰उ॰~३.१.२)}~–\end{sloppypar}
\centering\textcolor{red}{बेद तत्त्व नृप तव सुत चारी॥}\footnote{एतद्रूपान्तरम्–\textcolor{red}{चत्वारोऽपि सुता भूप वेदतत्त्वानि सन्ति ते} (मा॰भा॰~१.१९८.१)। दशरथं प्रति वसिष्ठस्य वचनमिदम्।}\nopagebreak\\
\raggedleft{–~रा॰च॰मा॰~१.१९८.१}\\
\begin{sloppypar}\hyphenrules{nohyphenation}\justifying\noindent इदं सर्वं कण्ठ\-रवेण कथितं विस्तर\-भयान्न निरूप्यते। एवं लक्ष्मणः साक्षाद्भगवान्निषादश्च नित्यो भगवत्परिकरः किं तयोः किमप्यज्ञातम्।\footnote{\textcolor{red}{तयोः} इत्यत्र \textcolor{red}{क्तस्य च वर्तमाने} (पा॰सू॰~२.३.६७) इत्यनेन षष्ठी। \textcolor{red}{मति\-बुद्धि\-पूजार्थेभ्यश्च} (पा॰सू॰~३.२.१८८) इत्यनेन \textcolor{red}{ज्ञानम्} इत्यत्र बुद्ध्यर्थे वर्तमाने क्तः। \textcolor{red}{बुद्धिर्ज्ञानम्} (का॰वृ॰~३.२.१८८) इति काशिका।} केवलं लोक\-लीलार्थं प्रश्नं प्रश्नोत्तरं कुर्वन्ताविव लक्ष्येते। अतो भावतो विदित\-राम\-तत्त्वतयाऽभाष\-माणयोरेव भाषणमाचक्षाणयोरिवानयोर्बाह्य\-परिवेषः।\end{sloppypar}
\section[निरहङ्कारिणः]{निरहङ्कारिणः}
\centering\textcolor{blue}{निरहङ्कारिणः शान्ता ये रागद्वेषवर्जिताः।\nopagebreak\\
समलोष्टाश्मकनकास्तेषां ते हृदयं गृहम्॥}\nopagebreak\\
\raggedleft{–~अ॰रा॰~२.६.५७}\\
\begin{sloppypar}\hyphenrules{nohyphenation}\justifying\noindent\hspace{10mm} अत्र प्राचेतसाश्रमं गत्वा सुनिवास\-स्थानं श्रीरामचन्द्रेण पृष्टो हृष्टो वाल्मीकिः प्रणिगदति \textcolor{red}{निरहङ्कारिणः} इति। यद्यप्यत्र द्वेधा समासः कर्तुं पार्यते तत्पुरुषो बहुव्रीहिश्च तत्पुरुषो यथा \textcolor{red}{अहङ्कारान्निष्क्रान्ता निरहङ्काराः} इति \textcolor{red}{निरादयः क्रान्ताद्यर्थे पञ्चम्या} (वा॰~२.२.१८) इति वचन\-सहायेन \textcolor{red}{कु\-गति\-प्रादयः} (पा॰सू॰~२.२.१८) इति सूत्रेण प्रथमः समासो द्वितीयश्च \textcolor{red}{प्रादिभ्यो धातुजस्य वाच्यो वा चोत्तर\-पद\-लोपश्च} (वा॰~२.२.२२) इत्यनेन बहुव्रीहिरपि \textcolor{red}{निर्गतोऽहङ्कारो येषामिति निरहङ्काराः} इत्थं द्वयोः समासयोः कृतयोरपि \textcolor{red}{निरहङ्कारिणः} इति विधाय तत्पुरुषं पुनर्मत्वर्थीय \textcolor{red}{इनिः} इति। \textcolor{red}{न कर्मधारयान्मत्वर्थीयो बहुव्रीहिश्चेत्तदर्थ\-प्रतिपत्ति\-करः}\footnote{मूलं मृग्यम्।} इति पाणिनि\-सम्मत\-नियममुल्लङ्घ्य किमेतेन
द्रविड\-प्राणायामेन। द्रविड\-प्राणायाम\-प्रकार\-निदर्शनं हि \textcolor{red}{निर्गतोऽहङ्कारो निरहङ्कारः} इत्यत्र \textcolor{red}{कु\-गति\-प्रादयः} (पा॰सू॰~२.२.१८) इत्यनेन तत्पुरुष\-समासः पश्चात् \textcolor{red}{निरहङ्कार एषां निरहङ्कारिणः} इति चेत्।\footnote{\textcolor{red}{अत इनिठनौ} (पा॰सू॰~५.२.११५) इत्यनेनेनिः।} उच्यते। यदि बहुव्रीहावभीष्टार्थ\-लाभः स्यात्तदेदमनुधावनं न स्यात्। तदैव नियमोल्लङ्घन\-रूपाऽपाणिनीयता स्यात्। बहुव्रीहौ नित्य\-निरहङ्कार\-रूपेऽभीष्टेऽर्थे न प्राप्ते नित्य\-योग\-विवक्षायामिनिरत एव न दोषः।\footnote{\textcolor{red}{नित्यं निरहङ्कार एषां निरहङ्कारिणः} इति विग्रहप्रकारः।} अतो गीतायामपि \textcolor{red}{निरहङ्कारः} इति प्रयुक्तं यथा~–\end{sloppypar}
\centering\textcolor{red}{अद्वेष्टा सर्वभूतानां मैत्रः करुण एव च।\nopagebreak\\
निर्ममो निरहङ्कारः समदुःखसुखः क्षमी॥}\nopagebreak\\
\raggedleft{–~भ॰गी॰~१२.१३}\\
\begin{sloppypar}\hyphenrules{nohyphenation}\justifying\noindent अर्थान्निरहङ्कारस्तु मम प्रियः किन्तु निरहङ्कारिणो हृदये वसामीत्येवान्तरं बहुव्रीहि\-मत्वर्थीययोः। एवं \textcolor{red}{निरहङ्कारिणः} इत्यत्र नित्यमहङ्काराभाव इति गूढमर्थं ध्वनयितुं तत्पुरुष\-बहुव्रीहि\-वर्त्म विहाय विकटः पन्था आश्रितः। यद्वा अन्यमपि दर्शयामि। \textcolor{red}{निरहं कर्तुं शीलं येषां ते निरहङ्कारिणः} इति \textcolor{red}{सुप्यजातौ णिनिस्ताच्छील्ये} (पा॰सू॰~३.२.७८) इत्यनेन सूत्रेण णिनि\-विधाने \textcolor{red}{निरहङ्कारिणः}।\end{sloppypar}
\section[प्रकाशन्तः]{प्रकाशन्तः}
\centering\textcolor{blue}{साक्षान्मया प्रकाशन्तो ज्वलनार्कसमप्रभाः।\nopagebreak\\
तानन्वधावं लोभेन तेषां सर्वपरिच्छदान्॥}\nopagebreak\\
\raggedleft{–~अ॰रा॰~२.६.६८}\\
\begin{sloppypar}\hyphenrules{nohyphenation}\justifying\noindent\hspace{10mm} अत्र वाल्मीकिर्भगवन्तं श्रीरामं प्रति पूर्व\-वृत्तान्त\-वर्णन\-प्रसङ्गे सप्तर्षि\-वर्णनं प्रस्तौति। अत्र \textcolor{red}{प्रकाशन्तः} इति प्रयोगः। प्रपूर्वको \textcolor{red}{काश्‌}\-धातुर्दीप्त्यर्थः (\textcolor{red}{काशृँ दीप्तौ} धा॰पा॰~११६२) अकर्मक आत्मनेपदी च। अत्र शानचा भवितव्यं शत्रन्त\-प्रयोगो ह्यपाणिनीय इति चेत्।
\textcolor{red}{प्रकाशन्तः} इत्थमाचार\-क्विबन्ताद्रूप\-सिद्धिः।\footnote{प्रकाशं कुर्वन्ति प्रकाशन्ति। प्रकाश~\arrow \textcolor{red}{सर्वप्राति\-पदिकेभ्य आचारे क्विब्वा वक्तव्यः} (वा॰~३.१.११)~\arrow प्रकाश~क्विँप्~\arrow प्रकाश~व्~\arrow \textcolor{red}{वेरपृक्तस्य} (पा॰सू॰~६.१.६७)~\arrow प्रकाश~\arrow \textcolor{red}{सनाद्यन्ता धातवः} (पा॰सू॰~३.१.३२)~\arrow धातुसञ्ज्ञा~\arrow \textcolor{red}{शेषात्कर्तरि परस्मैपदम्} (पा॰सू॰~१.३.७८)~\arrow \textcolor{red}{वर्तमाने लट्} (पा॰सू॰~३.२.१२३)~\arrow प्रकाश~लट्~\arrow प्रकाश~झि~\arrow \textcolor{red}{कर्तरि शप्‌} (पा॰सू॰~३.१.६८)~\arrow प्रकाश~शप्~झि~\arrow प्रकाश~अ~झि~\arrow \textcolor{red}{अतो गुणे} (पा॰सू॰~६.१.९७)~\arrow प्रकाश~झि~\arrow \textcolor{red}{झोऽन्तः} (पा॰सू॰~७.१.३)~\arrow प्रकाश~अन्ति~\arrow \textcolor{red}{अतो गुणे} (पा॰सू॰~६.१.९७)~\arrow प्रकाशन्ति। प्रकाशन्तीति प्रकाशन्तः। प्रकाश~\arrow धातुसञ्ज्ञा (पूर्ववत्)~\arrow \textcolor{red}{शेषात्कर्तरि परस्मैपदम्} (पा॰सू॰~१.३.७८)~\arrow \textcolor{red}{वर्तमाने लट्} (पा॰सू॰~३.२.१२३)~\arrow प्रकाश~लट्~\arrow \textcolor{red}{लटः शतृशानचावप्रथमा\-समानाधिकरणे} (पा॰सू॰~३.२.१२४)~\arrow प्रकाश~शतृँ~\arrow प्रकाश~अत्~\arrow \textcolor{red}{अतो गुणे} (पा॰सू॰~६.१.९७)~\arrow प्रकाशत्~\arrow \textcolor{red}{कृत्तद्धित\-समासाश्च} (पा॰सू॰~१.२.४६)~\arrow प्रातिपादिक\-सञ्ज्ञा~\arrow विभक्ति\-कार्यम्~\arrow प्रकाशत्~जस्~\arrow प्रकाशत्~अस्~\arrow \textcolor{red}{उगिदचां सर्वनामस्थानेऽधातोः} (पा॰सू॰~७.१.७०)~\arrow \textcolor{red}{मिदचोऽन्त्यात्परः} (पा॰सू॰~१.१.४७)~\arrow प्रकाश~नुँम्~त्~अस्~\arrow प्रकाश~न्~त्~अस्~\arrow प्रकाशन्तस्~\arrow \textcolor{red}{ससजुषो रुः} (पा॰सू॰~८.२.६६)~\arrow प्रकाशन्तरुँ~\arrow \textcolor{red}{खरवसानयोर्विसर्जनीयः} (पा॰सू॰~८.३.१५)~\arrow प्रकाशन्तः।} यद्वा \textcolor{red}{अनुदात्तेत्त्व\-लक्षणमात्मने\-पदमनित्यम्} (प॰शे॰~९३.४) मत्वा परस्मैपदत्वाच्छतृ\-प्रत्ययः।\footnote{प्र~\textcolor{red}{काशृँ दीप्तौ} (धा॰पा॰~११६२)~\arrow प्र~काश्~\arrow \textcolor{red}{अनुदात्तेत्त्व\-लक्षणमात्मने\-पदमनित्यम्} (प॰शे॰~९३.४)~\arrow \textcolor{red}{शेषात्कर्तरि परस्मैपदम्} (पा॰सू॰~१.३.७८)~\arrow \textcolor{red}{वर्तमाने लट्} (पा॰सू॰~३.२.१२३)~\arrow प्र~काश्~लट्~\arrow \textcolor{red}{लटः शतृशानचावप्रथमा\-समानाधिकरणे} (पा॰सू॰~३.२.१२४)~\arrow प्र~काश्~शतृँ~\arrow प्र~काश्~अत्~\arrow प्रकाशत्~\arrow \textcolor{red}{कृत्तद्धित\-समासाश्च} (पा॰सू॰~१.२.४६)~\arrow प्रातिपादिक\-सञ्ज्ञा। शेषा प्रक्रिया पूर्ववत्।}\end{sloppypar}
\section[निश्चयः]{निश्चयः}
\centering\textcolor{blue}{वयं स्थास्यामहे तावदागमिष्यसि निश्चयः।\nopagebreak\\
तथेत्युक्त्वा गृहं गत्वा मुनिभिर्यदुदीरितम्॥}\nopagebreak\\
\raggedleft{–~अ॰रा॰~२.६.७३}\\
\begin{sloppypar}\hyphenrules{nohyphenation}\justifying\noindent\hspace{10mm} अत्र सप्तर्षि\-कथितमेवानुवदति वाल्मीकिर्यत् \textcolor{red}{तावद्वयमत्र स्थास्यामो यावत्त्वं कृत\-निश्चय आगमिष्यसि}। \textcolor{red}{आगमिष्यसि निश्चयः} इति वाक्य\-खण्डे \textcolor{red}{निश्चयः} इति हि युष्मत्पद\-वाच्य\-वाल्मीकि\-रूप\-कर्तृ\-विशेषणम्। \textcolor{red}{त्वं निश्चयः सन् यावदागमिष्यसि तावद्वयं स्थास्यामः} इति भावः। \textcolor{red}{निश्चय}\-शब्दः प्रायो भावेऽबन्तः।\footnote{\textcolor{red}{निस्‌}\-पूर्वकात् \textcolor{red}{चिञ् चयने} (धा॰पा॰~१२५१) इति धातोः \textcolor{red}{एरच्} (पा॰सू॰~३.३.५६) इत्यनेन प्राप्तं \textcolor{red}{अच्‌}\-प्रत्ययं बाधित्वा \textcolor{red}{ग्रहवृदृ\-निश्चिगमश्च} (पा॰सू॰~३.३.५८) इत्यनेन भावेऽकर्तरि च कारके सञ्ज्ञायाम् \textcolor{red}{अप्‌}\-प्रत्यये निश्चय\-शब्दो निष्पन्नः।
} भाव\-प्रत्ययान्त\-शब्दः कथं कर्तृ\-वाचकस्य विशेषणमसमानाधि\-करणात्। असामानाधि\-करण्ये कथं स्याद्विभक्तिः कथमन्वयश्च। सामानाधि\-करण्याभावे \textcolor{red}{आगमिष्यसि} इति प्रयोगोऽपि नैव सम्भविष्यति। तथा च सूत्रम् \textcolor{red}{युष्मद्युपपदे समानाधिकरणे स्थानिन्यपि मध्यमः} (पा॰सू॰~१.४.१०५)। अस्यार्थः \textcolor{red}{तिङ्वाच्य\-कारक\-वाचिनि युष्मद्युपपदे समानाधिकरणे प्रयुज्यमानेऽप्रयुज्यमाने वा मध्यमः}। न च क्वचिदसमानाधि\-करणमपि विशेषणम्। विशेषण\-विशेष्ययोः सामानाधिकरण्ये नैव राजाज्ञा। \textcolor{red}{भूतले घटः}, \textcolor{red}{राज्ञः पुरुषः} इत्यादौ यथा सप्तम्यन्त\-षष्ठ्यन्तयोरपि विशेषणत्व\-दर्शनात्। सत्यम्। किन्त्वत्र समानाधिकरणत्वं नाम समान\-विभक्तिकत्वम्। समान\-विभक्तिकत्वं प्राय एकार्थ\-वाचकयोर्विशेषण\-विशेष्ययोरेव भवति।
यथा \textcolor{red}{नीलमुत्पलम्} अत्रोत्पल\-शब्दोत्तर\-सुब्विभक्तिर्नील\-पदोत्तर\-सुब्विभक्तिश्चोभे अप्येकमेवोत्पल\-रूपमर्थं नीलत्व\-विशिष्टं कथयतः। यतो हि \textcolor{red}{नील}\-शब्दस्य \textcolor{red}{उत्पल}\-शब्देनाभेद\-सम्बन्धेनान्वयः। तथाऽपि व्युत्पत्तिवादे श्रीगदाधरभट्टा व्युत्पादयन्ति \textcolor{red}{यत्र विशेष्य\-वाचक\-पदोत्तर\-विभक्ति\-तात्पर्य\-विषय\-सङ्ख्या\-विरुद्ध\-सङ्ख्याया अविवक्षितत्वं तत्रैव विशेष्य\-विशेषण\-वाचक\-पदयोः समान\-वचनकत्व\-नियमः} (व्यु॰वा॰ का॰प्र॰)। यथा \textcolor{red}{सुन्दरो रामः} इत्यत्र विशेष्य\-वाचक\-पदं \textcolor{red}{रामः} इति तदुत्तर\-विभक्तिः \textcolor{red}{सु} इति तत्तात्पर्य\-विषय\-सङ्ख्यैकत्व\-रूपा तद्विरुद्ध\-सङ्ख्या द्वित्व\-त्रित्वादयस्तासां विशेष्य\-वाचक\-सुन्दर\-पदोत्तर\-\textcolor{red}{सु}\-विभक्त्या न विवक्षितत्वमतोऽत्र समान\-वचनकत्वम्। इति मीमांसा\-मात्रे \textcolor{red}{आगमिष्यसि} इत्यस्य समानाधिकरणः \textcolor{red}{त्वम्} इति। तस्य सामानाधिकरणं \textcolor{red}{निश्चयः} इति। अन्यथा कथं प्रथमा\-विभक्तिः क्रियेत। स च तदैव \textcolor{red}{त्वम्} इत्यस्य सामानाधिकरणो भविष्यति यदा तदर्थमेव भाषेत। भावे विहिताजन्त\-निश्चय\-शब्दः कथं कर्त्रर्थमनुवदिष्यति। अस्यैकार्थ\-वाचकत्वे कथमत्र समान\-वचनकत्वमिति महत्पङ्कमिति चेत्। अत्र \textcolor{red}{निश्चिनोतीति निश्चयः} इति विग्रहे पचादित्वादच्।\footnote{\textcolor{red}{नन्दि\-ग्रहि\-पचादिभ्यो ल्युणिन्यचः} (पा॰सू॰~३.१.१३४) इत्यनेन।} यद्वा \textcolor{red}{निश्चयः} इति भावेऽबन्त एव तथाऽपि \textcolor{red}{निश्चयोऽस्त्यस्य} इति विग्रहेऽर्श\-आदित्वान्मत्वर्थीयोऽच्।\footnote{\textcolor{red}{अर्शआदिभ्योऽच्} (पा॰सू॰~५.२.१२७) इत्यनेन।} \textcolor{red}{निश्चयः} इत्यस्य \textcolor{red}{निश्चयवान्} इत्यर्थः। अथवा पृथक्पदं वाक्यभेदश्च। यदाऽऽगमिष्यसि तदा निश्चयो भविष्यतीति व्यञ्जनया ध्वन्यते। काव्यस्यात्मा हि ध्वनिस्तथा च ध्वन्यालोकेऽभिनन्दयन्त्यानन्द\-वर्धनाचार्याः~–\end{sloppypar}
\centering\textcolor{red}{काव्यस्यात्मा ध्वनिरिति बुधैर्यः समाम्नातपूर्व-\nopagebreak\\
स्तस्याभावं जगदुरपरे भाक्तमाहुस्तमन्ये।\nopagebreak\\
केचिद्वाचां स्थितमविषये तत्त्वमूचुस्तदीयं\nopagebreak\\
तेन ब्रूमः सहृदयमनःप्रीतये तत्स्वरूपम्॥}\nopagebreak\\
\raggedleft{–~ध्व॰~१.१}\\
\begin{sloppypar}\hyphenrules{nohyphenation}\justifying\noindent न च साहित्यिक\-मतेन वैयाकरणानां किमिति चेत्। आनन्दवर्धनाचार्यैरपि \textcolor{red}{बुध}\-शब्दे वैयाकरणानां ध्वनि\-स्वीकृतत्व\-प्रतिपादनात्। यथा बुधैर्वैयाकरणैः स एव ध्वनिः स्फोट\-रूपेण पूर्ण\-समाम्नातोऽभ्यस्तः। \textcolor{red}{येनोच्चारितेन सास्ना\-लाङ्गूल\-ककुद\-खुर\-विषाणिनां सम्प्रत्ययो भवति स शब्दः} (भा॰प॰)। न च काव्यात्मभूत\-ध्वनिरत्र किमायातम्। अध्यात्म\-रामायणस्य सर्वत्र काव्य\-रूपेण चर्चितत्वात् यथा~–\end{sloppypar}
\centering\textcolor{blue}{रामायणं जनमनोहरमादिकाव्यं\nopagebreak\\
ब्रह्मादिभिः सुरवरैरपि संस्तुतं च।\nopagebreak\\
श्रद्धान्वितः पठति यः शृणुयात्तु नित्यं\nopagebreak\\
विष्णोः प्रयाति सदनं स विशुद्धदेहः॥}\nopagebreak\\
\raggedleft{–~अ॰रा॰~७.९.७३}\\
\begin{sloppypar}\hyphenrules{nohyphenation}\justifying\noindent इत्थं \textcolor{red}{यदा आगमिष्यसि तदा निश्चयो भविष्यति} इति न दोषः।\end{sloppypar}
\section[स्थाप्य]{स्थाप्य}
\label{sec:sthapya}
\centering\textcolor{blue}{बहिरेव रथं स्थाप्य राजानं द्रष्टुमाययौ।\nopagebreak\\
जय शब्देन राजानं स्तुत्वा तं प्रणनाम ह॥}\nopagebreak\\
\raggedleft{–~अ॰रा॰~२.७.२}\\
\begin{sloppypar}\hyphenrules{nohyphenation}\justifying\noindent\hspace{10mm} सुमन्त्रः ससौमित्रि\-सीतं श्रीरामभद्रं रथेन गङ्गातटं यावत्प्रापय्य\footnote{\textcolor{red}{प्रापय्य} इत्यत्र \textcolor{red}{विभाषाऽऽपः} (पा॰सू॰~६.४.५७) इत्यनेन णेरयादेशे वैकल्पिके द्वियं रूपं सिद्धम्। पक्षे \textcolor{red}{प्राप्य} इति।} तदाज्ञयाऽयोध्यां परावर्तमानो रथं बहिः स्थापयित्वा म्रियमाणं राजानं दशरथं ददर्श। अत्र \textcolor{red}{बहिरेव रथं स्थाप्य} इत्यपाणिनीयमिव। यतो हि \textcolor{red}{ष्ठा गति\-निवृत्तौ} (धा॰पा॰~९२८) इति धातोर्ण्यन्ते \textcolor{red}{अर्ति\-ह्री\-व्ली\-री\-क्नूयी\-क्ष्माय्यातां पुङ्णौ} (पा॰सू॰~७.३.३६) इत्यनेन पुगागमे \textcolor{red}{क्त्वा}\-प्रत्यय इटि गुणेऽयादेशे \textcolor{red}{स्थापयित्वा} इति हि वरम्।\footnote{ष्ठा~\arrow \textcolor{red}{धात्वादेः षः सः} (पा॰सू॰~६.१.६४)~\arrow \textcolor{red}{निमित्तापाये नैमित्तिकस्याप्यपायः}~\arrow स्था~\arrow \textcolor{red}{हेतुमति च} (पा॰सू॰~३.१.२६)~\arrow स्था~णिच्~\arrow स्था~इ~\arrow \textcolor{red}{अर्ति\-ह्री\-व्ली\-री\-क्नूयी\-क्ष्माय्यातां पुङ्णौ} (पा॰सू॰~७.३.३६)~\arrow \textcolor{red}{आद्यन्तौ टकितौ} (पा॰सू॰~१.१.४६)~\arrow स्था~पुँक्~इ~\arrow स्था~प्~इ~\arrow स्थापि~\arrow \textcolor{red}{सनाद्यन्ता धातवः} (पा॰सू॰~३.१.३२)~\arrow धातुसञ्ज्ञा~\arrow \textcolor{red}{समानकर्तृकयोः पूर्वकाले} (पा॰सू॰~३.४.२१)~\arrow स्थापि~क्त्वा~\arrow स्थापि~त्वा~\arrow \textcolor{red}{आर्धधातुकस्येड्वलादेः} (पा॰सू॰~७.२.३५)~\arrow \textcolor{red}{आद्यन्तौ टकितौ} (पा॰सू॰~१.१.४६)~\arrow स्थापि~इट्~त्वा~\arrow स्थापि~इ~त्वा~\arrow \textcolor{red}{सार्वधातुकार्ध\-धातुकयोः} (पा॰सू॰~७.३.८४)~\arrow स्थापे~इ~त्वा~\arrow \textcolor{red}{एचोऽयवायावः} (पा॰सू॰~६.१.७८)~\arrow स्थापय्~इ~त्वा~\arrow स्थापयित्वा।} \textcolor{red}{रथं स्थाप्य} इति कथम्। \textcolor{red}{ल्यप्}\-प्रत्ययस्तु समासमन्तरेण भवितुं शक्य एव नहि। तथा च सूत्रम् \textcolor{red}{समासेऽनञ्पूर्वे क्त्वो ल्यप्} (पा॰सू॰~७.१.३७)। इदं ह्यनञ्पूर्वके समासे सत्येव क्त्वा\-प्रत्ययस्य स्थाने ल्यपं विदधाति। अत्र समासोऽपि नास्ति। अतः \textcolor{red}{स्थाप्य} इति विमृश्यते। अत्र \textcolor{red}{संस्थाप्य} इति प्रयोग आसीत्। तदा सम्पूर्वक\-णिजन्त\-स्था\-धातोः क्त्वा\-प्रत्यये ल्यबादेशः। एवं \textcolor{red}{विनाऽपि प्रत्ययं पूर्वोत्तर\-पद\-लोपो वक्तव्यः} (वा॰~५.३.८३) इत्यनेन \textcolor{red}{सम्} उपसर्गस्य लोपः। न च समासे कृते कुतोऽत्र पदत्वम्। अन्तर्वर्तिनीं विभक्तिमाश्रित्यैवानेन वार्त्तिकेन लोप\-करणात्।\footnote{\textcolor{red}{प्राग्रीश्वरान्निपाताः} (पा॰सू॰~१.४.५६) इत्यस्याधिकारे पठितत्वात् \textcolor{red}{प्रादयः} (पा॰सू॰~१.४.५८) इति सूत्रेण प्रादीनां \textcolor{red}{निपात}\-सञ्ज्ञायां जातायाम् \textcolor{red}{उपसर्गाः क्रियायोगे} (पा॰सू॰~१.४.५९) इत्यनेन क्रियायोगे तेषामेव \textcolor{red}{उपसर्ग}\-सञ्ज्ञायामुप\-सर्गाणां निपातत्वे सिद्धे \textcolor{red}{स्वरादि\-निपातमव्ययम्} (पा॰सू॰~१.१.३७) इत्यनेन निपातानाम् \textcolor{red}{अव्यय}\-सञ्ज्ञायां विहितायामुप\-सर्गाणामव्यय\-सञ्ज्ञायां सिद्धायाम् \textcolor{red}{अव्ययादाप्सुपः} (पा॰सू॰~२.४.८२) इत्यनेनोपसर्गात्सुपां लुक्यपि \textcolor{red}{प्रत्ययलोपे प्रत्ययलक्षणम्} (पा॰सू॰~१.१.६२) इति सूत्रबलेन \textcolor{red}{सुप्तिङन्तं पदम्} (पा॰सू॰~१.४.१४) इत्युपसर्गाणां प्रत्यय\-लक्षणा पदसञ्ज्ञाऽक्षतैवेति भावः।} \textcolor{red}{सत्यभामा भामा} (भा॰प॰, भा॰पा॰सू॰~१.१.४५) इत्यादावपि समस्त\-पदस्यैव लोप\-दर्शनात्। इत्थं \textcolor{red}{सम्} इत्यस्य लोपे सति प्रथमं \textcolor{red}{स्थाप्य} इति प्रामाणिक एव प्रयोगः। न च \textcolor{red}{निमित्तापाये नैमित्तिकस्याप्यपायः} इति परिभाषया निमित्त\-भूते समुपसर्गे लुप्ते नैमित्तिकस्य \textcolor{red}{ल्यप्} इत्यस्याप्यपायस्ततः \textcolor{red}{स्थाप्य} इत्यनिष्टं रूपमिति वाच्यम्। अस्याः परिभाषाया भाष्येऽदर्शनात्।\footnote{तस्मादनित्या परिभाषेयमिति भावः।} \textcolor{red}{जात\-संस्कारो न निवर्तते} इति वचन\-बलेन लुप्तेऽपि समुपसर्गे तन्निमित्तक\-ल्यपो न निवर्तनम्। यद्वा \textcolor{red}{स्थापयितुं शक्यं योग्यं वेति स्थाप्यम्} इति विग्रहे णिजन्त\-पुगन्त\-स्था\-धातोः \textcolor{red}{ऋहलोर्ण्यत्} (पा॰सू॰~३.१.१२४) इत्यनेन \textcolor{red}{ण्यत्} प्रत्ययः। अर्थादशक्यत्वादेकं स्थापयितुं योग्यं राजानं ददर्श। साम्प्रतं हि राज्ञो गतिर्निवृत्ता बुद्धिरपि शान्ता प्रत्यवसानमप्यवरुद्धं शब्दोऽपि शिथिलोऽतः \textcolor{red}{स्थाप्य}\-शब्दः \textcolor{red}{राजन्} शब्दस्य विशेषणम्। तथा \textcolor{red}{स्थाप्यश्चासौ राजा चेति स्थाप्यराजा तं स्थाप्य\-राजानम्}। \textcolor{red}{देवराजानम्} (अ॰रा॰~१.५.२६) इतिवट्टजभावः।\footnote{\pageref{sec:devarajanam}तमे पृष्ठे \ref{sec:devarajanam} \nameref{sec:devarajanam} इति प्रयोगस्य विमर्शं पश्यन्तु।} यद्वा \textcolor{red}{स्थीयत इति स्था}।\footnote{\textcolor{red}{ष्ठा गतिनिवृत्तौ} (धा॰पा॰~९२८) इति धातोः \textcolor{red}{अन्येष्वपि दृश्यते} (पा॰सू॰~३.२.१०१) इत्यनेन डे \textcolor{red}{डित्यभस्याप्यनु\-बन्धकरण\-सामर्थ्यात्} (वा॰~६.४.१४३) इत्यनेन टिलोपे \textcolor{red}{अजाद्यतष्टाप्‌} (पा॰सू॰~४.१.४) इत्यनेन टापि। दृशिग्रहणादपि\-ग्रहणाच्चानुप\-पदेऽपि। \textcolor{red}{अपिशब्दः सर्वोपाधि\-व्यभिचारार्थः} (का॰वृ॰~३.२.१०१, वै॰सि॰कौ॰~३०११)। यद्वा \textcolor{red}{अन्येभ्योऽपि दृश्यते} (पा॰सू॰~३.२.१७८) इत्यनेन क्विपि। दृशिग्रहणाद्भावेऽपि। यद्वा भिदादि\-गणमाकृति\-गणं मत्वा \textcolor{red}{षिद्भिदादिभ्योऽङ्} (पा॰सू॰~३.३.१०४) इत्यनेनाङि \textcolor{red}{अजाद्यतष्टाप्‌} (पा॰सू॰~४.१.४) इत्यनेन टापि। कथं तर्हि \textcolor{red}{स्थितिः} इति क्तिन्। \textcolor{red}{अनर्थकास्तु प्रतिवर्णमर्थानुपलब्धेः} (भा॰शि॰) इति भाष्य\-प्रयोगात्षिद्भिदादिभ्यो बाहुलकात्क्तिन्नपि। अन्यथा \textcolor{red}{डुलभँष् प्राप्तौ} (धा॰पा॰~९७५) इत्यस्य षित्त्वात्क्तिन्बाधो दुर्वारः।
}
तथा च \textcolor{red}{स्थयाऽऽप्तुं योग्यमिति स्थाप्यम्} इत्थं भाव\-साधित\-स्था\-शब्देनाऽप्य\-शब्दस्य समासः। \textcolor{red}{आप्य}\-शब्दोऽपि ण्यत्प्रत्ययान्त इति। तथा गति\-निवृत्त्याऽव्याप्तमिति तात्पर्यम्। अथवा म्रियमाणत्वाच्चितायां स्थापयितुं योग्यं राजानं ददर्श। न च \textcolor{red}{स्थाप्य}\-शब्दस्य \textcolor{red}{राजन्} शब्देन सह तत्पुरुष\-समासे \textcolor{red}{राजाऽहस्सखिभ्यष्टच्} (पा॰सू॰~५.४.९१) इत्यनेन \textcolor{red}{टच्} प्रत्यये टिलोपे विभक्ति\-कार्येऽमि \textcolor{red}{स्थाप्य\-राजम्} स्यादिति वाच्यम् \textcolor{red}{महाराजम्} इतिवत्। समासान्त\-प्रत्यय\-प्रकरणं ह्यनित्यम्। प्रमाणं चात्र \textcolor{red}{यचि भम्} (पा॰सू॰~१.४.१८) इति सूत्रम्। अत्र \textcolor{red}{यश्चाच्च यच्} इति समाहार\-द्वन्द्वः। इह \textcolor{red}{द्वन्द्वाच्चु\-दषहान्तात्समाहारे} (पा॰सू॰~५.४.१०६) इत्यनेन चान्तत्वाट्टच्प्रत्ययः प्रयोक्तव्य आसीत्। तस्मिन् प्रयुक्ते \textcolor{red}{यचे भम्} इति स्यात्। यतो न प्रयुक्तोऽतः समासान्त\-प्रत्ययस्यानित्यता ज्ञायते।\footnote{अन्यत्राप्येतज्ज्ञापितं भगवता पाणिनिना। प्रथमसूत्र एव \textcolor{red}{वृद्धिरादैच्} (पा॰सू॰~१.१.१) इति प्रयुक्तम्। समासान्त\-प्रत्यय\-नित्यत्वे तु \textcolor{red}{द्वन्द्वाच्चु\-दषहान्तात्समाहारे} (पा॰सू॰~५.४.१०६) इत्यनेन चान्तत्वाट्टजन्तरूपेण \textcolor{red}{वृद्धिरादैचम्} इत्यनेन भवितव्यमासीत्। हलन्तप्रयोगोऽत्रापि समासान्त\-प्रत्ययानित्यत्व\-ज्ञापनार्थम्।} अतः \textcolor{red}{स्थाप्य\-राजानम्} अस्मिन्प्रयोगे नापाणिनीयता।\end{sloppypar}
\section[अवेक्षती]{अवेक्षती}
\label{sec:aveksati}
\centering\textcolor{blue}{सीता चाश्रुपरीताक्षी मामाह नृपसत्तम।\nopagebreak\\
दुःखगद्गदया वाचा रामं किञ्चिदवेक्षती॥}\nopagebreak\\
\raggedleft{–~अ॰रा॰~२.७.१२}\\
\begin{sloppypar}\hyphenrules{nohyphenation}\justifying\noindent\hspace{10mm} शोकाकुल\-दशरथं प्रति राम\-सन्देशं वर्णयित्वा सुमन्त्रः साम्प्रतं सीता\-मनोदशां प्रतिपादयति यदश्रु\-परीत\-लोचना सीता निरीक्ष्य राम\-भद्रं किमपि समादिशत्। यदिह \textcolor{red}{किञ्चिदवेक्षती} अयं प्रयोगः \textcolor{red}{अवेक्षमाणा} इति प्रयोक्तव्ये \textcolor{red}{अवेक्षती} इति
प्रयुक्तमत्रांश\-द्वयेऽपाणिनीयता\-भ्रान्तिः।
\textcolor{red}{अव}पूर्वको हि \textcolor{red}{ईक्ष्}धातुः (\textcolor{red}{ईक्षँ दर्शने} धा॰पा॰~६१०) आत्मनेपदी। एवं \textcolor{red}{तङानावात्मनेपदम्} (पा॰सू॰~१.४.१००) इत्यनेन विधीयमानः \textcolor{red}{शानच्} प्रत्ययोऽप्यात्मनेपद\-सञ्ज्ञकः। तथा च \textcolor{red}{आने मुक्} (पा॰सू॰~७.२.८२) इत्यनेन मुकि कृते \textcolor{red}{अवेक्षमाणा} इत्येव प्रयोगः प्रचलितः। एवं \textcolor{red}{अवेक्षती} इति विमृश्यते। प्रथमं त्वात्मनेपद\-प्रयुक्त\-धातोः परस्मैपदवद्व्यवहारः। सति परस्मैपदे प्रत्यय\-विचारस्तत्रैव नुमभाव\-निस्तारः। \textcolor{red}{अनुदात्तेत्त्व\-लक्षणमात्मने\-पदमनित्यम्} (प॰शे॰~९३.४) इति वचनेन सति परस्मैपदे \textcolor{red}{अवेक्षति} इति विग्रहे शतृ\-प्रत्ययान्त\-रूपम्। ननु \textcolor{red}{अनुदात्तेत्त्व\-लक्षणमात्मने\-पदमनित्यम्} (प॰शे॰~९३.४) अत्र न किमपि प्रमाणमिति चेन्न। \textcolor{red}{चक्षिँङ् व्यक्तायां वाचि} (धा॰पा॰~१०१७) इति हि धातुः। अयं चानुदात्तेन्ङिच्च। द्वयोरप्यनुबन्धयोरेकमेव फलमात्मनेपदम्। तत्र सूत्रम् \textcolor{red}{अनुदात्तङित आत्मनेपदम्} (पा॰सू॰~१.३.१२) एव। अस्यैव फलस्य कृते द्वयोरनुबन्ध\-करणत्वं किमर्थम्। अनुदात्तेत्त्व\-करणेनैवाऽत्मनेपद\-सिद्धिः। ङित्करणेन किं प्रयोजनम्। तदेव व्यर्थं सज्ज्ञापयति यदनुदातेत्त्व\-लक्षणमात्मनेपदं तत्कार्यं चानित्यम्। अतस्तस्यानित्यत्वात्परस्मैपद\-शतृ\-प्रयोगः। शतृ\-प्रयोगे सति कथं नुमभावः इति चेत्। आगम\-शास्त्रस्यानित्यत्वात्।\footnote{\textcolor{red}{आगम\-शास्त्रमनित्यम्} (प॰शे॰~९३.२)। \textcolor{red}{शप्श्यनोर्नित्यम्} (पा॰सू॰~७.१.८१) इत्यनेन शपि नुमागमस्य नित्यत्वादागम\-शास्त्रस्यानित्यत्वमाश्रितम्।} यद्वाऽत्र न शतृ\-प्रत्ययः। अत्र \textcolor{red}{तृँच्} प्रत्यय औणादिकः।\footnote{नायं \textcolor{red}{बहुलमन्यत्रापि} (प॰उ॰~२.९५) इति तृच्। स नोगित्। \textcolor{red}{कार्याद्विद्यादनूबन्धम्} (भा॰पा॰सू॰~३.३.१) \textcolor{red}{केचिदविहिता अप्यूह्याः} (वै॰सि॰कौ॰~३१६९) इत्यनुसारमूह्योऽ\-यमविहित उगित्प्रत्ययः। \textcolor{red}{तृँच्} प्रत्यये चात्र शबागमोऽप्यूह्यः। \textcolor{red}{नयतेः षुगागमः} (प॰उ॰ श्वे॰वृ॰~२.९६) इतिवत्।}
एवं नुमभावः। तथा च स्त्रीत्व\-विवक्षायां ङीप्प्रत्यये\footnote{\textcolor{red}{उगितश्च} (पा॰सू॰~४.१.६) इत्यनेन ङीप्।} \textcolor{red}{अवेक्षती} इति सिद्धम्।\end{sloppypar}
\section[रुदन्ती]{रुदन्ती}
\centering\textcolor{blue}{ततो दुःखेन महता पुनरेवाहमागतः।\nopagebreak\\
ततो रुदन्ती कौसल्या राजानमिदमब्रवीत्॥}\nopagebreak\\
\raggedleft{–~अ॰रा॰~२.७.१५}\\
\begin{sloppypar}\hyphenrules{nohyphenation}\justifying\noindent\hspace{10mm} भगवतः श्रीरामचन्द्रस्य वनवासे सत्ययोध्यामागते सुमन्त्रे निशम्य सकलं समाचारं विलपन्तं रघुनन्दं निरानन्दं म्रियमाणं राजानं विलोक्य रुदती कौशल्याऽब्रवीत्। अत्र \textcolor{red}{रुदन्ती} इति प्रयुक्तम्। तच्चासङ्गतमिव। यतो हि \textcolor{red}{रुद्‌}\-धातुः (\textcolor{red}{रुदिँर् अश्रुविमोचने} धा॰पा॰~१०६७) अदादि\-गण\-पठितः। तत्र \textcolor{red}{अदिप्रभृतिभ्यः शपः} (पा॰सू॰~२.४.७२) इत्यनेन शपो लुग्भवति। ततश्च शतृ\-प्रत्यये नुमभावः स्वत एव सिद्धः।\footnote{\textcolor{red}{रुदिँर् अश्रुविमोचने} धा॰पा॰~१०६७)~\arrow रुद्~\arrow \textcolor{red}{लक्षणहेत्वोः क्रियायाः} (पा॰सू॰~३.२.१२६)~\arrow रुद्~शतृँ~\arrow \textcolor{red}{सार्वधातुकमपित्} (पा॰सू॰~१.२.४)~\arrow शतुर्ङिद्वत्त्वम्~\arrow रुद्~अत्~\arrow \textcolor{red}{कर्तरि शप्‌} (पा॰सू॰~३.१.६८)~\arrow रुद्~शप्~अत्~\arrow \textcolor{red}{अदिप्रभृतिभ्यः शपः} (पा॰सू॰~२.४.७२)~\arrow रुद्~अत्~\arrow \textcolor{red}{ग्क्ङिति च} (पा॰सू॰~१.१.५)~\arrow लघूपध\-गुणनिषेधः~\arrow रुद्~अत्~\arrow रुदत्~\arrow \textcolor{red}{उगितश्च} (पा॰सू॰~४.१.६)~\arrow रुदत्~ङीप्~\arrow रुदत्~ई~\arrow रुदती~\arrow विभक्ति\-कार्यम्~\arrow रुदती~सुँ~\arrow \textcolor{red}{हल्ङ्याब्भ्यो दीर्घात्सुतिस्यपृक्तं हल्} (पा॰सू॰~६.१.६८)~\arrow सुलोपः~\arrow रुदती। \textcolor{red}{आच्छीनद्योर्नुम्} (पा॰सू॰~७.१.८०) इति सूत्रेणावर्णान्तादङ्गादुत्तरस्य शतुः शीनद्योः परतः पाक्षिकनुमागमो भवति। शपो लुक्यङ्गस्य हलन्तत्वादेतत्सूत्रं न प्रवर्तते।} अत्र नुमपाणिनीय इति कृत्वा विमृश्यते। \textcolor{red}{गण\-कार्यमनित्यम्} (प॰शे॰~९३.३)। अतः शपो न लुक्। तथा च \textcolor{red}{रुदतीति रुदन्ती} इति विग्रहे नुम्। यद्वा \textcolor{red}{रोदनं रुदः}। भावेऽच्प्रत्ययः।\footnote{\textcolor{red}{अजपि सर्वधातुभ्यः} (वा॰~३.१.१३४) इत्यनेन। बाहुलकाद्भावे। यथा \textcolor{red}{विरुदः} इत्यत्र। \textcolor{red}{विरुद}\-शब्दो भावेऽपीत्याप्टे\-कोशः।} \textcolor{red}{रुदमाचरतीति रुदति}। आचारार्थे क्विप्। ततो धातु\-सञ्ज्ञातो लटि तिपि शपि \textcolor{red}{रुदति}।\footnote{रुद~\arrow \textcolor{red}{सर्वप्राति\-पदिकेभ्य आचारे क्विब्वा वक्तव्यः} (वा॰~३.१.११)~\arrow रुद~क्विँप्~\arrow रुद~व्~\arrow \textcolor{red}{वेरपृक्तस्य} (पा॰सू॰~६.१.६७)~\arrow रुद~\arrow \textcolor{red}{सनाद्यन्ता धातवः} (पा॰सू॰~३.१.३२)~\arrow धातुसञ्ज्ञा~\arrow \textcolor{red}{शेषात्कर्तरि परस्मैपदम्} (पा॰सू॰~१.३.७८)~\arrow \textcolor{red}{वर्तमाने लट्} (पा॰सू॰~३.२.१२३)~\arrow रुद~लट्~\arrow रुद~तिप्~\arrow रुद~ति~\arrow \textcolor{red}{कर्तरि शप्‌} (पा॰सू॰~३.१.६८)~\arrow रुद~शप्~ति~\arrow रुद~अ~ति~\arrow \textcolor{red}{अतो गुणे} (पा॰सू॰~६.१.९७)~\arrow रुद~ति~\arrow रुदति।} अतः \textcolor{red}{रुदतीति रुदन्ती} इत्थं साधु।\footnote{रुद~\arrow धातुसञ्ज्ञा (पूर्ववत्)~\arrow \textcolor{red}{शेषात्कर्तरि परस्मैपदम्} (पा॰सू॰~१.३.७८)~\arrow \textcolor{red}{वर्तमाने लट्} (पा॰सू॰~३.२.१२३)~\arrow रुद~लट्~\arrow \textcolor{red}{लटः शतृशानचावप्रथमा\-समानाधिकरणे} (पा॰सू॰~३.२.१२४)~\arrow रुद~शतृँ~\arrow रुद~अत्~\arrow \textcolor{red}{अतो गुणे} (पा॰सू॰~६.१.९७)~\arrow रुदत्~\arrow \textcolor{red}{उगितश्च} (पा॰सू॰~४.१.६)~\arrow रुदत्~ङीप्~\arrow रुदत्~ई~\arrow \textcolor{red}{शप्श्यनोर्नित्यम्} (पा॰सू॰~७.१.८१)~\arrow \textcolor{red}{आद्यन्तौ टकितौ} (पा॰सू॰~१.१.४६)~\arrow रुद~नुँम्~त्~ई~\arrow रुद~न्~त्~ई~\arrow रुदन्ती~\arrow विभक्ति\-कार्यम्~\arrow रुदन्ती~सुँ~\arrow \textcolor{red}{हल्ङ्याब्भ्यो दीर्घात्सुतिस्यपृक्तं हल्} (पा॰सू॰~६.१.६८)~\arrow रुदन्ती। कर्तर्यच्यपि न दोषः। एवं तर्हि रुदतीति रुदः। रुद इवाऽचरतीति रुदति। रुदतीति रुदन्ती। प्रक्रिया पूर्ववत्।}\end{sloppypar}
\section[गृह्य]{गृह्य}
\centering\textcolor{blue}{अतिवृद्धावन्धदृशौ क्षुत्पिपासार्दितौ निशि।\nopagebreak\\
नायाति सलिलं गृह्य पुत्रः किं वात्र कारणम्॥}\nopagebreak\\
\raggedleft{–~अ॰रा॰~२.७.३१}\\
\begin{sloppypar}\hyphenrules{nohyphenation}\justifying\noindent\hspace{10mm} अत्र महाराजो दशरथः श्रमणकुमारात्सजलं कलशं गृहीत्वा वृद्ध\-दम्पती गतः। \textcolor{red}{सलिलं गृह्य पुत्रो नायाति अत्र किं वा कारणम्}~– इति तौ चिन्तितवन्तावास्ताम्। ल्यबन्त\-प्रयोगोऽयम् \textcolor{red}{गृह्य}। स च समस्त\-पदं विनाऽसम्भवः। इह कोऽपि समासो नहीति। मैवम्। \textcolor{red}{प्रगृह्य} इति पदम्। तस्य च प्रोपसर्गस्य \textcolor{red}{विनाऽपि प्रत्ययं पूर्वोत्तर\-पद\-लोपो वक्तव्यः} (वा॰~५.३.८३) इति वार्त्तिकेन लोपे तथा च \textcolor{red}{जात\-संस्कारो न निवर्तते} इति परिभाषा\-बलेन ल्यबपि न निवृत्त इत्थं नापाणिनीयता। यद्वा \textcolor{red}{गृह्य}\-शब्दो हि क्यप्प्रत्ययान्तः।\footnote{\textcolor{red}{पदास्वैरिबाह्यापक्ष्येषु च} (पा॰सू॰~३.१.११९) इत्यनेन।} कित्वात्सम्प्रसारणम्।\footnote{\textcolor{red}{ग्रहि\-ज्या\-वयि\-व्यधि\-वष्टि\-विचति\-वृश्चति\-पृच्छति\-भृज्जतीनां ङिति च} (पा॰सू॰~६.१.१६) इत्यनेन।}
एवं \textcolor{red}{गृह्यो ग्रहीतुं योग्यः पुत्रः सौम्यः} इति तात्पर्यम्।\footnote{अस्वैरीति भावः। \textcolor{red}{गृह्यः, त्रि॰~(गृह्यते स्वाम्यादिभिरिति। ग्रह + क्यप्।), अस्वैरी। अस्वतन्त्रः। पक्षः। इति विश्वमेदिन्यौ} इति शब्द\-कल्पद्रुमः।} तथा च \textcolor{red}{गृह्यश्चासौ पुत्रश्चेति गृह्यपुत्रः} इति कर्मधारयः। \textcolor{red}{आदाय} इति चाध्याहार्यम्।\footnote{\textcolor{red}{सलिलमादाय गृह्यपुत्रो नायाति अत्र किं वा कारणम्} इति योजना।}\end{sloppypar}
\section[क्रन्दमानौ]{क्रन्दमानौ}
\centering\textcolor{blue}{हाहेति क्रन्दमानौ तौ पुत्रपुत्रेत्यवोचताम्।\nopagebreak\\
जलं देहीति पुत्रेति किमर्थं न ददास्यलम्॥}\nopagebreak\\
\raggedleft{–~अ॰रा॰~२.७.४३}\\
\begin{sloppypar}\hyphenrules{nohyphenation}\justifying\noindent\hspace{10mm} अत्र मुमूर्षु\-दम्पत्योर्दशां कौशल्या\-समक्षं दशरथो वर्णयति \textcolor{red}{हाहेति क्रन्दमानौ तौ}। \textcolor{red}{क्रन्द्‌}\-धातुः (\textcolor{red}{क्रदिँ आह्वाने रोदने च} धा॰पा॰~७१) परस्मैपदी। \textcolor{red}{इदितो नुम् धातोः} (पा॰सू॰~७.१.५८) इत्यनेन \textcolor{red}{नुम्}। ततश्च \textcolor{red}{क्रन्दत इति क्रन्दन्तौ} अयमेव प्रयोगः सामान्यतः पाणिनीयः। \textcolor{red}{क्रन्दमानौ} इति कथमिति चेत्। \textcolor{red}{कर्तरि कर्म\-व्यतिहारे} (पा॰सू॰~१.३.१४) इत्यनेनात्रात्मनेपदम्। कर्म\-व्यतिहारो नाम कार्य\-परिवर्तनम्। \textcolor{red}{अन्यस्य योग्यं लवनमन्यः करोति व्यतिलुनीते} (वै॰सि॰कौ॰~२६८०)। तथैवाऽत्र साधारण\-जीवाचरितं क्रन्दनं तपःपूतावपीमौ कुरुतः। अत आत्मनेपदम्। \textcolor{red}{क्रन्देते इति क्रन्दमानौ}। शानच्प्रत्यये शप्यनुबन्ध\-कार्ये \textcolor{red}{आने मुक्} (पा॰सू॰~७.२.८२) इत्यनेन मुगागमे विभक्ति\-कार्ये \textcolor{red}{क्रन्दमानौ}। यद्वा \textcolor{red}{क्रन्दतीति क्रन्दम्}। क्रन्दनशीलम्। पचादित्वादच्।\footnote{\textcolor{red}{नन्दि\-ग्रहि\-पचादिभ्यो ल्युणिन्यचः} (पा॰सू॰~३.१.१३४) इत्यनेन।} \textcolor{red}{क्रन्दं मानं शरीर\-दशा ययोस्तौ क्रन्दमानौ} इति बहुव्रीहौ नापाणिनीयता\-शङ्का\-पङ्क\-कलङ्कावकाशः।\end{sloppypar}
\section[विलपितम्]{विलपितम्}
\centering\textcolor{blue}{असमर्प्यैव रामाय राज्ञे मां क्व गतोऽसि भोः।\nopagebreak\\
इति विलपितं पुत्रं पतितं मुक्तमूर्धजम्॥}\nopagebreak\\
\raggedleft{–~अ॰रा॰~२.७.६७}\\
\begin{sloppypar}\hyphenrules{nohyphenation}\justifying\noindent\hspace{10mm} दशरथ\-मरणानन्तरं वसिष्ठ\-निर्देशाद्धावकैरयोध्यां नीतो भरतः कैकयीतः सकल\-वृत्तान्तं विज्ञायोपरतं पितरमुद्दिश्य विलपति। \textcolor{red}{असमर्प्यैव} इत्यादि। अत्र \textcolor{red}{विलपितम्} इति भरतस्य विशेषणं \textcolor{red}{विलपितं पुत्रम्} इति। \textcolor{red}{क्त}\-प्रत्ययस्य कर्मणि भावे च विधानात् \textcolor{red}{क्तवतु}\-प्रत्ययस्य पारिशेष्यात्कर्तरि विधानात् \textcolor{red}{विलपितवन्तं पुत्रम्} इति प्रयोक्तव्यमासीत्। किन्त्वविवक्षातः कर्मणो धातुमिममकर्मकं मत्वा\footnote{\textcolor{red}{धातोरर्थान्तरे वृत्तेर्धात्वर्थेनोपसङ्ग्रहात्। प्रसिद्धेरविवक्षातः कर्मणोऽकर्मिका क्रिया॥} (वा॰प॰~३.७.८८)।} \textcolor{red}{गत्यर्थाकर्मक\-श्लिष\-शीङ्स्थास\-वस\-जन\-रुह\-जी\-र्यतिभ्यश्च} (पा॰सू॰~३.४.७२) इत्यनेन कर्तरि \textcolor{red}{क्त}\-प्रत्ययः। यद्वा भावे \textcolor{red}{क्त}\-प्रत्यये कृते\footnote{\textcolor{red}{नपुंसके भावे क्तः} (पा॰सू॰~३.३.११४) इत्यनेन।} \textcolor{red}{विलपितमस्त्यस्य} इत्यर्शआद्यजन्तं \textcolor{red}{विलपितम्}।\footnote{\textcolor{red}{अर्शआदिभ्योऽच्} (पा॰सू॰~५.२.१२७) इत्यनेन।}
अथवा \textcolor{red}{विलपनं विलपः}। 
पचाद्यच्।\footnote{\textcolor{red}{अजपि सर्वधातुभ्यः} (वा॰~३.१.१३४) इत्यनेन। बाहुलकाद्भावे। 
} \textcolor{red}{विलपमितो विलपितस्तं विलपितम्} इति द्वितीयान्त\-विलप\-शब्दस्य क्तान्त\-\textcolor{red}{इत}\-इत्यनेन समासः। शकन्ध्वादित्वात्पूर्वोक्त\-दिशा पर\-रूपे \textcolor{red}{विलपितम्}। यद्वा \textcolor{red}{विलप}\-शब्दं तारकादि\-गणे मत्वा 
\textcolor{red}{विलपोऽस्य सञ्जातः} 
इति विग्रहे \textcolor{red}{तदस्य सञ्जातं तारकादिभ्य इतच्} (पा॰सू॰~५.२.३६) इत्यनेनेतच्यनुबन्ध\-कार्ये
भ\-सञ्ज्ञालोपयोर्विभक्ति\-कार्ये
\textcolor{red}{विलपितम्}।\footnote{\textcolor{red}{यचि भम्} (पा॰सू॰~१.४.१८) इत्यनेन भसञ्ज्ञा। \textcolor{red}{यस्येति च} (पा॰सू॰~६.४.१४८) इत्यनेनालोपः। अमि विभक्तौ \textcolor{red}{अमि पूर्वः} (पा॰सू॰~६.१.१०७) इत्यनेन पूर्वरूपम्।} यद्वा \textcolor{red}{विलपमाचष्टे विलपं करोति वा विलपयति}।\footnote{विलप~\arrow \textcolor{red}{तत्करोति तदाचष्टे} (धा॰पा॰ ग॰सू॰)~\arrow विलप~णिच्~\arrow विलप~इ~\arrow \textcolor{red}{णाविष्ठवत्प्राति\-पदिकस्य पुंवद्भाव\-रभाव\-टिलोप\-यणादि\-परार्थम्} (वा॰~६.४.४८)~\arrow विलप्~इ~\arrow विलपि~\arrow \textcolor{red}{सनाद्यन्ता धातवः} (पा॰सू॰~३.१.३२)~\arrow धातु\-सञ्ज्ञा~\arrow \textcolor{red}{शेषात्कर्तरि परस्मैपदम्} (पा॰सू॰~१.३.७८)~\arrow \textcolor{red}{वर्तमाने लट्} (पा॰सू॰~३.२.१२३)~\arrow विलपि~तिप्~\arrow विलपि~ति~\arrow \textcolor{red}{कर्तरि शप्‌} (पा॰सू॰~३.१.६८)~\arrow विलपि~शप्~ति~\arrow विलपि~अ~ति~\arrow \textcolor{red}{सार्वधातुकार्ध\-धातुकयोः} (पा॰सू॰~७.३.८४)~\arrow विलपे~अ~ति~\arrow \textcolor{red}{एचोऽयवायावः} (पा॰सू॰~६.१.७८)~\arrow विलपय्~अ~ति~\arrow विलपयति।} \textcolor{red}{विलपयतीति विलपितस्तं विलपितम्}।\footnote{विलपि~\arrow पूर्ववद्धातुसञ्ज्ञा~\arrow \textcolor{red}{गत्यर्थाकर्मक\-श्लिष\-शीङ्स्थास\-वस\-जन\-रुह\-जी\-र्यतिभ्यश्च} (पा॰सू॰~३.४.७२)~\arrow विलपि~क्त~\arrow विलपि~त~\arrow \textcolor{red}{आर्धधातुकस्येड्वलादेः} (पा॰सू॰~७.२.३५)~\arrow विलपि~इट्~त~\arrow विलपि~इ~त~\arrow \textcolor{red}{निष्ठायां सेटि} (पा॰सू॰~६.४.५२)~\arrow विलप्~इ~त~\arrow विलपित~\arrow विभक्तिकार्यम्~\arrow विलपित~अम्~\arrow \textcolor{red}{अमि पूर्वः} (पा॰सू॰~६.१.१०७)~\arrow विलपितम्।} इत्थमाचक्षाण\-णिजन्तात्कर्तरि क्त\-प्रत्ययः।\footnote{\textcolor{red}{गत्यर्थाकर्मक\-श्लिष\-शीङ्स्थास\-वस\-जन\-रुह\-जीर्यतिभ्यश्च} (पा॰सू॰~३.४.७२) इत्यनेन कर्तरि क्तः।} श्रीराम एव स्वयं विलपित इव। वस्तुतस्तु तस्य हृदये भक्ति\-भावेन सदाऽपि श्रीरामः सन्निहित एव। केवलं भावुकेभ्यो भरतः प्रेम\-दिग्दर्शनं कारयति।\end{sloppypar}
\section[दूरे स्थाप्य]{दूरे स्थाप्य}
\centering\textcolor{blue}{तीर्त्वा गङ्गां ययौ शीघ्रं भरद्वाजाश्रमं प्रति।\nopagebreak\\
दूरे स्थाप्य महासैन्यं भरतः सानुजो ययौ॥}\nopagebreak\\
\raggedleft{–~अ॰रा॰~२.८.४१}\\
\begin{sloppypar}\hyphenrules{nohyphenation}\justifying\noindent\hspace{10mm} अत्र समस्त\-पदं विनाऽपि \textcolor{red}{स्थाप्य} इति ल्यबन्त\-प्रयोगः कथमिति चेत्। अत्र मयूर\-व्यंसकादित्वात्समासः।\footnote{\textcolor{red}{मयूरव्यंसकादयश्च} (पा॰सू॰~२.१.७२) इत्यनेन।} तथा च \textcolor{red}{हलदन्तात्सप्तम्या सञ्ज्ञायाम्} (पा॰सू॰~६.३.९) इत्यनेन सप्तम्या अलुक्ततो ल्यप्प्रत्यये\footnote{\textcolor{red}{समासेऽनञ्पूर्वे क्त्वो ल्यप्‌} (पा॰सू॰~७.१.३७) इत्यनेन क्त्वो ल्यबादेश इति भावः।} \textcolor{red}{दूरेस्थाप्य} इति। यद्वा \textcolor{red}{दूरे} इति सप्तमी\-प्रतिरूपकाव्ययं ततश्च सुविभक्तेरेव \textcolor{red}{अव्ययादाप्सुपः} (पा॰सू॰~२.४.८२) इत्यनेन लुक्। इत्थं \textcolor{red}{दूरेस्थाप्य} इति साधु।\footnote{यद्वाऽत्र \textcolor{red}{संस्थाप्य} इति प्रयोगः। \textcolor{red}{विनाऽपि प्रत्ययं पूर्वोत्तर\-पद\-लोपो वक्तव्यः} (वा॰~५.३.८३) इत्यनेन \textcolor{red}{सम्} उपसर्गस्य लोपः।}\end{sloppypar}
\vspace{2mm}
\centering ॥ इत्ययोध्याकाण्डीयप्रयोगाणां विमर्शः ॥\nopagebreak\\
\vspace{4mm}
\centering इत्यध्यात्म\-रामायणेऽपाणिनीय\-प्रयोगाणां\-विमर्श\-नामके शोध\-प्रबन्धे द्वितीयाध्याये प्रथम\-परिच्छेदः।\\
\pagebreak
\pdfbookmark[1]{द्वितीयः परिच्छेदः}{Chap2Part2}
\phantomsection
\addtocontents{toc}{\protect\setcounter{tocdepth}{1}}
\addcontentsline{toc}{section}{द्वितीयः परिच्छेदः}
\addtocontents{toc}{\protect\setcounter{tocdepth}{0}}
\centering ॥ अथ द्वितीयाध्याये द्वितीयः परिच्छेदः ॥\nopagebreak\\
\vspace{4mm}
\pdfbookmark[2]{अरण्यकाण्डम्}{Chap2Part2Kanda3}
\phantomsection
\addtocontents{toc}{\protect\setcounter{tocdepth}{2}}
\addcontentsline{toc}{subsection}{अरण्यकाण्डीयप्रयोगाणां विमर्शः}
\addtocontents{toc}{\protect\setcounter{tocdepth}{0}}
\centering ॥ अथारण्यकाण्डीयप्रयोगाणां विमर्शः ॥\nopagebreak\\
\section[यातः]{यातः}
\centering\textcolor{blue}{को वा दयालुः स्मृतकामधेनुरन्यो जगत्यां रघुनायकादहो।\nopagebreak\\
स्मृतो मया नित्यमनन्यभाजा ज्ञात्वा स्मृतिं मे स्वयमेव यातः॥}\nopagebreak\\
\raggedleft{–~अ॰रा॰~३.२.८}\\
\begin{sloppypar}\hyphenrules{nohyphenation}\justifying\noindent\hspace{10mm} अत्र \textcolor{red}{आयातः} इत्यर्थे \textcolor{red}{यातः} इति प्रयुक्तम्। यतो हि \textcolor{red}{आङ्‌}\-उपसर्गस्य लोपे सति\footnote{\textcolor{red}{विनाऽपि प्रत्ययं पूर्वोत्तर\-पद\-लोपो वक्तव्यः} (वा॰~५.३.८३) इत्यनेन।} \textcolor{red}{यः शिष्यते स लुप्यमानार्थाभिधायी}\footnote{मूलं मृग्यम्।} इत्यनेन तादृशार्थस्य बोधकत्वात्।\end{sloppypar}
\section[अरूपिणः]{अरूपिणः}
\centering\textcolor{blue}{पश्यामि राम तव रूपमरूपिणोऽपि\nopagebreak\\
मायाविडम्बनकृतं सुमनुष्यवेषम्।\nopagebreak\\
कन्दर्पकोटिसुभगं कमनीयचाप-\nopagebreak\\
बाणं दयार्द्रहृदयं स्मितचारुवक्त्रम्॥}\nopagebreak\\
\raggedleft{–~अ॰रा॰~३.२.३२}\\
\begin{sloppypar}\hyphenrules{nohyphenation}\justifying\noindent\hspace{10mm} अत्र \textcolor{red}{न रूपमरूपम्}।\footnote{\textcolor{red}{नञ्‌} (पा॰सू॰~२.२.६) इत्यनेन नञ्तत्पुरुष\-समासे \textcolor{red}{नलोपो नञः} (पा॰सू॰~६.३.७३) इत्यनेन नलोपे विभक्तिकार्ये सिद्धम्।} \textcolor{red}{अरूपमस्त्यस्येत्यरूपी तस्यारूपिणः}।\footnote{\textcolor{red}{अत इनिठनौ} (पा॰सू॰~५.२.११५) इत्यनेन \textcolor{red}{अरूप}\-प्रातिपदिकान्मत्वर्थीये \textcolor{red}{इनि}प्रत्यये विभक्तिकार्ये सिद्धम्।} अत्र \textcolor{red}{न कर्मधारयान्मत्वर्थीयो बहुव्रीहिश्चेत्तदर्थ\-प्रतिपत्ति\-करः}\footnote{मूलं मृग्यम्।} अयं नियमो जागरूकः। तस्मात् \textcolor{red}{नास्ति रूपं यस्य सोऽरूपस्तस्यारूपस्य} इति बहुव्रीहावेवार्थ\-बोधे \textcolor{red}{अरूपिणः} इति कथमिति चेत्। \textcolor{red}{अरूपिणः} इत्यत्र बहुव्रीहेरीप्सितोऽर्थो न लभ्यते। यतो ह्यत्र प्राशस्त्ये भूमत्वे नित्ययोगे चेनिः।\footnote{\textcolor{red}{भूम\-निन्दा\-प्रशंसासु नित्ययोगेऽति\-शायने। सम्बन्धेऽस्ति\-विवक्षायां भवन्ति मतुबादयः॥} (भा॰पा॰सू॰~५.२.९४)।} भगवतो रूपस्य भूयस्त्वं प्रशस्तत्वञ्च नित्य\-योगत्वमपि सिद्धमेव। तद्बहुव्रीहिणा कथमपि लब्धुमशक्यत्वात्।\footnote{न भूम\-प्रशस्त\-नित्ययुक्तं रूपमस्येत्यरूपी तस्यारूपिणोऽपि तद्वद्रूपं पश्यामीति भावः। अस्मिन्नेव पद्ये \textcolor{red}{सुमनुष्य\-वेषं कन्दर्प\-कोटि\-सुभगं कमनीय\-चाप\-बाणं स्मित\-चारुवक्त्रम्} इति प्राशस्त्यमुक्तम्। अन्यच्चास्मिन् ग्रन्थे \textcolor{red}{रामश्चापधरो नित्यं तूणीबाणान्वितः प्रभुः} (अ॰रा॰~१.३.६२) इत्यत्र चाप\-तूणी\-बाण\-सहित\-रूपस्य नित्ययोग उक्तः। अहल्या\-स्तुतौ च \textcolor{red}{कार्य\-कारण\-कर्तृत्व\-फल\-साधन\-भेदतः। एको विभासि राम त्वं मायया बहुरूपया॥} (अ॰रा॰~१.५.५४) इत्यत्र श्रीरामभद्रस्य रूपभूमत्वं चोक्तम्।} अथवा \textcolor{red}{अरूपयितुं तच्छील इत्यरूपी तस्यारूपिणः} इति ताच्छील्ये णिनिः। \textcolor{red}{सुप्यजातौ णिनिस्ताच्छील्ये} (पा॰सू॰~३.२.७८) इति सूत्रेण।\end{sloppypar}
\section[मदुपासनात्]{मदुपासनात्}
\centering\textcolor{blue}{इत्येवं स्तुवस्तस्य रामः सुस्मितमब्रवीत्।\nopagebreak\\
मुने जानामि ते चित्तं निर्मलं मदुपासनात्॥}\nopagebreak\\
\raggedleft{–~अ॰रा॰~३.२.३५}\\
\begin{sloppypar}\hyphenrules{nohyphenation}\justifying\noindent\hspace{10mm} अत्र \textcolor{red}{ण्यास\-श्रन्थो युच्} (पा॰सू॰~३.३.१०७) इत्यनेन युच्कथं न। \textcolor{red}{मदुपासनात्} इत्यपाणिनीय इव।\footnote{कर्तृभिन्न\-कारके युचि तु \textcolor{red}{उपासना} इति रूपम्। तस्मात् \textcolor{red}{मदुपासनायाः} इति पञ्चम्यन्त\-रूपम्। यथा योगसूत्रे भोज\-वृत्तौ \textcolor{red}{उपासनायाः फलमाह} (यो॰सू॰ भो॰वृ॰~१.२८) इति षष्ठ्यन्त\-रूपम्।} अत्र \textcolor{red}{ल्युट् च} (पा॰सू॰~३.३.११५) इत्यनेन भावे नपुंसके \textcolor{red}{ल्युट्}।\footnote{एतेन \textcolor{red}{उपासनम्} इत्यपि पाणिनीयम्। \textcolor{red}{वरिवस्या तु शुश्रूषा परिचर्याऽप्युपासनम्} (अ॰पु॰~३६५.७) इत्यग्नि\-पुराण\-प्रयोगादपि। अमरकोषे तु \textcolor{red}{वरिवस्या तु शुश्रूषा परिचर्याऽप्युपासना} (अ॰को॰~२.६.३५) इति पाठः।} लकार\-टकारयोरनुबन्ध\-कार्ये \textcolor{red}{युवोरनाकौ} (पा॰सू॰~७.१.१) इत्यनेनानादेशे पञ्चम्येक\-वचने \textcolor{red}{मदुपासनात्}।\footnote{यथा तत्रैव योगसूत्रे भोज\-वृत्तौ \textcolor{red}{उपासनमाह} (यो॰सू॰ भो॰वृ॰~१.२८) इति द्वितीयान्त\-रूपम्।}\end{sloppypar}
\section[ध्यायमानः काङ्क्षमाणः]{ध्यायमानः काङ्क्षमाणः}
\centering\textcolor{blue}{शीघ्रमानय भद्रं ते रामं मम हृदि स्थितम्।\nopagebreak\\
तमेव ध्यायमानोऽहं काङ्क्षमाणोऽत्र संस्थितः॥}\nopagebreak\\
\raggedleft{–~अ॰रा॰~३.३.१०}\\
\begin{sloppypar}\hyphenrules{nohyphenation}\justifying\noindent\hspace{10mm} प्रयोगाविमावध्यात्म\-रामायणेऽरण्य\-काण्डस्य तृतीय\-सर्गीयौ। श्रीरामभद्रः सीता\-लक्ष्मण\-समेतः पावित\-मुनि\-जन\-निकेतोऽगस्त्यस्याश्रममुप\-तिष्ठमानः\footnote{\textcolor{red}{उपतिष्ठमानः} इत्यत्र \textcolor{red}{उपाद्देवपूजा\-सङ्गतकरण\-मित्रकरण\-पथिष्विति} (वा॰~१.३.२४) इत्यनेन सङ्गतकरणे पथि वाऽऽत्मनेपदम्। यद्वा \textcolor{red}{वा लिप्सायाम्} (वा॰~१.३.२४) इत्यनेन।} सुतीक्ष्णेन सूचयति। रामागमन\-श्रवण\-सञ्जात\-हर्षोऽगस्त्यः प्रोवाच यत् \textcolor{red}{राममानयाहं तमेव ध्यायमानस्तिष्ठामि}। \textcolor{red}{ध्यै}\-धातुः (\textcolor{red}{ध्यै चिन्तायाम्} धा॰पा॰~९०८) परस्मैपदी। तथा सति लटि तिपि शपि गुणेऽयादेशे \textcolor{red}{ध्यायति}। शतरि च \textcolor{red}{ध्यायतीति ध्यायन्} इत्येव प्रयोगः। \textcolor{red}{ध्यायमानः} इति कथम्। अत्र \textcolor{red}{कर्तरि कर्म\-व्यतिहारे} (पा॰सू॰~१.३.१४) इत्यात्मनेपदम्। यतो हि ध्यानस्य बुद्धि\-विषयत्वात्तदेव ध्यानमहं\-पद\-वाच्य\-प्रत्यग्भिन्न\-चैतन्य आत्मन्यारोप्यतेऽतः कर्म\-व्यतिहारतयाऽऽत्मनेपदम्। ततश्च \textcolor{red}{ध्यायत इति ध्यायमानः} इति विग्रहे \textcolor{red}{ध्यै}\-धातोर्लटि\footnote{\textcolor{red}{वर्तमाने लट्} (पा॰सू॰~३.२.१२३) इत्यनेन।} तस्य शानजादेशे\footnote{\textcolor{red}{लटः शतृशानचावप्रथमा\-समानाधिकरणे} (पा॰सू॰~३.२.१२४) इत्यनेन।} शपि \textcolor{red}{आने मुक्} (पा॰सू॰~७.२.८२) इत्यनेन मुगागमे विभक्ति\-कार्ये \textcolor{red}{ध्यायमानः}। अथवा \textcolor{red}{ध्यायम् ध्यायम्} इत्याभीक्ष्ण्ये णमुल्।\footnote{\textcolor{red}{आभीक्ष्ण्ये णमुल् च} (पा॰सू॰~३.४.२२) इत्यनेन।} प्रथमस्य \textcolor{red}{ध्यायम्} इत्यस्य लोपे \textcolor{red}{आसमन्तादनितीत्यानः}।\footnote{\textcolor{red}{आङ्‌}\-उपसर्ग\-पूर्वकात् \textcolor{red}{अनँ प्राणने} (धा॰पा॰~१०७०) इति धातोः \textcolor{red}{नन्दि\-ग्रह\-पचादिभ्यो ल्युणिन्यचः} (पा॰सू॰~३.१.१३४) इत्यनेन पचाद्यचि।} ततो वर्णसम्मेलने \textcolor{red}{ध्यायमानः}। एवमेव \textcolor{red}{काङ्क्षमाणः} इत्यत्रापि \textcolor{red}{काङ्क्षति} इति परस्मैपदी। तत्रापि कर्म\-व्यतिहार आत्मने\-पदं ततः शानच्।\footnote{काङ्क्षा तु सांसारिकाणां जनानां कृते। अगस्त्यश्चर्षिः। दिव्यदृष्ट्या भगवद्रूप\-द्रष्टर्षिर्भूत्वाऽपि लौकिक\-नेत्र\-लाभार्थं भगवद्दर्शन\-काङ्क्षामाचरतीति कर्म\-व्यतिहार\-विवक्षायामात्मने\-पदम्। एवमेव \textcolor{red}{न काङ्क्षे विजयं कृष्ण न च राज्यं सुखानि च} (भ॰गी॰~१.३२) इत्यत्र। विजय\-राज्य\-सुखानामकाङ्क्षा न राजोचिता।}\end{sloppypar}
\section[प्रतीक्षन्]{प्रतीक्षन्}
\centering\textcolor{blue}{त्वदागमनमेवाहं प्रतीक्षन्समवस्थितः।\nopagebreak\\
यदा क्षीरसमुद्रान्ते ब्रह्मणा प्रार्थितः पुरा॥}\nopagebreak\\
\raggedleft{–~अ॰रा॰~३.३.१८}\\
\begin{sloppypar}\hyphenrules{nohyphenation}\justifying\noindent\hspace{10mm} सीता\-लक्ष्मण\-सेव्यमानो निर्मानो भगवान् श्रीरामोऽगस्त्य\-दर्शनार्थमाश्रम\-द्वार्युपस्थितः। तदागमन\-श्रवण\-सञ्जात\-कुतूहलो विह्वलोऽगस्त्यः प्रणिजगाद यत् \textcolor{red}{अहमपि भगवत आगमनं प्रतीक्षे}। अत्र \textcolor{red}{प्रति}\-पूर्वकः \textcolor{red}{ईक्ष्‌}\-धातुः (\textcolor{red}{ईक्षँ दर्शने} धा॰पा॰~६१०) आत्मनेपदी दर्शनार्थः। ततश्च \textcolor{red}{प्रतीक्षते} इति विग्रहे शानच्प्रत्यये कृते\footnote{\textcolor{red}{लटः शतृशानचावप्रथमा\-समानाधिकरणे} (पा॰सू॰~३.२.१२४) इत्यनेन।} मुगागमे\footnote{\textcolor{red}{आने मुक्} (पा॰सू॰~७.२.८२) इत्यनेन।} विभक्ति\-कार्ये \textcolor{red}{प्रतीक्षमाणः} इति पाणिनीयः। आत्मनेपदीयत्वाद्दुर्गमः \textcolor{red}{शतृ}\-प्रत्ययः। एवं \textcolor{red}{प्रतीक्षन्} इति कथमिति चेत्। उच्यते। \textcolor{red}{अनुदात्तेत्त्व\-लक्षणमात्मने\-पदमनित्यम्} (प॰शे॰~९३.४) इति नियमेन परस्मैपदीयत्वाच्छतृ\-प्रत्यये \textcolor{red}{प्रतीक्षन्}। यद्वा \textcolor{red}{प्रतीक्षणं प्रतीक्षा} इति विग्रहे भावे \textcolor{red}{आङ्} औणादिक\-प्रत्यये\footnote{\textcolor{red}{गुरोश्च हलः} (पा॰सू॰~३.३.१०३) इत्यनेन स्त्रियां भावे \textcolor{red}{अ}\-प्रत्यये \textcolor{red}{अजाद्यतष्टाप्‌} (पा॰सू॰~४.१.४) इत्यनेन टापि वा। \pageref{sec:aveksati}तमे पृष्ठे \ref{sec:aveksati} \nameref{sec:aveksati} इति प्रयोगस्य विमर्शं पश्यन्तु।} \textcolor{red}{प्रतीक्षां करोतीति प्रतीक्षयति}।\footnote{प्रतीक्षा~\arrow \textcolor{red}{तत्करोति तदाचष्टे} (धा॰पा॰ ग॰सू॰~१८७)~\arrow प्रतीक्षा~णिच्~\arrow प्रतीक्षा~इ~\arrow \textcolor{red}{णाविष्ठवत्प्राति\-पदिकस्य पुंवद्भाव\-रभाव\-टिलोप\-यणादि\-परार्थम्} (वा॰~६.४.४८)~\arrow प्रतीक्ष्~इ~\arrow प्रतीक्षि~\arrow \textcolor{red}{सनाद्यन्ता धातवः} (पा॰सू॰~३.१.३२)~\arrow धातुसञ्ज्ञा~\arrow \textcolor{red}{शेषात्कर्तरि परस्मैपदम्} (पा॰सू॰~१.३.७८)~\arrow \textcolor{red}{वर्तमाने लट्} (पा॰सू॰~३.२.१२३)~\arrow प्रतीक्षि~लट्~\arrow प्रतीक्षि~तिप्~\arrow \textcolor{red}{कर्तरि शप्‌} (पा॰सू॰~३.१.६८)~\arrow प्रतीक्षि~शप्~तिप्~\arrow प्रतीक्षि~अ~ति~\arrow \textcolor{red}{सार्वधातुकार्ध\-धातुकयोः} (पा॰सू॰~७.३.८४)~\arrow गुणः~\arrow प्रतीक्षे~अ~ति~\arrow \textcolor{red}{एचोऽयवायावः} (पा॰सू॰~६.१.७८)~\arrow प्रतीक्षय्~अ~ति~\arrow प्रतीक्षयति।} \textcolor{red}{प्रतीक्षयतीति प्रतीक्षः}।\footnote{प्रतीक्षि~\arrow पूर्ववद्धातु\-सञ्ज्ञा~\arrow \textcolor{red}{नन्दि\-ग्रहि\-पचादिभ्यो ल्युणिन्यचः} (पा॰सू॰~३.१.१३४)~\arrow प्रतीक्षि~अच्~\arrow प्रतीक्षि~अ~\arrow \textcolor{red}{णेरनिटि} (पा॰सू॰~६.४.५१)~\arrow प्रतीक्ष्~अ~\arrow प्रतीक्ष~\arrow विभक्तिकार्यम्~\arrow प्रतीक्षः।} \textcolor{red}{प्रतीक्ष इवाऽचरन्निति प्रतीक्षन्}।\footnote{प्रतीक्ष~\arrow \textcolor{red}{सर्वप्राति\-पदिकेभ्य आचारे क्विब्वा वक्तव्यः} (वा॰~३.१.११)~\arrow प्रतीक्ष~क्विँप्~\arrow प्रतीक्ष~व्~\arrow \textcolor{red}{वेरपृक्तस्य} (पा॰सू॰~६.१.६७)~\arrow प्रतीक्ष~\arrow \textcolor{red}{सनाद्यन्ता धातवः} (पा॰सू॰~३.१.३२)~\arrow धातुसञ्ज्ञा~\arrow \textcolor{red}{शेषात्कर्तरि परस्मैपदम्} (पा॰सू॰~१.३.७८)~\arrow \textcolor{red}{वर्तमाने लट्} (पा॰सू॰~३.२.१२३)~\arrow प्रतीक्ष~लट्~\arrow \textcolor{red}{लटः शतृ\-शानचावप्रथमा\-समानाधिकरणे} (पा॰सू॰~३.२.१२४)~\arrow प्रतीक्ष~शतृँ~\arrow प्रतीक्ष~अत्~\arrow \textcolor{red}{कर्तरि शप्} (पा॰सू॰~३.१.६८)~\arrow प्रतीक्ष~शप्~अत्~\arrow प्रतीक्ष~अ~अत्~\arrow \textcolor{red}{अतो गुणे} (पा॰सू॰~६.१.९७)~\arrow प्रतीक्ष~अत्~\arrow \textcolor{red}{अतो गुणे} (पा॰सू॰~६.१.९७)~\arrow प्रतीक्षत्~\arrow \textcolor{red}{कृत्तद्धित\-समासाश्च} (पा॰सू॰~१.२.४६)~\arrow प्रातिपदिक\-सञ्ज्ञा~\arrow विभक्ति\-कार्यम्~\arrow प्रतीक्षत्~सुँ~\arrow \textcolor{red}{उगिदचां सर्वनामस्थानेऽधातोः} (पा॰सू॰~७.१.७०)~\arrow \textcolor{red}{मिदचोऽन्त्यात्परः} (पा॰सू॰~१.१.४७)~\arrow प्रतीक्ष~नुँम्~त्~सुँ~\arrow प्रतीक्ष~न्~त्~सुँ~\arrow \textcolor{red}{हल्ङ्याब्भ्यो दीर्घात्सुतिस्यपृक्तं हल्} (पा॰सू॰~६.१.६८)~\arrow प्रतीक्ष~न्~त्~\arrow \textcolor{red}{संयोगान्तस्य लोपः} (पा॰सू॰~८.२.२३)~\arrow प्रतीक्षन्।} एवं आचक्षाण\-णिजन्तान्त\-कर्तर्यच्। तत आचारे क्विप्प्रत्यये लटि शतरि शपि पररूपे विभक्तिकार्ये \textcolor{red}{प्रतीक्षन्}।\end{sloppypar}
\section[समीपतः]{समीपतः}
\centering\textcolor{blue}{एकदा गौतमीतीरे पञ्चवट्यां समीपतः।\nopagebreak\\
पद्मवज्राङ्कुशाङ्गानि पदानि जगतीपतेः॥}\nopagebreak\\
\raggedleft{–~अ॰रा॰~३.५.२}\\
\begin{sloppypar}\hyphenrules{nohyphenation}\justifying\noindent\hspace{10mm} अत्र विश्लेषाभावात् \textcolor{red}{ध्रुवमपायेऽपादानम्} (पा॰सू॰~१.४.२४) इत्यनेनापादान\-सञ्ज्ञा तु नैव विचारसहा। तदभावे \textcolor{red}{पञ्चम्यास्तसिल्} (पा॰सू॰~५.३.७) इत्यनेन तसिल्प्रत्ययोऽपि नोपयुक्तः। तथा च सामीप्य\-वाचक\-शब्दात्सप्तम्यर्थ एव \textcolor{red}{तसि\-प्रकरण आद्यादिभ्य उपसङ्ख्यानम्} (वा॰~५.४.४४) इत्यनेन \textcolor{red}{तसि}\-प्रत्यये सिद्धमिदम्।\footnote{तसेः सार्व\-विभक्तिकत्वं तदन्तानामाकृति\-गणत्वं च \pageref{fn:yatah}तमे पृष्ठे \ref{fn:yatah}तम्यां पादटिप्पण्यां स्पष्टीकृतम्।
}
 \textcolor{red}{समीप एव समीपतः}। न चात्र समीप\-वाचक\-शब्दे कथं सप्तमीति चेत्। \textcolor{red}{सप्तम्यधिकरणे च} (पा॰सू॰~२.३.३६) इत्यत्र \textcolor{red}{च}\-कार\-ग्रहणेन दूरान्तिकार्थ\-वाचक\-शब्देभ्यः सप्तमी।\footnote{\textcolor{red}{चकाराद्दूरान्तिकार्थेभ्यश्च ... दूरान्तिकार्थेभ्यः खल्वपि। दूरे ग्रामस्य। अन्तिके ग्रामस्य। अभ्याशे ग्रामस्य} (का॰वृ॰~२.३.३६)। \textcolor{red}{चकाराद्दूरान्तिकार्थेभ्यः} (वै॰सि॰कौ॰~६३३)।} समीपं ह्यन्तिकार्थ\-वाचकं ततः सप्तमी।\end{sloppypar}
\section[कनीयान्]{कनीयान्}
\centering\textcolor{blue}{एषा मे सुन्दरी भार्या सीता जनकनन्दिनी।\nopagebreak\\
स तु भ्राता कनीयान्मे लक्ष्मणोऽतीवसुन्दरः॥}\nopagebreak\\
\raggedleft{–~अ॰रा॰~३.५.९}\\
\begin{sloppypar}\hyphenrules{nohyphenation}\justifying\noindent\hspace{10mm} अत्र श्रीरामो लक्ष्मणस्य परिचयं कारयति यत् \textcolor{red}{स मे कनीयान् भ्राता}। \textcolor{red}{ईयसुन्} प्रत्ययो हि यत्र द्वयोर्विभागः।\footnote{\textcolor{red}{द्विवचनविभज्योपपदे तरबीयसुनौ} (पा॰सू॰~५.३.५७) इत्यनेन।} अत्र रामापेक्षया कनीयान् भरतः। कथं लक्ष्मणं कनीयांसं कथयतीति चेत्। अत्र राम\-लक्ष्मणयोर्द्वयोरेव विवक्षितत्वात्। भरत\-शत्रुघ्नावयोध्यायामरण्ये राम\-लक्षणौ। अनयोर्द्वन्द्वः शाश्वतः शेष\-शेषि\-भावात्। लीलायामपि राम\-लक्ष्मणयोः संयोगः सर्वत्र प्रसिद्धः। अतः शब्दार्थ\-सन्देहे विशेष\-स्मृति\-हेतूनां परिगणनावसरे साहचर्य उदाहरणं \textcolor{red}{राम\-लक्ष्मणौ}।\footnote{\textcolor{red}{संसर्गो विप्रयोगश्च साहचर्यं विरोधिता। अर्थः प्रकरणं लिङ्गं शब्दस्यान्यस्य सन्निधिः॥ सामर्थ्यमौचिती देशः कालो व्यक्तिः स्वरादयः। शब्दार्थस्यानवच्छेदे विशेषस्मृतिहेतवः॥} (वा॰प॰~२.३१५-३१६)।} यद्यपि \textcolor{red}{लक्ष्मण}\-शब्दस्य द्वावर्थौ लक्ष्मणो दाशरथि\-लक्ष्मणो दुर्योधन\-पुत्रश्च किन्तु राम\-साहचर्यादत्र दाशरथिर्लक्ष्मण एवार्थ\-बोध\-विषयो भवति। एवं \textcolor{red}{राम}\-शब्दस्य त्रयोऽर्था रामो जामदग्न्यो रामो दाशरथी रामो वासुदेवश्चेति। अत्रोच्चरिते राम\-शब्द उपस्थितेषु त्रिष्वर्थेषु को रामो गृह्यतामिति चेत् \textcolor{red}{राम\-लक्ष्मणौ} इति कथनेन दशरथ\-पुत्रस्य ग्रहणं भवति। अध्यात्म\-रामायणेऽपि~– \end{sloppypar}
\centering\textcolor{blue}{लक्ष्मणो रामचन्द्रेण शत्रुघ्नो भरतेन च।\nopagebreak\\
द्वन्द्वीभूय चरन्तौ तौ पायसांशानुसारतः॥}\nopagebreak\\
\raggedleft{–~अ॰रा॰~१.३.४२}\\
\begin{sloppypar}\hyphenrules{nohyphenation}\justifying\noindent\hspace{10mm} यद्वा \textcolor{red}{मे} इत्युच्चारण आत्मानं भ्रातृ\-समुदायात्पृथक्करोति। ज्येष्ठस्य भ्रातुः पितृवद्दायित्वं लोक\-प्रसिद्धम्।\footnote{\textcolor{red}{पितेव पालयेत्पुत्रान् ज्येष्ठो भ्रातॄन् यवीयसः। पुत्रवच्चापि वर्तेरन् ज्येष्ठे भ्रातरि धर्मतः॥} (म॰स्मृ॰~९.१०८)।} अतस्ते सन्ति त्रयो भ्रातरः। षष्ठी पालक\-पाल्य\-भाव\-सम्बन्धे। अहं त्रयाणां भ्रातॄणां पालको ज्येष्ठत्वात्। उपरते पितरि मे मय्येव सर्वेषामुत्तर\-दायित्वम्। वाल्मीकीयेऽपि विभीषणं शरणागतं स्वीकुर्वञ्छ्रीरामभद्रो रावण\-हननाय त्रयाणां भ्रातॄणां शपथं करोति। यथा~–\end{sloppypar}
\centering\textcolor{red}{अहत्वा रावणं सङ्ख्ये सपुत्रबलवाहनम्।\nopagebreak\\
अयोध्यां न प्रवेक्ष्यामि भ्रातृभिश्च त्रिभिः शपे॥}\nopagebreak\\
\raggedleft{–~वा॰रा॰~६.४१.७}\\
\begin{sloppypar}\hyphenrules{nohyphenation}\justifying\noindent इत्थं मम त्रिषु भ्रातृषु ज्येष्ठो भ्राता भरतोऽयोध्यामधितिष्ठति ततो लक्ष्मणः कनीयानिति श्रीरामचन्द्रस्य तात्पर्यं प्रतिभाति। अस्मादेव कारणात् \textcolor{red}{स तु भ्राता कनीयान्मे} इत्यत्र षष्ठी। अन्यथा \textcolor{red}{पञ्चमी विभक्ते} (पा॰सू॰~२.३.४२) इति सूत्रेण विभज्यमाने पञ्चमी स्यात्। तथा च \textcolor{red}{द्विवचनविभज्योपपदे तरबीयसुनौ} (पा॰सू॰~५.३.५७) इत्यनेन द्विवचन\-विभज्यमान\-वाचक उपपदे \textcolor{red}{तरप्} \textcolor{red}{ईयसुन्} च प्रत्ययो भवति। तत्र च पञ्चम्यनिवार्या। यथा \textcolor{red}{रामाच्छ्यामो लघुतरः}। \textcolor{red}{रामाद्भरतः कनीयान्}। अतः षष्ठीं दृष्ट्वाऽत्र रामस्य समुदायात्पृथग्भूतत्वं प्रतीयते। यद्वा
\textcolor{red}{भ्रियन्ते पुष्यन्त इति भ्रातरः}।\footnote{\textcolor{red}{भ्रातृ}\-शब्दो \textcolor{red}{भ्रातृपुत्रौ स्वसृ\-दुहितृभ्याम्} (पा॰सू॰~१.२.६८) इति सूत्रकार\-प्रयोगादेव सिद्धः। \textcolor{red}{नप्तृ\-नेष्टृ\-त्वष्टृ\-होतृ\-पोतृ\-भ्रातृ\-जामातृ\-मातृ\-पितृ\-दुहितृ} (प॰उ॰~२.९५) इत्युणादि\-सूत्र\-पाठभेदेन तृनन्तस्तृजन्तो वा निपात्यते। अन्यत्र \textcolor{red}{नप्तृ\-नेष्टृ\-त्वष्टृ\-क्षत्तृ\-होतृ\-पोतृ\- जामातृ\-पितृ\-दुहितृ} (प॰उ॰~२.९५) इति \textcolor{red}{भ्रातृ}\-शब्द\-रहितः पाठः। कोशेषूणादि\-सूत्र\-टीकासु विविधा व्युत्पत्तयः। \textcolor{red}{भ्राज-तृच् पृषो॰} इति वाचस्पत्यम्। \textcolor{red}{भ्राजते इति भ्राज + नप्तृ\-नेष्टृ\-त्वष्टृ\-होत्रिति प॰उ॰~२.९६ इति तृन्। निपात्यते च} इति शब्द\-कल्प\-द्रुमः। \textcolor{red}{भ्राजते। भ्राजृँ दीप्तौ (भ्वा॰आ॰से॰)। ‘नप्तृ\-नेष्टृ-’ इति साधु} (अ॰को॰ व्या॰सु॰~२.६.३६) इति व्याख्या\-सुधा। \textcolor{red}{भ्राजते दीप्यतेऽसौ भ्राता सोदर्यो वा} (उ॰को॰~२.९५) इति दयानन्द\-सरस्वती। \textcolor{red}{भ्राजृँ दीप्तौ} (पा॰सू॰~१८१) इत्यस्मात् \textcolor{red}{टुभ्राजृँ दीप्तौ} (धा॰पा॰~८३३) इत्यस्माद्वा तृनि तृचि वा पृषोदरादित्वाज्जकार\-लोप इत्येषां भावः। \textcolor{red}{‘भज सेवायाम्’ इति भ्राता} (प॰उ॰~श्वे॰वृ॰~२.९५) इति श्वेत\-वनवासि\-वृत्ति\-पाठभेदः। \textcolor{red}{भजँ सेवायाम्} (धा॰पा॰~९९८) इत्यस्मात्तृनि तृचि वा पृषोदरादित्वाज्जकार\-लोपो भ्रादेश्चेति भावः। अत्र प्रणेतारो \textcolor{red}{भ्रियन्ते पुष्यन्त इति भ्रातरः} इति। \textcolor{red}{भृञ् भरणे} (धा॰पा॰~८९८) इत्यस्माद्धातोः कर्मणि तृनि तृचि वा गुणे रपरत्वे पृषोदरादित्वात् \textcolor{red}{भर्} इत्यस्य \textcolor{red}{भ्रा} इत्यादेशे \textcolor{red}{भ्रातृ} इति भावः। न च कथं कर्मणि तृन्तृजौ \textcolor{red}{कर्तरि कृत्‌} (पा॰सू॰~३.४.६७) इत्यनेन कृतां कर्तर्येव विधानात्। \textcolor{red}{ताभ्यामन्यत्रोणादयः} (पा॰सू॰~३.४.७५) इत्यनेन कर्मण्यपीति दिक्। यद्वा \textcolor{red}{उणादयो बहुलम्} (धा॰पा॰~८९८) इत्यत्र \textcolor{red}{बहुल}\-ग्रहणादुणादयः क्वचित्कर्मण्यपि।}
तथा च रामः खलु निर्गुणो महा\-विष्णू राम एव पर\-ब्रह्मेत्युत्तर\-तापनीय\-श्रुतेः।\footnote{\textcolor{red}{ॐ यो वै श्रीरामचन्द्रः स भगवान् य ॐ नमो भगवते वासुदेवाय महाविष्णुर्भूर्भुवःस्वस्तस्मै वै नमो नमः} (रा॰उ॰ता॰उ॰~५.४४)।} एवं तस्य महा\-विष्णोर्भगवतः परात्परब्रह्मणः श्रीरामचन्द्रस्य त्रयोंऽशा ब्रह्म\-विष्णु\-महेशाख्याः।
ब्रह्मा खलु शत्रुघ्नो विष्णुर्भरतो लक्ष्मणः शिवः। कर्पूर\-गौरत्वाच्छिवस्य लक्ष्मणो गौरो विष्णोश्च श्यामतया भरतः श्यामो विष्णुत्वेन जनकत्वाद्ब्रह्मावतारः शत्रुघ्नो भरतं जनकमिवान्वञ्चति। अतस्तेषु पोष्यमाणेषु भ्रातृ\-रूपेषु त्रिष्वंशेषु लक्ष्मणः कनीयान्। यद्वा~–\end{sloppypar}
\centering\textcolor{red}{स्थूलं चाष्टभुजं प्रोक्तं सूक्ष्मं चैव चतुर्भुजम्।\nopagebreak\\
परं च द्विभुजं रूपं तस्मादेतत्त्रयं यजेत्॥}\nopagebreak\\
\raggedleft{–~आ॰सं॰}\footnote{मूलं मृग्यम्। शिवसहाय\-कृताया रामायण\-शिरोमण्याख्यायाष्टीकाया मङ्गलाचरणे एष श्लोक उद्धृतः। तत्र \textcolor{red}{इत्यानन्द\-संहिता\-वचनं च} इत्युक्तम्।}\\
\begin{sloppypar}\hyphenrules{nohyphenation}\justifying\noindent
द्वाभ्यां भुजाभ्यां भक्तस्य योगं क्षेमं च वहत्यथवा द्वाभ्यां भुजाभ्यां ज्ञान\-प्रधानांश्च भागवतान् भुनक्ति। त्रिपुर\-सुन्दरी\-तन्त्रे तस्य भगवतो महाविष्णोः श्रीरामभद्रस्येमे त्रयो विष्णव एवांशाः। तत्र क्षीरशायी भरतो वैकुण्ठाधीशो लक्ष्मणः।\footnote{मूलं वैष्णवागम\-ग्रन्थेषु मृग्यम्।} वैकुण्ठाधीशस्य शुक्ल\-वर्णता पुराणे प्रसिद्धा यथा~–\end{sloppypar}
\centering\textcolor{red}{केनोपयान चैतेषां दुःखनाशो भवेद्ध्रुवम्।\nopagebreak\\
इति सञ्चिन्त्य मनसा विष्णुलोकं गतस्तदा॥\nopagebreak\\
तत्र नारायणं देवं शुक्लवर्णं चतुर्भुजम्।\nopagebreak\\
शङ्खचक्रगदापद्मवनमालाविभूषितम्॥}\nopagebreak\\
\raggedleft{–~स्क॰पु॰~रे॰ख॰~२३३.५,६}\\
\begin{sloppypar}\hyphenrules{nohyphenation}\justifying\noindent एवमेव लक्ष्मणस्याऽपि गौराङ्गता स्पष्टा। क्षीराब्धि\-स्वामी श्यामलो यथा~–\end{sloppypar}
\centering\textcolor{red}{शान्ताकारं भुजगशयनं पद्मनाभं सुरेशं\nopagebreak\\
विश्वाधारं गगनसदृशं मेघवर्णं शुभाङ्गम्।\nopagebreak\\
लक्ष्मीकान्तं कमलनयनं योगिभिर्ध्यानगम्यं\nopagebreak\\
वन्दे विष्णुं भवभयहरं सर्वलोकैकनाथम्॥}\nopagebreak\\
\begin{sloppypar}\hyphenrules{nohyphenation}\justifying\noindent तथा क्षीराब्धि\-स्वामी भरतो वैकुण्ठ\-विहारी विष्णुर्लक्ष्मणो भूमा शत्रुघ्नो रामः सनातनं ब्रह्म। तथा चोक्तं बृहद्ब्रह्मसंहितायां यत्~–\end{sloppypar}
\centering\textcolor{red}{क्षीराब्धीशस्तु भरतो वैकुण्ठेशस्तु लक्ष्मणः।\nopagebreak\\
भूमा तु शत्रुघ्नो ज्ञेयो रामो ब्रह्म सनातनम्॥}\nopagebreak\\
\raggedleft{–~बृ॰ब्र॰सं॰}\\
\begin{sloppypar}\hyphenrules{nohyphenation}\justifying\noindent अतो मे त्रयो भ्रातरोंऽशाः। तत्र क्षीर\-शायि\-भरतापेक्षया वैकुण्ठेशो लक्ष्मणः कनीयानित्येव हार्दं हरेः।\end{sloppypar}
\section[क्रन्दमाना]{क्रन्दमाना}
\centering\textcolor{blue}{क्रन्दमाना पपाताग्रे खरस्य परुषाक्षरा।\nopagebreak\\
किमेतदिति तामाह खरः खरतराक्षरः॥}\nopagebreak\\
\raggedleft{–~अ॰रा॰~३.५.२१}\\
\begin{sloppypar}\hyphenrules{nohyphenation}\justifying\noindent\hspace{10mm} अत्र शूर्पणखा\-परिस्थितिं वर्णयति। \textcolor{red}{क्रन्दमाना} इति। आह्वाने रोदने च \textcolor{red}{क्रन्द्‌}\-धातुः (\textcolor{red}{क्रदिँ आह्वाने रोदने च} धा॰पा॰~७१) परस्मैपदी। ततश्चात्र शत्रा भवितव्यम्। एवं \textcolor{red}{क्रन्दतीति क्रन्दन्ती} इत्येव पाणिनीयम्।\footnote{यथा भागवते~– \textcolor{red}{गौर्भूत्वाऽश्रुमुखी खिन्ना क्रन्दन्ती करुणं विभोः} (भा॰पु॰~१०.१.१८)। अत्र \textcolor{red}{रुदन्ती} इत्यपि पाठः।} अत्र तु \textcolor{red}{क्रन्दमाना} इति।\footnote{एवमेव भारते शल्यपर्वणि~– \textcolor{red}{शोचन्त्यस्तत्र रुरुदुः क्रन्दमाना विशाम्पते} (म॰भा॰~९.२९.७०)।} 
\textcolor{red}{क्रन्दते} इति कर्म\-व्यतिहारादात्मनेपदम्।\footnote{\textcolor{red}{कर्तरि कर्म\-व्यतिहारे} (पा॰सू॰~१.३.१४) इत्यनेन।} \textcolor{red}{क्रन्दत इति क्रन्दमाना} इति शानचि शपि मुकि टापि च कृते सिद्धम्।\footnote{\textcolor{red}{क्रदिँ आह्वाने रोदने च} (धा॰पा॰~७१)~\arrow क्रद्~\arrow \textcolor{red}{इदितो नुम् धातोः} (पा॰सू॰~७.१.५८)~\arrow \textcolor{red}{मिदचोऽन्त्यात्परः} (पा॰सू॰~१.१.४७)~\arrow क्र~नुँम्~द्~\arrow क्र~न्~द्~\arrow \textcolor{red}{नश्चापदान्तस्य झलि} (पा॰सू॰~८.३.२४)~\arrow क्रंद्~\arrow \textcolor{red}{अनुस्वारस्य ययि परसवर्णः} (पा॰सू॰~८.४.५८)~\arrow क्रन्द्~\arrow \textcolor{red}{कर्तरि कर्म\-व्यतिहारे} (पा॰सू॰~१.३.१४)~\arrow \textcolor{red}{वर्तमाने लट्} (पा॰सू॰~३.२.१२३)~\arrow क्रन्द्~लट्~\arrow \textcolor{red}{लटः शतृशानचावप्रथमा\-समानाधिकरणे} (पा॰सू॰~३.२.१२४)~\arrow क्रन्द्~शानच्~\arrow क्रन्द्~आन~\arrow \textcolor{red}{कर्तरि शप्‌} (पा॰सू॰~३.१.६८)~\arrow क्रन्द्~शप्~आन~\arrow क्रन्द्~अ~आन~\arrow \textcolor{red}{आने मुक्} (पा॰सू॰~७.२.८२)~\arrow \textcolor{red}{आद्यन्तौ टकितौ} (पा॰सू॰~१.१.४६)~\arrow क्रन्द्~अ~मुँक्~आन~\arrow क्रन्द्~अ~म्~आन~\arrow क्रन्दमान~\arrow \textcolor{red}{अजाद्यतष्टाप्‌} (पा॰सू॰~४.१.४)~\arrow क्रन्दमान~टाप्~\arrow क्रन्दमान~आ~\arrow \textcolor{red}{अकः सवर्णे दीर्घः} (पा॰सू॰~६.१.१०१)~\arrow क्रन्दमाना।} यद्वाऽत्र वैक्लव्ये \textcolor{red}{क्रन्द्‌}\-धातुः (\textcolor{red}{क्रदिँ वैकल्ये} धा॰पा॰~७७३) आत्मनेपदी। ततः शानचि टापि \textcolor{red}{क्रन्दमाना}। यद्वेममाकृति\-गणत्वात्स्वरितेतं पठित्वा \textcolor{red}{स्वरितञितः कर्त्रभिप्राये क्रिया\-फले} (पा॰सू॰~१.३.७२) इत्यनेनाऽत्मनेपदत्वाच्छानचि शपि मुकि टापि \textcolor{red}{क्रन्दमाना}।\footnote{\textcolor{red}{बहुलमेतन्निदर्शनम्} (धा॰पा॰ ग॰सू॰~१९३८) \textcolor{red}{आकृतिगणोऽयम्} (धा॰पा॰ ग॰सू॰~१९९२) \textcolor{red}{भूवादिष्वेतदन्तेषु दशगणीषु धातूनां पाठो निदर्शनाय तेन स्तम्भुप्रभृतयः सौत्राश्चुलुम्पादयो वाक्यकारीयाः प्रयोगसिद्धा विक्लवत्यादयश्च} (मा॰धा॰वृ॰~१०.३२८) इत्यनुसारमाकृति\-गणत्वाद्भ्वादि\-गण ऊह्योऽयं स्वरितेद्धातुः। तेन \textcolor{red}{क्रन्दति} \textcolor{red}{क्रन्दते} इति सिध्यतः। \textcolor{red}{क्रन्दति} यथा वामनपुराणे~– \textcolor{red}{सिंहाभिपन्नो विपिने यथैव मत्तो गजः क्रन्दति वेदनार्तः} (वाम॰पु॰~१०.४७) गरुडपुराणे च~– \textcolor{red}{गच्छन्वनानि रौद्राणि दृष्ट्वा क्रन्दति तत्र सः} (ग॰पु॰~२.१६.१३)। \textcolor{red}{क्रन्दते} यथा ब्रह्मपुराणे~– \textcolor{red}{यदैव क्रन्दते जन्तुर्दुःखार्तः पतितः क्वचित्} (ब्र॰पु॰~२१४.१००) गरुडपुराणे च~– \textcolor{red}{हाहेति क्रन्दते नित्यं कीदृशं तु मया कृतम्} (ग॰पु॰~२.१५.८५)। एतेन पूर्वोक्त\-भारत\-प्रयोगोऽपि व्याख्यातः।}\end{sloppypar}
\section[घोर\-रूपिणः]{घोर\-रूपिणः}
\label{sec:ghorarupinah}
\centering\textcolor{blue}{सीतां नीत्वा गुहां गत्वा तत्र तिष्ठ महाबल।\nopagebreak\\
हन्तुमिच्छाम्यहं सर्वान् राक्षसान् घोररूपिणः॥}\nopagebreak\\
\raggedleft{–~अ॰रा॰~३.५.३०}\\
\begin{sloppypar}\hyphenrules{nohyphenation}\justifying\noindent\hspace{10mm} खर\-दूषणौ दृष्ट्वा निर्वासित\-खर\-दूषणो निरस्त\-दूषणो रघु\-कुल\-भूषणः श्रीरामो लक्ष्मणं सतर्कयति यत्त्वं \textcolor{red}{सीतां नीत्वा गुहां प्रविश तावदहमद्यैव घोर\-रूपिणो राक्षसान् हन्तुमिच्छामि}। अत्र \textcolor{red}{घोर\-रूपिणः} इति प्रयोगः कथं \textcolor{red}{न कर्मधारयान्मत्वर्थीयो बहुव्रीहिश्चेत्तदर्थ\-प्रतिपत्तिकरः} इति नियमस्य जागरूकत्वे \textcolor{red}{घोरं च तद्रूपं चेति घोर\-रूपं तदस्ति येषां ते घोर\-रूपिणस्तान् घोर\-रूपिणः} इति चेत्। बहुव्रीहौ सतीप्सितार्थस्यानव\-गतावेष पन्था अनुगतः। \textcolor{red}{घोरं रूपं येषां ते} इत्यर्थे सति न किमप्यर्थ\-वैलक्षण्यम्। मत्वर्थीय इनिर्निन्दायाम्।\footnote{\textcolor{red}{अत इनिठनौ} (पा॰सू॰~५.२.११५) इत्यनेन। \textcolor{red}{भूम\-निन्दा\-प्रशंसासु नित्ययोगेऽति\-शायने। सम्बन्धेऽस्ति\-विवक्षायां भवन्ति मतुबादयः॥} (भा॰पा॰सू॰~५.२.९४)। भाष्यकार\-मते निन्दायामिनिरेव। यथा \textcolor{red}{निन्दायाम्। ककुदावर्ती। सङ्खादकी} (भा॰पा॰सू॰~५.२.९४)।} \textcolor{red}{इन्‌}\-प्रत्यय\-विधानात् \textcolor{red}{निन्दित\-घोर\-रूप\-युक्ताः} इत्यर्थो ध्वन्यते। अथवा \textcolor{red}{घोरं यथा स्यात्तथा रूपयितुं तच्छीलाः} इति विग्रहे \textcolor{red}{सुप्यजातौ णिनिस्ताच्छील्ये} (पा॰सू॰~३.२.७८) इत्यनेन णिनि\-प्रत्यये समाधानम्।\end{sloppypar}
\section[शापितः]{शापितः}
\centering\textcolor{blue}{अत्र किञ्चिन्न वक्तव्यं शापितोऽसि ममोपरि।\nopagebreak\\
तथेति सीतामादाय लक्ष्मणो गह्वरं ययौ॥}\nopagebreak\\
\raggedleft{–~अ॰रा॰~३.५.३१}\\
\begin{sloppypar}\hyphenrules{nohyphenation}\justifying\noindent\hspace{10mm} \textcolor{red}{शप्‌}\-धातोः (\textcolor{red}{शपँ आक्रोशे} धा॰पा॰~१०००, ११६८) कर्मणि \textcolor{red}{क्त}\-प्रत्ययेऽनिट्कत्वात्\footnote{\textcolor{red}{एकाच उपदेशेऽनुदात्तात्‌} (पा॰सू॰~७.२.१०) इत्यनेनानिट्कत्वम्।} \textcolor{red}{शप्तः} इत्येव।\footnote{यथाऽस्मिन्नेव काण्डे \textcolor{red}{दुर्वाससाऽकारणकोपमूर्तिना शप्तः पुरा सोऽद्य विमोचितस्त्वया} (अ॰रा॰~३.१३८)।} \textcolor{red}{आगम\-शास्त्रमनित्यम्} (प॰शे॰~९३.२) इति कृत्वेडागमेऽपि \textcolor{red}{शपितः}।\footnote{यथा भारते दाक्षिणात्य\-पाठे शकुन्तलां प्रति कण्वः~– \textcolor{red}{प्रतिवाक्यं न दद्यास्त्वं शपिता मम पादयोः} (म॰भा॰~१.९६.९)। एवमेव नान्दी\-पुराणे~– \textcolor{red}{मत्कृते येऽत्र शपिताः सावित्र्या ब्राह्मणाः सुराः} (नान्दी॰पु॰~२७.७)।} \textcolor{red}{शापितः} इति कथम्। अत्र स्वार्थे णिचि प्रत्यये ततः क्तान्तेन समाधानम्।\footnote{\textcolor{red}{शपँ आक्रोशे} (धा॰पा॰~१०००, ११६८)~\arrow शप्~\arrow स्वार्थे णिच्~\arrow शप्~णिच्~\arrow शप्~इ~\arrow \textcolor{red}{अत उपधायाः} (पा॰सू॰~७.२.११६)~\arrow शाप्~इ~\arrow शापि~\arrow \textcolor{red}{सनाद्यन्ता धातवः} (पा॰सू॰~३.१.३२)~\arrow धातु\-सञ्ज्ञा~\arrow \textcolor{red}{तयोरेव कृत्य\-क्तखलर्थाः} (पा॰सू॰~३.४.७०)~\arrow शापि~क्त~\arrow शापि~त~\arrow \textcolor{red}{आर्धधातुकस्येड्वलादेः} (पा॰सू॰~७.२.३५)~\arrow \textcolor{red}{आद्यन्तौ टकितौ} (पा॰सू॰~१.१.४६)~\arrow शापि~इट्~त~\arrow \textcolor{red}{णेरनिटि} (पा॰सू॰~६.४.५१)~\arrow शापि~त~\arrow शापित~\arrow विभक्तिकार्यम्~\arrow शापितः।}\end{sloppypar}
\section[समाधिविरमे]{समाधिविरमे}
\centering\textcolor{blue}{ध्यायन् हृदि परात्मानं निर्गुणं गुणभासकम्।\nopagebreak\\
समाधिविरमेऽपश्यद्रावणं गृहमागतम्॥}\nopagebreak\\
\raggedleft{–~अ॰रा॰~३.६.३}\\
\begin{sloppypar}\hyphenrules{nohyphenation}\justifying\noindent\hspace{10mm} सङ्ग्रामे श्रीरामेण स्वधामनीतेषु खर\-दूषण\-त्रिशीर्षेषु प्रतिशोधं चिकीर्षन् रावणः श्रीरामं ध्यायन्तं मारीचं प्रत्यगात्। तत्र समाधि\-विरामे मारीचो रावणमपश्यत्। तत्र \textcolor{red}{समाधि\-विरमे} इति प्रयोगस्तु विचारकोटावागच्छति। न च \textcolor{red}{समाधेर्विरामः समाधि\-विरामस्तस्मिन् समाधि\-विरामे}। \textcolor{red}{विरमणं विरामः}। \textcolor{red}{भावे} (पा॰सू॰~३.३.१८) इति घञ्। \textcolor{red}{विरामोऽवसानम्} (पा॰सू॰~१.४.११०) इति सूत्रं प्रमाणम्। यद्वा \textcolor{red}{विरमन्तेऽस्मिन्निति विरामः}। \textcolor{red}{हलश्च} (पा॰सू॰~३.३.१२१) इति घञ्।\footnote{स चाधिकरणे। \textcolor{red}{हलश्च} (पा॰सू॰~३.३.१२१) इत्यत्र \textcolor{red}{करणाधिकरणयोश्च} (पा॰सू॰~३.३.११७) इत्यस्यानुवृत्तेः।} \textcolor{red}{अच्‌}\-प्रत्ययः कर्तरि। करणाधि\-करणयोः सञ्ज्ञायाञ्चान्य\-प्रत्ययानां \textcolor{red}{हलश्च} (पा॰सू॰~३.३.१२१) इत्यनेन बाधः। अत्रोच्यते। \textcolor{red}{विगता रमा यस्मात्तद्विरमम्}। समाधौ हि भक्ति\-रूपिणी रमा। ब्रह्मानुभवत्वात्। तदभावे रमाया निर्गमः स्वभाविक एवातः। यद्वा \textcolor{red}{रमाऽस्त्यस्मिन्निति रमम्}। अर्शआदित्वादच्।\footnote{\textcolor{red}{अर्शआदिभ्योऽच्} (पा॰सू॰~५.२.१२७) इत्यनेन।} \textcolor{red}{विगतं रममिति विरमम्}।\footnote{\textcolor{red}{कु\-गति\-प्रादयः} (पा॰सू॰~२.२.१८) इत्यनेन प्रादि\-समासः।} \textcolor{red}{समाधेर्विरमं समाधि\-विरमं तस्मिन् समाधि\-विरमे}।\footnote{यद्वा \textcolor{red}{विरमणं विरमः} इति विग्रहे \textcolor{red}{वि}\-पूर्वकात् \textcolor{red}{रम्‌}\-धातोः (\textcolor{red}{रमुँ क्रीडायाम्} धा॰पा॰~८५३) \textcolor{red}{अज्विधौ भयादीनामुपसंख्यानं नपुंसके क्तादिनिवृत्त्यर्थम्} (वा॰~३.३.५६) इत्यनेन \textcolor{red}{अच्‌}\-प्रत्यये विरमम्। अबन्तो \textcolor{red}{विरमः} (वि~रम्~अप्) शब्द इत्याप्टे\-कोशः। तस्मिन् विरमे। यथा \textcolor{red}{सोऽहं नृणां क्षुल्लसुखाय दुःखं महद्गतानां विरमाय तस्य} (भा॰पु॰~३.८.२) इति भागवते \textcolor{red}{विरमाय} इत्यत्र। अत्र \textcolor{red}{विरमाय नाशाय} (भा॰पु॰ वी॰रा॰व्या॰~३.८.२) \textcolor{red}{विरमाय निवृत्तये} (भा॰पु॰ सि॰प्र॰~३.८.२, भा॰पु॰ बा॰प्र॰~३.८.२) इति टीकाकाराः। एवमेव \textcolor{red}{अथ धूमस्य विरमे द्वितीयं रूपदर्शनम्} (म॰भा॰~१२.२४२.१८) इति भारत\-प्रयोगेऽपि।}\end{sloppypar}
\section[सहायं मे]{सहायं मे}
\label{sec:sahayam_me}
\centering\textcolor{blue}{त्वं तु तावत्सहायं मे कृत्वा स्थास्यसि पूर्ववत्।\nopagebreak\\
इत्येवं भाषमाणं तं रावणं वीक्ष्य विस्मितः॥}\nopagebreak\\
\raggedleft{–~अ॰रा॰~३.६.१४}\\
\begin{sloppypar}\hyphenrules{nohyphenation}\justifying\noindent\hspace{10mm} अत्र भावे \textcolor{red}{ष्यञ्} प्रत्यये \textcolor{red}{साहाय्यम्} इति तु पाणिनीयमेव।\footnote{\textcolor{red}{सह अयते एति वेति सहायः।} \textcolor{red}{अयँ गतौ} (धा॰पा॰~४७४) इत्यस्मात् \textcolor{red}{इण् गतौ} (धा॰पा॰~१०४५) इत्यस्माद्वा \textcolor{red}{नन्दि\-ग्रहि\-पचादिभ्यो ल्युणिन्यचः} (पा॰सू॰~३.१.१३४) इत्यनेन कर्तरि पचाद्यच्। ततः \textcolor{red}{सह सुपा} (पा॰सू॰~२.१.४) इत्यनेन सुप्सुपा\-समासः। \textcolor{red}{अनुप्लवः सहायश्चानुचरोऽभिसरः समाः} (अ॰को॰~२.८.७१) इत्यमरः। \textcolor{red}{सहाय}\-शब्दात् \textcolor{red}{गुण\-वचन\-ब्राह्मणादिभ्यः कर्मणि च} (पा॰सू॰~५.१.१२४) इत्यनेन भावे कर्मणि वा \textcolor{red}{ष्यञ्‌}\-प्रत्ययेऽनुबन्ध\-लोपे \textcolor{red}{तद्धितेष्वचामादेः} (पा॰सू॰~७.२.११७) इत्यनेनादि\-वृद्धौ \textcolor{red}{यचि भम्} (पा॰सू॰~१.४.१८) इत्यनेन भसञ्ज्ञायाम् \textcolor{red}{यस्येति च} (पा॰सू॰~६.४.१४८) इत्यनेनाकार\-लोपे विभक्तिकार्ये \textcolor{red}{साहाय्यम्} इति रूपम्। पक्षे \textcolor{red}{सहायाद्वेति वक्तव्यम्} (वा॰~५.१.१३१) इत्यनेन भावे कर्मणि वा \textcolor{red}{वुञ्‌}\-प्रत्ययेऽनुबन्ध\-लोपे \textcolor{red}{तद्धितेष्वचामादेः} (पा॰सू॰~७.२.११७) इत्यनेनादि\-वृद्धौ \textcolor{red}{युवोरनाकौ} (पा॰सू॰~७.१.१) इत्यनेनाकादेशे विभक्तिकार्ये \textcolor{red}{साहायकम्} इति रूपम्।} किन्तु \textcolor{red}{अयनम् अयः} इति भावे \textcolor{red}{एरच्} (पा॰सू॰~३.३.५६) इत्यनेन \textcolor{red}{इ}\-धातोः (\textcolor{red}{इण् गतौ} धा॰पा॰~१०४५) अचि गुणेऽयादेशे विभक्ति\-कार्ये \textcolor{red}{अयः}। \textcolor{red}{सह अयः सहायस्तं सहायम्} इति पूर्वोक्त\-दिशा साधनीयम्।\footnote{\pageref{sec:sahayam}तमे पृष्ठे \ref{sec:sahayam} \nameref{sec:sahayam} इति प्रयोगस्य विमर्शं पश्यन्तु। \textcolor{red}{सह अयः सहायः} इत्यत्र \textcolor{red}{सह सुपा} (पा॰सू॰~२.१.४) इत्यनेन सुप्सुपा\-समासः।}\end{sloppypar}
\section[स्थाप्य]{स्थाप्य}
\centering\textcolor{blue}{श्रुत्वा रामोदितं वाक्यं सापि तत्र तथाकरोत्।\nopagebreak\\
मायासीतां बहिः स्थाप्य स्वयमन्तर्दधेऽनले॥}\nopagebreak\\
\raggedleft{–~अ॰रा॰~३.७.४}\\
\begin{sloppypar}\hyphenrules{nohyphenation}\justifying\noindent\hspace{10mm} अत्र गति\-निवृत्त्यर्थक\-णिजन्त\-\textcolor{red}{स्था}\-धातोः (\textcolor{red}{ष्ठा गतिनिवृत्तौ} धा॰पा॰~९२८) पुकि\footnote{\textcolor{red}{अर्ति\-ह्रीव्ली\-रीक्नूयी\-क्ष्माय्यातां पुङ्णौ} (पा॰सू॰~७.३.३६) इत्यनेन।} ल्यबन्तमेवम्।\footnote{ल्यपि च \textcolor{red}{णेरनिटि} (पा॰सू॰~६.४.५१) इत्यनेन णिलोपः।} किन्त्विदं समासं विना कथं सम्भवमिति चेत्।
अत्र साक्षात्प्रभृति\-गणे \textcolor{red}{बहिस्} इत्यपि पठित्वा \textcolor{red}{साक्षात्प्रभृतीनि च} (पा॰सू॰~१.४.७४) इत्यनेन गतिसञ्ज्ञायां समासः।\footnote{\textcolor{red}{साक्षात्प्रभृतीनि च} (पा॰सू॰~१.४.७४) इत्यत्र \textcolor{red}{क्वचिदेक\-देशोऽप्यनुवर्तते} (प॰शे॰~१८) इति परिभाषया \textcolor{red}{कृञि} (पा॰सू॰~१.४.७२) इत्यस्य मण्डूक\-प्लुत्या निवृत्तौ \textcolor{red}{विभाषा} (पा॰सू॰~१.४.७२) इत्यस्यानुवृत्तौ कृञभावेऽपि साक्षात्प्रभृतीनां गति\-सञ्ज्ञा क्वाचित्का। यद्वा चकार\-ग्रहणात्कृञभावेऽपि गति\-सञ्ज्ञा क्वाचित्का। सत्यां गतिसञ्ज्ञायां \textcolor{red}{कुगतिप्रादयः} (पा॰सू॰~२.२.१८) इत्यनेन समास इति भावः।} ततो \textcolor{red}{समासेऽनञ्पूर्वे क्त्वो ल्यप्} (पा॰सू॰~७.१.३७) इत्यनेन क्त्वा\-स्थाने ल्यबादेशे \textcolor{red}{बहिःस्थाप्य} इति।\end{sloppypar}
\section[कानकम्]{कानकम्}
\centering\textcolor{blue}{पश्य राम मृगं चित्रं कानकं रत्नभूषितम्।\nopagebreak\\
विचित्रबिन्दुभिर्युक्तं चरन्तमकुतोभयम्।\nopagebreak\\
बद्ध्वा देहि मम क्रीडामृगो भवतु सुन्दरः॥}\nopagebreak\\
\raggedleft{–~अ॰रा॰~३.७.६}\\
\begin{sloppypar}\hyphenrules{nohyphenation}\justifying\noindent\hspace{10mm} कनक\-मृगं दृष्ट्वा माया\-सीता माया\-मनुष्यं रामं माया\-मृगं हन्तुं चोदयति यत् \textcolor{red}{कानकं मृगं पश्य}। अत्र विकारार्थे मयटि कृते\footnote{\textcolor{red}{मयड्वैतयोर्भाषायामभक्ष्याच्छादनयोः} (पा॰सू॰~४.३.१४३) इत्यनेन।} \textcolor{red}{कनक\-मयम्}। परन्तु माया\-मृगत्वात्कनक\-विकाराभावे \textcolor{red}{कनकस्यायं कानकस्तं कानकम्}। \textcolor{red}{तस्येदम्} (पा॰सू॰~४.३.१२०) इत्यनेन \textcolor{red}{अण्}। वृद्धौ भत्वादलोपे विभक्तिकार्ये \textcolor{red}{कानकम्}।\footnote{कनक~\arrow \textcolor{red}{तस्येदम्} (पा॰सू॰~४.३.१२०)~\arrow कनक~अण्~\arrow कनक~अ~\arrow \textcolor{red}{तद्धितेष्वचामादेः} (पा॰सू॰~७.२.११७)~\arrow कानक~अ~\arrow \textcolor{red}{यचि भम्} (पा॰सू॰~१.४.१८)~\arrow \textcolor{red}{यस्येति च} (पा॰सू॰~६.४.१४८)~\arrow कानक्~अ~\arrow कानक~\arrow \textcolor{red}{कृत्तद्धित\-समासाश्च} (पा॰सू॰~१.२.४६)~\arrow प्रातिपदिक\-सञ्ज्ञा~\arrow विभक्तिकार्यम्~\arrow कानक~सुँ~\arrow \textcolor{red}{अतोऽम्} (पा॰सू॰~७.१.२४)~\arrow कानक~अम्~\arrow \textcolor{red}{अमि पूर्वः} (पा॰सू॰~६.१.१०७)~\arrow कानकम्।} यद्वा \textcolor{red}{कनकानां समूहः कानकम्}।\footnote{\textcolor{red}{भिक्षादिभ्योऽण्} (पा॰सू॰~४.२.३८) इत्यनेन \textcolor{red}{अण्}। प्रक्रिया पूर्ववत्।} \textcolor{red}{तदस्त्यस्मिन्निति कानकस्तं कानकम्} इति समूहार्थाणन्तान्मत्वर्थीयेऽचि\footnote{\textcolor{red}{अर्शआदिभ्योऽच्} (पा॰सू॰~५.२.१२७) इत्यनेन \textcolor{red}{अच्}। भत्वादलोपः पूर्ववत्।} विभक्ति\-कार्ये।
न च तद्धित\-प्रत्ययान्तादत्र कथं तद्धित\-प्रत्ययः। तथा च कारिका~–\end{sloppypar}
\centering\textcolor{red}{शैषिकान्मतुबर्थीयाच्छैषिको मतुबर्थकः।\nopagebreak\\
सरूपः प्रत्ययो नेष्टः सन्नन्तान्न सनिष्यते॥}\nopagebreak\\
\raggedleft{–~भा॰पा॰सू॰~३.१.७}\\
\begin{sloppypar}\hyphenrules{nohyphenation}\justifying\noindent इतिवचनाच्छैषिकाच्छैषिकः सरूप\-प्रत्ययो न मत्वर्थीयान्मत्वर्थीयो न। सरूपत्वं नाम समान\-देशत्वे सति समानार्थ\-बोधकत्वम्। अतो न दोषः।\footnote{यतो ह्यत्र चातुरर्थिक\-तद्धित\-प्रत्ययान्मत्वर्थीय\-तद्धित\-प्रत्ययः। तौ न सरूपौ। भाष्येऽपि \textcolor{red}{शाकल्यस्य च्छात्राः} इत्यर्थे \textcolor{red}{शाकलाः} (भा॰पा॰सू॰~४.१.१८) इत्युदाहरणेऽपत्यार्थक\-तद्धित\-प्रत्ययाच्छैषिक\-तद्धित\-प्रत्ययः।} यद्वा \textcolor{red}{कनकमेव कानकस्तं कानकम्} इति स्वार्थेऽण्।\footnote{\textcolor{red}{प्रज्ञादिभ्यश्च} (पा॰सू॰~५.४.३८) इत्यनेन। प्रक्रिया पूर्ववत्।}  यद्वा \textcolor{red}{कनके भवः कानकस्तं कानकम्}।\footnote{\textcolor{red}{तत्र भवः} (पा॰सू॰~४.३.५३) इत्यनेन \textcolor{red}{अण्‌}। प्रक्रिया पूर्ववत्।} यद्वा \textcolor{red}{कनके जातः कानकस्तं कानकम्}।\footnote{\textcolor{red}{तत्र जातः} (पा॰सू॰~४.३.२५) इत्यनेन \textcolor{red}{अण्‌}। प्रक्रिया पूर्ववत्।} \textcolor{red}{कानकं मृगं बद्ध्वाऽऽनय} इति भगवत्यादिशति।\end{sloppypar}
\section[वध्यमाना]{वध्यमाना}
\centering\textcolor{blue}{इत्युक्त्वा वध्यमाना सा स्वबाहुभ्यां रुरोद ह।\nopagebreak\\
तच्छ्रुत्वा लक्ष्मणः कर्णौ पिधायातीव दुःखितः॥}\nopagebreak\\
\raggedleft{–~अ॰रा॰~३.७.३५}\\
\begin{sloppypar}\hyphenrules{nohyphenation}\justifying\noindent\hspace{10mm} माया\-मृगस्य श्रीरामानुकारि\-स्वरं श्रुत्वा लक्ष्मणं गमयितुमिच्छन्ती यातुमनिच्छन्तं तं भर्त्सयित्वा बाहुभ्यां हृदयं ताडयन्ती रुरोद माया\-सीता। तत्र \textcolor{red}{वध्यमाना} इति प्रयुक्तम्। \textcolor{red}{वध} आदेशो हि \textcolor{red}{हन्‌}\-धातोः (\textcolor{red}{हनँ हिंसागत्योः} धा॰पा॰~१०१२) लिङि लुङि चार्ध\-धातुक\-प्रत्यये भवति। तथा च सूत्र\-द्वयम्~– \textcolor{red}{हनो वध लिङि} (पा॰सू॰~२.४.४२) \textcolor{red}{लुङि च} (पा॰सू॰~२.४.४३)। अतः \textcolor{red}{वध्यमाना} अपाणिनीयमेव \textcolor{red}{बाहुभ्यां हृदये हन्यते} इति विग्रहे \textcolor{red}{हन्यमाना} इत्येव पाणिनीयमिति चेत्। हिंसार्थे \textcolor{red}{वध्‌}\-धातुरपि।\footnote{स च \textcolor{red}{जनिवध्योश्च} (पा॰सू॰~७.३.३५) इति सूत्रेण ज्ञापितः। \textcolor{red}{‘जनिवध्योश्च’। जनकः। ‘वधँ हिंसायाम्’। वधकः} (वै॰सि॰कौ॰~२८९५)। \textcolor{red}{‘वधँ हिंसायामिति’। धात्वन्तरं भौवादिकम्। भ्वादेराकृतिगणत्वात्} (बा॰म॰~२८९५)।} तस्य \textcolor{red}{वध्यते} इति कर्म\-वाच्ये \textcolor{red}{सार्वधातुके यक्} (पा॰सू॰~३.१.६७) इत्यनेन यकि ततश्च शानच्प्रत्यये मुगागमे टापि \textcolor{red}{वध्यमाना}।\footnote{वध्~\arrow \textcolor{red}{भावकर्मणोः} (पा॰सू॰~१.३.१३)~\arrow \textcolor{red}{वर्तमाने लट्} (पा॰सू॰~३.२.१२३)~\arrow \textcolor{red}{लटः शतृशानचावप्रथमा\-समानाधिकरणे} (पा॰सू॰~३.२.१२४)~\arrow वध्~शानच्~\arrow वध्~आन~\arrow \textcolor{red}{सार्वधातुके यक्} (पा॰सू॰~३.१.६७)~\arrow \textcolor{red}{आद्यन्तौ टकितौ} (पा॰सू॰~१.१.४६)~\arrow वध्~यक्~आन~\arrow वध्~य~आन~\arrow \textcolor{red}{आने मुक्} (पा॰सू॰~७.२.८२)~\arrow \textcolor{red}{आद्यन्तौ टकितौ} (पा॰सू॰~१.१.४६)~\arrow वध्~य~मुँक्~आन~\arrow वध्~य~म्~आन~\arrow वध्यमान~\arrow \textcolor{red}{अजाद्यतष्टाप्‌} (पा॰सू॰~४.१.४)~\arrow वध्यमान~टाप्~\arrow वध्यमान~आ~\arrow \textcolor{red}{अकः सवर्णे दीर्घः} (पा॰सू॰~६.१.१०१)~\arrow वध्यमाना~\arrow विभक्तिकार्यम्~\arrow वध्यमाना।} यद्वा \textcolor{red}{हननं वधः}।\footnote{\textcolor{red}{हनँ हिंसागत्योः} (धा॰पा॰~१०१२)~\arrow हन्~\arrow \textcolor{red}{हनश्च वधः} (पा॰सू॰~३.३.७६)~\arrow वध्~अप्~\arrow वध्~अ~\arrow वध~\arrow विभक्तिकार्यम्~\arrow वधः।} \textcolor{red}{वधमाचरतीति वधति}।\footnote{वध~\arrow \textcolor{red}{सर्वप्राति\-पदिकेभ्य आचारे क्विब्वा वक्तव्यः} (वा॰~३.१.११)~\arrow वध~क्विँप्~\arrow वध~व्~\arrow \textcolor{red}{वेरपृक्तस्य} (पा॰सू॰~६.१.६७)~\arrow वध~\arrow \textcolor{red}{सनाद्यन्ता धातवः} (पा॰सू॰~३.१.३२)~\arrow धातुसञ्ज्ञा~\arrow \textcolor{red}{शेषात्कर्तरि परस्मैपदम्} (पा॰सू॰~१.३.७८)~\arrow \textcolor{red}{वर्तमाने लट्} (पा॰सू॰~३.२.१२३)~\arrow वध~लट्~\arrow वध~तिप्~\arrow वध~ति~\arrow \textcolor{red}{कर्तरि शप्‌} (पा॰सू॰~३.१.६८)~\arrow वध~शप्~ति~\arrow वध~अ~ति~\arrow \textcolor{red}{अतो गुणे} (पा॰सू॰~६.१.९७)~\arrow वध~ति~\arrow वधति।} \textcolor{red}{बाहू हृदये वधमाचरत इति वधतः}\footnote{वध~\arrow \textcolor{red}{धातुसञ्ज्ञा} (पूर्ववत्)~\arrow \textcolor{red}{शेषात्कर्तरि परस्मैपदम्} (पा॰सू॰~१.३.७८)~\arrow \textcolor{red}{वर्तमाने लट्} (पा॰सू॰~३.२.१२३)~\arrow वध~लट्~\arrow वध~तस्~\arrow \textcolor{red}{कर्तरि शप्‌} (पा॰सू॰~३.१.६८)~\arrow वध~शप्~तस्~\arrow वध~अ~तस्~\arrow \textcolor{red}{अतो गुणे} (पा॰सू॰~६.१.९७)~\arrow वध~तस्~\arrow \textcolor{red}{ससजुषो रुः} (पा॰सू॰~८.२.६६)~\arrow वधतरुँ~\arrow \textcolor{red}{खरवसानयोर्विसर्जनीयः} (पा॰सू॰~८.३.१५)~\arrow वधतः।} विग्रहेऽस्मिन् \textcolor{red}{वध}\-शब्दादाचार\-क्विबन्तात्कर्मणि लकारे \textcolor{red}{बाहुभ्यां हृदये वध्यते} इति विग्रहे \textcolor{red}{वध्यमाना}।\footnote{वध~\arrow धातु\-सञ्ज्ञा (पूर्ववत्)~\arrow \textcolor{red}{भावकर्मणोः} (पा॰सू॰~१.३.१३)~\arrow \textcolor{red}{वर्तमाने लट्} (पा॰सू॰~३.२.१२३)~\arrow \textcolor{red}{लटः शतृशानचावप्रथमा\-समानाधिकरणे} (पा॰सू॰~३.२.१२४)~\arrow वध~शानच्~\arrow वध~आन~\arrow \textcolor{red}{सार्वधातुके यक्} (पा॰सू॰~३.१.६७)~\arrow \textcolor{red}{आद्यन्तौ टकितौ} (पा॰सू॰~१.१.४६)~\arrow वध~यक्~आन~\arrow वध~य~आन~\arrow \textcolor{red}{अतो लोपः} (पा॰सू॰~६.४.४८)~\arrow वध्~य~आन~\arrow \textcolor{red}{आने मुक्} (पा॰सू॰~७.२.८२)~\arrow \textcolor{red}{आद्यन्तौ टकितौ} (पा॰सू॰~१.१.४६)~\arrow वध्~य~मुँक्~आन~\arrow वध्~य~म्~आन~\arrow वध्यमान~\arrow \textcolor{red}{अजाद्यतष्टाप्‌} (पा॰सू॰~४.१.४)~\arrow वध्यमान~टाप्~\arrow वध्यमान~आ~\arrow \textcolor{red}{अकः सवर्णे दीर्घः} (पा॰सू॰~६.१.१०१)~\arrow वध्यमाना~\arrow विभक्तिकार्यम्~\arrow वध्यमाना।} अत आचार\-क्विबन्त\-\textcolor{red}{वध}\-धातोः कर्म\-वाच्ये शानचि कृते सिद्धं रूपमिदम्।\end{sloppypar}
\section[क्रोशमानाम्]{क्रोशमानाम्}
\centering\textcolor{blue}{वाक्शरेण हतस्त्वं मे क्षन्तुमर्हसि देवर।\nopagebreak\\
इत्येवं क्रोशमानां तां रामागमनशङ्कया॥}\nopagebreak\\
\raggedleft{–~अ॰रा॰~३.७.६१}\\
\begin{sloppypar}\hyphenrules{nohyphenation}\justifying\noindent\hspace{10mm} \textcolor{red}{क्रुश्‌}\-धातुः (\textcolor{red}{क्रुशँ आह्वाने रोदने च} धा॰पा॰~८५६) परस्मैपदी। ततः \textcolor{red}{क्रोशति} इति विग्रहे शतृ\-प्रत्यये ङीपि नुमि \textcolor{red}{क्रोशन्ती} इति पाणिनीय\-प्रयोगः।\footnote{यथा \textcolor{red}{क्रोशन्ती राम रामेति लक्ष्मणेति च विस्वरम्} (वा॰रा॰~४.६.९) इति वाल्मीकि\-प्रयोगे। अस्मिन्नेव ग्रन्थे च~– \textcolor{red}{लङ्कां गत्वा सभामध्ये क्रोशन्ती पादसन्निधौ} (अ॰रा॰~३.५.३८) \textcolor{red}{क्रोशन्ती करुणं दीना जगाद दशकन्धरम्} (अ॰रा॰~६.१०.२९)।} \textcolor{red}{क्रोशमानाम्} इत्यपि तथा हि \textcolor{red}{कर्तरि कर्म\-व्यतिहारे} (पा॰सू॰~१.३.१४) 
इत्यत्राऽत्मनेपदे। यतो हि सत्यपि वेदवती जानन्त्यपि राम\-पराक्रमं प्रवृत्ति\-योग्यं क्रोशनं रावण\-वधार्थं स्वयं करोति तत्राऽत्मनेपदम्। \textcolor{red}{क्रोशत इति क्रोशमाना ताम्} इति विग्रह आत्मने\-पदीयत्वाच्छानचि प्रत्यये मुगागमे टापि विभक्ति\-कार्ये सुलोपे \textcolor{red}{क्रोशमानाम्} शब्दोऽपि पाणिनीयः।\end{sloppypar}
\section[विलप्यमाना]{विलप्यमाना}
\centering\textcolor{blue}{कृशातिदीना परिकर्मवर्जिता दुःखेन शुष्यद्वदनातिविह्वला।\nopagebreak\\
हा राम रामेति विलप्यमाना सीता स्थिता राक्षसवृन्दमध्ये॥}\nopagebreak\\
\raggedleft{–~अ॰रा॰~३.७.६६}\\
\begin{sloppypar}\hyphenrules{nohyphenation}\justifying\noindent\hspace{10mm} अत्र रावण\-नीतां सीतां वर्णयति ग्रन्थकृद्यत् \textcolor{red}{विलप्यमाना} इति। \textcolor{red}{वि}\-पूर्वको \textcolor{red}{लप्‌}\-धातुः (\textcolor{red}{लपँ व्यक्तायां वाचि} धा॰पा॰~४०२) परस्मैपदी। ततः \textcolor{red}{शतृ}\-प्रत्यये \textcolor{red}{विलपन्ती} इति।\footnote{यथा \textcolor{red}{आदीप्य चानुमरणे विलपन्ती मनो दधे} (भा॰पु॰~४.२८.५०) इति भागवत\-प्रयोगे।} \textcolor{red}{विलप्यमाना} इति कथम्। अत्र\footnote{\textcolor{red}{लप्‌}\-धातोः कर्तरि शतरि प्रत्यये कृते।} यक्प्रत्ययाभाव आत्मनेपदाभावश्च। न च \textcolor{red}{कर्तरि कर्म\-व्यतिहारे} (पा॰सू॰~१.३.१४) इत्यनेनाऽत्मनेपदं क्रियतां सीता साधारण\-विलापं करोतीत्यर्थ\-स्वीकारे ग्रन्थ\-गौरव\-पुरः\-सरं भाव\-माधुर्यमपीति चेत्। आत्मनेपदे सत्यपि \textcolor{red}{यक्} कुतः सीताया विलपन\-कर्तृत्वादिति चेत्।\footnote{यतो हि \textcolor{red}{सार्वधातुके यक्} (पा॰सू॰~३.१.६७) इति हि सूत्रं भावकर्मणोरेव प्रवर्तते। \textcolor{red}{चिण् भावकर्मणोः} (पा॰सू॰~३.१.६६) इत्यतो \textcolor{red}{भावकर्मणोः} इत्यस्यानुवृत्तेः।} \textcolor{red}{विलप्यत इति विलप्यमानम्} भावे शानचि \textcolor{red}{भाव\-कर्मणोः} (पा॰सू॰~१.३.१३) इत्यात्मने\-पदत्वात् \textcolor{red}{विलप्यमानमस्ति नित्यमस्यामिति विलप्यमाना}।\footnote{\textcolor{red}{अर्शआदिभ्योऽच्} (पा॰सू॰~५.२.१२७) इत्यनेनाचि \textcolor{red}{यचि भम्} (पा॰सू॰~१.४.१८) इत्यनेन भसञ्ज्ञायां \textcolor{red}{यस्येति च} (पा॰सू॰~६.४.१४८) इत्यनेनालोपे \textcolor{red}{अजाद्यतष्टाप्‌} (पा॰सू॰~४.१.४) इत्यनेन टापि विभक्ति\-कार्ये। विग्रहे \textcolor{red}{नित्यम्} इति तु \textcolor{red}{भूम\-निन्दा\-प्रशंसासु नित्ययोगेऽति\-शायने। सम्बन्धेऽस्ति\-विवक्षायां भवन्ति मतुबादयः॥} (भा॰पा॰सू॰~५.२.९४) इत्यनुसारम्।} अथवेमं दिवादिगणे मत्वा\footnote{\textcolor{red}{बहुलमेतन्निदर्शनम्} (धा॰पा॰ ग॰सू॰~१९३८) \textcolor{red}{आकृतिगणोऽयम्} (धा॰पा॰ ग॰सू॰~१९९२) \textcolor{red}{भूवादिष्वेतदन्तेषु दशगणीषु धातूनां पाठो निदर्शनाय तेन स्तम्भुप्रभृतयः सौत्राश्चुलुम्पादयो वाक्यकारीयाः प्रयोगसिद्धा विक्लवत्यादयश्च} (मा॰धा॰वृ॰~१०.३२८) इत्यनुसारमाकृति\-गणत्वाद्दिवादि\-गण ऊह्योऽयमात्मने\-पदी धातुरिति भावः।} दिवादित्वाच्छ्यन्यात्मनेपदे \textcolor{red}{विलप्यमाना}।\end{sloppypar}
\section[घातितः]{घातितः}
\centering\textcolor{blue}{रावणं तत्र युद्धं मे बभूवारिविमर्दन।\nopagebreak\\
तस्य वाहान् रथं चापं छित्त्वाहं तेन घातितः॥}\nopagebreak\\
\raggedleft{–~अ॰रा॰~३.८.२८}\\
\begin{sloppypar}\hyphenrules{nohyphenation}\justifying\noindent\hspace{10mm} अत्र जटायु\-समीपं गत्वा श्रीरामभद्रस्तद्दशां विलोक्य तत्पराभव\-कारणमपृच्छत्। अत्र जटायुषा निगद्यते यत् \textcolor{red}{तेनाहं घातितः}। तत्र \textcolor{red}{हन्‌}\-धातोः (\textcolor{red}{हनँ हिंसागत्योः} धा॰पा॰~१०१२) कर्मणि \textcolor{red}{क्त}\-प्रत्यये कृते \textcolor{red}{हतः} इत्यनेन भवितव्यम्। यतो \textcolor{red}{रावणो मामहन्} पुनः कर्म\-वाच्ये \textcolor{red}{रावणेनाहमहन्ये} इत्यर्थे \textcolor{red}{रावणेनाहं हतः}। किन्तु \textcolor{red}{घातितः} अयं प्रयोगो हि ण्यन्त\-क्तान्तस्येत्येवापाणिनीयः प्रतिभाति। यतो हि ण्यन्त\-प्रयोगास्तु प्रायः प्रेरक\-कर्तृके भवन्ति। यथा \textcolor{red}{रामो रावणं हन्ति विभीषणस्तं प्रेरयति} इत्यर्थे \textcolor{red}{विभीषणो रामेण रावणं घातयति}। अत्र तु कश्चन प्रेरको नासीत्। अतः प्रयोजकाभावे \textcolor{red}{हेतुमति च} (पा॰सू॰~३.१.२६) इत्यनेन कथं णिजिति चेत्सत्यम्।
\textcolor{red}{रावणो खड्गेन जटायुषमघातयत्}।\footnote{\textcolor{red}{हनँ हिंसागत्योः} (धा॰पा॰~१०१२)~\arrow हन्~\arrow \textcolor{red}{हेतुमति च} (पा॰सू॰~३.१.२६)~\arrow हन्~णिच्~\arrow हन्~इ~\arrow \textcolor{red}{हनस्तोऽचिण्णलोः} (पा॰सू॰~७.३.३२)~\arrow हत्~इ~\arrow \textcolor{red}{हो हन्तेर्ञ्णिन्नेषु} (पा॰सू॰~७.३.५४)~\arrow घत्~इ~\arrow \textcolor{red}{अत उपधायाः} (पा॰सू॰~७.२.११६)~\arrow घात्~इ~\arrow घाति~\arrow \textcolor{red}{सनाद्यन्ता धातवः} (पा॰सू॰~३.१.३२)~\arrow धातुसञ्ज्ञा~\arrow \textcolor{red}{शेषात्कर्तरि परस्मैपदम्} (पा॰सू॰~१.३.७८)~\arrow \textcolor{red}{अनद्यतने लङ्} (पा॰सू॰~३.२.१११)~\arrow घाति~लङ्~\arrow घाति~तिप्~\arrow घाति~ति~\arrow \textcolor{red}{लुङ्लङ्लृङ्क्ष्वडुदात्तः} (पा॰सू॰~६.४.७१)~\arrow अट्~घाति~ति~\arrow अ~घाति~ति~\arrow \textcolor{red}{कर्तरि शप्‌} (पा॰सू॰~३.१.६८)~\arrow अ~घाति~शप्~ति~\arrow अ~घाति~अ~ति~\arrow \textcolor{red}{सार्वधातुकार्ध\-धातुकयोः} (पा॰सू॰~७.३.८४)~\arrow अ~घाते~अ~ति~\arrow \textcolor{red}{एचोऽयवायावः} (पा॰सू॰~६.१.७८)~\arrow अ~घातय्~अ~ति~\arrow \textcolor{red}{इतश्च} (पा॰सू॰~३.४.१००)~\arrow अ~घातय्~अ~त्~\arrow अघातयत्।} पुनः कर्म\-वाच्ये \textcolor{red}{रावणेन खड्गेन जटायुरघात्यत}\footnote{घाति~\arrow धातुसञ्ज्ञा (पूर्ववत्)~\arrow \textcolor{red}{भावकर्मणोः} (पा॰सू॰~१.३.१३)~\arrow \textcolor{red}{अनद्यतने लङ्} (पा॰सू॰~३.२.१११)~\arrow घाति~लङ्~\arrow घाति~त~\arrow \textcolor{red}{लुङ्लङ्लृङ्क्ष्वडुदात्तः} (पा॰सू॰~६.४.७१)~\arrow अट्~घाति~त~\arrow अ~घाति~त~\arrow \textcolor{red}{सार्वधातुके यक्} (पा॰सू॰~३.१.६७)~\arrow \textcolor{red}{आद्यन्तौ टकितौ} (पा॰सू॰~१.१.४६)~\arrow अ~घाति~यक्~ति~\arrow अ~घाति~य~त~\arrow \textcolor{red}{णेरनिटि} (पा॰सू॰~६.४.५१)~\arrow अ~घात्~य~त्~\arrow अघात्यत।} इत्यर्थे हिंसार्थक\-\textcolor{red}{हन्‌}\-धातोर्णिचि \textcolor{red}{हनस्तोऽचिण्णलोः} (पा॰सू॰~७.३.३२) इत्यनेन तान्तादेशे \textcolor{red}{हो हन्तेर्ञ्णिन्नेषु} (पा॰सू॰~७.३.५४) इत्यनेन घकारे \textcolor{red}{अत उपधायाः} (पा॰सू॰~७.२.११६) इत्यनेन वृद्धौ ततः \textcolor{red}{समान\-कर्तृकयोः पूर्व\-काले} (पा॰सू॰~३.४.२१) इत्यनेन कर्मणि \textcolor{red}{क्त}\-प्रत्यये विभक्ति\-कार्ये \textcolor{red}{घातितः} इति।\footnote{घाति~\arrow धातुसञ्ज्ञा पूर्ववत्~\arrow \textcolor{red}{समान\-कर्तृकयोः पूर्व\-काले} (पा॰सू॰~३.४.२१)~\arrow घाति~क्त~\arrow घाति~त~\arrow \textcolor{red}{आर्धधातुकस्येड्वलादेः} (पा॰सू॰~७.२.३५)~\arrow \textcolor{red}{आद्यन्तौ टकितौ} (पा॰सू॰~१.१.४६)~\arrow घाति~इट्~त~\arrow घाति~इ~त~\arrow \textcolor{red}{निष्ठायां सेटि} (पा॰सू॰~६.४.५२)~\arrow घात्~इ~त~\arrow घातित~\arrow विभक्तिकार्यम्~\arrow घातितः।} यद्वाऽत्र स्वार्थे णिचि \textcolor{red}{रावणेन जटायुरहन्यत} इत्यर्थ एव \textcolor{red}{रावणेन जटायुरघात्यत}।\footnote{प्रक्रिया पूर्ववत्।} पुनरस्मिन्स्वार्थे \textcolor{red}{क्त}\-प्रत्यये \textcolor{red}{घातितः}।\footnote{प्रक्रिया पूर्ववत्।} यद्वा \textcolor{red}{हननमेव घातः}।\footnote{\textcolor{red}{हनँ हिंसागत्योः} (धा॰पा॰~१०१२)~\arrow हन्~\arrow \textcolor{red}{भावे} (पा॰सू॰~३.३.१८)~\arrow हन्~घञ्~\arrow हन्~अ~\arrow \textcolor{red}{हनस्तोऽचिण्णलोः} (पा॰सू॰~७.३.३२)~\arrow हत्~अ~\arrow \textcolor{red}{हो हन्तेर्ञ्णिन्नेषु} (पा॰सू॰~७.३.५४)~\arrow घत्~अ~\arrow \textcolor{red}{अत उपधायाः} (पा॰सू॰~७.२.११६)~\arrow घात्~अ~\arrow घात~\arrow विभक्ति\-कार्यम्~\arrow घातः। यथा \textcolor{red}{चरेद्व्रतमहत्वाऽपि घातार्थं चेत्समागतः} (या॰स्मृ॰~३.२५२) \textcolor{red}{वियोगो मुग्धाक्ष्याः स खलु रिपुघातावधिरभूत्} (उ॰रा॰च॰~३.४४) \textcolor{red}{यो हन्यात्तस्य पापं स्याच्छत\-ब्राह्मण\-घातजम्} (प॰त॰~१.३१२) \textcolor{red}{सदय\-हृदय\-दर्शित\-पशुघातम्} (गी॰गो॰~१.९) इत्यादि\-प्रयोगेषु।} \textcolor{red}{घातमितो घातितः}।\footnote{\textcolor{red}{द्वितीया श्रितातीत\-पतित\-गतात्यस्त\-प्राप्तापन्नैः} (पा॰सू॰~२.१.२४) इत्यत्र \textcolor{red}{द्वितीया} इति योग\-विभागेन समासे \textcolor{red}{घात~इत} इति स्थिते \textcolor{red}{कृत्तद्धित\-समासाश्च} (पा॰सू॰~१.२.४६) इत्यनेन प्रातिपदिक\-सञ्ज्ञायां \textcolor{red}{सुपो धातु\-प्रातिपदिकयोः} (पा॰सू॰~२.४.७१) इत्यनेन विभक्ति\-लुकि \textcolor{red}{शकन्ध्वादिषु पर\-रूपं वाच्यम्} (वा॰~६.१.९४) इत्यनेन पररूपे विभक्तिकार्ये सिद्धम्।}\end{sloppypar}
\section[स्मयन्]{स्मयन्}
\centering\textcolor{blue}{तथेति रामः पस्पर्श तदङ्गं पाणिना स्मयन्।\nopagebreak\\
ततः प्राणान्परित्यज्य जटायुः पतितो भुवि॥}\nopagebreak\\
\raggedleft{–~अ॰रा॰~३.८.३६}\\
\begin{sloppypar}\hyphenrules{nohyphenation}\justifying\noindent\hspace{10mm} जटायुषो वृत्तान्तं समाकर्ण्य राजीव\-लोचनो रामः स्मयमानस्तदङ्गं पस्पर्श। अत्र \textcolor{red}{स्मयन्} प्रयोगो ह्यपाणिनीय इव। यतो हीषद्धासार्थकः \textcolor{red}{स्मि}\-धातुः (\textcolor{red}{ष्मिङ् ईषद्धसने} धा॰पा॰~९४८) आत्मनेपदी। ततश्च \textcolor{red}{स्मयत इति स्मयमानः} इत्येव पाणिनीयः। किन्तु \textcolor{red}{स्मयन्} इत्यपि। तथा च \textcolor{red}{स्मयत इति स्मयः}। पचादित्वादच्।\footnote{\textcolor{red}{नन्दि\-ग्रहि\-पचादिभ्यो ल्युणिन्यचः} (पा॰सू॰~३.१.१३४) इत्यनेन।} \textcolor{red}{स्मय इवाऽचरति} इत्यर्थे क्विपि सर्वापहारि\-लोपे सनाद्यन्तत्वाद्धातु\-सञ्ज्ञायां लटि तिपि शपि पररूपे \textcolor{red}{स्मयति}।\footnote{स्मय~\arrow \textcolor{red}{सर्वप्राति\-पदिकेभ्य आचारे क्विब्वा वक्तव्यः} (वा॰~३.१.११)~\arrow स्मय~क्विँप्~\arrow स्मय~व्~\arrow \textcolor{red}{वेरपृक्तस्य} (पा॰सू॰~६.१.६७)~\arrow स्मय~\arrow \textcolor{red}{सनाद्यन्ता धातवः} (पा॰सू॰~३.१.३२)~\arrow धातुसञ्ज्ञा~\arrow \textcolor{red}{शेषात्कर्तरि परस्मैपदम्} (पा॰सू॰~१.३.७८)~\arrow \textcolor{red}{वर्तमाने लट्} (पा॰सू॰~३.२.१२३)~\arrow स्मय~लट्~\arrow स्मय~तिप्~\arrow स्मय~ति~\arrow \textcolor{red}{कर्तरि शप्‌} (पा॰सू॰~३.१.६८)~\arrow स्मय~शप्~ति~\arrow स्मय~अ~ति~\arrow \textcolor{red}{अतो गुणे} (पा॰सू॰~६.१.९७)~\arrow स्मय~ति~\arrow स्मयति।} \textcolor{red}{स्मयतीति स्मयन्} इत्यर्थे शतृ\-प्रयोगे न दोषः।\footnote{स्मय~\arrow धातुसञ्ज्ञा (पूर्ववत्)~\arrow \textcolor{red}{शेषात्कर्तरि परस्मैपदम्} (पा॰सू॰~१.३.७८)~\arrow \textcolor{red}{वर्तमाने लट्} (पा॰सू॰~३.२.१२३)~\arrow स्मय~लट्~\arrow \textcolor{red}{लटः शतृशानचावप्रथमा\-समानाधिकरणे} (पा॰सू॰~३.२.१२४)~\arrow स्मय~शतृँ~\arrow स्मय~अत्~\arrow \textcolor{red}{अतो गुणे} (पा॰सू॰~६.१.९७)~\arrow स्मयत्~\arrow \textcolor{red}{कृत्तद्धित\-समासाश्च} (पा॰सू॰~१.२.४६)~\arrow प्रातिपादिक\-सञ्ज्ञा~\arrow विभक्ति\-कार्यम्~\arrow स्मयत्~सुँ~\arrow स्मयत्~स्~\arrow \textcolor{red}{उगिदचां सर्वनामस्थानेऽधातोः} (पा॰सू॰~७.१.७०)~\arrow \textcolor{red}{मिदचोऽन्त्यात्परः} (पा॰सू॰~१.१.४७)~\arrow स्मय~नुँम्~त्~स्~\arrow स्मय~न्~त्~स्~\arrow \textcolor{red}{हल्ङ्याब्भ्यो दीर्घात्सुतिस्यपृक्तं हल्} (पा॰सू॰~६.१.६८)~\arrow स्मय~न्~त्~\arrow \textcolor{red}{संयोगान्तस्य लोपः} (पा॰सू॰~८.२.२३)~\arrow स्मय~न्~त्~स्~\arrow स्मयन्।} \textcolor{red}{श्रीराम ईषद्धासानुकूल\-व्यापार\-सदृशाचरणानुकूल\-व्यापाराश्रयः} इति शाब्द\-बोधः। यतो हि श्रीरामचन्द्रस्तु जटायुषं पितरं मन्यमानस्तन्म्रियमाण\-दशां दृष्ट्वा शोकातुर आसीत्। किन्तु मरण\-काले जटायुषो मनसि व्यथा मा भूदिति कृत्वा \textcolor{red}{स्मयमान इव प्रतीयते स्म}। अतः \textcolor{red}{स्मयन्} इति सम्यक्पाणिनीयः। जटायुषं प्रति राघवेन्द्रस्य व्यथामत्रैव ग्रन्थकृत्स्पष्टयति यथा~–\end{sloppypar}
\centering\textcolor{blue}{रामस्तमनुशोचित्वा बन्धुवत्साश्रुलोचनः।\nopagebreak\\
लक्ष्मणेन समानाय्य काष्ठानि प्रददाह तम्॥}\nopagebreak\\
\raggedleft{–~अ॰रा॰~३.८.३७}\\
\begin{sloppypar}\hyphenrules{nohyphenation}\justifying\noindent अतः \textcolor{red}{स्मयन्} इत्याचार\-क्विबन्ताच्छतृ\-प्रत्ययः कमपि निगूढं भावं व्यञ्जयति।\end{sloppypar}
\section[अनुशोचित्वा]{अनुशोचित्वा}
\centering\textcolor{blue}{रामस्तमनुशोचित्वा बन्धुवत्साश्रुलोचनः।\nopagebreak\\
लक्ष्मणेन समानाय्य काष्ठानि प्रददाह तम्॥}\nopagebreak\\
\raggedleft{–~अ॰रा॰~३.८.३७}\\
\begin{sloppypar}\hyphenrules{nohyphenation}\justifying\noindent\hspace{10mm} जटायुषो मृत्युं दृष्ट्वा भक्त\-वत्सलः श्रीरामस्तमनुशोच्य तद्दाह\-संस्कारं चक्रे। अत्रानूपसर्ग\-योगेन \textcolor{red}{समासेऽनञ्पूर्वे क्त्वो ल्यप्} (पा॰सू॰~७.१.३७) इत्यनेन ल्यपि \textcolor{red}{अनुशोच्य} इत्येव पाणिनि\-सम्मतम्। \textcolor{red}{अनुशोचित्वा} अपि शोकार्थकात् \textcolor{red}{शुच्} धातोः (\textcolor{red}{शुचँ शोके} धा॰पा॰~१८३)
\textcolor{red}{क्त्वा} प्रत्यय इति कृते गुणे। न च कित्वाल्लघूपध\-गुण\-निषेधः शङ्क्यः।\footnote{\textcolor{red}{ग्क्ङिति च} (पा॰सू॰~१.१.५) इत्यनेन कित्त्वे गुणनिषेधः प्राप्तः।} \textcolor{red}{न क्त्वा सेट्} (पा॰सू॰~१.२.१८) इत्यनेन कित्व\-निषेधे गुणे \textcolor{red}{अनुशोचित्वा}। न च समासे सति \textcolor{red}{अनुशोच्य} इति भविष्यति कथम् \textcolor{red}{अनुशोचित्वा} इति। \textcolor{red}{अनु}\-शब्दस्य \textcolor{red}{शोचित्वा} इत्यनेन सह न योगोऽपि तु \textcolor{red}{तम्} इत्यनेन सह। तथा च \textcolor{red}{अनुर्लक्षणे} (पा॰सू॰~१.४.८४) इत्यनेन कर्म\-प्रवचनीय\-सञ्ज्ञायां \textcolor{red}{कर्म\-प्रवचनीय\-युक्ते द्वितीया} (पा॰सू॰~२.३.८) इत्यनेन द्वितीया\-विभक्तिः। अथ चानूपयोगाभावे \textcolor{red}{क्त्वा}\-प्रत्ययः निर्विवादः। यद्वा \textcolor{red}{लक्षणेत्थम्भूताख्यान\-भाग\-वीप्सासु प्रतिपर्यनवः} (पा॰सू॰~१.४.९०) इत्यनेन \textcolor{red}{अनु}\-शब्दस्य कर्म\-प्रवचनीय\-सञ्ज्ञायां द्वितीयायाञ्च सत्यां \textcolor{red}{तमनु शोचित्वा} इति साधु।\end{sloppypar}
\section[प्राणरिरक्षया]{प्राणरिरक्षया}
\centering\textcolor{blue}{बाहुभ्यां वेष्टितावत्र तव प्राणरिरक्षया।\nopagebreak\\
छिन्नौ तव भुजौ त्वं च को वा विकटरूपधृक्॥}\nopagebreak\\
\raggedleft{–~अ॰रा॰~३.९.१४}\\
\begin{sloppypar}\hyphenrules{nohyphenation}\justifying\noindent\hspace{10mm} अत्र कबन्धः \textcolor{red}{तव प्राण\-रिरक्षया} इति प्रयुङ्क्ते। \textcolor{red}{प्राणस्य रक्षितुमिच्छा} इति \textcolor{red}{प्राण\-रिरक्षा}। किन्त्वत्र \textcolor{red}{रिरक्षा} इत्यपाणिनीयः। तथा हि \textcolor{red}{रक्षितुमिच्छा} इति विग्रहे \textcolor{red}{रक्ष्‌}\-धातोः (\textcolor{red}{रक्षँ पालने} धा॰पा॰~६५८) \textcolor{red}{धातोः कर्मणः समान\-कर्तृकादिच्छायां वा} (पा॰सू॰~३.१.७) इत्यनेन \textcolor{red}{सन्} प्रत्यय इटि षत्वे \textcolor{red}{सन्यङोः} (पा॰सू॰~६.१.९) इत्यनेन द्वित्वे \textcolor{red}{पूर्वोऽभ्यासः} (पा॰सू॰~६.१.४) इत्यनेनाभ्यास\-सञ्ज्ञायामभ्यास\-कार्ये \textcolor{red}{हलादिः शेषः} (पा॰सू॰~७.४.६०) इत्यनेन रकार\-भावे शिष्टे \textcolor{red}{सन्यतः} (पा॰सू॰~७.४.७९) इत्यनेनेकारे \textcolor{red}{अ प्रत्ययात्} (पा॰सू॰~३.३.१०२) इत्यनेन \textcolor{red}{अ}\-प्रत्यये टापि \textcolor{red}{रिरक्षिषा} इत्येव।\footnote{यथा \textcolor{red}{जगद्रिरक्षिषया} (भा॰पु॰~५.१५.६) इत्यत्र। \textcolor{red}{रक्षँ पालने} (धा॰पा॰~६५८)~\arrow रक्ष्~\arrow \textcolor{red}{धातोः कर्मणः समान\-कर्तृकादिच्छायां वा} (पा॰सू॰~३.१.७)~\arrow रक्ष्~सन्~\arrow रक्ष्~स~\arrow \textcolor{red}{आर्धधातुकस्येड्वलादेः} (पा॰सू॰~७.२.३५)~\arrow रक्ष्~इट्~स~\arrow रक्ष्~इ~स~\arrow \textcolor{red}{आदेशप्रत्यययोः} (पा॰सू॰~८.३.५९)~\arrow रक्ष्~इ~ष~\arrow रक्षिष~\arrow \textcolor{red}{सन्यङोः} (पा॰सू॰~६.१.९)~\arrow रक्ष्~रक्षिष~\arrow \textcolor{red}{हलादिः शेषः} (पा॰सू॰~७.४.६०)~\arrow र~रक्षिष~\arrow \textcolor{red}{सन्यतः} (पा॰सू॰~७.४.७९)~\arrow रि~रक्षिष~\arrow रिरक्षिष~\arrow \textcolor{red}{सनाद्यन्ता धातवः} (पा॰सू॰~३.१.३२)~\arrow धातु\-सञ्ज्ञा~\arrow \textcolor{red}{अ प्रत्ययात्} (पा॰सू॰~३.३.१०२)~\arrow रिरक्षिष~अ~\arrow \textcolor{red}{अतो लोपः} (पा॰सू॰~६.४.४८)~\arrow रिरक्षिष्~अ~\arrow रिरक्षिष~\arrow \textcolor{red}{अजाद्यतष्टाप्‌} (पा॰सू॰~४.१.४)~\arrow रिरक्षिष~टाप्~\arrow रिरक्षिष~आ~\arrow \textcolor{red}{अकः सवर्णे दीर्घः} (पा॰सू॰~६.१.१०१)~\arrow रिरक्षिषा।} अत्र \textcolor{red}{अ प्रत्ययात्} (पा॰सू॰~३.३.१०२) इत्यनेनाकारे पृषोदरादि\-त्वात्सन्प्रत्ययस्य लोपे\footnote{\textcolor{red}{पृषोदरादीनि यथोपदिष्टम्} (पा॰सू॰~६.३.१०९)। पृषोदरादित्वादिटोऽपि लोपो बोध्यः। विकल्पत्वात्पक्षे \textcolor{red}{रिरक्षिषा} इत्यपि यत्र न लोपकार्यौ।} टापि समासे तृतीयैक\-वचने \textcolor{red}{प्राणरिरक्षया}।\footnote{एवमेव \textcolor{red}{एष साक्षाद्धरेरंशो जातो लोकरिरक्षया} (भा॰पु॰~४.१५.६) इत्यत्रापि।} यद्वा \textcolor{red}{प्राणस्यारयः प्राणारयः क्षुत्पिपासादयस्तेभ्यो रक्षेति प्राणरिरक्षा तया} इति विग्रहे \textcolor{red}{प्राण}\-शब्दस्य \textcolor{red}{अरि}\-शब्देन समासे शकन्ध्वादित्वात्पर\-रूपे पुनः \textcolor{red}{प्राणरि}\-शब्दस्य \textcolor{red}{रक्षा}\-शब्देन सह समासे तृतीयैक\-वचने \textcolor{red}{प्राणरिरक्षया}।\footnote{एवमेव \textcolor{red}{पुरोऽवतस्थे कृष्णस्य पुत्रप्राणरिरक्षया} (भा॰पु॰~१०.६३.२०) इत्यत्रापि।}\end{sloppypar}
\section[आदृता]{आदृता}
\centering\textcolor{blue}{रामलक्ष्मणयोः सम्यक्पादौ प्रक्षाल्य भक्तितः।\nopagebreak\\
तज्जलेनाभिषिच्याङ्गमथार्घ्यादिभिरादृता॥}\nopagebreak\\
\raggedleft{–~अ॰रा॰~३.१०.७}\\
\begin{sloppypar}\hyphenrules{nohyphenation}\justifying\noindent\hspace{10mm} भक्तवत्सलः श्रीरामः शबरीमुद्दिधीर्षन् तया बहुमानितः। सा भगवतः श्रीचरणारविन्दं प्रक्षाल्यार्घ्यादिभिरादृता। \textcolor{red}{आदृतवती} इति प्रयोक्तव्ये \textcolor{red}{आदृता} इति प्रयुक्तम्। यद्यपि \textcolor{red}{आ}\-पूर्वकात् \textcolor{red}{दृ}\-धातोः (\textcolor{red}{दृङ् आदरे} धा॰पा॰~१४११) सकर्मकतया \textcolor{red}{तयोरेव कृत्य\-क्त\-खलर्थाः} (पा॰सू॰~३.४.७०) इति सूत्रेण \textcolor{red}{क्त}\-प्रत्यय\-विधानं कर्मण्येव पाणिन्यनुकूलं तथाऽपि \textcolor{red}{गत्यर्थाकर्मक\-श्लिष\-शीङ्स्थास\-वस\-जन\-रुह\-जी\-र्यतिभ्यश्च} (पा॰सू॰~३.४.७२) इत्यत्र \textcolor{red}{च}कारात्क्वचित्सकर्मकादपि। तेनात्र सकर्मक\-धातोः कर्तरि \textcolor{red}{आदृता}। यद्वा \textcolor{red}{आदर एव आदृतम्}। भावे \textcolor{red}{क्त}\-प्रत्ययः।\footnote{\textcolor{red}{नपुंसके भावे क्तः} (पा॰सू॰~३.३.११४) इत्यनेन।} \textcolor{red}{आदृतमस्त्यस्या इत्यादृता}। अर्शआद्यजन्तः प्रयोगः।\footnote{\textcolor{red}{अर्शआदिभ्योऽच्} (पा॰सू॰~५.२.१२७) इत्यनेन।} यद्वा कर्मणोऽविवक्षायामकर्मकत्वात्\footnote{\textcolor{red}{धातोरर्थान्तरे वृत्तेर्धात्वर्थेनोपसङ्ग्रहात्। प्रसिद्धेरविवक्षातः कर्मणोऽकर्मिका क्रिया॥} (वा॰प॰~३.७.८८)} \textcolor{red}{गत्यर्थाकर्मक\-श्लिष\-शीङ्स्थास\-वस\-जन\-रुह\-जीर्यतिभ्यश्च} (पा॰सू॰~३.४.७२) इत्यनेन कर्तरि \textcolor{red}{क्त}\-प्रत्यये टापि \textcolor{red}{आदृता} इति सम्यक्।\footnote{यथाऽऽह मनुः~– \textcolor{red}{सर्वेष्वेव व्रतेष्वेवं प्रायश्चित्तार्थमादृतः} (म॰स्मृ॰~११.२२५)। अत्र टीकाकाराः~– \textcolor{red}{आदृतो यत्नवान्} (म॰स्मृ॰ मे॰टी॰~११.२२५) \textcolor{red}{यत्नवान्} (म॰स्मृ॰ कु॰टी॰~११.२२५) \textcolor{red}{आदृतः श्रद्धालुः} (म॰स्मृ॰ राघ॰टी॰~११.२२५)। एवमेव भागवते~– \textcolor{red}{यावद्‍ब्रह्म विजानीयान्मामेव गुरुमादृतः} (भा॰पु॰~११.१८.३९)। अत्र वीरराघव\-टीका~– \textcolor{red}{तत्राऽदृत आदरयुक्तः} (भा॰पु॰ वी॰रा॰व्या॰~११.१८.३९)। रघुवंशेऽपि कालिदासः~– \textcolor{red}{इत्यादृतेन कथितौ रघुनन्दनेन व्युत्क्रम्य लक्ष्मणमुभौ भरतो ववन्दे} (र॰वं॰~१३.७२)। अत्र सञ्जीविन्यां मल्लिनाथः~– \textcolor{red}{इत्यादृतेन आदरवता। कर्तरि क्तः} (र॰वं॰ स॰व्या॰~१३.७२)। दर्पण\-टीकाकारो हेमाद्रिस्तु \textcolor{red}{इत्यादरेण कथितौ रघुनन्दनेन} (र॰वं॰ द॰टी॰~१३.७२) इति पाठं स्वीचक्रे। पञ्चतन्त्रे विष्णुशर्मा च~– \textcolor{red}{आत्मानमादृतो रक्षेत्प्रमादाद्धि विनश्यति} (प॰त॰~३.२२९)। अत्राभिनव\-राज\-लक्ष्मी टीका~– \textcolor{red}{आदृतः सावधानः सन्} (प॰त॰ अ॰टी॰~३.२२९)।}\end{sloppypar}
\section[विरागित्वम्]{विरागित्वम्}
\centering\textcolor{blue}{मद्भक्तेष्वधिका पूजा सर्वभूतेषु मन्मतिः।\nopagebreak\\
बाह्यार्थेषु विरागित्वं शमादिसहितं तथा॥}\nopagebreak\\
\raggedleft{–~अ॰रा॰~३.१०.२६}\\
\begin{sloppypar}\hyphenrules{nohyphenation}\justifying\noindent\hspace{10mm} \textcolor{red}{विगतो रागो यस्य स विरागस्तस्य भावो विरागत्वम्} इति बहुव्रीहि\-जन्य\-\textcolor{red}{विराग}\-शब्दात् \textcolor{red}{त्व}\-प्रत्यये\footnote{\textcolor{red}{तस्य भावस्त्वतलौ} (पा॰सू॰~५.१.११९) इत्यनेन।} यद्यप्यर्थ\-सिद्धिः किं \textcolor{red}{न कर्मधारयान्मत्वर्थीयो बहुव्रीहिश्चेत्तदर्थ\-प्रतिपत्तिकरः}\footnote{मूलं मृग्यम्।} इत्यस्य नियमस्योल्लङ्घनेन तथाऽपि \textcolor{red}{विगतो रागो विरागः} इत्यत्र \textcolor{red}{प्रगत आचार्यः प्राचार्यः} (भा॰पा॰सू॰~२.२.१८)\footnote{\textcolor{red}{प्रगत आचार्यः प्राचार्यः} इत्यत्र \textcolor{red}{प्रादयः क्तार्थे} (वा॰~२.२.१८) इत्यनेन \textcolor{red}{प्रादयो गताद्यर्थे प्रथमया} (वा॰~२.२.१८) इत्यनेन च समासो भाष्ये दर्शितः। उभे वार्तिके \textcolor{red}{कु\-गति\-प्रादयः} (पा॰सू॰~२.२.१८) इति सूत्रे पठिते।} इतिवत् \textcolor{red}{कु\-गति\-प्रादयः} (पा॰सू॰~२.२.१८) इत्यनेन समासे \textcolor{red}{विरागः प्रशस्तो नित्यो वाऽस्त्यस्मिन्} इति बहुव्रीह्यलब्ध\-विशिष्टार्थं बोधयितुं कर्मधारयादिनिः।\footnote{\textcolor{red}{भूम\-निन्दा\-प्रशंसासु नित्ययोगेऽति\-शायने। सम्बन्धेऽस्ति\-विवक्षायां भवन्ति मतुबादयः॥} (भा॰पा॰सू॰~५.२.९४)।} ततश्च \textcolor{red}{तस्य भावस्त्वतलौ} (पा॰सू॰~५.१.११९) इत्यनेन \textcolor{red}{त्व}\-प्रत्यये विभक्ति\-कार्ये \textcolor{red}{विरागित्वम्}।\end{sloppypar}
\vspace{2mm}
\centering ॥ इत्यरण्यकाण्डीयप्रयोगाणां विमर्शः ॥\nopagebreak\\
\vspace{4mm}
\pdfbookmark[2]{किष्किन्धाकाण्डम्}{Chap2Part2Kanda4}
\phantomsection
\addtocontents{toc}{\protect\setcounter{tocdepth}{2}}\addtocontents{toc}{\protect\setcounter{tocdepth}{2}}
\addcontentsline{toc}{subsection}{किष्किन्धाकाण्डीयप्रयोगाणां विमर्शः}
\addtocontents{toc}{\protect\setcounter{tocdepth}{0}}
\centering ॥ अथ किष्किन्धाकाण्डीयप्रयोगाणां विमर्शः ॥\nopagebreak\\
\section[दाशरथो रामः]{दाशरथो रामः}
\centering\textcolor{blue}{अहं दाशरथो रामस्त्वयं मे लक्ष्मणोऽनुजः।\nopagebreak\\
सीतया भार्यया सार्धं पितुर्वचनगौरवात्॥}\nopagebreak\\
\raggedleft{–~अ॰रा॰~४.१.१९}\\
\begin{sloppypar}\hyphenrules{nohyphenation}\justifying\noindent\hspace{10mm} सीतामन्वीक्षमाणौ धनुर्बाणधरौ श्रीराम\-लक्ष्मणौ विलोक्य
तदाशङ्कया
सुग्रीवेण प्रेषितं वटु\-वेष\-धारिणं मारुतिं दृष्ट्वा तेन नामादि\-परिचयं पृष्टः श्रीरामभद्रः सङ्क्षिप्य परिचयं प्रस्तौति \textcolor{red}{दाशरथो रामः} इति। अत्र \textcolor{red}{दशरथस्यापत्यं पुमान् दाशरथिः} इति विग्रहे \textcolor{red}{तस्यापत्यम्} (पा॰सू॰~४.१.९२) इति सूत्रार्थानुसारमपत्यार्थे षष्ठ्यन्त\-दशरथ\-प्रातिपदिकात् \textcolor{red}{अत इञ्} (पा॰सू॰~४.१.९५) इत्यनेन \textcolor{red}{इञ्} प्रत्यये \textcolor{red}{दाशरथिः} इत्येव प्रसिद्ध\-प्रयोगः \textcolor{red}{दाशरथः} इति कथमकारान्तात् \textcolor{red}{इञ्} प्रत्ययस्य दुर्निवारत्वात्। श्रीरामो वस्तुतो दशरथस्य नापत्यं तत्क्षेत्र\-जन्य\-व्यवहाराद्दाशरथिरित्युपचर्यते। अतो हनुमतः समक्षमपत्य\-रूपमर्थं न कथयन्नाह \textcolor{red}{दाशरथः}। \textcolor{red}{दशरथस्यायं दाशरथः} इति विग्रहे \textcolor{red}{तस्येदम्} (पा॰सू॰~४.३.१२०) इत्यनेन \textcolor{red}{अण्‌}\-प्रत्यये भत्वादकार\-लोपे\footnote{\textcolor{red}{यचि भम्} (पा॰सू॰~१.४.१८) इत्यनेन भत्वम्। \textcolor{red}{यस्येति च} (पा॰सू॰~६.४.१४८) इत्यनेनाकार\-लोपः।} \textcolor{red}{दाशरथः}। दशरथस्य स्वेन सह केवलं पाल्य\-पालक\-भाव\-रूप\-सम्बन्धस्यैव विवक्षा भक्त\-प्रवर\-हनुमतः सम्मुखे राघवेन्द्रस्य। यद्वा \textcolor{red}{दशरथादागतो दाशरथः} इति विग्रहे पञ्चम्यन्त\-दशरथ\-शब्दात् \textcolor{red}{तत आगतः} (पा॰सू॰~४.३.७४) इति \textcolor{red}{अण्‌}\-प्रत्ययः। लोपादि\-कार्ये \textcolor{red}{तद्धितेष्वचामादेः} (पा॰सू॰~७.२.११७) इत्यनेन वृद्धौ \textcolor{red}{दाशरथः}। अर्थादण् पार्थक्ये। यतो हि दशरथस्य सकाशादहमागतः। अथवाऽपि \textcolor{red}{विभाषा गुणेऽस्त्रियाम्} (पा॰सू॰~२.३.२५) इत्यनेन पञ्चमी। ततोऽण्। अनेन \textcolor{red}{प्रदीयतां दाशरथाय मैथिली} (वा॰रा॰~६.१४.३) इति वाल्मीकीय\-रामायण\-प्रयोगोऽपि व्याख्यातः।\end{sloppypar}
\section[अभिषेचनम्]{अभिषेचनम्}
\centering\textcolor{blue}{तच्छ्रुत्वा दुःखिताः सर्वे मामनिच्छन्तमप्युत।\nopagebreak\\
राज्येऽभिषेचनं चक्रुः सर्वे वानरमन्त्रिणः॥}\nopagebreak\\
\raggedleft{–~अ॰रा॰~४.१.५३}\\
\begin{sloppypar}\hyphenrules{nohyphenation}\justifying\noindent\hspace{10mm} अत्र \textcolor{red}{मामभिषेचनं चक्रुः} इति सामानाधिकरण्य\-दर्शनात् \textcolor{red}{अभिषेचनम्} इत्यत्र प्रत्यय\-सन्देह\-परं भवति \textcolor{red}{माम्}। तथा च भावे \textcolor{red}{ल्युट्}\footnote{\textcolor{red}{ल्युट् च} (पा॰सू॰~३.३.११५) इत्यनेन।} चेत्कृद्योगे \textcolor{red}{कर्तृ\-कर्मणोः कृति} (पा॰सू॰~२.३.६५) इति सूत्रेण षष्ठ्यां \textcolor{red}{ममाभिषेचनम्} इति स्यात्। द्वितीयायां प्रत्यय\-जिज्ञासा तदवस्थेति चेत्।
\textcolor{red}{अभिषिच्यत इत्यभिषेचनः} इति कर्मणि \textcolor{red}{कृत्य\-ल्युटो बहुलम्} (पा॰सू॰~३.३.११३) इत्यनेन ल्युट्। यद्वा \textcolor{red}{सेचनम्} इति भाव\-ल्युडन्तम्। \textcolor{red}{अभितः सेचनं यस्य सोऽभिषेचनस्तमभिषेचनम्} इति \textcolor{red}{प्रादिभ्यो धातुजस्य वाच्यो वा चोत्तरपद\-लोपश्च} (वा॰~२.२.२२) इत्यनेन समासे \textcolor{red}{अभिषेचनम्} इति सिद्धम्।\end{sloppypar}
\section[बलवतां बली]{बलवतां बली}
\centering\textcolor{blue}{सुग्रीवोऽप्याह राजेन्द्र वाली बलवतां बली।\nopagebreak\\
कथं हनिष्यति भवान्देवैरपि दुरासदम्॥}\nopagebreak\\
\raggedleft{–~अ॰रा॰~४.१.६०}\\
\begin{sloppypar}\hyphenrules{nohyphenation}\justifying\noindent\hspace{10mm} अत्र बहूनां निर्धारणतया \textcolor{red}{तमप्} प्रत्ययः \textcolor{red}{इष्ठन्} प्रत्ययो वा प्राप्तः\footnote{\textcolor{red}{अतिशायने तमबिष्ठनौ} (पा॰सू॰~५.३.५५) इत्यनेन।} किन्त्वविवक्षणतया न।\end{sloppypar}
\section[चेतनम्]{चेतनम्}
\centering\textcolor{blue}{तदा मुहूर्त्तं निःसंज्ञो भूत्वा चेतनमाप सः।\nopagebreak\\
ततो वाली ददर्शाग्रे रामं राजीवलोचनम्।\nopagebreak\\
धनुरालम्ब्य वामेन हस्तेनान्येन सायकम्॥}\nopagebreak\\
\raggedleft{–~अ॰रा॰~४.२.४८}\\
\begin{sloppypar}\hyphenrules{nohyphenation}\justifying\noindent\hspace{10mm} श्रीराम\-बाण\-भिन्न\-शरीरो भूमौ पतितो वाली चेतनां प्राप्तत्वान्। अत्र \textcolor{red}{चेतनम्} इति प्रयुक्तम्। \textcolor{red}{चितीँ सञ्ज्ञाने} (धा॰पा॰~३९) इति धातोः स्वार्थे णिचि\footnote{\textcolor{red}{चितँ सञ्चेतने} (धा॰पा॰~१६७३) इत्यस्मात्स्वार्थे णिचि वा।} ततश्च \textcolor{red}{चेत्यत इति चेतना}\footnote{चेति~यक्~त इति स्थिते \textcolor{red}{णेरनिटि} (पा॰सू॰~६.४.५१) इत्यनेन णिलोपे चेत्यते।} इति विग्रहे \textcolor{red}{ण्यास\-श्रन्थो युच्} (पा॰सू॰~३.३.१०७) इत्यनेन भावे युचि प्रत्ययेऽनादेशे स्त्रीत्वाट्टाप्प्रत्यये \textcolor{red}{चेतना}।\footnote{यद्वा \textcolor{red}{चितँ सञ्चेतने} (धा॰पा॰~१६७३) इत्यतः \textcolor{red}{सत्याप\-पाश\-रूप\-वीणा\-तूल\-श्लोक\-सेना\-लोम\-त्वच\-वर्म\-वर्ण\-चूर्ण\-चुरादिभ्यो णिच्} (पा॰सू॰~३.१.२५) इत्यनेन स्वार्थे णिचि \textcolor{red}{ण्यास\-श्रन्थो युच्} (पा॰सू॰~३.३.१०७) इत्यनेन भावे युच्यनादेशे टापि \textcolor{red}{चेतना}।} \textcolor{red}{चेतनम्} इत्यत्र हि शुद्धात् \textcolor{red}{चेतति} इत्यस्माद्भावे ल्युट्।\footnote{\textcolor{red}{ल्युट् च} (पा॰सू॰~३.३.११५) इत्यनेन।} \textcolor{red}{चेतनम्}। \end{sloppypar}
\section[भ्राजद्वनमालाविभूषितम्]{भ्राजद्वनमालाविभूषितम्}
\centering\textcolor{blue}{बिभ्राणं चीरवसनं जटामुकुटधारिणम्।\nopagebreak\\
विशालवक्षसं भ्राजद्वनमालाविभूषितम्॥}\nopagebreak\\
\raggedleft{–~अ॰रा॰~४.२.४९}\\
\begin{sloppypar}\hyphenrules{nohyphenation}\justifying\noindent\hspace{10mm} भूमिपतितो वाली समर\-धीर\-रघु\-वीरस्य भुवन\-मोहन\-सौन्दर्यं लोचनातिथी\-करोति यच्छ्रीरामो वल्कल\-धरो विविध\-भूषण\-भूषितः। तत्र \textcolor{red}{भ्राजद्वनमाला\-विभूषितम्} इति शब्द\-घटिते \textcolor{red}{भ्राजत्} इत्यत्र \textcolor{red}{शतृ}\-प्रयोगोऽनुचितः।\footnote{\textcolor{red}{भ्राजृँ दीप्तौ} (पा॰सू॰~१८१) \textcolor{red}{टुभ्राजृँ दीप्तौ} (धा॰पा॰~८३३) इत्यनयोरात्मने\-पदीयत्वाच्छानचि \textcolor{red}{भ्राजमान\-वनमाला\-विभूषितम्} इति वक्तव्यमिति भावः।} तथाऽप्यस्याऽत्मने\-पदीयत्वं त्वौप\-चारिकमेव। अनुदात्तेत्त्व\-लक्षणस्याऽत्मने\-पदस्यानित्यत्वात्।\footnote{\textcolor{red}{अनुदात्तेत्त्व\-लक्षणमात्मने\-पदमनित्यम्} (प॰शे॰~९३.४)।} \textcolor{red}{भ्राजन्ती चासौ वनमाला चेति भ्राजद्वनमाला तया भूषितम्} इति।\end{sloppypar}
\section[तिरोभूत्वा]{तिरोभूत्वा}
\centering\textcolor{blue}{राजधर्ममविज्ञाय गर्हितं कर्म ते कृतम्।\nopagebreak\\
वृक्षखण्डे तिरोभूत्वा त्यजता मयि सायकम्॥}\nopagebreak\\
\raggedleft{–~अ॰रा॰~४.२.५२}\\
\begin{sloppypar}\hyphenrules{nohyphenation}\justifying\noindent\hspace{10mm} वाली श्रीरामं भर्त्सयन्नाह यन्मां गुप्तवेषो हतवान्। अत्र \textcolor{red}{तिरोभूत्वा} इति प्रयुक्तम्। \textcolor{red}{तिरस्‌}\-शब्दस्य हि \textcolor{red}{भू}\-शब्देन समासे क्त्वो ल्यपि\footnote{\textcolor{red}{समासेऽनञ्पूर्वे क्त्वो ल्यप्‌} (पा॰सू॰~७.१.३७) इत्यनेन।} \textcolor{red}{तिरोभूय} इत्येव। असति समासे \textcolor{red}{तिरो} इति पृथक्पदम्। कथं न \textcolor{red}{तिरस्} तर्हि। संहिताया विवक्षणात्। अतः समासाभावे \textcolor{red}{तिरो भूत्वा} इति न दोषः।\end{sloppypar}
\section[वानरम्]{वानरम्}
\centering\textcolor{blue}{ वानरं व्याधवद्धत्वा धर्मं कं लप्स्यसे वद।\nopagebreak\\
अभक्ष्यं वानरं मांसं हत्वा मां किं करिष्यसि॥}\nopagebreak\\
\raggedleft{–~अ॰रा॰~४.२.५८}\\
\begin{sloppypar}\hyphenrules{nohyphenation}\justifying\noindent\hspace{10mm} वाली कथयति यत् \textcolor{red}{वानरं मांसं} विगर्हितम्। अत्र \textcolor{red}{वानरे भवमिति वानरीयम्}। \textcolor{red}{वृद्धाच्छः} (पा॰सू॰~४.२.११४) इत्यनेन \textcolor{red}{छ}\-प्रत्यये \textcolor{red}{वानरीयम्} इति पाणिनीयम्। \textcolor{red}{वानरम्} इति कथम्। \textcolor{red}{नीलो घटः} इतिवत् \textcolor{red}{वानरं मांसम्}। \textcolor{red}{शाब्द\-बोधे चैक\-पदार्थेऽपर\-पदार्थस्य संसर्गः संसर्ग\-मर्यादया भासते} (व्यु॰वा॰ का॰प्र॰) इति व्युत्पत्ति\-वाद\-प्रयोगात् \textcolor{red}{वानराभिन्नं मांसम्}। यद्वा \textcolor{red}{वानरस्येदं वानरम्} इति विग्रहे \textcolor{red}{तस्येदम्} (पा॰सू॰~४.३.१२०) इति \textcolor{red}{अण्‌}\-प्रत्ययः।\end{sloppypar}
\section[बहु भाषन्तम्]{बहु भाषन्तम्}
\centering\textcolor{blue}{इत्येवं बहु भाषन्तं वालिनं राघवोऽब्रवीत्।\nopagebreak\\
धर्मस्य गोप्ता लोकेऽस्मिंश्चरामि सशरासनः॥}\nopagebreak\\
\raggedleft{–~अ॰रा॰~४.२.५९}\\
\begin{sloppypar}\hyphenrules{nohyphenation}\justifying\noindent\hspace{10mm} पतितं वालिनं बहु भाषमाणं श्रीरामभद्रः प्रतिवक्ति। अत्र \textcolor{red}{भाषमाणम्} इत्यर्थे \textcolor{red}{भाषन्तम्} इति प्रयुक्तम्। यतो हि \textcolor{red}{भाष्} धातुः (\textcolor{red}{भाषँ व्यक्तायां वाचि} धा॰पा॰~६१२) आत्मनेपदीयस्तथा च \textcolor{red}{भाषत इति भाषमाणस्तं भाषमाणम्} इति शानचा भवितव्यमासीत्। किन्तु \textcolor{red}{भाषत इति भाषः} पचादित्वादच्।\footnote{\textcolor{red}{नन्दि\-ग्रहि\-पचादिभ्यो ल्युणिन्यचः} (पा॰सू॰~३.१.१३४) इत्यनेन।} \textcolor{red}{भाष इवाऽचरति}
इति क्विपि लटि तिपि शपि पररूपे \textcolor{red}{भाषति}।\footnote{भाष~\arrow \textcolor{red}{सर्वप्राति\-पदिकेभ्य आचारे क्विब्वा वक्तव्यः} (वा॰~३.१.११)~\arrow भाष~क्विँप्~\arrow भाष~व्~\arrow \textcolor{red}{वेरपृक्तस्य} (पा॰सू॰~६.१.६७)~\arrow भाष~\arrow \textcolor{red}{सनाद्यन्ता धातवः} (पा॰सू॰~३.१.३२)~\arrow धातुसञ्ज्ञा~\arrow \textcolor{red}{शेषात्कर्तरि परस्मैपदम्} (पा॰सू॰~१.३.७८)~\arrow \textcolor{red}{वर्तमाने लट्} (पा॰सू॰~३.२.१२३)~\arrow भाष~लट्~\arrow भाष~तिप्~\arrow भाष~ति~\arrow \textcolor{red}{कर्तरि शप्‌} (पा॰सू॰~३.१.६८)~\arrow भाष~शप्~ति~\arrow भाष~अ~ति~\arrow \textcolor{red}{अतो गुणे} (पा॰सू॰~६.१.९७)~\arrow भाष~ति~\arrow भाषति।} \textcolor{red}{भाषतीति भाषन् तं भाषन्तम्} इत्याचार\-क्विबन्तात् \textcolor{red}{शतृ}\-प्रत्यये न दोषः।\footnote{भाष~\arrow धातुसञ्ज्ञा (पूर्ववत्)~\arrow \textcolor{red}{शेषात्कर्तरि परस्मैपदम्} (पा॰सू॰~१.३.७८)~\arrow \textcolor{red}{वर्तमाने लट्} (पा॰सू॰~३.२.१२३)~\arrow भाष~लट्~\arrow \textcolor{red}{लटः शतृशानचावप्रथमा\-समानाधिकरणे} (पा॰सू॰~३.२.१२४)~\arrow भाष~शतृँ~\arrow भाष~अत्~\arrow \textcolor{red}{अतो गुणे} (पा॰सू॰~६.१.९७)~\arrow भाषत्~\arrow \textcolor{red}{कृत्तद्धित\-समासाश्च} (पा॰सू॰~१.२.४६)~\arrow प्रातिपादिक\-सञ्ज्ञा~\arrow विभक्ति\-कार्यम्~\arrow भाषत्~सुँ~\arrow भाषत्~स्~\arrow \textcolor{red}{उगिदचां सर्वनामस्थानेऽधातोः} (पा॰सू॰~७.१.७०)~\arrow \textcolor{red}{मिदचोऽन्त्यात्परः} (पा॰सू॰~१.१.४७)~\arrow भाष~नुँम्~त्~स्~\arrow भाष~न्~त्~स्~\arrow \textcolor{red}{हल्ङ्याब्भ्यो दीर्घात्सुतिस्यपृक्तं हल्} (पा॰सू॰~६.१.६८)~\arrow भाष~न्~त्~\arrow \textcolor{red}{संयोगान्तस्य लोपः} (पा॰सू॰~८.२.२३)~\arrow भाष~न्~त्~स्~\arrow भाषन्। भाषत्~\arrow प्रातिपदिक\-सञ्ज्ञा (पूर्ववत्)~\arrow विभक्तिकार्यम्~\arrow भाषत्~अम्~\arrow \textcolor{red}{उगिदचां सर्वनामस्थानेऽधातोः} (पा॰सू॰~७.१.७०)~\arrow \textcolor{red}{मिदचोऽन्त्यात्परः} (पा॰सू॰~१.१.४७)~\arrow भाष~नुँम्~त्~अम्~\arrow भाष~न्~त्~अम्~\arrow भाषन्तम्।} \textcolor{red}{वालिर्भाषते नह्यपि तु भाष इवाऽचरति} इति परस्मैपदस्य तात्पर्यम्।\end{sloppypar}
\section[दापितम्]{दापितम्}
\centering\textcolor{blue}{सुग्रीवं त्वं सुखं राज्यं दापितं वालिघातिना।\nopagebreak\\
रामेण रुमया सार्धं भुङ्क्ष्व सापत्नवर्जितम्॥}\nopagebreak\\
\raggedleft{–~अ॰रा॰~४.३.११}\\
\begin{sloppypar}\hyphenrules{nohyphenation}\justifying\noindent\hspace{10mm} अत्र \textcolor{red}{दत्तम्} इति न कथयित्वा \textcolor{red}{दापितम्} इति प्रयुक्तम्। स्वार्थे णिचि \textcolor{red}{अर्ति\-ह्री\-व्ली\-री\-क्नूयी\-क्ष्माय्यातां पुङ्णौ} (पा॰सू॰~७.३.३६) इत्यनेन पुकि \textcolor{red}{क्त}\-प्रत्यये \textcolor{red}{दापितम्}। अर्थात् \textcolor{red}{रामेण राज्यमदाप्यत}। यद्वा \textcolor{red}{वाल्यददाद्रामः प्रेरयत्} इत्यर्थे \textcolor{red}{वालिना राज्यमदाप्यत} कर्म\-वाच्ये \textcolor{red}{वालि\-घातिना वालिना राज्यमदाप्यत} इत्यस्मिन्नर्थे \textcolor{red}{क्त}\-प्रत्ययः। अत्र \textcolor{red}{हेतुमति च} (पा॰सू॰~३.१.२६) इत्यनेन णिच्।\end{sloppypar}
\section[कुर्वन्ती]{कुर्वन्ती}
\centering\textcolor{blue}{ध्यात्वा मद्रूपमनिशमालोचय मयोदितम्।\nopagebreak\\
प्रवाहपतितं कार्यं कुर्वन्त्यपि न लिप्यसे॥}\nopagebreak\\
\raggedleft{–~अ॰रा॰~४.३.३५}\\
\begin{sloppypar}\hyphenrules{nohyphenation}\justifying\noindent\hspace{10mm} श्रीरामस्तारां प्रति कथयति \textcolor{red}{कार्यं कुर्वत्यपि मत्कृपया न लिप्ता भविष्यसि}। अत्र \textcolor{red}{नुम्} अपाणिनीय इव। \textcolor{red}{करोतीति कुर्वती} इति तनादित्वाच्छबभावे\footnote{शाभावे श्यनभावे चेति बोध्यम्।} \textcolor{red}{नुम्} कथमिति चेत्। \textcolor{red}{गण\-कार्यमनित्यम्} (प॰शे॰~९३.३) इत्यनेन शपि \textcolor{red}{नुम्} सङ्गत एव। यद्वा सौत्र\-धातव इवात्राप्याकृति\-गणतया क्रियार्थः \textcolor{red}{कुर्वँ धातुः} भ्वादिगणे पठ्यतां\footnote{\textcolor{red}{बहुलमेतन्निदर्शनम्} (धा॰पा॰ ग॰सू॰~१९३८) \textcolor{red}{आकृतिगणोऽयम्} (धा॰पा॰ ग॰सू॰~१९९२) \textcolor{red}{भूवादिष्वेतदन्तेषु दशगणीषु धातूनां पाठो निदर्शनाय तेन स्तम्भुप्रभृतयः सौत्राश्चुलुम्पादयो वाक्यकारीयाः प्रयोगसिद्धा विक्लवत्यादयश्च} (मा॰धा॰वृ॰~१०.३२८) इत्यनुसारमाकृति\-गणत्वाद्भ्वादि\-गण ऊह्योऽयं धातुः।} तथा च \textcolor{red}{कुर्वतीति कुर्वन्ती} इति \textcolor{red}{शतृ}\-प्रत्यये \textcolor{red}{नुम्} साधुः।\end{sloppypar}
\section[असहन्]{असहन्}
\centering\textcolor{blue}{रामस्तु पर्वतस्याग्रे मणिसानौ निशामुखे।\nopagebreak\\
सीताविरहजं शोकमसहन्निदमब्रवीत्॥}\nopagebreak\\
\raggedleft{–~अ॰रा॰~४.५.१}\\
\begin{sloppypar}\hyphenrules{nohyphenation}\justifying\noindent\hspace{10mm} \textcolor{red}{षहँ मर्षणे} (धा॰पा॰~८५२, १८०९) इत्यात्मनेपदीय\-धातुः। तत्र शानचि \textcolor{red}{सहमानः} इति पाणिनीयः। \textcolor{red}{असहन्} इति तु \textcolor{red}{सहत इति सहो न सह इत्यसहः} पचादित्वादच्\footnote{\textcolor{red}{नन्दि\-ग्रहि\-पचादिभ्यो ल्युणिन्यचः} (पा॰सू॰~३.१.१३४) इत्यनेन।} नञ्समासश्च।\footnote{\textcolor{red}{नलोपो नञः} (पा॰सू॰~६.३.७३) इत्यनेन नलोपः।} \textcolor{red}{असह इवाऽचरतीत्यसहति}।\footnote{असह~\arrow \textcolor{red}{सर्वप्राति\-पदिकेभ्य आचारे क्विब्वा वक्तव्यः} (वा॰~३.१.११)~\arrow असह~क्विँप्~\arrow असह~व्~\arrow \textcolor{red}{वेरपृक्तस्य} (पा॰सू॰~६.१.६७)~\arrow असह~\arrow \textcolor{red}{सनाद्यन्ता धातवः} (पा॰सू॰~३.१.३२)~\arrow धातुसञ्ज्ञा~\arrow \textcolor{red}{शेषात्कर्तरि परस्मैपदम्} (पा॰सू॰~१.३.७८)~\arrow \textcolor{red}{वर्तमाने लट्} (पा॰सू॰~३.२.१२३)~\arrow असह~लट्~\arrow असह~तिप्~\arrow असह~ति~\arrow \textcolor{red}{कर्तरि शप्‌} (पा॰सू॰~३.१.६८)~\arrow असह~शप्~ति~\arrow असह~अ~ति~\arrow \textcolor{red}{अतो गुणे} (पा॰सू॰~६.१.९७)~\arrow असह~ति~\arrow असहति।} \textcolor{red}{असहतीत्यसहन्}। आचार\-क्विबन्ताच्छतृ\-प्रत्ययः।\footnote{असह~\arrow धातुसञ्ज्ञा (पूर्ववत्)~\arrow \textcolor{red}{शेषात्कर्तरि परस्मैपदम्} (पा॰सू॰~१.३.७८)~\arrow \textcolor{red}{वर्तमाने लट्} (पा॰सू॰~३.२.१२३)~\arrow असह~लट्~\arrow \textcolor{red}{लटः शतृशानचावप्रथमा\-समानाधिकरणे} (पा॰सू॰~३.२.१२४)~\arrow असह~शतृँ~\arrow असह~अत्~\arrow \textcolor{red}{अतो गुणे} (पा॰सू॰~६.१.९७)~\arrow असहत्~\arrow \textcolor{red}{कृत्तद्धित\-समासाश्च} (पा॰सू॰~१.२.४६)~\arrow प्रातिपादिक\-सञ्ज्ञा~\arrow विभक्ति\-कार्यम्~\arrow असहत्~सुँ~\arrow असहत्~स्~\arrow \textcolor{red}{उगिदचां सर्वनामस्थानेऽधातोः} (पा॰सू॰~७.१.७०)~\arrow \textcolor{red}{मिदचोऽन्त्यात्परः} (पा॰सू॰~१.१.४७)~\arrow असह~नुँम्~त्~स्~\arrow असह~न्~त्~स्~\arrow \textcolor{red}{हल्ङ्याब्भ्यो दीर्घात्सुतिस्यपृक्तं हल्} (पा॰सू॰~६.१.६८)~\arrow असह~न्~त्~\arrow \textcolor{red}{संयोगान्तस्य लोपः} (पा॰सू॰~८.२.२३)~\arrow असह~न्~त्~स्~\arrow असहन्।
} \textcolor{red}{असहन\-शील\-समानमाचरणं करोति}। वस्तुतस्तु तस्य क्व वियोग इत्येवाऽचार\-क्विबन्ताच्छतृ\-प्रत्ययस्याऽध्यात्मिकं तात्पर्यं प्रतिभाति।\end{sloppypar}
\section[विस्मृतः]{विस्मृतः}
\centering\textcolor{blue}{कृतघ्नवत्त्वया नूनं विस्मृतः प्रतिभाति मे।\nopagebreak\\
त्वत्कृते निहितो वाली वीरस्त्रैलोक्यसम्मतः॥}\nopagebreak\\
\raggedleft{–~अ॰रा॰~४.४.४५}\\
\centering\textcolor{blue}{रामकार्यार्थमनिशं जागर्ति न तु विस्मृतः।\nopagebreak\\
आगताः परितः पश्य वानराः कोटिशः प्रभो॥}\nopagebreak\\
\raggedleft{–~अ॰रा॰~४.५.५५}\\
\begin{sloppypar}\hyphenrules{nohyphenation}\justifying\noindent\hspace{10mm} सकर्मक\-स्मृ\-धातोः (\textcolor{red}{स्मृ आध्याने} धा॰पा॰~८०७) \textcolor{red}{वि}\-उपसर्ग\-पूर्वकात्कर्तरि \textcolor{red}{क्तवतु}\-प्रत्यये \textcolor{red}{विस्मृतवान्} इति पाणिनीयः। किन्तु \textcolor{red}{विस्मरणं विस्मृतम्} इति भाव\-क्तान्त\-\textcolor{red}{विस्मृत}\-शब्दात्\footnote{\textcolor{red}{नपुंसके भावे क्तः} (पा॰सू॰~३.३.११४) इत्यनेन भाव\-क्तान्त\-शब्दः।} तदस्त्यस्येत्यर्शआद्यजन्तात्\footnote{\textcolor{red}{अर्शआदिभ्योऽच्} पा॰सू॰~५.२.१२७) इत्यनेन।} \textcolor{red}{विस्मृतः} अपि पाणिनीयं कर्तृ\-विशेषणं सदपि।\footnote{\textcolor{red}{विस्मृतमस्त्यस्येति विस्मृतः} इति भावः। यद्वा कर्मणोऽविवक्षायामकर्मकत्वात् \textcolor{red}{गत्यर्थाकर्मक\-श्लिष\-शीङ्स्थास\-वस\-जन\-रुह\-जीर्यतिभ्यश्च} (पा॰सू॰~३.४.७२) इत्यनेन कर्तरि क्तः।}\end{sloppypar}
\section[गृह्य]{गृह्य}
\centering\textcolor{blue}{इत्युक्त्वा लक्ष्मणं भक्त्या करे गृह्य स मारुतिः।\nopagebreak\\
आनयामास नगरमध्याद्राजगृहं प्रति॥}\nopagebreak\\
\raggedleft{–~अ॰रा॰~४.५.३९}\\
\begin{sloppypar}\hyphenrules{nohyphenation}\justifying\noindent\hspace{10mm} अत्रोपसर्गं विना कथं ल्यबिति चेत्।
\textcolor{red}{करे} इत्यस्य साक्षाद्गणे पाठात् \textcolor{red}{साक्षात्प्रभृतीनि च} (पा॰सू॰~१.४.७४) इत्यनेन गतिसञ्ज्ञायां समासे ल्यबादेशे न दोषः।\footnote{\textcolor{red}{साक्षात्प्रभृतीनि च} (पा॰सू॰~१.४.७४) इत्यत्र \textcolor{red}{क्वचिदेक\-देशोऽप्यनुवर्तते} (प॰शे॰~१८) इति परिभाषया \textcolor{red}{कृञि} (पा॰सू॰~१.४.७२) इत्यस्य मण्डूक\-प्लुत्या निवृत्तौ \textcolor{red}{विभाषा} (पा॰सू॰~१.४.७२) इत्यस्यानुवृत्तौ कृञभावेऽपि साक्षात्प्रभृतीनां गति\-सञ्ज्ञा क्वाचित्का। यद्वा चकार\-ग्रहणात्कृञभावेऽपि गति\-सञ्ज्ञा क्वाचित्का। सत्यां गतिसञ्ज्ञायां \textcolor{red}{कुगतिप्रादयः} (पा॰सू॰~२.२.१८) इत्यनेन समास इति भावः।}
यद्वा \textcolor{red}{प्र}\-उपसर्ग आसीत्तस्य \textcolor{red}{विनाऽपि प्रत्ययं पूर्वोत्तर\-पद\-लोपो वक्तव्यः} (वा॰~५.३.८३) इति वार्त्तिकेन लोपः।\end{sloppypar}
\section[मारुतिः]{मारुतिः}
\centering\textcolor{blue}{इत्युक्त्वा लक्ष्मणं भक्त्या करे गृह्य स मारुतिः।\nopagebreak\\
आनयामास नगरमध्याद्राजगृहं प्रति॥}\nopagebreak\\
\raggedleft{–~अ॰रा॰~४.५.३९}\\
\begin{sloppypar}\hyphenrules{nohyphenation}\justifying\noindent\hspace{10mm} यद्यपि \textcolor{red}{मरुतोऽयम्} इति विग्रहे तु \textcolor{red}{मारुतः}।\footnote{\textcolor{red}{तस्येदम्} (पा॰सू॰~४.३.१२०) इत्यनेन।} अकारान्ताभावात् \textcolor{red}{इञ्} प्रत्ययस्याऽप्यभावः। किन्तु \textcolor{red}{मरुदेव मारुतः}।\footnote{\textcolor{red}{समीरमारुतमरुज्जगत्प्राणसमीरणाः} (अ॰को॰~१.१.६२)} प्रज्ञादित्वात्स्वार्थे \textcolor{red}{अण्}।\footnote{\textcolor{red}{मरुत्‌}\-प्रातिपदिकात् \textcolor{red}{प्रज्ञादिभ्यश्च} (पा॰सू॰~५.४.३८) इत्यनेन \textcolor{red}{अण्‌}\-प्रत्यये \textcolor{red}{तद्धितेष्वचामादेः} (पा॰सू॰~७.२.११७) इत्यनेनाऽदिवृद्धौ विभक्तिकार्ये।} \textcolor{red}{मारुतस्यापत्यं पुमान् मारुतिः} इति \textcolor{red}{अत इञ्} (पा॰सू॰~४.१.९५) इत्यनेनेञ्प्रत्यये विभक्ति\-कार्ये \textcolor{red}{मारुतिः}।\footnote{\textcolor{red}{यचि भम्} (पा॰सू॰~१.४.१८) इत्यनेन भत्वम्। \textcolor{red}{यस्येति च} (पा॰सू॰~६.४.१४८) इत्यनेनाकार\-लोपः।}\end{sloppypar}
\section[दशसाहस्राः]{दशसाहस्राः}
\centering\textcolor{blue}{प्रेषिता दशसाहस्रा हरयो रघुसत्तम।\nopagebreak\\
आनेतुं वानरान् दिग्भ्यो महापर्वतसन्निभान्॥}\nopagebreak\\
\raggedleft{–~अ॰रा॰~४.५.४६}\\
\begin{sloppypar}\hyphenrules{nohyphenation}\justifying\noindent\hspace{10mm} \textcolor{red}{दशसाहस्रमस्त्येषामिति दशसाहस्राः} इत्यर्शआद्यजन्तम्।\footnote{\textcolor{red}{अर्शआदिभ्योऽच्} पा॰सू॰~५.२.१२७) इत्यनेन। \textcolor{red}{यचि भम्} (पा॰सू॰~१.४.१८) इत्यनेन भत्वम्। \textcolor{red}{यस्येति च} (पा॰सू॰~६.४.१४८) इत्यनेनाकार\-लोपः।}\end{sloppypar}
\vspace{2mm}
\centering ॥ इति किष्किन्धाकाण्डीयप्रयोगाणां विमर्शः ॥\nopagebreak\\
\vspace{4mm}
\pdfbookmark[2]{सुन्दरकाण्डम्}{Chap2Part2Kanda5}
\phantomsection
\addtocontents{toc}{\protect\setcounter{tocdepth}{2}}
\addcontentsline{toc}{subsection}{सुन्दरकाण्डीयप्रयोगाणां विमर्शः}
\addtocontents{toc}{\protect\setcounter{tocdepth}{0}}
\centering ॥ अथ सुन्दरकाण्डीयप्रयोगाणां विमर्शः ॥\nopagebreak\\
\section[उद्वमती]{उद्वमती}
\centering\textcolor{blue}{हनूमानपि तां वाममुष्टिनाऽवज्ञयाऽहनत्।\nopagebreak\\
तदैव पतिता भूमौ रक्तमुद्वमती भृशम्॥}\nopagebreak\\
\raggedleft{–~अ॰रा॰~५.१.४६}\\
\begin{sloppypar}\hyphenrules{nohyphenation}\justifying\noindent\hspace{10mm} अत्र हनुमन्मुष्टि\-प्रहारेण रक्तमुद्वमन्ती लङ्किनी पपात। \textcolor{red}{वम्} धातोः (\textcolor{red}{टुवमँ उद्गिरणे} धा॰पा॰~९८४) भ्वादित्वात् \textcolor{red}{नुम्} प्रयोक्तव्यः।\footnote{\textcolor{red}{शप्श्यनोर्नित्यम्} (पा॰सू॰~७.१.८१) इत्यनेन \textcolor{red}{नुम्} नित्यमिति भावः।} किन्त्वत्र \textcolor{red}{उद्वमति} इति विग्रह उणादिः \textcolor{red}{तृँच्} प्रत्ययः।\footnote{नायं \textcolor{red}{बहुलमन्यत्रापि} (प॰उ॰~२.९५) इति तृच्। स नोगित्। \textcolor{red}{कार्याद्विद्यादनूबन्धम्} (भा॰पा॰सू॰~३.३.१) \textcolor{red}{केचिदविहिता अप्यूह्याः} (वै॰सि॰कौ॰~३१६९) इत्यनुसारमूह्योऽ\-यमविहित उगित्प्रत्ययः। \textcolor{red}{तृँच्‌}\-प्रत्यये चात्र शबागमोऽप्यूह्यः। \textcolor{red}{नयतेः षुगागमः} (प॰उ॰ श्वे॰वृ॰~२.९६) इतिवत्।} ततो ङीपि\footnote{\textcolor{red}{उगितश्च} (पा॰सू॰~४.१.६) इत्यनेन।} \textcolor{red}{उद्वमती}। यद्वा \textcolor{red}{गण\-कार्यमनित्यम्} (प॰शे॰~९३.३) इत्यनेन शबभावे नुमभावः।\footnote{शतर्येव शम्नुमभावे \textcolor{red}{उद्वमती} इति भावः। यद्वा \textcolor{red}{आगम\-शास्त्रमनित्यम्} (प॰शे॰~९३.२) इत्यनेन नुमभावः। उत् \textcolor{red}{टुवमँ उद्गिरणे} (धा॰पा॰~९८४)~\arrow उत्~वम्~\arrow \textcolor{red}{शेषात्कर्तरि परस्मैपदम्} (पा॰सू॰~१.३.७८)~\arrow \textcolor{red}{वर्तमाने लट्} (पा॰सू॰~३.२.१२३)~\arrow उत्~वम्~लट्~\arrow \textcolor{red}{लटः शतृशानचावप्रथमा\-समानाधिकरणे} (पा॰सू॰~३.२.१२४)~\arrow उत्~वम्~शतृँ~\arrow उत्~वम्~अत्~\arrow \textcolor{red}{गण\-कार्यमनित्यम्} (प॰शे॰~९३.३)~\arrow शबभावः~\arrow उत्~वमत्~\arrow \textcolor{red}{झलां जशोऽन्ते} (पा॰सू॰~८.२.३९)~\arrow उद्~वमत्~\arrow उद्वमत्~\arrow \textcolor{red}{उगितश्च} (पा॰सू॰~४.१.६)~\arrow उद्वमत्~ङीप्‌~\arrow उद्वमत्~ई~\arrow उद्वमती~\arrow \textcolor{red}{कृत्तद्धित\-समासाश्च} (पा॰सू॰~१.२.४६)~\arrow प्रातिपदिक\-सञ्ज्ञा~\arrow उद्वमती~सुँ~\arrow \textcolor{red}{हल्ङ्याब्भ्यो दीर्घात्सुतिस्यपृक्तं हल्} (पा॰सू॰~६.१.६८)~\arrow उद्वमती। \textcolor{red}{यद्वा उत्~वम्~अत्} (पूर्ववत्)~\arrow \textcolor{red}{कर्तरि शप्‌} (पा॰सू॰~३.१.६८)~\arrow उत्~वम्~शप्~अत्~\arrow उत्~वम्~अ~अत्~\arrow \textcolor{red}{अतो गुणे} (पा॰सू॰~६.१.९७)~\arrow उत्~वमत्~\arrow \textcolor{red}{झलां जशोऽन्ते} (पा॰सू॰~८.२.३९)~\arrow उद्~वमत्~\arrow उद्वमत्~\arrow \textcolor{red}{उगितश्च} (पा॰सू॰~४.१.६)~\arrow उद्वमत्~ङीप्‌~\arrow उद्वमत्~ई~\arrow \textcolor{red}{शप्श्यनोर्नित्यम्} (पा॰सू॰~७.१.८१)~\arrow नित्यनुम्प्राप्तिः~\arrow \textcolor{red}{आगम\-शास्त्रमनित्यम्} (प॰शे॰~९३.२)~\arrow नुमभावः~\arrow उद्वमती~\arrow शेषं पूर्ववत्।}\end{sloppypar}
\section[ऐन्द्रः]{ऐन्द्रः}
\centering\textcolor{blue}{ऐन्द्रः काकस्तदागत्य नखैस्तुण्डेन चासकृत्।\nopagebreak\\
मत्पादाङ्गुष्ठमारक्तं विददाराऽमिषाशया॥}\nopagebreak\\
\raggedleft{–~अ॰रा॰~५.३.५४}\\
\begin{sloppypar}\hyphenrules{nohyphenation}\justifying\noindent\hspace{10mm} \textcolor{red}{इन्द्रस्यापत्यं पुमान्} इति विग्रहे \textcolor{red}{अत इञ्} (पा॰सू॰~४.१.९५) इत्यनेन \textcolor{red}{इञ्} प्रत्यये \textcolor{red}{ऐन्द्रिः}। किन्तु \textcolor{red}{इन्द्रस्यायम्} इति विग्रहे \textcolor{red}{तस्येदम्} (पा॰सू॰~४.३.१२०) इत्येन \textcolor{red}{अण्} प्रत्यये भत्वादकार\-लोपे\footnote{\textcolor{red}{यचि भम्} (पा॰सू॰~१.४.१८) इत्यनेन भत्वम्। \textcolor{red}{यस्येति च} (पा॰सू॰~६.४.१४८) इत्यनेनाकार\-लोपः।} \textcolor{red}{ऐन्द्रः}। सीतापहार\-रूप\-गर्हित\-कर्मत्वादपत्यत्व\-कलङ्क\-धियाऽत्र तदपत्यत्वं न विवक्षितम्।\end{sloppypar}
\section[ब्रह्मपाशतः]{ब्रह्मपाशतः}
\centering\textcolor{blue}{बद्ध्वाऽऽनेष्ये द्रुतं तात वानरं ब्रह्मपाशतः।\nopagebreak\\
इत्युक्त्वा रथमारुह्य राक्षसैर्बहुभिर्वृतः॥}\nopagebreak\\
\raggedleft{–~अ॰रा॰~५.३.९२}\\
\begin{sloppypar}\hyphenrules{nohyphenation}\justifying\noindent\hspace{10mm} अत्र \textcolor{red}{ब्रह्म\-पाशतः} इति \textcolor{red}{ब्रह्मपाशेन} विग्रहेऽस्मिन्तृतीयार्थे सार्वविभक्तिकस्तसिः।\footnote{तसेः सार्व\-विभक्तिकत्वं तदन्तानामाकृति\-गणत्वं च \pageref{fn:yatah}तमे पृष्ठे \ref{fn:yatah}तम्यां पादटिप्पण्यां स्पष्टीकृतम्।
\textcolor{red}{ब्रह्मपाशतः} इत्यत्र तृतीयायां तसिरिति भावः।}\end{sloppypar}
\section[भेदयित्वा]{भेदयित्वा}
\centering\textcolor{blue}{भेदयित्वा ततो घोरं सिंहनादमथाकरोत्।\nopagebreak\\
ततोऽतिहर्षाद्धनुमान स्तम्भमुद्यस्य वीर्यवान्॥}\nopagebreak\\
\raggedleft{–~अ॰रा॰~५.३.९६}\\
\begin{sloppypar}\hyphenrules{nohyphenation}\justifying\noindent\hspace{10mm} \textcolor{red}{भिदिँर विदारणे} (धा॰पा॰~१४३९) इत्यत्र स्वार्थिको णिच्।\footnote{\textcolor{red}{निवृत्तप्रेषणाद्धातोः प्राकृतेऽर्थे णिजुच्यते} (वा॰प॰~३.७.६०)।} ततो \textcolor{red}{क्त्वा}\-प्रत्यय इटि गुणेऽयादेशे \textcolor{red}{भेदयित्वा}।\footnote{\textcolor{red}{भिदिँर विदारणे} (धा॰पा॰~१४३९)~\arrow भिद्~\arrow स्वार्थे णिच्~\arrow भिद्~णिच्~\arrow भिद्~इ~\arrow \textcolor{red}{पुगन्त\-लघूपधस्य च} (पा॰सू॰~७.३.८६)~\arrow भेद्~इ~\arrow भेदि~\arrow \textcolor{red}{सनाद्यन्ता धातवः} (पा॰सू॰~३.१.३२)~\arrow धातु\-सञ्ज्ञा~\arrow भेदि~क्त्वा~\arrow \textcolor{red}{समान\-कर्तृकयोः पूर्वकाले} (पा॰सू॰~३.४.२१)~\arrow भेदि~क्त्वा~\arrow भेदि~त्वा~\arrow \textcolor{red}{आर्धधातुकस्येड्वलादेः} (पा॰सू॰~७.२.३५)~\arrow भेदि~इट्~त्वा~\arrow भेदि~इ~त्वा~\arrow \textcolor{red}{सार्वधातुकार्ध\-धातुकयोः} (पा॰सू॰~७.३.८४)~\arrow भेदे~इ~त्वा~\arrow \textcolor{red}{एचोऽयवायावः} (पा॰सू॰~६.१.७८)~\arrow भेदय्~इ~त्वा~\arrow भेदयित्वा~\arrow \textcolor{red}{तद्धितश्चासर्व\-विभक्तिः} (पा॰सू॰~१.१.३८)~\arrow अव्यय\-सञ्ज्ञा~\arrow \textcolor{red}{कृत्तद्धित\-समासाश्च} (पा॰सू॰~१.२.४६)~\arrow प्रातिपदिक\-सञ्ज्ञा~\arrow विभक्तिकार्यम्~\arrow भेदयित्वा~सुँ~\arrow \textcolor{red}{अव्ययादाप्सुपः} (पा॰सू॰~२.४.८२)~\arrow भेदयित्वा। यद्वा \textcolor{red}{भेदनं भेदः}। \textcolor{red}{भावे} (पा॰सू॰~३.३.१८) इत्यनेन घञि। \textcolor{red}{तत्करोति तदाचष्टे} (धा॰पा॰ ग॰सू॰) इत्यनेन णिचि लटि शपि तिपि \textcolor{red}{भेदं करोति} इत्यर्थे  \textcolor{red}{भेदयति}। ततः \textcolor{red}{भेदं कृत्वा} इत्यर्थे \textcolor{red}{भेदयित्वा}। प्रकृत\-धातोरनिट्कत्वाण्णिजभावे \textcolor{red}{भित्त्वा} इति रूपम्।}\end{sloppypar}
\vspace{2mm}
\centering ॥ इति सुन्दरकाण्डीयप्रयोगाणां विमर्शः ॥\nopagebreak\\
\vspace{4mm}
\pdfbookmark[2]{युद्धकाण्डम्}{Chap2Part2Kanda6}
\phantomsection
\addtocontents{toc}{\protect\setcounter{tocdepth}{2}}
\addcontentsline{toc}{subsection}{युद्धकाण्डीयप्रयोगाणां विमर्शः}
\addtocontents{toc}{\protect\setcounter{tocdepth}{0}}
\centering ॥ अथ युद्धकाण्डीयप्रयोगाणां विमर्शः ॥\nopagebreak\\
\section[आरोहयन्तः]{आरोहयन्तः}
\centering\textcolor{blue}{शैलानारोहयन्तश्च जग्मुर्मारुतवेगतः।\nopagebreak\\
असङ्ख्याताश्च सर्वत्र वानराः परिपूरिताः॥}\nopagebreak\\
\raggedleft{–~अ॰रा॰~६.१.३९}\\
\begin{sloppypar}\hyphenrules{nohyphenation}\justifying\noindent\hspace{10mm} \textcolor{red}{आ}\-पूर्वकं \textcolor{red}{रुह्‌}\-धातुं (\textcolor{red}{रुहँ बीजजन्मनि प्रादुर्भावे च} धा॰पा॰~८५९) स्वार्थे णिजन्तं कृत्वा ततः शतरि शपि गुणेऽयादेशे प्रथमा\-बहुवचने \textcolor{red}{आरोहयन्तः}।\footnote{\textcolor{red}{न्यग्भावना न्यग्भवनं रुहौ शुद्धे प्रतीयते। न्यग्भावना न्यग्भवनं ण्यन्तेऽपि प्रतिपद्यते॥ अवस्थां पञ्चमीमाहुर्ण्यन्ते तां कर्मकर्तरि। निवृत्तप्रेषणाद्धातोः प्राकृतेऽर्थे णिजुच्यते॥} (वा॰प॰~३.७.५९–६०)। \textcolor{red}{रुहँ बीजजन्मनि प्रादुर्भावे च} (धा॰पा॰~८५९)~\arrow रुह्~\arrow स्वार्थे णिच्~\arrow रुह्~णिच्~\arrow रुह्~इ~\arrow \textcolor{red}{पुगन्त\-लघूपधस्य च} (पा॰सू॰~७.३.८६)~\arrow रोह्~इ~\arrow रोहि~\arrow \textcolor{red}{सनाद्यन्ता धातवः} (पा॰सू॰~३.१.३२)~\arrow धातु\-सञ्ज्ञा। आङ्~रोहि~\arrow आ~रोहि~\arrow \textcolor{red}{शेषात्कर्तरि परस्मैपदम्} (पा॰सू॰~१.३.७८)~\arrow \textcolor{red}{वर्तमाने लट्} (पा॰सू॰~३.२.१२३)~\arrow आ~रोहि~लट्~\arrow \textcolor{red}{लटः शतृशानचावप्रथमा\-समानाधिकरणे} (पा॰सू॰~३.२.१२४)~\arrow आ~रोहि~शतृँ~\arrow आ~रोहि~अत्~\arrow \textcolor{red}{सार्वधातुकार्ध\-धातुकयोः} (पा॰सू॰~७.३.८४)~\arrow आ~रोहे~अत्~\arrow \textcolor{red}{एचोऽयवायावः} (पा॰सू॰~६.१.७८)~\arrow आ~रोहय्~अत्~\arrow आरोहयत्~\arrow \textcolor{red}{कृत्तद्धित\-समासाश्च} (पा॰सू॰~१.२.४६)~\arrow प्रातिपादिक\-सञ्ज्ञा~\arrow विभक्ति\-कार्यम्~\arrow आरोहयत्~जस्~\arrow आरोहयत्~अस्~\arrow \textcolor{red}{ उगिदचां सर्वनामस्थानेऽधातोः} (पा॰सू॰~७.१.७०)~\arrow \textcolor{red}{मिदचोऽन्त्यात्परः} (पा॰सू॰~१.१.४७)~\arrow आरोहय~नुँम्~त्~अस्~\arrow आरोहय~न्~त्~अस्~\arrow आरोहयन्तस्~\arrow \textcolor{red}{ससजुषो रुः} (पा॰सू॰~८.२.६६)~\arrow आरोहयन्तरुँ~\arrow \textcolor{red}{खरवसानयोर्विसर्जनीयः} (पा॰सू॰~८.३.१५)~\arrow आरोहयन्तः।}\end{sloppypar}
\section[याचते]{याचते}
\centering\textcolor{blue}{सकृदेव प्रपन्नाय तवास्मीति च याचते।\nopagebreak\\
अभयं सर्वभूतेभ्यो ददाम्येतद्व्रतं मम॥}\nopagebreak\\
\raggedleft{–~अ॰रा॰~६.३.१२}\footnote{वाल्मीकीय\-रामायणे च~– \textcolor{red}{सकृदेव प्रपन्नाय तवास्मीति च याचते। अभयं सर्वभूतेभ्यो ददाम्येतद्व्रतं मम॥} (वा॰रा॰~६.१८.३३)।}\\
\begin{sloppypar}\hyphenrules{nohyphenation}\justifying\noindent\hspace{10mm} अत्र शरणागतं विभीषणं प्रति स्व\-स्वभावं प्रकटयन् रामभद्रः प्राह यत् \textcolor{red}{प्रपन्नाय जनाय तवास्मीति याचमानायाहं सर्व\-प्राणिभ्योऽभयं ददामि}। आत्मनेपदीयः \textcolor{red}{याच्} धातुः (\textcolor{red}{टुयाचृँ याञ्चायाम्} धा॰पा॰~८६३)।\footnote{पूर्वपक्षोऽयम्।} तस्मात् \textcolor{red}{याचते} इति हि प्रयोगः।\footnote{यथा \textcolor{red}{बलिं याचते वसुधाम्} (वै॰सि॰कौ॰~५३९, ल॰सि॰कौ॰~८९५) इति प्रसिद्धोदाहरणे।} अस्माच्छानचि चतुर्थ्यैक\-वचने \textcolor{red}{याचमानाय} इति हि पाणिनीयम्। किन्तु \textcolor{red}{याचत इति याचः}।\footnote{\textcolor{red}{नन्दि\-ग्रहि\-पचादिभ्यो ल्युणिन्यचः} (पा॰सू॰~३.१.१३४) इत्यनेन कर्तरि पचाद्यच्।} \textcolor{red}{याच इवाऽचरतीति याचति}।\footnote{याच~\arrow \textcolor{red}{सर्वप्राति\-पदिकेभ्य आचारे क्विब्वा वक्तव्यः} (वा॰~३.१.११)~\arrow याच~क्विँप्~\arrow याच~व्~\arrow \textcolor{red}{वेरपृक्तस्य} (पा॰सू॰~६.१.६७)~\arrow याच~\arrow \textcolor{red}{सनाद्यन्ता धातवः} (पा॰सू॰~३.१.३२)~\arrow धातुसञ्ज्ञा~\arrow \textcolor{red}{शेषात्कर्तरि परस्मैपदम्} (पा॰सू॰~१.३.७८)~\arrow \textcolor{red}{वर्तमाने लट्} (पा॰सू॰~३.२.१२३)~\arrow याच~लट्~\arrow याच~तिप्~\arrow याच~ति~\arrow \textcolor{red}{कर्तरि शप्‌} (पा॰सू॰~३.१.६८)~\arrow याच~शप्~ति~\arrow याच~अ~ति~\arrow \textcolor{red}{अतो गुणे} (पा॰सू॰~६.१.९७)~\arrow याच~ति~\arrow याचति।} \textcolor{red}{याचतीति याचन्}।\footnote{याच~\arrow धातुसञ्ज्ञा (पूर्ववत्)~\arrow \textcolor{red}{शेषात्कर्तरि परस्मैपदम्} (पा॰सू॰~१.३.७८)~\arrow \textcolor{red}{वर्तमाने लट्} (पा॰सू॰~३.२.१२३)~\arrow याच~लट्~\arrow \textcolor{red}{लटः शतृशानचावप्रथमा\-समानाधिकरणे} (पा॰सू॰~३.२.१२४)~\arrow याच~शतृँ~\arrow याच~अत्~\arrow \textcolor{red}{अतो गुणे} (पा॰सू॰~६.१.९७)~\arrow याचत्~\arrow \textcolor{red}{कृत्तद्धित\-समासाश्च} (पा॰सू॰~१.२.४६)~\arrow प्रातिपादिक\-सञ्ज्ञा~\arrow विभक्ति\-कार्यम्~\arrow याचत्~सुँ~\arrow याचत्~स्~\arrow \textcolor{red}{उगिदचां सर्वनामस्थानेऽधातोः} (पा॰सू॰~७.१.७०)~\arrow \textcolor{red}{मिदचोऽन्त्यात्परः} (पा॰सू॰~१.१.४७)~\arrow याच~नुँम्~त्~स्~\arrow याच~न्~त्~स्~\arrow \textcolor{red}{हल्ङ्याब्भ्यो दीर्घात्सुतिस्यपृक्तं हल्} (पा॰सू॰~६.१.६८)~\arrow याच~न्~त्~\arrow \textcolor{red}{संयोगान्तस्य लोपः} (पा॰सू॰~८.२.२३)~\arrow याच~न्~त्~स्~\arrow याचन्।} \textcolor{red}{तस्मै याचते} इत्याचार\-क्विबन्ताच्छतृ\-प्रत्यये \textcolor{red}{याचते}।\footnote{याचत्~\arrow प्रातिपादिक\-सञ्ज्ञा (पूर्ववत्)~\arrow विभक्ति\-कार्यम्~\arrow याचत्~ङे~\arrow याचत्~ए~\arrow याचते।} याचकवदाचरण\-कारिणेऽप्यभयं ददामि तदा याचमानाय किं दद्यामिति स्वकारुण्यात्स्वयं ध्वनयितुं स्वयं निखिल\-निगमार्णवो राघवो \textcolor{red}{याचते} इति प्रायुङ्क्त। अथवोभयपदी धातुरयम्। तथा च भगवान्पाणिनिः \textcolor{red}{टुयाचृँ याच्ञायाम्} (धा॰पा॰~८६३)। स्वरितेदयम्। एष कर्त्रभिप्राये क्रियाफल आत्मनेपदी।\footnote{\textcolor{red}{स्वरितञितः कर्त्रभिप्राये क्रियाफले} (पा॰सू॰~१.३.७२) इत्यनेन।} अन्याभिप्राये क्रियाफले परस्मैपदी।\footnote{\textcolor{red}{शेषात्कर्तरि परस्मैपदम्} (पा॰सू॰~१.३.७८) इत्यनेन।} भगवतोऽभिप्रायो यत्~– यः स्वार्थं त्यक्त्वा प्रेमभक्तये मां प्रपन्नः \textcolor{red}{तवास्मि} इति वदन् मामभयं ब्रह्मैव याचति तस्मा अहं सर्वभूतेभ्योऽभयं ददामि। अतोऽत्र परस्मैपदं ततः शतृप्रत्ययः। तस्य चतुर्थ्येकवचनरूपं \textcolor{red}{याचते}। \textcolor{red}{याचतीति याचन् तस्मै याचते}।\end{sloppypar}
\section[अभिषेकम्]{अभिषेकम्}
\centering\textcolor{blue}{लङ्काराज्याधिपत्यार्थमभिषेकं रमापतिः।\nopagebreak\\
कारयामास सचिवैर्लक्ष्मणेन विशेषतः॥}\nopagebreak\\
\raggedleft{–~अ॰रा॰~६.३.४५}\\
\begin{sloppypar}\hyphenrules{nohyphenation}\justifying\noindent\hspace{10mm} अत्र नायं भाव\-साधनोऽपि तु \textcolor{red}{पुंसि सञ्ज्ञायां घः प्रायेण} (पा॰सू॰~३.३.११८) इत्यनेन \textcolor{red}{घ}\-प्रत्ययः सञ्ज्ञायामित्यस्य प्रायिकत्वात्।\end{sloppypar}
\section[शासिता]{शासिता}
\centering\textcolor{blue}{अनुजीव्य सुदुर्बुद्धे गुरुवद्भाषसे कथम्।\nopagebreak\\
शासिताऽहं त्रिजगतां त्वं मां शिक्षन्न लज्जसे॥}\nopagebreak\\
\raggedleft{–~अ॰रा॰~६.५.२}\\
\begin{sloppypar}\hyphenrules{nohyphenation}\justifying\noindent\hspace{10mm} अत्र \textcolor{red}{शास्तीति शास्ता} इत्यदादेर्धातोः (\textcolor{red}{शासुँ अनुशिष्टौ} धा॰पा॰~१०७५) तु नेट्।\footnote{\textcolor{red}{तृंस्तृचौ शंसिक्षदादिभ्यः संज्ञायां चानिटौ} (प॰उ॰~२.९४, द॰उ॰~२.१)। \textcolor{red}{शास्ता वरुणः। प्रशास्ता । शास्ता गुरुः} (प॰उ॰श्वे॰वृ॰~२.९४)। \textcolor{red}{प्रशास्तीति प्रशास्ता। राजा गुरुर्वा} (द॰उ॰वृ॰~२.१)।} किन्तु \textcolor{red}{आगम\-शास्त्रमनित्यम्} (प॰शे॰~९३.२) इति कृत्वा सेट्।\footnote{यथा मनुस्मृतौ~– \textcolor{red}{प्रशासितारं सर्वेषामणीयांसमणोरपि} (म॰स्मृ॰~१२.१२२)। शकुन्तला\-नाटकेऽपि~– \textcolor{red}{कः पौरवे वसुमतीं शासति शासितरि दुर्विनीतानाम्} (अ॰शा॰~१.२४)। यद्वाऽसञ्ज्ञायां \textcolor{red}{तृंस्तृचौ शंसिक्षदादिभ्यः संज्ञायां चानिटौ} (प॰उ॰~२.९४, द॰उ॰~२.१) इत्यस्याप्रवृत्तौ \textcolor{red}{ण्वुल्तृचौ} (पा॰सू॰~३.१.१३३) इत्यनेन तृचि \textcolor{red}{तृन्} (पा॰सू॰~३.२.१३५) इत्यनेन तृनि वा \textcolor{red}{आर्धधातुकस्येड्वलादेः} (पा॰सू॰~७.२.३५) इत्यनेनेडागमे विभक्ति\-कार्ये \textcolor{red}{शासिता}।}\end{sloppypar}
\section[शिक्षन्]{शिक्षन्}
\centering\textcolor{blue}{अनुजीव्य सुदुर्बुद्धे गुरुवद्भाषसे कथम्।\nopagebreak\\
शासिताऽहं त्रिजगतां त्वं मां शिक्षन्न लज्जसे॥}\nopagebreak\\
\raggedleft{–~अ॰रा॰~६.५.२}\\
\begin{sloppypar}\hyphenrules{nohyphenation}\justifying\noindent\hspace{10mm} \textcolor{red}{शिक्ष्} धातुः (\textcolor{red}{शिक्षँ विद्योपादाने} धा॰पा॰~६०५) आत्मनेपदी। स च शिक्षा\-ग्रहणार्थको न तु शिक्षा\-दानार्थकः।\footnote{यथा शकुन्तला\-नाटकस्य मैथिलबङ्गीयपाठयोः~– \textcolor{red}{विवर्तित\-भ्रूरियमद्य शिक्षते भयादकामाऽपि हि दृष्टि\-विभ्रमम्} (अ॰शा॰~१.२४)। रघुवंशे च~– \textcolor{red}{अशिक्षतास्रं पितुरेव मन्त्रवत्} (र॰वं॰~३.३१)।} अत्र \textcolor{red}{शिक्षयन्} इति पाणिनीयम्। किन्त्वन्तर्भावित\-ण्यर्थत्वादनित्यमात्मनेपदम्। तस्मात् \textcolor{red}{शिक्षन्} प्रयोगोऽयं पाणिनीयोऽस्ति सर्वतः।\end{sloppypar}
\section[आप्लवन्तः]{आप्लवन्तः}
\centering\textcolor{blue}{कोटिशतयुताश्चान्ये रुरुधुर्नगरं भृशम्।\nopagebreak\\
आप्लवन्तः प्लवन्तश्च गर्जन्तश्च प्लवङ्गमाः॥}\nopagebreak\\
\raggedleft{–~अ॰रा॰~६.५.५२}\\
\begin{sloppypar}\hyphenrules{nohyphenation}\justifying\noindent\hspace{10mm} \textcolor{red}{प्लु}\-धातुः (\textcolor{red}{प्लुङ् गतौ} धा॰पा॰~९५८) आत्मनेपदी। अत्रापि \textcolor{red}{शतृ}\-प्रत्यय आत्मनेपदस्यानित्यता\-स्वीकारेण।\footnote{\textcolor{red}{व्यत्ययो बहुलम्} (पा॰सू॰~३.१.८५) इत्यनेनात्मनेपदस्य छान्दसानित्यताया परस्मैपदमिति भावः। यद्वा \textcolor{red}{अनुदात्तेत्त्व\-लक्षणमात्मने\-पदमनित्यम्} (प॰शे॰~९३.४) इतिवत् ङित्त्व\-लक्षणमात्मने\-पदमप्यनित्यम्।}\end{sloppypar}
\section[ग्रसन्ती]{ग्रसन्ती}
\centering\textcolor{blue}{ततो ददर्श हनुमान् ग्रसन्तीं मकरीं रुषा।\nopagebreak\\
दारयामास हस्ताभ्यां वदनं सा ममार हे॥}\nopagebreak\\
\raggedleft{–~अ॰रा॰~६.७.२३}\\
\begin{sloppypar}\hyphenrules{nohyphenation}\justifying\noindent\hspace{10mm} \textcolor{red}{ग्रस्‌}\-धातुः (\textcolor{red}{ग्रसुँ अदने} धा॰पा॰~६३०) आत्मनेपदी। \textcolor{red}{ग्रसमाना} इति वक्तव्ये \textcolor{red}{ग्रसन्ती} इति \textcolor{red}{शतृ}\-प्रत्ययान्तम् \textcolor{red}{अनुदात्तेत्त्व\-लक्षणमात्मने\-पदमनित्यम्} (प॰शे॰~९३.४) स्वीकृत्योक्तम्।\end{sloppypar}
\section[शासयन्तम्]{शासयन्तम्}
\centering\textcolor{blue}{पादुके ते पुरस्कृत्य शासयन्तं वसुन्धराम्।\nopagebreak\\
मन्त्रिभिः पौरमुख्यैश्च काषायाम्बरधारिभिः॥}\nopagebreak\\
\raggedleft{–~अ॰रा॰~६.१४.५३}\\
\begin{sloppypar}\hyphenrules{nohyphenation}\justifying\noindent\hspace{10mm} अत्र \textcolor{red}{शास्‌}\-धातुं (\textcolor{red}{शासुँ अनुशिष्टौ} धा॰पा॰~१०७५) स्वार्थे णिजन्तं मत्वा शतरि शपि गुणेऽयादेशे नुमि विभक्तिकार्ये च \textcolor{red}{शासयन्तम्}।\footnote{\textcolor{red}{शासुँ अनुशिष्टौ} (धा॰पा॰~१०७५)~\arrow शास्~\arrow स्वार्थे णिच्~\arrow शास्~णिच्~\arrow शास्~इ~\arrow शासि~\arrow \textcolor{red}{सनाद्यन्ता धातवः} (पा॰सू॰~३.१.३२)~\arrow धातु\-सञ्ज्ञा~\arrow \textcolor{red}{शेषात्कर्तरि परस्मैपदम्} (पा॰सू॰~१.३.७८)~\arrow \textcolor{red}{वर्तमाने लट्} (पा॰सू॰~३.२.१२३)~\arrow शासि~लट्~\arrow \textcolor{red}{लटः शतृशानचावप्रथमा\-समानाधिकरणे} (पा॰सू॰~३.२.१२४)~\arrow शासि~शतृँ~\arrow शासि~अत्~\arrow \textcolor{red}{सार्वधातुकार्ध\-धातुकयोः} (पा॰सू॰~७.३.८४)~\arrow शासे~अत्~\arrow \textcolor{red}{एचोऽयवायावः} (पा॰सू॰~६.१.७८)~\arrow शासय्~अत्~\arrow शासयत्~\arrow \textcolor{red}{कृत्तद्धित\-समासाश्च} (पा॰सू॰~१.२.४६)~\arrow प्रातिपादिक\-सञ्ज्ञा~\arrow विभक्ति\-कार्यम्~\arrow शासयत्~अम्~\arrow \textcolor{red}{ उगिदचां सर्वनामस्थानेऽधातोः} (पा॰सू॰~७.१.७०)~\arrow \textcolor{red}{मिदचोऽन्त्यात्परः} (पा॰सू॰~१.१.४७)~\arrow शासय~नुँम्~त्~अम्~\arrow शासय~न्~त्~अम्~\arrow शासयन्तम्।}\end{sloppypar}
\section[गायमानाः]{गायमानाः}
\centering\textcolor{blue}{पश्चाद्दुरात्मना राम रावणेनाभिविद्रुताः।\nopagebreak\\
तमेव गायमानाश्च तदाराधनतत्पराः॥}\nopagebreak\\
\raggedleft{–~अ॰रा॰~६.१५.६९}\\
\begin{sloppypar}\hyphenrules{nohyphenation}\justifying\noindent\hspace{10mm} \textcolor{red}{गै}\-धातुः (\textcolor{red}{गै शब्दे} धा॰पा॰~९१८) परस्मैपदी। \textcolor{red}{गायन्तीति गायन्तः} इति पाणिनीयम्। किन्तु \textcolor{red}{कर्तरि कर्म\-व्यतिहारे} (पा॰सू॰~१.३.१४) इत्यात्मनेपदे \textcolor{red}{गायन्त इति गायमानाः}। न च कर्म\-व्यतिहार\-द्योतकः शब्दो नास्तीति वाच्यम्। \textcolor{red}{व्यतिगायन्ते} इति प्रयोगः। \textcolor{red}{विनाऽपि प्रत्ययं पूर्वोत्तर\-पद\-लोपो वक्तव्यः} (वा॰~५.३.८३) इत्यनेन \textcolor{red}{व्यति} इत्यस्य लोपः। ततः शानचि \textcolor{red}{गायमानाः}। \textcolor{red}{अयोग्यं गायं कुर्वाणाः} इति भावः।\end{sloppypar}
\vspace{2mm}
\centering ॥ इति युद्धकाण्डीयप्रयोगाणां विमर्शः ॥\nopagebreak\\
\vspace{4mm}
\pdfbookmark[2]{उत्तरकाण्डम्}{Chap2Part2Kanda7}
\phantomsection
\addtocontents{toc}{\protect\setcounter{tocdepth}{2}}
\addcontentsline{toc}{subsection}{उत्तरकाण्डीयप्रयोगाणां विमर्शः}
\addtocontents{toc}{\protect\setcounter{tocdepth}{0}}
\centering ॥ अथोत्तरकाण्डीयप्रयोगाणां विमर्शः ॥\nopagebreak\\
\section[पौत्रान्]{पौत्रान्}
\centering\textcolor{blue}{सुमाली वरलब्धांस्ताञ्ज्ञात्वा पौत्रान् निशाचरान्।\nopagebreak\\
पातालान्निर्भयः प्रायात्प्रहस्तादिभिरन्वितः॥}\nopagebreak\\
\raggedleft{–~अ॰रा॰~७.२.२४}\\
\begin{sloppypar}\hyphenrules{nohyphenation}\justifying\noindent\hspace{10mm} \textcolor{red}{पुत्र्या अपत्यानि पुमांसः} इति विग्रहे \textcolor{red}{स्त्रीभ्यो ढक्} (पा॰सू॰~४.१.१२०) इत्यनेन \textcolor{red}{ढक्} प्रत्यये \textcolor{red}{आयनेयीनीयियः
फढखच्छघां प्रत्ययादीनाम्} (पा॰सू॰~७.१.२) इत्यनेन \textcolor{red}{एय्} आदेशे \textcolor{red}{किति च} (पा॰सू॰~७.२.११८) इत्यनेन वृद्धौ विभक्ति\-कार्ये \textcolor{red}{पौत्रेयान्} इति पाणिनीयम्। किन्तु \textcolor{red}{पुत्र्या इमे} इति विग्रहे \textcolor{red}{तस्येदम्} (पा॰सू॰~४.३.१२०) इत्यनेन \textcolor{red}{अण्}। \textcolor{red}{यस्येति च} (पा॰सू॰~६.४.१४८) इत्यनेन भत्वादीकार\-लोपे विभक्तिकार्ये \textcolor{red}{पौत्रान्}।\end{sloppypar}
\section[विकल्पोज्झितः]{विकल्पोज्झितः}
\centering\textcolor{blue}{राम त्वं परमेश्वरोऽसि सकलं जानासि विज्ञानदृग्\nopagebreak\\
भूतं भव्यमिदं त्रिकालकलनासाक्षी विकल्पोज्झितः।\nopagebreak\\
भक्तानामनुवर्तनाय सकलां कुर्वन् क्रियासंहतिं\nopagebreak\\
त्वं शृण्वन्मनुजाकृतिर्मुनिवचो भासीश लोकार्चितः॥}\nopagebreak\\
\raggedleft{–~अ॰रा॰~७.४.१२}\\
\begin{sloppypar}\hyphenrules{nohyphenation}\justifying\noindent\hspace{10mm} \textcolor{red}{उज्झितो विकल्पो येन} इति विकल्पे \textcolor{red}{सप्तमीविशेषणे बहुव्रीहौ} (पा॰सू॰~२.२.३५) इत्यनेन विशेषणस्य पूर्वं प्रयोक्तव्ये विकल्प\-शब्दस्य प्रयोगो नापाणिनीयः। पूर्व\-निपात\-प्रकरणस्यानित्यत्वात्। \textcolor{red}{समुद्राभ्राद्घः} (पा॰सू॰~४.४.११८) इत्यत्र समुद्र\-शब्दस्य पूर्व\-प्रयोगात्।\footnote{\textcolor{red}{समुद्राभ्रात्} इत्यत्र द्वन्द्व\-समासे \textcolor{red}{अल्पाच्तरम्} (पा॰सू॰~२.२.३४) इत्यनेन \textcolor{red}{अभ्र}\-शब्दस्य पूर्व\-निपाते \textcolor{red}{अभ्रसमुद्रात्} इत्यनेन भवितव्यमासीत्। \textcolor{red}{लक्षण\-हेत्वोः क्रियायाः} (पा॰सू॰~३.२.१२६) इतिवदिदं सूत्रमपि पूर्व\-निपात\-प्रकरणस्यानित्यत्वं ज्ञापयति। \textcolor{red}{लक्षण\-हेत्वोरिति निर्देशः पूर्वनिपात\-व्यभिचार\-लिङ्गम्} (का॰वृ॰~३.२.१२६)।}\end{sloppypar}
\section[पूज्य]{पूज्य}
\centering\textcolor{blue}{तासां भावानुगं राम प्रसादं कर्तुमर्हसि।\nopagebreak\\
श्रुत्वा वसिष्ठवचनं ताः समुत्थाप्य पूज्य च॥}\nopagebreak\\
\raggedleft{–~अ॰रा॰~७.९.१०}\\
\begin{sloppypar}\hyphenrules{nohyphenation}\justifying\noindent\hspace{10mm} अत्र साकेत\-गमनाय कृत\-सङ्कल्पानां प्रजानां विषये वसिष्ठस्य प्रार्थनं\footnote{\textcolor{red}{प्रार्थना प्रार्थनम्} इति द्वावपि शब्दौ भावे। प्र~\textcolor{red}{अर्थँ उपयाच्ञायाम्} (धा॰पा॰~१९०५)~\arrow प्र~अर्थ्~\arrow \textcolor{red}{सत्याप\-पाश\-रूप\-वीणा\-तूल\-श्लोक\-सेना\-लोम\-त्वच\-वर्म\-वर्ण\-चूर्ण\-चुरादिभ्यो णिच्} (पा॰सू॰~३.१.२५)~\arrow प्र~अर्थ्~णिच्~\arrow प्र~अर्थ्~इ~\arrow प्र~अर्थि~\arrow~\arrow \textcolor{red}{सनाद्यन्ता धातवः} (पा॰सू॰~३.१.३२)~\arrow धातु\-सञ्ज्ञा~\arrow \textcolor{red}{ण्यासश्रन्थो युच्} (पा॰सू॰~३.३.१०७)~\arrow प्र~अर्थि~युच्~\arrow प्र~अर्थि~यु~\arrow \textcolor{red}{णेरनिटि} (पा॰सू॰~६.४.५१)~\arrow प्र~अर्थ्~यु~\arrow \textcolor{red}{युवोरनाकौ} (पा॰सू॰~७.१.१)~\arrow प्र~अर्थ्~अन~\arrow \textcolor{red}{अजाद्यतष्टाप्‌} (पा॰सू॰~४.१.४)~\arrow प्र~अर्थ्~अन~टाप्~\arrow प्र~अर्थ्~अन~आ~\arrow \textcolor{red}{अकः सवर्णे दीर्घः} (पा॰सू॰~६.१.१०१)~\arrow प्र~अर्थना~\arrow \textcolor{red}{अकः सवर्णे दीर्घः} (पा॰सू॰~६.१.१०१)~\arrow प्रार्थना~\arrow विभक्तिकार्यम्~\arrow प्रार्थना। प्र~अर्थि (पूर्ववत्)~\arrow \textcolor{red}{ल्युट् च} (पा॰सू॰~३.३.११५)~\arrow प्र~अर्थि~ल्युट्~\arrow प्र~अर्थि~यु~\arrow \textcolor{red}{णेरनिटि} (पा॰सू॰~६.४.५१)~\arrow प्र~अर्थ्~यु~\arrow \textcolor{red}{युवोरनाकौ} (पा॰सू॰~७.१.१)~\arrow प्र~अर्थ्~अन~\arrow \textcolor{red}{अकः सवर्णे दीर्घः} (पा॰सू॰~६.१.१०१)~\arrow प्रार्थन~\arrow विभक्तिकार्यम्~\arrow प्रार्थनम्।} श्रुत्वा करुणा\-वरुणालयो भगवाञ्छ्रीरामोऽनुगन्तुं ता आज्ञप्तवान्। अत्र \textcolor{red}{पूज्य} इति प्रयुक्तम्। ल्यप्प्रत्ययः समासं विना सम्भवो नहि अतः \textcolor{red}{पूज्य} इति कथं पाणिनीयमिति चेत्। \textcolor{red}{सम्पूज्य} इति प्रयोगः। अस्य च \textcolor{red}{विनाऽपि प्रत्ययं पूर्वोत्तर\-पद\-लोपो वक्तव्यः} (वा॰~५.३.८३) इति वार्त्तिकेन लोपे \textcolor{red}{जात\-संस्कारो न निवर्तते} इति परिभाषया ल्यम्निवृत्त्यभावे \textcolor{red}{पूज्य} इति पाणिनीयमेव।\end{sloppypar}
\begin{sloppypar}\hyphenrules{nohyphenation}\justifying\noindent\hspace{10mm} \end{sloppypar}
\vspace{2mm}
\centering ॥ इत्युत्तरकाण्डीयप्रयोगाणां विमर्शः ॥\nopagebreak\\
\vspace{4mm}
\centering इत्यध्यात्म\-रामायणेऽपाणिनीय\-प्रयोगाणां\-विमर्श\-नामके शोध\-प्रबन्धे द्वितीयाध्याये द्वितीय\-परिच्छेदः।\nopagebreak\\
\vspace{4mm}
\centering इत्यध्यात्म\-रामायणेऽपाणिनीय\-प्रयोगाणां\-विमर्श\-नामके शोध\-प्रबन्धे द्वितीयोऽध्यायः।


% Nityanand Misra: LaTeX code to typeset a book in Sanskrit
% Copyright (C) 2016 Nityanand Misra
%
% This program is free software: you can redistribute it and/or modify it under
% the terms of the GNU General Public License as published by the Free Software
% Foundation, either version 3 of the License, or (at your option) any later
% version.
%
% This program is distributed in the hope that it will be useful, but WITHOUT
% ANY WARRANTY; without even the implied warranty of  MERCHANTABILITY or FITNESS
% FOR A PARTICULAR PURPOSE. See the GNU General Public License for more details.
%
% You should have received a copy of the GNU General Public License along with
% this program.  If not, see <http://www.gnu.org/licenses/>.

\renewcommand\chaptername{अथ तृतीयोऽध्यायः}
\chapter[\texorpdfstring{धातुप्रकरणम्}{तृतीयोऽध्यायः}]{धातुप्रकरणम्}
\vspace{-5mm}
\fontsize{16}{24}\selectfont\centering\hyphenrules{nohyphenation}\textcolor{blue}{प्रणम्य सीतापतिपादपद्मं गौरीं गिरीशं किल धातुशब्दान्।\nopagebreak\\
अपाणिनीयांश्च विमर्शयेऽत्र\footnote{\textcolor{red}{विमर्शये} इत्यत्र स्वार्थे णिच्। \textcolor{red}{निवृत्त\-प्रेषणाद्धातोः प्राकृतेऽर्थे णिजुच्यते} (वा॰प॰~३.७.६०)। स्वान्तःसुखाय विमृशामीति कर्त्रभिप्रायं ध्वनयितुमात्मने\-पदप्रयोगः। \textcolor{red}{णिचश्च} (पा॰सू॰~१.३.७४) इत्यनेन। वि~\textcolor{red}{मृशँ आमर्शने} (धा॰पा॰~१४२५)~\arrow वि~मृश्~\arrow स्वार्थे णिच्~\arrow वि~मृश्~णिच्~\arrow वि~मृश्~इ~\arrow \textcolor{red}{पुगन्त\-लघूपधस्य च} (पा॰सू॰~७.३.८६)~\arrow \textcolor{red}{उरण् रपरः} (पा॰सू॰~१.१.५१)~\arrow वि~मर्श्~इ~\arrow विमर्शि~\arrow \textcolor{red}{सनाद्यन्ता धातवः} (पा॰सू॰~३.१.३२)~\arrow धातु\-सञ्ज्ञा~\arrow \textcolor{red}{णिचश्च} (पा॰सू॰~१.३.७४)~\arrow \textcolor{red}{वर्तमाने लट्} (पा॰सू॰~३.२.१२३)~\arrow विमर्शि~इट्~\arrow विमर्शि~इ~\arrow \textcolor{red}{कर्तरि शप्‌} (पा॰सू॰~३.१.६८)~\arrow विमर्शि~शप्~इ~\arrow विमर्शि~अ~इ~\arrow \textcolor{red}{सार्वधातुकार्ध\-धातुकयोः} (पा॰सू॰~७.३.८४)~\arrow विमर्शे~अ~इ~\arrow \textcolor{red}{एचोऽयवायावः} (पा॰सू॰~६.१.७८)~\arrow विमर्शय्~अ~इ~\arrow \textcolor{red}{आद्गुणः} (पा॰सू॰~६.१.८७)~\arrow विमर्शय्~ए~\arrow विमर्शये। यद्वा \textcolor{red}{विमर्शं कुर्वे} इति विग्रहे \textcolor{red}{विमर्शये}। अत्रापि कर्त्रभिप्राये \textcolor{red}{णिचश्च} (पा॰सू॰~१.३.७४) इत्यनेनाऽत्मने\-पदम्। विमर्श~\arrow \textcolor{red}{तत्करोति तदाचष्टे} (धा॰पा॰ ग॰सू॰)~\arrow विमर्श~णिच्~\arrow विमर्श~इ~\arrow \textcolor{red}{णाविष्ठवत्प्राति\-पदिकस्य पुंवद्भाव\-रभाव\-टिलोप\-यणादि\-परार्थम्} (वा॰~६.४.४८)~\arrow विमर्श्~इ~\arrow विमर्शि~\arrow \textcolor{red}{सनाद्यन्ता धातवः} (पा॰सू॰~३.१.३२)~\arrow धातु\-सञ्ज्ञा~\arrow \textcolor{red}{णिचश्च} (पा॰सू॰~१.३.७४)~\arrow \textcolor{red}{वर्तमाने लट्} (पा॰सू॰~३.२.१२३)~\arrow विमर्शि~इट्~\arrow शेषं पूर्ववत्।} अध्यात्मरामायणमध्यगान्वै॥}\nopagebreak\\
\vspace{4mm}
\pdfbookmark[1]{प्रथमः परिच्छेदः}{Chap3Part1}
\phantomsection
\addtocontents{toc}{\protect\setcounter{tocdepth}{1}}
\addcontentsline{toc}{section}{प्रथमः परिच्छेदः}
\addtocontents{toc}{\protect\setcounter{tocdepth}{0}}
\centering ॥ अथ तृतीयाध्याये प्रथमः परिच्छेदः ॥\nopagebreak\\
\vspace{4mm}
\pdfbookmark[2]{बालकाण्डम्}{Chap3Part1Kanda1}
\phantomsection
\addtocontents{toc}{\protect\setcounter{tocdepth}{2}}
\addcontentsline{toc}{subsection}{बालकाण्डीयप्रयोगाणां विमर्शः}
\addtocontents{toc}{\protect\setcounter{tocdepth}{0}}
\centering ॥ अथ बालकाण्डीयप्रयोगाणां विमर्शः ॥\nopagebreak\\
\fontsize{14}{21}\selectfont
\section[पठन्ति शृण्वन्ति यान्ति]{पठन्ति शृण्वन्ति यान्ति}
\centering\textcolor{blue}{पठन्ति ये नित्यमनन्यचेतसः शृण्वन्ति चाध्यात्मिकसञ्ज्ञितं शुभम्।\nopagebreak\\
रामायणं सर्वपुराणसम्मतं निर्धूतपापा हरिमेव यान्ति ते॥}\nopagebreak\\
\raggedleft{–~अ॰रा॰~१.१.३}\\
\fontsize{14}{21}\selectfont\begin{sloppypar}\hyphenrules{nohyphenation}\justifying\noindent\hspace{10mm} साम्प्रतमहं तृतीयाध्याये धातु\-सम्बन्धिनोऽपाणिनीय\-प्रयोगान् विमर्शये। प्रथमे सर्गे बाल\-काण्डस्य फल\-श्रुतिं वर्णयन्नाह ग्रन्थ\-कृद्यद् \textcolor{red}{य इमामध्यात्म\-रामायण\-संहितां पठन्ति ते निर्धूत\-पापा भगवन्तमेव प्राप्नुवन्ति}। आरभ्यमाण\-ग्रन्थस्य फल\-श्रुतिरियम्। पूर्णतामपि न गतेऽस्मिन् वर्तमान\-कालीन\-पाठः कथं सङ्गंस्यत इति चेत्। \textcolor{red}{पठन्ति} इति वर्तमान\-कालिक\-प्रयोगो भविष्यत्तात्पर्य\-वाचकः।\footnote{\textcolor{red}{पठँ व्यक्तायां वाचि} (धा॰पा॰~३३०)~\arrow पठ्~\arrow \textcolor{red}{शेषात्कर्तरि परस्मैपदम्} (पा॰सू॰~१.३.७८)~\arrow \textcolor{red}{वर्तमान\-सामीप्ये वर्तमानवद्वा} (पा॰सू॰~३.३.१३१)~\arrow \textcolor{red}{वर्तमाने लट्} (पा॰सू॰~३.२.१२३)~\arrow पठ्~लट्~\arrow पठ्~झि~\arrow \textcolor{red}{झोऽन्तः} (पा॰सू॰~७.१.३)~\arrow पठ्~अन्ति~\arrow \textcolor{red}{कर्तरि शप्} (पा॰सू॰~३.१.६८)~\arrow पठ्~शप्~अन्ति~\arrow पठ्~अ~अन्ति~\arrow \textcolor{red}{अतो गुणे} (पा॰सू॰~६.१.९७)~\arrow पठ्~अन्ति~\arrow पठन्ति।} एवं च \textcolor{red}{वर्तमान\-सामीप्ये वर्तमानवद्वा} (पा॰सू॰~३.३.१३१) इत्यनेन वर्तमान\-समीप\-भविष्यत्काले विकल्पेन वर्तमान\-काल\-निमित्त\-कार्याण्यर्थाद्वर्तमान\-वद्भावः। यथा च \textcolor{red}{पठिष्यन्ति}\footnote{\textcolor{red}{पठँ व्यक्तायां वाचि} (धा॰पा॰~३३०)~\arrow पठ्~\arrow \textcolor{red}{शेषात्कर्तरि परस्मैपदम्} (पा॰सू॰~१.३.७८)~\arrow \textcolor{red}{लृट् शेषे च} (पा॰सू॰~३.३.१३)~\arrow पठ्~लृट्~\arrow पठ्~झि~\arrow \textcolor{red}{झोऽन्तः} (पा॰सू॰~७.१.३)~\arrow पठ्~अन्ति~\arrow \textcolor{red}{स्यतासी लृलुटोः} (पा॰सू॰~३.१.३३)~\arrow पठ्~स्य~अन्ति~\arrow \textcolor{red}{आर्धधातुकस्येड्वलादेः} (पा॰सू॰~७.२.३५)~\arrow पठ्~इट्~स्य~अन्ति~\arrow पठ्~इ~स्य~अन्ति~\arrow \textcolor{red}{आदेश\-प्रत्यययोः} (पा॰सू॰~८.३.५९)~\arrow पठ्~इ~ष्य~अन्ति~\arrow \textcolor{red}{अतो गुणे} (पा॰सू॰~६.१.९७)~\arrow पठ्~इ~ष्यन्ति~\arrow पठिष्यन्ति।} इत्यर्थे पठन्ति। एवं \textcolor{red}{नाना\-कर्तृकाध्यात्म\-रामायण\-कर्मक\-भविष्यत्कालावच्छिन्न\-वर्तमान\-कालाभासिक उच्चारणानुकूलो व्यापारः} इति शाब्द\-बोधः। धातोः खलु व्यापार एव शक्तिरेवं \textcolor{red}{तिङ्} प्रत्ययस्याऽश्रये। व्यापार\-वाचकत्वे प्रधानतया व्यापारः प्रधानमेवं तिङर्थो विशेषणम्। अर्थाद्वैयाकरणा व्यापार\-मुख्य\-विशेष्यकं शाब्द\-बोधमङ्गीकुर्वन्ति। यद्यपि मीमांसका अपि भावना\-मुख्य\-विशेष्यकं शाब्द\-बोधं मन्यन्ते किन्तु कुत्रचिदप\-सिद्धान्तितमपि तैः। नैयायिकाः प्रथमान्त\-मुख्य\-विशेष्यकं शाब्द\-बोधं मन्यन्ते। किन्तु \textcolor{red}{पश्य मृगो धावति} इत्यत्र भाष्य\-सम्मतैक\-वाक्यता भग्ना भवति प्रथमान्त\-मुख्य\-विशेष्यके शाब्द\-बोधे स्वीकृते।\footnote{\textcolor{red}{क्रियाऽपि क्रिययेप्सिततमा भवति। कया क्रियया। सन्दर्शनक्रियया वा प्रार्थयतिक्रियया वाऽध्यवस्यतिक्रियया वा। इह य एष मनुष्यः प्रेक्षापूर्वकारी भवति स बुद्ध्या तावत्कञ्चिदर्थं सम्पश्यति सन्दृष्टे प्रार्थना प्रार्थनायामध्यवसायोऽध्यवसाय आरम्भ आरम्भे निर्वृत्तिर्निर्वृत्तौ फलावाप्तिः। एवं क्रियाऽपि कृत्रिमं कर्म} (भा॰पा॰सू॰~१.४.३२)।} \textcolor{red}{धावनानुकूल\-कृतिमान् मृगः}। \textcolor{red}{मृग\-कर्तृक\-धावन\-कर्मक\-दर्शनानुकूल\-कृतिमांस्त्वम्}। अस्मन्मते तु \textcolor{red}{मृग\-कर्तृक\-धावनानुकूल\-व्यापार\-कर्मक\-त्वत्कर्तृक\-दर्शनानुकूल\-व्यापारः}। इत्थं तिङर्थाश्चत्वारः कर्तृ\-कर्म\-सङ्ख्या\-कालाः। कर्तुर्व्यापारेऽन्वयः कर्मणश्च फले सङ्ख्यायाश्च कर्म\-प्रत्यये कर्मणि कर्तृ\-प्रत्यये च कर्तरि। कालस्य च व्यापारेऽन्वयः। \textcolor{red}{वर्तमाने लट्} (पा॰सू॰~३.२.१२३) इत्यादि\-सूत्र\-निर्देशात्। इत्थं तत्र कारिका~–\end{sloppypar}
\centering\textcolor{red}{फलव्यापारयोर्धातुराश्रये तु तिङः स्मृताः।\nopagebreak\\
फले प्रधानं व्यापारस्तिङर्थस्तु विशेषणम्॥}\nopagebreak\\
\raggedleft{–~वै॰सि॰का॰~१.२}\\
\fontsize{14}{21}\selectfont\begin{sloppypar}\hyphenrules{nohyphenation}\justifying\noindent फलस्य व्यापारे स्वानुकूलत्व\-सम्बन्धाश्रयः। एवं \textcolor{red}{भक्ताभिन्नैक\-कर्तृक\-भविष्यत्कालावच्छिन्न\-वर्तमान\-कालाभासिक\-स्पष्टोच्चारणानुकूल\-व्यापारः}। प्राचीन\-नवीनयोरिदमन्तरम्। प्राचीनास्तूभयत्र समान\-व्यवस्थां मन्यन्ते। नवीनास्तु कर्तृ\-भाव\-वाच्ययोः फलानुकूल\-व्यापार एवं कर्म\-वाच्ये व्यापार\-विशिष्ट\-फलम्।\footnote{\textcolor{red}{इति कथयन्ति} इति शेषः।} यथा \textcolor{red}{रामेणायोध्या गम्यते}~– \textcolor{red}{रामाभिन्नैक\-कर्तृक\-वर्तमानकालावच्छिन्न\-व्यापार\-जन्यं गृह\-निष्ठोत्तर\-देश\-संयोग\-रूपं फलम्} अयमत्र विवेकः।\footnote{नवीन\-वैयाकरण\-शाब्दबोधोऽयम्।} किं फले व्यापारे च धातोः पृथक्शक्तिरुताहो समुदिता चेत्। अनुकूलत्व\-सम्बन्धेन फलस्य व्यापारेऽन्वयः। तदा \textcolor{red}{पदार्थः पदार्थेनान्वेति न तु तदेकदेशेन} इति नियमेन कथमत्रान्वयः। अतो विशिष्टे शक्तिः।\footnote{फल\-विशिष्ट\-व्यापारे व्यापार\-विशिष्ट\-फले च शक्तिरिति भावः। \textcolor{red}{तस्मात्फलावच्छिन्ने व्यापारे व्यापारावच्छिन्ने फले च धातूनां शक्तिः कर्तृ\-कर्मार्थक\-तत्तत्प्रत्यय\-समभि\-व्याहारश्च ततद्बोधे नियामक इत्याहुः} (प॰ल॰म॰~४७)।} गुरु\-चरणास्त्वेक\-वृन्तावलम्बि\-फलद्वयवदेकस्मिन् धातावेव फल\-व्यापारयोः शक्तिः।\footnote{\textcolor{red}{मम त्वेक\-वृन्ताव\-लम्बि\-फल\-द्वय\-वदुभयांश एका खण्डशश्शक्तिरिति न शक्ति\-द्वय\-कल्पनं न वा बोध\-जनकत्व\-सम्बन्ध\-द्वय\-कल्पनम्} (प॰ल॰म॰ ज्यो॰टी॰~४७) इति प्रणेतॄणां गुरुचरणाः कालिका\-प्रसाद\-शुक्ल\-वर्याः परम\-लघु\-मञ्जूषाया ज्योत्स्ना\-टीकायाम्।} विस्तार\-भयाद्विरम्यते। एवं \textcolor{red}{भक्ताभिन्नैक\-कर्तृक\-भविष्यत्कालिक\-वर्तमान\-कालाभासिक\-स्पष्टोच्चारणानुकूल\-व्यापारः} इति शाब्द\-बोधः।\footnote{एवमेव \textcolor{red}{शृण्वन्ति} इति प्रयोगः \textcolor{red}{श्रोष्यन्ति} इत्यर्थे \textcolor{red}{यान्ति} इति च \textcolor{red}{यास्यन्ति} इत्यर्थे। उभयत्र वर्तमानवद्भावः। \textcolor{red}{श्रु श्रवणे} (धा॰पा॰~९४२)~\arrow \textcolor{red}{शेषात्कर्तरि परस्मैपदम्} (पा॰सू॰~१.३.७८)~\arrow \textcolor{red}{वर्तमान\-सामीप्ये वर्तमानवद्वा} (पा॰सू॰~३.३.१३१)~\arrow \textcolor{red}{वर्तमाने लट्} (पा॰सू॰~३.२.१२३)~\arrow श्रु~लट्~\arrow श्रु~झि~\arrow \textcolor{red}{झोऽन्तः} (पा॰सू॰~७.१.३)~\arrow श्रु~अन्ति~\arrow \textcolor{red}{श्रुवः शृ च} (पा॰सू॰~३.१.७४)~\arrow शृ~श्नु~अन्ति~\arrow शृ~नु~अन्ति~\arrow \textcolor{red}{सार्वधातुकमपित्} (पा॰सू॰~१.२.४)~\arrow ङित्त्वम्~\arrow \textcolor{red}{ग्क्ङिति च} (पा॰सू॰~१.१.५)~\arrow सार्वधातुक\-गुण\-निषेधः~\arrow \textcolor{red}{रषाभ्यां णत्व ऋकारग्रहणम्} (वा॰~८.४.१)~\arrow शृ~णु~अन्ति~\arrow \textcolor{red}{हुश्नुवोः सार्वधातुके} (पा॰सू॰~६.४.८७)~\arrow शृ~ण्व्~अन्ति~\arrow शृण्वन्ति। \textcolor{red}{या प्रापणे} (धा॰पा॰~१०४९)~\arrow या~\arrow \textcolor{red}{शेषात्कर्तरि परस्मैपदम्} (पा॰सू॰~१.३.७८)~\arrow \textcolor{red}{वर्तमान\-सामीप्ये वर्तमानवद्वा} (पा॰सू॰~३.३.१३१)~\arrow \textcolor{red}{वर्तमाने लट्} (पा॰सू॰~३.२.१२३)~\arrow या~लट्~\arrow या~झि~\arrow \textcolor{red}{झोऽन्तः} (पा॰सू॰~७.१.३)~\arrow या~अन्ति~\arrow \textcolor{red}{कर्तरि शप्‌} (पा॰सू॰~३.१.६८)~\arrow या~शप्~अन्ति~\arrow \textcolor{red}{अदिप्रभृतिभ्यः शपः} (पा॰सू॰~२.४.७२)~\arrow या~अन्ति~\arrow \textcolor{red}{अकः सवर्णे दीर्घः} (पा॰सू॰~६.१.१०१)~\arrow यान्ति। \textcolor{red}{श्रु श्रवणे} (धा॰पा॰~९४२)~\arrow \textcolor{red}{शेषात्कर्तरि परस्मैपदम्} (पा॰सू॰~१.३.७८)~\arrow \textcolor{red}{लृट् शेषे च} (पा॰सू॰~३.३.१३)~\arrow श्रु~लृट्~\arrow श्रु~झि~\arrow \textcolor{red}{झोऽन्तः} (पा॰सू॰~७.१.३)~\arrow श्रु~अन्ति~\arrow \textcolor{red}{स्यतासी लृलुटोः} (पा॰सू॰~३.१.३३)~\arrow श्रु~स्य~अन्ति~\arrow \textcolor{red}{एकाच उपदेशेऽनुदात्तात्‌} (पा॰सू॰~७.२.१०)~\arrow इडागम\-निषेधः~\arrow \textcolor{red}{सार्वधातुकार्धधातुकयोः} (पा॰सू॰~७.३.८४)~\arrow श्रो~स्य~अन्ति~\arrow \textcolor{red}{आदेश\-प्रत्यययोः} (पा॰सू॰~८.३.५९)~\arrow श्रो~ष्य~अन्ति~\arrow \textcolor{red}{अतो गुणे} (पा॰सू॰~६.१.९७)~\arrow श्रो~ष्यन्ति~\arrow श्रोष्यन्ति। \textcolor{red}{या प्रापणे} (धा॰पा॰~१०४९)~\arrow \textcolor{red}{शेषात्कर्तरि परस्मैपदम्} (पा॰सू॰~१.३.७८)~\arrow \textcolor{red}{लृट् शेषे च} (पा॰सू॰~३.३.१३)~\arrow या~लृट्~\arrow या~झि~\arrow \textcolor{red}{झोऽन्तः} (पा॰सू॰~७.१.३)~\arrow या~अन्ति~\arrow \textcolor{red}{स्यतासी लृलुटोः} (पा॰सू॰~३.१.३३)~\arrow या~स्य~अन्ति~\arrow \textcolor{red}{अतो गुणे} (पा॰सू॰~६.१.९७)~\arrow या~स्यन्ति~\arrow यास्यन्ति।}\end{sloppypar}
\section[लभेत्]{लभेत्}
\centering\textcolor{blue}{अध्यात्मरामायणमेव नित्यं पठेद्यदीच्छेद्भवबन्धमुक्तिम्।\nopagebreak\\
गवां सहस्रायुतकोटिदानात्फलं लभेद्यः शृणुयात्स नित्यम्॥}\nopagebreak\\
\raggedleft{–~अ॰रा॰~१.१.४}\\
\fontsize{14}{21}\selectfont\begin{sloppypar}\hyphenrules{nohyphenation}\justifying\noindent\hspace{10mm} अत्र फल\-श्रुतावेव \textcolor{red}{लभेत्} इति हलन्त\-प्रयोगः। \textcolor{red}{लभ्} धातुः (\textcolor{red}{डुलभँष् प्राप्तौ} धा॰पा॰~९७५) आत्मनेपदी। \textcolor{red}{विधि\-निमन्त्रणामन्त्रणाधीष्ट\-सम्प्रश्न\-प्रार्थनेषु लिङ्} (पा॰सू॰~३.३.१६१) इत्यनेन लिङ्लकारे \textcolor{red}{लिङः सीयुट्} (पा॰सू॰~३.४.१०२) इत्यनेन सीयुट्यनु\-बन्धलोपे सकारस्य च लोपे गुणे यकार\-लोपे \textcolor{red}{लभेत} इत्येव पाणिनीयम्।\footnote{\textcolor{red}{डुलभँष् प्राप्तौ} (धा॰पा॰~९७५)~\arrow लभ्~\arrow \textcolor{red}{अनुदात्तङित आत्मने\-पदम्} (पा॰सू॰~१.३.१२)~\arrow \textcolor{red}{विधि\-निमन्‍त्रणामन्‍त्रणाधीष्‍ट\-सम्प्रश्‍न\-प्रार्थनेषु लिङ्} (पा॰सू॰~३.३.१६१)~\arrow लभ्~लिङ्~\arrow लभ्~त~\arrow \textcolor{red}{कर्तरि शप्} (पा॰सू॰~३.१.६८)~\arrow लभ्~शप्~त~\arrow लभ्~अ~त~\arrow \textcolor{red}{लिङः सीयुट्} (पा॰सू॰~३.४.१०२)~\arrow लभ्~अ~सीयुँट्~त~\arrow लभ्~अ~सीय्~त~\arrow \textcolor{red}{सुट् तिथोः} (पा॰सू॰~३.४.१०७)~\arrow \textcolor{red}{आद्यन्तौ टकितौ} (पा॰सू॰~१.१.४६)~\arrow लभ्~अ~सीय्~सुँट्~त~\arrow लभ्~अ~सीय्~स्~त~\arrow \textcolor{red}{लिङः सलोपोऽनन्त्यस्य} (पा॰सू॰~७.२.७९)~\arrow लभ्~अ~ईय्~त~\arrow \textcolor{red}{लोपो व्योर्वलि} (पा॰सू॰~६.१.६६)~\arrow लभ्~अ~ई~त~\arrow \textcolor{red}{आद्गुणः} (पा॰सू॰~६.१.८७)~\arrow लभ्~ए~त~\arrow लभेत।} \textcolor{red}{लभेत्} इति कथमिति चेत्। \textcolor{red}{अनुदात्तेत्त्व\-लक्षणमात्मने\-पदमनित्यम्} (प॰शे॰~९३.४) इति वचनेनात्राऽत्मने\-पदाभावे परस्मैपद उक्तं रूपम्।\footnote{\textcolor{red}{डुलभँष् प्राप्तौ} (धा॰पा॰~९७५)~\arrow लभ्~\arrow \textcolor{red}{अनुदात्तेत्त्व\-लक्षणमात्मने\-पदमनित्यम्} (प॰शे॰~९३.४)~\arrow \textcolor{red}{शेषात्कर्तरि परस्मैपदम्} (पा॰सू॰~१.३.७८)~\arrow \textcolor{red}{विधि\-निमन्‍त्रणामन्‍त्रणाधीष्‍ट\-सम्प्रश्‍न\-प्रार्थनेषु लिङ्} (पा॰सू॰~३.३.१६१)~\arrow लभ्~लिङ्~\arrow लभ्~तिप्~\arrow लभ्~ति~\arrow \textcolor{red}{कर्तरि शप्} (पा॰सू॰~३.१.६८)~\arrow लभ्~शप्~ति~\arrow लभ्~अ~ति~\arrow \textcolor{red}{यासुट् परस्मैपदेषूदात्तो ङिच्च} (पा॰सू॰~३.४.१०३)~\arrow \textcolor{red}{आद्यन्तौ टकितौ} (पा॰सू॰~१.१.४६)~\arrow लभ्~अ~यासुँट्~ति~\arrow लभ्~अ~यास्~ति~\arrow \textcolor{red}{सुट् तिथोः} (पा॰सू॰~३.४.१०७)~\arrow \textcolor{red}{आद्यन्तौ टकितौ} (पा॰सू॰~१.१.४६)~\arrow लभ्~अ~यास्~सुँट्~ति~\arrow  लभ्~अ~यास्~स्~ति~\arrow \textcolor{red}{लिङः सलोपोऽनन्त्यस्य} (पा॰सू॰~७.२.७९)~\arrow लभ्~अ~या~ति~\arrow \textcolor{red}{अतो येयः} (पा॰सू॰~७.२.८०)~\arrow लभ्~अ~इय्~ति~\arrow \textcolor{red}{लोपो व्योर्वलि} (पा॰सू॰~६.१.६६)~\arrow लभ्~अ~इ~ति~\arrow \textcolor{red}{आद्गुणः} (पा॰सू॰~६.१.८७)~\arrow लभ्~ए~ति~\arrow \textcolor{red}{इतश्च} (पा॰सू॰~३.४.१००)~\arrow लभ्~ए~त्~\arrow लभेत्।} यद्वा \textcolor{red}{लभत इति लभः}।\footnote{\textcolor{red}{नन्दि\-ग्रहि\-पचादिभ्यो ल्युणिन्यचः} (पा॰सू॰~३.१.१३४) इत्यनेन कर्तरि पचाद्यच्।} \textcolor{red}{लभ इवाऽचरतीति लभति}।\footnote{लभ~\arrow \textcolor{red}{सर्वप्राति\-पदिकेभ्य आचारे क्विब्वा वक्तव्यः} (वा॰~३.१.११)~\arrow लभ~क्विँप्~\arrow लभ~व्~\arrow \textcolor{red}{वेरपृक्तस्य} (पा॰सू॰~६.१.६७)~\arrow लभ~\arrow \textcolor{red}{सनाद्यन्ता धातवः} (पा॰सू॰~३.१.३२)~\arrow धातुसञ्ज्ञा~\arrow \textcolor{red}{शेषात्कर्तरि परस्मैपदम्} (पा॰सू॰~१.३.७८)~\arrow \textcolor{red}{वर्तमाने लट्} (पा॰सू॰~३.२.१२३)~\arrow लभ~लट्~\arrow लभ~तिप्~\arrow लभ~ति~\arrow \textcolor{red}{कर्तरि शप्‌} (पा॰सू॰~३.१.६८)~\arrow लभ~शप्~ति~\arrow लभ~अ~ति~\arrow \textcolor{red}{अतो गुणे} (पा॰सू॰~६.१.९७)~\arrow लभ~ति~\arrow लभति।} तत आचारक्विबन्ताल्लिङि तिपि शपि पररूपे \textcolor{red}{यासुट् परस्मैपदेषूदात्तो ङिच्च} (पा॰सू॰~३.४.१०३) इत्यनेन यासुटि कृते सुडागमे सकारद्वयलोपे \textcolor{red}{अतो येयः} (पा॰सू॰~७.२.८०) इत्यनेनेयादेशे गुणे यलोपे \textcolor{red}{इतश्च} (पा॰सू॰~३.४.१००) इत्यनेनेकार\-लोपे \textcolor{red}{लभेत्}।\footnote{लभ~\arrow \textcolor{red}{सनाद्यन्ता धातवः} (पा॰सू॰~३.१.३२)~\arrow धातुसञ्ज्ञा (पूर्ववत्)~\arrow \textcolor{red}{शेषात्कर्तरि परस्मैपदम्} (पा॰सू॰~१.३.७८)~\arrow \textcolor{red}{विधि\-निमन्‍त्रणामन्‍त्रणाधीष्‍ट\-सम्प्रश्‍न\-प्रार्थनेषु लिङ्} (पा॰सू॰~३.३.१६१)~\arrow लभ~लिङ~\arrow लभ~तिप्~\arrow लभ~ति~\arrow \textcolor{red}{कर्तरि शप्‌} (पा॰सू॰~३.१.६८)~\arrow लभ~शप्~ति~\arrow लभ~अ~ति~\arrow \textcolor{red}{अतो गुणे} (पा॰सू॰~६.१.९७)~\arrow लभ~ति~\arrow \textcolor{red}{यासुट् परस्मैपदेषूदात्तो ङिच्च} (पा॰सू॰~३.४.१०३)~\arrow \textcolor{red}{आद्यन्तौ टकितौ} (पा॰सू॰~१.१.४६)~\arrow लभ~यासुँट्~ति~\arrow लभ~यास्~ति~\arrow \textcolor{red}{सुट् तिथोः} (पा॰सू॰~३.४.१०७)~\arrow \textcolor{red}{आद्यन्तौ टकितौ} (पा॰सू॰~१.१.४६)~\arrow लभ~यास्~सुँट्~ति~\arrow  लभ~यास्~स्~ति~\arrow \textcolor{red}{लिङः सलोपोऽनन्त्यस्य} (पा॰सू॰~७.२.७९)~\arrow लभ~या~ति~\arrow\textcolor{red}{अतो येयः} (पा॰सू॰~७.२.८०)~\arrow लभ~इय्~ति~\arrow \textcolor{red}{लोपो व्योर्वलि} (पा॰सू॰~६.१.६६)~\arrow लभ~इ~ति~\arrow \textcolor{red}{आद्गुणः} (पा॰सू॰~६.१.८७)~\arrow लभे~ति~\arrow \textcolor{red}{इतश्च} (पा॰सू॰~३.४.१००)~\arrow लभे~त्~\arrow लभेत्।}\end{sloppypar}
\section[जानामि]{जानामि}
\centering\textcolor{blue}{ज्ञानं सविज्ञानमथानुभक्तिवैराग्ययुक्तं च मितं विभास्वत्।\nopagebreak\\
जानाम्यहं योषिदपि त्वदुक्तं यथा तथा ब्रूहि तरन्ति येन॥}\nopagebreak\\
\raggedleft{–~अ॰रा॰~१.१.९}\\
\fontsize{14}{21}\selectfont\begin{sloppypar}\hyphenrules{nohyphenation}\justifying\noindent\hspace{10mm} अत्र भगवती पार्वती भगवन्तं शिवं प्रार्थयते यत् \textcolor{red}{भक्ति\-वैराग्य\-सहितं ज्ञानं यथा जानीयां तथा कुर्वन्तु देवाः}। अत्र \textcolor{red}{जानीयाम्}\footnote{\textcolor{red}{ज्ञा अवबोधने} (धा॰पा॰~१५०७)~\arrow ज्ञा~\arrow \textcolor{red}{शेषात्कर्तरि परस्मैपदम्} (पा॰सू॰~१.३.७८)~\arrow \textcolor{red}{विधि\-निमन्‍त्रणामन्‍त्रणाधीष्‍ट\-सम्प्रश्‍न\-प्रार्थनेषु लिङ्} (पा॰सू॰~३.३.१६१)~\arrow ज्ञा~लिङ्~\arrow ज्ञा~मिप्~\arrow ज्ञा~मि~\arrow \textcolor{red}{क्र्यादिभ्यः श्ना} (पा॰सू॰~३.१.८१)~\arrow ज्ञा~श्ना~मि~\arrow ज्ञा~ना~मि~\arrow \textcolor{red}{ज्ञाजनोर्जा} (पा॰सू॰~७.३.७९)~\arrow जा~ना~मि~\arrow \textcolor{red}{इतश्च} (पा॰सू॰~३.४.१००)~\arrow जा~ना~म्~\arrow \textcolor{red}{यासुट् परस्मैपदेषूदात्तो ङिच्च} (पा॰सू॰~३.४.१०३)~\arrow \textcolor{red}{आद्यन्तौ टकितौ} (पा॰सू॰~१.१.४६)~\arrow जा~ना~यासुँट्~म्~\arrow जा~ना~यास्~म्~\arrow \textcolor{red}{लिङः सलोपोऽनन्त्यस्य} (पा॰सू॰~७.२.७९)~\arrow जा~ना~या~म्~\arrow \textcolor{red}{ई हल्यघोः} (पा॰सू॰~६.४.११३)~\arrow जा~नी~या~म्~\arrow जानीयाम्।} इति प्रयोक्तव्ये \textcolor{red}{जानामि} इति प्रयुक्तम्। \textcolor{red}{वर्तमान\-सामीप्ये वर्तमानवद्वा} (पा॰सू॰~३.३.१३१) इत्यनेन वर्तमानवद्भावः।\footnote{\textcolor{red}{ज्ञा अवबोधने} (धा॰पा॰~१५०७)~\arrow ज्ञा~\arrow \textcolor{red}{शेषात्कर्तरि परस्मैपदम्} (पा॰सू॰~१.३.७८)~\arrow \textcolor{red}{वर्तमान\-सामीप्ये वर्तमानवद्वा} (पा॰सू॰~३.३.१३१)~\arrow \textcolor{red}{वर्तमाने लट्} (पा॰सू॰~३.२.१२३)~\arrow ज्ञा~लट्~\arrow ज्ञा~मिप्~\arrow ज्ञा~मि~\arrow \textcolor{red}{क्र्यादिभ्यः श्ना} (पा॰सू॰~३.१.८१)~\arrow ज्ञा~श्ना~मि~\arrow ज्ञा~ना~मि~\arrow \textcolor{red}{ज्ञाजनोर्जा} (पा॰सू॰~७.३.७९)~\arrow जा~ना~मि~\arrow जानामि।}\end{sloppypar}
\section[जानाति]{जानाति}
\centering\textcolor{blue}{वदन्ति केचित्परमोऽपि रामः स्वाविद्यया संवृतमात्मसञ्ज्ञम्।\nopagebreak\\
जानाति नात्मानमतः परेण सम्बोधितो वेद परात्मतत्त्वम्॥}\nopagebreak\\
\raggedleft{–~अ॰रा॰~१.१.१३}\\
\fontsize{14}{21}\selectfont\begin{sloppypar}\hyphenrules{nohyphenation}\justifying\noindent\hspace{10mm} भगवती पार्वती श्रीराम\-लीला\-विषयकं संशयं करोति यत् \textcolor{red}{श्रीरामः स्वाविद्यया संवृतमात्म\-सञ्ज्ञं नाजानात्}। अत्र भूत\-कालिक\-क्रियायां प्रयोक्तव्यायां \textcolor{red}{जानाति} इति वर्तमान\-कालिक\-क्रिया प्रयुक्ताऽपाणिनीयेव। किन्तु \textcolor{red}{पुरा} इत्यस्याध्याहारे \textcolor{red}{पुरि लुङ् चास्मे} (पा॰सू॰~३.२.१२२) इत्यनेन पुरा\-योगे लड्लकारः। अर्थात् \textcolor{red}{पुरा न जानाति}।\footnote{\textcolor{red}{ज्ञा अवबोधने} (धा॰पा॰~१५०७)~\arrow ज्ञा~\arrow \textcolor{red}{शेषात्कर्तरि परस्मैपदम्} (पा॰सू॰~१.३.७८)~\arrow \textcolor{red}{पुरि लुङ् चास्मे} (पा॰सू॰~३.२.१२२)~\arrow ज्ञा~लट्~\arrow ज्ञा~तिप्~\arrow ज्ञा~ति~\arrow \textcolor{red}{क्र्यादिभ्यः श्ना} (पा॰सू॰~३.१.८१)~\arrow ज्ञा~श्ना~ति~\arrow ज्ञा~ना~ति~\arrow \textcolor{red}{ज्ञाजनोर्जा} (पा॰सू॰~७.३.७९)~\arrow जा~ना~ति~\arrow जानाति।} \textcolor{red}{पूर्वं नाजानात्} इत्यर्थः।\footnote{\textcolor{red}{ज्ञा अवबोधने} (धा॰पा॰~१५०७)~\arrow ज्ञा~\arrow \textcolor{red}{शेषात्कर्तरि परस्मैपदम्} (पा॰सू॰~१.३.७८)~\arrow \textcolor{red}{अनद्यतने लङ्} (पा॰सू॰~३.२.१११)~\arrow ज्ञा~लङ्~\arrow ज्ञा~तिप्~\arrow ज्ञा~ति~\arrow \textcolor{red}{लुङ्लङ्लृङ्क्ष्वडुदात्तः} (पा॰सू॰~६.४.७१)~\arrow \textcolor{red}{आद्यन्तौ टकितौ} (पा॰सू॰~१.१.४६)~\arrow अट्~ज्ञा~ति~\arrow अ~ज्ञा~ति~\arrow \textcolor{red}{क्र्यादिभ्यः श्ना} (पा॰सू॰~३.१.८१)~\arrow अ~ज्ञा~श्ना~ति~\arrow अ~ज्ञा~ना~ति~\arrow \textcolor{red}{ज्ञाजनोर्जा} (पा॰सू॰~७.३.७९)~\arrow अ~जा~ना~ति~\arrow \textcolor{red}{इतश्च} (पा॰सू॰~३.४.१००)~\arrow अ~जा~ना~त्~\arrow अजानात्।} ब्रह्मणा बोधितः सन् पश्चाद्व्यजानादिति पार्वत्यास्तात्पर्यम्। न च पुरा शब्दो दृश्यत इति वाच्यम्। गम्यमानः सः।\end{sloppypar}
\section[ब्रूत]{ब्रूत}
\centering\textcolor{blue}{अत्रोत्तरं किं विदितं भवद्भिस्तद्ब्रूत मे संशयभेदि वाक्यम्॥}\nopagebreak\\
\raggedleft{–~अ॰रा॰~१.१.१५}\\
\fontsize{14}{21}\selectfont\begin{sloppypar}\hyphenrules{nohyphenation}\justifying\noindent\hspace{10mm} भगवती पार्वती भगवन्तं शशाङ्क\-शेखरं पृच्छति \textcolor{red}{तत्संशय\-भेदि\-वाक्यं ब्रूत}। \textcolor{red}{यूयं ब्रूत} इत्यध्याहार्यम्। प्रथमपुरुषे प्रयोक्तव्ये\footnote{\textcolor{red}{विदितं भवद्भभिः} इति समभि\-व्याहारेण प्रथमपुरुषः प्रयोक्तव्यः।} मध्यम\-पुरुष\-प्रयोग अपाणिनीयः। एवं च भगवती भवानी पतिव्रता\-शिरोमणिः। सा च स्वकीय\-प्राण\-वल्लभाय मध्यम\-पुरुषस्य विशेषणं ददाति तत्रापि बहुवचनान्तमिति न परं प्रतिभाति। तथा च नियमः~– \textcolor{red}{एकत्वं न प्रयुञ्जीत गुरावात्मनि चेश्वरे}। पार्वत्यास्तु भगवाञ्छिव एव गुरुः। ऐश्वर्ये त्रिभुवन\-गुरुत्वान्माधुर्ये प्राण\-वल्लभत्वाच्च \textcolor{red}{पतिरेव गुरुः स्त्रीणाम्} (म॰भा॰~१४.१०८.६, ब्र॰पु॰~८०.४८, कू॰पु॰~१२.४९, चा॰नी॰~५.१) इति वचनाच्च। अतो बहुवचन\-प्रयोगः। तथाऽपि कथं \textcolor{red}{यूयं ब्रूत} इति चेत्। \textcolor{red}{अत्रोत्तरं किं विदितं भवद्भिः} इत्यत्र \textcolor{red}{भवद्भिः} इति विशेषणं शिवस्य कृते दत्तम्। अतः \textcolor{red}{यूयम्} इति तत्सम्बन्धि\-कर्तृ\-विशेषणं नैव विचारसहम्। \textcolor{red}{भवद्भिः} इत्यस्य हि \textcolor{red}{भवन्तः} इत्यनेन सम्बन्धः स्यात्। यथा \textcolor{red}{यदुत्तरं भवद्भिर्विदितं तद्भवन्तो वदन्तु}। किन्तु प्रश्ने कृते शिवः समाधौ न्यलीयत। तं समाधिस्थं विलोक्य तस्य पञ्च वक्त्राण्युद्दिश्य कथयति यत् \textcolor{red}{भोः पञ्च\-मुखानि यूयमेव संशय\-भेदकं वाक्यं ब्रूत}।\footnote{\textcolor{red}{ब्रूञ् व्यक्तायां वाचि} (धा॰पा॰~१०४४)~\arrow ब्रू~\arrow \textcolor{red}{शेषात्कर्तरि परस्मैपदम्} (पा॰सू॰~१.३.७८)~\arrow \textcolor{red}{लोट् च} (पा॰सू॰~३.३.१६२)~\arrow ब्रू~लोट्~\arrow ब्रू~थस्~\arrow \textcolor{red}{कर्तरि शप्} (पा॰सू॰~३.१.६८)~\arrow ब्रू~शप्~थस्~\arrow \textcolor{red}{अदिप्रभृतिभ्यः शपः} (पा॰सू॰~२.४.७२)~\arrow ब्रू~थस्~\arrow \textcolor{red}{तस्थस्थमिपां तान्तन्तामः} (पा॰सू॰~३.४.१०१)~\arrow ब्रू~ त~\arrow ब्रूत।}\end{sloppypar}
\section[वक्ष्ये]{वक्ष्ये}
\centering\textcolor{blue}{त्वयाऽद्य भक्त्या परिनोदितोऽहं वक्ष्ये नमस्कृत्य रघूत्तमं ते।\nopagebreak\\
रामः परात्मा प्रकृतेरनादिरानन्द एकः पुरुषोत्तमो हि॥}\nopagebreak\\
\raggedleft{–~अ॰रा॰~१.१.१७}\\
\fontsize{14}{21}\selectfont\begin{sloppypar}\hyphenrules{nohyphenation}\justifying\noindent\hspace{10mm}
भगवती\-पार्वती\-प्रश्नं श्रुत्वा शिवः कथयति \textcolor{red}{अहं भगवन्तं नमस्कृत्य वक्ष्ये}। \textcolor{red}{ब्रू}\-धातोः (\textcolor{red}{ब्रूञ् व्यक्तायां वाचि} धा॰पा॰~१०४४) \textcolor{red}{भूवादयो धातवः} (पा॰सू॰~१.३.१) इत्यनेन धातु\-सञ्ज्ञायां \textcolor{red}{लृट् शेषे च} (पा॰सू॰~३.३.१३) इत्यनेन लृड्लकार उत्तम\-पुरुषे \textcolor{red}{तिप्तस्झि\-सिप्थस्थ\-मिब्वस्मस्ताताञ्झ\-थासाथान्ध्वमिड्वहिमहिङ्} (पा॰सू॰~३.४.७८) इत्यनेन \textcolor{red}{तङानावात्मने\-पदम्} (पा॰सू॰~१.४.१००) इति सूत्रानुसारमिट्प्रत्यये \textcolor{red}{स्यतासी लृलुटोः} (पा॰सू॰~३.१.३३) इत्यनेन \textcolor{red}{स्य}\-प्रत्यये \textcolor{red}{ब्रुवो वचिः} (पा॰सू॰~२.४.५३) इत्यनेन ब्रू\-धातोर्वचादेशे \textcolor{red}{टित आत्मने\-पदानां टेरे} (पा॰सू॰~३.४.७९) इत्यनेनैकारादेशे \textcolor{red}{अतो गुणे} (पा॰सू॰~६.१.९७) इत्यनेन पररूपे पश्चाच्चकारस्य कुत्वे षत्वे \textcolor{red}{वक्ष्ये} इति। अत्र \textcolor{red}{वक्ष्यामि} इति कथं नेति चेत्।\footnote{\textcolor{red}{ब्रूञ् व्यक्तायां वाचि} (धा॰पा॰~१०४४)~\arrow ब्रू~\arrow \textcolor{red}{शेषात्कर्तरि परस्मैपदम्} (पा॰सू॰~१.३.७८)~\arrow \textcolor{red}{लृट् शेषे च} (पा॰सू॰~३.३.१३)~\arrow ब्रू~लृट्~\arrow ब्रू~मिप्~\arrow ब्रू~मि~\arrow \textcolor{red}{स्यतासी लृलुटोः} (पा॰सू॰~३.१.३३)~\arrow ब्रू~स्य~मि~\arrow \textcolor{red}{ब्रुवो वचिः} (पा॰सू॰~२.४.५३)~\arrow वच्~स्य~मि~\arrow \textcolor{red}{अतो दीर्घो यञि} (पा॰सू॰~७.३.१०१)~\arrow वच्~स्या~मि~\arrow \textcolor{red}{चोः कुः} (पा॰सू॰~८.२.३०)~\arrow वक्~स्या~मि~\arrow \textcolor{red}{आदेश\-प्रत्यययोः} (पा॰सू॰~८.३.५९)~\arrow वक्~ष्या~मि~\arrow वक्ष्यामि।} क्रिया\-फलस्य कर्तृ\-गामित्वात्।\footnote{\textcolor{red}{स्वरितञितः कर्त्रभिप्राये क्रियाफले} (पा॰सू॰~१.३.७२) इत्यनेन। \textcolor{red}{ब्रूञ् व्यक्तायां वाचि} (धा॰पा॰~१०४४)~\arrow ब्रू~\arrow \textcolor{red}{स्वरितञितः कर्त्रभिप्राये क्रियाफले} (पा॰सू॰~१.३.७२)~\arrow \textcolor{red}{लृट् शेषे च} (पा॰सू॰~३.३.१३)~\arrow ब्रू~लृट्~\arrow ब्रू~इट्~\arrow ब्रू~इ~\arrow \textcolor{red}{स्यतासी लृलुटोः} (पा॰सू॰~३.१.३३)~\arrow ब्रू~स्य~इ~\arrow \textcolor{red}{ब्रुवो वचिः} (पा॰सू॰~२.४.५३)~\arrow वच्~स्य~इ~\arrow \textcolor{red}{टित आत्मने\-पदानां टेरे} (पा॰सू॰~३.४.७९)~\arrow वच्~स्य~ए~\arrow \textcolor{red}{अतो गुणे} (पा॰सू॰~६.१.९७)~\arrow वच्~स्ये~\arrow \textcolor{red}{चोः कुः} (पा॰सू॰~८.२.३०)~\arrow वक्~स्ये~\arrow \textcolor{red}{आदेश\-प्रत्यययोः} (पा॰सू॰~८.३.५९)~\arrow वक्~ष्ये~\arrow वक्ष्ये।} रामायण\-कथा\-प्रश्नेन पार्वत्यास्तु सन्देहो नष्टो भविष्यत्येव किन्तु तत्कथनेन वक्तुः शशाङ्क\-मौलेरपि स्वान्तः\-सुखमुत्पत्स्यते। भगवत्कथा\-प्रश्नो वक्तारं प्रश्न\-कर्तारं श्रोतारमिति त्रीनपि पुनाति। तथा चोक्तं भागवते~–\end{sloppypar}
\centering\textcolor{red}{वासुदेवकथाप्रश्नः पुरुषांस्त्रीन् पुनाति हि।\nopagebreak\\
वक्तारं पृच्छकं श्रोतृंस्तत्पादसलिलं यथा॥}\nopagebreak\\
\raggedleft{–~भा॰पु॰~१०.१.१६}\\
\fontsize{14}{21}\selectfont\begin{sloppypar}\hyphenrules{nohyphenation}\justifying\noindent इयं हि गङ्गा। तथा चोक्तमत्रैव ग्रन्थे~–\end{sloppypar}
\centering\textcolor{blue}{पुरारिगिरिसम्भूता श्रीरामार्णवसङ्गता।\nopagebreak\\
अध्यात्मरामगङ्गेयं पुनाति भुवनत्रयम्॥}\nopagebreak\\
\raggedleft{–~अ॰रा॰~१.१.५}\\
\fontsize{14}{21}\selectfont\begin{sloppypar}\hyphenrules{nohyphenation}\justifying\noindent अतो \textcolor{red}{वक्ष्ये} इति समूलमेव।\end{sloppypar}
\section[भ्रमतीव दृश्यते]{भ्रमतीव दृश्यते}
\centering\textcolor{blue}{यथा हि चाक्ष्णा भ्रमता गृहादिकं विनष्टदृष्टेर्भ्रमतीव दृश्यते।\nopagebreak\\
तथैव देहेन्द्रियकर्तुरात्मनः कृते परेऽध्यस्य जनो विमुह्यति॥}\nopagebreak\\
\raggedleft{–~अ॰रा॰~१.१.२२}\\
\fontsize{14}{21}\selectfont\begin{sloppypar}\hyphenrules{nohyphenation}\justifying\noindent\hspace{10mm} अत्र \textcolor{red}{भ्रम्‌}\-धातुः (\textcolor{red}{भ्रमुँ अनवस्थाने} धा॰पा॰~१२०४) दिवादिः। एवं \textcolor{red}{दिवादिभ्यः श्यन्} (पा॰सू॰~३.१.६९) इत्यनेन लटि तिपि श्यन्प्रत्यये कृते \textcolor{red}{शमामष्टानां दीर्घः श्यनि} (पा॰सू॰~७.३.७४) इत्यनेन दीर्घे \textcolor{red}{भ्राम्यति} इति पाणिनीयम्।\footnote{\textcolor{red}{भ्रमुँ अनवस्थाने} (धा॰पा॰~१२०४)~\arrow भ्रम्~\arrow \textcolor{red}{शेषात्कर्तरि परस्मैपदम्} (पा॰सू॰~१.३.७८)~\arrow \textcolor{red}{वर्तमाने लट्} (पा॰सू॰~३.२.१२३)~\arrow भ्रम्~लट्~\arrow भ्रम्~तिप्~\arrow भ्रम्~ति~\arrow \textcolor{red}{दिवादिभ्यः श्यन्} (पा॰सू॰~३.१.६९)~\arrow भ्रम्~श्यन्~ति~\arrow भ्रम्~य~ति~\arrow \textcolor{red}{शमामष्टानां दीर्घः श्यनि} (पा॰सू॰~७.३.६४)~\arrow भ्राम्~य~ति~\arrow भ्राम्यति।} \textcolor{red}{भ्रमति} इति कथमिति चेत्। \textcolor{red}{वा भ्राश\-भ्लाश\-भ्रमु\-क्रमु\-क्लमु\-त्रसि\-त्रुटि\-लषः} (पा॰सू॰~३.१.७०) इत्यनेन श्यनो विकल्पात् \textcolor{red}{भ्रमति}।\footnote{\textcolor{red}{भ्रमुँ अनवस्थाने} (धा॰पा॰~१२०४)~\arrow भ्रम्~\arrow \textcolor{red}{शेषात्कर्तरि परस्मैपदम्} (पा॰सू॰~१.३.७८)~\arrow \textcolor{red}{वर्तमाने लट्} (पा॰सू॰~३.२.१२३)~\arrow भ्रम्~लट्~\arrow भ्रम्~तिप्~\arrow भ्रम्~ति~\arrow \textcolor{red}{वा भ्राश\-भ्लाश\-भ्रमु\-क्रमु\-क्लमु\-त्रसि\-त्रुटि\-लषः} (पा॰सू॰~३.१.७०)~\arrow भ्रम्~शप्~ति~\arrow भ्रम्~अ~ति~\arrow भ्रमति।} यद्वा \textcolor{red}{भ्रममाचरतीति भ्रमति} अनेन प्रकारेण सिद्धम्।\footnote{भ्रमणं भ्रमः। \textcolor{red}{भावे} (पा॰सू॰~३.३.१८) इत्यनेन भावे घञ्। भ्रम~\arrow \textcolor{red}{सर्वप्राति\-पदिकेभ्य आचारे क्विब्वा वक्तव्यः} (वा॰~३.१.११)~\arrow भ्रम~क्विँप्~\arrow भ्रम~व्~\arrow \textcolor{red}{वेरपृक्तस्य} (पा॰सू॰~६.१.६७)~\arrow भ्रम~\arrow \textcolor{red}{सनाद्यन्ता धातवः} (पा॰सू॰~३.१.३२)~\arrow धातुसञ्ज्ञा~\arrow \textcolor{red}{शेषात्कर्तरि परस्मैपदम्} (पा॰सू॰~१.३.७८)~\arrow वर्तमाने लट्~\arrow भ्रम~लट्~\arrow भ्रम~तिप्~\arrow भ्रम~ति~\arrow \textcolor{red}{कर्तरि शप्‌} (पा॰सू॰~३.१.६८)~\arrow भ्रम~शप्~ति~\arrow भ्रम~अ~ति~\arrow \textcolor{red}{अतो गुणे} (पा॰सू॰~६.१.९७)~\arrow भ्रम~ति~\arrow भ्रमति।} यद्वा \textcolor{red}{गण\-कार्यमनित्यम्} (प॰शे॰~९३.३) इत्यनेन श्यनभावे भ्वादित्वाच्छपि।\footnote{\textcolor{red}{भ्रमुँ अनवस्थाने} (धा॰पा॰~१२०४)~\arrow भ्रम्~\arrow \textcolor{red}{शेषात्कर्तरि परस्मैपदम्} (पा॰सू॰~१.३.७८)~\arrow \textcolor{red}{वर्तमाने लट्} (पा॰सू॰~३.२.१२३)~\arrow भ्रम्~लट्~\arrow भ्रम्~तिप्~\arrow भ्रम्~ति~\arrow \textcolor{red}{गण\-कार्यमनित्यम्} (प॰शे॰~९३.३)~\arrow \textcolor{red}{कर्तरि शप्} (पा॰सू॰~३.१.६८)~\arrow भ्रम्~शप्~ति~\arrow भ्रम्~अ~ति~\arrow भ्रमति।} यद्वा \textcolor{red}{भ्रमति} इति नास्ति क्रियाऽपि तु 
\textcolor{red}{भ्रमती इव} इति \textcolor{red}{भ्रम्‌}\-धातोरौणादिक\-तृच्प्रत्ययान्तो \textcolor{red}{भ्रमती}।\footnote{अत्र \textcolor{red}{तृँच्} प्रत्यय इति भावः। नायं \textcolor{red}{बहुलमन्यत्रापि} (प॰उ॰~२.९५) इति तृच्। स नोगित्। \textcolor{red}{कार्याद्विद्यादनूबन्धम्} (भा॰पा॰सू॰~३.३.१) \textcolor{red}{केचिदविहिता अप्यूह्याः} (वै॰सि॰कौ॰~३१६९) इत्यनुसारमूह्योऽ\-यमविहित उगित्प्रत्ययः। \textcolor{red}{तृँच्} प्रत्यये चात्र शबागमोऽप्यूह्यः। \textcolor{red}{नयतेः षुगागमः} (प॰उ॰ श्वे॰वृ॰~२.९६) इतिवत्। \textcolor{red}{भ्रमुँ अनवस्थाने} (धा॰पा॰~१२०४)~\arrow भ्रम्~\arrow \textcolor{red}{उणादयो बहुलम्} (पा॰सू॰~३.३.१)~\arrow भ्रम्~तृँच्~\arrow भ्रम्~शप्~अत्~\arrow भ्रम्~अ~त्~\arrow भ्रमत्~\arrow \textcolor{red}{उगितश्च} (पा॰सू॰~४.१.६)~\arrow भ्रमत्~ङीप्~\arrow भ्रमत्~ई~\arrow भ्रमती~\arrow विभक्तिकार्यम्~\arrow भ्रमती~सुँ~\arrow भ्रमती~स्~\arrow \textcolor{red}{हल्ङ्याब्भ्यो दीर्घात्सुतिस्यपृक्तं हल्} (पा॰सू॰~६.१.६८)~\arrow भ्रमती।}\end{sloppypar}
\section[आकाङ्क्षते]{आकाङ्क्षते}
\centering\textcolor{blue}{रामो न गच्छति न तिष्ठति नानुशोचत्याकाङ्क्षते त्यजति नो न करोति किञ्चित्।\nopagebreak\\
आनन्दमूर्तिरचलः परिणामहीनो मायागुणाननुगतो हि तथा विभाति॥}\nopagebreak\\
\raggedleft{–~अ॰रा॰~१.१.४३}\\
\fontsize{14}{21}\selectfont\begin{sloppypar}\hyphenrules{nohyphenation}\justifying\noindent\hspace{10mm} अत्र सीता श्रीराम\-तत्त्वं वर्णयन्ती \textcolor{red}{आकाङ्क्षते} इति प्रयुङ्क्ते। यत् \textcolor{red}{रामो नाऽकाङ्क्षते}। \textcolor{red}{काङ्क्ष्‌}\-धातुः (\textcolor{red}{काक्षिँ काङ्क्षायाम्} धा॰पा॰~६६७) परस्मैपदी। एवं ततो लटि तिपि शपि \textcolor{red}{आकाङ्क्षति}।\footnote{आङ्~\textcolor{red}{काक्षिँ काङ्क्षायाम्} (धा॰पा॰~६६७)~\arrow आ~काक्ष्~\arrow \textcolor{red}{इदितो नुम् धातोः} (पा॰सू॰~७.१.५८)~\arrow \textcolor{red}{मिदचोऽन्त्यात्परः} (पा॰सू॰~१.१.४७)~\arrow आ~का~नुँम्~क्ष्~\arrow आ~कान्~क्ष्~\arrow \textcolor{red}{नश्चापदान्तस्य झलि} (पा॰सू॰~८.३.२४)~\arrow आ~कांक्ष्~\arrow \textcolor{red}{अनुस्वारस्य ययि परसवर्णः} (पा॰सू॰~८.४.५८)~\arrow आ~काङ्क्ष्~\arrow \textcolor{red}{शेषात्कर्तरि परस्मैपदम्} (पा॰सू॰~१.३.७८)~\arrow \textcolor{red}{वर्तमाने लट्} (पा॰सू॰~३.२.१२३)~\arrow आ~काङ्क्ष्~लट्~\arrow आ~काङ्क्ष्~तिप्~\arrow आ~काङ्क्ष्~ति~\arrow \textcolor{red}{कर्तरि शप्‌} (पा॰सू॰~३.१.६८)~\arrow आ~काङ्क्ष्~शप्~ति~\arrow आ~काङ्क्ष्~अ~ति~\arrow आकाङ्क्षति।} तथा च \textcolor{red}{नहि प्रफुल्लं सहकारमेत्य वृक्षान्तरं काङ्क्षति षट्पदाली} (र॰वं॰~६.६९) इति कालिदासोऽपि प्रायुङ्क्तेति चेत्कथमत्रात्मनेपदम्। अत्र विमृश्यते। \textcolor{red}{कर्तरि कर्म\-व्यतिहारे} (पा॰सू॰~१.३.१४) इत्यनेनाऽत्मनेपदम्।\footnote{आ~काङ्क्ष् (पूर्ववत्)~\arrow \textcolor{red}{कर्तरि कर्म\-व्यतिहारे} (पा॰सू॰~१.३.१४)~\arrow \textcolor{red}{वर्तमाने लट्} (पा॰सू॰~३.२.१२३)~\arrow आ~काङ्क्ष्~लट्~\arrow आ~काङ्क्ष्~त~\arrow \textcolor{red}{कर्तरि शप्‌} (पा॰सू॰~३.१.६८)~\arrow आ~काङ्क्ष्~शप्~त~\arrow आ~काङ्क्ष्~अ~त~\arrow \textcolor{red}{टित आत्मनेपदानां टेरे} (पा॰सू॰~३.४.७९)~\arrow आ~काङ्क्ष्~अ~ते~\arrow आकाङ्क्षते।} कर्म\-व्यतिहारो हि क्रिया\-विनिमयः। जीवस्याऽकाङ्क्षा भगवत्यारोपिता। तामेव निरस्यति \textcolor{red}{नाकाङ्क्षते} इति।\end{sloppypar}
\section[पठति]{पठति}
\centering\textcolor{blue}{योऽतिभ्रष्टोऽतिपापी परधनपरदारेषु नित्योद्यतो वा\nopagebreak\\
स्तेयी ब्रह्मघ्नमातापितृवधनिरतो योगिवृन्दापकारी।\nopagebreak\\
यः सम्पूज्याभिरामं पठति च हृदयं रामचन्द्रस्य भक्त्या\nopagebreak\\
योगीन्द्रैरप्यलभ्यं पदमिह लभते सर्वदेवैः स पूज्यम्॥}\nopagebreak\\
\raggedleft{–~अ॰रा॰~१.१.५६}\\
\fontsize{14}{21}\selectfont\begin{sloppypar}\hyphenrules{nohyphenation}\justifying\noindent\hspace{10mm} \textcolor{red}{पठति} इति भविष्यत्कालार्थे वर्तमान\-कालं प्रयुङ्क्त इति चेत्। \textcolor{red}{वर्तमान\-सामीप्ये वर्तमानवद्वा} (पा॰सू॰~३.३.१३१) इत्यनेन वर्तमान\-कालः।\footnote{\textcolor{red}{पठँ व्यक्तायां वाचि} (धा॰पा॰~३३०)~\arrow पठ्~\arrow \textcolor{red}{शेषात्कर्तरि परस्मैपदम्} (पा॰सू॰~१.३.७८)~\arrow \textcolor{red}{वर्तमान\-सामीप्ये वर्तमानवद्वा} (पा॰सू॰~३.३.१३१)~\arrow \textcolor{red}{वर्तमाने लट्} (पा॰सू॰~३.२.१२३)~\arrow पठ्~लट्~\arrow पठ्~तिप्~\arrow पठ्~ति~\arrow \textcolor{red}{कर्तरि शप्} (पा॰सू॰~३.१.६८)~\arrow पठ्~शप्~ति~\arrow पठ्~अ~ति~\arrow पठति। लृटि च \textcolor{red}{पठिष्यति} इति रूपम्। \textcolor{red}{पठँ व्यक्तायां वाचि} (धा॰पा॰~३३०)~\arrow पठ्~\arrow \textcolor{red}{शेषात्कर्तरि परस्मैपदम्} (पा॰सू॰~१.३.७८)~\arrow \textcolor{red}{लृट् शेषे च} (पा॰सू॰~३.३.१३)~\arrow पठ्~लृट्~\arrow पठ्~तिप्~\arrow पठ्~ति~\arrow \textcolor{red}{स्यतासी लृलुटोः} (पा॰सू॰~३.१.३३)~\arrow पठ्~स्य~ति~\arrow \textcolor{red}{आर्धधातुकस्येड्वलादेः} (पा॰सू॰~७.२.३५)~\arrow पठ्~इट्~स्य~ति~\arrow पठ्~इ~स्य~ति~\arrow \textcolor{red}{आदेश\-प्रत्यययोः} (पा॰सू॰~८.३.५९)~\arrow पठ्~इ~ष्य~ति~\arrow पठिष्यति।} एवमेव \textcolor{red}{लभते} इत्यत्रापि। अर्थात्त्वरितमेव फलसिद्धिर्भविष्यति।\footnote{\textcolor{red}{डुलभँष् प्राप्तौ} (धा॰पा॰~९७५)~\arrow \textcolor{red}{अनुदात्तङित आत्मने\-पदम्} (पा॰सू॰~१.३.१२)~\arrow \textcolor{red}{वर्तमान\-सामीप्ये वर्तमानवद्वा} (पा॰सू॰~३.३.१३१)~\arrow \textcolor{red}{वर्तमाने लट्} (पा॰सू॰~३.२.१२३)~\arrow लभ्~लट्~\arrow लभ्~त~\arrow \textcolor{red}{कर्तरि शप्} (पा॰सू॰~३.१.६८)~\arrow लभ्~शप्~त~\arrow लभ्~अ~त~\arrow \textcolor{red}{टित आत्मनेपदानां टेरे} (पा॰सू॰~३.४.७९)~\arrow लभ्~अ~ते~\arrow लभते। लृटि च \textcolor{red}{लप्स्यते} इति रूपम्। \textcolor{red}{डुलभँष् प्राप्तौ} (धा॰पा॰~९७५)~\arrow \textcolor{red}{अनुदात्तङित आत्मने\-पदम्} (पा॰सू॰~१.३.१२)~\arrow \textcolor{red}{लृट् शेषे च} (पा॰सू॰~३.३.१३)~\arrow लभ्~लृट्~\arrow लभ्~त~\arrow \textcolor{red}{स्यतासी लृलुटोः} (पा॰सू॰~३.१.३३)~\arrow लभ्~स्य~त~\arrow \textcolor{red}{एकाच उपदेशेऽनुदात्तात्‌} (पा॰सू॰~७.२.१०)~\arrow इडागम\-निषेधः~\arrow \textcolor{red}{खरि च} (पा॰सू॰~८.४.५५)~\arrow लप्~स्य~त~\arrow \textcolor{red}{टित आत्मनेपदानां टेरे} (पा॰सू॰~३.४.७९)~\arrow लप्~स्य~ते~\arrow लप्स्यते।}\end{sloppypar}
\section[मुच्यते]{मुच्यते}
\centering\textcolor{blue}{तदद्य कथयिष्यामि शृणु तापत्रयापहम्।\nopagebreak\\
यच्छ्रुत्वा मुच्यते जन्तुरज्ञानोत्थमहाभयात्।\nopagebreak\\
प्राप्नोति परमामृद्धिं दीर्घायुः पुत्रसन्ततिम्॥}\nopagebreak\\
\raggedleft{–~अ॰रा॰~१.२.५}\\
\fontsize{14}{21}\selectfont\begin{sloppypar}\hyphenrules{nohyphenation}\justifying\noindent\hspace{10mm} अत्र रामायणस्य शाश्वतत्वाद्वर्तमान\-कालः।\footnote{रामायणस्य शाश्वतत्वं रामायण एवोक्तम्~– \textcolor{red}{यावत्स्थास्यन्ति गिरयः सरितः च महीतले॥ तावद्रामायणकथा लोकेषु प्रचरिष्यति।} (वा॰रा॰~१.२.३६-३७)। \textcolor{red}{मुचॢँ मोक्षणे} (धा॰पा॰~१४३०)~\arrow मुच्~\arrow \textcolor{red}{भावकर्मणोः} (पा॰सू॰~१.३.१३)~\arrow \textcolor{red}{वर्तमाने लट्} (पा॰सू॰~३.२.१२३)~\arrow मुच्~लट्~\arrow मुच्~त~\arrow \textcolor{red}{सार्वधातुके यक्} (पा॰सू॰~३.१.६७)~\arrow मुच्~यक्~त~\arrow मुच्~य~त~\arrow \textcolor{red}{ग्क्ङिति च} (पा॰सू॰~१.१.५)~\arrow लघूपध\-गुण\-निषेधः~\arrow मुच्~य~त~\arrow \textcolor{red}{टित आत्मनेपदानां टेरे} (पा॰सू॰~३.४.७९)~\arrow मुच्~य~ते~\arrow मुच्यते।} वैयाकरणानां मते शब्दानां नित्यत्वात्। एवं भक्तानां मते भगवतस्तत्कथायाश्च नित्यत्वात्। अतो जीवानां दृष्टौ वर्तमान\-भूत\-भविष्यत्कालाः। त्रिकालाबाध्यत्वात्कालातीतत्वाच्चेश्वरस्य समक्षं निरन्तरं वर्तमान\-कालः।\end{sloppypar}
\section[सृजामि]{सृजामि}
\centering\textcolor{blue}{तस्याहं पुत्रतामेत्य कौसल्यायां शुभे दिने।\nopagebreak\\
चतुर्धाऽऽत्मानमेवाहं सृजामीतरयोः पृथक्॥}\nopagebreak\\
\raggedleft{–~अ॰रा॰~१.२.२७}\\
\fontsize{14}{21}\selectfont\begin{sloppypar}\hyphenrules{nohyphenation}\justifying\noindent\hspace{10mm} अत्र भगवाञ्छ्रीरामभद्रोऽवतार\-कथां वर्णयन्नाह यत् \textcolor{red}{कौशल्यायामितरयोश्चाऽत्मानं चतुर्धा सृजामि}। अत्र \textcolor{red}{स्रक्ष्यामि} इति हि पाणिनीयम्।\footnote{\textcolor{red}{सृजँ विसर्गे} (धा॰पा॰~१४१४)~\arrow सृज्~\arrow \textcolor{red}{शेषात्कर्तरि परस्मैपदम्} (पा॰सू॰~१.३.७८)~\arrow \textcolor{red}{लृट् शेषे च} (पा॰सू॰~३.३.१३)~\arrow सृज्~लृट्~\arrow सृज्~मिप्~\arrow सृज्~मि~\arrow \textcolor{red}{स्यतासी लृलुटोः} (पा॰सू॰~३.१.३३)~\arrow सृज्~स्य~मि~\arrow \textcolor{red}{सृजि\-दृशोर्झल्यमकिति} (पा॰सू॰~६.१.५८)~\arrow \textcolor{red}{मिदचोऽन्त्यात्परः} (पा॰सू॰~१.१.४७)~\arrow सृ~अम्~ज्~स्य~मि~\arrow सृ~अ~ज्~स्य~मि~\arrow \textcolor{red}{इको यणचि} (पा॰सू॰~६.१.७७)~\arrow स्र्~अ~ज्~स्य~मि~\arrow \textcolor{red}{व्रश्चभ्रस्ज\-सृजमृज\-यजराज\-भ्राजच्छशां षः} (पा॰सू॰~८.२.३६)~\arrow स्र्~अ~ष्~स्य~मि~\arrow \textcolor{red}{षढोः कः सि} (पा॰सू॰~८.२.४१)~\arrow स्र्~अ~क्~स्य~मि~\arrow \textcolor{red}{आदेश\-प्रत्यययोः} (पा॰सू॰~८.३.५९)~\arrow स्र्~अ~क्~ष्य~मि~\arrow \textcolor{red}{अतो दीर्घो यञि} (पा॰सू॰~७.३.१०१)~\arrow स्र्~अ~क्~ष्या~मि~\arrow स्रक्ष्यामि।} किन्तु वर्तमान\-सामीप्यात् \textcolor{red}{सृजामि} इति न दोषः।\footnote{\textcolor{red}{वर्तमान\-सामीप्ये वर्तमानवद्वा} (पा॰सू॰~३.३.१३१) इत्यनेन। \textcolor{red}{सृजँ विसर्गे} (धा॰पा॰~१४१४)~\arrow सृज्~\arrow \textcolor{red}{शेषात्कर्तरि परस्मैपदम्} (पा॰सू॰~१.३.७८)~\arrow \textcolor{red}{वर्तमान\-सामीप्ये वर्तमानवद्वा} (पा॰सू॰~३.३.१३१)~\arrow \textcolor{red}{वर्तमाने लट्} (पा॰सू॰~३.२.१२३)~\arrow सृज्~लट्~\arrow सृज्~मिप्~\arrow सृज्~मि~\arrow \textcolor{red}{तुदादिभ्यः शः} (पा॰सू॰~३.१.७७)~\arrow सृज्~श~मि~\arrow सृज्~अ~मि~\arrow \textcolor{red}{सार्वधातुकमपित्} (पा॰सू॰~१.२.४)~\arrow ङित्त्वम्~\arrow \textcolor{red}{ग्क्ङिति च} (पा॰सू॰~१.१.५)~\arrow लघूपध\-गुण\-निषेधः~\arrow सृज्~अ~मि~\arrow \textcolor{red}{अतो दीर्घो यञि} (पा॰सू॰~७.३.१०१)~\arrow सृज्~आ~मि~\arrow सृजामि।}\end{sloppypar}
\section[सृजध्वम्]{सृजध्वम्}
\centering\textcolor{blue}{यूयं सृजध्वं सर्वेऽपि वानरेष्वंशसम्भवान्।\nopagebreak\\
विष्णोः सहायं कुरुत यावत्स्थास्यति भूतले॥}\nopagebreak\\
\raggedleft{–~अ॰रा॰~१.२.३०}\\
\fontsize{14}{21}\selectfont\begin{sloppypar}\hyphenrules{nohyphenation}\justifying\noindent\hspace{10mm} ब्रह्मा देवानादिशति यत् \textcolor{red}{यूयं वानरेष्वात्मानं सृजध्वम्}। रचयध्वमिति भावः। \textcolor{red}{सृज्‌}\-धातुः (\textcolor{red}{सृजँ विसर्गे} धा॰पा॰~१४१४) आत्मने\-पदी न।\footnote{तस्माल्लोड्लकारे मध्यमपुरुषे बहुवचने \textcolor{red}{सृजथ} इति रूपम्। यथा~– \textcolor{red}{यूयमप्यहह पूजनमस्या यन्निजैः सृजथ पादपयोजैः} (नै॰च॰~५.९६) इति श्रीहर्षप्रयोगे। \textcolor{red}{सृजँ विसर्गे} (धा॰पा॰~१४१४)~\arrow सृज्~\arrow \textcolor{red}{शेषात्कर्तरि परस्मैपदम्} (पा॰सू॰~१.३.७८)~\arrow \textcolor{red}{लोट् च} (पा॰सू॰~३.३.१६२)~\arrow सृज्~लोट्~\arrow सृज्~थ~\arrow \textcolor{red}{तुदादिभ्यः शः} (पा॰सू॰~३.१.७७)~\arrow सृज्~श~थ~\arrow सृज्~अ~ध्वम्~\arrow \textcolor{red}{सार्वधातुकमपित्} (पा॰सू॰~१.२.४)~\arrow ङित्त्वम्~\arrow \textcolor{red}{ग्क्ङिति च} (पा॰सू॰~१.१.५)~\arrow लघूपध\-गुण\-निषेधः~\arrow सृज्~अ~थ~\arrow सृजथ।} कथमत्र \textcolor{red}{सृजध्वम्} इति चेत्। अत्र कर्म\-व्यतिहारादात्मने\-पदम्।\footnote{\textcolor{red}{कर्तरि कर्म\-व्यतिहारे} पा॰सू॰~१.३.१४) इत्यनेन।} सर्जनं ब्रह्मणः कर्म तदेव देवेभ्यो दीयत इति क्रिया\-विनिमयः। आत्मनेपदे लोड्लकारे ध्वमि शे \textcolor{red}{सृजध्वम्}।\footnote{\textcolor{red}{सृजँ विसर्गे} (धा॰पा॰~१४१४)~\arrow सृज्~\arrow \textcolor{red}{कर्तरि कर्म\-व्यतिहारे} (पा॰सू॰~१.३.१४)~\arrow \textcolor{red}{लोट् च} (पा॰सू॰~३.३.१६२)~\arrow सृज्~लोट्~\arrow सृज्~ध्वम्~\arrow \textcolor{red}{तुदादिभ्यः शः} (पा॰सू॰~३.१.७७)~\arrow सृज्~श~ध्वम्~\arrow सृज्~अ~ध्वम्~\arrow \textcolor{red}{सार्वधातुकमपित्} (पा॰सू॰~१.२.४)~\arrow ङित्त्वम्~\arrow \textcolor{red}{ग्क्ङिति च} (पा॰सू॰~१.१.५)~\arrow लघूपध\-गुण\-निषेधः~\arrow सृज्~अ~ध्वम्~\arrow सृजध्वम्।}\end{sloppypar}
\section[दर्शयस्व]{दर्शयस्व}
\centering\textcolor{blue}{उपसंहर विश्वात्मन्नदो रूपमलौकिकम्।\nopagebreak\\
दर्शयस्व महानन्दबालभावं सुकोमलम्।\nopagebreak\\
ललितालिङ्गनालापैस्तरिष्याम्युत्कटं तमः॥}\nopagebreak\\
\raggedleft{–~अ॰रा॰~१.३.२९}\\
\fontsize{14}{21}\selectfont\begin{sloppypar}\hyphenrules{nohyphenation}\justifying\noindent\hspace{10mm} अत्र भगवती कौशल्या स्व\-समक्षं प्रकटं श्रीराम\-भद्रं प्रार्थयते यत् \textcolor{red}{बाल\-भावं दर्शयस्व}। अत्र \textcolor{red}{दृश्‌}\-धातोः (\textcolor{red}{दृशिँर् प्रेक्षणे} धा॰पा॰~९८८) 
णिचि लघूपध\-गुणे रपरत्वे धातु\-सञ्ज्ञायां पुनर्लोड्लकार आत्मने\-पदे \textcolor{red}{थास्‌}\-प्रत्यये शपि गुणेऽयादेशे स्वादेशे \textcolor{red}{दर्शयस्व}।\footnote{\textcolor{red}{दृशिँर् प्रेक्षणे} (धा॰पा॰~९८८)~\arrow दृश्~\arrow \textcolor{red}{हेतुमति च} (पा॰सू॰~३.१.२६)~\arrow दृश्~णिच्~\arrow दृश्~इ~\arrow \textcolor{red}{पुगन्त\-लघूपधस्य च} (पा॰सू॰~७.३.८६)~\arrow दश्~इ~\arrow \textcolor{red}{उरण् रपरः} (पा॰सू॰~१.१.५१)~\arrow दर्श्~इ~\arrow दर्शि~\arrow \textcolor{red}{सनाद्यन्ता धातवः} (पा॰सू॰~३.१.३२)~\arrow धातुसञ्ज्ञा~\arrow \textcolor{red}{णिचश्च} (पा॰सू॰~१.३.७४)~\arrow \textcolor{red}{लोट् च} (पा॰सू॰~३.३.१६२)~\arrow दर्शि~लोट्~\arrow दर्शि~थास्~\arrow \textcolor{red}{कर्तरि शप्} (पा॰सू॰~३.१.६८)~\arrow दर्शि~शप्~थास्~\arrow दर्शि~अ~थास्~\arrow \textcolor{red}{सार्वधातुकार्धधातुकयोः} (पा॰सू॰~७.३.८४)~\arrow दर्शे~अ~थास्~\arrow \textcolor{red}{एचोऽयवायावः} (पा॰सू॰~६.१.७८)~\arrow दर्शय्~अ~थास्~\arrow \textcolor{red}{थासस्से} (पा॰सू॰~३.४.८०)~\arrow दर्शय्~अ~से~\arrow \textcolor{red}{सवाभ्यां वामौ} (पा॰सू॰~३.४.९१)~\arrow दर्शय्~अ~स्व~\arrow दर्शयस्व।} आत्मनेपदं यदा कर्तरि फलं दृश्येत। अत्र राम\-रूपे कर्तरि किं फलं श्रीरामस्य फलानपेक्षत्वादिति चेत्। भक्तानन्द एव तस्यापूर्वं फलम्। भक्त\-हितार्थमेव गृहीत\-जन्मत्वात्। यद्वा \textcolor{red}{स्व} इति सम्बोधनम्। \textcolor{red}{हे स्व दर्शय बाल\-भावम्}।\footnote{दर्शि~\arrow धातुसञ्ज्ञा (पूर्ववत्)~\arrow \textcolor{red}{शेषात्कर्तरि परस्मैपदम्} (पा॰सू॰~१.३.७८)~\arrow \textcolor{red}{लोट् च} (पा॰सू॰~३.३.१६२)~\arrow दर्शि~लोट्~\arrow दर्शि~सिप्~\arrow दर्शि~सि~\arrow \textcolor{red}{कर्तरि शप्} (पा॰सू॰~३.१.६८)~\arrow दर्शि~शप्~सि~\arrow दर्शि~अ~सि~\arrow \textcolor{red}{सार्वधातुकार्धधातुकयोः} (पा॰सू॰~७.३.८४)~\arrow दर्शे~अ~सि~\arrow \textcolor{red}{एचोऽयवायावः} (पा॰सू॰~६.१.७८)~\arrow दर्शय्~अ~सि~\arrow \textcolor{red}{सेर्ह्यपिच्च} (पा॰सू॰~३.४.८७)~\arrow दर्शय्~अ~हि~\arrow \textcolor{red}{अतो हेः} (पा॰सू॰~६.४.१०५)~\arrow दर्शय्~अ~\arrow दर्शय।} स्व\-शब्दस्य चत्वारोऽर्था आत्माऽऽत्मीयो ज्ञातिर्धनञ्च।\footnote{\textcolor{red}{स्वे स्वाः। आत्मीया इत्यर्थः। आत्मान इति वा। ज्ञाति\-धन\-वाचिनस्तु स्वाः। ज्ञातयोऽर्था वा} (वै॰सि॰कौ॰~२२०)।} अतः \textcolor{red}{हे स्व हे मम आत्मन्मम धन} इति निजं दर्शय। यद्वा \textcolor{red}{स्व} इति \textcolor{red}{महानन्दम्}
इत्यस्य विशेषणम्। एवं च \textcolor{red}{स्वेभ्य आत्मीयेभ्यो महानानन्दो यस्मात्स स्वमहानन्दः}। \textcolor{red}{स्वमहानन्दश्चासौ बालभावश्चेति स्वमहानन्द\-बालभावस्तं स्वमहानन्द\-बालभावम्}।\end{sloppypar}
\section[याति]{याति}
\centering\textcolor{blue}{संवादमावयोर्यस्तु पठेद्वा शृणुयादपि।\nopagebreak\\
स याति मम सारूप्यं मरणे मत्स्मृतिं लभेत्॥}\nopagebreak\\
\raggedleft{–~अ॰रा॰~१.३.३४}\\
\fontsize{14}{21}\selectfont\begin{sloppypar}\hyphenrules{nohyphenation}\justifying\noindent\hspace{10mm} अत्र भगवाञ्छ्रीरामो मात्रा सह स्वकीय\-संवादस्य फल\-श्रुतिं वर्णयति \textcolor{red}{य आवयोः संवादं पठेच्छृणुयाद्वा स मम सारूप्यं यायात्}। \textcolor{red}{यायात्}\footnote{आशीर्वादार्थे लिङि \textcolor{red}{यायात्} इति रूपम्। \textcolor{red}{या प्रापणे} (धा॰पा॰~१०४९)~\arrow या~\arrow \textcolor{red}{शेषात्कर्तरि परस्मैपदम्} (पा॰सू॰~१.३.७८)~\arrow \textcolor{red}{आशिषि लिङ्लोटौ} (पा॰सू॰~३.३.१७३)~\arrow या~लिङ्~\arrow या~तिप्~\arrow या~ति~\arrow \textcolor{red}{यासुट् परस्मैपदेषूदात्तो ङिच्च} (पा॰सू॰~३.४.१०३)~\arrow \textcolor{red}{आद्यन्तौ टकितौ} (पा॰सू॰~१.१.४६)~\arrow या~यासुँट्~ति~\arrow या~यास्~ति~\arrow \textcolor{red}{सुट् तिथोः} (पा॰सू॰~३.४.१०७)~\arrow \textcolor{red}{आद्यन्तौ टकितौ} (पा॰सू॰~१.१.४६)~\arrow या~यास्~सुँट्~ति~\arrow या~यास्~स्~ति~\arrow \textcolor{red}{स्कोः संयोगाद्योरन्ते च} (पा॰सू॰~८.२.२९)~\arrow या~या~ति~\arrow \textcolor{red}{इतश्च} (पा॰सू॰~३.४.१००)~\arrow या~या~त्~\arrow यायात्। विध्यर्थे लिङ्यपि \textcolor{red}{यायात्} इत्येव रूपं प्रक्रिया तु भिन्ना। \textcolor{red}{या प्रापणे} (धा॰पा॰~१०४९)~\arrow या~\arrow \textcolor{red}{शेषात्कर्तरि परस्मैपदम्} (पा॰सू॰~१.३.७८)~\arrow \textcolor{red}{विधि\-निमन्‍त्रणामन्‍त्रणाधीष्‍ट\-सम्प्रश्‍न\-प्रार्थनेषु लिङ्} (पा॰सू॰~३.३.१६१)~\arrow या~लिङ्~\arrow या~तिप्~\arrow या~ति~\arrow \textcolor{red}{कर्तरि शप्} (पा॰सू॰~३.१.६८)~\arrow या~शप्~ति~\arrow \textcolor{red}{अदिप्रभृतिभ्यः शपः} (पा॰सू॰~२.४.७२)~\arrow या~ति~\arrow \textcolor{red}{यासुट् परस्मैपदेषूदात्तो ङिच्च} (पा॰सू॰~३.४.१०३)~\arrow \textcolor{red}{आद्यन्तौ टकितौ} (पा॰सू॰~१.१.४६)~\arrow या~यासुँट्~ति~\arrow या~यास्~ति~\arrow \textcolor{red}{सुट् तिथोः} (पा॰सू॰~३.४.१०७)~\arrow \textcolor{red}{आद्यन्तौ टकितौ} (पा॰सू॰~१.१.४६)~\arrow या~यास्~सुँट्~ति~\arrow या~यास्~स्~ति~\arrow \textcolor{red}{लिङः सलोपोऽनन्त्यस्य} (पा॰सू॰~७.२.७९)~\arrow या~या~ति~\arrow \textcolor{red}{इतश्च} (पा॰सू॰~३.४.१००)~\arrow या~या~त्~\arrow यायात्।} इति प्रयोक्तव्ये \textcolor{red}{याति} इति प्रयुक्तम्। \textcolor{red}{वर्तमान\-सामीप्ये वर्तमानवद्वा} (पा॰सू॰~३.३.१३१) इत्यनेन वर्तमान\-कालः।\footnote{\textcolor{red}{या प्रापणे} (धा॰पा॰~१०४९)~\arrow या~\arrow \textcolor{red}{शेषात्कर्तरि परस्मैपदम्} (पा॰सू॰~१.३.७८)~\arrow \textcolor{red}{वर्तमान\-सामीप्ये वर्तमानवद्वा} (पा॰सू॰~३.३.१३१)~\arrow \textcolor{red}{वर्तमाने लट्} (पा॰सू॰~३.२.१२३)~\arrow या~लट्~\arrow या~तिप्~\arrow या~ति~\arrow \textcolor{red}{कर्तरि शप्‌} (पा॰सू॰~३.१.६८)~\arrow या~शप्~ति~\arrow \textcolor{red}{अदिप्रभृतिभ्यः शपः} (पा॰सू॰~२.४.७२)~\arrow या~ति~\arrow या~ति~\arrow याति।} अर्थात् \textcolor{red}{संवादमिममनुशील्य सद्य एव मम सारूप्यं प्राप्स्यति} इति भगवतस्तात्पर्यम्।\end{sloppypar}
\section[अहनत्]{अहनत्}
\centering\textcolor{blue}{भोजनं देहि मे मातर्न श्रुतं कार्यसक्तया।\nopagebreak\\
ततः क्रोधेन भाण्डानि लगुडेनाहनत्तदा॥}\nopagebreak\\
\raggedleft{–~अ॰रा॰~१.३.५३}\\
\fontsize{14}{21}\selectfont\begin{sloppypar}\hyphenrules{nohyphenation}\justifying\noindent\hspace{10mm} भगवाञ्छ्रीरामो बाल\-लीलामाचरन्मातरं भोजनं याचमानोऽप्राप्य बाल\-क्रोधं विडम्बयल्लँगुडेन भाण्डान्यचूर्णयत्। अत्र \textcolor{red}{अहनत्} इति प्रयोगः। वस्तुतस्तु \textcolor{red}{हन्‌}\-धातोः (\textcolor{red}{हनँ हिंसागत्योः} धा॰पा॰~१०१२) धातु\-सञ्ज्ञायाम् \textcolor{red}{अनद्यतने लङ्} (पा॰सू॰~३.२.१११) इत्यनेन लङ्लकारे \textcolor{red}{तिप्तस्झि\-सिप्थस्थ\-मिब्वस्मस्ताताञ्झ\-थासाथान्ध्वमिड्वहिमहिङ्} (पा॰सू॰~३.४.७८) इत्यनेन तिप्प्रत्यये \textcolor{red}{कर्तरि शप्} (पा॰सू॰~३.१.६८) इत्यनेन शब्विकरणे \textcolor{red}{अदि\-प्रभृतिभ्यः शपः} (पा॰सू॰~२.४.७२) इत्यनेन शपो लुकि \textcolor{red}{लुङ्लङ्लृङ्क्ष्वडुदात्तः} (पा॰सू॰~६.४.७१) इत्यनेनाडागमे \textcolor{red}{इतश्च} (पा॰सू॰~३.४.१००) इत्यनेनेकार\-लोपे \textcolor{red}{हल्ङ्याब्भ्यो दीर्घात्सुतिस्यपृक्तं हल्} (पा॰सू॰~६.१.६८) इत्यनेन तकार\-लोपे \textcolor{red}{अहन्} इति पाणिनीयम्।\footnote{\textcolor{red}{हनँ हिंसागत्योः} (धा॰पा॰~१०१२)~\arrow हन्~\arrow \textcolor{red}{शेषात्कर्तरि परस्मैपदम्} (पा॰सू॰~१.३.७८)~\arrow \textcolor{red}{अनद्यतने लङ्} (पा॰सू॰~३.२.१११)~\arrow हन्~लङ्~\arrow हन्~तिप्~\arrow हन्~ति~\arrow \textcolor{red}{लुङ्लङ्लृङ्क्ष्वडुदात्तः} (पा॰सू॰~६.४.७१)~\arrow \textcolor{red}{आद्यन्तौ टकितौ} (पा॰सू॰~१.१.४६)~\arrow अट्~हन्~ति~\arrow अ~हन्~ति~\arrow \textcolor{red}{कर्तरि शप्‌} (पा॰सू॰~३.१.६८)~\arrow अ~हन्~शप्~ति~\arrow \textcolor{red}{अदि\-प्रभृतिभ्यः शपः} (पा॰सू॰~२.४.७२)~\arrow अ~हन्~ति~\arrow \textcolor{red}{इतश्च} (पा॰सू॰~३.४.१००)~\arrow अ~हन्~त्~\arrow \textcolor{red}{हल्ङ्याब्भ्यो दीर्घात्सुतिस्यपृक्तं हल्} (पा॰सू॰~६.१.६८)~\arrow अ~हन्~\arrow अहन्।} \textcolor{red}{अहनत्} इत्यत्र \textcolor{red}{गण\-कार्यमनित्यम्} (प॰शे॰~९३.३) इत्यनेन शब्लुगभावे हलन्तत्वाभावान्न तकारलोपः।\footnote{\textcolor{red}{हनँ हिंसागत्योः} (धा॰पा॰~१०१२)~\arrow हन्~\arrow \textcolor{red}{शेषात्कर्तरि परस्मैपदम्} (पा॰सू॰~१.३.७८)~\arrow \textcolor{red}{अनद्यतने लङ्} (पा॰सू॰~३.२.१११)~\arrow हन्~लङ्~\arrow हन्~तिप्~\arrow हन्~ति~\arrow \textcolor{red}{लुङ्लङ्लृङ्क्ष्वडुदात्तः} (पा॰सू॰~६.४.७१)~\arrow \textcolor{red}{आद्यन्तौ टकितौ} (पा॰सू॰~१.१.४६)~\arrow अट्~हन्~ति~\arrow अ~हन्~ति~\arrow \textcolor{red}{कर्तरि शप्‌} (पा॰सू॰~३.१.६८)~\arrow अ~हन्~शप्~ति~\arrow \textcolor{red}{गण\-कार्यमनित्यम्} (प॰शे॰~९३.३)~\arrow शब्लुगभावः~\arrow अ~हन्~अ~ति~\arrow \textcolor{red}{इतश्च} (पा॰सू॰~३.४.१००)~\arrow अ~हन्~अ~त्~\arrow अहनत्।} यद्वा \textcolor{red}{हन्तीति हनः} इति विग्रहे पचादित्वादच्।\footnote{\textcolor{red}{नन्दि\-ग्रहि\-पचादिभ्यो ल्युणिन्यचः} (पा॰सू॰~३.१.१३४) इत्यनेन कर्तरि।} \textcolor{red}{हन इवाऽचरतीति हनति}\footnote{हन~\arrow \textcolor{red}{सर्वप्राति\-पदिकेभ्य आचारे क्विब्वा वक्तव्यः} (वा॰~३.१.११)~\arrow हन~क्विँप्~\arrow हन~व्~\arrow \textcolor{red}{वेरपृक्तस्य} (पा॰सू॰~६.१.६७)~\arrow हन~\arrow \textcolor{red}{सनाद्यन्ता धातवः} (पा॰सू॰~३.१.३२)~\arrow धातुसञ्ज्ञा~\arrow \textcolor{red}{शेषात्कर्तरि परस्मैपदम्} (पा॰सू॰~१.३.७८)~\arrow वर्तमाने लट्~\arrow हन~लट्~\arrow हन~तिप्~\arrow हन~ति~\arrow \textcolor{red}{कर्तरि शप्‌} (पा॰सू॰~३.१.६८)~\arrow हन~शप्~ति~\arrow हन~अ~ति~\arrow \textcolor{red}{अतो गुणे} (पा॰सू॰~६.१.९७)~\arrow हन~ति~\arrow हनति।} इति विग्रह आचार\-क्विबन्ताल्लङ्लकारे प्रथम\-पुरुष एकवचन एवम् \textcolor{red}{अहनत्}।\footnote{हन~\arrow धातुसञ्ज्ञा (पूर्ववत्)~\arrow \textcolor{red}{शेषात्कर्तरि परस्मैपदम्} (पा॰सू॰~१.३.७८)~\arrow \textcolor{red}{अनद्यतने लङ्} (पा॰सू॰~३.२.१११)~\arrow हन्~लङ्~\arrow हन्~तिप्~\arrow हन्~ति~\arrow \textcolor{red}{लुङ्लङ्लृङ्क्ष्वडुदात्तः} (पा॰सू॰~६.४.७१)~\arrow \textcolor{red}{आद्यन्तौ टकितौ} (पा॰सू॰~१.१.४६)~\arrow अट्~हन्~ति~\arrow अ~हन्~ति~\arrow \textcolor{red}{कर्तरि शप्‌} (पा॰सू॰~३.१.६८)~\arrow अ~हन~शप्~ति~\arrow अ~हन~अ~ति~\arrow \textcolor{red}{अतो गुणे} (पा॰सू॰~६.१.९७)~\arrow अ~हन~ति~\arrow \textcolor{red}{इतश्च} (पा॰सू॰~३.४.१००)~\arrow अ~हन~त्~\arrow अहनत्।} श्रीरामो निर्लेपत्वादकर्तृत्वाच्च किमपि न करोति। \textcolor{red}{रामो न गच्छति न तिष्ठति नानुशोचति} इत्यादि राम\-हृदये सीतया स्पष्टं निगदितत्वात्। यथा~–\end{sloppypar}
\centering\textcolor{blue}{रामो न गच्छति न तिष्ठति नानुशोचत्याकाङ्क्षते त्यजति नो न करोति किञ्च।\nopagebreak\\
आनन्दमूर्तिरचलः परिणामहीनो मायागुणाननुगतो हि तथा विभाति॥}\nopagebreak\\
\raggedleft{–~अ॰रा॰~१.१.५३}\\
\fontsize{14}{21}\selectfont\begin{sloppypar}\hyphenrules{nohyphenation}\justifying\noindent एवं \textcolor{red}{रामाभिन्नैक\-कर्तृकानद्यतन\-भूत\-कालावच्छिन्न\-भाण्ड\-कर्मक\-हननानुकूल\-व्यापाराश्रय\-सदृश\-व्यापारः} इति शाब्द\-बोधः।\end{sloppypar}
\section[चक्रे]{चक्रे}
\centering\textcolor{blue}{एवं परात्मा मनुजावतारो मनुष्यलोकाननुसृत्य सर्वम्।\nopagebreak\\
चक्रेऽविकारी परिणामहीनो विचार्यमाणे न करोति किञ्चित्॥}\nopagebreak\\
\raggedleft{–~अ॰रा॰~१.३.६६}\\
\fontsize{14}{21}\selectfont\begin{sloppypar}\hyphenrules{nohyphenation}\justifying\noindent\hspace{10mm} अत्र भगवतः स्वरूपं वर्णयति \textcolor{red}{अकर्ता सन् सर्वं चकार विचार्यमाणे सति किमपि नाकरोत्}। अत्र \textcolor{red}{चकार} इति हि पाणिनीयः।\footnote{\textcolor{red}{डुकृञ् करणे} (धा॰पा॰~१४७२)~\arrow कृ~\arrow \textcolor{red}{शेषात्कर्तरि परस्मैपदम्} (पा॰सू॰~१.३.७८)~\arrow \textcolor{red}{परोक्षे लिट्} (पा॰सू॰~३.२.११५)~\arrow कृ~लिट्~\arrow कृ~तिप्~\arrow कृ~ति~\arrow \textcolor{red}{परस्मैपदानां णलतुसुस्थलथुस\-णल्वमाः} (पा॰सू॰~३.४.८२)~\arrow कृ~णल्~\arrow कृ~अ~\arrow \textcolor{red}{लिटि धातोरनभ्यासस्य} (पा॰सू॰~६.१.८)~\arrow कृ~कृ~अ~\arrow \textcolor{red}{उरत्‌} (पा॰सू॰~७.४.६६)~\arrow \textcolor{red}{उरण् रपरः} (पा॰सू॰~१.१.५१)~\arrow कर्~कृ~अ~\arrow \textcolor{red}{हलादिः शेषः} (पा॰सू॰~७.४.६०)~\arrow क~कृ~अ~\arrow \textcolor{red}{कुहोश्चुः} (पा॰सू॰~७.४.६२)~\arrow च~कृ~अ~\arrow \textcolor{red}{अचो ञ्णिति} (पा॰सू॰~७.२.११५)~\arrow \textcolor{red}{उरण् रपरः} (पा॰सू॰~१.१.५१)~\arrow च~कार्~अ~\arrow चकार।} तच्चरित्रेण भक्तेष्वानन्द\-जननात्क्रिया\-फलस्य पर\-निष्ठत्वात्। किन्तु कर्म\-व्यतिहारेणात्राऽत्मने\-पदम्।\footnote{\textcolor{red}{कर्तरि कर्म\-व्यतिहारे} पा॰सू॰~१.३.१४) इत्यनेन।} साङ्ख्य\-दृष्ट्या भगवति प्रकृति\-गत\-कर्तव्यमारोपितम्। राम\-हृदये श्रीसीतया स्पष्टं प्रतिपादितत्वात्। वेदान्त\-दृष्ट्याऽपि जीव\-गत\-कर्तृत्वं भगवत्यारोपितम्। अत आत्मनेपदम्। तथा च \textcolor{red}{कृ}\-धातोः (\textcolor{red}{डुकृञ् करणे} धा॰पा॰~१४७२) लिड्लकारे \textcolor{red}{परोक्षे लिट्} (पा॰सू॰~३.२.११५) इत्यनेन \textcolor{red}{त}\-प्रत्यये \textcolor{red}{लिटस्तझयोरेशिरेच्} (पा॰सू॰~३.४.८१) इत्यनेन \textcolor{red}{एश्} आदेशे \textcolor{red}{लिटि धातोरनभ्यासस्य} (पा॰सू॰~६.१.८) इत्यनेन द्वित्वे \textcolor{red}{पूर्वोऽभ्यासः} (पा॰सू॰~६.१.४) इत्यनेनाभ्यास\-सञ्ज्ञायां \textcolor{red}{उरत्‌} (पा॰सू॰~७.४.६६) इत्यनेनर्कारस्याकारे \textcolor{red}{उरण् रपरः} (पा॰सू॰~१.१.५१) इत्यनेन रपरत्वे \textcolor{red}{हलादिः शेषः} (पा॰सू॰~७.४.०) इत्यनेन रलोपे \textcolor{red}{कुहोश्चुः} (पा॰सू॰~७.४.६२) इत्यनेन चुत्वे \textcolor{red}{इको यणचि} (पा॰सू॰~६.१.७७) इत्यनेन यणि \textcolor{red}{चक्रे}।\footnote{\textcolor{red}{डुकृञ् करणे} (धा॰पा॰~१४७२)~\arrow कृ~\arrow \textcolor{red}{कर्तरि कर्म\-व्यतिहारे} (पा॰सू॰~१.३.१४)~\arrow \textcolor{red}{परोक्षे लिट्} (पा॰सू॰~३.२.११५)~\arrow कृ~लिट्~\arrow कृ~त~\arrow \textcolor{red}{लिटस्तझयोरेशिरेच्} (पा॰सू॰~३.४.८१)~\arrow कृ~एश्~\arrow कृ~ए~\arrow \textcolor{red}{लिटि धातोरनभ्यासस्य} (पा॰सू॰~६.१.८)~\arrow कृ~कृ~ए~\arrow \textcolor{red}{उरत्‌} (पा॰सू॰~७.४.६६)~\arrow \textcolor{red}{उरण् रपरः} (पा॰सू॰~१.१.५१)~\arrow कर्~कृ~ए~\arrow \textcolor{red}{हलादिः शेषः} (पा॰सू॰~७.४.६०)~\arrow क~कृ~ए~\arrow \textcolor{red}{कुहोश्चुः} (पा॰सू॰~७.४.६२)~\arrow च~कृ~ए~\arrow \textcolor{red}{असंयोगाल्लिट् कित्} (पा॰सू॰~१.२.५)~\arrow कित्त्वम्~\arrow \textcolor{red}{ग्क्ङिति च} (पा॰सू॰~१.१.५)~\arrow गुणनिषेधः~\arrow \textcolor{red}{इको यणचि} (पा॰सू॰~६.१.७७)~\arrow च~क्र्~ए~\arrow चक्रे।} एवं \textcolor{red}{करोति} इत्यपि भूत\-काले प्रयोक्तव्ये वर्तमाने प्रयुक्तं रूपम्। अत्र \textcolor{red}{लट् स्मे} (पा॰सू॰~३.२.११८) इत्यनेन स्म\-योगे लड्लकारो भूतकालेऽपि।\footnote{\textcolor{red}{डुकृञ् करणे} (धा॰पा॰~१४७२)~\arrow कृ~\arrow \textcolor{red}{शेषात्कर्तरि परस्मैपदम्} (पा॰सू॰~१.३.७८)~\arrow \textcolor{red}{लट् स्मे} (पा॰सू॰~३.२.११८)~\arrow कृ~लट्~\arrow कृ~तिप्~\arrow कृ~ति~\arrow \textcolor{red}{तनादि\-कृञ्भ्य उः} (पा॰सू॰~३.१.७९)~\arrow कृ~उ~ति~\arrow \textcolor{red}{सार्वधातुकार्ध\-धातुकयोः} (पा॰सू॰~७.३.८४)~\arrow \textcolor{red}{उरण् रपरः} (पा॰सू॰~१.१.५१)~\arrow कर्~उ~ति~\arrow \textcolor{red}{सार्वधातुकार्ध\-धातुकयोः} (पा॰सू॰~७.३.८४)~\arrow कर्~ओ~ति~\arrow करोति।} न चाऽत्र \textcolor{red}{स्म} इति न दृश्यते। \textcolor{red}{विनाऽपि प्रत्ययं पूर्वोत्तर\-पद\-लोपो वक्तव्यः} (वा॰~५.३.८३) इत्यनेन लोपात्।\end{sloppypar}
\section[करोमि]{करोमि}
\centering\textcolor{blue}{त्वद्विधा यद्गृहं यान्ति तत्रैवाऽयान्ति सम्पदः।\nopagebreak\\
यदर्थमागतोऽसि त्वं ब्रूहि सत्यं करोमि तत्॥}\nopagebreak\\
\raggedleft{–~अ॰रा॰~१.४.४}\\
\fontsize{14}{21}\selectfont\begin{sloppypar}\hyphenrules{nohyphenation}\justifying\noindent\hspace{10mm} अत्र \textcolor{red}{करिष्यामि}\footnote{\textcolor{red}{डुकृञ् करणे} (धा॰पा॰~१४७२)~\arrow कृ~\arrow \textcolor{red}{शेषात्कर्तरि परस्मैपदम्} (पा॰सू॰~१.३.७८)~\arrow \textcolor{red}{लृट् शेषे च} (पा॰सू॰~३.३.१३)~\arrow कृ~लृट्~\arrow कृ~मिप्~\arrow कृ~मि~\arrow \textcolor{red}{स्यतासी लृलुटोः} (पा॰सू॰~३.१.३३)~\arrow कृ~स्य~मि~\arrow \textcolor{red}{ऋद्धनोः स्ये} (पा॰सू॰~७.२.७०)~\arrow कृ~इट्~स्य~मि~\arrow कृ~इ~स्य~मि~\arrow \textcolor{red}{सार्वधातुकार्ध\-धातुकयोः} (पा॰सू॰~७.३.८४)~\arrow \textcolor{red}{उरण् रपरः} (पा॰सू॰~१.१.५१)~\arrow कर्~इ~स्य~मि~\arrow \textcolor{red}{अतो दीर्घो यञि} (पा॰सू॰~७.३.१०१)~\arrow कर्~इ~स्या~मि~\arrow \textcolor{red}{आदेश\-प्रत्यययोः} (पा॰सू॰~८.३.५९)~\arrow कर्~इ~ष्या~मि~\arrow करिष्यामि।} इति प्रयोक्तव्ये वर्तमान\-सामीप्य\-लकारात् \textcolor{red}{करोमि}।\footnote{\textcolor{red}{वर्तमान\-सामीप्ये वर्तमानवद्वा} (पा॰सू॰~३.३.१३१) इत्यनेन। \textcolor{red}{डुकृञ् करणे} (धा॰पा॰~१४७२)~\arrow कृ~\arrow \textcolor{red}{शेषात्कर्तरि परस्मैपदम्} (पा॰सू॰~१.३.७८)~\arrow \textcolor{red}{वर्तमान\-सामीप्ये वर्तमानवद्वा} (पा॰सू॰~३.३.१३१)~\arrow \textcolor{red}{वर्तमाने लट्} (पा॰सू॰~३.२.१२३)~\arrow कृ~लट्~\arrow कृ~मिप्~\arrow कृ~मि~\arrow \textcolor{red}{तनादि\-कृञ्भ्य उः} (पा॰सू॰~३.१.७९)~\arrow कृ~उ~मि~\arrow \textcolor{red}{सार्वधातुकार्ध\-धातुकयोः} (पा॰सू॰~७.३.८४)~\arrow \textcolor{red}{उरण् रपरः} (पा॰सू॰~१.१.५१)~\arrow कर्~उ~मि~\arrow \textcolor{red}{सार्वधातुकार्ध\-धातुकयोः} (पा॰सू॰~७.३.८४)~\arrow कर्~ओ~मि~\arrow करोमि।}\end{sloppypar}
\section[किं करोमि]{किं करोमि}
\centering\textcolor{blue}{किं करोमि गुरो रामं त्यक्तुं नोत्सहते मनः।\nopagebreak\\
बहुवर्षसहस्रान्ते कष्टेनोत्पादिताः सुताः॥}\nopagebreak\\
\raggedleft{–~अ॰रा॰~१.४.९}\\
\fontsize{14}{21}\selectfont\begin{sloppypar}\hyphenrules{nohyphenation}\justifying\noindent\hspace{10mm} अत्र \textcolor{red}{किंवृत्ते लिप्सायाम्} (पा॰सू॰~३.३.६) इत्यनेन लट्।\footnote{\textcolor{red}{डुकृञ् करणे} (धा॰पा॰~१४७२)~\arrow कृ~\arrow \textcolor{red}{शेषात्कर्तरि परस्मैपदम्} (पा॰सू॰~१.३.७८)~\arrow \textcolor{red}{किंवृत्ते लिप्सायाम्} (पा॰सू॰~३.३.६)~\arrow कृ~लट्~\arrow कृ~मिप्~\arrow कृ~मि~\arrow \textcolor{red}{तनादि\-कृञ्भ्य उः} (पा॰सू॰~३.१.७९)~\arrow कृ~उ~मि~\arrow \textcolor{red}{सार्वधातुकार्ध\-धातुकयोः} (पा॰सू॰~७.३.८४)~\arrow \textcolor{red}{उरण् रपरः} (पा॰सू॰~१.१.५१)~\arrow कर्~उ~मि~\arrow \textcolor{red}{सार्वधातुकार्ध\-धातुकयोः} (पा॰सू॰~७.३.८४)~\arrow कर्~ओ~मि~\arrow करोमि।}\end{sloppypar}
\section[न जीवामि]{न जीवामि}
\centering\textcolor{blue}{चत्वारोऽमरतुल्यास्ते तेषां रामोऽतिवल्लभः।\nopagebreak\\
रामस्त्वितो गच्छति चेन्न जीवामि कथञ्चन॥}\nopagebreak\\
\raggedleft{–~अ॰रा॰~१.४.१०}\\
\fontsize{14}{21}\selectfont\begin{sloppypar}\hyphenrules{nohyphenation}\justifying\noindent\hspace{10mm} \textcolor{red}{न जीवामि} इति वर्तमान\-सामीप्याल्लट्।\footnote{\textcolor{red}{वर्तमान\-सामीप्ये वर्तमानवद्वा} (पा॰सू॰~३.३.१३१) इत्यनेन। \textcolor{red}{जीवँ प्राणधारणे} (धा॰पा॰~५६२)~\arrow जीव्~\arrow \textcolor{red}{शेषात्कर्तरि परस्मैपदम्} (पा॰सू॰~१.३.७८)~\arrow \textcolor{red}{वर्तमान\-सामीप्ये वर्तमानवद्वा} (पा॰सू॰~३.३.१३१)~\arrow \textcolor{red}{वर्तमाने लट्} (पा॰सू॰~३.२.१२३)~\arrow जीव्~लट्~\arrow जीव्~मिप्~\arrow जीव्~मि~\arrow \textcolor{red}{कर्तरि शप्‌} (पा॰सू॰~३.१.६८)~\arrow जीव्~शप्~मि~\arrow जीव्~अ~मि~\arrow \textcolor{red}{अतो दीर्घो यञि} (पा॰सू॰~७.३.१०१)~\arrow जीव्~आ~मि~\arrow जीवामि।} \textcolor{red}{सद्यो मरिष्यामि} इति दशरथस्य तात्पर्यम्।\end{sloppypar}
\section[तेपाथे]{तेपाथे}
\centering\textcolor{blue}{त्वं तु प्रजापतिः पूर्वं कश्यपो ब्रह्मणः सुतः।\nopagebreak\\
कौसल्या चादितिर्देवमाता पूर्वं यशस्विनी।\nopagebreak\\
भवन्तौ तप उग्रं वै तेपाथे बहुवत्सरम्॥}\nopagebreak\\
\raggedleft{–~अ॰रा॰~१.४.१४}\\
\fontsize{14}{21}\selectfont\begin{sloppypar}\hyphenrules{nohyphenation}\justifying\noindent\hspace{10mm} अत्र वसिष्ठो दशरथस्य पूर्व\-जन्म स्मारयति। अत्र \textcolor{red}{तप्‌}\-धातुः (\textcolor{red}{तपँ दाहे ऐश्वर्ये वा} धा॰पा॰~११५९) दिवादिरात्मनेपदीयो न तु भ्वादिः परस्मैपदी (\textcolor{red}{तपँ सन्तापे} धा॰पा॰~९८५)।\footnote{भौवादिकाद्धातोस्तु \textcolor{red}{तेपथुः} इति रूपम्। \textcolor{red}{तपँ सन्तापे} (धा॰पा॰~९८५)~\arrow तप्~\arrow \textcolor{red}{शेषात्कर्तरि परस्मैपदम्} (पा॰सू॰~१.३.७८)~\arrow \textcolor{red}{परोक्षे लिट्} (पा॰सू॰~३.२.११५)~\arrow तप्~लिट्~\arrow तप्~थस्~\arrow \textcolor{red}{परस्मैपदानां णलतुसुस्थलथुस\-णल्वमाः} (पा॰सू॰~३.४.८२)~\arrow तप्~अथुस्~\arrow \textcolor{red}{लिटि धातोरनभ्यासस्य} (पा॰सू॰~६.१.८)~\arrow तप्~तप्~अथुस्~\arrow \textcolor{red}{हलादिः शेषः} (पा॰सू॰~७.४.६०)~\arrow त~तप्~अथुस्~\arrow \textcolor{red}{अत एकहल्मध्येऽनादेशादेर्लिटि} (पा॰सू॰~६.४.१२०)~\arrow तेप्~अथुस्~\arrow तेपथुः।} तस्माल्लिटि \textcolor{red}{आथाम्} प्रत्यये द्वित्वेऽभ्यास\-कार्ये \textcolor{red}{अत एकहल्मध्येऽनादेशादेर्लिटि} (पा॰सू॰~६.४.१२०) इत्यनेनाभ्यास\-लोप एकार एत्वे च \textcolor{red}{तेपाथे} इति साधु।\footnote{\textcolor{red}{तपँ दाहे ऐश्वर्ये वा} (धा॰पा॰~११५९)~\arrow तप्~\arrow \textcolor{red}{अनुदात्तङित आत्मने\-पदम्} (पा॰सू॰~१.३.१२)~\arrow \textcolor{red}{परोक्षे लिट्} (पा॰सू॰~३.२.११५)~\arrow तप्~लिट्~\arrow तप्~आथाम्~\arrow \textcolor{red}{लिटि धातोरनभ्यासस्य} (पा॰सू॰~६.१.८)~\arrow तप्~तप्~आथाम्~\arrow \textcolor{red}{हलादिः शेषः} (पा॰सू॰~७.४.६०)~\arrow त~तप्~आथाम्~\arrow \textcolor{red}{अत एकहल्मध्येऽनादेशादेर्लिटि} (पा॰सू॰~६.४.१२०)~\arrow तेप्~आथाम्~\arrow \textcolor{red}{टित आत्मनेपदानां टेरे} (पा॰सू॰~३.४.७९)~\arrow तेप्~आथे~\arrow तेपाथे। न चात्र \textcolor{red}{तप्‌}\-धातुश्चुरादिराधृषीय उभयपदी (\textcolor{red}{तपँ दाहे} धा॰पा॰~१८१९)। तस्माल्लिटि परस्मैपदे णिच्पक्षे \textcolor{red}{तापयाञ्चक्रथुः तापयाम्बभूवथुः तापयामासथुः} इति रूपाणि णिजभाव\-पक्षे \textcolor{red}{तेपथुः} इति रूपमात्मने\-पदे च णिच्पक्षे \textcolor{red}{तापयाञ्चक्राथे} इति रूपं णिजभाव\-पक्षे \textcolor{red}{तेपाथे} इति रूपम्। अत्र क्रियाफलं रामावतरणम्। तन्नादिति\-कश्यप\-कर्तृगाम्यपि तु परगामि। सकल\-संसार\-कल्याणकारित्वात्। रामावतारेण साधूनां परित्राणं दुष्कृतानां विनाशो धर्मस्य संस्थापना च। तेन \textcolor{red}{स्वरितञितः कर्त्रभिप्राये क्रियाफले} (पा॰सू॰~१.३.७२) इत्यस्याप्रवृत्तिरत्र। विस्तार\-भिया प्रक्रिया न दीयन्ते।}\end{sloppypar}
\section[प्रेषयस्व]{प्रेषयस्व}
\centering\textcolor{blue}{अतः प्रीतेन मनसा पूजयित्वाथ कौशिकम्।\nopagebreak\\
प्रेषयस्व रमानाथं राघवं सहलक्ष्मणम्॥}\nopagebreak\\
\raggedleft{–~अ॰रा॰~१.४.२०}\\
\fontsize{14}{21}\selectfont\begin{sloppypar}\hyphenrules{nohyphenation}\justifying\noindent\hspace{10mm} वसिष्ठोऽनुजानाति यत् \textcolor{red}{रमा\-नाथं श्रीराम\-भद्रं विश्वामित्राय प्रयच्छ}। अत्र \textcolor{red}{प्रेषयस्व} इति प्रयुक्तं \textcolor{red}{प्रेषय} इति प्रयोक्तव्यम्।\footnote{\textcolor{red}{प्रेषृँ गतौ} (धा॰पा॰~६१९)~\arrow प्रेष्~\arrow \textcolor{red}{हेतुमति च} (पा॰सू॰~३.१.२६)~\arrow प्रेष्~णिच्~\arrow प्रेष्~इ~\arrow प्रेषि~\arrow \textcolor{red}{सनाद्यन्ता धातवः} (पा॰सू॰~३.१.३२)~\arrow धातुसञ्ज्ञा~\arrow \textcolor{red}{शेषात्कर्तरि परस्मैपदम्} (पा॰सू॰~१.३.७८)~\arrow \textcolor{red}{लोट् च} (पा॰सू॰~३.३.१६२)~\arrow प्रेषि~लोट्~\arrow प्रेषि~सिप्~\arrow प्रेषि~सि~\arrow \textcolor{red}{कर्तरि शप्} (पा॰सू॰~३.१.६८)~\arrow प्रेषि~शप्~सि~\arrow प्रेषि~अ~सि~\arrow \textcolor{red}{सार्वधातुकार्धधातुकयोः} (पा॰सू॰~७.३.८४)~\arrow प्रेषे~अ~सि~\arrow \textcolor{red}{एचोऽयवायावः} (पा॰सू॰~६.१.७८)~\arrow प्रेषय्~अ~सि~\arrow \textcolor{red}{सेर्ह्यपिच्च} (पा॰सू॰~३.४.८७)~\arrow प्रेषय्~अ~हि~\arrow \textcolor{red}{अतो हेः} (पा॰सू॰~६.४.१०५)~\arrow प्रेषय्~अ~\arrow प्रेषय।} यतो हि \textcolor{red}{णिचश्च} (पा॰सू॰~१.३.७४) इत्यात्मनेपदम्।\footnote{\textcolor{red}{प्रेषृँ गतौ} (धा॰पा॰~६१९)~\arrow प्रेष्~\arrow \textcolor{red}{हेतुमति च} (पा॰सू॰~३.१.२६)~\arrow प्रेष्~णिच्~\arrow प्रेष्~इ~\arrow प्रेषि~\arrow धातुसञ्ज्ञा~\arrow \textcolor{red}{णिचश्च} (पा॰सू॰~१.३.७४)~\arrow \textcolor{red}{लोट् च} (पा॰सू॰~३.३.१६२)~\arrow प्रेषि~लोट्~\arrow प्रेषि~थास्~\arrow \textcolor{red}{कर्तरि शप्} (पा॰सू॰~३.१.६८)~\arrow प्रेषि~शप्~थास्~\arrow प्रेषि~अ~थास्~\arrow \textcolor{red}{सार्वधातुकार्ध\-धातुकयोः} (पा॰सू॰~७.३.८४)~\arrow प्रेषे~अ~थास्~\arrow \textcolor{red}{एचोऽयवायावः} (पा॰सू॰~६.१.७८)~\arrow प्रेषय्~अ~थास्~\arrow \textcolor{red}{थासस्से} (पा॰सू॰~३.४.८०)~\arrow प्रेषय्~अ~से~\arrow \textcolor{red}{सवाभ्यां वामौ} (पा॰सू॰~३.४.९१)~\arrow प्रेषय्~अ~स्व~\arrow प्रेषयस्व।} तदपि क्रिया\-फले कर्तृ\-गामिनि सति। \textcolor{red}{राघव\-प्रेषणेन त्वामपि यशो\-रूपं फलं मिलिष्यति} इति तात्पर्यादत्राऽत्मनेपदम्। यद्वा \textcolor{red}{स्व} इति सम्बोधनम्। \textcolor{red}{हे स्व आत्मीय रमा\-नाथं रामं प्रेषय} इति नापाणिनीयता।\end{sloppypar}
\section[दर्शयस्व]{दर्शयस्व}
\centering\textcolor{blue}{दर्शयस्व महाभाग कुतस्तौ राक्षसाधमौ।\nopagebreak\\
तथेत्युक्त्वा मुनिर्यष्टुमारेभे मुनिभिः सह॥}\nopagebreak\\
\raggedleft{–~अ॰रा॰~१.५.४}\\
\fontsize{14}{21}\selectfont\begin{sloppypar}\hyphenrules{nohyphenation}\justifying\noindent\hspace{10mm} श्रीरामो राक्षसौ प्रति पृच्छन् \textcolor{red}{दर्शयस्व} इति प्रयुङ्क्ते। अत्र राक्षस\-निधनेन दर्शन\-कारयितरि विश्वामित्रे फलमिति क्रिया\-फलस्य कर्तृ\-गामित्वादात्मने\-पदम्।\footnote{\textcolor{red}{णिचश्च} (पा॰सू॰~१.३.७४) इत्यनेन। \textcolor{red}{दृशिँर् प्रेक्षणे} (धा॰पा॰~९८८)~\arrow दृश्~\arrow \textcolor{red}{हेतुमति च} (पा॰सू॰~३.१.२६)~\arrow दृश्~णिच्~\arrow दृश्~इ~\arrow \textcolor{red}{पुगन्त\-लघूपधस्य च} (पा॰सू॰~७.३.८६)~\arrow \textcolor{red}{उरण् रपरः} (पा॰सू॰~१.१.५१)~\arrow दर्श्~इ~\arrow दर्शि~\arrow \textcolor{red}{सनाद्यन्ता धातवः} (पा॰सू॰~३.१.३२)~\arrow धातुसञ्ज्ञा~\arrow \textcolor{red}{णिचश्च} (पा॰सू॰~१.३.७४)~\arrow \textcolor{red}{लोट् च} (पा॰सू॰~३.३.१६२)~\arrow दर्शि~लोट्~\arrow दर्शि~थास्~\arrow \textcolor{red}{कर्तरि शप्} (पा॰सू॰~३.१.६८)~\arrow दर्शि~शप्~थास्~\arrow दर्शि~अ~थास्~\arrow \textcolor{red}{सार्वधातुकार्धधातुकयोः} (पा॰सू॰~७.३.८४)~\arrow दर्शे~अ~थास्~\arrow \textcolor{red}{एचोऽयवायावः} (पा॰सू॰~६.१.७८)~\arrow दर्शय्~अ~थास्~\arrow \textcolor{red}{थासस्से} (पा॰सू॰~३.४.८०)~\arrow दर्शय्~अ~से~\arrow \textcolor{red}{सवाभ्यां वामौ} (पा॰सू॰~३.४.९१)~\arrow दर्शय्~अ~स्व~\arrow दर्शयस्व।}\end{sloppypar}
\section[ददृशाते]{ददृशाते}
\centering\textcolor{blue}{मध्याह्ने ददृशाते तौ राक्षसौ कामरूपिणौ।\nopagebreak\\
मारीचश्च सुबाहुश्च वर्षन्तौ रुधिरास्थिनी॥}\nopagebreak\\
\raggedleft{–~अ॰रा॰~१.५.५}\\
\fontsize{14}{21}\selectfont\begin{sloppypar}\hyphenrules{nohyphenation}\justifying\noindent\hspace{10mm} \textcolor{red}{ददृशाते} इति कर्म\-वाच्य\-प्रयोगः।\footnote{अत्र \textcolor{red}{कर्तरि कर्म\-व्यतिहारे} (पा॰सू॰~१.३.१४) इत्यनेन कर्तर्यात्मनेपदं न वेत्याशङ्कां परिहर्तुं विमर्शः प्रारब्धः। \textcolor{red}{दृशिँर् प्रेक्षणे} (धा॰पा॰~९८८)~\arrow दृश्~\arrow \textcolor{red}{भावकर्मणोः} (पा॰सू॰~१.३.१३)~\arrow \textcolor{red}{परोक्षे लिट्} (पा॰सू॰~३.२.११५)~\arrow दृश्~लिट्~\arrow दृश्~आताम्~\arrow \textcolor{red}{लिटि धातोरनभ्यासस्य} (पा॰सू॰~६.१.८)~\arrow दृश्~दृश्~आताम्~\arrow \textcolor{red}{उरत्‌} (पा॰सू॰~७.४.६६)~\arrow \textcolor{red}{उरण् रपरः} (पा॰सू॰~१.१.५१)~\arrow दर्श्~दृश्~आताम्~\arrow \textcolor{red}{हलादिः शेषः} (पा॰सू॰~७.४.६०)~\arrow द~दृश्~आताम्~\arrow \textcolor{red}{टित आत्मनेपदानां टेरे} (पा॰सू॰~३.४.७९)~\arrow द~दृश्~आते~\arrow \textcolor{red}{असंयोगाल्लिट् कित्} (पा॰सू॰~१.२.५)~\arrow कित्त्वम्~\arrow \textcolor{red}{ग्क्ङिति च} (पा॰सू॰~१.१.५)~\arrow गुणनिषेधः~\arrow ददृशाते।} \textcolor{red}{राम\-लक्ष्मणाभ्याम्} इति शेषः। अत्र \textcolor{red}{परोक्षे लिट्} (पा॰सू॰~३.२.११५) इत्यनेन लिड्लकारः। प्रथम\-पुरुष\-द्वि\-वचने रूपम्। परोक्षत्वं नाम \textcolor{red}{साक्षात्करोमीति क्रियाशालि\-ज्ञानाविषयतावच्छेदकत्वम्}।\footnote{\textcolor{red}{परोक्षत्वं च साक्षात्करोमीत्येतादृश\-विषयता\-शालि\-ज्ञानाविषयत्वम्} (वै॰भू॰सा॰~२२)।}\end{sloppypar}
\section[गच्छामहे]{गच्छामहे}
\centering\textcolor{blue}{चतुर्थेऽहनि संप्राप्ते कौशिको राममब्रवीत्।\nopagebreak\\
राम राम महायज्ञं द्रष्टुं गच्छामहे वयम्॥}\nopagebreak\\
\raggedleft{–~अ॰रा॰~१.५.१२}\\
\fontsize{14}{21}\selectfont\begin{sloppypar}\hyphenrules{nohyphenation}\justifying\noindent\hspace{10mm} ताटका\-मारीच\-सुबाहून् व्यापाद्य यज्ञ\-रक्षां विधाय दिन\-त्रयं यावत्तत्र प्रोष्यायोध्यां प्रति गन्तुकामं लक्ष्मणाभिरामं श्रीरामं विश्वामित्रो मिथिलां प्रस्थातुं समामन्त्रयत्। तत्रैव प्रयुक्तम् \textcolor{red}{गच्छामहे}। इदं कथम्। \textcolor{red}{गमॢँ}\-धातुः (\textcolor{red}{गमॢँ गतौ} धा॰पा॰~९८२) परस्मैपदी। ततो लटि मसि शपि \textcolor{red}{अतो दीर्घो यञि} (पा॰सू॰~७.३.१०१) इत्यनेन दीर्घे \textcolor{red}{इषुगमियमां छः} (पा॰सू॰~७.३.७७) इत्यनेन छत्वे \textcolor{red}{छे च} (पा॰सू॰~६.१.७३) इत्यनेन तुकि \textcolor{red}{स्तोः श्चुना श्चुः} (पा॰सू॰~८.४.४०) इत्यनेन श्चुत्वे विसर्गे \textcolor{red}{गच्छामः} इति पाणिनीयम्।\footnote{\textcolor{red}{गमॢँ गतौ} (धा॰पा॰~९८२)~\arrow गम्~\arrow \textcolor{red}{शेषात्कर्तरि परस्मैपदम्} (पा॰सू॰~१.३.७८)~\arrow \textcolor{red}{वर्तमाने लट्} (पा॰सू॰~३.२.१२३)~\arrow गम्~लट्~\arrow गम्~मस्~\arrow \textcolor{red}{कर्तरि शप्‌} (पा॰सू॰~३.१.६८)~\arrow गम्~शप्~मस्~\arrow गम्~अ~मस्~\arrow \textcolor{red}{इषुगमियमां छः} (पा॰सू॰~७.३.७७)~\arrow गछ्~अ~मस्~\arrow \textcolor{red}{छे च} (पा॰सू॰~६.१.७३)~\arrow \textcolor{red}{आद्यन्तौ टकितौ} (पा॰सू॰~१.१.४६)~\arrow गतुँक्~छ्~अ~मस्~\arrow गत्~छ्~अ~मस्~\arrow \textcolor{red}{स्तोः श्चुना श्चुः} (पा॰सू॰~८.४.४०)~\arrow गच्~छ्~अ~मस्~\arrow \textcolor{red}{अतो दीर्घो यञि} (पा॰सू॰~७.३.१०१)~\arrow गच्~छ्~आ~मस्~\arrow \textcolor{red}{ससजुषो रुः} (पा॰सू॰~८.२.६६)~\arrow गच्~छ्~आ~मरुँ~\arrow \textcolor{red}{खरवसानयोर्विसर्जनीयः} (पा॰सू॰~८.३.१५)~\arrow गच्~छ्~आ~मः~\arrow गच्छामः।} \textcolor{red}{गच्छामहे} इति \textcolor{red}{सम्} उपसर्ग\-संयोजने \textcolor{red}{समो गम्यृच्छिभ्याम्} (पा॰सू॰~१.३.२९) इत्यात्मने\-पदे \textcolor{red}{महिङ्} प्रत्यये ङकारानुबन्ध\-कार्ये \textcolor{red}{टित आत्मनेपदानां टेरे} (पा॰सू॰~३.४.७९) इत्यनेन चैत्वे \textcolor{red}{सङ्गच्छामहे}।\footnote{सम्~\textcolor{red}{गमॢँ गतौ} (धा॰पा॰~९८२)~\arrow सम्~गम्~\arrow \textcolor{red}{समो गम्यृच्छिभ्याम्} (पा॰सू॰~१.३.२९)~\arrow \textcolor{red}{वर्तमान\-सामीप्ये वर्तमानवद्वा} (पा॰सू॰~३.३.१३१)~\arrow \textcolor{red}{वर्तमाने लट्} (पा॰सू॰~३.२.१२३)~\arrow सम्~गम्~लट्~\arrow सम्~गम्~महिङ्~\arrow सम्~गम्~महि~\arrow \textcolor{red}{कर्तरि शप्‌} (पा॰सू॰~३.१.६८)~\arrow सम्~गम्~शप्~महि~\arrow सम्~गम्~अ~महि~\arrow \textcolor{red}{इषुगमियमां छः} (पा॰सू॰~७.३.७७)~\arrow सम्~गछ्~अ~महि~\arrow \textcolor{red}{छे च} (पा॰सू॰~६.१.७३)~\arrow \textcolor{red}{आद्यन्तौ टकितौ} (पा॰सू॰~१.१.४६)~\arrow सम्~गतुँक्~छ्~अ~महि~\arrow सम्~गत्~छ्~अ~महि~\arrow \textcolor{red}{स्तोः श्चुना श्चुः} (पा॰सू॰~८.४.४०)~\arrow सम्~गच्~छ्~अ~महि~\arrow \textcolor{red}{अतो दीर्घो यञि} (पा॰सू॰~७.३.१०१)~\arrow सम्~गच्~छ्~आ~महि~\arrow \textcolor{red}{टित आत्मनेपदानां टेरे} (पा॰सू॰~३.४.७९)~\arrow सम्~गच्~छ्~आ~महे~\arrow \textcolor{red}{मोऽनुस्वारः} (पा॰सू॰~८.३.२३)~\arrow सं~गच्~छ्~आ~महे~\arrow \textcolor{red}{वा पदान्तस्य} (पा॰सू॰~८.४.५९)~\arrow सङ्~गच्~छ्~आ~महे~\arrow सङ्गच्छामहे।} \textcolor{red}{यज्ञं द्रष्टुं सङ्गता भवामः}। वर्तमान\-सामीप्याल्लट्।\footnote{\textcolor{red}{वर्तमान\-सामीप्ये वर्तमानवद्वा} (पा॰सू॰~३.३.१३१) इत्यनेन।} उपसर्ग\-लोपस्तु \textcolor{red}{विनाऽपि प्रत्ययं पूर्वोत्तर\-पद\-लोपो वक्तव्यः} (वा॰~५.३.८३) इति वचनात्। यद्वा \textcolor{red}{हे} इति पृथक्पदम्। \textcolor{red}{गच्छाम} पदं लोड्लकारोत्तम\-पुरुषस्य बहुवचनस्य।\footnote{\textcolor{red}{गमॢँ गतौ} (धा॰पा॰~९८२)~\arrow गम्~\arrow \textcolor{red}{शेषात्कर्तरि परस्मैपदम्} (पा॰सू॰~१.३.७८)~\arrow \textcolor{red}{इच्छार्थेषु लिङ्लोटौ} (पा॰सू॰~३.३.१५७)~\arrow गम्~लोट्~\arrow गम्~मस्~\arrow \textcolor{red}{कर्तरि शप्‌} (पा॰सू॰~३.१.६८)~\arrow गम्~शप्~मस्~\arrow गम्~अ~मस्~\arrow \textcolor{red}{इषुगमियमां छः} (पा॰सू॰~७.३.७७)~\arrow गछ्~अ~मस्~\arrow \textcolor{red}{छे च} (पा॰सू॰~६.१.७३)~\arrow \textcolor{red}{आद्यन्तौ टकितौ} (पा॰सू॰~१.१.४६)~\arrow गतुँक्~छ्~अ~मस्~\arrow गत्~छ्~अ~मस्~\arrow \textcolor{red}{स्तोः श्चुना श्चुः} (पा॰सू॰~८.४.४०)~\arrow गच्~छ्~अ~मस्~\arrow \textcolor{red}{आडुत्तमस्य पिच्च} (पा॰सू॰~३.४.९२)~\arrow गच्~छ्~अ~आट्~मस्~\arrow \textcolor{red}{अकः सवर्णे दीर्घः} (पा॰सू॰~६.१.१०१)~\arrow गच्~छ्~आ~मस्~\arrow \textcolor{red}{लोटो लङ्वत्‌} (पा॰सू॰~३.४.८५)~\arrow ङिद्वत्त्वम्~\arrow \textcolor{red}{नित्यं ङितः} (पा॰सू॰~३.४.९९)~\arrow गच्~छ्~आ~म~\arrow गच्छाम।} \textcolor{red}{हे राम हे राम महायज्ञं द्रष्टुं वयं गच्छामेतीच्छामः} इत्थमन्वये \textcolor{red}{इच्छार्थेषु लिङ्लोटौ} (पा॰सू॰~३.३.१५७) इत्यनेन \textcolor{red}{इच्छामः} इत्युपपदे लोड्लकारे कृते मस्प्रत्यये \textcolor{red}{आडुत्तमस्य पिच्च} (पा॰सू॰~३.४.९२) इत्यनेनाऽडागमे \textcolor{red}{लोटो लङ्वत्} (पा॰सू॰~३.४.८५) इत्यनेन लङ्वद्भावे \textcolor{red}{नित्यं ङितः} (पा॰सू॰~३.४.९९) इत्यनेन सकार\-लोपे \textcolor{red}{गच्छाम हे}। \textcolor{red}{हे राम, इच्छामो यन्महायज्ञं द्रष्टुं गच्छाम} इति योजना।\footnote{\textcolor{red}{इच्छामः} इत्यध्याहार्यमिति भावः। ततः \textcolor{red}{इच्छार्थेषु लिङ्लोटौ} (पा॰सू॰~३.३.१५७) इत्यनेन सर्वलकारापवादेन लोट्।}\end{sloppypar}
\section[पूज्यसे]{पूज्यसे}
\centering\textcolor{blue}{द्रक्ष्यसि त्वं महासत्त्वं पूज्यसे जनकेन च।\nopagebreak\\
इत्युक्त्वा मुनिभिस्ताभ्यां ययौ गङ्गासमीपगम्॥}\nopagebreak\\
\raggedleft{–~अ॰रा॰~१.५.१४}\\
\fontsize{14}{21}\selectfont\begin{sloppypar}\hyphenrules{nohyphenation}\justifying\noindent\hspace{10mm} \textcolor{red}{द्रक्ष्यसि} इति समभिव्याहारेण \textcolor{red}{पूजयिष्यसे}\footnote{\textcolor{red}{पूजँ पूजायाम्} (धा॰पा॰~१६४२)~\arrow पूज्~\arrow \textcolor{red}{सत्याप\-पाश\-रूप\-वीणा\-तूल\-श्लोक\-सेना\-लोम\-त्वच\-वर्म\-वर्ण\-चूर्ण\-चुरादिभ्यो णिच्} (पा॰सू॰~३.१.२५)~\arrow पूज्~णिच्~\arrow पूज्~इ~\arrow पूजि~\arrow \textcolor{red}{सनाद्यन्ता धातवः} (पा॰सू॰~३.१.३२)~\arrow धातु\-सञ्ज्ञा~\arrow \textcolor{red}{भावकर्मणोः} (पा॰सू॰~१.३.१३)~\arrow \textcolor{red}{लृट् शेषे च} (पा॰सू॰~३.३.१३)~\arrow पूजि~लृट्~\arrow पूजि~थास्~\arrow \textcolor{red}{स्यतासी लृलुटोः} (पा॰सू॰~३.१.३३)~\arrow पूजि~स्य~थास्~\arrow \textcolor{red}{आर्धधातुकस्येड्वलादेः} (पा॰सू॰~७.२.३५)~\arrow पूजि~इट्~स्य~थास्~\arrow पूजि~इ~स्य~थास्~\arrow \textcolor{red}{सार्वधातुकार्धधातुकयोः} (पा॰सू॰~७.३.८४)~\arrow पूजे~इ~स्य~थास्~\arrow \textcolor{red}{एचोऽयवायावः} (पा॰सू॰~६.१.७८)~\arrow पूजय्~इ~स्य~थास्~\arrow \textcolor{red}{थासस्से} (पा॰सू॰~३.४.८०)~\arrow पूजय्~इ~स्य~से~\arrow \textcolor{red}{आदेश\-प्रत्यययोः} (पा॰सू॰~८.३.५९)~\arrow पूजय्~इ~ष्य~से~\arrow पूजयिष्यसे।} इति हि पाणिनीय\-प्रयोगः स्याच्चेत्सङ्गतिः स्यात्। किन्तु \textcolor{red}{पूज्यसे} इति प्रयोगे त्वसङ्गतिरेकत्र \textcolor{red}{द्रक्ष्यसि} इति भविष्यत्प्रयोगोऽपरत्र \textcolor{red}{पूज्यसे} इति वर्तमान\-कालिक इति चेत्। वर्तमान\-सामीप्याद्भविष्यत्काले लटि नासङ्गतिः।\footnote{\textcolor{red}{वर्तमान\-सामीप्ये वर्तमानवद्वा} (पा॰सू॰~३.३.१३१) इत्यनेन। पूजि~\arrow धातु\-सञ्ज्ञा (पूर्ववत्)~\arrow \textcolor{red}{भावकर्मणोः} (पा॰सू॰~१.३.१३)~\arrow \textcolor{red}{वर्तमान\-सामीप्ये वर्तमानवद्वा} (पा॰सू॰~३.३.१३१)~\arrow \textcolor{red}{वर्तमाने लट्} (पा॰सू॰~३.२.१२३)~\arrow पूजि~लट्~\arrow पूजि~थास्~\arrow \textcolor{red}{सार्वधातुके यक्} (पा॰सू॰~३.१.६७)~\arrow पूजि~यक्~थास्~\arrow पूजि~य~थास्~\arrow \textcolor{red}{णेरनिटि} (पा॰सू॰~६.४.५१)~\arrow पूज्~य~थास्~\arrow \textcolor{red}{थासस्से} (पा॰सू॰~३.४.८०)~\arrow पूज्~य~से~\arrow पूज्यसे।} किमर्थमिदमिति चेत्। \textcolor{red}{त्वं यदा महासत्त्वं द्रक्ष्यसि ततो धनुषि भग्ने शीघ्रमेव जनकेन पूजयिष्यसे} इति शीघ्रतां ध्वनयितुं भविष्यति वर्तमान\-वत्कार्यम्।\end{sloppypar}
\section[आगमिष्यति]{आगमिष्यति}
\centering\textcolor{blue}{एवं वर्षसहस्रेषु ह्यनेकेषु गतेषु च।\nopagebreak\\
रामो दाशरथिः श्रीमानागमिष्यति सानुजः॥}\nopagebreak\\
\raggedleft{–~अ॰रा॰~१.५.३०}\\
\fontsize{14}{21}\selectfont\begin{sloppypar}\hyphenrules{nohyphenation}\justifying\noindent\hspace{10mm} अत्र यद्यप्यनद्य\-तनत्वाल्लुड्लकारः\footnote{लुटि \textcolor{red}{आगन्ता} इति रूपम्। आङ्~\textcolor{red}{गमॢँ गतौ} (धा॰पा॰~९८२)~\arrow आ~गम्~\arrow \textcolor{red}{शेषात्कर्तरि परस्मैपदम्} (पा॰सू॰~१.३.७८)~\arrow \textcolor{red}{अनद्यतने लुट्} (पा॰सू॰~३.३.१५)~\arrow आ~गम्~\arrow आ~गम्~लुट्~\arrow आ~गम्~तिप्~\arrow आ~गम्~ति~\arrow \textcolor{red}{स्यतासी लृलुटोः} (पा॰सू॰~३.१.३३)~\arrow आ~गम्~तास्~ति~\arrow \textcolor{red}{लुटः प्रथमस्य डारौरसः} (पा॰सू॰~२.४.८५)~\arrow आ~गम्~तास्~डा~\arrow आ~गम्~तास्~आ~\arrow \textcolor{red}{डित्यभस्याप्यनु\-बन्धकरण\-सामर्थ्यात्} (वा॰~६.४.१४३)~\arrow आ~गम्~त्~आ~\arrow \textcolor{red}{मोऽनुस्वारः} (पा॰सू॰~८.३.२३)~\arrow आ~गं~त्~आ~\arrow \textcolor{red}{अनुस्वारस्य ययि परसवर्णः} (पा॰सू॰~८.४.५८)~\arrow आ~गन्~त्~आ~\arrow आगन्ता।} \textcolor{red}{अनद्यतने लुट्} (पा॰सू॰~३.३.१५) इति सूत्रानुशासनात्किन्तु
सामान्य\-समयतोऽनिश्चित\-भविष्यद्विवक्षायां
लृडेव।\footnote{आङ्~\textcolor{red}{गमॢँ गतौ} (धा॰पा॰~९८२)~\arrow आ~गम्~\arrow \textcolor{red}{शेषात्कर्तरि परस्मैपदम्} (पा॰सू॰~१.३.७८)~\arrow \textcolor{red}{लृट् शेषे च} (पा॰सू॰~३.३.१३)~\arrow आ~गम्~लृँट्~\arrow आ~गम्~तिप्~\arrow आ~गम्~ति~\arrow \textcolor{red}{स्यतासी लृलुटोः} (पा॰सू॰~३.१.३३)~\arrow आ~गम्~स्य~ति~\arrow \textcolor{red}{गमेरिट् परस्मैपदेषु} (पा॰सू॰~७.२.५८)~\arrow आ~गम्~इट्~स्य~ति~\arrow आ~गम्~इ~स्य~ति~\arrow \textcolor{red}{आदेशप्रत्यययोः} (पा॰सू॰~८.३.५९)~\arrow आ~गम्~इ~ष्य~ति~\arrow आगमिष्यति।} अतो न दोषः।\end{sloppypar}
\section[काङ्क्षते]{काङ्क्षते}
\centering\textcolor{blue}{तव पादरजःस्पर्शं काङ्क्षते पवनाशना।\nopagebreak\\
आस्तेऽद्यापि रघुश्रेष्ठ तपो दुष्करमास्थिता॥}\nopagebreak\\
\raggedleft{–~अ॰रा॰~१.५.३४}\\
\fontsize{14}{21}\selectfont\begin{sloppypar}\hyphenrules{nohyphenation}\justifying\noindent\hspace{10mm} विश्वामित्रः श्रीरामं प्रति वेदयति यत् \textcolor{red}{अहल्या भवच्चरण\-धूलिं वाञ्छति}। तत्र \textcolor{red}{काङ्क्षते} इति प्रयुक्तम्। अत्र कर्म\-व्यतिहार आत्मनेपदम्।\footnote{\textcolor{red}{कर्तरि कर्म\-व्यतिहारे} (पा॰सू॰~१.३.१४) इत्यनेन। \textcolor{red}{काक्षिँ काङ्क्षायाम्} (धा॰पा॰~६६७)~\arrow काक्ष्~\arrow \textcolor{red}{इदितो नुम् धातोः} (पा॰सू॰~७.१.५८)~\arrow \textcolor{red}{मिदचोऽन्त्यात्परः} (पा॰सू॰~१.१.४७)~\arrow का~नुँम्~क्ष्~\arrow कान्~क्ष्~\arrow \textcolor{red}{नश्चापदान्तस्य झलि} (पा॰सू॰~८.३.२४)~\arrow कांक्ष्~\arrow \textcolor{red}{अनुस्वारस्य ययि परसवर्णः} (पा॰सू॰~८.४.५८)~\arrow काङ्क्ष्~\arrow \textcolor{red}{कर्तरि कर्म\-व्यतिहारे} (पा॰सू॰~१.३.१४)~\arrow \textcolor{red}{वर्तमाने लट्} (पा॰सू॰~३.२.१२३)~\arrow काङ्क्ष्~लट्~\arrow काङ्क्ष्~त~\arrow \textcolor{red}{कर्तरि शप्‌} (पा॰सू॰~३.१.६८)~\arrow काङ्क्ष्~शप्~त~\arrow काङ्क्ष्~अ~त~\arrow \textcolor{red}{टित आत्मनेपदानां टेरे} (पा॰सू॰~३.४.७९)~\arrow काङ्क्ष्~अ~ते~\arrow काङ्क्षते।} अन्यस्य योग्यं कार्यमन्यः करोति यदा तदा कर्म\-व्यतिहारः। तव पाद\-रजो योग्यतया श्रीर्वाञ्छति। यथा श्रीमद्भागवते~–\end{sloppypar}
\centering\textcolor{red}{श्रीर्यत्पदाम्बुजरजश्चकमे तुलस्या लब्ध्वाऽपि वक्षसि पदं किल भृत्यजुष्टम्।\nopagebreak\\
यस्याः स्ववीक्षणकृतेऽन्यसुरप्रयासस्तद्वद्वयं च तव पादरजःप्रपन्नाः॥}\nopagebreak\\
\raggedleft{–~भा॰पु॰~१०.२९.३७}\\
\fontsize{14}{21}\selectfont\begin{sloppypar}\hyphenrules{nohyphenation}\justifying\noindent अर्थात्साक्षान्महा\-लक्ष्मीर्भगवती त्वत्पाद\-रजोऽधिकारिणी किन्तु परम\-पातक\-कारिणी व्यभिचारिणी सत्यपीयं रजो\-गुण\-नाशाय तव चरण\-धूलिं काङ्क्षतीत्यन्य\-योग्य\-काङ्क्षण आत्मनेपदम्। अहल्याऽऽत्मनः पराभव\-मूलं कामं तदुद्भवं च रजो\-गुणं मन्यमाना तं च रजो\-गुणं मुकुन्द\-श्रीराम\-भद्रचरणारविन्द\-रजसा परिमार्ष्टुमिच्छतीति कौशिक\-हार्दम्। कामो रजो\-गुणोद्भव इति गीतायां निर्दिष्टं यथा~–\end{sloppypar}
\centering\textcolor{red}{काम एष क्रोध एष रजोगुणसमुद्भवः।\nopagebreak\\
महाशनो महापाप्मा विद्ध्येनमिह वैरिणम्॥}\nopagebreak\\
\raggedleft{–~भ॰गी॰~३.३७}\\
\fontsize{14}{21}\selectfont\begin{sloppypar}\hyphenrules{nohyphenation}\justifying\noindent यद्वा ब्राह्मणी सत्यपि क्षत्त्रियस्य ते चरणारविन्द\-रजः काङ्क्षत्यतो दयस्वेत्यन्य\-योग\-काङ्क्षण\-रूपे कर्म\-व्यतिहार आत्मनेपदे \textcolor{red}{काङ्क्षते}।\end{sloppypar}
\section[पावयस्व]{पावयस्व}
\centering\textcolor{blue}{पावयस्व मुनेर्भार्यामहल्यां ब्रह्मणः सुताम्।\nopagebreak\\
इत्युक्त्वा राघवं हस्ते गृहीत्वा मुनिपुङ्गवः॥}\nopagebreak\\
\raggedleft{–~अ॰रा॰~१.५.३५}\\
\fontsize{14}{21}\selectfont\begin{sloppypar}\hyphenrules{nohyphenation}\justifying\noindent\hspace{10mm} विश्वामित्रः कथयति यत्
\textcolor{red}{भगवन्नहल्यां पुनीहि}।
अत्र \textcolor{red}{पावय} इति प्रयोक्तव्यम्।\footnote{\textcolor{red}{पूञ् पवने} (धा॰पा॰~१४८२)~\arrow पू~\arrow \textcolor{red}{हेतुमति च} (पा॰सू॰~३.१.२६) (यद्वा स्वार्थे णिच्)~\arrow पू~णिच्~\arrow पू~इ~\arrow \textcolor{red}{अचो ञ्णिति} (पा॰सू॰~७.२.११५)~\arrow पौ~इ~\arrow \textcolor{red}{एचोऽयवायावः} (पा॰सू॰~६.१.७८)~\arrow पाव्~इ~\arrow पावि~\arrow \textcolor{red}{सनाद्यन्ता धातवः} (पा॰सू॰~३.१.३२)~\arrow धातु\-सञ्ज्ञा~\arrow \textcolor{red}{शेषात्कर्तरि परस्मैपदम्} (पा॰सू॰~१.३.७८)~\arrow \textcolor{red}{लोट् च} (पा॰सू॰~३.३.१६२)~\arrow पावि~लोट्~\arrow पावि~सिप्~\arrow पावि~सि~\arrow \textcolor{red}{कर्तरि शप्} (पा॰सू॰~३.१.६८)~\arrow पावि~शप्~सि~\arrow पावि~अ~सि~\arrow \textcolor{red}{सार्वधातुकार्धधातुकयोः} (पा॰सू॰~७.३.८४)~\arrow पावे~अ~सि~\arrow \textcolor{red}{एचोऽयवायावः} (पा॰सू॰~६.१.७८)~\arrow पावय्~अ~सि~\arrow \textcolor{red}{सेर्ह्यपिच्च} (पा॰सू॰~३.४.८७)~\arrow पावय्~अ~हि~\arrow \textcolor{red}{अतो हेः} (पा॰सू॰~६.४.१०५)~\arrow पावय्~अ~\arrow पावय।} \textcolor{red}{पावयस्व} इति पावन\-रूपं फलं त्वय्येव स्थास्यति तव भक्त\-वात्सल्यं पतित\-पावनत्वं वा दिगन्ते प्रसरिष्यत्यत आत्मनेपदम्।\footnote{पावि~\arrow धातु\-सञ्ज्ञा (पूर्ववत्)~\arrow \textcolor{red}{णिचश्च} (पा॰सू॰~१.३.७४)~\arrow \textcolor{red}{लोट् च} (पा॰सू॰~३.३.१६२)~\arrow पावि~लोट्~\arrow पावि~थास्~\arrow \textcolor{red}{कर्तरि शप्} (पा॰सू॰~३.१.६८)~\arrow पावि~शप्~थास्~\arrow पावि~अ~थास्~\arrow \textcolor{red}{सार्वधातुकार्धधातुकयोः} (पा॰सू॰~७.३.८४)~\arrow पावे~अ~थास्~\arrow \textcolor{red}{एचोऽयवायावः} (पा॰सू॰~६.१.७८)~\arrow पावय्~अ~थास्~\arrow \textcolor{red}{थासस्से} (पा॰सू॰~३.४.८०)~\arrow पावय्~अ~से~\arrow \textcolor{red}{सवाभ्यां वामौ} (पा॰सू॰~३.४.९१)~\arrow पावय्~अ~स्व~\arrow पावयस्व।} यद्वा \textcolor{red}{हे स्व} अर्थात् \textcolor{red}{हे आत्मीय हे आत्मन् हे धन पावय} इत्यन्वये परिहारः। \textcolor{red}{स्वमज्ञाति\-धनाख्यायाम्} (पा॰सू॰~१.१.३५) इति सूत्रस्य भाष्ये \textcolor{red}{स्व}\-शब्दस्य चत्वारोऽर्थाः सङ्केतेन कथिताः~– आत्माऽऽत्मीयो ज्ञातिर्धनमिति।\footnote{\textcolor{red}{आख्याग्रहणं किमर्थम्। ज्ञाति\-धन\-पर्यायवाची यः स्वशब्दस्तस्य यथा स्यात्। इह मा भूत्~– स्वे पुत्राः स्वाः पुत्राः स्वे गावः स्वाः गावः} (भा॰पा॰सू॰~१.१.३५)। \textcolor{red}{स्वे स्वाः। आत्मीया इत्यर्थः। आत्मान इति वा। ज्ञाति\-धन\-वाचिनस्तु स्वाः। ज्ञातयोऽर्था वा} (वै॰सि॰कौ॰~२२०, १.१.३५)।} अत्र चत्वारोऽप्यर्था अनुसन्धेयाः। विश्वामित्रः कथयति यत्त्वमात्मा त्वमात्मीयस्त्वमेवास्मज्ज्ञातिर्ब्रह्मण्य\-देवत्वात्त्वमेवास्मद्धनमाराध्यत्वात्। अतः \textcolor{red}{हे स्व पावय} इति सम्यक्पाणिनीयता।\end{sloppypar}
\section[किं वर्ण्यते]{किं वर्ण्यते}
\centering\textcolor{blue}{यत्पादपङ्कजपरागपवित्रगात्रा भागीरथी भवविरिञ्चिमुखान्पुनाति।\nopagebreak\\
साक्षात्स एव मम दृग्विषयो यदास्ते किं वर्ण्यते मम पुराकृतभागधेयम्॥}\nopagebreak\\
\raggedleft{–~अ॰रा॰~१.५.४५}\\
\fontsize{14}{21}\selectfont\begin{sloppypar}\hyphenrules{nohyphenation}\justifying\noindent\hspace{10mm} अहल्या स्वकीयं भाग्यं प्रशंसति। \textcolor{red}{किं वर्ण्यते} इति।
अत्र \textcolor{red}{वर्णयितुं शक्यते} इति वर्तमान\-सामीप्याल्लट्।\footnote{\textcolor{red}{वर्तमान\-सामीप्ये वर्तमानवद्वा} (पा॰सू॰~३.३.१३१) इत्यनेन। \textcolor{red}{वर्ण वर्णक्रिया\-विस्तार\-गुण\-वचनेषु} (धा॰पा॰~१९३९)~\arrow वर्ण~\arrow \textcolor{red}{सत्याप\-पाश\-रूप\-वीणा\-तूल\-श्लोक\-सेना\-लोम\-त्वच\-वर्म\-वर्ण\-चूर्ण\-चुरादिभ्यो णिच्} (पा॰सू॰~३.१.२५)~\arrow वर्ण~णिच्~\arrow वर्ण~इ~\arrow \textcolor{red}{णाविष्ठवत्प्राति\-पदिकस्य पुंवद्भाव\-रभाव\-टिलोप\-यणादि\-परार्थम्} (वा॰~६.४.४८)~\arrow वर्ण्~इ~\arrow वर्णि~\arrow धातु\-सञ्ज्ञा~\arrow \textcolor{red}{भावकर्मणोः} (पा॰सू॰~१.३.१३)~\arrow \textcolor{red}{वर्तमान\-सामीप्ये वर्तमानवद्वा} (पा॰सू॰~३.३.१३१)~\arrow \textcolor{red}{वर्तमाने लट्} (पा॰सू॰~३.२.१२३)~\arrow वर्णि~लट्~\arrow वर्णि~त~\arrow वर्णि~त~\arrow \textcolor{red}{सार्वधातुके यक्} (पा॰सू॰~३.१.६७)~\arrow वर्णि~यक्~त~\arrow वर्णि~य~त~\arrow \textcolor{red}{णेरनिटि} (पा॰सू॰~६.४.५१)~\arrow वर्ण्~य~त~\arrow \textcolor{red}{टित आत्मनेपदानां टेरे} (पा॰सू॰~३.४.७९)~\arrow वर्ण्~य~ते~\arrow वर्ण्यते।}\end{sloppypar}
\section[अधिगच्छति]{अधिगच्छति}
\centering\textcolor{blue}{अहल्यया कृतं स्तोत्रं यः पठेद्भक्तिसंयुतः।\nopagebreak\\
स मुच्यतेऽखिलैः पापैः परं ब्रह्माधिगच्छति॥}\nopagebreak\\
\raggedleft{–~अ॰रा॰~१.५.६२}\\
\fontsize{14}{21}\selectfont\begin{sloppypar}\hyphenrules{nohyphenation}\justifying\noindent\hspace{10mm} अत्र वर्तमान\-सामीप्याल्लट्।\footnote{\textcolor{red}{वर्तमान\-सामीप्ये वर्तमानवद्वा} (पा॰सू॰~३.३.१३१) इत्यनेन। अधि~\textcolor{red}{गमॢँ गतौ} (धा॰पा॰~९८२)~\arrow अधि~गम्~\arrow \textcolor{red}{शेषात्कर्तरि परस्मैपदम्} (पा॰सू॰~१.३.७८)~\arrow \textcolor{red}{वर्तमान\-सामीप्ये वर्तमानवद्वा} (पा॰सू॰~३.३.१३१)~\arrow \textcolor{red}{वर्तमाने लट्} (पा॰सू॰~३.२.१२३)~\arrow अधि~गम्~लट्~\arrow अधि~गम्~तिप्~\arrow अधि~गम्~ति~\arrow \textcolor{red}{कर्तरि शप्} (पा॰सू॰~३.१.६८)~\arrow अधि~गम्~शप्~ति~\arrow अधि~गम्~अ~ति~\arrow \textcolor{red}{इषुगमियमां छः} (पा॰सू॰~७.३.७७)~\arrow अधि~गछ्~अ~ति~\arrow \textcolor{red}{छे च} (पा॰सू॰~६.१.७३)~\arrow \textcolor{red}{आद्यन्तौ टकितौ} (पा॰सू॰~१.१.४६)~\arrow अधि~गतुँक्~छ्~अ~ति~\arrow अधि~गत्~छ्~अ~ति~\arrow \textcolor{red}{स्तोः श्चुना श्चुः} (पा॰सू॰~८.४.४०)~\arrow अधि~गच्~छ्~अ~ति~\arrow अधिगच्छति।} \textcolor{red}{अहल्या\-कृत\-स्तोत्रं पठित्वा सद्यः परं ब्रह्माधि\-गमिष्यति} इति ध्वनयितुं वर्तमान\-प्रत्ययः। अत्र कर्म\-कर्तृ\-प्रयोगः। यदा कार्य\-सौकर्यातिशयं बोधयितुं कर्तृ\-व्यापारो न विवक्ष्यते तदा कारकान्तराण्यपि कर्तृ\-सञ्ज्ञां लभन्ते स्व\-व्यापारे स्वतन्त्रत्वात्। यथा \textcolor{red}{सीता कन्द\-मूलं पचति}। सीतया किं कन्द\-मूलं स्वयमेव पच्यते तथैवात्रापि। \textcolor{red}{स ब्रह्माधिगच्छति}। तेन किं पर\-ब्रह्म स्वयमेवाधिगम्यते। \textcolor{red}{अधिगच्छति शास्त्रार्थः} (वै॰सि॰कौ॰~२७६६) इतिवत्।
\end{sloppypar}
\section[क्षालयामि]{क्षालयामि}
\centering\textcolor{blue}{क्षालयामि तव पादपङ्कजं नाथ दारुदृषदोः किमन्तरम्।\nopagebreak\\
मानुषीकरणचूर्णमस्ति ते पादयोरिति कथा प्रथीयसी॥}\nopagebreak\\
\raggedleft{–~अ॰रा॰~१.६.३}\\
\fontsize{14}{21}\selectfont\begin{sloppypar}\hyphenrules{nohyphenation}\justifying\noindent\hspace{10mm} अत्र वर्तमान\-सामीप्याल्लट्।\footnote{\textcolor{red}{वर्तमान\-सामीप्ये वर्तमानवद्वा} (पा॰सू॰~३.३.१३१) इत्यनेन। \textcolor{red}{क्षलँ शौचकर्मणि} (धा॰पा॰~१५९७)~\arrow क्षल्~\arrow \textcolor{red}{सत्याप\-पाश\-रूप\-वीणा\-तूल\-श्लोक\-सेना\-लोम\-त्वच\-वर्म\-वर्ण\-चूर्ण\-चुरादिभ्यो णिच्} (पा॰सू॰~३.१.२५)~\arrow क्षल्~णिच्~\arrow क्षल्~इ~\arrow \textcolor{red}{अत उपधायाः} (पा॰सू॰~७.२.११६)~\arrow क्षाल्~इ~\arrow क्षालि~\arrow \textcolor{red}{सनाद्यन्ता धातवः} (पा॰सू॰~३.१.३२)~\arrow धातु\-सञ्ज्ञा~\arrow \textcolor{red}{शेषात्कर्तरि परस्मैपदम्} (पा॰सू॰~१.३.७८)~\arrow \textcolor{red}{वर्तमान\-सामीप्ये वर्तमानवद्वा} (पा॰सू॰~३.३.१३१)~\arrow \textcolor{red}{वर्तमाने लट्} (पा॰सू॰~३.२.१२३)~\arrow क्षालि~लट्~\arrow क्षालि~मिप्~\arrow क्षालि~मि~\arrow \textcolor{red}{कर्तरि शप्} (पा॰सू॰~३.१.६८)~\arrow क्षालि~शप्~मि~\arrow क्षालि~अ~मि~\arrow \textcolor{red}{सार्वधातुकार्धधातुकयोः} (पा॰सू॰~७.३.८४)~\arrow क्षाले~अ~मि~\arrow \textcolor{red}{एचोऽयवायावः} (पा॰सू॰~६.१.७८)~\arrow क्षालय्~अ~मि~\arrow \textcolor{red}{अतो दीर्घो यञि} (पा॰सू॰~७.३.१०१)~\arrow क्षालय्~आ~मि~\arrow क्षालयामि।}\end{sloppypar}
\section[अनुशुश्रुवे]{अनुशुश्रुवे}
\centering\textcolor{blue}{पूजितं राजभिः सर्वैर्दृष्टमित्यनुशुश्रुवे।\nopagebreak\\
अतो दर्शय राजेन्द्र शैवं चापमनुत्तमम्।\nopagebreak\\
दृष्ट्वाऽयोध्यां जिगमिषुः पितरं द्रष्टुमिच्छति॥}\nopagebreak\\
\raggedleft{–~अ॰रा॰~१.६.१६}\\
\fontsize{14}{21}\selectfont\begin{sloppypar}\hyphenrules{nohyphenation}\justifying\noindent\hspace{10mm} अत्र विश्वामित्रो लिड्लकार\-क्रियां प्रयुङ्क्ते। तस्य कथं पारोक्ष्यं त्रिकाल\-दर्शित्वात्। वस्तुतस्त्वपरोक्षानुभूतेः सरूपं सगुणं ब्रह्म श्रीरामं दर्शं दर्शं विस्मृत\-सकल\-व्यापारतया भावातिरेके प्रेम\-सिन्धौ ज्ञान\-प्लवस्य मग्नतां सूचयितुं परोक्ष\-प्रयोगः।
\textcolor{red}{बहु जगद पुरस्तात्तस्य मत्ता किलाहम्} (शि॰~११.३९) इति प्रयोग इव।\end{sloppypar}
\section[आकर्षयामास]{आकर्षयामास}
\centering\textcolor{blue}{ईषदाकर्षयामास पाणिना दक्षिणेन सः।\nopagebreak\\
बभञ्जाखिलहृत्सारो दिशः शब्देन पूरयन्॥}\nopagebreak\\
\raggedleft{–~अ॰रा॰~१.६.२५}\\
\fontsize{14}{21}\selectfont\begin{sloppypar}\hyphenrules{nohyphenation}\justifying\noindent\hspace{10mm} अत्र स्वार्थे णिजन्तत्वात् \textcolor{red}{आकर्षयामास} इति।\footnote{\textcolor{red}{कृषँ विलेखने} (धा॰पा॰~९९०)~\arrow कृष्~\arrow स्वार्थे णिच्~\arrow कृष्~णिच्~\arrow कृष्~इ~\arrow \textcolor{red}{पुगन्त\-लघूपधस्य च} (पा॰सू॰~७.३.८६)~\arrow \textcolor{red}{उरण् रपरः} (पा॰सू॰~१.१.५१)~\arrow कर्ष्~इ~\arrow कर्षि~\arrow \textcolor{red}{सनाद्यन्ता धातवः} (पा॰सू॰~३.१.३२)~\arrow धातुसञ्ज्ञा। आङ्~कर्षि~\arrow आ~कर्षि~\arrow \textcolor{red}{शेषात्कर्तरि परस्मैपदम्} (पा॰सू॰~१.३.७८)~\arrow \textcolor{red}{परोक्षे लिट्} (पा॰सू॰~३.२.११५)~\arrow आ~कर्षि~लिट्~\arrow आ~कर्षि~तिप्~\arrow \textcolor{red}{परस्मैपदानां णलतुसुस्थलथुस\-णल्वमाः} (पा॰सू॰~३.४.८२)~\arrow आ~कर्षि~ण~\arrow आ~कर्षि~अ~\arrow \textcolor{red}{कास्यनेकाचश्चुलुम्पाद्यर्थम्} (वा॰~३.३.३५)~\arrow आ~कर्षि~आम्~अ~\arrow \textcolor{red}{सार्वधातुकार्धधातुकयोः} (पा॰सू॰~७.३.८४)~\arrow आ~कर्षे~आम्~अ~\arrow \textcolor{red}{एचोऽयवायावः} (पा॰सू॰~६.१.७८)~\arrow आ~कर्षय्~आम्~अ~\arrow \textcolor{red}{आमः} (पा॰सू॰~२.४.८१)~\arrow आ~कर्षय्~आम्~\arrow \textcolor{red}{कृञ्चानुप्रयुज्यते लिटि} (पा॰सू॰~३.१.४०)~\arrow आ~कर्षय्~आम्~अस्~लिट्~\arrow आ~कर्षय्~आम्~अस्~तिप्~\arrow \textcolor{red}{परस्मैपदानां णलतुसुस्थलथुस\-णल्वमाः} (पा॰सू॰~३.४.८२)~\arrow आ~कर्षय्~आम्~अस्~ण~\arrow आ~कर्षय्~आम्~अस्~अ~\arrow \textcolor{red}{लिटि धातोरनभ्यासस्य} (पा॰सू॰~६.१.८)~\arrow आ~कर्षय्~आम्~अस्~अस्~अ~\arrow \textcolor{red}{हलादिः शेषः} (पा॰सू॰~७.४.६०)~\arrow आ~कर्षय्~आम्~अ~अस्~अ~\arrow \textcolor{red}{अत आदेः} (पा॰सू॰~७.४.७०)~\arrow आ~कर्षय्~आम्~आ~अस्~अ~\arrow \textcolor{red}{अत उपधायाः} (पा॰सू॰~७.२.११६)~\arrow आ~कर्षय्~आम्~आ~आस्~अ~\arrow \textcolor{red}{अकः सवर्णे दीर्घः} (पा॰सू॰~६.१.१०१)~\arrow आ~कर्षय्~आम्~आस्~अ~\arrow आकर्षयामास। यद्वा \textcolor{red}{आकर्षणमाकर्षः}। \textcolor{red}{भावे} (पा॰सू॰~३.३.१८) इत्यनेन भावे घञ्। ततः \textcolor{red}{तत्करोति तदाचष्टे} (धा॰पा॰ ग॰सू॰~१८७) इत्यनेन णिचि गुणे रपरत्वे \textcolor{red}{आकर्षि} इति धातुसञ्ज्ञायां लिटि तिप्यामि लुकि \textcolor{red}{अस्‌}\-धात्वनु\-प्रयोगे पूर्वोक्तदिशा \textcolor{red}{आकर्षयामास}।}\end{sloppypar}
\section[आनयामास]{आनयामास}
\centering\textcolor{blue}{ततः शुभे दिने लग्ने सुमुहूर्ते रघूत्तमम्।\nopagebreak\\
आनयामास धर्मज्ञो रामं सभ्रातृकं तदा॥}\nopagebreak\\
\raggedleft{–~अ॰रा॰~१.६.४५}\\
\fontsize{14}{21}\selectfont\begin{sloppypar}\hyphenrules{nohyphenation}\justifying\noindent\hspace{10mm} \textcolor{red}{आङ्} पूर्वकः \textcolor{red}{णीञ् प्रापणे} (धा॰पा॰~९०१) इति धातुः। \textcolor{red}{णो नः} (पा॰सू॰~६.१.६५) इत्यनेन नकारः। अत्र परोक्षे लिड्लकारः। \textcolor{red}{परस्मैपदानां णलतुसुस्थलथुस\-णल्वमाः} (पा॰सू॰~३.४.८२) इत्यनेन णलादेशे वृद्धौ द्वित्वेऽभ्यास\-लोपेऽयादेशे \textcolor{red}{आनिनाय} इत्येव।\footnote{आङ्~\textcolor{red}{णीञ् प्रापणे} (धा॰पा॰~९०१)~\arrow आ~नी~\arrow \textcolor{red}{शेषात्कर्तरि परस्मैपदम्} (पा॰सू॰~१.३.७८)~\arrow \textcolor{red}{परोक्षे लिट्} (पा॰सू॰~३.२.११५)~\arrow आ~नी~लिट्~\arrow आ~नी~तिप्~\arrow आ~नी~ति~\arrow \textcolor{red}{परस्मैपदानां णलतुसुस्थलथुस\-णल्वमाः} (पा॰सू॰~३.४.८२)~\arrow आ~नी~ण~\arrow आ~नी~अ~\arrow \textcolor{red}{लिटि धातोरनभ्यासस्य} (पा॰सू॰~६.१.८)~\arrow आ~नी~नी~अ~\arrow \textcolor{red}{ह्रस्वः} (पा॰सू॰~७.४.५९)~\arrow आ~नि~नी~अ~\arrow \textcolor{red}{अचो ञ्णिति} (पा॰सू॰~७.२.११५)~\arrow आ~नि~नै~अ~\arrow \textcolor{red}{एचोऽयवायावः} (पा॰सू॰~६.१.७८)~\arrow आ~नि~नाय्~अ~\arrow आनिनाय।} \textcolor{red}{आनयामास} इति कथम्। उच्यते।
\textcolor{red}{आनयतीत्यानया} भावे \textcolor{red}{अच्‌}\-प्रत्यये गुणेऽयादेशे टापि।\footnote{\textcolor{red}{नन्दि\-ग्रहि\-पचादिभ्यो ल्युणिन्यचः} (पा॰सू॰~३.१.१३४) इत्यनेन। बाहुलकाद्भावे स्त्रियाम्।}
\textcolor{red}{आनयाम्}। क्रिया\-विशेषणत्वाद्द्वितीया। \textcolor{red}{आस} बभूवेति।\footnote{\textcolor{red}{बभूव} इत्यर्थे \textcolor{red}{आस} इति बहुधा शिष्टप्रयोगेषु पुराणेषु च दृश्यते। यथा लघुत्रय्याम्~– \textcolor{red}{लावण्य उत्पाद्य इवास यत्नः} (कु॰स॰~१.३५) \textcolor{red}{निष्प्रभश्च रिपुरास भूभृताम्} (र॰वं॰~११.८१) \textcolor{red}{तेनास लोकः पितृमान्विनेत्रा} (र॰वं॰~१४.२३)। अत्र मल्लिनाथः~– \textcolor{red}{आसेति बभूवार्थे “तिङन्त\-प्रतिरूपकमव्ययम्” इत्याह शाकटायनः। वल्लभस्तु “न तिङन्त\-प्रतिरूपकमव्ययम् ‘अस्तेर्भूः’ इति भ्वादेश\-नियमात्तादृक्तिङन्तस्यैवाभावात्। किन्तु कवीनामयं प्रामादिकः प्रयोगः” इत्याह। वामनस्तु “अस गतिदीप्त्यादानेष्विति धातोर्लिटि रूपमिदम्” इत्याह। अस इत्यनुदात्तेद्दीप्त्यर्थे। आस दिदीपे। प्रवृत्त इत्यर्थः}। भागवते च~– \textcolor{red}{मैत्रेयेणास सङ्गमः} (भा॰पु॰~३.१.३) \textcolor{red}{भगवानेक आसेदम्} (भा॰पु॰~३.५.२३) \textcolor{red}{निष्क्रामति निर्विशती द्विधाऽऽस सा} (भा॰पु॰~४.४.१) \textcolor{red}{असन्नपि क्लेशद आस देहः} (भा॰पु॰~५.५.४) \textcolor{red}{न यदिदमग्र आस न भविष्यदतो निधनात्} (भा॰पु॰~१०.८७.३७)। यद्वा \textcolor{red}{अस्‌}\-धातोर्लिटि छान्दस\-वैकल्पिक\-प्रयोगोऽयम्। तथा च \textcolor{red}{बहुलं छन्दसि} (पा॰सू॰~२.४.७३) इति सूत्रे मण्डूकप्लुत्या \textcolor{red}{अस्तेर्भूः} (पा॰सू॰~२.४.५२) इति सूत्रं चाप्यनुवर्तनीयम्। तेन क्वचिदादेशाप्रवृत्तिः। यथा~– \textcolor{red}{तस्य ह नचिकेता नाम पुत्र आस} (क॰उ॰~१.१.१) इत्यत्र श्रीराघव\-कृपा\-भाष्ये ग्रन्थ\-प्रणेतारः~– \textcolor{red}{आस बभूव। अत्र ‘बहुलं छन्दसि’ (पा॰सू॰~२.४.७३) इत्यप्रवृत्ति\-लक्षण\-बाहुलकेन ‘अस्तेर्भूः’ (पा॰सू॰~२.४.५२) इत्यस्याप्रवृत्तौ ‘आस’ इति छान्दस\-प्रयोगः} (क॰उ॰ रा॰कृ॰भा॰~१.१.१)। \textcolor{red}{असँ भुवि} (धा॰पा॰~१०६५)~\arrow अस्~\arrow \textcolor{red}{शेषात्कर्तरि परस्मैपदम्} (पा॰सू॰~१.३.७८)~\arrow \textcolor{red}{परोक्षे लिट्} (पा॰सू॰~३.२.११५)~\arrow अस्~लिट्~\arrow अस्~तिप्~\arrow \textcolor{red}{परस्मैपदानां णलतुसुस्थलथुस\-णल्वमाः} (पा॰सू॰~३.४.८२)~\arrow अस्~णल्~\arrow अस्~अ~\arrow \textcolor{red}{बहुलं छन्दसि} (पा॰सू॰~२.४.७३)~\arrow भ्वादेशाप्रवृत्तिः~\arrow \textcolor{red}{लिटि धातोरनभ्यासस्य} (पा॰सू॰~६.१.८)~\arrow अस्~अस्~अ~\arrow \textcolor{red}{हलादिः शेषः} (पा॰सू॰~७.४.६०)~\arrow अ~अस्~अ~\arrow \textcolor{red}{अत आदेः} (पा॰सू॰~७.४.७०)~\arrow आ~अस्~अ~\arrow \textcolor{red}{अत उपधायाः} (पा॰सू॰~७.२.११६)~\arrow आ~आस्~अ~\arrow \textcolor{red}{अकः सवर्णे दीर्घः} (पा॰सू॰~६.१.१०१)~\arrow आस्~अ~\arrow आस। छन्दसि द्वयोरपि रूपयोः प्रयोगः। \textcolor{red}{द॒क्षाय्यो॒ यो दम॒ आस॒ नित्य॑} (ऋ॰वे॰सं॰~७.१.२) इत्यत्र \textcolor{red}{आस}। \textcolor{red}{अ॒रान्न ने॒मिः परि॒ ता ब॑भूव} (ऋ॰वे॰सं॰~१.३२.१५) इत्यत्र \textcolor{red}{बभूव} च।}\end{sloppypar}
\section[मुमोद]{मुमोद}
\centering\textcolor{blue}{मुमोद जनको लक्ष्मीं क्षीराब्धिरिव विष्णवे।\nopagebreak\\
ऊर्मिलां चौरसीं कन्यां लक्ष्मणाय ददौ मुदा॥}\nopagebreak\\
\raggedleft{–~अ॰रा॰~१.६.५५}\\
\fontsize{14}{21}\selectfont\begin{sloppypar}\hyphenrules{nohyphenation}\justifying\noindent\hspace{10mm} अत्र \textcolor{red}{मुद्‌}\-धातुः (\textcolor{red}{मुदँ हर्षे} धा॰पा॰~१६) आत्मनेपदी। एवं तत्र लिटि लकारे \textcolor{red}{मुमुदे} इति पाणिनीयम्।\footnote{\textcolor{red}{मुदँ हर्षे} (धा॰पा॰~१६)~\arrow मुद्~\arrow \textcolor{red}{अनुदात्तङित आत्मने\-पदम्} (पा॰सू॰~१.३.१२)~\arrow \textcolor{red}{परोक्षे लिट्} (पा॰सू॰~३.२.११५)~\arrow मुद्~लिट्~\arrow मुद्~त~\arrow \textcolor{red}{लिटस्तझयोरेशिरेच्} (पा॰सू॰~३.४.८१)~\arrow मुद्~एश्~\arrow मुद्~ए~\arrow \textcolor{red}{लिटि धातोरनभ्यासस्य} (पा॰सू॰~६.१.८)~\arrow मुद्~मुद्~ए~\arrow हलादिः शेषः~\arrow मु~मुद्~ए~\arrow \textcolor{red}{असंयोगाल्लिट् कित्} (पा॰सू॰~१.२.५)~\arrow कित्त्वम्~\arrow \textcolor{red}{ग्क्ङिति च} (पा॰सू॰~१.१.५)~\arrow लघूपध\-गुणनिषेधः~\arrow मु~मुद्~ए~\arrow मुमुदे।} \textcolor{red}{मुमोद} इत्यपि। \textcolor{red}{मोदनं मोदः}। भावे घञ्।\footnote{\textcolor{red}{भावे} (पा॰सू॰~३.३.१८) इत्यनेन।}
\textcolor{red}{मोदं करोति मोदयति}।\footnote{मोद~\arrow \textcolor{red}{तत्करोति तदाचष्टे} (धा॰पा॰ ग॰सू॰~१८७)~\arrow मोद~णिच्~\arrow मोद~इ~\arrow \textcolor{red}{णाविष्ठवत्प्राति\-पदिकस्य पुंवद्भाव\-रभाव\-टिलोप\-यणादि\-परार्थम्} (वा॰~६.४.४८)~\arrow मोद्~इ~\arrow मोदि~\arrow \textcolor{red}{सनाद्यन्ता धातवः} (पा॰सू॰~३.१.३२)~\arrow धातुसञ्ज्ञा~\arrow \textcolor{red}{शेषात्कर्तरि परस्मैपदम्} (पा॰सू॰~१.३.७८)~\arrow \textcolor{red}{वर्तमाने लट्} (पा॰सू॰~३.२.१२३)~\arrow मोदि~लट्~\arrow मोदि~तिप्~\arrow मोदि~ति~\arrow \textcolor{red}{कर्तरि शप्‌} (पा॰सू॰~३.१.६८)~\arrow मोदि~शप्~ति~\arrow मोदि~अ~ति~\arrow \textcolor{red}{सार्वधातुकार्धधातुकयोः} (पा॰सू॰~७.३.८४)~\arrow मोदे~अ~ति~\arrow \textcolor{red}{एचोऽयवायावः} (पा॰सू॰~६.१.७८)~\arrow मोदय्~अ~ति~\arrow मोदयति।} \textcolor{red}{मोदयतीति मोद्}।\footnote{मोदि~\arrow धातुसञ्ज्ञा (पूर्ववत्)~\arrow \textcolor{red}{क्विप् च} (पा॰सू॰~३.२.७६)~\arrow मोदि~क्विँप्~\arrow मोदि~व्~\arrow \textcolor{red}{वेरपृक्तस्य} (पा॰सू॰~६.१.६७)~\arrow मोदि~\arrow \textcolor{red}{णेरनिटि} (पा॰सू॰~६.४.५१)~\arrow मोद्~\arrow विभक्तिकार्यम्~\arrow मोद्~सुँ~\arrow \textcolor{red}{हल्ङ्याब्भ्यो दीर्घात्सुतिस्यपृक्तं हल्} (पा॰सू॰~६.१.६८)~\arrow मोद्~\arrow \textcolor{red}{वाऽवसाने} (पा॰सू॰~८.४.५६)~\arrow मोद्, मोत्।} \textcolor{red}{मोदिवाऽचरति मोदति}।\footnote{मोद्~\arrow \textcolor{red}{सर्वप्रातिपतिकेभ्य आचारे क्विब्वा वक्तव्यः} (वा॰~३.१.११)~\arrow मोद्~क्विँप्~\arrow मोद्~व्~\arrow \textcolor{red}{वेरपृक्तस्य} (पा॰सू॰~६.१.६७)~\arrow मोद्~\arrow \textcolor{red}{सनाद्यन्ता धातवः} (पा॰सू॰~३.१.३२)~\arrow धातु\-सञ्ज्ञा~\arrow \textcolor{red}{शेषात्कर्तरि परस्मैपदम्} (पा॰सू॰~१.३.७८)~\arrow \textcolor{red}{वर्तमाने लट्} (पा॰सू॰~३.२.१२३)~\arrow मोद्~लट्~\arrow मोद्~तिप्~\arrow मोद्~ति~\arrow \textcolor{red}{कर्तरि शप्‌} (पा॰सू॰~३.१.६८)~\arrow मोद्~शप्~ति~\arrow मोद्~अ~ति~\arrow मोदति।} ततो लिटि लकारे णल्प्रत्यये द्वित्वेऽभ्यास\-कार्ये ह्रस्वे \textcolor{red}{मुमोद}।\footnote{मोद्~\arrow \textcolor{red}{धातु\-सञ्ज्ञा} (पूर्ववत्)~\arrow \textcolor{red}{शेषात्कर्तरि परस्मैपदम्} (पा॰सू॰~१.३.७८)~\arrow \textcolor{red}{परोक्षे लिट्} (पा॰सू॰~३.२.११५)~\arrow मोद्~लिट्~\arrow मोद्~तिप्~\arrow मोद्~ति~\arrow \textcolor{red}{परस्मैपदानां णलतुसुस्थलथुस\-णल्वमाः} (पा॰सू॰~३.४.८२)~\arrow मोद्~ण~\arrow मोद्~अ~\arrow \textcolor{red}{लिटि धातोरनभ्यासस्य} (पा॰सू॰~६.१.८)~\arrow मोद्~मोद्~अ~\arrow हलादिः शेषः~\arrow मो~मोद्~अ~\arrow \textcolor{red}{ह्रस्वः} (पा॰सू॰~७.४.५९)~\arrow \textcolor{red}{एच इग्घ्रस्वादेशे} (पा॰सू॰~१.१.४८)~\arrow मु~मोद्~अ~\arrow मुमोद। मोदयितेवाऽचचारेति मुमोद। वस्तुतस्तु जनको योगिराजस्तस्य कुतो मोद इति भावः। यद्वा \textcolor{red}{अनुदात्तेत्त्व\-लक्षणमात्मने\-पदमनित्यम्} (प॰शे॰~९३.४) इत्यपि समाधानम्। \textcolor{red}{मुदँ हर्षे} (धा॰पा॰~१६)~\arrow मुद्~\arrow \textcolor{red}{अनुदात्तेत्त्व\-लक्षणमात्मने\-पदमनित्यम्} (प॰शे॰~९३.४)~\arrow \textcolor{red}{शेषात्कर्तरि परस्मैपदम्} (पा॰सू॰~१.३.७८)~\arrow \textcolor{red}{परोक्षे लिट्} (पा॰सू॰~३.२.११५)~\arrow मुद्~लिट्~\arrow मुद्~तिप्~\arrow मुद्~ति~\arrow \textcolor{red}{परस्मैपदानां णलतुसुस्थलथुस\-णल्वमाः} (पा॰सू॰~३.४.८२)~\arrow मुद्~ण~\arrow मुद्~अ~\arrow \textcolor{red}{लिटि धातोरनभ्यासस्य} (पा॰सू॰~६.१.८)~\arrow मुद्~मुद्~अ~\arrow हलादिः शेषः~\arrow मु~मुद्~अ~\arrow \textcolor{red}{पुगन्त\-लघूपधस्य च} (पा॰सू॰~७.३.८६)~\arrow मु~मोद्~अ~\arrow मुमोद।}\end{sloppypar}
\section[दीयते]{दीयते}
\centering\textcolor{blue}{तदारभ्य मया सीता विष्णोर्लक्ष्मीर्विभाव्यते।\nopagebreak\\
कथं मया राघवाय दीयते जानकी शुभा॥}\nopagebreak\\
\raggedleft{–~अ॰रा॰~१.६.६७}\\
\fontsize{14}{21}\selectfont\begin{sloppypar}\hyphenrules{nohyphenation}\justifying\noindent\hspace{10mm} अत्र कथं \textcolor{red}{मया दीयते} इति। \textcolor{red}{दीयताम्}\footnote{\textcolor{red}{डुदाञ् दाने} (धा॰पा॰~१०९१)~\arrow दा~\arrow \textcolor{red}{भावकर्मणोः} (पा॰सू॰~१.३.१३)~\arrow \textcolor{red}{लोट् च} (पा॰सू॰~३.३.१६२)~\arrow दा~लोट्~\arrow दा~त~\arrow \textcolor{red}{सार्वधातुके यक्} (पा॰सू॰~३.१.६७)~\arrow दा~यक्~त~\arrow दा~य~त~\arrow \textcolor{red}{घुमा\-स्थागापा\-जहातिसां हलि} (पा॰सू॰~६.४.६६)~\arrow दी~य~त~\arrow \textcolor{red}{टित आत्मनेपदानां टेरे} (पा॰सू॰~३.४.७९)~\arrow दी~य~ते~\arrow \textcolor{red}{आमेतः} (पा॰सू॰~३.४.९०)~\arrow दी~य~ताम्~\arrow दीयताम्।} इति \textcolor{red}{लोट् च} (पा॰सू॰~३.३.१६२) इत्यनेन प्रश्नार्थे\footnote{सम्प्रश्नार्थ इति भावः। \textcolor{red}{विधि\-निमन्‍त्रणामन्‍त्रणाधीष्‍ट\-सम्प्रश्न\-प्रार्थनेषु लिङ्} (पा॰सू॰~३.३.१६१) इत्यत्र \textcolor{red}{सम्प्रश्न} इत्युपादानेन।} लोट्। यद्वा \textcolor{red}{दीयेत}\footnote{\textcolor{red}{डुदाञ् दाने} (धा॰पा॰~१०९१)~\arrow दा~\arrow \textcolor{red}{भावकर्मणोः} (पा॰सू॰~१.३.१३)~\arrow \textcolor{red}{आशंसावचने लिङ्} (पा॰सू॰~३.३.१३४)~\arrow दा~लिङ्~\arrow दा~त~\arrow \textcolor{red}{सार्वधातुके यक्} (पा॰सू॰~३.१.६७)~\arrow दा~यक्~त~\arrow दा~य~त~\arrow \textcolor{red}{घुमा\-स्थागापा\-जहातिसां हलि} (पा॰सू॰~६.४.६६)~\arrow दी~य~त~\arrow \textcolor{red}{लिङः सीयुट्} (पा॰सू॰~३.४.१०२)~\arrow दी~य~सीयुँट्~त~\arrow दी~य~सीय्~त~\arrow \textcolor{red}{सुट् तिथोः} (पा॰सू॰~३.४.१०७)~\arrow \textcolor{red}{आद्यन्तौ टकितौ} (पा॰सू॰~१.१.४६)~\arrow दी~य~सीय्~सुँट्~त~\arrow दी~य~सीय्~स्~त~\arrow \textcolor{red}{लिङः सलोपोऽनन्त्यस्य} (पा॰सू॰~७.२.७९)~\arrow दी~य~ईय्~त~\arrow \textcolor{red}{लोपो व्योर्वलि} (पा॰सू॰~६.१.६६)~\arrow दी~य~ई~त~\arrow \textcolor{red}{आद्गुणः} (पा॰सू॰~६.१.८७)~\arrow दी~ये~त~\arrow दीयेत।} इति \textcolor{red}{आशंसावचने लिङ्} (पा॰सू॰~३.३.१३४) इत्यनेन लिङ् प्रयोक्तव्यमासीत्। किन्तु \textcolor{red}{धातु\-सम्बन्धे प्रत्ययाः} (पा॰सू॰~३.४.१) इत्यनेन लड्लकारः।\footnote{\textcolor{red}{डुदाञ् दाने} (धा॰पा॰~१०९१)~\arrow दा~\arrow \textcolor{red}{भावकर्मणोः} (पा॰सू॰~१.३.१३)~\arrow \textcolor{red}{धातुसम्बन्धे प्रत्ययाः} (पा॰सू॰~३.४.१)~\arrow दा~लट्~\arrow दा~त~\arrow \textcolor{red}{सार्वधातुके यक्} (पा॰सू॰~३.१.६७)~\arrow दा~यक्~त~\arrow दा~य~त~\arrow \textcolor{red}{घुमा\-स्थागापा\-जहातिसां हलि} (पा॰सू॰~६.४.६६)~\arrow दी~य~त~\arrow \textcolor{red}{टित आत्मनेपदानां टेरे} (पा॰सू॰~३.४.७९)~\arrow दी~य~ते~\arrow दीयते।}\end{sloppypar}
\section[त्राहि त्राहि]{त्राहि त्राहि}
\centering\textcolor{blue}{तं दृष्ट्वा भयसन्त्रस्तो राजा दशरथस्तदा।\nopagebreak\\
अर्घ्यादिपूजां विस्मृत्य त्राहि त्राहीति चाब्रवीत्॥}\nopagebreak\\
\raggedleft{–~अ॰रा॰~१.७.९}\\
\fontsize{14}{21}\selectfont\begin{sloppypar}\hyphenrules{nohyphenation}\justifying\noindent\hspace{10mm} विवाहं
कृत्वा 
परावर्तमानो दशरथः सपरिकरः परशुरामं दृष्ट्वा \textcolor{red}{त्राहि त्राहि} इति ब्रवीति। \textcolor{red}{त्रै}\-धातोस्तु (\textcolor{red}{त्रैङ् पालने} धा॰पा॰~९६५) लोटि मध्यम\-पुरुष एक\-वचने गुणेऽयादेशे \textcolor{red}{त्रायस्व} इति पाणिनीयम्।\footnote{\textcolor{red}{त्रैङ् पालने} (धा॰पा॰~९६५)~\arrow त्रै~\arrow \textcolor{red}{अनुदात्तङित आत्मने\-पदम्} (पा॰सू॰~१.३.१२)~\arrow \textcolor{red}{लोट् च} (पा॰सू॰~३.३.१६२)~\arrow त्रै~लोट्~\arrow त्रै~थास्~\arrow \textcolor{red}{कर्तरि शप्} (पा॰सू॰~३.१.६८)~\arrow त्रै~शप्~थास्~\arrow त्रै~अ~थास्~\arrow \textcolor{red}{एचोऽयवायावः} (पा॰सू॰~६.१.७८)~\arrow त्राय्~अ~थास्~\arrow \textcolor{red}{थासस्से} (पा॰सू॰~३.४.८०)~\arrow त्राय्~अ~से~\arrow \textcolor{red}{सवाभ्यां वामौ} (पा॰सू॰~३.४.९१)~\arrow त्राय्~अ~स्व~\arrow त्रायस्व।} \textcolor{red}{त्राहि} इति कथम्। आकृति\-गणत्वाददादि\-गणे \textcolor{red}{पा} (\textcolor{red}{पा रक्षणे} धा॰पा॰~१०५६) \textcolor{red}{रा} (\textcolor{red}{रा दाने} धा॰पा॰~१०५७) \textcolor{red}{प्रा} (\textcolor{red}{प्रा पूरणे} धा॰पा॰~१०६१) इत्यादिवद्रक्षणार्थः \textcolor{red}{त्रा} इति पठितव्यः। ततश्च \textcolor{red}{सेर्ह्यपिच्च} (पा॰सू॰~३.४.८७) इत्यनेन \textcolor{red}{हि} आदेशे \textcolor{red}{त्राहि} इति पाणिनीय एव।\footnote{\textcolor{red}{भूवादिष्वेतदन्तेषु दशगणीषु धातूनां पाठो निदर्शनाय तेन स्तम्भुप्रभृतयः सौत्राश्चुलुम्पादयो वाक्यकारीयाः प्रयोगसिद्धा विक्लवत्यादयश्च} (मा॰धा॰वृ॰~१०.३२८) इत्यनुसारमाकृति\-गणत्वाददादि\-गण ऊह्योऽयं त्राधातू रक्षण इति भावः। \textcolor{red}{त्रा रक्षणे} (अदादिगण ऊह्यः)~\arrow \textcolor{red}{शेषात्कर्तरि परस्मैपदम्} (पा॰सू॰~१.३.७८)~\arrow \textcolor{red}{लोट् च} (पा॰सू॰~३.३.१६२)~\arrow त्रा~लोट्~\arrow त्रा~सिप्~\arrow त्रा~सि~\arrow \textcolor{red}{कर्तरि शप्} (पा॰सू॰~३.१.६८)~\arrow त्रा~शप्~सि~\arrow \textcolor{red}{अदिप्रभृतिभ्यः शपः} (पा॰सू॰~२.४.७२)~\arrow त्रा~सि~\arrow \textcolor{red}{सेर्ह्यपिच्च} (पा॰सू॰~३.४.८७)~\arrow त्रा~हि~\arrow त्राहि।}\end{sloppypar}
\section[भूयात्]{भूयात्}
\label{sec:bhuyat1}
\centering\textcolor{blue}{जडस्य चित्समायोगाच्चित्त्वं भूयाच्चितेस्तथा।\nopagebreak\\
जडसङ्गाज्जडत्वं हि जलाग्न्योर्मेलनं यथा॥}\nopagebreak\\
\raggedleft{–~अ॰रा॰~१.७.३७}\\
\centering\textcolor{blue}{देव यद्यत्कृतं पुण्यं मया लोकजिगीषया।\nopagebreak\\
तत्सर्वं तव बाणाय भूयाद्राम नमोऽस्तु ते॥}\nopagebreak\\
\raggedleft{–~अ॰रा॰~१.७.४५}\\
\fontsize{14}{21}\selectfont\begin{sloppypar}\hyphenrules{nohyphenation}\justifying\noindent\hspace{10mm} यद्यपि सत्तार्थक\-\textcolor{red}{भू}\-धातोः (\textcolor{red}{भू सत्तायाम्} धा॰पा॰~१) लिङ्लकारे तु \textcolor{red}{भवेत्}\footnote{\textcolor{red}{भू सत्तायाम्} (धा॰पा॰~१)~\arrow भू~\arrow \textcolor{red}{शेषात्कर्तरि परस्मैपदम्} (पा॰सू॰~१.३.७८)~\arrow \textcolor{red}{विधि\-निमन्‍त्रणामन्‍त्रणाधीष्‍ट\-सम्प्रश्‍न\-प्रार्थनेषु लिङ्} (पा॰सू॰~३.३.१६१)~\arrow भू~लिङ्~\arrow भू~तिप्~\arrow भू~ति~\arrow \textcolor{red}{कर्तरि शप्} (पा॰सू॰~३.१.६८)~\arrow भू~शप्~ति~\arrow भू~अ~ति~\arrow \textcolor{red}{सार्वधातुकार्धधातुकयोः} (पा॰सू॰~७.३.८४)~\arrow भो~अ~ति~\arrow \textcolor{red}{एचोऽयवायावः} (पा॰सू॰~६.१.७८)~\arrow भव्~अ~ति~\arrow \textcolor{red}{यासुट् परस्मैपदेषूदात्तो ङिच्च} (पा॰सू॰~३.४.१०३)~\arrow \textcolor{red}{आद्यन्तौ टकितौ} (पा॰सू॰~१.१.४६)~\arrow भव्~अ~यासुँट्~ति~\arrow भव्~अ~यास्~ति~\arrow \textcolor{red}{सुट् तिथोः} (पा॰सू॰~३.४.१०७)~\arrow \textcolor{red}{आद्यन्तौ टकितौ} (पा॰सू॰~१.१.४६)~\arrow भव्~अ~यास्~सुँट्~ति~\arrow भव्~अ~यास्~स्~ति~\arrow \textcolor{red}{लिङः सलोपोऽनन्त्यस्य} (पा॰सू॰~७.२.७९)~\arrow भव्~अ~या~ति~\arrow \textcolor{red}{अतो येयः} (पा॰सू॰~७.२.८०)~\arrow भव्~अ~इय्~ति~\arrow \textcolor{red}{लोपो व्योर्वलि} (पा॰सू॰~६.१.६६)~\arrow भव्~अ~इ~ति~\arrow \textcolor{red}{आद्गुणः} (पा॰सू॰~६.१.८७)~\arrow भव्~ए~ति~\arrow \textcolor{red}{इतश्च} (पा॰सू॰~३.४.१००)~\arrow भव्~ए~त्~\arrow भवेत्।} इति प्रयोग एवं लड्लकारे च \textcolor{red}{भवति}\footnote{\textcolor{red}{भू सत्तायाम्} (धा॰पा॰~१)~\arrow भू~\arrow \textcolor{red}{शेषात्कर्तरि परस्मैपदम्} (पा॰सू॰~१.३.७८)~\arrow \textcolor{red}{वर्तमाने लट्} (पा॰सू॰~३.२.१२३)~\arrow भू~लट्~\arrow भू~तिप्~\arrow भू~ति~\arrow \textcolor{red}{कर्तरि शप्} (पा॰सू॰~३.१.६८)~\arrow भू~शप्~ति~\arrow भू~अ~ति~\arrow \textcolor{red}{सार्वधातुकार्धधातुकयोः} (पा॰सू॰~७.३.८४)~\arrow भो~अ~ति~\arrow \textcolor{red}{एचोऽयवायावः} (पा॰सू॰~६.१.७८)~\arrow भव्~अ~ति~\arrow भवति।} इति प्रयोगः पाणिनीयः। \textcolor{red}{भूयात्} इति प्रयोग आशिषि।\footnote{\textcolor{red}{भू सत्तायाम्} (धा॰पा॰~१)~\arrow भू~\arrow \textcolor{red}{शेषात्कर्तरि परस्मैपदम्} (पा॰सू॰~१.३.७८)~\arrow \textcolor{red}{आशिषि लिङ्लोटौ} (पा॰सू॰~३.३.१७३)~\arrow भू~लिङ्~\arrow भू~तिप्~\arrow भू~ति~\arrow \textcolor{red}{लिङाशिषि} (पा॰सू॰~३.४.११६)~\arrow शबभावः~\arrow \textcolor{red}{यासुट् परस्मैपदेषूदात्तो ङिच्च} (पा॰सू॰~३.४.१०३)~\arrow \textcolor{red}{आद्यन्तौ टकितौ} (पा॰सू॰~१.१.४६)~\arrow भू~यासुँट्~ति~\arrow भू~यास्~ति~\arrow \textcolor{red}{सुट् तिथोः} (पा॰सू॰~३.४.१०७)~\arrow \textcolor{red}{आद्यन्तौ टकितौ} (पा॰सू॰~१.१.४६)~\arrow भू~यास्~सुँट्~ति~\arrow भू~यास्~स्~ति~\arrow \textcolor{red}{किदाशिषि} (पा॰सू॰~३.४.१०४)~\arrow कित्त्वम्~\arrow \textcolor{red}{ग्क्ङिति च} (पा॰सू॰~१.१.५)~\arrow गुणनिषेधः~\arrow \textcolor{red}{इतश्च} (पा॰सू॰~३.४.१००)~\arrow भू~यास्~स्~त~\arrow \textcolor{red}{स्कोः संयोगाद्योरन्ते च} (पा॰सू॰~८.२.२९)~\arrow भू~या~त्~\arrow भूयात्।} \textcolor{red}{आशिषि लिङ्\-लोटौ} (पा॰सू॰~३.३.१७३) इत्यनेन। किन्तु \textcolor{red}{यु} (\textcolor{red}{यु मिश्रणेऽमिश्रणे च} धा॰पा॰~१०३३) \textcolor{red}{णु} (\textcolor{red}{णु स्तुतौ} धा॰पा॰~१०३५) इत्यादि\-वद्भूधातुरप्यदादिस्तस्यैव लिङ्लकारे \textcolor{red}{भूयात्}।\footnote{\textcolor{red}{भूवादिष्वेतदन्तेषु दशगणीषु धातूनां पाठो निदर्शनाय तेन स्तम्भुप्रभृतयः सौत्राश्चुलुम्पादयो वाक्यकारीयाः प्रयोगसिद्धा विक्लवत्यादयश्च} (मा॰धा॰वृ॰~१०.३२८) इत्यनुसारमाकृति\-गणत्वाददादि\-गण ऊह्योऽयं भूधातुः सत्तायामिति भावः। \textcolor{red}{भू सत्तायाम्} (अदादिगण ऊह्यः)~\arrow \textcolor{red}{शेषात्कर्तरि परस्मैपदम्} (पा॰सू॰~१.३.७८)~\arrow \textcolor{red}{आशंसावचने लिङ्} (पा॰सू॰~३.३.१३४)~\arrow भू~लिङ्~\arrow भू~ति~\arrow \textcolor{red}{कर्तरि शप्} (पा॰सू॰~३.१.६८)~\arrow भू~शप्~ति~\arrow \textcolor{red}{अदिप्रभृतिभ्यः शपः} (पा॰सू॰~२.४.७२)~\arrow भू~ति~\arrow \textcolor{red}{यासुट् परस्मैपदेषूदात्तो ङिच्च} (पा॰सू॰~३.४.१०३)~\arrow \textcolor{red}{आद्यन्तौ टकितौ} (पा॰सू॰~१.१.४६)~\arrow भू~यासुँट्~ति~\arrow भू~यास्~ति~\arrow \textcolor{red}{सुट् तिथोः} (पा॰सू॰~३.४.१०७)~\arrow \textcolor{red}{आद्यन्तौ टकितौ} (पा॰सू॰~१.१.४६)~\arrow भू~यास्~सुँट्~ति~\arrow भू~यास्~स्~ति~\arrow \textcolor{red}{ग्क्ङिति च} (पा॰सू॰~१.१.५)~\arrow गुणनिषेधः~\arrow \textcolor{red}{इतश्च} (पा॰सू॰~३.४.१००)~\arrow भू~यास्~स्~त्~\arrow \textcolor{red}{स्कोः संयोगाद्योरन्ते च} (पा॰सू॰~८.२.२९)~\arrow भू~या~त्~\arrow भूयात्।} \textcolor{red}{आशंसावचने लिङ्} (पा॰सू॰~३.३.१३४) इत्यनेन।\end{sloppypar}
\vspace{2mm}
\centering ॥ इति बालकाण्डीयप्रयोगाणां विमर्शः ॥\nopagebreak\\
\vspace{4mm}
\pdfbookmark[2]{अयोध्याकाण्डम्}{Chap3Part1Kanda2}
\phantomsection
\addtocontents{toc}{\protect\setcounter{tocdepth}{2}}
\addcontentsline{toc}{subsection}{अयोध्याकाण्डीयप्रयोगाणां विमर्शः}
\addtocontents{toc}{\protect\setcounter{tocdepth}{0}}
\centering ॥ अथायोध्याकाण्डीयप्रयोगाणां विमर्शः ॥\nopagebreak\\
\section[मोहयस्व]{मोहयस्व}
\centering\textcolor{blue}{अहं त्वद्भक्तभक्तानां तद्भक्तानां च किङ्करः।\nopagebreak\\
अतो मामनुगृह्णीष्व मोहयस्व न मां प्रभो॥}\nopagebreak\\
\raggedleft{–~अ॰रा॰~२.१.३०}\\
\fontsize{14}{21}\selectfont\begin{sloppypar}\hyphenrules{nohyphenation}\justifying\noindent\hspace{10mm} नारदः कथयति \textcolor{red}{मां मा मोहयस्व}। अत्र \textcolor{red}{मोहय} इति समीचीनं प्रतीयते।\footnote{\textcolor{red}{मुहँ वैचित्ये} (धा॰पा॰~११९८)~\arrow मुह्~\arrow \textcolor{red}{हेतुमति च} (पा॰सू॰~३.१.२६)~\arrow मुह्~णिच्~\arrow मुह्~इ~\arrow \textcolor{red}{पुगन्त\-लघूपधस्य च} (पा॰सू॰~७.३.८६)~\arrow मोह्~इ~\arrow मोहि~\arrow \textcolor{red}{सनाद्यन्ता धातवः} (पा॰सू॰~३.१.३२)~\arrow \textcolor{red}{शेषात्कर्तरि परस्मैपदम्} (पा॰सू॰~१.३.७८)~\arrow \textcolor{red}{लोट् च} (पा॰सू॰~३.३.१६२)~\arrow मोहि~लोट्~\arrow मोहि~सिप्~\arrow मोहि~सि~\arrow \textcolor{red}{कर्तरि शप्} (पा॰सू॰~३.१.६८)~\arrow मोहि~शप्~सि~\arrow मोहि~अ~सि~\arrow \textcolor{red}{सार्वधातुकार्धधातुकयोः} (पा॰सू॰~७.३.८४)~\arrow मोहे~अ~सि~\arrow \textcolor{red}{एचोऽयवायावः} (पा॰सू॰~६.१.७८)~\arrow मोहय्~अ~सि~\arrow \textcolor{red}{सेर्ह्यपिच्च} (पा॰सू॰~३.४.८७)~\arrow मोहय्~अ~हि~\arrow \textcolor{red}{अतो हेः} (पा॰सू॰~६.४.१०५)~\arrow मोहय्~अ~\arrow मोहय।} किन्तु \textcolor{red}{स्व} इति पृथगुपादानेन \textcolor{red}{हे स्व हे आत्मीय मां मा मोहय} इति परिहारः।\end{sloppypar}
\section[नाशयामि]{नाशयामि}
\centering\textcolor{blue}{सीतामिषेण तं दुष्टं सकुलं नाशयाम्यहम्।\nopagebreak\\
एवं रामे प्रतिज्ञाते नारदः प्रमुमोद ह॥}\nopagebreak\\
\raggedleft{–~अ॰रा॰~२.१.३९}\\
\fontsize{14}{21}\selectfont\begin{sloppypar}\hyphenrules{nohyphenation}\justifying\noindent\hspace{10mm} अत्र वर्तमान\-सामीप्ये लट्।\footnote{\textcolor{red}{णशँ अदर्शने} (धा॰पा॰~११९४)~\arrow णश्~\arrow \textcolor{red}{णो नः} (पा॰सू॰~६.१.६५)~\arrow नश्~\arrow \textcolor{red}{हेतुमति च} (पा॰सू॰~३.१.२६)~\arrow नश्~णिच्~\arrow नश्~इ~\arrow \textcolor{red}{अत उपधायाः} (पा॰सू॰~७.२.११६)~\arrow नाश्~इ~\arrow नाशि~\arrow \textcolor{red}{सनाद्यन्ता धातवः} (पा॰सू॰~३.१.३२)~\arrow धातु\-सञ्ज्ञा~\arrow \textcolor{red}{बुध\-युध\-नश\-जनेङ्प्रु\-द्रु\-स्रुभ्यो णेः} (पा॰सू॰~१.३.८६)~\arrow \textcolor{red}{वर्तमान\-सामीप्ये वर्तमानवद्वा} (पा॰सू॰~३.३.१३१)~\arrow \textcolor{red}{वर्तमाने लट्} (पा॰सू॰~३.२.१२३)~\arrow नाशि~लट्~\arrow नाशि~मिप्~\arrow नाशि~मि~\arrow \textcolor{red}{कर्तरि शप्‌} (पा॰सू॰~३.१.६८)~\arrow नाशि~शप्~मि~\arrow नाशि~अ~मि~\arrow \textcolor{red}{सार्वधातुकार्ध\-धातुकयोः} (पा॰सू॰~७.३.८४)~\arrow नाशे~अ~मि~\arrow \textcolor{red}{एचोऽयवायावः} (पा॰सू॰~६.१.७८)~\arrow नाशय्~अ~मि~\arrow \textcolor{red}{अतो दीर्घो यञि} (पा॰सू॰~७.३.१०१)~\arrow नाशय्~आ~मि~\arrow नाशयामि।} यद्वा \textcolor{red}{नाशमाचष्ट इति नाशयति}।\footnote{नाश~\arrow \textcolor{red}{तत्करोति तदाचष्टे} (धा॰पा॰ ग॰सू॰~१८७)~\arrow नाश~णिच्~\arrow नाश~इ~\arrow \textcolor{red}{णाविष्ठवत्प्राति\-पदिकस्य पुंवद्भाव\-रभाव\-टिलोप\-यणादि\-परार्थम्} (वा॰~६.४.४८)~\arrow नाश्~इ~\arrow नाशि~\arrow \textcolor{red}{सनाद्यन्ता धातवः} (पा॰सू॰~३.१.३२)~\arrow धातुसञ्ज्ञा~\arrow \textcolor{red}{शेषात्कर्तरि परस्मैपदम्} (पा॰सू॰~१.३.७८)~\arrow \textcolor{red}{वर्तमाने लट्} (पा॰सू॰~३.२.१२३)~\arrow नाशि~लट्~\arrow नाशि~तिप्~\arrow नाशि~ति~\arrow \textcolor{red}{कर्तरि शप्‌} (पा॰सू॰~३.१.६८)~\arrow नाशि~शप्~ति~\arrow नाशि~अ~ति~\arrow \textcolor{red}{सार्वधातुकार्ध\-धातुकयोः} (पा॰सू॰~७.३.८४)~\arrow नाशे~अ~ति~\arrow \textcolor{red}{एचोऽयवायावः} (पा॰सू॰~६.१.७८)~\arrow नाशय्~अ~ति~\arrow नाशयति।} \textcolor{red}{नाशयतीति नाश्}।\footnote{नाशि~\arrow धातुसञ्ज्ञा (पूर्ववत्)~\arrow \textcolor{red}{क्विप् च} (पा॰सू॰~३.२.७६)~\arrow नाशि~क्विँप्~\arrow नाशि~व्~\arrow \textcolor{red}{वेरपृक्तस्य} (पा॰सू॰~६.१.६७)~\arrow नाशि~\arrow \textcolor{red}{णेरनिटि} (पा॰सू॰~६.४.५१)~\arrow नाश्~\arrow विभक्तिकार्यम्~\arrow नाश्~सुँ~\arrow \textcolor{red}{हल्ङ्याब्भ्यो दीर्घात्सुतिस्यपृक्तं हल्} (पा॰सू॰~६.१.६८)~\arrow नाश्~\arrow \textcolor{red}{अयस्मयादीनि च्छन्दसि} (पा॰सू॰~१.४.२०)~\arrow षत्वाभावः~\arrow जश्त्वाभावः~\arrow नाश्।} क्विप्कर्तरि। \textcolor{red}{अयस्मयादीनि च्छन्दसि} (पा॰सू॰~१.४.२०) इत्यनेन छान्दस\-भत्वात्षत्वाभावो\footnote{\textcolor{red}{व्रश्चभ्रस्ज\-सृजमृज\-यजराज\-भ्राजच्छशां षः} (पा॰सू॰~८.२.३६) इत्यस्याप्रवृत्तेः।} जश्त्वाभावश्च।\footnote{\textcolor{red}{झलां जशोऽन्ते} (पा॰सू॰~८.२.३९) इत्यस्याप्रवृत्तेः।} अहमेव \textcolor{red}{नाश्} रावण\-नाशयिता \textcolor{red}{अयामि}\footnote{\textcolor{red}{कटी गतौ} (धा॰पा॰~३२०) इत्यत्रेकारप्रश्लेषपक्षे गतौ \textcolor{red}{इ}\-धातुरपि। \textcolor{red}{इ इति चतुर्थधातुवादिनाम् – अयति} (मा॰धा॰वृ॰~१.२१५)। \textcolor{red}{इ गतौ} (धा॰पा॰~३२०)~\arrow \textcolor{red}{शेषात्कर्तरि परस्मैपदम्} (पा॰सू॰~१.३.७८)~\arrow \textcolor{red}{वर्तमाने लट्} (पा॰सू॰~३.२.१२३)~\arrow इ~लट्~\arrow इ~मिप्~\arrow इ~मि~\arrow \textcolor{red}{कर्तरि शप्‌} (पा॰सू॰~३.१.६८)~\arrow इ~शप्~मि~\arrow इ~अ~मि~\arrow \textcolor{red}{सार्वधातुकार्ध\-धातुकयोः} (पा॰सू॰~७.३.८४)~\arrow ए~अ~मि~\arrow \textcolor{red}{एचोऽयवायावः} (पा॰सू॰~६.१.७८)~\arrow अय्~अ~मि~\arrow \textcolor{red}{अतो दीर्घो यञि} (पा॰सू॰~७.३.१०१)~\arrow अय्~आ~मि~\arrow अयामि। यद्वा \textcolor{red}{अयत इत्ययः}। \textcolor{red}{अयँ गतौ} (धा॰पा॰~४७४) इति धातोः \textcolor{red}{नन्दि\-ग्रहि\-पचादिभ्यो ल्युणिन्यचः} (पा॰सू॰~३.१.१३४) इत्यनेन कर्तरि पचाद्यचि विभक्तिकार्ये। ततः \textcolor{red}{अय इवाऽचरामीत्ययामि}। अय~\arrow \textcolor{red}{सर्वप्राति\-पदिकेभ्य आचारे क्विब्वा वक्तव्यः} (वा॰~३.१.११)~\arrow अय~क्विँप्~\arrow अय~व्~\arrow \textcolor{red}{वेरपृक्तस्य} (पा॰सू॰~६.१.६७)~\arrow अय~\arrow \textcolor{red}{सनाद्यन्ता धातवः} (पा॰सू॰~३.१.३२)~\arrow धातुसञ्ज्ञा~\arrow \textcolor{red}{शेषात्कर्तरि परस्मैपदम्} (पा॰सू॰~१.३.७८)~\arrow \textcolor{red}{वर्तमाने लट्} (पा॰सू॰~३.२.१२३)~\arrow अय~लट्~\arrow अय~मिप्~\arrow अय~मि~\arrow \textcolor{red}{कर्तरि शप्‌} (पा॰सू॰~३.१.६८)~\arrow अय~शप्~मि~\arrow अय~अ~मि~\arrow \textcolor{red}{अतो गुणे} (पा॰सू॰~६.१.९७)~\arrow अय~मि~\arrow \textcolor{red}{अतो दीर्घो यञि} (पा॰सू॰~७.३.१०१)~\arrow अया~मि~\arrow अयामि।} अरण्यं व्रजामि।\end{sloppypar}
\section[अभिषेक्ष्यामि]{अभिषेक्ष्यामि}
\centering\textcolor{blue}{आज्ञापयति यद्यत्त्वां मुनिस्तत्तत्समानय।\nopagebreak\\
यौवराज्येऽभिषेक्ष्यामि श्वोभूते रघुनन्दनम्॥}\nopagebreak\\
\raggedleft{–~अ॰रा॰~२.२.७}\\
\fontsize{14}{21}\selectfont\begin{sloppypar}\hyphenrules{nohyphenation}\justifying\noindent\hspace{10mm} यद्यपि \textcolor{red}{श्वोभूते} इति पद\-समभिव्याहारेण लुड्लकारः प्राप्नोति\footnote{\textcolor{red}{अनद्यतने लुट्} (पा॰सू॰~३.३.१५) इत्यनेन।} तथा च स्पष्टं लकारार्थ\-निर्णये भूषणे~–\end{sloppypar}
\centering\textcolor{red}{वर्तमाने परोक्षे श्वोभाविन्यर्थे भविष्यति।\nopagebreak\\
विध्यादौ प्रार्थनादौ च क्रमाज्ज्ञेया लडादयः॥}\nopagebreak\\
\raggedleft{–~वै॰सि॰का॰~२२}\\
\fontsize{14}{21}\selectfont\begin{sloppypar}\hyphenrules{nohyphenation}\justifying\noindent तथाऽप्यनद्यतनत्वस्याविवक्षया लृट्।\footnote{अभि \textcolor{red}{षिचँ क्षरणे} (धा॰पा॰~१४३४)~\arrow अभि~षिच्~\arrow \textcolor{red}{धात्वादेः षः सः} (पा॰सू॰~६.१.६४)~\arrow अभि~सिच्~\arrow \textcolor{red}{शेषात्कर्तरि परस्मैपदम्} (पा॰सू॰~१.३.७८)~\arrow \textcolor{red}{लृट् शेषे च} (पा॰सू॰~३.३.१३)~\arrow अभि~सिच्~लृट्~\arrow अभि~सिच्~मिप्~\arrow अभि~सिच्~मि~\arrow \textcolor{red}{स्यतासी लृलुटोः} (पा॰सू॰~३.१.३३)~\arrow अभि~सिच्~स्य~मि~\arrow \textcolor{red}{एकाच उपदेशेऽनुदात्तात्‌} (पा॰सू॰~७.२.१०)~\arrow इडागम\-निषेधः~\arrow \textcolor{red}{पुगन्त\-लघूपधस्य च} (पा॰सू॰~७.३.८६)~\arrow अभि~सेच्~स्य~मि~\arrow \textcolor{red}{चोः कुः} (पा॰सू॰~८.२.३०)~\arrow अभि~सेक्~स्य~मि~\arrow \textcolor{red}{अतो दीर्घो यञि} (पा॰सू॰~७.३.१०१)~\arrow अभि~सेक्~स्या~मि~\arrow \textcolor{red}{आदेश\-प्रत्यययोः} (पा॰सू॰~८.३.५९)~\arrow अभि~षेक्~ष्या~मि~\arrow अभिषेक्ष्यामि।}\end{sloppypar}
\section[प्रविशस्व]{प्रविशस्व}
\centering\textcolor{blue}{रामाभिषेकविघ्नार्थं यतस्व ब्रह्मवाक्यतः।\nopagebreak\\
मन्थरां प्रविशस्वादौ कैकेयीं च ततः परम्॥}\nopagebreak\\
\raggedleft{–~अ॰रा॰~२.२.४५}\\
\fontsize{14}{21}\selectfont\begin{sloppypar}\hyphenrules{nohyphenation}\justifying\noindent\hspace{10mm} अत्र देवाः सरस्वतीं प्रार्थयन्ते यत् \textcolor{red}{त्वं मन्थरां प्रविशस्व}। \textcolor{red}{प्र}\-पूर्वको \textcolor{red}{विश्‌}\-धातुः (\textcolor{red}{विशँ प्रवेशने} धा॰पा॰~१४२४) परस्मैपदी। तत्र लोड्लकारे \textcolor{red}{प्रविश} इति पाणिनीयम्।\footnote{प्र~\textcolor{red}{विशँ प्रवेशने} (धा॰पा॰~१४२४)~\arrow प्र~विश्~\arrow \textcolor{red}{शेषात्कर्तरि परस्मैपदम्} (पा॰सू॰~१.३.७८)~\arrow \textcolor{red}{लोट् च} (पा॰सू॰~३.३.१६२)~\arrow प्र~विश्~लोट्~\arrow प्र~विश्~सिप्~\arrow प्र~विश्~सि~\arrow \textcolor{red}{तुदादिभ्यः शः} (पा॰सू॰~३.१.७७)~\arrow प्र~विश्~श~सि~\arrow प्र~विश्~अ~सि~\arrow \textcolor{red}{सार्वधातुकमपित्} (पा॰सू॰~१.२.४)~\arrow ङित्त्वम्~\arrow \textcolor{red}{ग्क्ङिति च} (पा॰सू॰~१.१.५)~\arrow लघूपध\-गुण\-निषेधः~\arrow \textcolor{red}{सेर्ह्यपिच्च} (पा॰सू॰~३.४.८७)~\arrow प्र~विश्~अ~हि~\arrow \textcolor{red}{अतो हेः} (पा॰सू॰~६.४.१०५)~\arrow प्रविश।} \textcolor{red}{प्रविशस्व} इति तु \textcolor{red}{कर्तरि कर्म\-व्यतिहारे} (पा॰सू॰~१.३.१४) इत्यनेन क्रिया\-विनिमय आत्मनेपदे सति \textcolor{red}{प्रविशस्व}।\footnote{प्र~\textcolor{red}{विशँ प्रवेशने} (धा॰पा॰~१४२४)~\arrow प्र~विश्~\arrow \textcolor{red}{कर्तरि कर्मव्यतिहारे} (पा॰सू॰~१.३.१४)~\arrow \textcolor{red}{लोट् च} (पा॰सू॰~३.३.१६२)~\arrow प्र~विश्~लोट्~\arrow प्र~विश्~थास्~\arrow प्र~विश्~थास्~\arrow \textcolor{red}{तुदादिभ्यः शः} (पा॰सू॰~३.१.७७)~\arrow प्र~विश्~श~थास्~\arrow प्र~विश्~अ~थास्~\arrow \textcolor{red}{सार्वधातुकमपित्} (पा॰सू॰~१.२.४)~\arrow ङित्त्वम्~\arrow \textcolor{red}{ग्क्ङिति च} (पा॰सू॰~१.१.५)~\arrow लघूपध\-गुण\-निषेधः~\arrow \textcolor{red}{थासस्से} (पा॰सू॰~३.४.८०)~\arrow प्र~विश्~अ~से~\arrow \textcolor{red}{सवाभ्यां वामौ} (पा॰सू॰~३.४.९१)~\arrow प्र~विश्~अ~स्व~\arrow प्रविशस्व।} 
यद्वा \textcolor{red}{प्रवेशनं प्रविश्}।\footnote{स्त्रियां भावे क्विप्। प्र~\textcolor{red}{विशँ प्रवेशने} (धा॰पा॰~१४२४)~\arrow प्र~विश्~\arrow \textcolor{red}{सम्पदादिभ्‍यः क्विप्} (वा॰~३.३.१०८)~\arrow प्र~विश्~क्विँप्~\arrow प्र~विश्~व्~\arrow \textcolor{red}{वेरपृक्तस्य} (पा॰सू॰~६.१.६७)~\arrow प्र~विश्~\arrow \textcolor{red}{ग्क्ङिति च} (पा॰सू॰~१.१.५)~\arrow लघूपध\-गुण\-निषेधः~\arrow प्रविश्~\arrow विभक्ति\-कार्यम्~\arrow प्रविश्~सुँ~\arrow \textcolor{red}{हल्ङ्याब्भ्यो दीर्घात्सुतिस्यपृक्तं हल्} (पा॰सू॰~६.१.६८)~\arrow \textcolor{red}{अयस्मयादीनि च्छन्दसि} (पा॰सू॰~१.४.२०)~\arrow षत्वाभावः~\arrow जश्त्वाभावः~\arrow प्रविश्।} \textcolor{red}{अयस्मयादीनि च्छन्दसि} (पा॰सू॰~१.४.२०) इत्यनेन छान्दस\-भत्वात्षत्वाभावो\footnote{\textcolor{red}{व्रश्चभ्रस्ज\-सृजमृज\-यजराज\-भ्राजच्छशां षः} (पा॰सू॰~८.२.३६) इत्यस्याप्रवृत्तेः।} जश्त्वाभावश्च।\footnote{\textcolor{red}{झलां जशोऽन्ते} (पा॰सू॰~८.२.३९) इत्यस्याप्रवृत्तेः।}
\textcolor{red}{प्रविशमस्वेनाऽत्मभिन्नेन कपटेनाऽतिक्रामतीति प्रविशस्वयति}।\footnote{अस्वमात्मभिन्नं कपटम्। अस्वेन कपटेनातिक्रामतीति \textcolor{red}{अस्वयति}। अस्व~\arrow \textcolor{red}{तेनातिक्रामति} (धा॰पा॰ ग॰सू॰~१८८)~\arrow अस्व णिच्~\arrow अस्व इ~\arrow \textcolor{red}{णाविष्ठवत्प्राति\-पदिकस्य पुंवद्भाव\-रभाव\-टिलोप\-यणादि\-परार्थम्} (वा॰~६.४.४८)~\arrow अस्व्~इ~\arrow अस्वि~\arrow \textcolor{red}{सनाद्यन्ता धातवः} (पा॰सू॰~३.१.३२)~\arrow धातु\-सञ्ज्ञा~\arrow \textcolor{red}{शेषात्कर्तरि परस्मैपदम्} (पा॰सू॰~१.३.७८)~\arrow \textcolor{red}{वर्तमाने लट्} (पा॰सू॰~३.२.१२३)~\arrow अस्वि~लट्~\arrow अस्वि~तिप्~\arrow अस्वि~ति~\arrow \textcolor{red}{कर्तरि शप्‌} (पा॰सू॰~३.१.६८)~\arrow अस्वि~शप्~ति~\arrow अस्वि~अ~ति~\arrow \textcolor{red}{सार्वधातुकार्धधातुकयोः} (पा॰सू॰~७.३.८४)~\arrow अस्वे~अ~ति~\arrow \textcolor{red}{एचोऽयवायावः} (पा॰सू॰~६.१.७८)~\arrow अस्वय्~अ~ति~\arrow अस्वयति। \textcolor{red}{प्रविशम् अस्वयति} इति स्थिते \textcolor{red}{सह सुपा} (पा॰सू॰~२.१.४) इत्यत्र \textcolor{red}{सह} इति योगविभागात्सुबन्तस्य तिङन्तेन समासे \textcolor{red}{प्रविशस्वयति}। \textcolor{red}{पर्यभूषयत्} (वै॰सि॰कौ॰~९३९) इतिवत्। तत्रत्या बालमनोरमा च~– \textcolor{red}{‘पर्यभूषयदिति॥’ ‘सह सुपा’ इत्यत्र सहेति योगविभागात् परीति सुबन्तस्य तिङन्तेन समासः} (बा॰म॰~९३९)।} \textcolor{red}{प्रविशस्वयतीति प्रविशस्वः}।\footnote{अस्वयतीत्यस्वः। \textcolor{red}{अस्वि}\-नामधातोः \textcolor{red}{नन्दि\-ग्रहि\-पचादिभ्यो ल्युणिन्यचः} (पा॰सू॰~३.१.१३४) इत्यनेन कर्तरि पचाद्यचि। \textcolor{red}{प्रविशि अस्वः} इति \textcolor{red}{प्रविशस्वः}। \textcolor{red}{सप्तमी शौण्डैः} (पा॰सू॰~२.१.४०) इत्यत्र \textcolor{red}{सप्तमी} इति योगविभागात्समासः। यद्वा \textcolor{red}{सह सुपा} (पा॰सू॰~२.१.४) इत्यनेन सुप्सुपासमासः।} \textcolor{red}{प्रविशस्व इवाऽचरतीति प्रविशस्वति}।\footnote{प्रविशस्व~\arrow \textcolor{red}{सर्वप्रातिपतिकेभ्य आचारे क्विब्वा वक्तव्यः} (वा॰~३.१.११)~\arrow प्रविशस्व~क्विप्~\arrow प्रविशस्व~व्~\arrow \textcolor{red}{वेरपृक्तस्य} (पा॰सू॰~६.१.६७)~\arrow प्रविशस्व~\arrow \textcolor{red}{सनाद्यन्ता धातवः} (पा॰सू॰~३.१.३२)~\arrow धातु\-सञ्ज्ञा~\arrow \textcolor{red}{शेषात्कर्तरि परस्मैपदम्} (पा॰सू॰~१.३.७८)~\arrow \textcolor{red}{वर्तमाने लट्} (पा॰सू॰~३.२.१२३)~\arrow प्रविशस्व~लट्~\arrow प्रविशस्व~तिप्~\arrow प्रविशस्व~ति~\arrow \textcolor{red}{कर्तरि शप्‌} (पा॰सू॰~३.१.६८)~\arrow प्रविशस्व~शप्~ति~\arrow प्रविशस्व~अ~ति~\arrow \textcolor{red}{अतो गुणे} (पा॰सू॰~६.१.९७)~\arrow प्रविशस्व~ति~\arrow प्रविशस्वति।} इत्थमाचक्षाण\-णिजन्तात्पचाद्यच्पुनराचारक्विप्ततो लोण्मध्यमपुरुष एकवचने \textcolor{red}{प्रविशस्व}।\footnote{प्रविशस्व~\arrow धातु\-सञ्ज्ञा (पूर्ववत्)~\arrow \textcolor{red}{शेषात्कर्तरि परस्मैपदम्} (पा॰सू॰~१.३.७८)~\arrow \textcolor{red}{लोट् च} (पा॰सू॰~३.३.१६२)~\arrow प्रविशस्व~लोट्~\arrow प्रविशस्व~सिप्~\arrow प्रविशस्व~सि~\arrow \textcolor{red}{कर्तरि शप्‌} (पा॰सू॰~३.१.६८)~\arrow प्रविशस्व~शप्~सि~\arrow प्रविशस्व~अ~सि~\arrow \textcolor{red}{अतो गुणे} (पा॰सू॰~६.१.९७)~\arrow प्रविशस्व~सि~\arrow \textcolor{red}{सेर्ह्यपिच्च} (पा॰सू॰~३.४.८७)~\arrow प्रविशस्व~हि~\arrow \textcolor{red}{अतो हेः} (पा॰सू॰~६.४.१०५)~\arrow प्रविशस्व।}\end{sloppypar}
\section[चक्रे]{चक्रे}
\centering\textcolor{blue}{ततो विघ्ने समुत्पन्ने पुनरेहि दिविं शुभे।\nopagebreak\\
तथेत्युक्त्वा तथा चक्रे प्रविवेशाथ मन्थराम्॥}\nopagebreak\\
\raggedleft{–~अ॰रा॰~२.२.४६}\\
\fontsize{14}{21}\selectfont\begin{sloppypar}\hyphenrules{nohyphenation}\justifying\noindent\hspace{10mm} क्रियाफलस्य कर्तृ\-भूतायां सरस्वत्यां गामित्वादात्मनेपदम्।\footnote{\textcolor{red}{स्वरितञितः कर्त्रभिप्राये क्रियाफले} (पा॰सू॰~१.३.७२) इत्यनेन। \textcolor{red}{डुकृञ् करणे} (धा॰पा॰~१४७२)~\arrow कृ~\arrow \textcolor{red}{स्वरितञितः कर्त्रभिप्राये क्रियाफले} (पा॰सू॰~१.३.७२)~\arrow \textcolor{red}{परोक्षे लिट्} (पा॰सू॰~३.२.११५)~\arrow कृ~लिट्~\arrow कृ~इट्~\arrow कृ~इ~\arrow \textcolor{red}{लिटि धातोरनभ्यासस्य} (पा॰सू॰~६.१.८)~\arrow कृ~कृ~इ~\arrow \textcolor{red}{उरत्‌} (पा॰सू॰~७.४.६६)~\arrow \textcolor{red}{उरण् रपरः} (पा॰सू॰~१.१.५१)~\arrow कर्~कृ~ए~\arrow \textcolor{red}{हलादिः शेषः} (पा॰सू॰~७.४.६०)~\arrow क~कृ~ए~\arrow \textcolor{red}{कुहोश्चुः} (पा॰सू॰~७.४.६२)~\arrow च~कृ~ए~\arrow \textcolor{red}{असंयोगाल्लिट् कित्} (पा॰सू॰~१.२.५)~\arrow कित्त्वम्~\arrow \textcolor{red}{ग्क्ङिति च} (पा॰सू॰~१.१.५)~\arrow गुणनिषेधः~\arrow \textcolor{red}{इको यणचि} (पा॰सू॰~६.१.७७)~\arrow च~क्र्~ए~\arrow चक्रे।} क्रिया\-फलं हि लीला\-साहाय्येन भगवदानुकूल्य\-रूपम्।\end{sloppypar}
\section[विवास्यते]{विवास्यते}
\centering\textcolor{blue}{भरतो राघवस्याग्रे किङ्करो वा भविष्यति।\nopagebreak\\
विवास्यते वा नगरात्प्राणैर्वा हायतेऽचिरात्॥}\nopagebreak\\
\raggedleft{–~अ॰रा॰~२.२.६२}\\
\fontsize{14}{21}\selectfont\begin{sloppypar}\hyphenrules{nohyphenation}\justifying\noindent\hspace{10mm} अत्रापि \textcolor{red}{विवासयिष्यते}\footnote{\textcolor{red}{वसँ निवासे} (धा॰पा॰~१००५)~\arrow वस्~\arrow \textcolor{red}{हेतुमति च} (पा॰सू॰~३.१.२६)~\arrow वस्~णिच्~\arrow वस्~इ~\arrow \textcolor{red}{अत उपधायाः} (पा॰सू॰~७.२.११६)~\arrow वास्~इ~\arrow वासि~\arrow \textcolor{red}{सनाद्यन्ता धातवः} (पा॰सू॰~३.१.३२)~\arrow धातु\-सञ्ज्ञा। वि~वासि~\arrow \textcolor{red}{भावकर्मणोः} (पा॰सू॰~१.३.१३)~\arrow \textcolor{red}{लृट् शेषे च} (पा॰सू॰~३.३.१३)~\arrow वि~वासि~लृट्~\arrow वि~वासि~त~\arrow \textcolor{red}{स्यतासी लृलुटोः} (पा॰सू॰~३.१.३३)~\arrow वि~वासि~स्य~त~\arrow \textcolor{red}{आर्धधातुकस्येड्वलादेः} (पा॰सू॰~७.२.३५)~\arrow वि~वासि~इट्~स्य~त~\arrow वि~वासि~इ~स्य~त~\arrow \textcolor{red}{सार्वधातुकार्ध\-धातुकयोः} (पा॰सू॰~७.३.८४)~\arrow वि~वासे~इ~स्य~त~\arrow \textcolor{red}{एचोऽयवायावः} (पा॰सू॰~६.१.७८)~\arrow वि~वासय्~इ~स्य~त~\arrow \textcolor{red}{टित आत्मनेपदानां टेरे} (पा॰सू॰~३.४.७९)~\arrow वि~वासय्~इ~स्य~ते~\arrow \textcolor{red}{आदेश\-प्रत्यययोः} (पा॰सू॰~८.३.५९)~\arrow वि~वासय्~इ~ष्य~ते~\arrow विवासयिष्यते।} इति प्रयोक्तव्ये \textcolor{red}{विवास्यते}\footnote{वि~वासि (पूर्ववत्)~\arrow \textcolor{red}{भावकर्मणोः} (पा॰सू॰~१.३.१३)~\arrow \textcolor{red}{वर्तमान\-सामीप्ये वर्तमानवद्वा} (पा॰सू॰~३.३.१३१)~\arrow \textcolor{red}{वर्तमाने लट्} (पा॰सू॰~३.२.१२३)~\arrow वि~वासि~लट्~\arrow वि~वासि~त~\arrow \textcolor{red}{सार्वधातुके यक्} (पा॰सू॰~३.१.६७)~\arrow \textcolor{red}{आद्यन्तौ टकितौ} (पा॰सू॰~१.१.४६)~\arrow वि~वासि~यक्~त~\arrow वि~वासि~य~त~\arrow \textcolor{red}{णेरनिटि} (पा॰सू॰~६.४.५१)~\arrow वि~वास्~य~त~\arrow \textcolor{red}{टित आत्मनेपदानां टेरे} (पा॰सू॰~३.४.७९)~\arrow वि~वास्~य~ते~\arrow विवास्यते।
} इति वर्तमान\-सामीप्ये लट्।\footnote{\textcolor{red}{वर्तमान\-सामीप्ये वर्तमानवद्वा} (पा॰सू॰~३.३.१३१) इत्यनेन।}\end{sloppypar}
\section[भूयात्]{भूयात्}
\centering\textcolor{blue}{त्वय्येव तिष्ठतु चिरं न्यासभूतं ममानघ।\nopagebreak\\
यदा मेऽवसरो भूयात्तदा देहि वरद्वयम्॥}\nopagebreak\\
\raggedleft{–~अ॰रा॰~२.२.७२}\\
\fontsize{14}{21}\selectfont\begin{sloppypar}\hyphenrules{nohyphenation}\justifying\noindent\hspace{10mm} अत्र पूर्वोक्तमेव समाधानम्।\footnote{\pageref{sec:bhuyat1}तमे पृष्ठे \ref{sec:bhuyat1} \nameref{sec:bhuyat1} इति प्रयोगस्य विमर्शं पश्यन्तु।}\end{sloppypar}
\section[न जाने]{न जाने}
\centering\textcolor{blue}{एवं त्वां बुद्धिसम्पन्नां न जाने वक्रसुन्दरि।\nopagebreak\\
भरतो यदि राजा मे भविष्यति सुतः प्रियः॥}\nopagebreak\\
\raggedleft{–~अ॰रा॰~२.२.७७}\\
\fontsize{14}{21}\selectfont\begin{sloppypar}\hyphenrules{nohyphenation}\justifying\noindent\hspace{10mm} मन्थरां कैकेयी प्रशंसति यत्पूर्वं \textcolor{red}{त्वां बुद्धि\-सम्पन्नां न जाने}। न चात्र \textcolor{red}{न अजाने} इत्येव लङ्प्रयोगः।\footnote{पूर्वपक्षोऽयम्।} अट्कथं न श्रूयत इति चेत्पर\-रूपत्वात् \textcolor{red}{विनाऽपि प्रत्ययं पूर्वोत्तर\-पद\-लोपो वक्तव्यः} (वा॰~५.३.८३) इत्यनेन लोपाद्वा।\footnote{अयमपि पूर्वपक्षः।} तथाऽप्याकार\-लोपस्तु दुर्वार एव।\footnote{उत्तरपक्षोऽयम्। \textcolor{red}{श्नाभ्यस्तयोरातः} (पा॰सू॰~६.४.११२) इत्यनेनाकार\-लोपे \textcolor{red}{अजानि} इति रूपम्। \textcolor{red}{ज्ञा अवबोधने} (धा॰पा॰~१५०७)~\arrow ज्ञा~\arrow \textcolor{red}{अनुपसर्गाज्ज्ञः} (पा॰सू॰~१.३.७६)~\arrow \textcolor{red}{अनद्यतने लङ्} (पा॰सू॰~३.२.१११)~\arrow ज्ञा~लङ्~\arrow ज्ञा~इट्~\arrow ज्ञा~इ~\arrow \textcolor{red}{लुङ्लङ्लृङ्क्ष्वडुदात्तः} (पा॰सू॰~६.४.७१)~\arrow \textcolor{red}{आद्यन्तौ टकितौ} (पा॰सू॰~१.१.४६)~\arrow अट्~ज्ञा~इ~\arrow अ~ज्ञा~इ~\arrow \textcolor{red}{क्र्यादिभ्यः श्ना} (पा॰सू॰~३.१.८१)~\arrow अ~ज्ञा~श्ना~इ~\arrow अ~ज्ञा~ना~इ~\arrow \textcolor{red}{ज्ञाजनोर्जा} (पा॰सू॰~७.३.७९)~\arrow अ~जा~ना~इ~\arrow \textcolor{red}{श्नाभ्यस्तयोरातः} (पा॰सू॰~६.४.११२)~\arrow अ~जा~न्~इ~\arrow अजानि।} उच्यते। अत्र स्मयोगे लट्।\footnote{\textcolor{red}{स्म} इत्यध्याहार्यमिति भावः। \textcolor{red}{ज्ञा अवबोधने} (धा॰पा॰~१५०७)~\arrow ज्ञा~\arrow \textcolor{red}{अनुपसर्गाज्ज्ञः} (पा॰सू॰~१.३.७६)~\arrow \textcolor{red}{लट् स्मे} (पा॰सू॰~३.२.११८)~\arrow ज्ञा~लट्~\arrow ज्ञा~इट्~\arrow ज्ञा~इ~\arrow \textcolor{red}{क्र्यादिभ्यः श्ना} (पा॰सू॰~३.१.८१)~\arrow ज्ञा~श्ना~इ~\arrow ज्ञा~ना~इ~\arrow \textcolor{red}{ज्ञाजनोर्जा} (पा॰सू॰~७.३.७९)~\arrow जा~ना~इ~\arrow \textcolor{red}{श्नाभ्यस्तयोरातः} (पा॰सू॰~६.४.११२)~\arrow जा~न्~इ~\arrow \textcolor{red}{टित आत्मनेपदानां टेरे} (पा॰सू॰~३.४.७९)~\arrow जा~न्~ए~\arrow जाने।} यद्वा पुरा\-योगे लट् \textcolor{red}{पुरि लुङ् चास्मे} (पा॰सू॰~३.२.१२२) इत्यनेन।\footnote{\textcolor{red}{पुरा} इत्यध्याहार्यमिति भावः। सिद्धिः पूर्ववत्।} \textcolor{red}{पुरा त्वामीदृशीं बुद्धिमतीं न जाने}।\end{sloppypar}
\section[त्यक्ष्ये]{त्यक्ष्ये}
\centering\textcolor{blue}{प्राणांस्त्यक्ष्येऽथ वा वक्रे शयिष्ये तावदेव हि।\nopagebreak\\
निश्चयं कुरु कल्याणि कल्याणं ते भविष्यति॥}\nopagebreak\\
\raggedleft{–~अ॰रा॰~२.२.८१}\\
\fontsize{14}{21}\selectfont\begin{sloppypar}\hyphenrules{nohyphenation}\justifying\noindent\hspace{10mm} \textcolor{red}{त्यज्}\-धातुः (\textcolor{red}{त्यजँ हानौ} धा॰पा॰~९८६) आत्मनेपदी न। अत्र \textcolor{red}{इट्} प्रत्ययस्य सम्भावना कथम्। अत्रापि कर्मव्यतिहारादात्मनेपदम्।\footnote{\textcolor{red}{कर्तरि कर्मव्यतिहारे} (पा॰सू॰~१.३.१४) इत्यनेन। \textcolor{red}{त्यजँ हानौ} (धा॰पा॰~९८६)~\arrow त्यज्~\arrow \textcolor{red}{कर्तरि कर्मव्यतिहारे} (पा॰सू॰~१.३.१४)~\arrow \textcolor{red}{लृट् शेषे च} (पा॰सू॰~३.३.१३)~\arrow त्यज्~लृट्~\arrow त्यज्~इट्~\arrow त्यज्~इ~\arrow \textcolor{red}{स्यतासी लृलुटोः} (पा॰सू॰~३.१.३३)~\arrow त्यज्~स्य~इ~\arrow \textcolor{red}{चोः कुः} (पा॰सू॰~८.२.३०)~\arrow त्यक्~स्य~इ~\arrow \textcolor{red}{टित आत्मनेपदानां टेरे} (पा॰सू॰~३.४.७९)~\arrow त्यक्~स्य~ए~\arrow \textcolor{red}{अतो गुणे} (पा॰सू॰~६.१.९७)~\arrow त्यक्~स्ये~\arrow \textcolor{red}{आदेश\-प्रत्यययोः} (पा॰सू॰~८.३.५९)~\arrow तयक्~ष्ये~\arrow \textcolor{red}{टित आत्मनेपदानां टेरे} (पा॰सू॰~३.४.७९)~\arrow त्यक्ष्ये।}\end{sloppypar}
\section[शयिष्ये]{शयिष्ये}
\centering\textcolor{blue}{प्राणांस्त्यक्ष्येऽथ वा वक्रे शयिष्ये तावदेव हि।\nopagebreak\\
निश्चयं कुरु कल्याणि कल्याणं ते भविष्यति॥}\nopagebreak\\
\raggedleft{–~अ॰रा॰~२.२.८१}\\
\fontsize{14}{21}\selectfont\begin{sloppypar}\hyphenrules{nohyphenation}\justifying\noindent\hspace{10mm} न च \textcolor{red}{शीङ्‌}\-धातुः (\textcolor{red}{शीङ् स्वप्ने} धा॰पा॰~१०३२) अदादिरनिट्।\footnote{पूर्वपक्षोऽयम्।} यथा \textcolor{red}{शेष्ये} इति वाल्मीकिः।\footnote{\textcolor{red}{शेष्ये पुरस्ताच्छालाया यावन्न प्रतियास्यति} (वा॰रा॰~२.१०३.१४) इति गोविन्द\-राजीय\-टीका\-पाठः।} \textcolor{red}{शयिष्ये} इति कथम्।\footnote{अयमपि पूर्वपक्ष इति बोध्यम्।} अत्र शयनं शयः।\footnote{\textcolor{red}{एरच्} (पा॰सू॰~३.३.५६) इत्यनेन \textcolor{red}{अच्‌}\-प्रत्यये। \textcolor{red}{शीङ् स्वप्ने} (धा॰पा॰~१०३२)~\arrow शी~\arrow \textcolor{red}{एरच्} (पा॰सू॰~३.३.५६)~\arrow शी~अच्~\arrow शी~अ~\arrow \textcolor{red}{सार्वधातुकार्ध\-धातुकयोः} (पा॰सू॰~७.३.८४)~\arrow शे~अ~\arrow \textcolor{red}{एचोऽयवायावः} (पा॰सू॰~६.१.७८)~\arrow शय्~अ~\arrow शय~\arrow विभक्तिकार्यम्~\arrow शयः। शयशब्दः शय्यायामपि। \textcolor{red}{शयः शय्याहिपाणिषु} (मे॰को॰~२६.५७) इति मेदिनी। तत्र \textcolor{red}{घ}\-प्रत्ययः। \textcolor{red}{शीङ् स्वप्ने} (धा॰पा॰~१०३२)~\arrow \textcolor{red}{पुंसि संज्ञायां घः प्रायेण} (पा॰सू॰~३.३.११८)~\arrow शी~घ~\arrow शी~अ~\arrow \textcolor{red}{सार्वधातुकार्ध\-धातुकयोः} (पा॰सू॰~७.३.८४)~\arrow शे~अ~\arrow \textcolor{red}{एचोऽयवायावः} (पा॰सू॰~६.१.७८)~\arrow शय्~अ~\arrow शय~\arrow विभक्तिकार्यम्~\arrow शयः।} \textcolor{red}{शयमाचरतीति शयते}\footnote{शय~\arrow \textcolor{red}{सर्वप्राति\-पदिकेभ्य आचारे क्विब्वा वक्तव्यः} (वा॰~३.१.११)~\arrow शय~क्विँप्~\arrow शय~व्~\arrow \textcolor{red}{वेरपृक्तस्य} (पा॰सू॰~६.१.६७)~\arrow शय~\arrow \textcolor{red}{सनाद्यन्ता धातवः} (पा॰सू॰~३.१.३२)~\arrow धातुसञ्ज्ञा~\arrow \textcolor{red}{कर्तरि कर्मव्यतिहारे} (पा॰सू॰~१.३.१४)~\arrow \textcolor{red}{वर्तमाने लट्} (पा॰सू॰~३.२.१२३)~\arrow शय~लट्~\arrow शय~त~\arrow \textcolor{red}{कर्तरि शप्‌} (पा॰सू॰~३.१.६८)~\arrow शय~शप्~त~\arrow शय~अ~त~\arrow \textcolor{red}{अतो गुणे} (पा॰सू॰~६.१.९७)~\arrow शय~त~\arrow \textcolor{red}{टित आत्मनेपदानां टेरे} (पा॰सू॰~३.४.७९)~\arrow शय~ते~\arrow शयते।} पुनर्लृड्लकार उत्तम\-पुरुष एकवचने \textcolor{red}{शयिष्ये}।\footnote{शय~\arrow धातुसञ्ज्ञा (पूर्ववत्)~\arrow \textcolor{red}{कर्तरि कर्मव्यतिहारे} (पा॰सू॰~१.३.१४)~\arrow \textcolor{red}{लृट् शेषे च} (पा॰सू॰~३.३.१३)~\arrow शय~लृट्~\arrow शय~इट्~\arrow शय~इ~\arrow \textcolor{red}{स्यतासी लृलुटोः} (पा॰सू॰~३.१.३३)~\arrow शय~स्य~इ~\arrow \textcolor{red}{आर्धधातुकस्येड्वलादेः} (पा॰सू॰~७.२.३५)~\arrow शय~इट्~स्य~इ~\arrow शय~इ~स्य~ति~\arrow \textcolor{red}{अतो लोपः} (पा॰सू॰~६.४.४८)~\arrow शय्~इ~स्य~इ~\arrow \textcolor{red}{टित आत्मनेपदानां टेरे} (पा॰सू॰~३.४.७९)~\arrow शय्~इ~स्य~ए~\arrow \textcolor{red}{अतो गुणे} (पा॰सू॰~६.१.९७)~\arrow शय्~इ~स्ये~\arrow \textcolor{red}{आदेश\-प्रत्यययोः} (पा॰सू॰~८.३.५९)~\arrow शय्~इ~ष्ये~\arrow शयिष्ये।} यद्वा \textcolor{red}{शय्} इति हलन्तं क्रिया\-विशेषणम्। शयनं कुर्वतीष्टास्मीति \textcolor{red}{इष्ये}।\footnote{\textcolor{red}{इषुँ इच्छायाम्} धा॰पा॰~१३५१)~\arrow इष्~\arrow \textcolor{red}{भावकर्मणोः} (पा॰सू॰~१.३.१३)~\arrow \textcolor{red}{वर्तमाने लट्} (पा॰सू॰~३.२.१२३)~\arrow इष्~लट्~\arrow इष्~इट्~\arrow इष्~इ~\arrow \textcolor{red}{सार्वधातुके यक्} (पा॰सू॰~३.१.६७)~\arrow \textcolor{red}{आद्यन्तौ टकितौ} (पा॰सू॰~१.१.४६) इष्~यक्~इ~\arrow इष्~य~इ~\arrow \textcolor{red}{ग्क्ङिति च} (पा॰सू॰~१.१.५)~\arrow लघूपध\-गुण\-निषेधः~\arrow इष्~य~इ~\arrow \textcolor{red}{टित आत्मनेपदानां टेरे} (पा॰सू॰~३.४.७९)~\arrow इष्~य~ए~\arrow \textcolor{red}{अतो गुणे} (पा॰सू॰~६.१.९७)~\arrow इष्~ये~\arrow इष्ये।} \textcolor{red}{इषु}\-धातोः (\textcolor{red}{इषुँ इच्छायाम्} धा॰पा॰~१३५१) कर्म\-वाच्ये वर्तमान\-काल उत्तम\-पुरुष एक\-वचने रूपम्।\footnote{पूर्वपक्षं मत्वैवैते द्वे समाधाने उक्ते।} वस्तुतस्तु \textcolor{red}{शीङ्‌}\-धातुः (\textcolor{red}{शीङ् स्वप्ने} धा॰पा॰~१०३२) सेडेव। यथा तत्रैव \textcolor{red}{शयित्वा}।\footnote{\textcolor{red}{शयित्वा पुरुषव्याघ्रः कथं शेते महीतले} (वा॰रा॰~२.८८.४)।} \textcolor{red}{शेष्ये} इत्यत्र त्विडभावोऽनित्यत्वात्।\footnote{\textcolor{red}{आगम\-शास्त्रमनित्यम्} (प॰शे॰~९३.२)। \textcolor{red}{शेष्ये} (वा॰रा॰~२.१०३.१४) इत्यत्र भूषण\-कारा गोविन्द\-राजाश्च~– \textcolor{red}{शेष्ये शयिष्ये। इडभाव आर्षः} (वा॰रा॰ भू॰टी॰~२.१०३.१४)।}\end{sloppypar}
\section[उपायाति]{उपायाति}
\centering\textcolor{blue}{हसन्ती मामुपायाति सा किं नैवाद्य दृश्यते।\nopagebreak\\
इत्यात्मन्येव सञ्चिन्त्य मनसाऽतिविदूयता॥}\nopagebreak\\
\raggedleft{–~अ॰रा॰~२.३.३}\\
\fontsize{14}{21}\selectfont\begin{sloppypar}\hyphenrules{nohyphenation}\justifying\noindent\hspace{10mm} \textcolor{red}{उपायात्}\footnote{उप~आङ्~\textcolor{red}{या प्रापणे} (धा॰पा॰~१०४९)~\arrow उप~आ~या~\arrow \textcolor{red}{शेषात्कर्तरि परस्मैपदम्} (पा॰सू॰~१.३.७८)~\arrow \textcolor{red}{अनद्यतने लङ्} (पा॰सू॰~३.२.१११)~\arrow उप~आ~या~लङ्~\arrow उप~आ~या~तिप्~\arrow उप~आ~या~ति~\arrow \textcolor{red}{लुङ्लङ्लृङ्क्ष्वडुदात्तः} (पा॰सू॰~६.४.७१)~\arrow \textcolor{red}{आद्यन्तौ टकितौ} (पा॰सू॰~१.१.४६)~\arrow उप~आ~अट्~या~ति~\arrow उप~आ~अ~या~ति~\arrow \textcolor{red}{कर्तरि शप्‌} (पा॰सू॰~३.१.६८)~\arrow उप~आ~अ~या~शप्~ति~\arrow \textcolor{red}{अदिप्रभृतिभ्यः शपः} (पा॰सू॰~२.४.७२)~\arrow उप~आ~अ~या~ति~\arrow \textcolor{red}{इतश्च} (पा॰सू॰~३.४.१००)~\arrow उप~आ~अ~या~त्~\arrow \textcolor{red}{अकः सवर्णे दीर्घः} (पा॰सू॰~६.१.१०१)~\arrow उपा~अ~या~त्~\arrow \textcolor{red}{अकः सवर्णे दीर्घः} (पा॰सू॰~६.१.१०१)~\arrow उपा~या~त~\arrow उपायात्।} इति प्रयोक्तव्ये \textcolor{red}{उपायाति} इति हि \textcolor{red}{स्म}\-शब्द\-योगे।\footnote{\textcolor{red}{स्म} इत्यध्याहार्यमिति भावः।} तथा च \textcolor{red}{लट् स्मे} (पा॰सू॰~३.२.११८) इत्यनेन लड्लकारः।\footnote{उप~आङ्~\textcolor{red}{या प्रापणे} (धा॰पा॰~१०४९)~\arrow उप~आ~या~\arrow \textcolor{red}{शेषात्कर्तरि परस्मैपदम्} (पा॰सू॰~१.३.७८)~\arrow \textcolor{red}{लट् स्मे} (पा॰सू॰~३.२.११८)~\arrow उप~आ~या~लट्~\arrow उप~आ~या~तिप्~\arrow उप~आ~या~ति~\arrow \textcolor{red}{कर्तरि शप्‌} (पा॰सू॰~३.१.६८)~\arrow उप~आ~या~शप्~ति~\arrow \textcolor{red}{अदिप्रभृतिभ्यः शपः} (पा॰सू॰~२.४.७२)~\arrow उप~आ~या~ति~\arrow \textcolor{red}{अकः सवर्णे दीर्घः} (पा॰सू॰~६.१.१०१)~\arrow उपा~या~ति~\arrow उपायाति।} लोपत्वात् \textcolor{red}{स्म} इत्यस्य श्रवणं न।\footnote{\textcolor{red}{विनाऽपि प्रत्ययं पूर्वोत्तर\-पद\-लोपो वक्तव्यः} (वा॰~५.३.८३) इत्यनेन।}\end{sloppypar}
\section[विद्महे]{विद्महे}
\centering\textcolor{blue}{ता ऊचुः क्रोधभवनं प्रविष्टा नैव विद्महे।\nopagebreak\\
कारणं तत्र देव त्वं गच्छ निश्चेतुमर्हसि॥}\nopagebreak\\
\raggedleft{–~अ॰रा॰~२.३.५}\\
\fontsize{14}{21}\selectfont\begin{sloppypar}\hyphenrules{nohyphenation}\justifying\noindent\hspace{10mm} \textcolor{red}{विद्महे} इत्यत्र \textcolor{red}{हे} इति पृथक्पदम्। \textcolor{red}{हे देव कारणं नैव विद्म} इत्यन्वयः करणीयः। किमिदं \textcolor{red}{विद्म}। न चात्र \textcolor{red}{वेदितुं वयं न शक्नुमह} इत्यर्थे \textcolor{red}{शकि लिङ् च} (पा॰सू॰~३.३.१७२) इत्यनेन यद्वा \textcolor{red}{कारणं वेदितुं वयं नार्हाः} इत्यर्थे \textcolor{red}{अर्हे कृत्यतृचश्च} (पा॰सू॰~३.३.१६९) इत्यनेन लिङ् लोड्वा।\footnote{\textcolor{red}{विद्म} इत्यत्राडभावान्न लुङ्लङ्लृङो विसर्गाभावान्न लड्द्वित्वाभावान्न लिट्स्यतास्यभावाच्च न लृलुटावित्याशङ्क्य पूर्वपक्षोऽयम्।} तत्र \textcolor{red}{विद्याम}\footnote{\textcolor{red}{विदँ ज्ञाने} (धा॰पा॰~१०६४)~\arrow विद्~\arrow \textcolor{red}{शेषात्कर्तरि परस्मैपदम्} (पा॰सू॰~१.३.७८)~\arrow \textcolor{red}{शकि लिङ् च} (पा॰सू॰~३.३.१७२)~\arrow विद्~लिङ्~\arrow विद्~मस्~\arrow \textcolor{red}{कर्तरि शप्‌} (पा॰सू॰~३.१.६८)~\arrow विद्~शप्~मस्~\arrow \textcolor{red}{अदिप्रभृतिभ्यः शपः} (पा॰सू॰~२.४.७२)~\arrow विद्~मस्~\arrow \textcolor{red}{यासुट् परस्मै\-पदेषूदात्तो ङिच्च} (पा॰सू॰~३.४.१०३)~\arrow विद्~यासुँट्~मस्~\arrow विद्~यास्~मस्~\arrow \textcolor{red}{ग्क्ङिति च} (पा॰सू॰~१.१.५)~\arrow पुगन्त\-लघूपध\-गुण\-निषेधः~\arrow \textcolor{red}{लिङः सलोपोऽनन्त्यस्य} (पा॰सू॰~७.२.७९)~\arrow विद्~या~मस्~\arrow नित्यं ङितः~\arrow विद्~या~म~\arrow विद्याम।} \textcolor{red}{वेदाम}\footnote{\textcolor{red}{विदँ ज्ञाने} (धा॰पा॰~१०६४)~\arrow विद्~\arrow \textcolor{red}{शेषात्कर्तरि परस्मैपदम्} (पा॰सू॰~१.३.७८)~\arrow \textcolor{red}{अर्हे कृत्यतृचश्च} (पा॰सू॰~३.३.१६९)~\arrow विद्~लोट्~\arrow विद्~मस्~\arrow \textcolor{red}{कर्तरि शप्‌} (पा॰सू॰~३.१.६८)~\arrow विद्~शप्~मस्~\arrow \textcolor{red}{अदिप्रभृतिभ्यः शपः} (पा॰सू॰~२.४.७२)~\arrow विद्~मस्~\arrow \textcolor{red}{आडुत्तमस्य पिच्च} (पा॰सू॰~३.४.९२)~\arrow विद्~आट्~मस्~\arrow विद्~आ~मस्~\arrow \textcolor{red}{पुगन्त\-लघूपधस्य च} (पा॰सू॰~७.३.८६)~\arrow वेद्~आ~मस्~\arrow \textcolor{red}{लोटो लङ्वत्‌} (पा॰सू॰~३.४.८५)~\arrow ङिद्वत्त्वम्~\arrow नित्यं ङितः~\arrow वेद्~आ~म~\arrow वेदाम।} इति रूपे। अत्र \textcolor{red}{विदो लटो वा} (पा॰सू॰~३.४.८३) इत्यनेन लटि मसो मादेशे \textcolor{red}{विद्म}।\footnote{\textcolor{red}{हे} इति पृथक्पदमिति भावः। \textcolor{red}{विदँ ज्ञाने} (धा॰पा॰~१०६४)~\arrow विद्~\arrow \textcolor{red}{शेषात्कर्तरि परस्मैपदम्} (पा॰सू॰~१.३.७८)~\arrow \textcolor{red}{वर्तमाने लट्} (पा॰सू॰~३.२.१२३)~\arrow विद्~लट्~\arrow विद्~मस्~\arrow \textcolor{red}{विदो लटो वा} (पा॰सू॰~३.४.८३)~\arrow विद्~म~\arrow \textcolor{red}{कर्तरि शप्‌} (पा॰सू॰~३.१.६८)~\arrow विद्~शप्~म~\arrow \textcolor{red}{अदिप्रभृतिभ्यः शपः} (पा॰सू॰~२.४.७२)~\arrow विद्~म~\arrow विद्म।} यद्वा कर्म\-व्यतिहार आत्मनेपदम्।\footnote{\textcolor{red}{कर्तरि कर्मव्यतिहारे} (पा॰सू॰~१.३.१४) इत्यनेन।} ततो लड्लकार उत्तमपुरुषे बहुवचने \textcolor{red}{महिङ्} प्रत्ययः।\footnote{\textcolor{red}{विदँ ज्ञाने} (धा॰पा॰~१०६४)~\arrow विद्~\arrow \textcolor{red}{कर्तरि कर्मव्यतिहारे} (पा॰सू॰~१.३.१४)~\arrow \textcolor{red}{वर्तमाने लट्} (पा॰सू॰~३.२.१२३)~\arrow विद्~लट्~\arrow विद्~महिङ्~\arrow विद्~महि~\arrow \textcolor{red}{कर्तरि शप्‌} (पा॰सू॰~३.१.६८)~\arrow विद्~शप्~महि~\arrow \textcolor{red}{अदिप्रभृतिभ्यः शपः} (पा॰सू॰~२.४.७२)~\arrow विद्~महि~\arrow \textcolor{red}{टित आत्मनेपदानां टेरे} (पा॰सू॰~३.४.७९)~\arrow विद्~महे~\arrow विद्महे।}
\end{sloppypar}
\section[वधिष्यामि]{वधिष्यामि}
\label{sec:vadhisyami}
\centering\textcolor{blue}{ब्रूहि किं वा वधिष्यामि वधार्हो वा विमोक्ष्यसे।\nopagebreak\\
किमत्र बहुनोक्तेन प्राणान्दास्यामि ते प्रिये॥}\nopagebreak\\
\raggedleft{–~अ॰रा॰~२.३.१३}\\
\fontsize{14}{21}\selectfont\begin{sloppypar}\hyphenrules{nohyphenation}\justifying\noindent\hspace{10mm} अत्र महाराजो दशरथः \textcolor{red}{हनिष्यामि}\footnote{\textcolor{red}{हनँ हिंसागत्योः} (धा॰पा॰~१०१२)~\arrow हन्~\arrow \textcolor{red}{शेषात्कर्तरि परस्मैपदम्} (पा॰सू॰~१.३.७८)~\arrow \textcolor{red}{लृट् शेषे च} (पा॰सू॰~३.३.१३)~\arrow हन्~लृट्~\arrow हन्~मिप्~\arrow हन्~मि~\arrow \textcolor{red}{स्यतासी लृलुटोः} (पा॰सू॰~३.१.३३)~\arrow हन्~स्य~मि~\arrow \textcolor{red}{ऋद्धनोः स्ये} (पा॰सू॰~७.२.७०)~\arrow हन्~इट्~स्य~मि~\arrow हन्~इ~स्य~मि~\arrow \textcolor{red}{अतो दीर्घो यञि} (पा॰सू॰~७.३.१०१)~\arrow हन्~इ~स्या~मि~\arrow \textcolor{red}{आदेश\-प्रत्यययोः} (पा॰सू॰~८.३.५९)~\arrow हन्~इ~ष्या~मि~\arrow हनिष्यामि।} इत्यर्थे \textcolor{red}{वधिष्यामि} इति प्रयोगं करोति। \textcolor{red}{वध्‌}\-धातुश्चुरादिः।\footnote{\textcolor{red}{बहुलमेतन्निदर्शनम्} (धा॰पा॰ ग॰सू॰~१९३८) \textcolor{red}{आकृतिगणोऽयम्} (धा॰पा॰ ग॰सू॰~१९९२) \textcolor{red}{भूवादिष्वेतदन्तेषु दशगणीषु धातूनां पाठो निदर्शनाय तेन स्तम्भुप्रभृतयः सौत्राश्चुलुम्पादयो वाक्यकारीयाः प्रयोगसिद्धा विक्लवत्यादयश्च} (मा॰धा॰वृ॰~१०.३२८) इत्यनुसारमाकृति\-गणत्वाच्चुरादि\-गण ऊह्योऽयमाधृषीयो धातुर्हिंसायाम्।} ततो णिजभावे\footnote{\textcolor{red}{आ धृषाद्वा} (धा॰पा॰ ग॰सू॰~१८०६) इत्यनेन वैकल्पिको णिच्। तेनात्र णिजभावः।} \textcolor{red}{वधिष्यामि} इति पाणिनीयः।\footnote{वध्~\arrow \textcolor{red}{आ धृषाद्वा} (धा॰पा॰ ग॰सू॰~१८०६)~\arrow णिजभावः~\arrow \textcolor{red}{शेषात्कर्तरि परस्मैपदम्} (पा॰सू॰~१.३.७८)~\arrow \textcolor{red}{लृट् शेषे च} (पा॰सू॰~३.३.१३)~\arrow वध्~लृट्~\arrow वध्~मिप्~\arrow वध्~मि~\arrow \textcolor{red}{स्यतासी लृलुटोः} (पा॰सू॰~३.१.३३)~\arrow वध्~स्य~मि~\arrow \textcolor{red}{आर्धधातुकस्येड्वलादेः} (पा॰सू॰~७.२.३५)~\arrow वध्~इट्~स्य~मि~\arrow वध्~इ~स्य~मि~\arrow \textcolor{red}{अतो दीर्घो यञि} (पा॰सू॰~७.३.१०१)~\arrow वध्~इ~स्या~मि~\arrow \textcolor{red}{आदेश\-प्रत्यययोः} (पा॰सू॰~८.३.५९)~\arrow वध्~इ~ष्या~मि~\arrow वधिष्यामि।} यद्वा \textcolor{red}{वधमाचरतीति वधति}\footnote{वध~\arrow \textcolor{red}{सर्वप्राति\-पदिकेभ्य आचारे क्विब्वा वक्तव्यः} (वा॰~३.१.११)~\arrow वध~क्विँप्~\arrow वध~व्~\arrow \textcolor{red}{वेरपृक्तस्य} (पा॰सू॰~६.१.६७)~\arrow वध~\arrow \textcolor{red}{सनाद्यन्ता धातवः} (पा॰सू॰~३.१.३२)~\arrow धातुसञ्ज्ञा~\arrow \textcolor{red}{शेषात्कर्तरि परस्मैपदम्} (पा॰सू॰~१.३.७८)~\arrow \textcolor{red}{वर्तमाने लट्} (पा॰सू॰~३.२.१२३)~\arrow वध~लट्~\arrow वध~तिप्~\arrow वध~ति~\arrow \textcolor{red}{कर्तरि शप्‌} (पा॰सू॰~३.१.६८)~\arrow वध~शप्~ति~\arrow वध~अ~ति~\arrow \textcolor{red}{अतो गुणे} (पा॰सू॰~६.१.९७)~\arrow वध~ति~\arrow वधति।} इति विग्रह आचार\-क्विबन्ताल्लृड्लकारे \textcolor{red}{वधिष्यामि}।\footnote{वध~\arrow धातुसञ्ज्ञा (पूर्ववत्)~\arrow \textcolor{red}{शेषात्कर्तरि परस्मैपदम्} (पा॰सू॰~१.३.७८)~\arrow \textcolor{red}{लृट् शेषे च} (पा॰सू॰~३.३.१३)~\arrow वध~लृट्~\arrow वध~मिप्~\arrow वध~मि~\arrow \textcolor{red}{स्यतासी लृलुटोः} (पा॰सू॰~३.१.३३)~\arrow वध~स्य~मि~\arrow \textcolor{red}{आर्धधातुकस्येड्वलादेः} (पा॰सू॰~७.२.३५)~\arrow वध~इट्~स्य~मि~\arrow वध~इ~स्य~मि~\arrow \textcolor{red}{अतो लोपः} (पा॰सू॰~६.४.४८)~\arrow वध्~इ~स्य~मि~\arrow \textcolor{red}{अतो दीर्घो यञि} (पा॰सू॰~७.३.१०१)~\arrow वध्~इ~स्या~मि~\arrow \textcolor{red}{आदेश\-प्रत्यययोः} (पा॰सू॰~८.३.५९)~\arrow वध्~इ~ष्या~मि~\arrow वधिष्यामि।} इति नापाणिनीयः।\footnote{\pageref{sec:vadhayisyati}तमे पृष्ठे \ref{sec:vadhayisyati} \nameref{sec:vadhayisyati} इति प्रयोगस्य विमर्शमपि पश्यन्तु।}\end{sloppypar}
\section[मरिष्ये]{मरिष्ये}
\centering\textcolor{blue}{वनं न गच्छेद्यदि रामचन्द्रः प्रभातकालेऽजिनचीरयुक्तः।\nopagebreak\\
उद्बन्धनं वा विषभक्षणं वा कृत्वा मरिष्ये पुरतस्तवाहम्॥}\nopagebreak\\
\raggedleft{–~अ॰रा॰~२.३.३१}\\
\fontsize{14}{21}\selectfont\begin{sloppypar}\hyphenrules{nohyphenation}\justifying\noindent\hspace{10mm} कैकेयी कथयति यद् \textcolor{red}{यदि राघवो वनं न गमिष्यति तदाऽहं मरिष्यामि}।\footnote{\textcolor{red}{मृङ् प्राणत्यागे} (धा॰पा॰~१४०३)~\arrow मृ~\arrow \textcolor{red}{शेषात्कर्तरि परस्मैपदम्} (पा॰सू॰~१.३.७८)~\arrow \textcolor{red}{लृट् शेषे च} (पा॰सू॰~३.३.१३)~\arrow मृ~लृट्~\arrow मृ~मिप्~\arrow मृ~मि~\arrow \textcolor{red}{स्यतासी लृलुटोः} (पा॰सू॰~३.१.३३)~\arrow मृ~स्य~मि~\arrow \textcolor{red}{ऋद्धनोः स्ये} (पा॰सू॰~७.२.७०)~\arrow \textcolor{red}{आद्यन्तौ टकितौ} (पा॰सू॰~१.१.४६)~\arrow मृ~इट्~स्य~मि~\arrow मृ~इ~स्य~मि~\arrow \textcolor{red}{सार्वधातुकार्ध\-धातुकयोः} (पा॰सू॰~७.३.८४)~\arrow म~इ~स्य~मि~\arrow \textcolor{red}{उरण् रपरः} (पा॰सू॰~१.१.५१)~\arrow मर्~इ~स्य~मि~\arrow \textcolor{red}{अतो दीर्घो यञि} (पा॰सू॰~७.३.१०१)~\arrow मर्~इ~स्या~मि~\arrow \textcolor{red}{आदेश\-प्रत्यययोः} (पा॰सू॰~८.३.५९)~\arrow मर्~इ~ष्या~मि~\arrow मरिष्यामि।} अत्र म्रियतेर्लुङ्लिङोः शिति चैवाऽत्मनेपद\-विधानात् \textcolor{red}{मरिष्ये} इति कथम्। \textcolor{red}{म्रियतेर्लुङ्लिङोश्च} (पा॰सू॰~१.३.६१) इति हि सूत्रं लृटि नैवात्मनेपदं करोतीति चेत्। \textcolor{red}{मरणं मरः}।\footnote{\textcolor{red}{कृत्यल्युटो बहुलम्} (पा॰सू॰~३.३.११३) इत्यनेन \textcolor{red}{मृ}\-धातोः \textcolor{red}{अप्‌}\-प्रत्ययः। \textcolor{red}{मरणं मरः। ‘कृत्यल्युटो बहुलम्’ (पा॰सू॰~३.३.११३) इत्यप्। तेनायमपि पूर्ववदाद्युदात्तः} (का॰वि॰प॰~६.२.११४) इति जिनेन्द्र\-बुद्धिः। यद्वा \textcolor{red}{नञो जर\-मर\-मित्र\-मृताः} (पा॰सू॰~६.२.११६) इत्यत्र निपातनात्सिद्धम्। \textcolor{red}{जरणं जरः ‘ॠदोरप्’ (पा॰सू॰~३.३.५७)। मरणं मरः। अमरम्। अस्मादेव निपातनादप्} (त॰बो॰~३८५०, ६.२.११६) इति ज्ञानेन्द्र\-सरस्वती च।} \textcolor{red}{मरमाचरतीति मरते} आचार\-क्विबन्तात्कर्म\-व्यतिहार आत्मनेपदम्।\footnote{मर~\arrow \textcolor{red}{सर्वप्राति\-पदिकेभ्य आचारे क्विब्वा वक्तव्यः} (वा॰~३.१.११)~\arrow मर~क्विँप्~\arrow मर~व्~\arrow \textcolor{red}{वेरपृक्तस्य} (पा॰सू॰~६.१.६७)~\arrow मर~\arrow \textcolor{red}{सनाद्यन्ता धातवः} (पा॰सू॰~३.१.३२)~\arrow धातुसञ्ज्ञा~\arrow \textcolor{red}{कर्तरि कर्मव्यतिहारे} (पा॰सू॰~१.३.१४)~\arrow \textcolor{red}{वर्तमाने लट्} (पा॰सू॰~३.२.१२३)~\arrow मर~लट्~\arrow मर~त~\arrow \textcolor{red}{कर्तरि शप्‌} (पा॰सू॰~३.१.६८)~\arrow मर~शप्~त~\arrow मर~अ~त~\arrow \textcolor{red}{अतो गुणे} (पा॰सू॰~६.१.९७)~\arrow मर~त~\arrow \textcolor{red}{टित आत्मनेपदानां टेरे} (पा॰सू॰~३.४.७९)~\arrow मर~ते~\arrow मरते।} लृड्लकार उत्तम\-पुरुष एक\-वचने \textcolor{red}{मरिष्ये}।\footnote{मर~\arrow धातुसञ्ज्ञा (पूर्ववत्)~\arrow \textcolor{red}{कर्तरि कर्मव्यतिहारे} (पा॰सू॰~१.३.१४)~\arrow \textcolor{red}{लृट् शेषे च} (पा॰सू॰~३.३.१३)~\arrow मर~लृट्~\arrow मर~इट्~\arrow मर~इ~\arrow \textcolor{red}{स्यतासी लृलुटोः} (पा॰सू॰~३.१.३३)~\arrow मर~स्य~इ~\arrow \textcolor{red}{आर्धधातुकस्येड्वलादेः} (पा॰सू॰~७.२.३५)~\arrow मर~इट्~स्य~इ~\arrow मर~इ~स्य~इ~\arrow \textcolor{red}{अतो लोपः} (पा॰सू॰~६.४.४८)~\arrow मर्~इ~स्य~इ~\arrow \textcolor{red}{टित आत्मनेपदानां टेरे} (पा॰सू॰~३.४.७९)~\arrow मर्~इ~स्य~ए~\arrow \textcolor{red}{अतो गुणे} (पा॰सू॰~६.१.९७)~\arrow मर्~इ~स्ये~\arrow \textcolor{red}{आदेश\-प्रत्यययोः} (पा॰सू॰~८.३.५९)~\arrow मर्~इ~ष्ये~\arrow मरिष्ये।} यद्वा \textcolor{red}{मरं मरणं गच्छतीति मरिष्यति}\footnote{\textcolor{red}{इषँ गतौ} (धा॰पा॰~११२७)~\arrow इष्~\arrow \textcolor{red}{शेषात्कर्तरि परस्मैपदम्} (पा॰सू॰~१.३.७८)~\arrow \textcolor{red}{वर्तमान\-सामीप्ये वर्तमानवद्वा} (पा॰सू॰~३.३.१३१)~\arrow \textcolor{red}{वर्तमाने लट्} (पा॰सू॰~३.२.१२३)~\arrow इष्~लट्~\arrow इष्~तिप्~\arrow इष्~ति~\arrow \textcolor{red}{दिवादिभ्यः श्यन्} (पा॰सू॰~३.१.६९)~\arrow इष्~शयन्~ति~\arrow इष्~य~ति~\arrow इष्यति। \textcolor{red}{मरम् इष्यति} इति स्थिते \textcolor{red}{शकन्ध्वादिषु पर\-रूपं वाच्यम्} (वा. ६.१.९१) इत्यनेन पररूपे \textcolor{red}{मरिष्यति}।} तस्यैव कर्मव्यतिहार आत्मनेपदत्वात्\footnote{\textcolor{red}{कर्तरि कर्म\-व्यतिहारे} (पा॰सू॰~१.३.१४) इत्यनेन।} \textcolor{red}{मरिष्ये} इति लड्लकार एवोत्तम\-पुरुष एक\-वचने।\footnote{\textcolor{red}{इषँ गतौ} (धा॰पा॰~११२७)~\arrow इष्~\arrow \textcolor{red}{कर्तरि कर्मव्यतिहारे} (पा॰सू॰~१.३.१४)~\arrow \textcolor{red}{वर्तमान\-सामीप्ये वर्तमानवद्वा} (पा॰सू॰~३.३.१३१)~\arrow \textcolor{red}{वर्तमाने लट्} (पा॰सू॰~३.२.१२३)~\arrow इष्~लट्~\arrow इष्~इट्~\arrow इष्~इ~\arrow \textcolor{red}{दिवादिभ्यः श्यन्} (पा॰सू॰~३.१.६९)~\arrow इष्~शयन्~इ~\arrow इष्~य~इ~\arrow \textcolor{red}{टित आत्मनेपदानां टेरे} (पा॰सू॰~३.४.७९)~\arrow इष्~य~ए~\arrow \textcolor{red}{अतो गुणे} (पा॰सू॰~६.१.९७)~\arrow इष्~ये~\arrow इष्ये। \textcolor{red}{मरम् इष्ये} इति स्थिते \textcolor{red}{शकन्ध्वादिषु पर\-रूपं वाच्यम्} (वा. ६.१.९१) इत्यनेन पररूपे \textcolor{red}{मरिष्ये}।}\end{sloppypar}
\section[द्रक्ष्यामहे]{द्रक्ष्यामहे}
\centering\textcolor{blue}{स्त्रियो बालाश्च वृद्धाश्च रात्रौ निद्रां न लेभिरे।\nopagebreak\\
कदा द्रक्ष्यामहे रामं पीतकौशेयवाससम्॥}\nopagebreak\\
\raggedleft{–~अ॰रा॰~२.३.३८}\\
\fontsize{14}{21}\selectfont\begin{sloppypar}\hyphenrules{nohyphenation}\justifying\noindent\hspace{10mm} अत्र कर्म\-व्यतिहार आत्मनेपदम्।\footnote{\textcolor{red}{कर्तरि कर्मव्यतिहारे} (पा॰सू॰~१.३.१४) इत्यनेन।} ततो लृड्लकारे \textcolor{red}{महिङ्} प्रत्यये दीर्घ एत्वे \textcolor{red}{द्रक्ष्यामहे}।\footnote{\textcolor{red}{दृशिँर प्रेक्षणे} (धा॰पा॰~९८८)~\arrow दृश्~\arrow \textcolor{red}{कर्तरि कर्मव्यतिहारे} (पा॰सू॰~१.३.१४)~\arrow \textcolor{red}{लृट् शेषे च} (पा॰सू॰~३.३.१३)~\arrow दृश्~लृट्~\arrow दृश्~महिङ्~\arrow दृश्~महि~\arrow \textcolor{red}{स्यतासी लृलुटोः} (पा॰सू॰~३.१.३३)~\arrow दृश्~स्य~महि~\arrow \textcolor{red}{सृजि\-दृशोर्झल्यमकिति} (पा॰सू॰~६.१.५८)~\arrow \textcolor{red}{मिदचोऽन्त्यात्परः} (पा॰सू॰~१.१.४७)~\arrow दृ~अम्~श्~स्य~महि~\arrow दृ~अ~श्~स्य~महि~\arrow \textcolor{red}{इको यणचि} (पा॰सू॰~६.१.७७)~\arrow द्र्~अ~श्~स्य~महि~\arrow \textcolor{red}{व्रश्चभ्रस्ज\-सृजमृज\-यजराज\-भ्राजच्छशां षः} (पा॰सू॰~८.२.३६)~\arrow द्र्~अ~ष्~स्य~महि~\arrow \textcolor{red}{षढोः कः सि} (पा॰सू॰~८.२.४१)~\arrow द्र्~अ~क्~स्य~महि~\arrow \textcolor{red}{अतो दीर्घो यञि} (पा॰सू॰~७.३.१०१)~\arrow द्र्~अ~क्~स्या~महि~\arrow \textcolor{red}{टित आत्मनेपदानां टेरे} (पा॰सू॰~३.४.७९)~\arrow द्र्~अ~क्~स्या~महे~\arrow \textcolor{red}{आदेश\-प्रत्यययोः} (पा॰सू॰~८.३.५९)~\arrow द्र्~अ~क्~ष्या~महे~\arrow द्रक्ष्यामहे।}\end{sloppypar}
\section[समपृच्छत]{समपृच्छत}
\centering\textcolor{blue}{वर्धयन् जयशब्देन प्रणमञ्छिरसा नृपम्।\nopagebreak\\
अतिखिन्नं नृपं दृष्ट्वा कैकेयीं समपृच्छत॥}\nopagebreak\\
\raggedleft{–~अ॰रा॰~२.३.४३}\\
\fontsize{14}{21}\selectfont\begin{sloppypar}\hyphenrules{nohyphenation}\justifying\noindent\hspace{10mm} अत्र कर्म\-व्यतिहार आत्मनेपदम्।\footnote{\textcolor{red}{कर्तरि कर्मव्यतिहारे} (पा॰सू॰~१.३.१४) इत्यनेन। सम्~\textcolor{red}{प्रच्छँ ज्ञीप्सायाम्} (धा॰पा॰~१४१३)~\arrow सम्~प्रच्छ्~\arrow \textcolor{red}{कर्तरि कर्मव्यतिहारे} (पा॰सू॰~१.३.१४)~\arrow \textcolor{red}{अनद्यतने लङ्} (पा॰सू॰~३.२.१११)~\arrow सम्~प्रच्छ्~लङ्~\arrow सम्~प्रच्छ्~त~\arrow \textcolor{red}{तुदादिभ्यः शः} (पा॰सू॰~३.१.७७)~\arrow सम्~प्रच्छ्~श~त~\arrow प्रच्छ्~अ~थास्~\arrow \textcolor{red}{सार्वधातुकमपित्} (पा॰सू॰~१.२.४)~\arrow ङिद्वत्त्वम्~\arrow \textcolor{red}{ग्रहिज्या\-वयिव्यधि\-वष्टिविचति\-वृश्चति\-पृच्छति\-भृज्जतीनां ङिति च} (पा॰सू॰~६.१.१६)~\arrow सम्~पृ~अ~च्छ्~अ~त~\arrow \textcolor{red}{सम्प्रसारणाच्च} (पा॰सू॰~६.१.१०८)~\arrow सम्~पृ~च्छ्~अ~त~\arrow सम्~पृ~च्छ्~अ~त~\arrow समपृच्छत। \textcolor{red}{विदि\-प्रच्छि\-स्वरतीनामुपसङ्ख्यानम्} इत्युप\-सङ्ख्यानस्याकर्मकाधिकारादत्राप्रवृत्तिः। यद्वा \textcolor{red}{अतिखिन्नं नृपं कैकेयीं च दृष्ट्वा} इत्यन्वयेन कर्मणोऽविवक्षायामकर्मकत्वादात्मने\-पदम्। अनेनैवोप\-सङ्ख्यानेन।}\end{sloppypar}
\section[शासतु]{शासतु}
\centering\textcolor{blue}{आश्वासयामास नृपं शनैः स नयकोविदः।\nopagebreak\\
किमत्र दुःखेन विभो राज्यं शासतु मेऽनुजः॥}\nopagebreak\\
\raggedleft{–~अ॰रा॰~२.३.७३}\\
\fontsize{14}{21}\selectfont\begin{sloppypar}\hyphenrules{nohyphenation}\justifying\noindent\hspace{10mm} अत्र \textcolor{red}{गण\-कार्यमनित्यम्} (प॰शे॰~९३.३) इति नियमाल्लोड्लकारे तिपि शपि \textcolor{red}{एरुः} (पा॰सू॰~३.४.८६) इत्यनेनोकारे \textcolor{red}{शासतु}।\footnote{\textcolor{red}{शासुँ अनुशिष्टौ} (धा॰पा॰~१०७४) इति धातोर्लोड्लकारे प्रथमपुरुष एकवचने \textcolor{red}{शास्तु} इति रूपम्। यथा \textcolor{red}{त्वया परिगृहीतोऽयमङ्गदः शास्तु मेदिनीम्} (वा॰रा॰~४.२१.९) इति वाल्मीकिप्रयोगे। शासुँ~\arrow शास्~\arrow \textcolor{red}{शेषात्कर्तरि परस्मैपदम्} (पा॰सू॰~१.३.७८)~\arrow \textcolor{red}{लोट् च} (पा॰सू॰~३.३.१६२)~\arrow शास्~लोट्~\arrow शास्~तिप्~\arrow शास्~ति~\arrow \textcolor{red}{कर्तरि शप्‌} (पा॰सू॰~३.१.६८)~\arrow शास्~शप्~ति~\arrow \textcolor{red}{अदिप्रभृतिभ्यः शपः} (पा॰सू॰~२.४.७२)~\arrow शब्लुक्~\arrow शास्~ति~\arrow \textcolor{red}{एरुः} (पा॰सू॰~३.४.८६)~\arrow शास्~तु~\arrow शास्तु। \textcolor{red}{गण\-कार्यमनित्यम्} (प॰शे॰~९३.३) इति परिभाषयाऽत्र \textcolor{red}{अदिप्रभृतिभ्यः शपः} (पा॰सू॰~२.४.७२) इत्यस्याप्रवृत्तौ शब्लुगभावे शास्~शप्~ति~\arrow शास्~अ~ति~\arrow \textcolor{red}{एरुः} (पा॰सू॰~३.४.८६)~\arrow शास्~अ~तु~\arrow \textcolor{red}{शासतु} इति सिद्धम्।}\end{sloppypar}
\section[यास्ये]{यास्ये}
\label{sec:yasye}
\centering\textcolor{blue}{मातरं समनुश्वास्य अनुनीय च जानकीम्।\nopagebreak\\
आगत्य पादौ वन्दित्वा तव यास्ये सुखं वनम्॥}\nopagebreak\\
\raggedleft{–~अ॰रा॰~२.३.७७}\\
\fontsize{14}{21}\selectfont\begin{sloppypar}\hyphenrules{nohyphenation}\justifying\noindent\hspace{10mm} श्रीरामः कैकेयीं प्रति कथयति यत् \textcolor{red}{अहं तव पादं वन्दित्वा वनं यास्यामि}। \textcolor{red}{यास्यामि}\footnote{\textcolor{red}{या प्रापणे} (धा॰पा॰~१०४९)~\arrow \textcolor{red}{शेषात्कर्तरि परस्मैपदम्} (पा॰सू॰~१.३.७८)~\arrow \textcolor{red}{लृट् शेषे च} (पा॰सू॰~३.३.१३)~\arrow या~लृट्~\arrow या~मिप्~\arrow या~मि~\arrow \textcolor{red}{स्यतासी लृलुटोः} (पा॰सू॰~३.१.३३)~\arrow या~स्य~मि~\arrow \textcolor{red}{अतो दीर्घो यञि} (पा॰सू॰~७.३.१०१)~\arrow या~स्या~मि~\arrow यास्यामि।} इति प्रयोक्तव्ये \textcolor{red}{यास्ये} इति प्रयुज्यते। अत्रापि कर्म\-व्यतिहारः।\footnote{तस्मात् \textcolor{red}{कर्तरि कर्मव्यतिहारे} (पा॰सू॰~१.३.१४) इत्यनेनाऽत्मनेपदमिति भावः। \textcolor{red}{या प्रापणे} (धा॰पा॰~१०४९)~\arrow \textcolor{red}{कर्तरि कर्मव्यतिहारे} (पा॰सू॰~१.३.१४)~\arrow \textcolor{red}{लृट् शेषे च} (पा॰सू॰~३.३.१३)~\arrow या~लृट्~\arrow या~इट्~\arrow या~इ~\arrow \textcolor{red}{स्यतासी लृलुटोः} (पा॰सू॰~३.१.३३)~\arrow या~स्य~इ~\arrow \textcolor{red}{टित आत्मनेपदानां टेरे} (पा॰सू॰~३.४.७९)~\arrow या~स्य~ए~\arrow \textcolor{red}{अतो गुणे} (पा॰सू॰~६.१.९७)~\arrow या~स्ये~\arrow यास्ये।} यतो हि वन\-गमनं तु वृद्धानां कृते। तथा चोक्तम्~–\end{sloppypar}
\centering\textcolor{red}{शैशवेऽभ्यस्तविद्यानां यौवने विषयैषिणाम्।\nopagebreak\\
वार्धके मुनिवृत्तीनां योगेनान्ते तनुत्यजाम्॥}\nopagebreak\\
\raggedleft{–~र॰वं॰~१.८}\\
\fontsize{14}{21}\selectfont\begin{sloppypar}\hyphenrules{nohyphenation}\justifying\noindent
 अतो वृद्धानां कर्म वन\-गमनं युवको भूत्वा श्रीराम आचरति तस्मादत्र \textcolor{red}{यास्ये} इत्येव प्रयोगः साधु। श्रीरामचरितमानसे कौसल्याऽपि कथयति~–\end{sloppypar}
\centering\textcolor{red}{अंतहुँ उचित नृपहि बनबासू। बय बिलोकि हिय होइ हरासू॥}\footnote{एतद्रूपान्तरम्–\textcolor{red}{अन्ते निवासो विपिने समीचीनो महीपतेः। किन्तु दृष्ट्वा तवावस्थां दुःखं मे मनसि स्थितम्॥} (मा॰भा॰~२.५६.४)।}\nopagebreak\\
\raggedleft{–~रा॰च॰मा॰~२.५६.४}\\
\fontsize{14}{21}\selectfont\begin{sloppypar}\hyphenrules{nohyphenation}\justifying\noindent तस्मात् \textcolor{red}{यास्ये}। वृद्धोचितमपि गमनं त्वत्सन्तोषार्थं करिष्यामीत्येव राजीव\-लोचनस्य विवक्षितोऽर्थ इट्प्रत्ययेन ध्वन्यते। न च \textcolor{red}{न गति\-हिंसार्थेभ्यः} (पा॰सू॰~१.३.१५) इत्यनेन कर्मव्यतिहार आत्मनेपद\-निषेधः। मण्डूक\-प्लुत्या \textcolor{red}{नपुंसकमनपुंसकेनैक\-वच्चास्यान्यतरस्याम्} (पा॰सू॰~१.२.६९) इत्यतः \textcolor{red}{अन्यतरस्याम्} इत्यनुवर्तनीयम्।\end{sloppypar}
\section[कुरुते]{कुरुते}
\centering\textcolor{blue}{इत्युक्त्वा तु परिक्रम्य मातरं द्रष्टुमाययौ।\nopagebreak\\
कौसल्याऽपि हरेः पूजां कुरुते रामकारणात्॥}\nopagebreak\\
\raggedleft{–~अ॰रा॰~२.३.७८}\\
\fontsize{14}{21}\selectfont\begin{sloppypar}\hyphenrules{nohyphenation}\justifying\noindent\hspace{10mm} अत्र \textcolor{red}{अकुरुत}\footnote{\textcolor{red}{डुकृञ् करणे} (धा॰पा॰~१४७२)~\arrow कृ~\arrow \textcolor{red}{स्वरितञितः कर्त्रभिप्राये क्रियाफले} (पा॰सू॰~१.३.७२)~\arrow \textcolor{red}{अनद्यतने लङ्} (पा॰सू॰~३.२.१११)~\arrow कृ~लङ्~\arrow कृ~त~\arrow \textcolor{red}{लुङ्लङ्लृङ्क्ष्वडुदात्तः} (पा॰सू॰~६.४.७१)~\arrow \textcolor{red}{आद्यन्तौ टकितौ} (पा॰सू॰~१.१.४६)~\arrow अट्~कृ~त~\arrow अ~कृ~त~\arrow \textcolor{red}{तनादि\-कृञ्भ्य उः} (पा॰सू॰~३.१.७९)~\arrow अ~कृ~उ~त~\arrow \textcolor{red}{सार्वधातुकार्ध\-धातुकयोः} (पा॰सू॰~७.३.८४)~\arrow \textcolor{red}{उरण् रपरः} (पा॰सू॰~१.१.५१)~\arrow अ~क्~अर्~उ~त~\arrow \textcolor{red}{अत उत्सार्वधातुके} (पा॰सू॰~६.४.११०)~\arrow अ~क्~उर्~उ~त~\arrow अकुरुत।} इति प्रयोक्तव्यम्। \textcolor{red}{कुरुते} इति प्रयुक्तम्। \textcolor{red}{स्म}\-योगे \textcolor{red}{लट् स्मे} (पा॰सू॰~३.२.११८) इत्यनेन लड्लकार\-विधानात्सम्यक्।\footnote{\textcolor{red}{स्म} इत्यध्याहार्यमिति भावः। \textcolor{red}{डुकृञ् करणे} (धा॰पा॰~१४७२)~\arrow कृ~\arrow \textcolor{red}{स्वरितञितः कर्त्रभिप्राये क्रियाफले} (पा॰सू॰~१.३.७२)~\arrow \textcolor{red}{लट् स्मे} (पा॰सू॰~३.२.११८)~\arrow कृ~लट्~\arrow कृ~त~\arrow \textcolor{red}{तनादि\-कृञ्भ्य उः} (पा॰सू॰~३.१.७९)~\arrow कृ~उ~त~\arrow \textcolor{red}{सार्वधातुकार्ध\-धातुकयोः} (पा॰सू॰~७.३.८४)~\arrow \textcolor{red}{उरण् रपरः} (पा॰सू॰~१.१.५१)~\arrow क्~अर्~उ~त~\arrow \textcolor{red}{अत उत्सार्वधातुके} (पा॰सू॰~६.४.११०)~\arrow क्~उर्~उ~त~\arrow \textcolor{red}{टित आत्मनेपदानां टेरे} (पा॰सू॰~३.४.७९)~\arrow क्~उर्~उ~ते~\arrow कुरुते।}\end{sloppypar}
\section[ध्यायते]{ध्यायते}
\centering\textcolor{blue}{होमं च कारयामास ब्राह्मणेभ्यो ददौ धनम्।\nopagebreak\\
ध्यायते विष्णुमेकाग्रमनसा मौनमास्थिता॥}\nopagebreak\\
\raggedleft{–~अ॰रा॰~२.३.७९}\\
\fontsize{14}{21}\selectfont\begin{sloppypar}\hyphenrules{nohyphenation}\justifying\noindent\hspace{10mm} अत्रापि कर्म\-व्यतिहारादेवाऽत्मने\-पदम्।\footnote{\textcolor{red}{कर्तरि कर्मव्यतिहारे} (पा॰सू॰~१.३.१४) इत्यनेन।} यतो हि महाविष्णुस्तु तस्याः पुत्रः। अंशिनि महाविष्णौ श्रीरामभद्रे पुत्र\-रूपेण वर्तमानेऽप्यंशं विष्णुं वामनावतारे तन्माताऽदितिः सत्यपि प्राकृत\-नारीव ध्यायतीत्येव कर्म\-व्यतिहारस्य सङ्गतिः। वर्तमान\-कालोऽपि पूर्वोक्त\-दिशा।\footnote{\textcolor{red}{लट् स्मे} (पा॰सू॰~३.२.११८) इत्यनेन। \textcolor{red}{स्म} इत्यध्याहार्यमिति भावः। \textcolor{red}{ध्यै चिन्तायाम्} (धा॰पा॰~९०८)~\arrow ध्यै~\arrow \textcolor{red}{कर्तरि कर्मव्यतिहारे} (पा॰सू॰~१.३.१४)~\arrow \textcolor{red}{लट् स्मे} (पा॰सू॰~३.२.११८)~\arrow ध्यै~लट्~\arrow ध्यै~त~\arrow \textcolor{red}{कर्तरि शप्‌} (पा॰सू॰~३.१.६८)~\arrow ध्यै~शप्~त~\arrow ध्यै~अ~त~\arrow \textcolor{red}{एचोऽयवायावः} (पा॰सू॰~६.१.७८)~\arrow ध्याय्~अ~त~\arrow \textcolor{red}{टित आत्मनेपदानां टेरे} (पा॰सू॰~३.४.७९)~\arrow ध्याय्~अ~ते~\arrow ध्यायते।}\end{sloppypar}
\section[आगमिष्ये]{आगमिष्ये}
\centering\textcolor{blue}{चतुर्दश समास्तत्र ह्युषित्वा मुनिवेषधृक्।\nopagebreak\\
आगमिष्ये पुनः शीघ्रं न चिन्तां कर्तुमर्हसि॥}\nopagebreak\\
\raggedleft{–~अ॰रा॰~२.४.६}\\
\fontsize{14}{21}\selectfont\begin{sloppypar}\hyphenrules{nohyphenation}\justifying\noindent\hspace{10mm} श्रीरामः कथयति वनवासान्ते \textcolor{red}{आगमिष्ये}। अत्र \textcolor{red}{आगमिष्यामि} इति हि पाणिनीयम्।\footnote{आङ् \textcolor{red}{गमॢँ गतौ} (धा॰पा॰~९८२)~\arrow आ~गम्~\arrow \textcolor{red}{शेषात्कर्तरि परस्मैपदम्} (पा॰सू॰~१.३.७८)~\arrow \textcolor{red}{लृट् शेषे च} (पा॰सू॰~३.३.१३)~\arrow आ~गम्~लृँट्~\arrow आ~गम्~मिप्~\arrow आ~गम्~मि~\arrow \textcolor{red}{स्यतासी लृलुटोः} (पा॰सू॰~३.१.३३)~\arrow आ~गम्~स्य~मि~\arrow \textcolor{red}{गमेरिट् परस्मैपदेषु} (पा॰सू॰~७.२.५८)~\arrow आ~गम्~इट्~स्य~मि~\arrow आ~गम्~इ~स्य~मि~\arrow \textcolor{red}{अतो दीर्घो यञि} (पा॰सू॰~७.३.१०१)~\arrow आ~गम्~इ~स्या~मि~\arrow \textcolor{red}{आदेशप्रत्यययोः} (पा॰सू॰~८.३.५९)~\arrow आ~गम्~इ~ष्या~ति~\arrow आगमिष्यामि।} समुपसर्गाध्याहार आत्मनेपदम् \textcolor{red}{समो गम्यृच्छिभ्याम्} (पा॰सू॰~१.३.२९) इत्यनेन। न च \textcolor{red}{आङ्‌}\-उपसर्गस्यैव \textcolor{red}{गम्‌}\-धातुनाऽन्वय इति चेत्। \textcolor{red}{व्यवहिते} इति वक्तव्यम्।
न च सत्यप्यात्मनेपद इट् कथं \textcolor{red}{गमेरिट् परस्मैपदेषु} (पा॰सू॰~७.२.५८) इति सूत्रेण परस्मैपद एवेड्विधानादिति चेत्सत्यम्। उच्यते। \textcolor{red}{आगच्छतीत्यागम्} आचारे क्विप्ततश्च सर्वापहारि\-लोपे।\footnote{आङ् \textcolor{red}{गमॢँ गतौ} (धा॰पा॰~९८२)~\arrow \textcolor{red}{क्विप् च} (पा॰सू॰~३.२.७६)~\arrow आ~गम्~क्विँप्~\arrow आ~गम्~व्~\arrow \textcolor{red}{वेरपृक्तस्य} (पा॰सू॰~६.१.६७)~\arrow आ~गम्~\arrow आगम्~\arrow \textcolor{red}{कृत्तद्धित\-समासाश्च} (पा॰सू॰~१.२.४६)~\arrow प्रातिपदिक\-सञ्ज्ञा~\arrow विभक्तिकार्यम्~\arrow आगम्~सुँ~\arrow \textcolor{red}{हल्ङ्याब्भ्यो दीर्घात्सुतिस्यपृक्तं हल्} (पा॰सू॰~६.१.६८)~\arrow आगम्।} \textcolor{red}{इष्‌}\-धातोः कर्म\-व्यतिहार आत्मनेपदे \textcolor{red}{इष्ये}।\footnote{\textcolor{red}{इषँ गतौ} (धा॰पा॰~११२७)~\arrow इष्~\arrow \textcolor{red}{कर्तरि कर्मव्यतिहारे} (पा॰सू॰~१.३.१४)~\arrow \textcolor{red}{वर्तमान\-सामीप्ये वर्तमानवद्वा} (पा॰सू॰~३.३.१३१)~\arrow \textcolor{red}{वर्तमाने लट्} (पा॰सू॰~३.२.१२३)~\arrow इष्~लट्~\arrow इष्~इट्~\arrow इष्~इ~\arrow \textcolor{red}{दिवादिभ्यः श्यन्} (पा॰सू॰~३.१.६९)~\arrow इष्~शयन्~इ~\arrow इष्~य~इ~\arrow \textcolor{red}{टित आत्मनेपदानां टेरे} (पा॰सू॰~३.४.७९)~\arrow इष्~य~ए~\arrow \textcolor{red}{अतो गुणे} (पा॰सू॰~६.१.९७)~\arrow इष्~ये~\arrow इष्ये।} \textcolor{red}{आगम् इष्ये} इति \textcolor{red}{आगमिष्ये}।\end{sloppypar}
\section[जायते]{जायते}
\centering\textcolor{blue}{यथा प्रवाहपतितप्लवानां सरितां तथा।\nopagebreak\\
चतुर्दशसमासङ्ख्या क्षणार्द्धमिव जायते॥}\nopagebreak\\
\raggedleft{–~अ॰रा॰~२.४.४६}\\
\fontsize{14}{21}\selectfont\begin{sloppypar}\hyphenrules{nohyphenation}\justifying\noindent\hspace{10mm} सामीप्याभिप्रायेणैव \textcolor{red}{क्षणार्द्धमिव} इति वाक्य\-खण्ड\-प्रयोगेण च शीघ्रतां द्योतयितुं वर्तमान\-समीपे लटं प्रयुङ्क्ते। \textcolor{red}{वर्तमान\-सामीप्ये वर्तमानवद्वा} (पा॰सू॰~३.३.१३१) इत्यनेन।\footnote{\textcolor{red}{जनीँ प्रादुर्भावे} (धा॰पा॰~११४९)~\arrow \textcolor{red}{अनुदात्तङित आत्मने\-पदम्} (पा॰सू॰~१.३.१२)~\arrow \textcolor{red}{वर्तमान\-सामीप्ये वर्तमानवद्वा} (पा॰सू॰~३.३.१३१)~\arrow \textcolor{red}{वर्तमाने लट्} (पा॰सू॰~३.२.१२३)~\arrow जन्~लट्~\arrow जन्~त~\arrow \textcolor{red}{दिवादिभ्यः श्यन्} (पा॰सू॰~३.१.६९)~\arrow जन्~श्यन्~त~\arrow जन्~य~त~\arrow \textcolor{red}{ज्ञाजनोर्जा} (पा॰सू॰~७.३.७९)~\arrow जा~य~त~\arrow \textcolor{red}{टित आत्मनेपदानां टेरे} (पा॰सू॰~३.४.७९)~\arrow जा~य~ते~\arrow जायते।}\end{sloppypar}
\section[नेष्ये]{नेष्ये}
\centering\textcolor{blue}{तामाह राघवः प्रीतः स्वप्रियां प्रियवादिनीम्।\nopagebreak\\
कथं वनं त्वां नेष्येऽहं बहुव्याघ्रमृगाकुलम्॥}\nopagebreak\\
\raggedleft{–~अ॰रा॰~२.४.६४}\\
\fontsize{14}{21}\selectfont\begin{sloppypar}\hyphenrules{nohyphenation}\justifying\noindent\hspace{10mm} \textcolor{red}{णीञ् प्रापणे} (धा॰पा॰~९०१) इति धातुरकर्त्रभिप्राये क्रियाफले परस्मैपदी। तथा च \textcolor{red}{नेष्यामि} इति प्रयोक्तव्यम्।\footnote{\textcolor{red}{णीञ् प्रापणे} (धा॰पा॰~९०१)~\arrow णी~\arrow \textcolor{red}{णो नः} (पा॰सू॰~६.१.६५)~\arrow नी~\arrow \textcolor{red}{शेषात्कर्तरि परस्मैपदम्} (पा॰सू॰~१.३.७८)~\arrow \textcolor{red}{लृट् शेषे च} (पा॰सू॰~३.३.१३)~\arrow नी~लृट्~\arrow नी~मिप्~\arrow नी~मि~\arrow \textcolor{red}{स्यतासी लृलुटोः} (पा॰सू॰~३.१.३३)~\arrow नी~स्य~मि~\arrow \textcolor{red}{सार्वधातुकार्ध\-धातुकयोः} (पा॰सू॰~७.३.८४)~\arrow ने~स्य~मि~\arrow \textcolor{red}{अतो दीर्घो यञि} (पा॰सू॰~७.३.१०१)~\arrow ने~स्या~मि~\arrow \textcolor{red}{आदेश\-प्रत्यययोः} (पा॰सू॰~८.३.५९)~\arrow ने~ष्या~मि~\arrow नेष्यामि।} अत्र \textcolor{red}{उपनेष्ये} इति हि पदम्।\footnote{\textcolor{red}{उप}उपसर्गस्य लोप इति भावः।} \textcolor{red}{कथमुपनेष्ये} इत्यर्थे सम्मानन आत्मनेपदम्।\footnote{उप~\textcolor{red}{णीञ् प्रापणे} (धा॰पा॰~९०१)~\arrow उप~णी~\arrow \textcolor{red}{णो नः} (पा॰सू॰~६.१.६५)~\arrow उप~नी~\arrow \textcolor{red}{सम्माननोत्सञ्जनाचार्य\-करण\-ज्ञान\-भृति\-विगणन\-व्ययेषु नियः} (पा॰सू॰~१.३.३६)~\arrow \textcolor{red}{लृट् शेषे च} (पा॰सू॰~३.३.१३)~\arrow उप~नी~लृट्~\arrow उप~नी~इट्~\arrow उप~नी~इ~\arrow \textcolor{red}{स्यतासी लृलुटोः} (पा॰सू॰~३.१.३३)~\arrow उप~नी~स्य~इ~\arrow \textcolor{red}{सार्वधातुकार्ध\-धातुकयोः} (पा॰सू॰~७.३.८४)~\arrow उप~ने~स्य~इ~\arrow \textcolor{red}{टित आत्मनेपदानां टेरे} (पा॰सू॰~३.४.७९)~\arrow उप~ने~स्य~ए~\arrow \textcolor{red}{अतो गुणे} (पा॰सू॰~६.१.९७)~\arrow उप~ने~स्ये~\arrow \textcolor{red}{आदेश\-प्रत्यययोः} (पा॰सू॰~८.३.५९)~\arrow उप~ने~ष्ये~\arrow \textcolor{red}{विनाऽपि प्रत्ययं पूर्वोत्तर\-पद\-लोपो वक्तव्यः} (वा॰~५.३.८३)~\arrow ने~ष्ये~\arrow नेष्ये।} \textcolor{red}{कथं सम्मानितं करिष्यामि} इति भावः। यद्वा व्यये। \textcolor{red}{कथं दिनानि यापयिष्यामि}। यद्वोत्सञ्जने। अत्र सूत्रं \textcolor{red}{सम्माननोत्सञ्जनाचार्य\-करण\-ज्ञान\-भृति\-विगणन\-व्ययेषु नियः} (पा॰सू॰~१.३.३६)।
\end{sloppypar}
\section[हास्यसे]{हास्यसे}
\centering\textcolor{blue}{पादचारेण गन्तव्यं शीतवातातपादिमत्।\nopagebreak\\
राक्षसादीन्वने दृष्ट्वा जीवितं हास्यसेऽचिरात्॥}\nopagebreak\\
\raggedleft{–~अ॰रा॰~२.४.६९}\\
\fontsize{14}{21}\selectfont\begin{sloppypar}\hyphenrules{nohyphenation}\justifying\noindent\hspace{10mm} अत्र \textcolor{red}{हास्यसि}\footnote{\textcolor{red}{ओँहाक् त्यागे} (धा॰पा॰~१०९०)~\arrow हा~\arrow \textcolor{red}{शेषात्कर्तरि परस्मैपदम्} (पा॰सू॰~१.३.७८)~\arrow \textcolor{red}{लृट् शेषे च} (पा॰सू॰~३.३.१३)~\arrow हा~लृट्~\arrow हा~सिप्~\arrow हा~सि~\arrow \textcolor{red}{स्यतासी लृलुटोः} (पा॰सू॰~३.१.३३)~\arrow हा~स्य~सि~\arrow हास्यसि।} इति प्रयोक्तव्यं किन्तु कर्म\-व्यतिहार आत्मनेपदम्।\footnote{\textcolor{red}{कर्तरि कर्मव्यतिहारे} (पा॰सू॰~१.३.१४) इत्यनेन। \textcolor{red}{ओँहाक् त्यागे} (धा॰पा॰~१०९०)~\arrow हा~\arrow \textcolor{red}{कर्तरि कर्मव्यतिहारे} (पा॰सू॰~१.३.१४)~\arrow \textcolor{red}{लृट् शेषे च} (पा॰सू॰~३.३.१३)~\arrow हा~लृट्~\arrow हा~थास्~\arrow \textcolor{red}{स्यतासी लृलुटोः} (पा॰सू॰~३.१.३३)~\arrow हा~स्य~थास्~\arrow \textcolor{red}{थासस्से} (पा॰सू॰~३.४.८०)~\arrow हा~स्य~से~\arrow हास्यसे।} यद्वा \textcolor{red}{जीवितम्} इति क्रिया\-विशेषणम् \textcolor{red}{उपेक्ष्य} इति वाऽध्याहार्यम्। एवं \textcolor{red}{जीवितमुपेक्ष्य त्वं प्राणैः हास्यसे} इति कर्म\-वाच्य आत्मनेपदम्।\footnote{\textcolor{red}{ओँहाक् त्यागे} (धा॰पा॰~१०९०) इति धातोः कर्मणि लृटि मध्यम\-पुरुष एकवचने \textcolor{red}{हास्यसे हायिष्यसे} इति रूपद्वयम्। अचिण्वद्भाव इडभावे \textcolor{red}{हास्यसे} इति रूपम्। हा~\arrow \textcolor{red}{भावकर्मणोः} (पा॰सू॰~१.३.१३)~\arrow \textcolor{red}{लृट् शेषे च} (पा॰सू॰~३.३.१३)~\arrow हा~लृट्~\arrow हा~थास्~\arrow \textcolor{red}{स्यतासी लृलुटोः} (पा॰सू॰~३.१.३३)~\arrow हा~स्य~थास्~\arrow \textcolor{red}{थासस्से} (पा॰सू॰~३.४.८०)~\arrow हा~स्य~से~\arrow हास्यसे। पक्षे चिण्वद्भाव इडागमे हायिष्यसे इति रूपम्। हा~\arrow \textcolor{red}{भावकर्मणोः} (पा॰सू॰~१.३.१३)~\arrow \textcolor{red}{लृट् शेषे च} (पा॰सू॰~३.३.१३)~\arrow हा~लृट्~\arrow हा~थास्~\arrow \textcolor{red}{स्यतासी लृलुटोः} (पा॰सू॰~३.१.३३)~\arrow हा~स्य~थास्~\arrow \textcolor{red}{स्यसिच्सीयुट्तासिषु भाव\-कर्मणोरुपदेशेऽज्झन\-ग्रहदृशां वा चिण्वदिट् च} (पा॰सू॰~६.४.६२)~\arrow हा~इट्~स्य~थास्~\arrow हा~इ~स्य~थास्~\arrow \textcolor{red}{आतो युक् चिण्कृतोः} (पा॰सू॰~७.३.३३)~\arrow हा~युँक्~इ~स्य~थास्~\arrow हा~य्~इ~स्य~थास्~\arrow \textcolor{red}{आदेश\-प्रत्यययोः} (पा॰सू॰~८.३.५९)~\arrow हा~य्~इ~ष्य~थास्~\arrow \textcolor{red}{थासस्से} (पा॰सू॰~३.४.८०)~\arrow हा~य्~इ~ष्य~से~\arrow हायिष्यसे।}\end{sloppypar}
\section[इच्छसे]{इच्छसे}
\centering\textcolor{blue}{प्रत्युवाच स्फुरद्वक्त्रा किञ्चित्कोपसमन्विता।\nopagebreak\\
कथं मामिच्छसे त्यक्तुं धर्मपत्नीं पतिव्रताम्॥}\nopagebreak\\
\raggedleft{–~अ॰रा॰~२.४.७१}\\
\fontsize{14}{21}\selectfont\begin{sloppypar}\hyphenrules{nohyphenation}\justifying\noindent\hspace{10mm} अत्रापि कर्म\-व्यतिहार आत्मनेपदम्।\footnote{\textcolor{red}{कर्तरि कर्मव्यतिहारे} (पा॰सू॰~१.३.१४) इत्यनेन। \textcolor{red}{इषँ इच्छायाम्} (धा॰पा॰~१३५१)~\arrow \textcolor{red}{कर्तरि कर्मव्यतिहारे} (पा॰सू॰~१.३.१४)~\arrow वर्तमाने लट्~\arrow इष्~लट्~\arrow इष्~थास्~\arrow \textcolor{red}{तुदादिभ्यः शः} (पा॰सू॰~३.१.७७)~\arrow इष्~श~थास्~\arrow इष्~अ~थास्~\arrow \textcolor{red}{इषुगमियमां छः} (पा॰सू॰~७.३.७७)~\arrow इछ्~अ~थास्~\arrow \textcolor{red}{छे च} (पा॰सू॰~६.१.७३)~\arrow \textcolor{red}{आद्यन्तौ टकितौ} (पा॰सू॰~१.१.४६)~\arrow इतुँक्~छ्~अ~थास्~\arrow इत्~छ्~अ~थास्~\arrow \textcolor{red}{स्तोः श्चुना श्चुः} (पा॰सू॰~८.४.४०)~\arrow इच्~छ्~अ~थास्~\arrow \textcolor{red}{थासस्से} (पा॰सू॰~३.४.८०)~\arrow इच्~छ्~अ~से~\arrow इच्छसे।}\end{sloppypar}
\section[रमामि]{रमामि}
\centering\textcolor{blue}{फलमूलादिकं यद्यत्तव भुक्तावशेषितम्।\nopagebreak\\
तदेवामृततुल्यं मे तेन तुष्टा रमाम्यहम्॥}\nopagebreak\\
\raggedleft{–~अ॰रा॰~२.४.७३}\\
\fontsize{14}{21}\selectfont\begin{sloppypar}\hyphenrules{nohyphenation}\justifying\noindent\hspace{10mm} अत्र \textcolor{red}{रमुँ क्रीडायाम्} (धा॰पा॰~८५३) इत्यात्मनेपदी धातुः। एवं भविष्यत्काले \textcolor{red}{रंस्ये} इति हि युक्तं पाणिनीयमपि।\footnote{\textcolor{red}{रमुँ क्रीडायाम्} (धा॰पा॰~८५३)~\arrow रम्~\arrow \textcolor{red}{अनुदात्तङित आत्मने\-पदम्} (पा॰सू॰~१.३.१२)~\arrow \textcolor{red}{लृट् शेषे च} (पा॰सू॰~३.३.१३)~\arrow रम्~लृट्~\arrow रम्~इट्~\arrow रम्~इ~\arrow \textcolor{red}{स्यतासी लृलुटोः} (पा॰सू॰~३.१.३३)~\arrow रम्~स्य~इ~\arrow \textcolor{red}{टित आत्मनेपदानां टेरे} (पा॰सू॰~३.४.७९)~\arrow रम्~स्य~ए~\arrow \textcolor{red}{अतो गुणे} (पा॰सू॰~६.१.९७)~\arrow रम्~स्ये~\arrow \textcolor{red}{मोऽनुस्वारः} (पा॰सू॰~८.३.२३)~\arrow रं~स्ये~\arrow रंस्ये।} अत्र \textcolor{red}{रमामि} इत्यत्र द्वावंशावयुक्ताविव। परस्मैपद\-प्रयोगः काल\-व्यत्ययश्च। कालव्यत्यये तु भविष्यत्कालेऽपि \textcolor{red}{लट् स्मे} (पा॰सू॰~३.२.११८) इत्यनेन \textcolor{red}{स्म}\-शब्द\-योगे वर्तमानवत्कार्यम्।\footnote{\textcolor{red}{स्म} इत्यध्याहार्यमिति भावः।} \textcolor{red}{आरमामि} इत्यत्र \textcolor{red}{व्याङ्परिभ्यो रमः} (पा॰सू॰~१.३.८३) इत्यनेन परस्मैपदम्।\footnote{आङ्~रम्~\arrow आ~रम्~\arrow \textcolor{red}{व्याङ्परिभ्यो रमः} (पा॰सू॰~१.३.८३)~\arrow \textcolor{red}{लट् स्मे} (पा॰सू॰~३.२.११८)~\arrow आ~रम्~लट्~\arrow आ~रम्~मिप्~\arrow आ~रम्~मि~\arrow \textcolor{red}{कर्तरि शप्‌} (पा॰सू॰~३.१.६८)~\arrow आ~रम्~शप्~मि~\arrow आ~रम्~अ~मि~\arrow \textcolor{red}{अतो दीर्घो यञि} (पा॰सू॰~७.३.१०१)~\arrow आ~रम्~आ~मि~\arrow आरमामि।} \textcolor{red}{आङ्} इत्यस्य \textcolor{red}{विनाऽपि प्रत्ययं पूर्वोत्तर\-पद\-लोपो वक्तव्यः} (वा॰~५.३.८३) इत्यनेन लोपः। यद्वा \textcolor{red}{तुष्टा आरमामि} इति स्थिते \textcolor{red}{अकः सवर्णे दीर्घः} (पा॰सू॰~६.१.१०१) इत्यनेन दीर्घैकादेशे \textcolor{red}{तुष्टाऽऽरमामि} इति।\footnote{यद्वा \textcolor{red}{अनुदात्तेत्त्व\-लक्षणमात्मने\-पदमनित्यम्} (प॰शे॰~९३.४) इत्यपि समाधानम्। \textcolor{red}{रमुँ क्रीडायाम्} (धा॰पा॰~८५३)~\arrow रम्~\arrow \textcolor{red}{अनुदात्तेत्त्व\-लक्षणमात्मने\-पदमनित्यम्} (प॰शे॰~९३.४)~\arrow \textcolor{red}{शेषात्कर्तरि परस्मैपदम्} (पा॰सू॰~१.३.७८)~\arrow \textcolor{red}{लट् स्मे} (पा॰सू॰~३.२.११८)~\arrow रम्~लट्~\arrow रम्~मिप्~\arrow रम्~मि~\arrow \textcolor{red}{कर्तरि शप्‌} (पा॰सू॰~३.१.६८)~\arrow रम्~शप्~मि~\arrow रम्~अ~मि~\arrow \textcolor{red}{अतो दीर्घो यञि} (पा॰सू॰~७.३.१०१)~\arrow रम्~आ~मि~\arrow रमामि।}\end{sloppypar}
\section[द्रक्ष्यथ]{द्रक्ष्यथ}
\centering\textcolor{blue}{रामोऽपि पादचारेण गजाश्वादिविवर्जितः।\nopagebreak\\
गच्छति द्रक्ष्यथ विभुं सर्वलोकैकसुन्दरम्॥}\nopagebreak\\
\raggedleft{–~अ॰रा॰~२.५.७}\\
\fontsize{14}{21}\selectfont\begin{sloppypar}\hyphenrules{nohyphenation}\justifying\noindent\hspace{10mm} अत्र \textcolor{red}{पश्यत} इति प्रयोक्तव्ये \textcolor{red}{द्रक्ष्यथ} इति प्रयुक्तम्। \textcolor{red}{द्रक्ष्} इति स्वतन्त्रो धातुः।\footnote{\textcolor{red}{बहुलमेतन्निदर्शनम्} (धा॰पा॰ ग॰सू॰~१९३८) \textcolor{red}{भूवादिष्वेतदन्तेषु दशगणीषु धातूनां पाठो निदर्शनाय तेन स्तम्भुप्रभृतयः सौत्राश्चुलुम्पादयो वाक्यकारीयाः प्रयोगसिद्धा विक्लवत्यादयश्च} (मा॰धा॰वृ॰~१०.३२८) इत्यनुसारं दिवादिगण ऊह्योऽयं \textcolor{red}{द्रक्ष्‌}\-धातुर्दर्शने।} तस्य लटि मध्यम\-पुरुष\-बहु\-वचनम् \textcolor{red}{द्रक्ष्यथ}।\footnote{ द्रक्ष्~\arrow \textcolor{red}{शेषात्कर्तरि परस्मैपदम्} (पा॰सू॰~१.३.७८)~\arrow \textcolor{red}{वर्तमाने लट्} (पा॰सू॰~३.२.१२३)~\arrow द्रक्ष्~लट्~\arrow द्रक्ष्~थ~\arrow \textcolor{red}{दिवादिभ्यः श्यन्} (पा॰सू॰~३.१.६९)~\arrow द्रक्ष्~श्यन्~थ~\arrow द्रक्ष्~य~थ~\arrow द्रक्ष्यथ।} यद्वा \textcolor{red}{अभिज्ञा\-वचने लृट्} (पा॰सू॰~३.२.११२)।\footnote{\textcolor{red}{दृशिँर प्रेक्षणे} (धा॰पा॰~९८८)~\arrow दृश्~\arrow \textcolor{red}{शेषात्कर्तरि परस्मैपदम्} (पा॰सू॰~१.३.७८)~\arrow \textcolor{red}{अभिज्ञा\-वचने लृट्} (पा॰सू॰~३.२.११२)~\arrow दृश्~लृट्~\arrow दृश्~थ~\arrow \textcolor{red}{स्यतासी लृलुटोः} (पा॰सू॰~३.१.३३)~\arrow दृश्~स्य~थ~\arrow \textcolor{red}{सृजि\-दृशोर्झल्यमकिति} (पा॰सू॰~६.१.५८)~\arrow \textcolor{red}{मिदचोऽन्त्यात्परः} (पा॰सू॰~१.१.४७)~\arrow दृ~अम्~श्~स्य~थ~\arrow दृ~अ~श्~स्य~थ~\arrow \textcolor{red}{इको यणचि} (पा॰सू॰~६.१.७७)~\arrow द्र्~अ~श्~स्य~थ~\arrow \textcolor{red}{व्रश्चभ्रस्ज\-सृजमृज\-यजराज\-भ्राजच्छशां षः} (पा॰सू॰~८.२.३६)~\arrow द्र्~अ~ष्~स्य~थ~\arrow \textcolor{red}{षढोः कः सि} (पा॰सू॰~८.२.४१)~\arrow द्र्~अ~क्~स्य~थ~\arrow \textcolor{red}{आदेश\-प्रत्यययोः} (पा॰सू॰~८.३.५९)~\arrow द्र्~अ~क्~ष्य~थ~\arrow द्रक्ष्यथ।} \textcolor{red}{स्मरथ किमीदृशं द्रक्ष्यथ रामं पूर्वम्}। अतो नापाणिनीयम्।\end{sloppypar}
\section[जीवामि]{जीवामि}
\centering\textcolor{blue}{किञ्चित्कालं भवेत्तत्र जीवनं दुःखितस्य मे।\nopagebreak\\
अत ऊर्ध्वं न जीवामि चिरं रामं विना कृतः॥}\nopagebreak\\
\raggedleft{–~अ॰रा॰~२.५.४९}\\
\fontsize{14}{21}\selectfont\begin{sloppypar}\hyphenrules{nohyphenation}\justifying\noindent\hspace{10mm} अत्र वर्तमान\-सामीप्ये लट्।\footnote{\textcolor{red}{वर्तमान\-सामीप्ये वर्तमानवद्वा} (पा॰सू॰~३.३.१३१) इत्यनेन। \textcolor{red}{जीवँ प्राणधारणे} (धा॰पा॰~५६२)~\arrow जीव्~\arrow \textcolor{red}{शेषात्कर्तरि परस्मैपदम्} (पा॰सू॰~१.३.७८)~\arrow \textcolor{red}{वर्तमान\-सामीप्ये वर्तमानवद्वा} (पा॰सू॰~३.३.१३१)~\arrow \textcolor{red}{वर्तमाने लट्} (पा॰सू॰~३.२.१२३)~\arrow जीव्~लट्~\arrow जीव्~मिप्~\arrow जीव्~मि~\arrow \textcolor{red}{कर्तरि शप्‌} (पा॰सू॰~३.१.६८)~\arrow जीव्~शप्~मि~\arrow जीव्~अ~मि~\arrow \textcolor{red}{अतो दीर्घो यञि} (पा॰सू॰~७.३.१०१)~\arrow जीव्~आ~मि~\arrow जीवामि।}\end{sloppypar}
\section[गच्छामहे]{गच्छामहे}
\centering\textcolor{blue}{पौराः सर्वे समागत्य स्थितास्तस्याविदूरतः।\nopagebreak\\
शक्ता रामं पुरं नेतुं नोचेद्गच्छामहे वनम्॥}\nopagebreak\\
\raggedleft{–~अ॰रा॰~२.५.५३}\\
\fontsize{14}{21}\selectfont\begin{sloppypar}\hyphenrules{nohyphenation}\justifying\noindent\hspace{10mm} \textcolor{red}{हे पौरा वनं गच्छाम} इत्यन्वये शक्यार्थे लोट्।\footnote{\textcolor{red}{शकि लिङ् च} (पा॰सू॰~३.३.१७२) इत्यत्र मण्डूक\-प्लुत्या \textcolor{red}{स्मे लोट्} (पा॰सू॰~३.३.१६५) इत्यतः \textcolor{red}{लोट्} इत्यनुवर्तनीयमिति भावः। \textcolor{red}{गमॢँ गतौ} (धा॰पा॰~९८२)~\arrow गम्~\arrow \textcolor{red}{शेषात्कर्तरि परस्मैपदम्} (पा॰सू॰~१.३.७८)~\arrow \textcolor{red}{शकि लिङ् च} (पा॰सू॰~३.३.१७२)~\arrow गम्~लोट्~\arrow गम्~मस्~\arrow \textcolor{red}{कर्तरि शप्‌} (पा॰सू॰~३.१.६८)~\arrow गम्~शप्~मस्~\arrow गम्~अ~मस्~\arrow \textcolor{red}{इषुगमियमां छः} (पा॰सू॰~७.३.७७)~\arrow गछ्~अ~मस्~\arrow \textcolor{red}{छे च} (पा॰सू॰~६.१.७३)~\arrow \textcolor{red}{आद्यन्तौ टकितौ} (पा॰सू॰~१.१.४६)~\arrow गतुँक्~छ्~अ~मस्~\arrow गत्~छ्~अ~मस्~\arrow \textcolor{red}{स्तोः श्चुना श्चुः} (पा॰सू॰~८.४.४०)~\arrow गच्~छ्~अ~मस्~\arrow \textcolor{red}{आडुत्तमस्य पिच्च} (पा॰सू॰~३.४.९२)~\arrow गच्~छ्~अ~आट्~मस्~\arrow \textcolor{red}{अकः सवर्णे दीर्घः} (पा॰सू॰~६.१.१०१)~\arrow गच्~छ्~आ~मस्~\arrow \textcolor{red}{लोटो लङ्वत्‌} (पा॰सू॰~३.४.८५)~\arrow ङिद्वत्त्वम्~\arrow नित्यं ङितः~\arrow गच्~छ्~आ~म~\arrow गच्छाम।} \textcolor{red}{वनं गन्तुं शक्ताः} इति भावः।\end{sloppypar}
\section[बभूव]{बभूव}
\centering\textcolor{blue}{बभूव परमानदः स्पृष्ट्वा तेऽङ्गं रघूत्तम।\nopagebreak\\
नैषादराज्यमेतत्ते किङ्करस्य रघूत्तम॥}\nopagebreak\\
\raggedleft{–~अ॰रा॰~२.५.६५}\\
\fontsize{14}{21}\selectfont\begin{sloppypar}\hyphenrules{nohyphenation}\justifying\noindent\hspace{10mm} इह प्रेमातिशयात्प्रभुं प्रत्यक्षं मत्वा सर्वं परोक्षं मनुते निषादः। अतो \textcolor{red}{बभूव} इति क्रियां प्रयुङ्क्ते।\footnote{\textcolor{red}{भू सत्तायाम्} (धा॰पा॰~१)~\arrow भू~\arrow \textcolor{red}{शेषात्कर्तरि परस्मैपदम्} (पा॰सू॰~१.३.७८)~\arrow \textcolor{red}{परोक्षे लिट्} (पा॰सू॰~३.२.११५)~\arrow भू~लिट्~\arrow भू~तिप्~\arrow \textcolor{red}{परस्मैपदानां णलतुसुस्थलथुस\-णल्वमाः} (पा॰सू॰~३.४.८२)~\arrow भू~ण~\arrow भू~अ~\arrow \textcolor{red}{भुवो वुग्लुङ्लिटोः} (पा॰सू॰~६.४.८८)~\arrow \textcolor{red}{आद्यन्तौ टकितौ} (पा॰सू॰~१.१.४६)~\arrow भूवुँक्~अ~\arrow भूव्~अ~\arrow \textcolor{red}{लिटि धातोरनभ्यासस्य} (पा॰सू॰~६.१.८)~\arrow भूव्~भूव्~अ~\arrow \textcolor{red}{हलादिः शेषः} (पा॰सू॰~७.४.६०)~\arrow भू~भूव्~अ~\arrow \textcolor{red}{ह्रस्वः} (पा॰सू॰~७.४.५९)~\arrow भु~भूव्~अ~\arrow \textcolor{red}{भवतेरः} (पा॰सू॰~७.४.७३)~\arrow भ~भूव्~अ~\arrow \textcolor{red}{अभ्यासे चर्च} (पा॰सू॰~८.४.५४)~\arrow ब~भूव्~अ~\arrow बभूव।}\end{sloppypar}
\section[शेते]{शेते}
\centering\textcolor{blue}{शयानं कुशपत्रौघसंस्तरे सीतया सह।\nopagebreak\\
यः शेते स्वर्णपर्यङ्के स्वास्तीर्णे भवनोत्तमे॥}\nopagebreak\\
\raggedleft{–~अ॰रा॰~२.६.२}\\
\fontsize{14}{21}\selectfont\begin{sloppypar}\hyphenrules{nohyphenation}\justifying\noindent\hspace{10mm} भूतकालविवक्षया \textcolor{red}{अशेत}\footnote{\textcolor{red}{शीङ् स्वप्ने} (धा॰पा॰~१०३२)~\arrow शी~\arrow \textcolor{red}{अनुदात्तङित आत्मने\-पदम्} (पा॰सू॰~१.३.१२)~\arrow \textcolor{red}{अनद्यतने लङ्} (पा॰सू॰~३.२.१११)~\arrow शी~लङ्~\arrow शी~त~\arrow \textcolor{red}{लुङ्लङ्लृङ्क्ष्वडुदात्तः} (पा॰सू॰~६.४.७१)~\arrow \textcolor{red}{आद्यन्तौ टकितौ} (पा॰सू॰~१.१.४६)~\arrow अट्~शी~त~\arrow अ~शी~त~\arrow \textcolor{red}{कर्तरि शप्} (पा॰सू॰~३.१.६८)~\arrow अ~शी~शप्~त~\arrow \textcolor{red}{शीङः सार्वधातुके गुणः} (पा॰सू॰~७.४.२१)~\arrow अ~शे~शप्~त~\arrow \textcolor{red}{अदिप्रभृतिभ्यः शपः} (पा॰सू॰~२.४.७२)~\arrow अ~शे~त~\arrow अशेत।} इति प्रयोक्तव्ये \textcolor{red}{शेते} इति प्रयुक्तम्। अत्र \textcolor{red}{पुरि लुङ् चास्मे} (पा॰सू॰~३.२.१२२) इति सूत्रेण वर्तमानवद्व्यवहारः।\footnote{\textcolor{red}{पुरा} इत्यध्याहार्यमिति भावः। \textcolor{red}{शीङ् स्वप्ने} (धा॰पा॰~१०३२)~\arrow शी~\arrow \textcolor{red}{अनुदात्तङित आत्मने\-पदम्} (पा॰सू॰~१.३.१२)~\arrow \textcolor{red}{पुरि लुङ् चास्मे} (पा॰सू॰~३.२.१२२)~\arrow शी~लट्~\arrow शी~त~\arrow \textcolor{red}{कर्तरि शप्} (पा॰सू॰~३.१.६८)~\arrow शी~शप्~त~\arrow \textcolor{red}{शीङः सार्वधातुके गुणः} (पा॰सू॰~७.४.२१)~\arrow शे~शप्~त~\arrow \textcolor{red}{अदिप्रभृतिभ्यः शपः} (पा॰सू॰~२.४.७२)~\arrow शे~त~\arrow \textcolor{red}{टित आत्मनेपदानां टेरे} (पा॰सू॰~३.४.७९)~\arrow शे~ते~\arrow शेते।}\end{sloppypar}
\section[प्रार्थयामास]{प्रार्थयामास}
\centering\textcolor{blue}{गुहस्तान्वाहयामास ज्ञातिभिः सहितः स्वयम्।\nopagebreak\\
गङ्गामध्ये गतां गङ्गां प्रार्थयामास जानकी॥}\nopagebreak\\
\raggedleft{–~अ॰रा॰~२.६.२१}\\
\fontsize{14}{21}\selectfont\begin{sloppypar}\hyphenrules{nohyphenation}\justifying\noindent\hspace{10mm} अनुदात्तेत्त्व\-लक्षणमात्मनेपदमनित्यमत एव \textcolor{red}{प्रार्थयामास}।\footnote{\pageref{sec:prarthaya}तमे पृष्ठे \ref{sec:prarthaya} \nameref{sec:prarthaya} इति प्रयोगस्य \pageref{sec:prarthayami}तमे पृष्ठे \ref{sec:prarthayami} \nameref{sec:prarthayami} इति प्रयोगस्य च विमर्शमपि पश्यन्तु।}\end{sloppypar}
\section[वदस्व]{वदस्व}
\centering\textcolor{blue}{भवन्तो यदि जानन्ति किं वक्ष्यामोऽत्र कारणम्।\nopagebreak\\
यत्र मे सुखवासाय भवेत्स्थानं वदस्व तत्॥}\nopagebreak\\
\raggedleft{–~अ॰रा॰~२.६.५०}\\
\fontsize{14}{21}\selectfont\begin{sloppypar}\hyphenrules{nohyphenation}\justifying\noindent\hspace{10mm} \textcolor{red}{भासनोपसम्भाषा\-ज्ञान\-यत्न\-विमत्युपमन्त्रणेषु वदः} (पा॰सू॰~१.३.४७) इत्यनेन भासन आत्मनेपदम्। \textcolor{red}{भासमानो वद इति} तात्पर्यम्।\footnote{\textcolor{red}{वदँ व्यक्तायां वाचि} (धा॰पा॰~१००९)~\arrow वद्~\arrow \textcolor{red}{भासनोपसम्भाषा\-ज्ञान\-यत्न\-विमत्युपमन्त्रणेषु वदः} (पा॰सू॰~१.३.४७)~\arrow \textcolor{red}{लोट् च} (पा॰सू॰~३.३.१६२)~\arrow वद्~लोट्~\arrow वद्~थास्~\arrow \textcolor{red}{कर्तरि शप्} (पा॰सू॰~३.१.६८)~\arrow वद्~शप्~थास्~\arrow वद्~अ~थास्~\arrow \textcolor{red}{थासस्से} (पा॰सू॰~३.४.८०)~\arrow वद्~अ~से~\arrow \textcolor{red}{सवाभ्यां वामौ} (पा॰सू॰~३.४.९१)~\arrow वद्~अ~स्~व~\arrow वदस्व।}\end{sloppypar}
\section[स्थास्यामहे]{स्थास्यामहे}
\centering\textcolor{blue}{वयं स्थास्यामहे तावदागमिष्यसि निश्चयः।\nopagebreak\\
तथेत्युक्त्वा गृहं गत्वा मुनिभिर्यदुदीरितम्॥}\nopagebreak\\
\raggedleft{–~अ॰रा॰~२.६.७३}\\
\fontsize{14}{21}\selectfont\begin{sloppypar}\hyphenrules{nohyphenation}\justifying\noindent\hspace{10mm} \textcolor{red}{अकर्मकाच्च} (पा॰सू॰~१.३.२६) इत्यनेन आत्मनेपदम्।\footnote{\textcolor{red}{उप}\-उपसर्गस्य लोप इति भावः। उप~\textcolor{red}{ष्ठा गतिनिवृत्तौ} (धा॰पा॰~९२८)~\arrow उप~ष्ठा~\arrow \textcolor{red}{धात्वादेः षः सः} (पा॰सू॰~६.१.६४)~\arrow निमित्तापाये नैमित्तिकस्याप्यपायः~\arrow उप~स्था~\arrow \textcolor{red}{अकर्मकाच्च} (पा॰सू॰~१.३.२६)~\arrow \textcolor{red}{लृट् शेषे च} (पा॰सू॰~३.३.१३)~\arrow उप~स्था~लृट्~\arrow उप~स्था~महिङ्~\arrow उप~स्था~महि~\arrow \textcolor{red}{स्यतासी लृलुटोः} (पा॰सू॰~३.१.३३)~\arrow उप~स्था~स्य~महि~\arrow \textcolor{red}{अतो दीर्घो यञि} (पा॰सू॰~७.३.१०१)~\arrow उप~स्था~स्या~महि~\arrow \textcolor{red}{टित आत्मनेपदानां टेरे} (पा॰सू॰~३.४.७९)~\arrow उप~स्था~स्या~महे~\arrow उप~स्थास्यामहे~\arrow \textcolor{red}{विनाऽपि प्रत्ययं पूर्वोत्तर\-पद\-लोपो वक्तव्यः} (वा॰~५.३.८३)~\arrow स्थास्यामहे।} यद्वा \textcolor{red}{प्रकाशन\-स्थेयाख्ययोश्च} (पा॰सू॰~१.३.२३) इत्यनेन स्थेयाख्यायामात्मने\-पदम्।\footnote{प्रक्रिया पूर्ववत्। \textcolor{red}{ष्ठा गतिनिवृत्तौ} (धा॰पा॰~९२८)~\arrow \textcolor{red}{स्था} (पूर्ववत्)~\arrow \textcolor{red}{प्रकाशन\-स्थेयाख्ययोश्च} (पा॰सू॰~१.३.२३)~\arrow \textcolor{red}{लृट् शेषे च} (पा॰सू॰~३.३.१३)~\arrow स्था~लृट्~\arrow स्था~महिङ्~\arrow स्था~महि~\arrow \textcolor{red}{स्यतासी लृलुटोः} (पा॰सू॰~३.१.३३)~\arrow स्था~स्य~महि~\arrow \textcolor{red}{अतो दीर्घो यञि} (पा॰सू॰~७.३.१०१)~\arrow स्था~स्या~महि~\arrow \textcolor{red}{टित आत्मनेपदानां टेरे} (पा॰सू॰~३.४.७९)~\arrow स्था~स्या~महे~\arrow स्थास्यामहे।}\end{sloppypar}
\section[तिष्ठन्ति]{तिष्ठन्ति}
\centering\textcolor{blue}{तच्छ्रुत्वा जातनिर्वेदो विचार्य पुनरागमम्।\nopagebreak\\
मुनयो यत्र तिष्ठन्ति करुणापूर्णमानसाः॥}\nopagebreak\\
\raggedleft{–~अ॰रा॰~२.६.७५}\\
\fontsize{14}{21}\selectfont\begin{sloppypar}\hyphenrules{nohyphenation}\justifying\noindent\hspace{10mm} अत्र \textcolor{red}{तिष्ठन्ति स्म} इति प्रयोगः। अतः \textcolor{red}{लट् स्मे} (पा॰सू॰~३.२.११८) इत्यनेन लड्लकारः।\footnote{\textcolor{red}{स्म} इत्यध्याहार्यमिति भावः। \textcolor{red}{ष्ठा गतिनिवृत्तौ} (धा॰पा॰~९२८)~\arrow ष्ठा~\arrow \textcolor{red}{धात्वादेः षः सः} (पा॰सू॰~६.१.६४)~\arrow निमित्तापाये नैमित्तिकस्याप्यपायः~\arrow स्था~\arrow \textcolor{red}{शेषात्कर्तरि परस्मैपदम्} (पा॰सू॰~१.३.७८)~\arrow \textcolor{red}{लट् स्मे} (पा॰सू॰~३.२.११८)~\arrow स्था~लट्~\arrow स्था~झि~\arrow \textcolor{red}{कर्तरि शप्‌} (पा॰सू॰~३.१.६८)~\arrow स्था~शप्~झि~\arrow स्था~अ~झि~\arrow \textcolor{red}{पाघ्रा\-ध्मास्थाम्ना\-दाण्दृश्यर्त्ति\-सर्त्तिशदसदां पिब\-जिघ्र\-धम\-तिष्ठ\-मन\-यच्छ\-पश्यर्च्छ\-धौ\-शीय\-सीदाः} (पा॰सू॰~७.३.७८)~\arrow तिष्ठ्~अ~झि~\arrow \textcolor{red}{झोऽन्तः} (पा॰सू॰~७.१.३)~\arrow तिष्ठ्~अ~न्ति~\arrow तिष्ठन्ति।}\end{sloppypar}
\section[अस्मि]{अस्मि}
\centering\textcolor{blue}{मुनीनां दर्शनादेव शुद्धान्तःकरणोऽभवम्।\nopagebreak\\
धनुरादीन्परित्यज्य दण्डवत्पतितोऽस्म्यहम्॥}\nopagebreak\\
\raggedleft{–~अ॰रा॰~२.६.७६}\\
\fontsize{14}{21}\selectfont\begin{sloppypar}\hyphenrules{nohyphenation}\justifying\noindent\hspace{10mm} अत्रापि \textcolor{red}{स्म}\-योगे वर्तमानो भावः।\footnote{\textcolor{red}{स्म} इत्यध्याहार्यमिति भावः। \textcolor{red}{असँ भुवि} (धा॰पा॰~१०६५)~\arrow अस्~\arrow \textcolor{red}{शेषात्कर्तरि परस्मैपदम्} (पा॰सू॰~१.३.७८)~\arrow \textcolor{red}{लट् स्मे} (पा॰सू॰~३.२.११८)~\arrow अस्~लट्~\arrow अस्~मिप्~\arrow अस्~मि~\arrow \textcolor{red}{कर्तरि शप्‌} (पा॰सू॰~३.१.६८)~\arrow अस्~शप्~मि~\arrow \textcolor{red}{अदिप्रभृतिभ्यः शपः} (पा॰सू॰~२.४.७२)~\arrow अस्~मि~\arrow अस्मि।} अथवा \textcolor{red}{अस्मि} इत्यव्ययम्।\footnote{तथा च वाचस्पत्ये~– \textcolor{red}{अस्मि। अव्य॰~अस्–मिन्। अहमर्थे, ‘त्वामस्मि वच्मि विदुषां समवायोऽत्र तिष्ठति’ सा॰द॰। ‘अस्मिता’। ‘उडुपेनास्मि सागरम्’ रघुः। ‘ब्रह्मैवास्मि न शोकभाक्।’} योगसूत्रे च~– \textcolor{red}{अविद्यास्मिता\-रागद्वेषाभि\-निवेशाः क्लेशाः} (यो॰सू॰~२.३)।}\end{sloppypar}
\section[रक्षध्वम्]{रक्षध्वम्}
\centering\textcolor{blue}{रक्षध्वं मां मुनिश्रेष्ठा गच्छन्तं निरयार्णवम्।\nopagebreak\\
इत्यग्रे पतितं दृष्ट्वा मामूचुर्मुनिसत्तमाः॥}\nopagebreak\\
\raggedleft{–~अ॰रा॰~२.६.७७}\\
\fontsize{14}{21}\selectfont\begin{sloppypar}\hyphenrules{nohyphenation}\justifying\noindent\hspace{10mm} इह परस्मैपदीय\-\textcolor{red}{रक्ष्‌}\-धातोः (\textcolor{red}{रक्षँ पालने} धा॰पा॰~६५७) आत्मनेपद\-प्रयोगस्तु कर्म\-व्यतिहारे।\footnote{\textcolor{red}{कर्तरि कर्मव्यतिहारे} (पा॰सू॰~१.३.१४) इत्यनेन। \textcolor{red}{रक्षँ पालने} (धा॰पा॰~६५७)~\arrow रक्ष्~\arrow \textcolor{red}{कर्तरि कर्मव्यतिहारे} (पा॰सू॰~१.३.१४)~\arrow \textcolor{red}{लोट् च} (पा॰सू॰~३.३.१६२)~\arrow रक्ष्~लोट्~\arrow रक्ष्~ध्वम्~\arrow \textcolor{red}{कर्तरि शप्‌} (पा॰सू॰~३.१.६८)~\arrow रक्ष्~शप्~ध्वम्~\arrow रक्ष्~अ~ध्वम्~\arrow रक्षध्वम्।}
यद्वा \textcolor{red}{रक्षस इव अध्वा यस्य}\footnote{\textcolor{red}{रक्षः} इत्यस्य राक्षस इत्यर्थः। \textcolor{red}{यातुधानः पुण्यजनो नैऋतो यातुरक्षसी} (अ॰को॰~१.१.६०) इत्यमरः।}
इति बहुव्रीहि\-समासे पृषोदरादित्वाट्टचि शकन्ध्वादित्वात्पर\-रूपे विभक्तिकार्ये \textcolor{red}{रक्षध्वः}।\footnote{रक्षस्~ङस्~अध्वन्~सुँ~\arrow \textcolor{red}{अनेकमन्यपदार्थे} (पा॰सू॰~२.२.२४)~\arrow \textcolor{red}{कृत्तद्धित\-समासाश्च} (पा॰सू॰~१.२.४६)~\arrow प्रातिपदिक\-सञ्ज्ञा~\arrow \textcolor{red}{सुपो धातु\-प्रातिपदिकयोः} (पा॰सू॰~२.४.७१)~\arrow रक्षस्~अध्वन्~सुँ~\arrow \textcolor{red}{शकन्ध्वादिषु पररूपं वाच्यम्} (वा॰~६.१.९४)~\arrow रक्षध्वन्~सुँ~\arrow \textcolor{red}{पृषोदरादीनि यथोपदिष्टम्} (पा॰सू॰~६.३.१०९)~\arrow टज्भावः~\arrow रक्षध्वन्~सुँ~\arrow रक्षध्वन्~टच्~सुँ~\arrow रक्षध्वन्~अ~सुँ~\arrow \textcolor{red}{नस्तद्धिते} (पा॰सू॰~६.४.१४४)~\arrow रक्षध्व्~अ~सुँ~\arrow रक्षध्व~सुँ~\arrow रक्षध्वस्~\arrow \textcolor{red}{ससजुषो रुः} (पा॰सू॰~८.२.६६)~\arrow रक्षध्वरुँ~\arrow \textcolor{red}{खरवसानयोर्विसर्जनीयः} (पा॰सू॰~८.३.१५)~\arrow रक्षध्वः।} तं \textcolor{red}{रक्षध्वम्}।\footnote{रक्षध्व~अम्~\arrow \textcolor{red}{अमि पूर्वः} (पा॰सू॰~६.१.१०७)~\arrow रक्षध्वम्।
} \textcolor{red}{रक्षध्वं मां पात} इति तात्पर्यम्।\footnote{\textcolor{red}{पात}~({\englishfont =}रक्षत) इत्यध्याहार्यमिति भावः।}\end{sloppypar}
\section[निष्क्रमस्व]{निष्क्रमस्व}
\centering\textcolor{blue}{ततो युगसहस्रान्ते ऋषयः पुनरागमन्।\nopagebreak\\
मामूचुर्निष्क्रमस्वेति तच्छ्रुत्वा तूर्णमुत्थितः॥}\nopagebreak\\
\raggedleft{–~अ॰रा॰~२.६.८४}\\
\fontsize{14}{21}\selectfont\begin{sloppypar}\hyphenrules{nohyphenation}\justifying\noindent\hspace{10mm} अस्मिन् प्रयोगे \textcolor{red}{वृत्ति\-सर्ग\-तायनेषु क्रमः} (पा॰सू॰~१.३.३८) इत्यनेनाऽत्मनेपदम्। \textcolor{red}{वर्धमानो निष्क्रमस्व} इति तायने \textcolor{red}{थास्}।\footnote{निस्~\textcolor{red}{क्रमुँ पादविक्षेपे} (धा॰पा॰~१९०५)~\arrow निस्~क्रम्~\arrow \textcolor{red}{वृत्ति\-सर्ग\-तायनेषु क्रमः} (पा॰सू॰~१.३.३८)~\arrow \textcolor{red}{लोट् च} (पा॰सू॰~३.३.१६२)~\arrow निस्~क्रम्~लोट्~\arrow निस्~क्रम्~थास्~\arrow \textcolor{red}{कर्तरि शप्‌} (पा॰सू॰~३.१.६८)~\arrow निस्~क्रम्~शप्~थास्~\arrow निस्~क्रम्~अ~थास्~\arrow \textcolor{red}{थासस्से} (पा॰सू॰~३.४.८०)~\arrow निस्~क्रम्~अ~से~\arrow \textcolor{red}{सवाभ्यां वामौ} (पा॰सू॰~३.४.९१)~\arrow निस्~क्रम्~अ~स्व~\arrow \textcolor{red}{नुम्विसर्जनीय\-शर्व्यवायेऽपि} (पा॰सू॰~८.३.५८)~\arrow निष्~क्रम्~अ~स्व~\arrow निष्क्रमस्व।}\end{sloppypar}
\section[प्रार्थय]{प्रार्थय}
\label{sec:prarthaya}
\centering\textcolor{blue}{मा भैषीरिति मां प्राह ब्रह्महत्याभयं न ते।\nopagebreak\\
मत्पित्रोः सलिलं दत्त्वा नत्वा प्रार्थय जीवितम्॥}\nopagebreak\\
\raggedleft{–~अ॰रा॰~२.७.३९}\\
\fontsize{14}{21}\selectfont\begin{sloppypar}\hyphenrules{nohyphenation}\justifying\noindent\hspace{10mm} \textcolor{red}{प्रार्थयस्व}\footnote{प्र~\textcolor{red}{अर्थँ उपयाच्ञायाम्} (धा॰पा॰~१९०५)~\arrow प्र~अर्थ्~\arrow \textcolor{red}{सत्याप\-पाश\-रूप\-वीणा\-तूल\-श्लोक\-सेना\-लोम\-त्वच\-वर्म\-वर्ण\-चूर्ण\-चुरादिभ्यो णिच्} (पा॰सू॰~३.१.२५)~\arrow प्र~अर्थ्~णिच्~\arrow प्र~अर्थ्~इ~\arrow प्र~अर्थि~\arrow \textcolor{red}{सनाद्यन्ता धातवः} (पा॰सू॰~३.१.३२)~\arrow धातु\-सञ्ज्ञा~\arrow \textcolor{red}{आगर्वादात्मने\-पदिनः} (धा॰पा॰ ग॰सू॰)~\arrow \textcolor{red}{लोट् च} (पा॰सू॰~३.३.१६२)~\arrow प्र~अर्थि~लोट्~\arrow प्र~अर्थि~थास्~\arrow \textcolor{red}{कर्तरि शप्‌} (पा॰सू॰~३.१.६८)~\arrow प्र~अर्थि~शप्~थास्~\arrow प्र~अर्थि~अ~थास्~\arrow \textcolor{red}{सार्वधातुकार्ध\-धातुकयोः} (पा॰सू॰~७.३.८४)~\arrow प्र~अर्थे~अ~थास्~\arrow \textcolor{red}{एचोऽयवायावः} (पा॰सू॰~६.१.७८)~\arrow प्र~अर्थय्~अ~थास्~\arrow \textcolor{red}{थासस्से} (पा॰सू॰~३.४.८०)~\arrow प्र~अर्थय्~अ~से~\arrow \textcolor{red}{सवाभ्यां वामौ} (पा॰सू॰~३.४.९१)~\arrow प्र~अर्थय्~अ~स्व~\arrow \textcolor{red}{अकः सवर्णे दीर्घः} (पा॰सू॰~६.१.१०१)~\arrow प्रार्थय्~अ~स्व~\arrow प्रार्थयस्व।} इति प्रयोक्तव्ये \textcolor{red}{प्रार्थय} इति प्रयोग आत्मने\-पदस्यानित्यत्वात्।\footnote{\textcolor{red}{‘अनुदात्तेत्त्व\-प्रयुक्तमात्मने\-पदमनित्यम्’ इति ज्ञापनार्थोऽयं ङकारः। तेन ‘तालैः शिञ्जद्वलयसुभगैः’ ‘तृष्णे जृम्भसि’ ‘प्रार्थयन्ति शयनोत्थितं प्रियाः’ इत्यादयः प्रयोगा उपपद्यन्ते} (मा॰धा॰वृ॰~२.९)। प्र~अर्थि~\arrow धातु\-सञ्ज्ञा (पूर्ववत्)~\arrow \textcolor{red}{अनुदात्तेत्त्व\-लक्षणमात्मने\-पदमनित्यम्} (प॰शे॰~९३.४)~\arrow \textcolor{red}{शेषात्कर्तरि परस्मैपदम्} (पा॰सू॰~१.३.७८)~\arrow \textcolor{red}{लोट् च} (पा॰सू॰~३.३.१६२)~\arrow प्र~अर्थि~लोट्~\arrow प्र~अर्थि~सिप्~\arrow प्र~अर्थि~सि~\arrow \textcolor{red}{कर्तरि शप्‌} (पा॰सू॰~३.१.६८)~\arrow प्र~अर्थि~शप्~सि~\arrow प्र~अर्थि~अ~सि~\arrow \textcolor{red}{सार्वधातुकार्ध\-धातुकयोः} (पा॰सू॰~७.३.८४)~\arrow प्र~अर्थे~अ~सि~\arrow \textcolor{red}{एचोऽयवायावः} (पा॰सू॰~६.१.७८)~\arrow प्र~अर्थय्~अ~से~\arrow \textcolor{red}{सेर्ह्यपिच्च} (पा॰सू॰~३.४.८७)~\arrow प्र~अर्थय्~अ~हि~\arrow \textcolor{red}{अतो हेः} (पा॰सू॰~६.४.१०५)~\arrow प्र~अर्थय्~अ~\arrow \textcolor{red}{अकः सवर्णे दीर्घः} (पा॰सू॰~६.१.१०१)~\arrow प्रार्थय्~अ~\arrow प्रार्थय।} यद्वा \textcolor{red}{प्रार्थयत इति प्रार्थयः}।\footnote{\textcolor{red}{प्रार्थि}\-धातोरौणादिके \textcolor{red}{अयच्} प्रत्यये। \textcolor{red}{कार्याद्विद्यादनूबन्धम्} (भा॰पा॰सू॰~३.३.१) \textcolor{red}{केचिदविहिता अप्यूह्याः} (वै॰सि॰कौ॰~३१६९) इत्यनुसारमूह्योऽ\-त्राविहितो \textcolor{red}{अयच्} प्रत्ययः। \textcolor{red}{अटच्} (प॰उ॰~४.११४, द॰उ॰~१०.१५) \textcolor{red}{अतच्} (प॰उ॰~३.१०३–१०५, द॰उ॰~६.१४–१६) \textcolor{red}{अभच्} (प॰उ॰~३.११६–१२०, द॰उ॰~७.१८–२२) \textcolor{red}{अमच्} (प॰उ॰~५.७३–७४, द॰उ॰~५.७३–७४) \textcolor{red}{अलच्} (प॰उ॰~५.८१, द॰उ॰~५.८१) \textcolor{red}{असच्} (प॰उ॰~३.१११–११४, द॰उ॰~९.४४–४७) इतिवत्। प्र~अर्थि~\arrow धातु\-सञ्ज्ञा (पूर्ववत्)~\arrow \textcolor{red}{उणादयो बहुलम्} (पा॰सू॰~३.३.१)~\arrow प्र~अर्थि~अयच्~\arrow प्र~अर्थि~अय~\arrow \textcolor{red}{णेरनिटि} (पा॰सू॰~६.४.५१)~\arrow प्र~अर्थ्~अय~\arrow \textcolor{red}{अकः सवर्णे दीर्घः} (पा॰सू॰~६.१.१०१)~\arrow प्रार्थ्~अय~\arrow प्रार्थय~\arrow विभक्ति\-कार्यम्~\arrow प्रार्थय~सुँ~\arrow प्रार्थय~स्~\arrow \textcolor{red}{ससजुषो रुः} (पा॰सू॰~८.२.६६)~\arrow प्रार्थयरुँ~\arrow प्रार्थयर्~\arrow \textcolor{red}{खरवसानयोर्विसर्जनीयः} (पा॰सू॰~८.३.१५)~\arrow प्रार्थयः।} \textcolor{red}{प्रार्थय इवाऽचर} इति \textcolor{red}{प्रार्थय}।\footnote{प्रार्थय~\arrow \textcolor{red}{सर्वप्राति\-पदिकेभ्य आचारे क्विब्वा वक्तव्यः} (वा॰~३.१.११)~\arrow प्रार्थय~क्विँप्~\arrow प्रार्थय~व्~\arrow \textcolor{red}{वेरपृक्तस्य} (पा॰सू॰~६.१.६७)~\arrow प्रार्थय~\arrow \textcolor{red}{सनाद्यन्ता धातवः} (पा॰सू॰~३.१.३२)~\arrow धातुसञ्ज्ञा~\arrow \textcolor{red}{शेषात्कर्तरि परस्मैपदम्} (पा॰सू॰~१.३.७८)~\arrow \textcolor{red}{लोट् च} (पा॰सू॰~३.३.१६२)~\arrow प्रार्थय~लोट्~\arrow प्रार्थय~सिप्~\arrow प्रार्थय~सि~\arrow \textcolor{red}{कर्तरि शप्‌} (पा॰सू॰~३.१.६८)~\arrow प्रार्थय~शप्~सि~\arrow प्रार्थय~अ~सि~\arrow \textcolor{red}{अतो गुणे} (पा॰सू॰~६.१.९७)~\arrow प्रार्थय~सि~\arrow \textcolor{red}{सेर्ह्यपिच्च} (पा॰सू॰~३.४.८७)~\arrow प्रार्थय~हि~\arrow \textcolor{red}{अतो हेः} (पा॰सू॰~६.४.१०५)~\arrow प्रार्थय।} आचार\-क्विबन्त\-प्रत्ययः।\footnote{यद्वा \textcolor{red}{निवृत्त\-प्रेषणाद्धातोः प्राकृतेऽर्थे णिजुच्यते} (वा॰प॰~३.७.६०) इत्यनुसारं प्राकृतेऽर्थे णावन्याभिप्राये क्रियाफले परस्मैपदे लोटि सिपि शपि हौ हेर्लुकि \textcolor{red}{प्रार्थय}। यथा \textcolor{red}{निवृत्त\-प्रेषणाद्धातोर्हेतुमण्णौ शुद्धेन तुल्योऽर्थः। तेन ‘प्रार्थयन्ति शयनोत्थितं प्रियाः’ इत्यादि सिद्धम्। एवं सकर्मकेषु सर्वमूह्यम्} (ल॰सि॰कौ॰~२६०७)। यद्वा \textcolor{red}{प्रार्थनं प्रार्थस्तं कुरु} इति विग्रह आचार\-णिजन्ताद्धातोर्लोटि सिपि शपि हौ हेर्लुकि \textcolor{red}{प्रार्थय}। यथा \textcolor{red}{प्रार्थनं प्रार्थस्तत्करोति णौ ‘प्रार्थयति’ इति} (मा॰धा॰वृ॰~२.९) \textcolor{red}{‘प्रार्थयन्ति शयनोत्थितं प्रियाः’ इत्यादि कृदन्तात्तत्करोतीति णिचि नेयम्} (मा॰धा॰वृ॰~१०.२८७)। यत्तु तत्त्व\-बोधिन्यां \textcolor{red}{केचित्तु परस्मैपद\-सिद्ध्यर्थं प्रार्थनं प्रार्थस्तं कुर्वन्ति प्रार्थयन्तीति व्याचक्षते तदसत्। धातुसंज्ञा\-प्रयोजक\-प्रत्यये चिकीर्षित उपसर्गाणां पृथक्करणस्य वक्ष्यमाणतया ‘अर्थवेदे’त्यापुगागमस्य दुर्वारत्वात्}। तच्चिन्त्यम्। \textcolor{red}{प्रार्थयित्वा अप्रार्थयत्} इत्यादीनां शिष्टप्रयुक्तत्वात् यथा \textcolor{red}{अप्रार्थयत्कामधेनुम्} (अ॰पु॰~४.१६) \textcolor{red}{प्रार्थयित्वा द्विजान् भोज्य} (अ॰पु॰~१८४.८) \textcolor{red}{प्रार्थयित्वा विरोधितम्} (अ॰शा॰~८.५.२८) \textcolor{red}{प्रार्थयित्वा द्विजान्नृपान्} (ग॰सं॰~१०.५७.१) \textcolor{red}{प्रार्थयित्वा निजेश्वरम्} (ना॰पु॰~६६.१४) \textcolor{red}{तं प्रार्थयित्वा विधिवत्} (ब्रह्मा॰पु॰~२.५४.२८) इत्यादिषु। \textcolor{red}{प्रार्थाप्य अप्रार्थापयत्} इत्यादि\-प्रयोगाणामनु\-पलब्धेश्च।
\pageref{sec:prarthayami}तमे पृष्ठे \ref{sec:prarthayami} \nameref{sec:prarthayami} इति प्रयोगस्य विमर्शमपि पश्यन्तु।}\end{sloppypar}
\section[पृच्छसे]{पृच्छसे}
\centering\textcolor{blue}{तथाऽपि पृच्छसे किञ्चित्तदनुग्रह एव मे।\nopagebreak\\
कैकेय्या मत्कृतं कर्म रामराज्यविघातनम्॥}\nopagebreak\\
\raggedleft{–~अ॰रा॰~२.८.४६}\\
\fontsize{14}{21}\selectfont\begin{sloppypar}\hyphenrules{nohyphenation}\justifying\noindent\hspace{10mm} अत्रापि \textcolor{red}{कर्तरि कर्म\-व्यतिहारे} (पा॰सू॰~१.३.१४) इत्यनेनैवाऽत्मनेपदम्।\footnote{\textcolor{red}{प्रच्छँ ज्ञीप्सायाम्} (धा॰पा॰~१४१३)~\arrow प्रच्छ्~\arrow \textcolor{red}{कर्तरि कर्मव्यतिहारे} (पा॰सू॰~१.३.१४)~\arrow \textcolor{red}{वर्तमाने लट्} (पा॰सू॰~३.२.१२३)~\arrow प्रच्छ्~लट्~\arrow प्रच्छ्~थास्~\arrow \textcolor{red}{तुदादिभ्यः शः} (पा॰सू॰~३.१.७७)~\arrow प्रच्छ्~श~थास्~\arrow प्रच्छ्~अ~थास्~\arrow \textcolor{red}{सार्वधातुकमपित्} (पा॰सू॰~१.२.४)~\arrow ङिद्वत्त्वम्~\arrow \textcolor{red}{ग्रहिज्या\-वयिव्यधि\-वष्टिविचति\-वृश्चति\-पृच्छति\-भृज्जतीनां ङिति च} (पा॰सू॰~६.१.१६)~\arrow पृ~अ~च्छ्~अ~थास्~\arrow \textcolor{red}{सम्प्रसारणाच्च} (पा॰सू॰~६.१.१०८)~\arrow पृ~च्छ्~अ~थास्~\arrow \textcolor{red}{थासस्से} (पा॰सू॰~३.४.८०)~\arrow पृ~च्छ्~अ~से~\arrow पृच्छसे।} यतो हि \textcolor{red}{ज्ञात्वाऽपि सर्वं त्वं प्राकृत इव प्रश्नं करोषि} इति तात्पर्यम्।\end{sloppypar}
\section[नेष्ये]{नेष्ये}
\centering\textcolor{blue}{अभिषेक्ष्ये वसिष्ठाद्यैः पौरजानपदैः सह।\nopagebreak\\
नेष्येऽयोध्यां रमानाथं दासः सेवेऽतिनीचवत्॥}\nopagebreak\\
\raggedleft{–~अ॰रा॰~२.८.५१}\\
\fontsize{14}{21}\selectfont\begin{sloppypar}\hyphenrules{nohyphenation}\justifying\noindent\hspace{10mm} अस्मिन्प्रयोगे \textcolor{red}{सम्माननोत्सञ्जनाचार्य\-करण\-ज्ञान\-भृति\-विगणन\-व्ययेषु नियः} (पा॰सू॰~१.३.३६) इत्यादिनोत्सञ्जनार्थमात्मनेपदम्।\footnote{ञित्त्वात्कर्त्रभिप्राये क्रियाफले तु सिद्धमेवात्मने\-पदम्। परन्त्वत्र कर्त्रभिप्रायो न। क्रियाफलं त्वत्र राममेवाभिप्रैति न कर्तारं मामिति भक्तशिरोमणेर्भरतस्य भावात्। अकर्त्रभिप्राये कथमात्मने\-पदमित्याशङ्क्याहुः समाधानम्। \textcolor{red}{णीञ् प्रापणे} (धा॰पा॰~९०१)~\arrow णी~\arrow \textcolor{red}{णो नः} (पा॰सू॰~६.१.६५)~\arrow नी~\arrow \textcolor{red}{सम्माननोत्सञ्जनाचार्य\-करण\-ज्ञान\-भृति\-विगणन\-व्ययेषु नियः} (पा॰सू॰~१.३.३६)~\arrow \textcolor{red}{लृट् शेषे च} (पा॰सू॰~३.३.१३)~\arrow नी~लृट्~\arrow नी~इट्~\arrow नी~इ~\arrow \textcolor{red}{स्यतासी लृलुटोः} (पा॰सू॰~३.१.३३)~\arrow नी~स्य~इ~\arrow \textcolor{red}{सार्वधातुकार्ध\-धातुकयोः} (पा॰सू॰~७.३.८४)~\arrow ने~स्य~इ~\arrow \textcolor{red}{टित आत्मनेपदानां टेरे} (पा॰सू॰~३.४.७९)~\arrow ने~स्य~ए~\arrow \textcolor{red}{अतो गुणे} (पा॰सू॰~६.१.९७)~\arrow ने~स्ये~\arrow \textcolor{red}{आदेश\-प्रत्यययोः} (पा॰सू॰~८.३.५९)~\arrow ने~ष्ये~\arrow नेष्ये।} \textcolor{red}{सेवे} इत्यत्र च वर्तमान\-सामीप्ये\footnote{\textcolor{red}{वर्तमान\-सामीप्ये वर्तमानवद्वा} (पा॰सू॰~३.३.१३१) इत्यनेन। \textcolor{red}{षेवृँ सेवने} (धा॰पा॰~५०१)~\arrow षेव्~\arrow \textcolor{red}{धात्वादेः षः सः} (पा॰सू॰~६.१.६४)~\arrow सेव्~\arrow \textcolor{red}{अनुदात्तङित आत्मने\-पदम्} (पा॰सू॰~१.३.१२)~\arrow \textcolor{red}{लट् स्मे} (पा॰सू॰~३.२.११८)~\arrow सेव्~लट्~\arrow सेव्~इट्~\arrow सेव्~इ~\arrow \textcolor{red}{कर्तरि शप्‌} (पा॰सू॰~३.१.६८)~\arrow सेव्~शप्~इ~\arrow सेव्~अ~इ~\arrow \textcolor{red}{टित आत्मनेपदानां टेरे} (पा॰सू॰~३.४.७९)~\arrow सेव्~ए~\arrow \textcolor{red}{अतो गुणे} (पा॰सू॰~६.१.९७)~\arrow सेवे।} यद्वा \textcolor{red}{स्म} इति योगे भविष्यत्काले लट्।\footnote{\textcolor{red}{स्म} इत्यध्याहार्यमिति भावः। प्रक्रिया पूर्ववत्।}\end{sloppypar}
\section[भाषयेत्]{भाषयेत्}
\centering\textcolor{blue}{सर्वं देवकृतं नोचेदेवं सा भाषयेत्कथम्।\nopagebreak\\
तस्मात्त्यजाग्रहं तात रामस्य विनिवर्तने॥}\nopagebreak\\
\raggedleft{–~अ॰रा॰~२.९.४६}\\
\fontsize{14}{21}\selectfont\begin{sloppypar}\hyphenrules{nohyphenation}\justifying\noindent\hspace{10mm} इह स्वार्थे णिचि परस्मैपदम्।\footnote{ \textcolor{red}{भाषँ व्यक्तायां वाचि} (धा॰पा॰~६१२)~\arrow भाष्~\arrow स्वार्थे णिच्~\arrow भाष्~णिच्~\arrow भाष्~इ~\arrow भाषि~\arrow \textcolor{red}{सनाद्यन्ता धातवः} (पा॰सू॰~३.१.३२)~\arrow धातु\-सञ्ज्ञा~\arrow \textcolor{red}{शेषात्कर्तरि परस्मैपदम्} (पा॰सू॰~१.३.७८)~\arrow \textcolor{red}{विधि\-निमन्‍त्रणामन्‍त्रणाधीष्‍ट\-सम्प्रश्‍न\-प्रार्थनेषु लिङ्} (पा॰सू॰~३.३.१६१)~\arrow भाषि~लिङ्~\arrow भाषि~तिप्~\arrow भाषि~ति~\arrow \textcolor{red}{कर्तरि शप्‌} (पा॰सू॰~३.१.६८)~\arrow भाषि~शप्~ति~\arrow भाषि~अ~ति~\arrow \textcolor{red}{सार्वधातुकार्ध\-धातुकयोः} (पा॰सू॰~७.३.८४)~\arrow भाषे~अ~ति~\arrow \textcolor{red}{यासुट् परस्मैपदेषूदात्तो ङिच्च} (पा॰सू॰~३.४.१०३)~\arrow \textcolor{red}{आद्यन्तौ टकितौ} (पा॰सू॰~१.१.४६)~\arrow भाषे~अ~यासुँट्~ति~\arrow भाषे~अ~यास्~ति~\arrow \textcolor{red}{सुट् तिथोः} (पा॰सू॰~३.४.१०७)~\arrow \textcolor{red}{आद्यन्तौ टकितौ} (पा॰सू॰~१.१.४६)~\arrow भाषे~अ~यास्~सुँट्~ति~\arrow भाषे~अ~यास्~स्~ति~\arrow \textcolor{red}{लिङः सलोपोऽनन्त्यस्य} (पा॰सू॰~७.२.७९)~\arrow भाषे~अ~या~ति~\arrow\textcolor{red}{अतो येयः} (पा॰सू॰~७.२.८०)~\arrow भाषे~अ~इय्~ति~\arrow \textcolor{red}{लोपो व्योर्वलि} (पा॰सू॰~६.१.६६)~\arrow भाषे~अ~इ~ति~\arrow \textcolor{red}{एचोऽयवायावः} (पा॰सू॰~६.१.७८)~\arrow भाषय्~अ~इ~ति~\arrow \textcolor{red}{आद्गुणः} (पा॰सू॰~६.१.८७)~\arrow भाषय्~ए~ति~\arrow \textcolor{red}{इतश्च} (पा॰सू॰~३.४.१००)~\arrow भाषय्~ए~त्~\arrow भाषयेत्। न च चुरादिगणे पाठाभावात्स्वार्थे न णिच्। शिष्टप्रयोगेषु दृश्यते। यथा भारते वनपर्वणि भीमं प्रति नव\-व्याकरणार्थ\-वेत्ता हनुमान्~– \textcolor{red}{दशवर्षसहस्राणि दशवर्षशतानि च। राज्यं कारितवान् रामस्ततः स्वभवनं गतः॥} (म॰भा॰~३.१४८.१९) । अत्र नीलकण्ठोऽपि~– \textcolor{red}{कारितवान् कृतवान्। स्वार्थे णिच्} (म॰भा॰ भा॰दी॰~३.१४८.१९)। 
}\end{sloppypar}
\section[त्यक्ष्यते]{त्यक्ष्यते}
\centering\textcolor{blue}{अङ्गरागं च सीतायै ददौ दिव्यं शुभानना।\nopagebreak\\
न त्यक्ष्यतेऽङ्गरागेण शोभा त्वां कमलानने॥}\nopagebreak\\
\raggedleft{–~अ॰रा॰~२.९.८९}\\
\fontsize{14}{21}\selectfont\begin{sloppypar}\hyphenrules{nohyphenation}\justifying\noindent\hspace{10mm} चित्रकूटं परित्यज्य दण्डकारण्यं गन्तुकामः श्रीरामोऽत्रि\-दर्शनं करोति। तत्रानुसूया सीताया अङ्गरागं प्रयच्छति। इह \textcolor{red}{त्यक्ष्यते} इति प्रयोगो विमृश्यते। \textcolor{red}{त्यज्‌}\-धातुः (\textcolor{red}{त्यजँ हानौ} धा॰पा॰~९८६) परस्मैपदीयः। अत्र कर्म\-व्यतिहार आत्मनेपदम्।\footnote{त्यज्~\arrow \textcolor{red}{कर्तरि कर्मव्यतिहारे} (पा॰सू॰~१.३.१४)~\arrow \textcolor{red}{लृट् शेषे च} (पा॰सू॰~३.३.१३)~\arrow त्यज्~लृट्~\arrow त्यज्~त~\arrow \textcolor{red}{स्यतासी लृलुटोः} (पा॰सू॰~३.१.३३)~\arrow त्यज्~स्य~त~\arrow \textcolor{red}{चोः कुः} (पा॰सू॰~८.२.३०)~\arrow त्यक्~स्य~त~\arrow \textcolor{red}{आदेश\-प्रत्यययोः} (पा॰सू॰~८.३.५९)~\arrow त्यक्~ष्य~त~\arrow \textcolor{red}{टित आत्मनेपदानां टेरे} (पा॰सू॰~३.४.७९)~\arrow त्यक्ष्यते। सामान्यतस्तु \textcolor{red}{त्यक्ष्यति} इति रूपम्। यथा \textcolor{red}{सोऽपि शोकसमाविष्टो ननु त्यक्ष्यति जीवितम्} (वा॰रा॰~२.६६.११) इति वाल्मीकि\-प्रयोगे। त्यज्~\arrow \textcolor{red}{शेषात्कर्तरि परस्मैपदम्} (पा॰सू॰~१.३.७८)~\arrow \textcolor{red}{लृट् शेषे च} (पा॰सू॰~३.३.१३)~\arrow त्यज्~लृट्~\arrow त्यज्~तिप्~\arrow त्यज्~ति~\arrow \textcolor{red}{स्यतासी लृलुटोः} (पा॰सू॰~३.१.३३)~\arrow त्यज्~स्य~ति~\arrow \textcolor{red}{चोः कुः} (पा॰सू॰~८.२.३०)~\arrow तयक्~स्य~ति~\arrow \textcolor{red}{आदेश\-प्रत्यययोः} (पा॰सू॰~८.३.५९)~\arrow तयक्~ष्य~ति~\arrow त्यक्ष्यति।} यद्वा कर्म\-कर्तृक\-प्रयोगे \textcolor{red}{त्वामधिश्रित्य शोभा स्वयमेव न त्यक्ष्यते त्वया किम्} इत्यात्मनेपदम्।\footnote{\textcolor{red}{अधिश्रित्य} इत्यध्याहार्यमिति शेषः। त्यज्~\arrow \textcolor{red}{कर्मवत्कर्मणा तुल्यक्रियः} (पा॰सू॰~३.१.८७)~\arrow \textcolor{red}{भावकर्मणोः} (पा॰सू॰~१.३.१३)~\arrow \textcolor{red}{लृट् शेषे च} (पा॰सू॰~३.३.१३)~\arrow त्यज्~लृट्~\arrow त्यज्~त~\arrow \textcolor{red}{स्यतासी लृलुटोः} (पा॰सू॰~३.१.३३)~\arrow त्यज्~स्य~त~\arrow \textcolor{red}{चोः कुः} (पा॰सू॰~८.२.३०)~\arrow तयक्~स्य~त~\arrow \textcolor{red}{आदेश\-प्रत्यययोः} (पा॰सू॰~८.३.५९)~\arrow तयक्~ष्य~त~\arrow \textcolor{red}{टित आत्मनेपदानां टेरे} (पा॰सू॰~३.४.७९)~\arrow त्यक्ष्यते।}\end{sloppypar}
\vspace{2mm}
\centering ॥ इत्ययोध्याकाण्डीयप्रयोगाणां विमर्शः ॥\nopagebreak\\
\vspace{4mm}
\centering इत्यध्यात्म\-रामायणेऽपाणिनीय\-प्रयोगाणां\-विमर्श\-नामके शोध\-प्रबन्धे तृतीयाध्याये प्रथम\-परिच्छेदः।\\
\pagebreak
\pdfbookmark[1]{द्वितीयः परिच्छेदः}{Chap3Part2}
\phantomsection
\addtocontents{toc}{\protect\setcounter{tocdepth}{1}}
\addcontentsline{toc}{section}{द्वितीयः परिच्छेदः}
\addtocontents{toc}{\protect\setcounter{tocdepth}{0}}
\centering ॥ अथ तृतीयाध्याये द्वितीयः परिच्छेदः ॥\nopagebreak\\
\vspace{4mm}
\pdfbookmark[2]{अरण्यकाण्डम्}{Chap3Part2Kanda3}
\phantomsection
\addtocontents{toc}{\protect\setcounter{tocdepth}{2}}
\addcontentsline{toc}{subsection}{अरण्यकाण्डीयप्रयोगाणां विमर्शः}
\addtocontents{toc}{\protect\setcounter{tocdepth}{0}}
\centering ॥ अथारण्यकाण्डीयप्रयोगाणां विमर्शः ॥\nopagebreak\\
\section[गच्छामहे]{गच्छामहे}
\centering\textcolor{blue}{मुने गच्छामहे सर्वे मुनिमण्डलमण्डितम्।\nopagebreak\\
विपिनं दण्डकं यत्र त्वमाज्ञातुमिहार्हसि॥}\nopagebreak\\
\raggedleft{–~अ॰रा॰~३.१.२}\\
\fontsize{14}{21}\selectfont\begin{sloppypar}\hyphenrules{nohyphenation}\justifying\noindent\hspace{10mm} इह श्रीरामभद्रोऽत्रिमाज्ञां याचते। अत्र \textcolor{red}{गच्छामहे} इति प्रयोगोऽपि नाऽपाणिनीयः। यद्यपि परस्मैपदत्वादात्मनेपदं न \textcolor{red}{गच्छामः}\footnote{\textcolor{red}{गमॢँ गतौ} (धा॰पा॰~९८२)~\arrow गम्~\arrow \textcolor{red}{शेषात्कर्तरि परस्मैपदम्} (पा॰सू॰~१.३.७८)~\arrow \textcolor{red}{वर्तमान\-सामीप्ये वर्तमानवद्वा} (पा॰सू॰~३.३.१३१)~\arrow \textcolor{red}{वर्तमाने लट्} (पा॰सू॰~३.२.१२३)~\arrow गम्~लट्~\arrow गम्~मस्~\arrow \textcolor{red}{कर्तरि शप्‌} (पा॰सू॰~३.१.६८)~\arrow गम्~शप्~मस्~\arrow गम्~अ~मस्~\arrow \textcolor{red}{इषुगमियमां छः} (पा॰सू॰~७.३.७७)~\arrow गछ्~अ~मस्~\arrow \textcolor{red}{छे च} (पा॰सू॰~६.१.७३)~\arrow \textcolor{red}{आद्यन्तौ टकितौ} (पा॰सू॰~१.१.४६)~\arrow गतुँक्~छ्~अ~मस्~\arrow गत्~छ्~अ~मस्~\arrow \textcolor{red}{स्तोः श्चुना श्चुः} (पा॰सू॰~८.४.४०)~\arrow गच्~छ्~अ~मस्~\arrow \textcolor{red}{अतो दीर्घो यञि} (पा॰सू॰~७.३.१०१)~\arrow गच्~छ्~आ~मस्~\arrow \textcolor{red}{ससजुषो रुः} (पा॰सू॰~८.२.६६)~\arrow गच्~छ्~आ~मरुँ~\arrow \textcolor{red}{खरवसानयोर्विसर्जनीयः} (पा॰सू॰~८.३.१५)~\arrow गच्~छ्~आ~मः~\arrow गच्छामः।} इति सर्व\-विदितं तथाऽपि \textcolor{red}{सम्‌}\-उपसर्ग\-संयोजने \textcolor{red}{समो गम्यृच्छिभ्याम्} (पा॰सू॰~१.३.२९) इत्यनेनाऽत्मनेपदम्। लुप्तत्वात् समुपसर्गः न श्रूयते।\footnote{सम् \textcolor{red}{गमॢँ गतौ} (धा॰पा॰~९८२)~\arrow सम्~गम्~\arrow \textcolor{red}{समो गम्यृच्छिभ्याम्} (पा॰सू॰~१.३.२९)~\arrow \textcolor{red}{वर्तमान\-सामीप्ये वर्तमानवद्वा} (पा॰सू॰~३.३.१३१)~\arrow \textcolor{red}{वर्तमाने लट्} (पा॰सू॰~३.२.१२३)~\arrow सम्~गम्~लट्~\arrow सम्~गम्~महिङ्~\arrow सम्~गम्~महि~\arrow \textcolor{red}{कर्तरि शप्‌} (पा॰सू॰~३.१.६८)~\arrow सम्~गम्~शप्~महि~\arrow सम्~गम्~अ~महि~\arrow \textcolor{red}{इषुगमियमां छः} (पा॰सू॰~७.३.७७)~\arrow सम्~गछ्~अ~महि~\arrow \textcolor{red}{छे च} (पा॰सू॰~६.१.७३)~\arrow \textcolor{red}{आद्यन्तौ टकितौ} (पा॰सू॰~१.१.४६)~\arrow सम्~गतुँक्~छ्~अ~महि~\arrow सम्~गत्~छ्~अ~महि~\arrow \textcolor{red}{स्तोः श्चुना श्चुः} (पा॰सू॰~८.४.४०)~\arrow सम्~गच्~छ्~अ~महि~\arrow \textcolor{red}{अतो दीर्घो यञि} (पा॰सू॰~७.३.१०१)~\arrow सम्~गच्~छ्~आ~महि~\arrow \textcolor{red}{टित आत्मनेपदानां टेरे} (पा॰सू॰~३.४.७९)~\arrow सम्~गच्~छ्~आ~महे~\arrow \textcolor{red}{विनाऽपि प्रत्ययं पूर्वोत्तर\-पद\-लोपो वक्तव्यः} (वा॰~५.३.८३)~\arrow गच्~छ्~आ~महे~\arrow गच्छामहे।} लडपि \textcolor{red}{वर्तमान\-सामीप्ये वर्तमानवद्वा} (पा॰सू॰~३.३.१३१) इत्यनेन। \textcolor{red}{मुनि\-मण्डलमाश्रित्याति\-शीघ्रं सङ्गता भविष्यामः} इति रामभद्रस्य तात्पर्यम्।\footnote{यद्वा \textcolor{red}{कर्तरि कर्मव्यतिहारे} (पा॰सू॰~१.३.१४) इत्यनेनात्मने\-पदम्। \pageref{sec:yasye}तमे पृष्ठे \ref{sec:yasye} \nameref{sec:yasye} इति प्रयोगस्य विमर्शं पश्यन्तु।}\end{sloppypar}
\section[तारयिष्यामहे]{तारयिष्यामहे}
\centering\textcolor{blue}{तारयिष्यामहे युष्मान्वयमेव क्षणादिह।\nopagebreak\\
ततो नावि समारोप्य सीतां राघवलक्ष्मणौ॥}\nopagebreak\\
\raggedleft{–~अ॰रा॰~३.१.८}\\
\fontsize{14}{21}\selectfont\begin{sloppypar}\hyphenrules{nohyphenation}\justifying\noindent\hspace{10mm} अत्र कर्म\-व्यतिहार आत्मनेपदम्।\footnote{\textcolor{red}{कर्तरि कर्मव्यतिहारे} (पा॰सू॰~१.३.१४) इत्यनेन।} \textcolor{red}{निखिलं जगत्संसार\-सागराद्भवान् तारयति किन्तु भवन्तमपि वयं सरितस्तारयिष्यामहे} इति क्रिया\-व्यत्यय आत्मनेपदम्।\footnote{\textcolor{red}{तॄ प्लवनतरणयोः} (धा॰पा॰~९६९)~\arrow तॄ~\arrow \textcolor{red}{हेतुमति च} (पा॰सू॰~३.१.२६)~\arrow तॄ~णिच्~\arrow तॄ~इ~\arrow \textcolor{red}{अचो ञ्णिति} (पा॰सू॰~७.२.११५)~\arrow ता~इ~\arrow \textcolor{red}{उरण् रपरः} (पा॰सू॰~१.१.५१)~\arrow तार्~इ~\arrow तारि~\arrow \textcolor{red}{सनाद्यन्ता धातवः} (पा॰सू॰~३.१.३२)~\arrow धातु\-सञ्ज्ञा~\arrow \textcolor{red}{कर्तरि कर्मव्यतिहारे} (पा॰सू॰~१.३.१४)~\arrow \textcolor{red}{लृट् शेषे च} (पा॰सू॰~३.३.१३)~\arrow तारि~लृट्~\arrow तारि~महिङ्~\arrow तारि~महि~\arrow \textcolor{red}{स्यतासी लृलुटोः} (पा॰सू॰~३.१.३३)~\arrow तारि~स्य~महि~\arrow \textcolor{red}{आर्धधातुकस्येड्वलादेः} (पा॰सू॰~७.२.३५)~\arrow तारि~इट्~स्य~महि~\arrow तारि~इ~स्य~महि~\arrow \textcolor{red}{सार्वधातुकार्ध\-धातुकयोः} (पा॰सू॰~७.३.८४)~\arrow तारे~इ~स्य~महि~\arrow \textcolor{red}{एचोऽयवायावः} (पा॰सू॰~६.१.७८)~\arrow तारय्~इ~स्य~महि~\arrow \textcolor{red}{अतो दीर्घो यञि} (पा॰सू॰~७.३.१०१)~\arrow तारय्~इ~स्या~महि~\arrow \textcolor{red}{टित आत्मनेपदानां टेरे} (पा॰सू॰~३.४.७९)~\arrow तारय्~इ~स्या~महे~\arrow \textcolor{red}{आदेश\-प्रत्यययोः} (पा॰सू॰~८.३.५९)~\arrow तारय्~इ~ष्या~महे~\arrow तारयिष्यामहे।}\end{sloppypar}
\section[आस्ते]{आस्ते}
\centering\textcolor{blue}{इत्येवं भाषमाणौ तौ जग्मतुः सार्धयोजनम्।\nopagebreak\\
तत्रैका पुष्करिण्यास्ते कह्लारकुमुदोत्पलैः॥}\nopagebreak\\
\raggedleft{–~अ॰रा॰~३.१.१५}\\
\fontsize{14}{21}\selectfont\begin{sloppypar}\hyphenrules{nohyphenation}\justifying\noindent\hspace{10mm} \textcolor{red}{आस्त}\footnote{\textcolor{red}{आसँ उपवेशने} (धा॰पा॰~१०२१)~\arrow आस्~\arrow \textcolor{red}{अनुदात्तङित आत्मने\-पदम्} (पा॰सू॰~१.३.१२)~\arrow \textcolor{red}{अनद्यतने लङ्} (पा॰सू॰~३.२.१११)~\arrow आस्~लङ्~\arrow आस्~त~\arrow \textcolor{red}{आडजादीनाम्} (पा॰सू॰~६.४.७२)~\arrow \textcolor{red}{आद्यन्तौ टकितौ} (पा॰सू॰~१.१.४६)~\arrow आट्~आस्~त~\arrow आ~आस्~त~\arrow \textcolor{red}{कर्तरि शप्‌} (पा॰सू॰~३.१.६८)~\arrow आ~आस्~शप्~त~\arrow \textcolor{red}{अदिप्रभृतिभ्यः शपः} (पा॰सू॰~२.४.७२)~\arrow आ~आस्~त~\arrow \textcolor{red}{अकः सवर्णे दीर्घः} (पा॰सू॰~६.१.१०१)~\arrow आस्~त~\arrow आस्त।
} इति प्रयोक्तव्ये \textcolor{red}{स्म}\-योगे लड्लकारः।\footnote{\textcolor{red}{स्म} इत्यध्याहार्यमिति भावः। \textcolor{red}{आसँ उपवेशने} (धा॰पा॰~१०२१)~\arrow आस्~\arrow \textcolor{red}{अनुदात्तङित आत्मने\-पदम्} (पा॰सू॰~१.३.१२)~\arrow \textcolor{red}{लट् स्मे} (पा॰सू॰~३.२.११८)~\arrow आस्~लट्~\arrow आस्~त~\arrow \textcolor{red}{कर्तरि शप्‌} (पा॰सू॰~३.१.६८)~\arrow आस्~शप्~त~\arrow \textcolor{red}{अदिप्रभृतिभ्यः शपः} (पा॰सू॰~२.४.७२)~\arrow आस्~त~\arrow \textcolor{red}{टित आत्मनेपदानां टेरे} (पा॰सू॰~३.४.७९)~\arrow आस्~ते~\arrow आस्ते।
}\end{sloppypar}
\section[पलायतम्]{पलायतम्}
\centering\textcolor{blue}{यदि जीवितुमिच्छाऽस्ति त्यक्त्वा सीतां निरायुधौ।\nopagebreak\\
पलायतं न चेच्छीघ्रं भक्षयामि युवामहम्॥}\nopagebreak\\
\raggedleft{–~अ॰रा॰~३.१.२९}\\
\fontsize{14}{21}\selectfont\begin{sloppypar}\hyphenrules{nohyphenation}\justifying\noindent\hspace{10mm} अनुदात्तेत्त्व\-लक्षणस्याऽत्मनेपदस्यानित्यत्वात्परस्मैपदे लोड्लकारे मध्यम\-पुरुषे द्वि\-वचने थसस्तमादेशे \textcolor{red}{पलायतम्}।\footnote{\textcolor{red}{अनुदात्तेत्त्व\-लक्षणमात्मने\-पदमनित्यम्} (प॰शे॰~९३.४)। परा~\textcolor{red}{अयँ गतौ} (धा॰पा॰~४७४)~\arrow परा~अय्~\arrow \textcolor{red}{अनुदात्तेत्त्व\-लक्षणमात्मने\-पदमनित्यम्} (प॰शे॰~९३.४)~\arrow \textcolor{red}{शेषात्कर्तरि परस्मैपदम्} (पा॰सू॰~१.३.७८)~\arrow \textcolor{red}{लोट् च} (पा॰सू॰~३.३.१६२)~\arrow परा~अय्~लोट्~\arrow परा~अय्~थस्~\arrow \textcolor{red}{कर्तरि शप्‌} (पा॰सू॰~३.१.६८)~\arrow परा~अय्~शप्~तस्~\arrow परा~अय्~अ~तस्~\arrow \textcolor{red}{तस्थस्थमिपां तान्तन्तामः} (पा॰सू॰~३.४.१०१)~\arrow परा~अय्~अ~तम्~\arrow \textcolor{red}{उपसर्गस्यायतौ} (पा॰सू॰~८.२.१९)~\arrow पला~अय्~अ~तम्~\arrow \textcolor{red}{अकः सवर्णे दीर्घः} (पा॰सू॰~६.१.१०१)~\arrow पलाय्~अ~तम्~\arrow पलायतम्। सामान्यतस्तु \textcolor{red}{पलायेथाम्} इति रूपम्। यथा \textcolor{red}{त्वरमाणौ पलायेथां न वां जीवितमाददे} (वा॰रा॰~३.३.८) इति वाल्मीकि\-प्रयोगे। परा~\textcolor{red}{अयँ गतौ} (धा॰पा॰~४७४)~\arrow परा~अय्~\arrow \textcolor{red}{अनुदात्तङित आत्मने\-पदम्} (पा॰सू॰~१.३.१२)~\arrow \textcolor{red}{लोट् च} (पा॰सू॰~३.३.१६२)~\arrow परा~अय्~लोट्~\arrow परा~अय्~आथाम्~\arrow \textcolor{red}{कर्तरि शप्‌} (पा॰सू॰~३.१.६८)~\arrow परा~अय्~शप्~आथाम्~\arrow परा~अय्~अ~आथाम्~\arrow \textcolor{red}{लोटो लङ्वत्‌} (पा॰सू॰~३.४.८५)~\arrow ङिद्वत्त्वम्~\arrow \textcolor{red}{आतो ङितः} (पा॰सू॰~७.२.८१)~\arrow परा~अय्~अ~इय्~थाम्~\arrow \textcolor{red}{लोपो व्योर्वलि} (पा॰सू॰~६.१.६६)~\arrow परा~अय्~अ~इ~थाम्~\arrow \textcolor{red}{आद्गुणः} (पा॰सू॰~६.१.८७)~\arrow परा~अय्~ए~थाम्~\arrow \textcolor{red}{उपसर्गस्यायतौ} (पा॰सू॰~८.२.१९)~\arrow पला~अय्~ए~थाम्~\arrow \textcolor{red}{अकः सवर्णे दीर्घः} (पा॰सू॰~६.१.१०१)~\arrow पलाय्~ए~थाम्~\arrow पलायेथाम्।}\end{sloppypar}
\section[अभिदुद्रुवे]{अभिदुद्रुवे}
\centering\textcolor{blue}{इत्युक्त्वा राक्षसः सीतामादातुमभिदुद्रुवे।\nopagebreak\\
रामश्चिच्छेद तद्बाहू शरेण प्रहसन्निव॥}\nopagebreak\\
\raggedleft{–~अ॰रा॰~३.१.३०}\\
\fontsize{14}{21}\selectfont\begin{sloppypar}\hyphenrules{nohyphenation}\justifying\noindent\hspace{10mm} \textcolor{red}{अभिदुद्राव}\footnote{यथा \textcolor{red}{वालिपुत्रं महावीर्यमभिदुद्राव वीर्यवान्} (वा॰रा॰~६.७०.२) इति वाल्मीकि\-प्रयोगे। अभि~\textcolor{red}{द्रु गतौ} (धा॰पा॰~९४५)~\arrow अभि~द्रु~\arrow \textcolor{red}{शेषात्कर्तरि परस्मैपदम्} (पा॰सू॰~१.३.७८)~\arrow \textcolor{red}{परोक्षे लिट्} (पा॰सू॰~३.२.११५)~\arrow अभि~द्रु~लिट्~\arrow अभि~द्रु~तिप्~\arrow \textcolor{red}{परस्मैपदानां णलतुसुस्थलथुसणल्वमाः} (पा॰सू॰~३.४.८२)~\arrow अभि~द्रु~णल्~\arrow \textcolor{red}{लिटि धातोरनभ्यासस्य} (पा॰सू॰~६.१.८)~\arrow अभि~द्रु~द्रु~अ~\arrow \textcolor{red}{हलादिः शेषः} (पा॰सू॰~७.४.६०)~\arrow अभि~दु~द्रु~अ~\arrow \textcolor{red}{अचो ञ्णिति} (पा॰सू॰~७.२.११५)~\arrow अभि~दु~द्रौ~अ~\arrow \textcolor{red}{एचोऽयवायावः} (पा॰सू॰~६.१.७८)~\arrow अभि~दु~द्राव्~अ~\arrow अभिदुद्राव।} इति प्रयोक्तव्ये \textcolor{red}{अभिदुद्रुवे} इति प्रयुक्तम्। अत्रापि क्रिया\-विनिमय एव आत्मनेपदम्।\footnote{\textcolor{red}{कर्तरि कर्म\-व्यतिहारे} (पा॰सू॰~१.३.१४) इत्यनेन। अभि \textcolor{red}{द्रु गतौ} (धा॰पा॰~९४५)~\arrow अभि~द्रु~\arrow \textcolor{red}{कर्तरि कर्म\-व्यतिहारे} (पा॰सू॰~१.३.१४)~\arrow \textcolor{red}{परोक्षे लिट्} (पा॰सू॰~३.२.११५)~\arrow अभि~द्रु~लिट्~\arrow अभि~द्रु~त~\arrow \textcolor{red}{लिटस्तझयोरेशिरेच्} (पा॰सू॰~३.४.८१)~\arrow अभि~द्रु~एश्~\arrow अभि~द्रु~ए~\arrow \textcolor{red}{लिटि धातोरनभ्यासस्य} (पा॰सू॰~६.१.८)~\arrow अभि~द्रु~द्रु~ए~\arrow \textcolor{red}{हलादिः शेषः} (पा॰सू॰~७.४.६०)~\arrow अभि~दु~द्रु~ए~\arrow \textcolor{red}{अचि श्नुधातुभ्रुवां य्वोरियङुवङौ} (पा॰सू॰~६.४.७७)~\arrow \textcolor{red}{ङिच्च} (पा॰सू॰~१.१.५३)~\arrow अभि~दु~द्रुवँङ्~ए~\arrow अभि~दु~द्रुव्~ए~\arrow अभिदुद्रुवे। \textcolor{red}{सार्वधातुकमपित्} (पा॰सू॰~१.२.४) इत्यनेन ङिद्वत्त्वम् \textcolor{red}{ग्क्ङिति च} (पा॰सू॰~१.१.५) इत्यनेन च गुणनिषेधः।} रामादृते स्पर्शस्य कस्यचिदप्यनधिकारत्वात्।\end{sloppypar}
\section[उपवेशयत्]{उपवेशयत्}
\centering\textcolor{blue}{अभिगम्य सुसम्पूज्य विष्टरेषूपवेशयत्।\nopagebreak\\
आतिथ्यमकरोत्तेषां कन्दमूलफलादिभिः॥}\nopagebreak\\
\raggedleft{–~अ॰रा॰~३.२.३}\\
\fontsize{14}{21}\selectfont\begin{sloppypar}\hyphenrules{nohyphenation}\justifying\noindent\hspace{10mm} \textcolor{red}{विश्‌}\-धातुः (\textcolor{red}{विशँ प्रवेशने} धा॰पा॰~१४२४) अत्र णिजन्तः। तस्य \textcolor{red}{उप}\-उपसर्गेण संयोजनम्। \textcolor{red}{पूर्वं धातुः साधनेन युज्यते पश्चादुपसर्गेण} (भा॰पा॰सू॰~२.२.१९, ६.१.१३५, ८.१.७०) इति नियमेनाडागम \textcolor{red}{उप}\-उपसर्ग\-संयोजने \textcolor{red}{उपावेशयत्}।\footnote{विश्~\arrow \textcolor{red}{हेतुमति च} (पा॰सू॰~३.१.२६)~\arrow विश्~णिच्~\arrow विश्~इ~\arrow \textcolor{red}{पुगन्त\-लघूपधस्य च} (पा॰सू॰~७.३.८६)~\arrow वेश्~इ~\arrow वेशि~\arrow \textcolor{red}{सनाद्यन्ता धातवः} (पा॰सू॰~३.१.३२)~\arrow धातु\-सञ्ज्ञा। उप~वेशि~\arrow \textcolor{red}{शेषात्कर्तरि परस्मैपदम्} (पा॰सू॰~१.३.७८)~\arrow \textcolor{red}{अनद्यतने लङ्} (पा॰सू॰~३.२.१११)~\arrow उप~वेशि~लङ्~\arrow \textcolor{red}{लुङ्लङ्लृङ्क्ष्वडुदात्तः} (पा॰सू॰~६.४.७१)~\arrow \textcolor{red}{आद्यन्तौ टकितौ} (पा॰सू॰~१.१.४६)~\arrow उप~अट्~वेशि~लङ्~\arrow उप~अ~वेशि~लङ्~\arrow उप~अ~वेशि~तिप्~\arrow उप~अ~वेशि~ति~\arrow \textcolor{red}{कर्तरि शप्‌} (पा॰सू॰~३.१.६८)~\arrow उप~अ~वेशि~शप्~ति~\arrow उप~अ~वेशि~अ~ति~\arrow \textcolor{red}{सार्वधातुकार्ध\-धातुकयोः} (पा॰सू॰~७.३.८४)~\arrow उप~अ~वेशे~अ~ति~\arrow \textcolor{red}{एचोऽयवायावः} (पा॰सू॰~६.१.७८)~\arrow उप~अ~वेशय्~अ~ति~\arrow \textcolor{red}{इतश्च} (पा॰सू॰~३.४.१००)~\arrow उप~अ~वेशय्~अ~त्~\arrow \textcolor{red}{अकः सवर्णे दीर्घः} (पा॰सू॰~६.१.१०१)~\arrow उपा~वेशय्~अ~त्~\arrow उपावेशयत्।} अत्र \textcolor{red}{विनाऽपि प्रत्ययं पूर्वोत्तर\-पद\-लोपो वक्तव्यः} (वा॰~५.३.८३) इत्यनेनाडागमस्य लोपः। आगम\-कार्यस्यानित्यत्वाद्वा।\footnote{\textcolor{red}{आगम\-शास्त्रमनित्यम्} (प॰शे॰~९३.२)।}\end{sloppypar}
\section[व्रजामि]{व्रजामि}
\centering\textcolor{blue}{अद्य मत्तपसा सिद्धं यत्पुण्यं बहु विद्यते।\nopagebreak\\
तत्सर्वं तव दास्यामि ततो मुक्तिं व्रजाम्यहं॥}\nopagebreak\\
\raggedleft{–~अ॰रा॰~३.२.५}\\
\fontsize{14}{21}\selectfont\begin{sloppypar}\hyphenrules{nohyphenation}\justifying\noindent\hspace{10mm} वर्तमान\-सामीप्याल्लट्।\footnote{\textcolor{red}{वर्तमान\-सामीप्ये वर्तमानवद्वा} (पा॰सू॰~३.३.१३१) इत्यनेन। \textcolor{red}{व्रजँ गतौ} (धा॰पा॰~२५३)~\arrow व्रज्~\arrow \textcolor{red}{शेषात्कर्तरि परस्मैपदम्} (पा॰सू॰~१.३.७८)~\arrow \textcolor{red}{वर्तमान\-सामीप्ये वर्तमानवद्वा} (पा॰सू॰~३.३.१३१)~\arrow \textcolor{red}{वर्तमाने लट्} (पा॰सू॰~३.२.१२३)~\arrow व्रज्~लट्~\arrow व्रज्~मिप्~\arrow व्रज्~मि~\arrow \textcolor{red}{कर्तरि शप्‌} (पा॰सू॰~३.१.६८)~\arrow व्रज्~शप्~मि~\arrow व्रज्~अ~मि~\arrow \textcolor{red}{अतो दीर्घो यञि} (पा॰सू॰~७.३.१०१)~\arrow व्रज्~आ~मि~\arrow व्रजामि।} \textcolor{red}{शीघ्रं मुक्तिं व्रजिष्यामि}\footnote{\textcolor{red}{व्रजँ गतौ} (धा॰पा॰~२५३)~\arrow व्रज्~\arrow \textcolor{red}{शेषात्कर्तरि परस्मैपदम्} (पा॰सू॰~१.३.७८)~\arrow \textcolor{red}{लृट् शेषे च} (पा॰सू॰~३.३.१३)~\arrow व्रज्~लृट्~\arrow व्रज्~मिप्~\arrow व्रज्~मि~\arrow \textcolor{red}{स्यतासी लृलुटोः} (पा॰सू॰~३.१.३३)~\arrow \textcolor{red}{वर्तमान\-सामीप्ये वर्तमानवद्वा} (पा॰सू॰~३.३.१३१)~\arrow व्रज्~स्य~मि~\arrow \textcolor{red}{आर्धधातुकस्येड्वलादेः} (पा॰सू॰~७.२.३५)~\arrow व्रज्~इट्~स्य~मि~\arrow व्रज्~इ~स्य~मि~\arrow \textcolor{red}{अतो दीर्घो यञि} (पा॰सू॰~७.३.१०१)~\arrow व्रज्~इ~स्या~मि~\arrow \textcolor{red}{आदेश\-प्रत्यययोः} (पा॰सू॰~८.३.५९)~\arrow व्रज्~इ~ष्या~मि~\arrow व्रजिष्यामि।} इति शरभङ्गस्य तात्पर्यम्।\end{sloppypar}
\section[बभूवुः]{बभूवुः}
\centering\textcolor{blue}{अद्य मे क्रतवः सर्वे बभूवुः सफलाः प्रभो।\nopagebreak\\
दीर्घकालं मया तप्तमनन्यमतिना तपः।\nopagebreak\\
तस्येह तपसो राम फलं तव यदर्चनम्॥}\nopagebreak\\
\raggedleft{–~अ॰रा॰~३.३.४३}\\
\fontsize{14}{21}\selectfont\begin{sloppypar}\hyphenrules{nohyphenation}\justifying\noindent\hspace{10mm} शरभङ्गः प्रेमविह्वलः श्रीरामं दृष्ट्वा कथयति \textcolor{red}{अद्य मे क्रतवः सफला अभूवन्}। \textcolor{red}{अभूवन्}\footnote{\textcolor{red}{भू सत्तायाम्} (धा॰पा॰~१)~\arrow भू~\arrow \textcolor{red}{शेषात्कर्तरि परस्मैपदम्} (पा॰सू॰~१.३.७८)~\arrow \textcolor{red}{लुङ्} (पा॰सू॰~३.२.११०)~\arrow भू~लङ्~\arrow भू~झि~\arrow \textcolor{red}{लुङ्लङ्लृङ्क्ष्वडुदात्तः} (पा॰सू॰~६.४.७१)~\arrow \textcolor{red}{आद्यन्तौ टकितौ} (पा॰सू॰~१.१.४६)~\arrow अट्~भू~झि~\arrow अ~भू~झि~\arrow \textcolor{red}{च्लि लुङि} (पा॰सू॰~३.१.४३)~\arrow अ~भू~च्लि~झि~\arrow \textcolor{red}{च्लेः सिच्} (पा॰सू॰~३.१.४४)~\arrow अ~भू~सिच्~झि~\arrow \textcolor{red}{गातिस्था\-घुपाभूभ्यः सिचः परस्मैपदेषु} (पा॰सू॰~२.४.७७)~\arrow अ~भू~झि~\arrow \textcolor{red}{झोऽन्तः} (पा॰सू॰~७.१.३)~\arrow अ~भू~अन्ति~\arrow \textcolor{red}{भुवो वुग्लुङ्लिटोः} (पा॰सू॰~६.४.८८)~\arrow अ~भू~वुँक्~अन्ति~\arrow अ~भू~व्~अन्ति~\arrow \textcolor{red}{इतश्च} (पा॰सू॰~३.४.१००)~\arrow अ~भू~व्~अन्त्~\arrow \textcolor{red}{संयोगान्तस्य लोपः} (पा॰सू॰~८.२.२३)~\arrow अ~भू~व्~अन्~\arrow अभूवन्।} इति प्रयोक्तव्ये \textcolor{red}{बभूवुः} इति परोक्ष\-प्रयोगोऽकार्यः। तथा हि \textcolor{red}{भू}\-धातोः (\textcolor{red}{भू सत्तायाम्} धा॰पा॰~१) \textcolor{red}{भूवादयो धातवः} (पा॰सू॰~१.३.१) इत्यनेन धातु\-सञ्ज्ञायां \textcolor{red}{परोक्षे लिट्} (पा॰सू॰~३.२.११५) इत्यनेन लिड्लकारे \textcolor{red}{तिप्तस्झि\-सिप्थस्थ\-मिब्वस्मस्ताताञ्झ\-थासाथान्ध्वमिड्वहिमहिङ्} (पा॰सू॰~३.४.७८) इत्यनेन प्रथम\-पुरुष\-बहु\-वचने \textcolor{red}{झि}\-प्रत्यये \textcolor{red}{परस्मैपदानां णलतुसुस्थलथुस\-णल्वमाः} (पा॰सू॰~३.४.८२) इत्यनेनोसादेशे \textcolor{red}{भुवो वुग्लुङ्लिटोः} (पा॰सू॰~६.४.८८) इत्यनेन वुगागमे \textcolor{red}{लिटि धातोरनभ्यासस्य} (पा॰सू॰~६.१.८) इत्यनेन द्वित्वे \textcolor{red}{पूर्वोऽभ्यासः} (पा॰सू॰~६.१.४) इत्यनेनाभ्यास\-सञ्ज्ञायां \textcolor{red}{हलादिः शेषः} (पा॰सू॰~७.४.६०) इत्यनेनाऽदिहल्शेषे \textcolor{red}{ह्रस्वः} (पा॰सू॰~७.४.५९) इत्यनेन ह्रस्वे \textcolor{red}{भवतेरः} (पा॰सू॰~७.४.७३) इत्यनेनोकारस्याकारे \textcolor{red}{अभ्यासे चर्च} (पा॰सू॰~८.४.५४) इत्यनेन जश्त्वे रुत्वे विसर्गे च \textcolor{red}{बभूवुः}।\footnote{\textcolor{red}{भू सत्तायाम्} (धा॰पा॰~१)~\arrow भू~\arrow \textcolor{red}{शेषात्कर्तरि परस्मैपदम्} (पा॰सू॰~१.३.७८)~\arrow \textcolor{red}{परोक्षे लिट्} (पा॰सू॰~३.२.११५)~\arrow (पा॰सू॰~३.२.११५)~\arrow भू~लिट्~\arrow भू~तिप्~\arrow \textcolor{red}{परस्मैपदानां णलतुसुस्थलथुस\-णल्वमाः} (पा॰सू॰~३.४.८२)~\arrow भू~उस्~\arrow \textcolor{red}{भुवो वुग्लुङ्लिटोः} (पा॰सू॰~६.४.८८)~\arrow \textcolor{red}{आद्यन्तौ टकितौ} (पा॰सू॰~१.१.४६)~\arrow भूवुँक्~अ~\arrow भूव्~उस्~\arrow \textcolor{red}{लिटि धातोरनभ्यासस्य} (पा॰सू॰~६.१.८)~\arrow भूव्~भूव्~उस्~\arrow \textcolor{red}{हलादिः शेषः} (पा॰सू॰~७.४.६०)~\arrow भू~भूव्~उस्~\arrow \textcolor{red}{ह्रस्वः} (पा॰सू॰~७.४.५९)~\arrow भु~भूव्~उस्~\arrow \textcolor{red}{भवतेरः} (पा॰सू॰~७.४.७३)~\arrow भ~भूव्~उस्~\arrow \textcolor{red}{अभ्यासे चर्च} (पा॰सू॰~८.४.५४)~\arrow ब~भूव्~उस्~\arrow बभूवुस्~\arrow \textcolor{red}{ससजुषो रुः} (पा॰सू॰~८.२.६६)~\arrow रुत्वम्~\arrow बभूवुरुँ~\arrow अनुबन्ध\-लोपः~\arrow बभूवुर्~\arrow \textcolor{red}{खरवसानयोर्विसर्जनीयः} (पा॰सू॰~८.३.१५)~\arrow बभूवुः।} प्रेम\-विह्वलतया भगवतश्चाक्षुष\-प्रत्यक्षे वा षट्सु प्रत्यक्षेषु सकल\-विस्मृततया पारोक्ष्याल्लिट्प्रयोगः।\end{sloppypar}
\section[वत्स्ये]{वत्स्ये}
\centering\textcolor{blue}{पञ्चवट्यामहं वत्स्ये तवैव प्रियकाम्यया।\nopagebreak\\
मृगयायां कदाचित्तु प्रयाते लक्ष्मणेऽपि च॥}\nopagebreak\\
\raggedleft{–~अ॰रा॰~३.४.५}\\
\fontsize{14}{21}\selectfont\begin{sloppypar}\hyphenrules{nohyphenation}\justifying\noindent\hspace{10mm} अत्र क्रिया\-विनिमय आत्मनेपदम्।\footnote{\textcolor{red}{कर्तरि कर्म\-व्यतिहारे} (पा॰सू॰~१.३.१४) इत्यनेन। \textcolor{red}{वसँ निवासे} (धा॰पा॰~१००४)~\arrow वस्~\arrow \textcolor{red}{कर्तरि कर्म\-व्यतिहारे} (पा॰सू॰~१.३.१४)~\arrow \textcolor{red}{लृट् शेषे च} (पा॰सू॰~३.३.१३)~\arrow वस्~लृट्~\arrow वस्~इट्~\arrow वस्~इ~\arrow \textcolor{red}{स्यतासी लृलुटोः} (पा॰सू॰~३.१.३३)~\arrow वस्~स्य~इ~\arrow \textcolor{red}{सः स्यार्धधातुके} (पा॰सू॰~७.४.४९)~\arrow वत्~स्य~इ~\arrow \textcolor{red}{टित आत्मनेपदानां टेरे} (पा॰सू॰~३.४.७९)~\arrow वत्~स्य~ए~\arrow \textcolor{red}{अतो गुणे} (पा॰सू॰~६.१.९७)~\arrow वत्~स्ये~\arrow वत्स्ये। सामान्यतस्तु \textcolor{red}{वत्स्यामि} इति रूपम्। यथा \textcolor{red}{चतुर्दश हि वर्षाणि वत्स्यामि विजने वने} (वा॰रा॰~२.२०.२९) इति वाल्मीकि\-प्रयोगे। \textcolor{red}{वसँ निवासे} (धा॰पा॰~१००४)~\arrow वस्~\arrow \textcolor{red}{शेषात्कर्तरि परस्मैपदम्} (पा॰सू॰~१.३.७८)~\arrow \textcolor{red}{लृट् शेषे च} (पा॰सू॰~३.३.१३)~\arrow वस्~लृट्~\arrow वस्~मिप्~\arrow वस्~मि~\arrow \textcolor{red}{स्यतासी लृलुटोः} (पा॰सू॰~३.१.३३)~\arrow वस्~स्य~मि~\arrow \textcolor{red}{सः स्यार्धधातुके} (पा॰सू॰~७.४.४९)~\arrow वत्~स्य~मि~\arrow \textcolor{red}{अतो दीर्घो यञि} (पा॰सू॰~७.३.१०१)~\arrow वत्~स्या~मि~\arrow वत्स्यामि।}\end{sloppypar}
\section[जागर्ति]{जागर्ति}
\centering\textcolor{blue}{आनीय प्रददौ रामसेवातत्परमानसः।\nopagebreak\\
धनुर्बाणधरो नित्यं रात्रौ जागर्ति सर्वतः॥}\nopagebreak\\
\raggedleft{–~अ॰रा॰~३.४.१३}\\
\fontsize{14}{21}\selectfont\begin{sloppypar}\hyphenrules{nohyphenation}\justifying\noindent\hspace{10mm} \textcolor{red}{स्म}\-योगे लट्।\footnote{\textcolor{red}{स्म} इत्यध्याहार्यमिति भावः। \textcolor{red}{जागृ निद्राक्षये} (धा॰पा॰~१०७२)~\arrow जागृ~\arrow \textcolor{red}{शेषात्कर्तरि परस्मैपदम्} (पा॰सू॰~१.३.७८)~\arrow \textcolor{red}{लट् स्मे} (पा॰सू॰~३.२.११८)~\arrow जागृ~लट्~\arrow जागृ~तिप्~\arrow जागृ~ति~\arrow \textcolor{red}{कर्तरि शप्‌} (पा॰सू॰~३.१.६८)~\arrow जागृ~शप्~ति~\arrow जागृ~अ~ति~\arrow \textcolor{red}{सार्वधातुकार्ध\-धातुकयोः} (पा॰सू॰~७.३.८४)~\arrow \textcolor{red}{उरण् रपरः} (पा॰सू॰~१.१.५१)~\arrow जागर्~अ~ति~\arrow \textcolor{red}{अदिप्रभृतिभ्यः शपः} (पा॰सू॰~२.४.७२)~\arrow जागर्~ति~\arrow जागर्ति।}\end{sloppypar}
\section[सेवते]{सेवते}
\centering\textcolor{blue}{आनीय सलिलं नित्यं लक्ष्मणः प्रीतमानसः।\nopagebreak\\
सेवतेऽहरहः प्रीत्या एवमासन् सुखं त्रयः॥}\nopagebreak\\
\raggedleft{–~अ॰रा॰~३.४.१५}\\
\fontsize{14}{21}\selectfont\begin{sloppypar}\hyphenrules{nohyphenation}\justifying\noindent\hspace{10mm} \textcolor{red}{स्म}\-योगे लट्।\footnote{\textcolor{red}{स्म} इत्यध्याहार्यमिति भावः। \textcolor{red}{षेवृँ सेवने} (धा॰पा॰~५०१)~\arrow षेव्~\arrow \textcolor{red}{धात्वादेः षः सः} (पा॰सू॰~६.१.६४)~\arrow सेव्~\arrow \textcolor{red}{अनुदात्तङित आत्मने\-पदम्} (पा॰सू॰~१.३.१२)~\arrow \textcolor{red}{लट् स्मे} (पा॰सू॰~३.२.११८)~\arrow सेव्~लट्~\arrow सेव्~त~\arrow \textcolor{red}{कर्तरि शप्‌} (पा॰सू॰~३.१.६८)~\arrow सेव्~शप्~त~\arrow सेव्~अ~त~\arrow \textcolor{red}{टित आत्मनेपदानां टेरे} (पा॰सू॰~३.४.७९)~\arrow सेव्~अ~ते~\arrow सेवते।}\end{sloppypar}
\section[सङ्गच्छावः]{सङ्गच्छावः}
\centering\textcolor{blue}{भ्रातुराज्ञां पुरस्कृत्य सङ्गच्छावोऽद्य मा चिरम्।\nopagebreak\\
इत्याह राक्षसी घोरा लक्ष्मणं काममोहिता॥}\nopagebreak\\
\raggedleft{–~अ॰रा॰~३.५.१५}\\
\fontsize{14}{21}\selectfont\begin{sloppypar}\hyphenrules{nohyphenation}\justifying\noindent\hspace{10mm} इह \textcolor{red}{समो गम्यृच्छिभ्याम्} (पा॰सू॰~१.३.२९) इत्यनेनाऽत्मनेपदत्वात् \textcolor{red}{सङ्गच्छावहे}।\footnote{सम् \textcolor{red}{गमॢँ गतौ} (धा॰पा॰~९८२)~\arrow सम्~गम्~\arrow \textcolor{red}{समो गम्यृच्छिभ्याम्} (पा॰सू॰~१.३.२९)~\arrow \textcolor{red}{वर्तमाने लट्} (पा॰सू॰~३.२.१२३)~\arrow सम्~गम्~लट्~\arrow सम्~गम्~वहिङ्~\arrow सम्~गम्~वहि~\arrow \textcolor{red}{कर्तरि शप्‌} (पा॰सू॰~३.१.६८)~\arrow सम्~गम्~शप्~वहि~\arrow सम्~गम्~अ~वहि~\arrow \textcolor{red}{इषुगमियमां छः} (पा॰सू॰~७.३.७७)~\arrow सम्~गछ्~अ~वहि~\arrow \textcolor{red}{छे च} (पा॰सू॰~६.१.७३)~\arrow \textcolor{red}{आद्यन्तौ टकितौ} (पा॰सू॰~१.१.४६)~\arrow सम्~गतुँक्~छ्~अ~वहि~\arrow सम्~गत्~छ्~अ~वहि~\arrow \textcolor{red}{स्तोः श्चुना श्चुः} (पा॰सू॰~८.४.४०)~\arrow सम्~गच्~छ्~अ~वहि~\arrow \textcolor{red}{अतो दीर्घो यञि} (पा॰सू॰~७.३.१०१)~\arrow सम्~गच्~छ्~आ~वहि~\arrow \textcolor{red}{टित आत्मनेपदानां टेरे} (पा॰सू॰~३.४.७९)~\arrow सम्~गच्~छ्~आ~वहे~\arrow \textcolor{red}{मोऽनुस्वारः} (पा॰सू॰~८.३.२३)~\arrow सं~गच्~छ्~आ~वहे~\arrow \textcolor{red}{वा पदान्तस्य} (पा॰सू॰~८.४.५९)~\arrow सङ्~गच्~छ्~आ~वहे~\arrow सङ्गच्छावहे।} किन्तु अत्र \textcolor{red}{शम्} इति पृथक्पदम्। अर्थात् \textcolor{red}{शं शान्तिं गच्छावः} इति तात्पर्यम्।\footnote{\textcolor{red}{गमॢँ गतौ} (धा॰पा॰~९८२)~\arrow गम्~\arrow \textcolor{red}{शेषात्कर्तरि परस्मैपदम्} (पा॰सू॰~१.३.७८)~\arrow \textcolor{red}{वर्तमाने लट्} (पा॰सू॰~३.२.१२३)~\arrow गम्~लट्~\arrow गम्~वस्~\arrow \textcolor{red}{कर्तरि शप्‌} (पा॰सू॰~३.१.६८)~\arrow गम्~शप्~वस्~\arrow गम्~अ~वस्~\arrow \textcolor{red}{इषुगमियमां छः} (पा॰सू॰~७.३.७७)~\arrow गछ्~अ~वस्~\arrow \textcolor{red}{छे च} (पा॰सू॰~६.१.७३)~\arrow \textcolor{red}{आद्यन्तौ टकितौ} (पा॰सू॰~१.१.४६)~\arrow गतुँक्~छ्~अ~वस्~\arrow गत्~छ्~अ~वस्~\arrow \textcolor{red}{स्तोः श्चुना श्चुः} (पा॰सू॰~८.४.४०)~\arrow गच्~छ्~अ~वस्~\arrow \textcolor{red}{अतो दीर्घो यञि} (पा॰सू॰~७.३.१०१)~\arrow गच्~छ्~आ~वस्~\arrow \textcolor{red}{ससजुषो रुः} (पा॰सू॰~८.२.६६)~\arrow गच्~छ्~आ~वरुँ~\arrow \textcolor{red}{खरवसानयोर्विसर्जनीयः} (पा॰सू॰~८.३.१५)~\arrow गच्~छ्~आ~वः~\arrow गच्छावः।} तालव्य\-शकारो लेखनप्रमादान्मुद्रण\-प्रमादाच्च दन्त्यो लिखितो हस्तलेखेषु पुस्तकेषु च।\end{sloppypar}
\section[अनुधावति]{अनुधावति}
\centering\textcolor{blue}{इत्युक्त्वा विकटाकरा जानकीमनुधावति।\nopagebreak\\
ततो रामाज्ञया खड्गमादाय परिगृह्य ताम्॥}\nopagebreak\\
\raggedleft{–~अ॰रा॰~३.५.१९}\\
\fontsize{14}{21}\selectfont\begin{sloppypar}\hyphenrules{nohyphenation}\justifying\noindent\hspace{10mm} इहापि \textcolor{red}{स्म}\-योगे लट्।\footnote{\textcolor{red}{स्म} इत्यध्याहार्यमिति भावः। अनु~\textcolor{red}{धावुँ गतिशुद्धयोः} (धा॰पा॰~६०१)~\arrow अनु~धाव्~\arrow \textcolor{red}{शेषात्कर्तरि परस्मैपदम्} (पा॰सू॰~१.३.७८)~\arrow \textcolor{red}{लट् स्मे} (पा॰सू॰~३.२.११८)~\arrow अनु~धाव्~लट्~\arrow अनु~धाव्~तिप्~\arrow अनु~धाव्~ति~\arrow \textcolor{red}{कर्तरि शप्‌} (पा॰सू॰~३.१.६८)~\arrow अनु~धाव्~शप्~ति~\arrow अनु~धाव्~अ~ति~\arrow अनु~धावति।} एवमेवैतादृशेषु सर्वेषु प्रयोगेषूह्यम्।\end{sloppypar}
\section[पास्ये]{पास्ये}
\centering\textcolor{blue}{तयोस्तु रुधिरं पास्ये भक्षयैतौ सुदुर्मदौ।\nopagebreak\\
नो चेत्प्राणान्परित्यज्य यास्यामि यमसादनम्॥}\nopagebreak\\
\raggedleft{–~अ॰रा॰~३.५.२५}\\
\fontsize{14}{21}\selectfont\begin{sloppypar}\hyphenrules{nohyphenation}\justifying\noindent\hspace{10mm} अत्र कर्म\-विनिमयादात्मने\-पदम्।\footnote{\textcolor{red}{पा पाने} (धा॰पा॰~९२५)~\arrow \textcolor{red}{कर्तरि कर्म\-व्यतिहारे} (पा॰सू॰~१.३.१४)~\arrow \textcolor{red}{लृट् शेषे च} (पा॰सू॰~३.३.१३)~\arrow पा~लृँट्~\arrow पा~इट्~\arrow पा~इ~\arrow \textcolor{red}{स्यतासी लृलुटोः} (पा॰सू॰~३.१.३३)~\arrow पा~स्य~इ~\arrow \textcolor{red}{टित आत्मनेपदानां टेरे} (पा॰सू॰~३.४.७९)~\arrow पा~स्य~ए~\arrow \textcolor{red}{अतो गुणे} (पा॰सू॰~६.१.९७)~\arrow पा~स्ये~\arrow पास्ये।} एवमन्यत्रापि विभाव्यम्।\end{sloppypar}
\section[आनयिष्यामि]{आनयिष्यामि}
\centering\textcolor{blue}{अतस्त्वया सहायेन गत्वा तत्प्राणवल्लभाम्।\nopagebreak\\
आनयिष्यामि विपिने रहिते राघवेण ताम्॥}\nopagebreak\\
\raggedleft{–~अ॰रा॰~३.६.१२}\\
\centering\textcolor{blue}{यदि मारीच एवायं तदा हन्मि न संशयः।\nopagebreak\\
मृगश्चेदानयिष्यामि सीताविश्रमहेतवे॥\\
गमिष्यामि मृगं बद्ध्वा ह्यानयिष्यामि सत्वरः।\nopagebreak\\
त्वं प्रयत्नेन सन्तिष्ठ सीतासंरक्षणोद्यतः॥}\nopagebreak\\
\raggedleft{–~अ॰रा॰~३.७.१०–११}\\
\fontsize{14}{21}\selectfont\begin{sloppypar}\hyphenrules{nohyphenation}\justifying\noindent\hspace{10mm} \textcolor{red}{नी}\-धातोः (\textcolor{red}{णीञ् प्रापणे} धा॰पा॰~९०१) अनिट्त्वाल्लृड्लकार उत्तम\-पुरुषैक\-वचने \textcolor{red}{सार्वधातुकार्धधातुकयोः} (पा॰सू॰~७.३.८४) इत्यनेन गुणे \textcolor{red}{अतो दीर्घो यञि} (पा॰सू॰~७.३.१०१) इत्यनेन दीर्घे \textcolor{red}{आदेश\-प्रत्ययोः} (पा॰सू॰~८.३.५९) इत्यनेन षत्वे \textcolor{red}{आनेष्यामि} इति पाणिनीयम्।\footnote{आ~णीञ्~\arrow णी~\arrow \textcolor{red}{णो नः} (पा॰सू॰~६.१.६५)~\arrow आ~नी~\arrow \textcolor{red}{शेषात्कर्तरि परस्मैपदम्} (पा॰सू॰~१.३.७८)~\arrow \textcolor{red}{लृट् शेषे च} (पा॰सू॰~३.३.१३)~\arrow आ~नी~लृँट्~\arrow आ~नी~मिप्~\arrow आ~नी~मि~\arrow \textcolor{red}{स्यतासी लृलुटोः} (पा॰सू॰~३.१.३३)~\arrow आ~नी~स्य~मि~\arrow \textcolor{red}{सार्वधातुकार्ध\-धातुकयोः} (पा॰सू॰~७.३.८४)~\arrow आ~ने~स्य~मि~\arrow \textcolor{red}{अतो दीर्घो यञि} (पा॰सू॰~७.३.१०१)~\arrow आ~ने~स्या~मि~\arrow \textcolor{red}{आदेश\-प्रत्यययोः} (पा॰सू॰~८.३.५९)~\arrow आ~ने~ष्या~मि~\arrow आनेष्यामि।} परम् \textcolor{red}{आनीयत इत्यानयः}।\footnote{\textcolor{red}{एरच्} (पा॰सू॰~३.३.५६) इत्यनेन भावेऽचि। आङ्~\textcolor{red}{णीञ् प्रापणे} धा॰पा॰~९०१)~\arrow आ~णी~\arrow \textcolor{red}{णो नः} (पा॰सू॰~६.१.६५)~\arrow आ~नी~\arrow \textcolor{red}{एरच्} (पा॰सू॰~३.३.५६)~\arrow आ~नी~अच्~\arrow आ~नी~अ~\arrow \textcolor{red}{सार्वधातुकार्ध\-धातुकयोः} (पा॰सू॰~७.३.८४)~\arrow आ~ने~अच्~\arrow \textcolor{red}{एचोऽयवायावः} (पा॰सू॰~६.१.७८)~\arrow आ~नय्~अ~\arrow आनय~\arrow विभक्तिकार्यम्~\arrow आनयः।} \textcolor{red}{आनयमाचरिष्याम्यानयिष्यामि} इत्यानय\-धातोर्लृटीट्सम्भवः।\footnote{प्रयोगस्यास्य सिद्धिः \textcolor{red}{कृष्णिष्यति} (बा॰म॰~२६६५) इतिवत्। आनय~\arrow \textcolor{red}{सर्वप्राति\-पदिकेभ्य आचारे क्विब्वा वक्तव्यः} (वा॰~३.१.११)~\arrow आनय~क्विँप्~\arrow आनय~व्~\arrow \textcolor{red}{वेरपृक्तस्य} (पा॰सू॰~६.१.६७)~\arrow आनय~\arrow \textcolor{red}{सनाद्यन्ता धातवः} (पा॰सू॰~३.१.३२)~\arrow \textcolor{red}{शेषात्कर्तरि परस्मैपदम्} (पा॰सू॰~१.३.७८)~\arrow \textcolor{red}{लृट् शेषे च} (पा॰सू॰~३.३.१३)~\arrow आनय~लृट्~\arrow आनय~मिप्~\arrow आनय~मि~\arrow \textcolor{red}{स्यतासी लृलुटोः} (पा॰सू॰~३.१.३३)~\arrow आनय~स्य~मि~\arrow \textcolor{red}{आर्धधातुकस्येड्वलादेः} (पा॰सू॰~७.२.३५)~\arrow आनय~इट्~स्य~मि~\arrow आनय~इ~स्य~मि~\arrow \textcolor{red}{अतो लोपः} (पा॰सू॰~६.४.४८)~\arrow आनय्~इ~स्य~मि~\arrow \textcolor{red}{अतो दीर्घो यञि} (पा॰सू॰~७.३.१०१)~\arrow आनय्~इ~स्या~मि~\arrow \textcolor{red}{आदेश\-प्रत्ययोः} (पा॰सू॰~८.३.५९)~\arrow आनय्~इ~ष्या~मि~\arrow आनयिष्यामि। यद्वा सप्तमाध्याये \textcolor{red}{आनय इवाऽचरिष्यामि आनयिष्यामि}। \textcolor{red}{आनयतीत्यानयः}। \textcolor{red}{नन्दि\-ग्रहि\-पचादिभ्यो ल्युणिन्यचः} (पा॰सू॰~३.१.१३४) इत्यनेन कर्तर्यच्। \textcolor{red}{नयतीति नयः} इतिवत्। यथा विष्णु\-सहस्र\-नाम\-स्तोत्रे \textcolor{red}{रामो विरामो विरतो मार्गो नेयो नयोऽनयः} (वि॰स॰ना॰~५६)। अत्र सत्यभाष्ये सत्यदेव\-वासिष्ठाः~– \textcolor{red}{‘णीञ् प्रापणे’ धातोरच्प्रत्ययो ‘नयतीति नयः’ सर्वस्य नेतेत्यर्थः} (वि॰स॰ना॰ स॰भा॰~५६)। \textcolor{red}{नय}\-शब्दो \textcolor{red}{नेतरि} इति शब्दकल्पद्रुम\-वाचस्पत्यौ च। एवं तर्हि \textcolor{red}{आनय इवाऽचरिष्यामि आनयिष्यामि}। सिद्धिः पूर्ववत्। \textcolor{red}{रामो न गच्छति न तिष्ठति नानुशोचत्याकाङ्क्षते त्यजति नो न करोति किञ्चित्} (अ॰रा॰~१.१.४३) इत्यस्मिन्नेव ग्रन्थ उक्तत्वात्।}\end{sloppypar}
\section[सन्तिष्ठ]{सन्तिष्ठ}
\centering\textcolor{blue}{गमिष्यामि मृगं बद्ध्वा ह्यानयिष्यामि सत्वरः।\nopagebreak\\
त्वं प्रयत्नेन सन्तिष्ठ सीतासंरक्षणोद्यतः॥}\nopagebreak\\
\raggedleft{–~अ॰रा॰~३.७.११}\\
\fontsize{14}{21}\selectfont\begin{sloppypar}\hyphenrules{nohyphenation}\justifying\noindent\hspace{10mm} \textcolor{red}{समवप्रविभ्यः स्थः} (पा॰सू॰~१.३.२२) इत्यनेनाऽत्मनेपदं प्राप्तम्। \textcolor{red}{सन्तिष्ठस्व}\footnote{सम्~\textcolor{red}{ष्ठा गतिनिवृत्तौ} (धा॰पा॰~९२८)~\arrow \textcolor{red}{धात्वादेः षः सः} (पा॰सू॰~६.१.६४)~\arrow निमित्तापाये नैमित्तिकस्याप्यपायः~\arrow सम्~स्था~\arrow \textcolor{red}{समवप्रविभ्यः स्थः} (पा॰सू॰~१.३.२२)~\arrow \textcolor{red}{लोट् च} (पा॰सू॰~३.३.१६२)~\arrow सम्~स्था~लोट्~\arrow सम्~स्था~थास्~\arrow \textcolor{red}{कर्तरि शप्} (पा॰सू॰~३.१.६८)~\arrow सम्~स्था~शप्~थास्~\arrow सम्~स्था~अ~थास्~\arrow \textcolor{red}{पाघ्रा\-ध्मास्थाम्ना\-दाण्दृश्यर्त्ति\-सर्त्तिशदसदां पिब\-जिघ्र\-धम\-तिष्ठ\-मन\-यच्छ\-पश्यर्च्छ\-धौ\-शीय\-सीदाः} (पा॰सू॰~७.३.७८)~\arrow सम्~तिष्ठ्~अ~थास्~\arrow \textcolor{red}{थासस्से} (पा॰सू॰~३.४.८०)~\arrow सम्~तिष्ठ्~अ~से~\arrow \textcolor{red}{सवाभ्यां वामौ} (पा॰सू॰~३.४.९१)~\arrow सम्~तिष्ठ्~अ~स्~व~\arrow \textcolor{red}{मोऽनुस्वारः} (पा॰सू॰~८.३.२३)~\arrow सं~तिष्ठ्~अ~स्~व~\arrow \textcolor{red}{वा पदान्तस्य} (पा॰सू॰~८.४.५९)~\arrow सन्~तिष्ठ्~अ~स्~व~\arrow सन्तिष्ठस्व।} इति प्रयोगः पाणिनीयः। किन्तु \textcolor{red}{तिष्ठ} इति यावत्संसाध्य\footnote{\textcolor{red}{ष्ठा गतिनिवृत्तौ} (धा॰पा॰~९२८)~\arrow \textcolor{red}{धात्वादेः षः सः} (पा॰सू॰~६.१.६४)~\arrow निमित्तापाये नैमित्तिकस्याप्यपायः~\arrow स्था~\arrow \textcolor{red}{शेषात्कर्तरि परस्मैपदम्} (पा॰सू॰~१.३.७८)~\arrow \textcolor{red}{लोट् च} (पा॰सू॰~३.३.१६२)~\arrow स्था~लोट्~\arrow स्था~सिप्~\arrow स्था~सि~\arrow \textcolor{red}{कर्तरि शप्} (पा॰सू॰~३.१.६८)~\arrow स्था~शप्~सि~\arrow स्था~अ~सि~\arrow \textcolor{red}{पाघ्रा\-ध्मास्थाम्ना\-दाण्दृश्यर्त्ति\-सर्त्तिशदसदां पिब\-जिघ्र\-धम\-तिष्ठ\-मन\-यच्छ\-पश्यर्च्छ\-धौ\-शीय\-सीदाः} (पा॰सू॰~७.३.७८)~\arrow तिष्ठ्~अ~सि~\arrow \textcolor{red}{सेर्ह्यपिच्च} (पा॰सू॰~३.४.८७)~\arrow तिष्ठ्~अ~हि~\arrow \textcolor{red}{अतो हेः} (पा॰सू॰~६.४.१०५)~\arrow तिष्ठ्~अ~\arrow तिष्ठ।} पादपूर्त्यर्थं \textcolor{red}{सम्} इति निपातः प्रयुक्तः। अतो नाऽत्मनेपदम्। यथा \textcolor{red}{ह हि नु ननु खलु किल हन्त} इत्यादयः।\footnote{यद्वा \textcolor{red}{सन्} इति शत्रन्तं पृथक्पदम्। \textcolor{red}{असँ भुवि} (धा॰पा॰~१०६५)~\arrow अस्~\arrow \textcolor{red}{शेषात्कर्तरि परस्मैपदम्} (पा॰सू॰~१.३.७८)~\arrow \textcolor{red}{वर्तमाने लट्} (पा॰सू॰~३.२.१२३)~\arrow अस्~लट्~\arrow \textcolor{red}{लटः शतृशानचावप्रथमा\-समानाधिकरणे} (पा॰सू॰~३.२.१२४)~\arrow अस्~शतृँ~\arrow अस्~अत्~\arrow \textcolor{red}{कर्तरि शप्‌} (पा॰सू॰~३.१.६८)~\arrow अस्~शप्~अत्~\arrow \textcolor{red}{अदिप्रभृतिभ्यः शपः} (पा॰सू॰~२.४.७२)~\arrow अस्~अत्~\arrow \textcolor{red}{श्नसोरल्लोपः} (पा॰सू॰~६.४.१११)~\arrow स्~अत्~\arrow सत्~\arrow \textcolor{red}{कृत्तद्धित\-समासाश्च} (पा॰सू॰~१.२.४६)~\arrow प्रातिपदिक\-सञ्ज्ञा~\arrow विभक्ति\-कार्यम्~\arrow सत्~सुँ~\arrow सत्~स्~\arrow \textcolor{red}{उगिदचां सर्वनामस्थानेऽधातोः} (पा॰सू॰~७.१.७०)~\arrow \textcolor{red}{मिदचोऽन्त्यात्परः} (पा॰सू॰~१.१.४७)~\arrow स~नुँम्~त्~स्~\arrow स~न्~त्~स्~\arrow \textcolor{red}{हल्ङ्याब्भ्यो दीर्घात्सुतिस्यपृक्तं हल्} (पा॰सू॰~६.१.६८)~\arrow स~न्~त्~\arrow \textcolor{red}{संयोगान्तस्य लोपः} (पा॰सू॰~८.२.२३)~\arrow स~न्~\arrow सन्। शतुः \textcolor{red}{तिङ्शित्सार्वधातुकम्} (पा॰सू॰~३.४.११३) इत्यनेन सार्वधातुकत्वात् \textcolor{red}{अस्तेर्भूः} (पा॰सू॰~२.४.५२) इत्यस्याप्रवृत्तिः। \textcolor{red}{हे लक्ष्मण त्वं प्रयत्नेन सीतासंरक्षणोद्यतः सन् तिष्ठ} इति श्रीरामतात्पर्यम्।}\end{sloppypar}
\section[विश्रमस्व]{विश्रमस्व}
\centering\textcolor{blue}{कन्दमूलफलादीनि दत्त्वा स्वागतमब्रवीत्।\nopagebreak\\
मुने भुङ्क्ष्व फलादीनि विश्रमस्व यथासुखम्॥}\nopagebreak\\
\raggedleft{–~अ॰रा॰~३.७.३९}\\
\fontsize{14}{21}\selectfont\begin{sloppypar}\hyphenrules{nohyphenation}\justifying\noindent\hspace{10mm} दिवादित्वात् \textcolor{red}{विश्राम्य} इति प्रयोगः पाणिनीयो दीर्घे श्यनि परस्मैपदे।\footnote{वि~\textcolor{red}{श्रमुँ तपसि खेदे च} (धा॰पा॰~१२०४)~\arrow वि~श्रम्~\arrow \textcolor{red}{शेषात्कर्तरि परस्मैपदम्} (पा॰सू॰~१.३.७८)~\arrow \textcolor{red}{लोट् च} (पा॰सू॰~३.३.१६२)~\arrow वि~श्रम्~लोट्~\arrow वि~श्रम्~सिप्~\arrow वि~श्रम्~सि~\arrow \textcolor{red}{दिवादिभ्यः श्यन्‌} (पा॰सू॰~३.१.६९)~\arrow वि~श्रम्~श्यन्~सि~\arrow वि~श्रम्~य~सि~\arrow \textcolor{red}{शमामष्टानां दीर्घः श्यनि} (पा॰सू॰~७.३.६४)~\arrow वि~श्राम्~य~सि~\arrow \textcolor{red}{सेर्ह्यपिच्च} (पा॰सू॰~३.४.८७)~\arrow वि~श्राम्~य~हि~\arrow \textcolor{red}{अतो हेः} (पा॰सू॰~६.४.१०५)~\arrow विश्राम्य।} अत्र त्रयोऽप्यंशा विमर्श\-कोटिमञ्चन्ति।\footnote{यतो ह्यत्र दीर्घाभावः श्यन्नभाव आत्मनेपदञ्च।} \textcolor{red}{विश्रमणं विश्रमः}।\footnote{वि~\textcolor{red}{श्रमुँ तपसि खेदे च} (धा॰पा॰~१२०४)~\arrow \textcolor{red}{भावे} (पा॰सू॰~३.३.१८)~\arrow वि~श्रम्~घञ्~\arrow वि~श्रम्~अ~\arrow \textcolor{red}{अत उपधायाः} (पा॰सू॰~७.२.११६)~\arrow वृद्धिप्राप्तिः~\arrow \textcolor{red}{नोदात्तोपदेशस्य मान्तस्यानाचमेः} (पा॰सू॰~७.३.३४)~\arrow वृद्धिनिषेधः~\arrow वि~श्रम्~अ~\arrow विश्रम~\arrow विभक्तिकार्यम्~\arrow विश्रमः।} \textcolor{red}{विश्रममाचरति विश्रमति}।\footnote{विश्रम~\arrow \textcolor{red}{सर्वप्राति\-पदिकेभ्य आचारे क्विब्वा वक्तव्यः} (वा॰~३.१.११)~\arrow विश्रम~क्विँप्~\arrow विश्रम~व्~\arrow \textcolor{red}{वेरपृक्तस्य} (पा॰सू॰~६.१.६७)~\arrow विश्रम~\arrow \textcolor{red}{सनाद्यन्ता धातवः} (पा॰सू॰~३.१.३२)~\arrow धातुसञ्ज्ञा~\arrow \textcolor{red}{शेषात्कर्तरि परस्मैपदम्} (पा॰सू॰~१.३.७८)~\arrow \textcolor{red}{वर्तमाने लट्} (पा॰सू॰~३.२.१२३)~\arrow विश्रम~लट्~\arrow विश्रम~तिप्~\arrow विश्रम~ति~\arrow \textcolor{red}{कर्तरि शप्‌} (पा॰सू॰~३.१.६८)~\arrow विश्रम~शप्~ति~\arrow विश्रम~अ~ति~\arrow \textcolor{red}{अतो गुणे} (पा॰सू॰~६.१.९७)~\arrow विश्रम~ति~\arrow विश्रमति।} तदेव लोटि \textcolor{red}{विश्रम}।\footnote{विश्रम~\arrow धातुसञ्ज्ञा (पूर्ववत्)~\arrow \textcolor{red}{शेषात्कर्तरि परस्मैपदम्} (पा॰सू॰~१.३.७८)~\arrow \textcolor{red}{लोट् च} (पा॰सू॰~३.३.१६२)~\arrow विश्रम~लोट्~\arrow विश्रम~सिप्~\arrow विश्रम~सि~\arrow \textcolor{red}{कर्तरि शप्‌} (पा॰सू॰~३.१.६८)~\arrow विश्रम~शप्~सि~\arrow विश्रम~अ~सि~\arrow \textcolor{red}{अतो गुणे} (पा॰सू॰~६.१.९७)~\arrow विश्रम~सि~\arrow \textcolor{red}{सेर्ह्यपिच्च} (पा॰सू॰~३.४.८७)~\arrow विश्रम~हि~\arrow \textcolor{red}{अतो हेः} (पा॰सू॰~६.४.१०५)~\arrow विश्रम।} \textcolor{red}{हे स्व आत्मीय विश्रम}।\end{sloppypar}
\section[भक्षन्तु]{भक्षन्तु}
\centering\textcolor{blue}{शाद्वले प्राक्षिपद्रामः पृथक्पृथगनेकधा।\nopagebreak\\
भक्षन्तु पक्षिणः सर्वे तृप्तो भवतु पक्षिराट्॥}\nopagebreak\\
\raggedleft{–~अ॰रा॰~३.८.३९}\\
\fontsize{14}{21}\selectfont\begin{sloppypar}\hyphenrules{nohyphenation}\justifying\noindent\hspace{10mm} णिजन्त\-प्रसिद्धोऽयं \textcolor{red}{भक्षयन्तु} इति।\footnote{\textcolor{red}{भक्षँ अदने} (धा॰पा॰~१५५७) (\textcolor{red}{भ्लक्षँ/भक्षँ} इति माधवीयाधातुवृत्तिः, \textcolor{red}{भक्षँ} इति मैत्रेयः)~\arrow भक्ष्~\arrow \textcolor{red}{सत्यापपाश\-रूप\-वीणा\-तूल\-श्लोक\-सेना\-लोम\-त्वच\-वर्म\-वर्ण\-चूर्ण\-चुरादिभ्यो णिच् } (पा॰सू॰~३.१.२५)~\arrow भक्ष्~णिच्~\arrow भक्ष्~इ~\arrow भक्षि~\arrow \textcolor{red}{सनाद्यन्ता धातवः} (पा॰सू॰~३.१.३२)~\arrow धातु\-सञ्ज्ञा~\arrow \textcolor{red}{शेषात्कर्तरि परस्मैपदम्} (पा॰सू॰~१.३.७८)~\arrow \textcolor{red}{लोट् च} (पा॰सू॰~३.३.१६२)~\arrow भक्षि~लोट्~\arrow भक्षि~झि~\arrow \textcolor{red}{कर्तरि शप्‌} (पा॰सू॰~३.१.६८)~\arrow भक्षि~शप्~झि~\arrow भक्षि~अ~झि~\arrow \textcolor{red}{झोऽन्तः} (पा॰सू॰~७.१.३)~\arrow भक्षि~अ~अन्ति~\arrow \textcolor{red}{सार्वधातुकार्ध\-धातुकयोः} (पा॰सू॰~७.३.८४)~\arrow भक्षे~अ~अन्ति~\arrow \textcolor{red}{एचोऽयवायावः} (पा॰सू॰~६.१.७८)~\arrow भक्षय्~अ~अन्ति~\arrow \textcolor{red}{अतो गुणे} (पा॰सू॰~६.१.९७)~\arrow भक्षय्~अन्ति~\arrow \textcolor{red}{एरुः} (पा॰सू॰~३.४.८६)~\arrow भक्षय्~अन्तु~\arrow भक्षयन्तु।} किन्तु णिजन्ताः शुद्धा अपि भवन्त्यतो नापाणिनीयता।\footnote{\textcolor{red}{भक्षँ अदने} (धा॰पा॰~१५५७) (\textcolor{red}{भ्लक्षँ/भक्षँ} इति माधवीयाधातुवृत्तिः, \textcolor{red}{भक्षँ} इति मैत्रेयः)~\arrow भक्ष्~\arrow \textcolor{red}{शेषात्कर्तरि परस्मैपदम्} (पा॰सू॰~१.३.७८)~\arrow \textcolor{red}{लोट् च} (पा॰सू॰~३.३.१६२)~\arrow भक्ष्~लोट्~\arrow भक्ष्~झि~\arrow \textcolor{red}{कर्तरि शप्‌} (पा॰सू॰~३.१.६८)~\arrow भक्ष्~शप्~झि~\arrow भक्ष्~अ~झि~\arrow \textcolor{red}{झोऽन्तः} (पा॰सू॰~७.१.३)~\arrow भक्ष्~अ~अन्ति~\arrow \textcolor{red}{अतो गुणे} (पा॰सू॰~६.१.९७)~\arrow भक्ष्~अन्ति~\arrow \textcolor{red}{एरुः} (पा॰सू॰~३.४.८६)~\arrow भक्ष्~अन्तु~\arrow भक्षन्तु। यद्वा भक्षयन्तीति भक्षाः। \textcolor{red}{नन्दि\-ग्रहि\-पचादिभ्यो ल्युणिन्यचः} (पा॰सू॰~३.१.१३४) इत्यनेन कर्तरि पचाद्यचि \textcolor{red}{णेरनिटि} (पा॰सू॰~६.४.५१) इत्यनेन णिलोपे विभक्तिकार्ये। भक्षा इवाऽचरन्तु भक्षन्तु। भक्ष~\arrow \textcolor{red}{सर्वप्राति\-पदिकेभ्य आचारे क्विब्वा वक्तव्यः} (वा॰~३.१.११)~\arrow भक्ष~क्विँप्~\arrow भक्ष~व्~\arrow \textcolor{red}{वेरपृक्तस्य} (पा॰सू॰~६.१.६७)~\arrow भक्ष~\arrow \textcolor{red}{सनाद्यन्ता धातवः} (पा॰सू॰~३.१.३२)~\arrow धातुसञ्ज्ञा~\arrow \textcolor{red}{शेषात्कर्तरि परस्मैपदम्} (पा॰सू॰~१.३.७८)~\arrow \textcolor{red}{लोट् च} (पा॰सू॰~३.३.१६२)~\arrow भक्ष~लोट्~\arrow भक्ष~झि~\arrow \textcolor{red}{कर्तरि शप्‌} (पा॰सू॰~३.१.६८)~\arrow भक्ष~शप्~झि~\arrow भक्ष~अ~झि~\arrow \textcolor{red}{झोऽन्तः} (पा॰सू॰~७.१.३)~\arrow भक्ष~अ~अन्ति~\arrow \textcolor{red}{अतो गुणे} (पा॰सू॰~६.१.९७)~\arrow भक्ष~अन्ति~\arrow \textcolor{red}{अतो गुणे} (पा॰सू॰~६.१.९७)~\arrow भक्षन्ति~\arrow \textcolor{red}{एरुः} (पा॰सू॰~३.४.८६)~\arrow भक्षन्तु।}\end{sloppypar}
\vspace{2mm}
\centering ॥ इत्यरण्यकाण्डीयप्रयोगाणां विमर्शः ॥\nopagebreak\\
\vspace{4mm}
\pdfbookmark[2]{किष्किन्धाकाण्डम्}{Chap3Part2Kanda4}
\phantomsection
\addtocontents{toc}{\protect\setcounter{tocdepth}{2}}\addtocontents{toc}{\protect\setcounter{tocdepth}{2}}
\addcontentsline{toc}{subsection}{किष्किन्धाकाण्डीयप्रयोगाणां विमर्शः}
\addtocontents{toc}{\protect\setcounter{tocdepth}{0}}
\centering ॥ अथ किष्किन्धाकाण्डीयप्रयोगाणां विमर्शः ॥\nopagebreak\\
\section[आकाङ्क्षे]{आकाङ्क्षे}
\centering\textcolor{blue}{दाराः पुत्रा धनं राज्यं सर्वं त्वन्मायया कृतम्।\nopagebreak\\
अतोऽहं देवदेवेश नाकाङ्क्षेऽन्यत्प्रसीद मे॥}\nopagebreak\\
\raggedleft{–~अ॰रा॰~४.१.७८}\\
\fontsize{14}{21}\selectfont\begin{sloppypar}\hyphenrules{nohyphenation}\justifying\noindent\hspace{10mm} \textcolor{red}{आकाङ्क्षे} इत्यत्र परस्मैपदेन भवितव्यम्। धातोः परस्मैपदीयत्वात्। अतः \textcolor{red}{आकाङ्क्षामि} इति पाणिनीयम्।\footnote{आ~\textcolor{red}{काक्षिँ काङ्क्षायाम्} (धा॰पा॰~६६७)~\arrow आ~काक्ष्~\arrow \textcolor{red}{इदितो नुम् धातोः} (पा॰सू॰~७.१.५८)~\arrow \textcolor{red}{मिदचोऽन्त्यात्परः} (पा॰सू॰~१.१.४७)~\arrow आ~का~नुँम्~क्ष्~\arrow आ~कान्~क्ष्~\arrow \textcolor{red}{नश्चापदान्तस्य झलि} (पा॰सू॰~८.३.२४)~\arrow आ~कांक्ष्~\arrow \textcolor{red}{अनुस्वारस्य ययि परसवर्णः} (पा॰सू॰~८.४.५८)~\arrow आ~काङ्क्ष्~\arrow \textcolor{red}{शेषात्कर्तरि परस्मैपदम्} (पा॰सू॰~१.३.७८)~\arrow \textcolor{red}{वर्तमाने लट्} (पा॰सू॰~३.२.१२३)~\arrow आ~काङ्क्ष्~लट्~\arrow आ~काङ्क्ष्~मिप्~\arrow आ~काङ्क्ष्~मि~\arrow \textcolor{red}{कर्तरि शप्‌} (पा॰सू॰~३.१.६८)~\arrow आ~काङ्क्ष्~शप्~मि~\arrow आ~काङ्क्ष्~अ~मि~\arrow \textcolor{red}{अतो दीर्घो यञि} (पा॰सू॰~७.३.१०१)~\arrow आ~काङ्क्ष्~आ~मि~\arrow आकाङ्क्षामि।} \textcolor{red}{आकाङ्क्षे} इति तु \textcolor{red}{कर्तरि कर्म\-व्यतिहारे} (पा॰सू॰~१.३.१४) इति सूत्रेणाऽत्मनेपदम्।\footnote{आ~काङ्क्ष् (पूर्ववत्)~\arrow \textcolor{red}{कर्तरि कर्म\-व्यतिहारे} (पा॰सू॰~१.३.१४)~\arrow \textcolor{red}{वर्तमाने लट्} (पा॰सू॰~३.२.१२३)~\arrow आ~काङ्क्ष्~लट्~\arrow आ~काङ्क्ष्~इट्~\arrow आ~काङ्क्ष्~इ~\arrow \textcolor{red}{कर्तरि शप्‌} (पा॰सू॰~३.१.६८)~\arrow आ~काङ्क्ष्~शप्~इ~\arrow आ~काङ्क्ष्~अ~इ~\arrow \textcolor{red}{टित आत्मनेपदानां टेरे} (पा॰सू॰~३.४.७९)~\arrow आ~काङ्क्ष्~अ~ए~\arrow \textcolor{red}{अतो गुणे} (पा॰सू॰~६.१.९७)~\arrow आ~काङ्क्ष्~ए~\arrow आकाङ्क्षे।} कर्म\-व्यतिहारो नाम क्रिया\-विनिमयः।
अन्य\-करणीय\-कार्यस्यान्येन सम्पादनम्।
यथा कौमुद्यां \textcolor{red}{क्रिया\-विनिमये द्योत्ये कर्तर्यात्मनेपदं स्यात्} (वै॰सि॰कौ॰~२६८०)। कृत\-भगवद्दर्शनस्यान्यत्काङ्क्षणं स्पष्टं क्रिया\-विनिमयः। यद्वा \textcolor{red}{आकाङ्क्षणमाकाङ्क्षः} पचाद्यज्भावे।\footnote{\textcolor{red}{नन्दि\-ग्रहि\-पचादिभ्यो ल्युणिन्यचः} (पा॰सू॰~३.१.१३४) इत्यनेन। बाहुलकाद्भावे।} तस्मिन् \textcolor{red}{आकाङ्क्षे}। विषयत्वात्सप्तमी। ममाऽकाङ्क्षण इच्छायामन्यन्न। यद्वा \textcolor{red}{आकाङ्क्षणमाकाङ्क्षा}।\footnote{आ~काङ्क्ष् (पूर्ववत्)~\arrow \textcolor{red}{गुरोश्च हलः} (पा॰सू॰~३.३.१०३)~\arrow आ~काङ्क्ष्~अ~\arrow \textcolor{red}{अजाद्यतष्टाप्‌} (पा॰सू॰~४.१.४)~\arrow आ~काङ्क्ष्~अ~टाप्~\arrow आ~काङ्क्ष्~अ~आ~\arrow \textcolor{red}{अकः सवर्णे दीर्घः} (पा॰सू॰~६.१.१०१)~\arrow आ~काङ्क्ष्~आ~\arrow आकाङ्क्षा~\arrow \textcolor{red}{कृत्तद्धित\-समासाश्च} (पा॰सू॰~१.२.४६)~\arrow प्रातिपदिक\-सञ्ज्ञा~\arrow विभक्ति\-कार्यम्~\arrow आकाङ्क्षा~सुँ~\arrow \textcolor{red}{हल्ङ्याब्भ्यो दीर्घात्सुतिस्यपृक्तं हल्} (पा॰सू॰~६.१.६८)~\arrow आकाङ्क्षा। \textcolor{red}{अथ दोहदम्। इच्छा काङ्क्षा स्पृहेहा तृड्वाञ्छा लिप्सा मनोरथः॥ कामोऽभिलाषस्तर्षश्च सोऽत्यर्थं लालसा द्वयोः।} (अ॰को॰~१.७.२७–२८) इत्यमरः।} \textcolor{red}{आकाङ्क्षाऽस्त्यस्येत्याकाङ्क्षम्}। मनः। अर्शआदित्वादच्।\footnote{\textcolor{red}{अर्शआदिभ्योऽच्} (पा॰सू॰~५.२.१२७) इत्यनेन। आकाङ्क्षा~\arrow \textcolor{red}{अर्शआदिभ्योऽच्} (पा॰सू॰~५.२.१२७)~\arrow आकाङ्क्षा~अच्~\arrow आकाङ्क्षा~अ~\arrow \textcolor{red}{यचि भम्} (पा॰सू॰~१.४.१८)~\arrow भ\-सञ्ज्ञा~\arrow \textcolor{red}{यस्येति च} (पा॰सू॰~६.४.१४८)~\arrow आकाङ्क्ष्~अ~\arrow आकाङ्क्ष~\arrow \textcolor{red}{कृत्तद्धित\-समासाश्च} (पा॰सू॰~१.२.४६)~\arrow प्रातिपदिक\-सञ्ज्ञा~\arrow विभक्ति\-कार्यम्~\arrow आकाङ्क्ष~सुँ~\arrow \textcolor{red}{अतोऽम्} (पा॰सू॰~७.१.२४)~\arrow आकाङ्क्ष~अम्~\arrow \textcolor{red}{अमि पूर्वः} (पा॰सू॰~६.१.१०७)~\arrow आकाङ्क्षम्।} तस्मिन् \textcolor{red}{आकाङ्क्षे} त्वद्दर्शनेच्छावति मे मनसि नान्यत्किञ्चित्।\end{sloppypar}
\section[रक्षामहे]{रक्षामहे}
\centering\textcolor{blue}{चतुर्द्वारकपाटादीन् बद्ध्वा रक्षामहे पुरीम्।\nopagebreak\\
वानराणां तु राजानमङ्गदं कुरु भामिनि॥}\nopagebreak\\
\raggedleft{–~अ॰रा॰~४.३.३}\\
\fontsize{14}{21}\selectfont\begin{sloppypar}\hyphenrules{nohyphenation}\justifying\noindent\hspace{10mm} इहाऽपि कर्म\-व्यत्ययादात्मनेपदम्।\footnote{\textcolor{red}{कर्तरि कर्म\-व्यतिहारे} (पा॰सू॰~१.३.१४) इत्यनेन। पुरीरक्षणं सैन्यवानराणां कर्म न तु सर्वेषां वानराणाम्। अत्र सर्वे वानराः (\textcolor{red}{दुद्रुवुर्वानराः सर्वे किष्किन्धां भयविह्वलाः} अ॰रा॰~४.३.१) बाला वृद्धा विकलाङ्गा अपि रक्षाकर्म कुर्वन्तीति कर्मव्यतिहारः। \textcolor{red}{रक्षँ पालने} (धा॰पा॰~६५८)~\arrow रक्ष्~\arrow \textcolor{red}{कर्तरि कर्म\-व्यतिहारे} (पा॰सू॰~१.३.१४)~\arrow (पा॰सू॰~३.३.१३)~\arrow \textcolor{red}{वर्तमाने लट्} (पा॰सू॰~३.२.१२३)~\arrow रक्ष्~लृँट्~\arrow रक्ष्~महिङ्~\arrow रक्ष्~महि~\arrow \textcolor{red}{कर्तरि शप्‌} (पा॰सू॰~३.१.६८)~\arrow रक्ष्~शप्~महि~\arrow रक्ष्~अ~महि~\arrow \textcolor{red}{अतो दीर्घो यञि} (पा॰सू॰~७.३.१०१)~\arrow रक्ष्~आ~महि~\arrow \textcolor{red}{टित आत्मनेपदानां टेरे} (पा॰सू॰~३.४.७९)~\arrow रक्ष्~आ~महे~\arrow रक्षामहे।} यद्वा \textcolor{red}{हे} इति पृथक्पदम्। \textcolor{red}{हे तारे वयं नगरीं रक्षाम} इति प्रार्थनायां लोट्। सम्प्रश्ने वा। \textcolor{red}{लोट् च} (पा॰सू॰~३.३.१६२) इत्यनेन।\footnote{\textcolor{red}{रक्षँ पालने} (धा॰पा॰~६५८)~\arrow रक्ष्~\arrow \textcolor{red}{शेषात्कर्तरि परस्मैपदम्} (पा॰सू॰~१.३.७८)~\arrow \textcolor{red}{लोट् च} (पा॰सू॰~३.३.१६२)~\arrow रक्ष्~लोट्~\arrow रक्ष्~मस्~\arrow \textcolor{red}{कर्तरि शप्‌} (पा॰सू॰~३.१.६८)~\arrow रक्ष्~शप्~मस्~\arrow रक्ष्~अ~मस्~\arrow \textcolor{red}{आडुत्तमस्य पिच्च} (पा॰सू॰~३.४.९२)~\arrow रक्ष्~अ~आट्~मस्~\arrow \textcolor{red}{अकः सवर्णे दीर्घः} (पा॰सू॰~६.१.१०१)~\arrow रक्ष्~आ~मस्~\arrow \textcolor{red}{लोटो लङ्वत्‌} (पा॰सू॰~३.४.८५)~\arrow ङिद्वत्त्वम्~\arrow नित्यं ङितः~\arrow रक्ष्~आ~म~\arrow रक्षाम।}\end{sloppypar}
\section[लिप्यसे]{लिप्यसे}
\centering\textcolor{blue}{ध्यात्वा मद्रूपमनिशमालोचय मयोदितम्।\nopagebreak\\
प्रवाहपतितं कार्यं कुर्वत्यपि न लिप्यसे॥}\nopagebreak\\
\raggedleft{–~अ॰रा॰~४.३.३५}\\
\fontsize{14}{21}\selectfont\begin{sloppypar}\hyphenrules{nohyphenation}\justifying\noindent\hspace{10mm} \textcolor{red}{वर्तमान\-सामीप्ये वर्तमानवद्वा} (पा॰सू॰~३.३.१३१) इत्यनेनाऽसन्न\-भविष्यत्काले वर्तमानम्।\footnote{\textcolor{red}{लिपँ उपदेहे} (धा॰पा॰~१४३३)~\arrow लिप्~\arrow \textcolor{red}{भावकर्मणोः} (पा॰सू॰~१.३.१३)~\arrow \textcolor{red}{वर्तमान\-सामीप्ये वर्तमानवद्वा} (पा॰सू॰~३.३.१३१)~\arrow \textcolor{red}{वर्तमाने लट्} (पा॰सू॰~३.२.१२३)~\arrow लिप्~लट्~\arrow लिप्~थास्~\arrow \textcolor{red}{सार्वधातुके यक्} (पा॰सू॰~३.१.६७)~\arrow लिप्~यक्~थास्~\arrow लिप्~य~थास्~\arrow \textcolor{red}{ग्क्ङिति च} (पा॰सू॰~१.१.५)~\arrow लघूपध\-गुण\-निषेधः~\arrow लिप्~य~थास्~\arrow \textcolor{red}{थासस्से} (पा॰सू॰~३.४.८०)~\arrow लिप्~य~से~\arrow लिप्यसे।}\end{sloppypar}
\section[अनुसेविरे]{अनुसेविरे}
\centering\textcolor{blue}{चरन्तं परमात्मानं ज्ञात्वा सिद्धगणा भुवि।\nopagebreak\\
मृगपक्षिगणा भूत्वा राममेवानुसेविरे॥}\nopagebreak\\
\raggedleft{–~अ॰रा॰~४.४.५}\\
\fontsize{14}{21}\selectfont\begin{sloppypar}\hyphenrules{nohyphenation}\justifying\noindent\hspace{10mm}
लिड्लकार\-प्रयोगात् \textcolor{red}{अनुसिषेविरे}\footnote{अनु~\textcolor{red}{षेवृँ सेवने} (धा॰पा॰~५०१)~\arrow अनु~षेव्~\arrow \textcolor{red}{धात्वादेः षः सः} (पा॰सू॰~६.१.६४)~\arrow अनु~सेव्~\arrow \textcolor{red}{अनुदात्तङित आत्मने\-पदम्} (पा॰सू॰~१.३.१२)~\arrow \textcolor{red}{परोक्षे लिट्} (पा॰सू॰~३.२.११५)~\arrow अनु~सेव्~लिट्~\arrow अनु~सेव्~झ~\arrow \textcolor{red}{लिटि धातोरनभ्यासस्य} (पा॰सू॰~६.१.८)~\arrow अनु~सेव्~सेव्~झ~\arrow \textcolor{red}{हलादिः शेषः} (पा॰सू॰~७.४.६०)~\arrow अनु~से~सेव्~झ~\arrow \textcolor{red}{ह्रस्वः} (पा॰सू॰~७.४.५९)~\arrow \textcolor{red}{एच इग्घ्रस्वादेशे} (पा॰सू॰~१.१.४८)~\arrow अनु~सि~सेव्~झ~\arrow \textcolor{red}{लिटस्तझयोरेशिरेच्} (पा॰सू॰~३.४.८१)~\arrow अनु~सि~सेव्~इरेच्~\arrow अनु~सि~सेव्~इरे~\arrow \textcolor{red}{आदेश\-प्रत्यययोः} (पा॰सू॰~८.३.५९)~\arrow अनु~सि~षेव्~इरे~\arrow अनुसिषेविरे।} इति प्रयोक्तव्ये सिलोपाच्च सिद्धम् \textcolor{red}{अनुसेविरे} इति। 
अर्थात् \textcolor{red}{विनाऽपि प्रत्ययं पूर्वोत्तर\-पद\-लोपो वक्तव्यः} (वा॰~५.३.८३) इत्यनेनाऽदेर्हलोऽचश्च लोपे\footnote{बाहुलकादपदस्यापि लोप इति शेषः।} 
\textcolor{red}{अनुसेविरे} इति। यद्वा \textcolor{red}{अनुसेवां कृतवन्तः} इति भूत\-काल औणादिके \textcolor{red}{डिरच्} प्रत्यये भावे सप्तम्यन्ते च \textcolor{red}{अनुसेविरे}।\footnote{\textcolor{red}{चेरुः} इति चाध्याहार्यम्। \textcolor{red}{कार्याद्विद्यादनूबन्धम्} (भा॰पा॰सू॰~३.३.१) \textcolor{red}{केचिदविहिता अप्यूह्याः} (वै॰सि॰कौ॰~३१६९) इत्यनुसारमूह्योऽ\-यमविहितो \textcolor{red}{डिरच्} प्रत्ययः। अनुसेवा~\arrow \textcolor{red}{उणादयो बहुलम्} (पा॰सू॰~३.३.१)~\arrow अनुसेवा~डिरच्~\arrow अनुसेवा~इर~\arrow \textcolor{red}{डित्यभस्याप्यनु\-बन्धकरण\-सामर्थ्यात्} (वा॰~६.४.१४३)~\arrow अनुसेव्~इर~\arrow अनुसेविर~\arrow \textcolor{red}{कृत्तद्धित\-समासाश्च} (पा॰सू॰~१.२.४६)~\arrow प्रातिपदिक\-सञ्ज्ञा~\arrow विभक्ति\-कार्यम्~\arrow अनुसेविर~ङि~\arrow अनुसेविर~इ~\arrow \textcolor{red}{आद्गुणः} (पा॰सू॰~६.१.८७)~\arrow अनुसेविरे।} यद्वा \textcolor{red}{अनुसेवनमनु\-सेवा}।\footnote{\textcolor{red}{अनु}\-पूर्वकात् \textcolor{red}{षेवृँ सेवने} (धा॰पा॰~५०१) धातोः \textcolor{red}{गुरोश्च हलः} (पा॰सू॰~३.३.१०३) इत्यनेन भावे स्त्रियां \textcolor{red}{अ}प्रत्यये ततश्च \textcolor{red}{अजाद्यतष्टाप्‌} (पा॰सू॰~४.१.४) इत्यनेन टापि विभक्तिकार्ये।} \textcolor{red}{अनुसेवामाचरत्यनु\-सेवयति}।\footnote{अनुसेवा~\arrow \textcolor{red}{तत्करोति तदाचष्टे} (धा॰पा॰ ग॰सू॰~१८७)~\arrow अनुसेवा~णिच्~\arrow अनुसेवा~इ~\arrow \textcolor{red}{णाविष्ठवत्प्राति\-पदिकस्य पुंवद्भाव\-रभाव\-टिलोप\-यणादि\-परार्थम्} (वा॰~६.४.४८)~\arrow अनुसेव्~इ~\arrow अनुसेवि~\arrow \textcolor{red}{सनाद्यन्ता धातवः} (पा॰सू॰~३.१.३२)~\arrow धातुसञ्ज्ञा~\arrow \textcolor{red}{शेषात्कर्तरि परस्मैपदम्} (पा॰सू॰~१.३.७८)~\arrow \textcolor{red}{वर्तमाने लट्} (पा॰सू॰~३.२.१२३)~\arrow अनुसेवि~तिप्~\arrow अनुसेवि~ति~\arrow \textcolor{red}{कर्तरि शप्‌} (पा॰सू॰~३.१.६८)~\arrow अनुसेवि~शप्~ति~\arrow अनुसेवि~अ~ति~\arrow \textcolor{red}{सार्वधातुकार्ध\-धातुकयोः} (पा॰सू॰~७.३.८४)~\arrow अनुसेवे~अ~ति~\arrow \textcolor{red}{एचोऽयवायावः} (पा॰सू॰~६.१.७८)~\arrow अनुसेवय्~अ~ति~\arrow अनुसेवयति।} पुनः \textcolor{red}{अनुसेव्यत\footnote{अनुसेवि~\arrow धातु\-सञ्ज्ञा (पूर्ववत्)~\arrow \textcolor{red}{भावकर्मणोः} (पा॰सू॰~१.३.१३)~\arrow \textcolor{red}{वर्तमाने लट्} (पा॰सू॰~३.२.१२३)~\arrow अनुसेवि~लट्~\arrow अनुसेवि~त~\arrow \textcolor{red}{सार्वधातुके यक्} (पा॰सू॰~३.१.६७)~\arrow अनुसेवि~यक्~त~\arrow अनुसेवि~य~त~\arrow \textcolor{red}{णेरनिटि} (पा॰सू॰~६.४.५१)~\arrow अनुसेव्~य~त~\arrow \textcolor{red}{टित आत्मनेपदानां टेरे} (पा॰सू॰~३.४.७९)~\arrow अनुसेव्~य~ते~\arrow अनुसेव्यते।} इति अनुसेव्} हलन्त\-नपुंसक\-लिङ्गे।\footnote{\textcolor{red}{अनुसेवि}\-धातोः \textcolor{red}{सम्पदादिभ्‍यः क्विप्‌} (वा॰~३.३.१०८) इत्यनेन भावे क्विपि \textcolor{red}{णेरनिटि} (पा॰सू॰~६.४.५१) इति णिलोपे। बाहुलकान्नपुंसकलिङ्गम्।} लिड्लकारे \textcolor{red}{इण्‌}\-धातोः (\textcolor{red}{इण् गतौ} धा॰पा॰~१०४५) \textcolor{red}{कर्तरि कर्म\-व्यतिहारे} (पा॰सू॰~१.३.१४) इत्यनेनाऽत्मनेपदम्। \textcolor{red}{लिटस्तझयोरेशिरेच्} (पा॰सू॰~३.४.८१) इत्यनेन \textcolor{red}{इरेच्} प्रत्यये \textcolor{red}{ईयिरे}।\footnote{\textcolor{red}{इण् गतौ} (धा॰पा॰~१०४५)~\arrow इ~\arrow \textcolor{red}{कर्तरि कर्म\-व्यतिहारे} (पा॰सू॰~१.३.१४)~\arrow \textcolor{red}{परोक्षे लिट्} (पा॰सू॰~३.२.११५)~\arrow इ~लिट्~\arrow इ~झ~\arrow \textcolor{red}{लिटि धातोरनभ्यासस्य} (पा॰सू॰~६.१.८)~\arrow इ~इ~झ~\arrow \textcolor{red}{लिटस्तझयोरेशिरेच्} (पा॰सू॰~३.४.८१)~\arrow इ~इ~इरेच्~\arrow इ~इ~इरे~\arrow \textcolor{red}{असंयोगाल्लिट् कित्} (पा॰सू॰~१.२.५)~\arrow \textcolor{red}{दीर्घ इणः किति} (पा॰सू॰~७.४.६९)~\arrow ई~इ~इरे~\arrow \textcolor{red}{इणो यण्} (पा॰सू॰~६.४.८१)~\arrow ई~य्~इरे~\arrow ईयिरे।}
\textcolor{red}{विनाऽपि प्रत्ययं पूर्वोत्तर\-पद\-लोपो वक्तव्यः} (वा॰~५.३.८३) इत्यनेन धात्विकार\-लोपे\footnote{बाहुलकादपदस्यापि लोप इति शेषः। धातोरिकारौ धात्विकारौ तयोर्लोपः धात्विकारलोपस्तस्मिन् धात्विकारलोपे।} \textcolor{red}{इरे}। 
\textcolor{red}{अनुसेव् इरे} इति \textcolor{red}{अनुसेविरे}।\end{sloppypar}
\section[समारभेत्]{समारभेत्}
\centering\textcolor{blue}{किं पुनर्भक्ष्यभोज्यादि गन्धपुष्पाक्षतादिकम्।\nopagebreak\\
पूजाद्रव्याणि सर्वाणि सम्पाद्यैवं समारभेत्॥}\nopagebreak\\
\raggedleft{–~अ॰रा॰~४.४.२०}\\
\fontsize{14}{21}\selectfont\begin{sloppypar}\hyphenrules{nohyphenation}\justifying\noindent\hspace{10mm} \textcolor{red}{समा}\-पूर्वकात् \textcolor{red}{रभ्‌}\-धातोः (\textcolor{red}{रभँ राभस्ये} धा॰पा॰~९७४) विधिलिङि \textcolor{red}{समारभेत}\footnote{सम्~आङ्~\textcolor{red}{रभँ राभस्ये} (धा॰पा॰~९७४)~\arrow सम्~आ~रभ्~\arrow \textcolor{red}{अनुदात्तङित आत्मने\-पदम्} (पा॰सू॰~१.३.१२)~\arrow \textcolor{red}{विधि\-निमन्‍त्रणामन्‍त्रणाधीष्‍ट\-सम्प्रश्‍न\-प्रार्थनेषु लिङ्} (पा॰सू॰~३.३.१६१)~\arrow सम्~आ~रभ्~लिङ्~\arrow सम्~आ~रभ्~त~\arrow \textcolor{red}{कर्तरि शप्‌} (पा॰सू॰~३.१.६८)~\arrow समारभ~शप्~त~\arrow समारभ~अ~त~\arrow \textcolor{red}{अतो गुणे} (पा॰सू॰~६.१.९७)~\arrow समारभ~त~\arrow \textcolor{red}{यासुट् परस्मैपदेषूदात्तो ङिच्च} (पा॰सू॰~३.४.१०३)~\arrow \textcolor{red}{आद्यन्तौ टकितौ} (पा॰सू॰~१.१.४६)~\arrow समारभ~यासुँट्~त~\arrow समारभ~यास्~त~\arrow \textcolor{red}{सुट् तिथोः} (पा॰सू॰~३.४.१०७)~\arrow \textcolor{red}{आद्यन्तौ टकितौ} (पा॰सू॰~१.१.४६)~\arrow समारभ~यास्~सुँट्~त~\arrow समारभ~यास्~स्~त~\arrow \textcolor{red}{लिङः सलोपोऽनन्त्यस्य} (पा॰सू॰~७.२.७९)~\arrow समारभ~या~त~\arrow \textcolor{red}{अतो येयः} (पा॰सू॰~७.२.८०)~\arrow समारभ~इय्~त~\arrow \textcolor{red}{लोपो व्योर्वलि} (पा॰सू॰~६.१.६६)~\arrow समारभ~इ~त~\arrow \textcolor{red}{आद्गुणः} (पा॰सू॰~६.१.८७)~\arrow समारभे~त~\arrow समारभेत।} इति प्रयोक्तव्ये \textcolor{red}{समारभेत्} इति प्रयुक्तम्। \textcolor{red}{समारभत इति समारभः}। पचाद्यच्।\footnote{\textcolor{red}{नन्दि\-ग्रहि\-पचादिभ्यो ल्युणिन्यचः} (पा॰सू॰~३.१.१३४) इत्यनेन।} \textcolor{red}{समारभ इवाचरेत्} इति विधिलिङि \textcolor{red}{भवेत्} इव।\footnote{समारभ~\arrow \textcolor{red}{सर्वप्राति\-पदिकेभ्य आचारे क्विब्वा वक्तव्यः} (वा॰~३.१.११)~\arrow समारभ~क्विँप्~\arrow समारभ~व्~\arrow \textcolor{red}{वेरपृक्तस्य} (पा॰सू॰~६.१.६७)~\arrow समारभ~\arrow \textcolor{red}{सनाद्यन्ता धातवः} (पा॰सू॰~३.१.३२)~\arrow धातुसञ्ज्ञा~\arrow \textcolor{red}{शेषात्कर्तरि परस्मैपदम्} (पा॰सू॰~१.३.७८)~\arrow \textcolor{red}{विधि\-निमन्‍त्रणामन्‍त्रणाधीष्‍ट\-सम्प्रश्‍न\-प्रार्थनेषु लिङ्} (पा॰सू॰~३.३.१६१)~\arrow समारभ~लिङ~\arrow समारभ~तिप्~\arrow समारभ~ति~\arrow \textcolor{red}{कर्तरि शप्‌} (पा॰सू॰~३.१.६८)~\arrow समारभ~शप्~ति~\arrow समारभ~अ~ति~\arrow \textcolor{red}{अतो गुणे} (पा॰सू॰~६.१.९७)~\arrow समारभ~ति~\arrow \textcolor{red}{यासुट् परस्मैपदेषूदात्तो ङिच्च} (पा॰सू॰~३.४.१०३)~\arrow \textcolor{red}{आद्यन्तौ टकितौ} (पा॰सू॰~१.१.४६)~\arrow समारभ~यासुँट्~ति~\arrow समारभ~यास्~ति~\arrow \textcolor{red}{सुट् तिथोः} (पा॰सू॰~३.४.१०७)~\arrow \textcolor{red}{आद्यन्तौ टकितौ} (पा॰सू॰~१.१.४६)~\arrow समारभ~यास्~सुँट्~ति~\arrow समारभ~यास्~स्~ति~\arrow \textcolor{red}{लिङः सलोपोऽनन्त्यस्य} (पा॰सू॰~७.२.७९)~\arrow समारभ~या~ति~\arrow\textcolor{red}{अतो येयः} (पा॰सू॰~७.२.८०)~\arrow समारभ~इय्~ति~\arrow \textcolor{red}{लोपो व्योर्वलि} (पा॰सू॰~६.१.६६)~\arrow समारभ~इ~ति~\arrow \textcolor{red}{आद्गुणः} (पा॰सू॰~६.१.८७)~\arrow समारभे~ति~\arrow \textcolor{red}{इतश्च} (पा॰सू॰~३.४.१००)~\arrow समारभे~त्~\arrow समारभेत्। यद्वा \textcolor{red}{अनुदात्तेत्त्व\-लक्षणमात्मने\-पदमनित्यम्} (प॰शे॰~९३.४) इत्यपि समाधानम्।
}\end{sloppypar}
\section[प्रकारयेत्]{प्रकारयेत्}
\centering\textcolor{blue}{दशावरणपूजां वै ह्यागमोक्तां प्रकारयेत्।\nopagebreak\\
नीराजनैर्धूपदीपैर्नैवेद्यैर्बहुविस्तरैः॥}\nopagebreak\\
\raggedleft{–~अ॰रा॰~४.४.२९}\\
\fontsize{14}{21}\selectfont\begin{sloppypar}\hyphenrules{nohyphenation}\justifying\noindent\hspace{10mm} \textcolor{red}{प्र}\-पूर्वकात् \textcolor{red}{कृ}\-धातोः (\textcolor{red}{डुकृञ् करणे} धा॰पा॰~१४७२) विधिलिङि \textcolor{red}{प्रकुर्यात्}\footnote{प्र~\textcolor{red}{डुकृञ् करणे} (धा॰पा॰~१४७२)~\arrow प्र~कृ~\arrow \textcolor{red}{शेषात्कर्तरि परस्मैपदम्} (पा॰सू॰~१.३.७८)~\arrow \textcolor{red}{विधि\-निमन्‍त्रणामन्‍त्रणाधीष्‍ट\-सम्प्रश्‍न\-प्रार्थनेषु लिङ्} (पा॰सू॰~३.३.१६१)~\arrow प्र~कृ~लिङ्~\arrow प्र~कृ~तिप्~\arrow प्र~कृ~ति~\arrow \textcolor{red}{तनादि\-कृञ्भ्य उः} (पा॰सू॰~३.१.७९)~\arrow प्र~कृ~उ~ति~\arrow \textcolor{red}{सार्वधातुकार्ध\-धातुकयोः} (पा॰सू॰~७.३.८४)~\arrow \textcolor{red}{उरण् रपरः} (पा॰सू॰~१.१.५१)~\arrow प्र~कर्~उ~ति~\arrow \textcolor{red}{यासुट् परस्मैपदेषूदात्तो ङिच्च} (पा॰सू॰~३.४.१०३)~\arrow \textcolor{red}{आद्यन्तौ टकितौ} (पा॰सू॰~१.१.४६)~\arrow प्र~कर्~उ~यासुँट्~ति~\arrow प्र~कर्~उ~यास्~ति~\arrow \textcolor{red}{सुट् तिथोः} (पा॰सू॰~३.४.१०७)~\arrow \textcolor{red}{आद्यन्तौ टकितौ} (पा॰सू॰~१.१.४६)~\arrow प्र~कर्~उ~यास्~सुँट्~ति~\arrow प्र~कर्~उ~यास्~स्~ति~\arrow 
\textcolor{red}{लिङः सलोपोऽनन्त्यस्य} (पा॰सू॰~७.२.७९)~\arrow प्र~कर्~उ~या~ति~\arrow \textcolor{red}{अत उत्सार्वधातुके} (पा॰सू॰~६.४.११०)~\arrow प्र~कुर्~उ~या~ति~\arrow \textcolor{red}{ये च} (पा॰सू॰~६.४.१०९)~\arrow प्र~कुर्~या~ति~\arrow \textcolor{red}{इतश्च} (पा॰सू॰~३.४.१००)~\arrow प्र~कुर्~या~त्~\arrow प्रकुर्यात्।} इति प्रयोक्तव्ये \textcolor{red}{प्रकारयेत्} इति प्रयुक्तम्। \textcolor{red}{प्रकारमाचक्षीत} इति \textcolor{red}{प्रकारयेत्}। आचक्षाण\-णिजन्ताद्विधि\-लिङ्।\footnote{अत्र हेतुमण्णिञ्न। अर्थानुपपत्तेः। प्रकार~\arrow \textcolor{red}{तत्करोति तदाचष्टे} (धा॰पा॰ ग॰सू॰~१८७)~\arrow प्रकार~णिच्~\arrow प्रकार~इ~\arrow \textcolor{red}{णाविष्ठवत्प्राति\-पदिकस्य पुंवद्भाव\-रभाव\-टिलोप\-यणादि\-परार्थम्} (वा॰~६.४.४८)~\arrow प्रकार्~इ~\arrow प्रकारि~\arrow \textcolor{red}{सनाद्यन्ता धातवः} (पा॰सू॰~३.१.३२)~\arrow धातुसञ्ज्ञा~\arrow \textcolor{red}{शेषात्कर्तरि परस्मैपदम्} (पा॰सू॰~१.३.७८)~\arrow \textcolor{red}{विधि\-निमन्‍त्रणामन्‍त्रणाधीष्‍ट\-सम्प्रश्‍न\-प्रार्थनेषु लिङ्} (पा॰सू॰~३.३.१६१)~\arrow प्रकारि~लिङ~\arrow प्रकारि~तिप्~\arrow प्रकारि~ति~\arrow \textcolor{red}{कर्तरि शप्‌} (पा॰सू॰~३.१.६८)~\arrow प्रकारि~शप्~ति~\arrow प्रकारि~अ~ति~\arrow \textcolor{red}{सार्वधातुकार्ध\-धातुकयोः} (पा॰सू॰~७.३.८४)~\arrow प्रकारे~अ~ति~\arrow \textcolor{red}{यासुट् परस्मैपदेषूदात्तो ङिच्च} (पा॰सू॰~३.४.१०३)~\arrow \textcolor{red}{आद्यन्तौ टकितौ} (पा॰सू॰~१.१.४६)~\arrow प्रकारे~अ~यासुँट्~ति~\arrow प्रकारे~अ~यास्~ति~\arrow \textcolor{red}{सुट् तिथोः} (पा॰सू॰~३.४.१०७)~\arrow \textcolor{red}{आद्यन्तौ टकितौ} (पा॰सू॰~१.१.४६)~\arrow प्रकारे~अ~यास्~सुँट्~ति~\arrow प्रकारे~अ~यास्~स्~ति~\arrow \textcolor{red}{लिङः सलोपोऽनन्त्यस्य} (पा॰सू॰~७.२.७९)~\arrow प्रकारे~अ~या~ति~\arrow \textcolor{red}{अतो येयः} (पा॰सू॰~७.२.८०)~\arrow प्रकारे~अ~इय्~ति~\arrow \textcolor{red}{लोपो व्योर्वलि} (पा॰सू॰~६.१.६६)~\arrow प्रकारे~अ~इ~ति~\arrow \textcolor{red}{एचोऽयवायावः} (पा॰सू॰~६.१.७८)~\arrow प्रकारय्~अ~इ~ति~\arrow \textcolor{red}{आद्गुणः} (पा॰सू॰~६.१.८७)~\arrow प्रकारय्~ए~ति~\arrow \textcolor{red}{इतश्च} (पा॰सू॰~३.४.१००)~\arrow प्रकारय्~ए~त्~\arrow प्रकारयेत्।}\end{sloppypar}
\section[हन्यसे]{हन्यसे}
\centering\textcolor{blue}{करोमीति प्रतिज्ञाय सीतायाः परिमार्गणम्।\nopagebreak\\
न करोषि कृतघ्नस्त्वं हन्यसे वालिवद्द्रुतम्॥}\nopagebreak\\
\raggedleft{–~अ॰रा॰~४.४.४८}\\
\fontsize{14}{21}\selectfont\begin{sloppypar}\hyphenrules{nohyphenation}\justifying\noindent\hspace{10mm} शीघ्रता\-बोधनार्थं वर्तमान\-सामीप्ये लट्।\footnote{\textcolor{red}{वर्तमान\-सामीप्ये वर्तमानवद्वा} (पा॰सू॰~३.३.१३१) इत्यनेन। \textcolor{red}{हनँ हिंसागत्योः} (धा॰पा॰~१०१२)~\arrow हन्~\arrow \textcolor{red}{भावकर्मणोः} (पा॰सू॰~१.३.१३)~\arrow \textcolor{red}{वर्तमान\-सामीप्ये वर्तमानवद्वा} (पा॰सू॰~३.३.१३१)~\arrow \textcolor{red}{वर्तमाने लट्} (पा॰सू॰~३.२.१२३)~\arrow हन्~लट्~\arrow हन्~थास्~\arrow \textcolor{red}{सार्वधातुके यक्} (पा॰सू॰~३.१.६७)~\arrow हन्~यक्~थास्~\arrow हन्~य~थास्~\arrow \textcolor{red}{थासस्से} (पा॰सू॰~३.४.८०)~\arrow हन्~य~से~\arrow हन्यसे। }\end{sloppypar}
\section[वधयिष्यति]{वधयिष्यति}
\label{sec:vadhayisyati}
\centering\textcolor{blue}{सुग्रीवः स्वयमागत्य सर्ववानरयूथपैः।\nopagebreak\\
वधयिष्यति दैत्यौघान् रावणं च हनिष्यति॥}\nopagebreak\\
\raggedleft{–~अ॰रा॰~४.५.४७}\\
\fontsize{14}{21}\selectfont\begin{sloppypar}\hyphenrules{nohyphenation}\justifying\noindent\hspace{10mm} \textcolor{red}{वधमाचरन्तीति वधन्ति}।\footnote{वध~\arrow \textcolor{red}{सर्वप्राति\-पदिकेभ्य आचारे क्विब्वा वक्तव्यः} (वा॰~३.१.११)~\arrow वध~क्विँप्~\arrow वध~व्~\arrow \textcolor{red}{वेरपृक्तस्य} (पा॰सू॰~६.१.६७)~\arrow वध~\arrow \textcolor{red}{सनाद्यन्ता धातवः} (पा॰सू॰~३.१.३२)~\arrow धातुसञ्ज्ञा~\arrow \textcolor{red}{शेषात्कर्तरि परस्मैपदम्} (पा॰सू॰~१.३.७८)~\arrow \textcolor{red}{वर्तमाने लट्} (पा॰सू॰~३.२.१२३)~\arrow वध~लट्~\arrow वध~झि~\arrow \textcolor{red}{झोऽन्तः} (पा॰सू॰~७.१.३)~\arrow वध~अन्ति~\arrow \textcolor{red}{कर्तरि शप्‌} (पा॰सू॰~३.१.६८)~\arrow वध~शप्~अन्ति~\arrow वध~अ~अन्ति~\arrow \textcolor{red}{अतो गुणे} (पा॰सू॰~६.१.९७)~\arrow वध~अन्ति~\arrow \textcolor{red}{अतो गुणे} (पा॰सू॰~६.१.९७)~\arrow वधन्ति।} \textcolor{red}{तान् प्रेरयति वधयति}।\footnote{वध~\arrow धातुसञ्ज्ञा (पूर्ववत्)~\arrow \textcolor{red}{हेतुमति च} (पा॰सू॰~३.१.२६)~\arrow वध~णिच्~\arrow वध~इ~\arrow \textcolor{red}{णाविष्ठवत्प्राति\-पदिकस्य पुंवद्भाव\-रभाव\-टिलोप\-यणादि\-परार्थम्} (वा॰~६.४.४८)~\arrow वध्~इ~\arrow वधि~\arrow \textcolor{red}{सनाद्यन्ता धातवः} (पा॰सू॰~३.१.३२)~\arrow धातु\-सञ्ज्ञा~\arrow \textcolor{red}{शेषात्कर्तरि परस्मैपदम्} (पा॰सू॰~१.३.७८)~\arrow \textcolor{red}{वर्तमाने लट्} (पा॰सू॰~३.२.१२३)~\arrow वधि~लट्~\arrow वधि~तिप्~\arrow वधि~ति~\arrow \textcolor{red}{कर्तरि शप्‌} (पा॰सू॰~३.१.६८)~\arrow वधि~शप्~ति~\arrow वधि~अ~ति~\arrow \textcolor{red}{सार्वधातुकार्ध\-धातुकयोः} (पा॰सू॰~७.३.८४)~\arrow वधे~अ~ति~\arrow \textcolor{red}{एचोऽयवायावः} (पा॰सू॰~६.१.७८)~\arrow वधय्~अ~ति~\arrow वधयति।} तद्भविष्यत्काले \textcolor{red}{वधयिष्यति}।\footnote{वधि~\arrow \textcolor{red}{धातु\-सञ्ज्ञा} (पूर्ववत्)~\arrow \textcolor{red}{शेषात्कर्तरि परस्मैपदम्} (पा॰सू॰~१.३.७८)~\arrow \textcolor{red}{लृट् शेषे च} (पा॰सू॰~३.३.१३)~\arrow वधि~लृट्~\arrow वधि~तिप्~\arrow वधि~ति~\arrow \textcolor{red}{स्यतासी लृलुटोः} (पा॰सू॰~३.१.३३)~\arrow वधि~स्य~ति~\arrow \textcolor{red}{आर्धधातुकस्येड्वलादेः} (पा॰सू॰~७.२.३५)~\arrow वधि~इट्~स्य~ति~\arrow वधि~इ~स्य~ति~\arrow \textcolor{red}{सार्वधातुकार्ध\-धातुकयोः} (पा॰सू॰~७.३.८४)~\arrow वधे~इ~स्य~ति~\arrow \textcolor{red}{एचोऽयवायावः} (पा॰सू॰~६.१.७८)~\arrow वधय्~इ~स्य~ति~\arrow \textcolor{red}{आदेश\-प्रत्यययोः} (पा॰सू॰~८.३.५९)~\arrow वधय्~इ~ष्य~ति~\arrow वधयिष्यति।} यद्वा \textcolor{red}{वध्} इति स्वतन्त्रश्चौरादिको धातुः। तस्माल्लृटि \textcolor{red}{वधयिष्यति}। \textcolor{red}{मितां ह्रस्वः} (पा॰सू॰~६.४.९२) इति ह्रस्वः।\footnote{\textcolor{red}{बहुलमेतन्निदर्शनम्} (धा॰पा॰ ग॰सू॰~१९३८) \textcolor{red}{आकृतिगणोऽयम्} (धा॰पा॰ ग॰सू॰~१९९२) \textcolor{red}{भूवादिष्वेतदन्तेषु दशगणीषु धातूनां पाठो निदर्शनाय तेन स्तम्भुप्रभृतयः सौत्राश्चुलुम्पादयो वाक्यकारीयाः प्रयोगसिद्धा विक्लवत्यादयश्च} (मा॰धा॰वृ॰~१०.३२८) इत्यनुसारमाकृति\-गणत्वाच्चुरादि\-गण ऊह्योऽयं नामधातुः। \textcolor{red}{सङ्ग्राम युद्धे} (धा॰पा॰~१९२२) \textcolor{red}{संवर संवरणे} (धा॰पा॰~१९९२) इतिवत्। वृद्ध्यभावात् \textcolor{red}{ज्ञपँ ज्ञान\-ज्ञापन\-मारण\-तोषण\-निशान\-निशामनेषु} (धा॰पा॰~१६२४) इतिवन्मिदप्ययम्। वध्~\arrow \textcolor{red}{सत्याप\-पाश\-रूप\-वीणा\-तूल\-श्लोक\-सेना\-लोम\-त्वच\-वर्म\-वर्ण\-चूर्ण\-चुरादिभ्यो णिच्} (पा॰सू॰~३.१.२५)~\arrow वध्~णिच्~\arrow वध्~इ~\arrow \textcolor{red}{अत उपधायाः} (पा॰सू॰~७.२.११६)~\arrow वाध्~इ~\arrow \textcolor{red}{मितां ह्रस्वः} (पा॰सू॰~६.४.९२)~\arrow वध्~इ~\arrow वधि~\arrow \textcolor{red}{सनाद्यन्ता धातवः} (पा॰सू॰~३.१.३२)~\arrow धातु\-सञ्ज्ञा~\arrow शेषा प्रक्रिया पूर्ववत्।} यद्वा \textcolor{red}{अयनमय्}। भावे क्विप्।\footnote{\textcolor{red}{अयँ गतौ} (धा॰पा॰~४७४)~\arrow अय्~\arrow \textcolor{red}{सम्पदादिभ्‍यः क्विप्} (वा॰~३.३.१०८)~\arrow अय्~क्विँप्~\arrow अय्~व्~\arrow \textcolor{red}{वेरपृक्तस्य} (पा॰सू॰~६.१.६७)~\arrow अय्~\arrow विभक्ति\-कार्यम्~\arrow अय्~सुँ~\arrow \textcolor{red}{हल्ङ्याब्भ्यो दीर्घात्सुतिस्यपृक्तं हल्} (पा॰सू॰~६.१.६८)~\arrow \textcolor{red}{अय्}।} \textcolor{red}{वधस्य अय् इति वधय्} शकन्ध्वादित्वात्पर\-रूपे।\footnote{\textcolor{red}{शकन्ध्वादिषु पररूपं वाच्यम्} (वा॰~६.१.९४) इत्यनेन।} \textcolor{red}{इषुँ इच्छायाम्} (धा॰पा॰~१३५१) दिवादिः। \textcolor{red}{वधय् वध\-प्राप्तिमिच्छतीष्यति वधयिष्यति}।\footnote{\textcolor{red}{लिङ्गमशिष्यं लोकाश्रयत्वाल्लिङ्गस्य} (भा॰पा॰सू॰~२.१.३६) इति नियमेन \textcolor{red}{अय्} इति शब्दं नपुंसक\-लिङ्गे पठित्वा द्वितीयायां विभक्तौ अय्~अम् इति स्थिते \textcolor{red}{स्वमोर्नपुंसकात्‌} (पा॰सू॰~७.१.२३) इत्यनेनाम्लुकि \textcolor{red}{अय्} इत्येव। \pageref{sec:vadhisyami}तमे पृष्ठे \ref{sec:vadhisyami} \nameref{sec:vadhisyami} इति प्रयोगस्य विमर्शमपि पश्यन्तु।}\end{sloppypar}
\section[प्रधर्षथ]{प्रधर्षथ}
\centering\textcolor{blue}{कुतो वा कस्य दूता वा मत्स्थानं किं प्रधर्षथ।\nopagebreak\\
तच्छ्रुत्वा हनुमानाह श्रृणु वक्ष्यामि देवि ते॥}\nopagebreak\\
\raggedleft{–~अ॰रा॰~४.६.४२}\\
\fontsize{14}{21}\selectfont\begin{sloppypar}\hyphenrules{nohyphenation}\justifying\noindent\hspace{10mm} ण्यजन्तत्यागे रूपमिदम्।\footnote{\textcolor{red}{धृष्‌}\-धातुः (\textcolor{red}{धृषँ प्रसहने} धा॰पा॰~१८५०) आधृषीयान्तर्गणे पठितः। तत्र \textcolor{red}{आ धृषाद्वा} (धा॰पा॰ ग॰सू॰~१८०६) इति गणसूत्रस्याधिकाराद्वैकल्पिक\-णिच्प्रत्ययः। णिजभावे \textcolor{red}{प्रधर्षथ} इति रूपम्। प्र~\textcolor{red}{धृषँ प्रसहने}~\arrow प्र~धृष्~\arrow \textcolor{red}{शेषात्कर्तरि परस्मैपदम्} (पा॰सू॰~१.३.७८)~\arrow \textcolor{red}{वर्तमाने लट्} (पा॰सू॰~३.२.१२३)~\arrow प्र~धृष्~लट्~\arrow प्र~धृष्~थ~\arrow \textcolor{red}{कर्तरि शप्‌} (पा॰सू॰~३.१.६८)~\arrow प्र~धृष्~शप्~थ~\arrow प्र~धृष्~अ~थ~\arrow \textcolor{red}{पुगन्त\-लघूपधस्य च} (पा॰सू॰~७.३.८६)~\arrow \textcolor{red}{उरण् रपरः} (पा॰सू॰~१.१.५१)~\arrow प्र~धर्ष्~अ~थ~\arrow प्रधर्षथ। णिचि तु \textcolor{red}{प्रधर्षयथ} इति रूपम्। धृष्~\arrow \textcolor{red}{सत्याप\-पाश\-रूप\-वीणा\-तूल\-श्लोक\-सेना\-लोम\-त्वच\-वर्म\-वर्ण\-चूर्ण\-चुरादिभ्यो णिच्} (पा॰सू॰~३.१.२५)~\arrow धृष्~णिच्~\arrow धृष्~इ~\arrow \textcolor{red}{पुगन्त\-लघूपधस्य च} (पा॰सू॰~७.३.८६)~\arrow \textcolor{red}{उरण् रपरः} (पा॰सू॰~१.१.५१)~\arrow धर्ष्~इ~\arrow धर्षि~\arrow \textcolor{red}{सनाद्यन्ता धातवः} (पा॰सू॰~३.१.३२)~\arrow धातु\-सञ्ज्ञा। प्र~धर्षि~\arrow \textcolor{red}{शेषात्कर्तरि परस्मैपदम्} (पा॰सू॰~१.३.७८)~\arrow \textcolor{red}{वर्तमाने लट्} (पा॰सू॰~३.२.१२३)~\arrow प्र~धर्षि~लट्~\arrow प्र~धर्षि~थ~\arrow \textcolor{red}{कर्तरि शप्‌} (पा॰सू॰~३.१.६८)~\arrow प्र~धर्षि~शप्~थ~\arrow प्र~धर्षि~अ~थ~\arrow \textcolor{red}{सार्वधातुकार्धधातुकयोः} (पा॰सू॰~७.३.८४)~\arrow प्र~धर्षे~अ~थ~\arrow \textcolor{red}{एचोऽयवायावः} (पा॰सू॰~६.१.७८)~\arrow प्र~धर्षय्~अ~थ~\arrow प्रधर्षयथ।}\end{sloppypar}
\section[मृगयध्वम्]{मृगयध्वम्}
\centering\textcolor{blue}{मृगयध्वमिति प्राह ततो वयमुपागताः।\nopagebreak\\
ततो वनं विचिन्वन्तो जानकीं जलकाङ्क्षिणः॥}\nopagebreak\\
\raggedleft{–~अ॰रा॰~४.६.४६}\\
\fontsize{14}{21}\selectfont\begin{sloppypar}\hyphenrules{nohyphenation}\justifying\noindent\hspace{10mm} अत्र श्रीरामेणाऽत्मीयत्वात्क्रिया\-फलस्य वानर\-रूप\-कर्तृ\-गामित्व आत्मनेपदम्।\footnote{\textcolor{red}{मृग अन्वेषणे} (धा॰पा॰~१९००) चौरादिक\-धातुरागर्वीयत्वादात्मने\-पदी। परन्तु \textcolor{red}{मृगणां कुरुत} इति प्राकृतेऽर्थे णिचि कृते तूभयपदी। यथा – \textcolor{red}{निवृत्तप्रेषणाद्धातोः प्राकृतेऽर्थे णिजुच्यते} (वा॰प॰~३.७.६०)। तत्रात्मनेपद\-प्राप्तेः कारणमिदं प्रादर्शि ग्रन्थकारैः। मृग~णिच्~\arrow मृग~इ~\arrow \textcolor{red}{अतो लोपः} (पा॰सू॰~६.४.४८)~\arrow मृग्~इ~\arrow \textcolor{red}{सनाद्यन्ता धातवः} (पा॰सू॰~३.१.३२)~\arrow \textcolor{red}{णिचश्च} (पा॰सू॰~१.३.७४)~\arrow \textcolor{red}{लोट् च} (पा॰सू॰~३.३.१६२)~\arrow मृग्~इ~लोट्~\arrow मृग्~इ~ध्वम्~\arrow \textcolor{red}{कर्तरि शप्} (पा॰सू॰~३.१.६८)~\arrow मृग्~इ~शप्~ध्वम्~\arrow मृग्~इ~अ~ध्वम्~\arrow \textcolor{red}{सार्वधातुकार्ध\-धातुकयोः} (पा॰सू॰~७.३.८४)~\arrow मृग्~ए~अ~ध्वम्~\arrow \textcolor{red}{एचोऽयवायावः} (पा॰सू॰~६.१.७८)~\arrow मृग्~अय्~अ~ध्वम्~\arrow मृगयध्वम्।}\end{sloppypar}
\section[स्थास्यामहे]{स्थास्यामहे}
\centering\textcolor{blue}{पुनर्वैकुण्ठमासाद्य सुखं स्थास्यामहे वयम्।\nopagebreak\\
इत्यङ्गदमथाऽश्वास्य गता विन्ध्यं महाचलम्॥}\nopagebreak\\
\raggedleft{–~अ॰रा॰~४.७.२२}\\
\fontsize{14}{21}\selectfont\begin{sloppypar}\hyphenrules{nohyphenation}\justifying\noindent\hspace{10mm} सीतान्वेषणेऽसफलान् मरणे कृत\-निश्चयान् वानरान् हनुमान् प्रतिबोधयति \textcolor{red}{वैकुण्ठे सुखं स्थास्यामहे}। इह \textcolor{red}{उपान्मन्त्र\-करणे} (पा॰सू॰~१.३.२५) इत्यनेनाऽत्मनेपदम्।\footnote{उप~\textcolor{red}{ष्ठा गतिनिवृत्तौ} (धा॰पा॰~९२८)~\arrow \textcolor{red}{धात्वादेः षः सः} (पा॰सू॰~६.१.६४)~\arrow निमित्तापाये नैमित्तिकस्याप्यपायः~\arrow उप~स्था~\arrow \textcolor{red}{उपान्मन्त्र\-करणे} (पा॰सू॰~१.३.२५)~\arrow \textcolor{red}{लृट् शेषे च} (पा॰सू॰~३.३.१३)~\arrow उप~लृँट्~\arrow उप~स्था~महिङ्~\arrow उप~स्था~महि~\arrow \textcolor{red}{स्यतासी लृलुटोः} (पा॰सू॰~३.१.३३)~\arrow उप~स्था~स्य~महि~\arrow \textcolor{red}{अतो दीर्घो यञि} (पा॰सू॰~७.३.१०१)~\arrow उप~स्था~स्या~महि~\arrow \textcolor{red}{टित आत्मनेपदानां टेरे} (पा॰सू॰~३.४.७९)~\arrow उप~स्था~स्या~महे~\arrow उपस्थास्यामहे।} उप\-शब्दस्य लोपः।\footnote{\textcolor{red}{विनाऽपि प्रत्ययं पूर्वोत्तर\-पद\-लोपो वक्तव्यः} (वा॰~५.३.८३) इत्यनेन।}\end{sloppypar}
\section[नयध्वम्]{नयध्वम्}
\centering\textcolor{blue}{वाक्यसाहाय्यं करिष्येऽहं भवतां प्लवगेश्वराः।\nopagebreak\\
भ्रातुः सलिलदानाय नयध्वं मां जलान्तिकम्॥}\nopagebreak\\
\raggedleft{–~अ॰रा॰~४.७.४८}\\
\fontsize{14}{21}\selectfont\begin{sloppypar}\hyphenrules{nohyphenation}\justifying\noindent\hspace{10mm} इह \textcolor{red}{सम्माननोत्सञ्जनाचार्य\-करण\-ज्ञान\-भृति\-विगणन\-व्ययेषु नियः} (पा॰सू॰~१.३.३६) इत्यनेनाऽत्मनेपदम्। सम्माननेऽयं प्रयोगः। \textcolor{red}{नयध्वम्}।\footnote{\textcolor{red}{णीञ् प्रापणे} (धा॰पा॰~९०१)~\arrow \textcolor{red}{णो नः} (पा॰सू॰~६.१.६५)~\arrow नीञ्~\arrow नी~\arrow \textcolor{red}{सम्माननोत्सञ्जनाचार्य\-करण\-ज्ञान\-भृति\-विगणन\-व्ययेषु नियः} (पा॰सू॰~१.३.३६)~\arrow \textcolor{red}{लोट् च} (पा॰सू॰~३.३.१६२)~\arrow नी~लोट्~\arrow नी~ध्वम्~\arrow \textcolor{red}{कर्तरि शप्} (पा॰सू॰~३.१.६८)~\arrow नी~शप्~ध्वम्~\arrow नी~अ~ध्वम्~\arrow \textcolor{red}{सार्वधातुकार्ध\-धातुकयोः} (पा॰सू॰~७.३.८४)~\arrow ने~अ~ध्वम्~\arrow \textcolor{red}{एचोऽयवायावः} (पा॰सू॰~६.१.७८)~\arrow नय्~अ~ध्वम्~\arrow नयध्वम्।} \textcolor{red}{सम्मानं कुरुत}। तदिच्छापालनमेव सम्माननम्। उत्सञ्जने वाऽऽत्मनेपदम्।\end{sloppypar}
\section[इयात्]{इयात्}
\centering\textcolor{blue}{तत्पुनः पञ्चरात्रेण बुद्बुदाकारतामियात्।\nopagebreak\\
सप्तरात्रेण तदपि मांसपेशित्वमाप्नुयात्॥}\nopagebreak\\
\raggedleft{–~अ॰रा॰~४.८.२३}\\
\fontsize{14}{21}\selectfont\begin{sloppypar}\hyphenrules{nohyphenation}\justifying\noindent\hspace{10mm} \textcolor{red}{एति}\footnote{\textcolor{red}{इण् गतौ} (पा॰सू॰~१०४५)~\arrow इ~\arrow \textcolor{red}{शेषात्कर्तरि परस्मैपदम्} (पा॰सू॰~१.३.७८)~\arrow \textcolor{red}{वर्तमाने लट्} (पा॰सू॰~३.२.१२३)~\arrow इ~लट्~\arrow इ~तिप्~\arrow इ~ति~\arrow \textcolor{red}{कर्तरि शप्‌} (पा॰सू॰~३.१.६८)~\arrow इ~अ~ति~\arrow \textcolor{red}{अदिप्रभृतिभ्यः शपः} (पा॰सू॰~२.४.७२)~\arrow इ~ति~\arrow \textcolor{red}{सार्वधातुकार्ध\-धातुकयोः} (पा॰सू॰~७.३.८४)~\arrow ए~ति~\arrow एति।} इति प्रयोक्तव्ये \textcolor{red}{इयात्}\footnote{\textcolor{red}{इण् गतौ} (पा॰सू॰~१०४५)~\arrow इ~\arrow \textcolor{red}{शेषात्कर्तरि परस्मैपदम्} (पा॰सू॰~१.३.७८)~\arrow \textcolor{red}{आशंसावचने लिङ्} (पा॰सू॰~३.३.१३४)~\arrow इ~लिङ्~\arrow इ~तिप्~\arrow इ~ति~\arrow \textcolor{red}{यासुट् परस्मैपदेषूदात्तो ङिच्च} (पा॰सू॰~३.४.१०३)~\arrow \textcolor{red}{आद्यन्तौ टकितौ} (पा॰सू॰~१.१.४६)~\arrow इ~यासुँट्~ति~\arrow इ~यास्~ति~\arrow \textcolor{red}{सुट् तिथोः} (पा॰सू॰~३.४.१०७)~\arrow \textcolor{red}{आद्यन्तौ टकितौ} (पा॰सू॰~१.१.४६)~\arrow इ~यास्~सुँट्~ति~\arrow इ~यास्~स्~ति~\arrow \textcolor{red}{लिङः सलोपोऽनन्त्यस्य} (पा॰सू॰~७.२.७९)~\arrow इ~या~ति~\arrow \textcolor{red}{ग्क्ङिति च} (पा॰सू॰~१.१.५)~\arrow गुणनिषेधः~\arrow इ~या~ति~\arrow \textcolor{red}{इतश्च} (पा॰सू॰~३.४.१००)~\arrow इ~या~त्~\arrow इयात्।} इति। \textcolor{red}{आशंसावचने लिङ्} (पा॰सू॰~३.३.१३४) इत्यनेन लिङ्लकारः।\footnote{एवमेव \textcolor{red}{आप्नुयात्} इत्यत्रापि बोध्यम्।}\end{sloppypar}
\section[म्रियेत]{म्रियेत}
\centering\textcolor{blue}{नाभिसूत्राल्परन्ध्रेण मातृभुक्तान्नसारतः।\nopagebreak\\
वर्धते गर्भतः पिण्डो न म्रियेत स्वकर्मतः॥}\nopagebreak\\
\raggedleft{–~अ॰रा॰~४.८.३२}\\
\fontsize{14}{21}\selectfont\begin{sloppypar}\hyphenrules{nohyphenation}\justifying\noindent\hspace{10mm} \textcolor{red}{म्रियते}\footnote{\textcolor{red}{मृङ् प्राणत्यागे} (धा॰पा॰~१४०३)~\arrow मृ~\arrow \textcolor{red}{म्रियतेर्लुङ्‌लिङोश्च} (पा॰सू॰~१.३.६१)~\arrow \textcolor{red}{वर्तमाने लट्} (पा॰सू॰~३.२.१२३)~\arrow मृ~लट्~\arrow मृ~त~\arrow \textcolor{red}{तुदादिभ्यः शः} (पा॰सू॰~३.१.७७)~\arrow मृ~श~त~\arrow मृ~अ~त~\arrow \textcolor{red}{रिङ् शयग्लिङ्क्षु} (पा॰सू॰~७.४.२८)~\arrow \textcolor{red}{ङिच्च} (पा॰सू॰~१.१.५३)~\arrow म्~रिङ्~अ~त~\arrow म्~रि~अ~त~\arrow \textcolor{red}{अचि श्नुधातुभ्रुवां य्वोरियङुवङौ} (पा॰सू॰~६.४.७७)~\arrow \textcolor{red}{ङिच्च} (पा॰सू॰~१.१.५३)~\arrow म्~र्~इयँङ्~अ~त~\arrow म्~र्~इय्~अ~त~\arrow \textcolor{red}{टित आत्मनेपदानां टेरे} (पा॰सू॰~३.४.७९)~\arrow म्~र्~इय्~अ~ते~\arrow म्रियते।} इति प्रयोक्तव्ये \textcolor{red}{म्रियेत} इति प्रयोगस्तु \textcolor{red}{शकि लिङ् च} (पा॰सू॰~३.३.१७२) इति सूत्रेण शक्यार्थे लिङ्प्रयोगे साधु।\footnote{\textcolor{red}{मृङ् प्राणत्यागे} (धा॰पा॰~१४०३)~\arrow मृ~\arrow \textcolor{red}{म्रियतेर्लुङ्‌लिङोश्च} (पा॰सू॰~१.३.६१)~\arrow \textcolor{red}{शकि लिङ् च} (पा॰सू॰~३.३.१७२)~\arrow मृ~लिङ्~\arrow मृ~त~\arrow \textcolor{red}{तुदादिभ्यः शः} (पा॰सू॰~३.१.७७)~\arrow मृ~श~त~\arrow मृ~अ~त~\arrow \textcolor{red}{रिङ् शयग्लिङ्क्षु} (पा॰सू॰~७.४.२८)~\arrow \textcolor{red}{ङिच्च} (पा॰सू॰~१.१.५३)~\arrow म्~रिङ्~अ~त~\arrow म्~रि~अ~त~\arrow \textcolor{red}{अचि श्नुधातुभ्रुवां य्वोरियङुवङौ} (पा॰सू॰~६.४.७७)~\arrow \textcolor{red}{ङिच्च} (पा॰सू॰~१.१.५३)~\arrow म्~र्~इयँङ्~अ~त~\arrow म्~र्~इय्~अ~त~\arrow \textcolor{red}{लिङः सीयुट्} (पा॰सू॰~३.४.१०२)~\arrow म्~र्~इय्~अ~सीयुँट्~त~\arrow म्~र्~इय्~अ~सीय्~त~\arrow \textcolor{red}{सुट् तिथोः} (पा॰सू॰~३.४.१०७)~\arrow \textcolor{red}{आद्यन्तौ टकितौ} (पा॰सू॰~१.१.४६)~\arrow म्~र्~इय्~अ~सीय्~सुँट्~त~\arrow म्~र्~इय्~अ~सीय्~स्~त~\arrow \textcolor{red}{लिङः सलोपोऽनन्त्यस्य} (पा॰सू॰~७.२.७९)~\arrow म्~र्~इय्~अ~ईय्~त~\arrow \textcolor{red}{लोपो व्योर्वलि} (पा॰सू॰~६.१.६६)~\arrow म्~र्~इय्~अ~ई~त~\arrow \textcolor{red}{आद्गुणः} (पा॰सू॰~६.१.८७)~\arrow म्~र्~इय्~ए~त~\arrow म्रियेत।}\end{sloppypar}
\section[क्षिपामि]{क्षिपामि}
\centering\textcolor{blue}{लङ्कां सपर्वतां धृत्वा रामस्याग्रे क्षिपाम्यहम्।\nopagebreak\\
यद्वा दृष्ट्वैव यास्यामि जानकीं शुभलक्षणाम्॥}\nopagebreak\\
\raggedleft{–~अ॰रा॰~४.९.२४}\\
\fontsize{14}{21}\selectfont\begin{sloppypar}\hyphenrules{nohyphenation}\justifying\noindent\hspace{10mm} \textcolor{red}{वर्तमान\-सामीप्ये वर्तमानवद्वा} (पा॰सू॰~३.३.१३१) इत्यनेन लट्प्रयोगः।\footnote{\textcolor{red}{क्षिपँ प्रेरणे} (धा॰पा॰~१२८५)~\arrow क्षिप्~\arrow \textcolor{red}{शेषात्कर्तरि परस्मैपदम्} (पा॰सू॰~१.३.७८)~\arrow \textcolor{red}{वर्तमान\-सामीप्ये वर्तमानवद्वा} (पा॰सू॰~३.३.१३१)~\arrow \textcolor{red}{वर्तमाने लट्} (पा॰सू॰~३.२.१२३)~\arrow क्षिप्~लट्~\arrow क्षिप्~मिप्~\arrow क्षिप्~मि~\arrow \textcolor{red}{तुदादिभ्यः शः} (पा॰सू॰~३.१.७७)~\arrow क्षिप्~श~मि~\arrow क्षिप्~अ~मि~\arrow \textcolor{red}{सार्वधातुकमपित्} (पा॰सू॰~१.२.४)~\arrow ङित्त्वम्~\arrow \textcolor{red}{ग्क्ङिति च} (पा॰सू॰~१.१.५)~\arrow लघूपध\-गुण\-निषेधः~\arrow क्षिप्~अ~मि~\arrow \textcolor{red}{अतो दीर्घो यञि} (पा॰सू॰~७.३.१०१)~\arrow क्षिप्~आ~मि~\arrow क्षिपामि।}\end{sloppypar}
\vspace{2mm}
\centering ॥ इति किष्किन्धाकाण्डीयप्रयोगाणां विमर्शः ॥\nopagebreak\\
\vspace{4mm}
\pdfbookmark[2]{सुन्दरकाण्डम्}{Chap3Part2Kanda5}
\phantomsection
\addtocontents{toc}{\protect\setcounter{tocdepth}{2}}
\addcontentsline{toc}{subsection}{सुन्दरकाण्डीयप्रयोगाणां विमर्शः}
\addtocontents{toc}{\protect\setcounter{tocdepth}{0}}
\centering ॥ अथ सुन्दरकाण्डीयप्रयोगाणां विमर्शः ॥\nopagebreak\\
\section[पश्यामि]{पश्यामि}
\centering\textcolor{blue}{अमोघं रामनिर्मुक्तं महाबाणमिवाखिलाः।\nopagebreak\\
पश्याम्यद्यैव रामस्य पत्नीं जनकनन्दिनीम्॥}\nopagebreak\\
\raggedleft{–~अ॰रा॰~५.१.३}\\
\centering\textcolor{blue}{कृतार्थोऽहं कृतार्थोऽहं पुनः पश्यामि राघवम्।\nopagebreak\\
प्राणप्रयाणसमये यस्य नाम सकृत्स्मरन्॥}\nopagebreak\\
\raggedleft{–~अ॰रा॰~५.१.४}\\
\fontsize{14}{21}\selectfont\begin{sloppypar}\hyphenrules{nohyphenation}\justifying\noindent\hspace{10mm} अत्र शीघ्रता\-द्योतनाय वर्तमान\-समीपे भविष्यति \textcolor{red}{वर्तमान\-सामीप्ये वर्तमानवद्वा} (पा॰सू॰~३.३.१३१) इत्यनेन लट्।\footnote{\textcolor{red}{दृशिँर प्रेक्षणे} (धा॰पा॰~९८८)~\arrow दृश्~\arrow \textcolor{red}{शेषात्कर्तरि परस्मैपदम्} (पा॰सू॰~१.३.७८)~\arrow \textcolor{red}{वर्तमान\-सामीप्ये वर्तमानवद्वा} (पा॰सू॰~३.३.१३१)~\arrow \textcolor{red}{वर्तमाने लट्} (पा॰सू॰~३.२.१२३)~\arrow दृश्~लट्~\arrow दृश्~मिप्~\arrow दृश्~मि~\arrow \textcolor{red}{कर्तरि शप्‌} (पा॰सू॰~३.१.६८)~\arrow दृश्~शप्~मि~\arrow दृश्~अ~मि~\arrow \textcolor{red}{पाघ्रा\-ध्मास्थाम्ना\-दाण्दृश्यर्त्ति\-सर्त्तिशदसदां पिब\-जिघ्र\-धम\-तिष्ठ\-मन\-यच्छ\-पश्यर्च्छ\-धौ\-शीय\-सीदाः} (पा॰सू॰~७.३.७८)~\arrow पश्य्~अ~मि~\arrow \textcolor{red}{अतो दीर्घो यञि} (पा॰सू॰~७.३.१०१)~\arrow पश्य्~आ~मि~\arrow पश्यामि।}\end{sloppypar}
\section[विवेक्ष्ये]{विवेक्ष्ये}
\centering\textcolor{blue}{विवेक्ष्ये\footnote{\textcolor{red}{निवेक्ष्ये} इति पाठभेदः। तत्र तु \textcolor{red}{नेर्विशः} (पा॰सू॰~१.३.१७) इत्यनेनात्मनेपदम्।} देहि मे मार्गं सुरसायै नमोऽस्तु ते।\nopagebreak\\
इत्युक्त्वा पुनरेवाह सुरसा क्षुधितास्म्यहम्॥}\nopagebreak\\
\raggedleft{–~अ॰रा॰~५.१.१६}\\
\fontsize{14}{21}\selectfont\begin{sloppypar}\hyphenrules{nohyphenation}\justifying\noindent\hspace{10mm} कर्म\-व्यतिहार आत्मनेपदम्।\footnote{\textcolor{red}{कर्तरि कर्मव्यतिहारे} (पा॰सू॰~१.३.१४) इत्यनेन। वि~\textcolor{red}{विशँ प्रवेशने} (धा॰पा॰~१४२४)~\arrow वि~विश्~\arrow \textcolor{red}{कर्तरि कर्म\-व्यतिहारे} (पा॰सू॰~१.३.१४)~\arrow \textcolor{red}{लृट् शेषे च} (पा॰सू॰~३.३.१३)~\arrow वि~विश्~लृट्~\arrow वि~विश्~इट्~\arrow वि~विश्~इ~\arrow \textcolor{red}{स्यतासी लृलुटोः} (पा॰सू॰~३.१.३३)~\arrow वि~विश्~स्य~इ~\arrow \textcolor{red}{पुगन्त\-लघूपधस्य च} (पा॰सू॰~७.३.८६)~\arrow वि~वेश्~स्य~इ~\arrow \textcolor{red}{व्रश्चभ्रस्ज\-सृजमृज\-यजराज\-भ्राजच्छशां षः} (पा॰सू॰~८.२.३६)~\arrow वि~वेष्~स्य~इ~\arrow \textcolor{red}{षढोः कः सि} (पा॰सू॰~८.२.४१)~\arrow वि~वेक्~स्य~इ~\arrow \textcolor{red}{आदेश\-प्रत्यययोः} (पा॰सू॰~८.३.५९)~\arrow वि~वेक्~ष्य~इ~\arrow \textcolor{red}{टित आत्मनेपदानां टेरे} (पा॰सू॰~३.४.७९)~\arrow वि~वेक्~ष्य~ए~\arrow \textcolor{red}{अतो गुणे} (पा॰सू॰~६.१.९७)~\arrow वि~वेक्~ष्ये~\arrow विवेक्ष्ये।}\end{sloppypar}
\section[भक्षयेत्]{भक्षयेत्}
\centering\textcolor{blue}{सिंहिका नाम सा घोरा जलमध्ये स्थिता सदा।\nopagebreak\\
आकाशगामिनां छायामाक्रम्याऽकृष्य भक्षयेत्॥}\nopagebreak\\
\raggedleft{–~अ॰रा॰~५.१.३५}\\
\fontsize{14}{21}\selectfont\begin{sloppypar}\hyphenrules{nohyphenation}\justifying\noindent\hspace{10mm} अत्र \textcolor{red}{विधि\-निमन्त्रणामन्त्रणाधीष्ट\-सम्प्रश्न\-प्रार्थनेषु लिङ्} (पा॰सू॰~३.३.१६१) इत्यनेन \textcolor{red}{हेतु\-हेतुमतोर्लिङ्} (पा॰सू॰~३.३.१५६) इत्यनेन वा लिङ्लकारः।\footnote{भक्षँ \textcolor{red}{अदने} (धा॰पा॰~१५५७)~\arrow भक्ष्~\arrow \textcolor{red}{सत्याप\-पाश\-रूप\-वीणा\-तूल\-श्लोक\-सेना\-लोम\-त्वच\-वर्म\-वर्ण\-चूर्ण\-चुरादिभ्यो णिच्} (पा॰सू॰~३.१.२५)~\arrow भक्ष्~णिच्~\arrow भक्ष्~इ~\arrow भक्षि~\arrow \textcolor{red}{सनाद्यन्ता धातवः} (पा॰सू॰~३.१.३२)~\arrow धातु\-सञ्ज्ञा~\arrow \textcolor{red}{शेषात्कर्तरि परस्मैपदम्} (पा॰सू॰~१.३.७८)~\arrow \textcolor{red}{विधि\-निमन्त्रणामन्त्रणाधीष्ट\-सम्प्रश्न\-प्रार्थनेषु लिङ्} (पा॰सू॰~३.३.१६१) \textcolor{red}{हेतु\-हेतुमतोर्लिङ्} (पा॰सू॰~३.३.१५६) वा~\arrow भक्षि~लिङ~\arrow भक्षि~तिप्~\arrow भक्षि~ति~\arrow \textcolor{red}{कर्तरि शप्‌} (पा॰सू॰~३.१.६८)~\arrow भक्षि~शप्~ति~\arrow भक्षि~अ~ति~\arrow \textcolor{red}{सार्वधातुकार्ध\-धातुकयोः} (पा॰सू॰~७.३.८४)~\arrow भक्षे~अ~ति~\arrow \textcolor{red}{यासुट् परस्मैपदेषूदात्तो ङिच्च} (पा॰सू॰~३.४.१०३)~\arrow \textcolor{red}{आद्यन्तौ टकितौ} (पा॰सू॰~१.१.४६)~\arrow भक्षे~अ~यासुँट्~ति~\arrow भक्षे~अ~यास्~ति~\arrow \textcolor{red}{सुट् तिथोः} (पा॰सू॰~३.४.१०७)~\arrow \textcolor{red}{आद्यन्तौ टकितौ} (पा॰सू॰~१.१.४६)~\arrow भक्षे~अ~यास्~सुँट्~ति~\arrow भक्षे~अ~यास्~स्~ति~\arrow \textcolor{red}{अतो येयः} (पा॰सू॰~७.२.८०)~\arrow भक्षे~अ~इय्~ति~\arrow \textcolor{red}{लोपो व्योर्वलि} (पा॰सू॰~६.१.६६)~\arrow भक्षे~अ~इ~ति~\arrow \textcolor{red}{एचोऽयवायावः} (पा॰सू॰~६.१.७८)~\arrow भक्षय्~अ~इ~ति~\arrow \textcolor{red}{आद्गुणः} (पा॰सू॰~६.१.८७)~\arrow भक्षय्~ए~ति~\arrow \textcolor{red}{इतश्च} (पा॰सू॰~३.४.१००)~\arrow भक्षय्~ए~त्~\arrow भक्षयेत्।}\end{sloppypar}
\label{sec:prasarayat}
\section[प्रसारयत्]{प्रसारयत्}
\centering\textcolor{blue}{दृश्यते नैव कोऽप्यत्र विस्मयो मे प्रजायते।\nopagebreak\\
एवं विचिन्त्य हनूमानधो दृष्टिं प्रसारयत्॥}\nopagebreak\\
\raggedleft{–~अ॰रा॰~५.१.३७}\\
\fontsize{14}{21}\selectfont\begin{sloppypar}\hyphenrules{nohyphenation}\justifying\noindent\hspace{10mm} अत्र \textcolor{red}{विनाऽपि प्रत्ययं पूर्वोत्तर\-पद\-लोपो वक्तव्यः} (वा॰~५.३.८३) इत्यनेनाकार\-लोपः। आगम\-कार्यस्यानित्यत्वाद्वा।\footnote{\textcolor{red}{आगम\-शास्त्रमनित्यम्} (प॰शे॰~९३.२)। अडागमे कृते सामान्यतः \textcolor{red}{प्रासारयत्} इति रूपम्। यथा \textcolor{red}{ततो॒ वै स प्र॒जाना॒न्दक्षि॑णं बा॒हुं प्रासा॑रयत्} (कृ॰य॰ तै॰ब्रा~१.६.४.२) इत्यत्र। \textcolor{red}{सृ गतौ} (धा॰पा॰~९३५)~\arrow \textcolor{red}{हेतुमति च}~\arrow सृ~णिच्~\arrow सृ~इ~\arrow \textcolor{red}{अचो ञ्णिति}~\arrow \textcolor{red}{उरण् रपरः}~\arrow सार्~इ~\arrow सारि~\arrow \textcolor{red}{सनाद्यन्ता धातवः}~\arrow धातुसञ्ज्ञा। प्र~सारि~\arrow \textcolor{red}{शेषात्कर्तरि परस्मैपदम्} (पा॰सू॰~१.३.७८)~\arrow \textcolor{red}{अनद्यतने लङ्} (पा॰सू॰~३.२.१११)~\arrow प्र~सारि~तिप्~\arrow प्र~सारि~ति~\arrow \textcolor{red}{लुङ्लङ्लृङ्क्ष्वडुदात्तः} (पा॰सू॰~६.४.७१)~\arrow \textcolor{red}{आद्यन्तौ टकितौ} (पा॰सू॰~१.१.४६)~\arrow प्र~अट्~सारि~ति~\arrow प्र~अ~सारि~ति~\arrow \textcolor{red}{कर्तरि शप्‌} (पा॰सू॰~३.१.६८)~\arrow प्र~अ~सारि~शप्~ति~\arrow प्र~अ~सारि~अ~ति~\arrow \textcolor{red}{सार्वधातुकार्ध\-धातुकयोः} (पा॰सू॰~७.३.८४)~\arrow प्र~अ~सारे~अ~ति~\arrow प्र~अ~सारय्~अ~ति~\arrow \textcolor{red}{इतश्च} (पा॰सू॰~३.४.१००)~\arrow प्र~अ~सारय्~अ~त्~\arrow \textcolor{red}{अकः सवर्णे दीर्घः} (पा॰सू॰~६.१.१०१)~\arrow प्रा~सारय्~अ~त्~\arrow प्रासारयत्।} प्रमाणं चात्र \textcolor{red}{इको यणचि} (पा॰सू॰~६.१.७७) इत्यत्र ङमुडागमाभावः।\footnote{यद्यागमकार्यं नित्यमभविष्यत्तर्हि \textcolor{red}{इको यणचि} (पा॰सू॰~६.१.७७) इति सूत्रे \textcolor{red}{यण् अचि} इति स्थिते \textcolor{red}{ङमो ह्रस्वादचि ङमुण्नित्यम्} (पा॰सू॰~८.३.३२) इत्यनेन ङमुडागमं कृत्वा \textcolor{red}{इको यण्णचि} इत्येवासूत्रयिष्यन् सूत्रकाराः। अत एव वाल्मीकीय\-रामायणे सुन्दर\-काण्डे नव\-व्याकरणार्थ\-वेत्ता हनुमान् लङि \textcolor{red}{प्रविशम्} इति प्रयुङ्क्ते~– \textcolor{red}{प्रदोषकाले प्रविशं भीतयाऽहं तयोदितः} (वा॰रा॰~५.५८.५०)। अत्र तिलक\-शिरोमणि\-टीका\-कारौ च~– \textcolor{red}{प्रविशं प्राविशम्} (वा॰रा॰ ति॰टी॰~५.५८.५०, वा॰रा॰ शि॰टी॰~५.५८.५०)।}\end{sloppypar}
\section[प्रसीदताम्]{प्रसीदताम्}
\centering\textcolor{blue}{धन्याहमप्यद्य चिराय राघव स्मृतिर्ममासीद्भवपाशमोचिनी।\nopagebreak\\
तद्भक्तसङ्गोऽप्यतिदुर्लभो मम प्रसीदतां दाशरथिः सदा हृदि॥}\nopagebreak\\
\raggedleft{–~अ॰रा॰~५.१.५७}\\
\fontsize{14}{21}\selectfont\begin{sloppypar}\hyphenrules{nohyphenation}\justifying\noindent\hspace{10mm} \textcolor{red}{प्रसीदतु}\footnote{प्र~\textcolor{red}{षद्ऌँ विशरण\-गत्यवसादनेषु} (धा॰पा॰~८५४, १४२७)~\arrow प्र~षद्~\arrow \textcolor{red}{धात्वादेः षः सः} (पा॰सू॰~६.१.६४)~\arrow प्र~सद्~\arrow \textcolor{red}{शेषात्कर्तरि परस्मैपदम्} (पा॰सू॰~१.३.७८)~\arrow \textcolor{red}{लोट् च} (पा॰सू॰~३.३.१६२)~\arrow प्र~सद्~लोट्~\arrow प्र~सद्~तिप्~\arrow प्र~सद्~ति~\arrow \textcolor{red}{कर्तरि शप्‌} (पा॰सू॰~३.१.६८)~\arrow प्र~सद्~शप्~ति~\arrow प्र~सद्~अ~ति~\arrow \textcolor{red}{पाघ्रा\-ध्मास्थाम्ना\-दाण्दृश्यर्त्ति\-सर्त्तिशदसदां पिब\-जिघ्र\-धम\-तिष्ठ\-मन\-यच्छ\-पश्यर्च्छ\-धौ\-शीय\-सीदाः} (पा॰सू॰~७.३.७८)~\arrow प्र~सीद्~अ~ति~\arrow \textcolor{red}{एरुः} (पा॰सू॰~३.४.८६)~\arrow प्र~सीद्~अ~तु~\arrow प्रसीदतु।} इति प्रयोक्तव्ये \textcolor{red}{प्रसीदताम्}\footnote{प्र~\textcolor{red}{षद्ऌँ विशरण\-गत्यवसादनेषु} (धा॰पा॰~८५४, १४२७)~\arrow प्र~षद्~\arrow \textcolor{red}{धात्वादेः षः सः} (पा॰सू॰~६.१.६४)~\arrow प्र~सद्~\arrow \textcolor{red}{कर्तरि कर्म\-व्यतिहारे} (पा॰सू॰~१.३.१४)~\arrow \textcolor{red}{लोट् च} (पा॰सू॰~३.३.१६२)~\arrow प्र~सद्~लोट्~\arrow प्र~सद्~त~\arrow \textcolor{red}{कर्तरि शप्‌} (पा॰सू॰~३.१.६८)~\arrow प्र~सद्~शप्~त~\arrow प्र~सद्~अ~त~\arrow \textcolor{red}{पाघ्रा\-ध्मास्थाम्ना\-दाण्दृश्यर्त्ति\-सर्त्तिशदसदां पिब\-जिघ्र\-धम\-तिष्ठ\-मन\-यच्छ\-पश्यर्च्छ\-धौ\-शीय\-सीदाः} (पा॰सू॰~७.३.७८)~\arrow प्र~सीद्~अ~त~\arrow \textcolor{red}{टित आत्मनेपदानां टेरे} (पा॰सू॰~३.४.७९)~\arrow प्र~सीद्~अ~ते~\arrow \textcolor{red}{आमेतः} (पा॰सू॰~३.४.९०)~\arrow प्र~सीद्~अ~ताम्~\arrow प्रसीदताम्।} इति प्रयोगः।
कर्म\-व्यतिहार आत्मनेपदम्।\footnote{\textcolor{red}{कर्तरि कर्म\-व्यतिहारे} (पा॰सू॰~१.३.१४) इत्यनेन। प्रसन्नताभावस्तु जीवस्यैव धर्मो न ब्रह्मण इति ध्वनयितुं कर्मव्यतिहार आत्मनेपदप्रयोग इति भावः। मानसे गोस्वामि\-पादाश्च~– \textcolor{red}{प्रसन्नतां या न गताऽभिषेकतस्तथा न मम्लौ वनवास\-दुःखतः। मुखाम्बुजश्री रघुनन्दनस्य मे सदाऽस्तु सा मञ्जुलमङ्गलप्रदा॥} (रा॰च॰मा॰~२.म॰२)। एतेन श्रीमद्भागवतेऽष्टम\-स्कन्धे पञ्चमाध्याये ब्रह्मस्तुतौ द्वादशवारं प्रयुक्तं \textcolor{red}{प्रसीदताम्} इत्यपि व्याख्यातम्। यथा~– \textcolor{red}{प्रसीदतां ब्रह्म महाविभूतिः} (भा॰पु॰~८.५.३२) \textcolor{red}{प्रसीदतां नः स महाविभूतिः} (भा॰पु॰~८.५.३३.४३)।}\end{sloppypar}
\section[भोक्ष्यति]{भोक्ष्यति}
\centering\textcolor{blue}{द्विमासाभ्यन्तरे सीता यदि मे वशगा भवेत्।\nopagebreak\\
तदा सर्वसुखोपेता राज्यं भोक्ष्यति सा मया॥}\nopagebreak\\
\raggedleft{–~अ॰रा॰~५.२.४१}\\
\fontsize{14}{21}\selectfont\begin{sloppypar}\hyphenrules{nohyphenation}\justifying\noindent\hspace{10mm} अत्र \textcolor{red}{भुजोऽनवने} (पा॰सू॰~१.३.६६) इत्यनेनाऽत्मनेपदे सति \textcolor{red}{भोक्ष्यते} इति पाणिनीयः।\footnote{भुज्~\arrow \textcolor{red}{भुजोऽनवने} (पा॰सू॰~१.३.७८)~\arrow \textcolor{red}{लृट् शेषे च} (पा॰सू॰~३.३.१३)~\arrow भुज्~लृँट्~\arrow भुज्~त~\arrow \textcolor{red}{स्यतासी लृलुटोः} (पा॰सू॰~३.१.३३)~\arrow भुज्~स्य~त~\arrow \textcolor{red}{एकाच उपदेशेऽनुदात्तात्‌} (पा॰सू॰~७.२.१०)~\arrow इडागम\-निषेधः~\arrow \textcolor{red}{पुगन्त\-लघूपधस्य च} (पा॰सू॰~७.३.८६)~\arrow भोज्~स्य~त~\arrow \textcolor{red}{चोः कुः} (पा॰सू॰~८.२.३०)~\arrow भोग्~स्य~त~\arrow \textcolor{red}{आदेश\-प्रत्यययोः} (पा॰सू॰~८.३.५९)~\arrow भोग्~ष्य~त~\arrow \textcolor{red}{खरि च} (पा॰सू॰~८.४.५५)~\arrow भोक्~ष्य~त~\arrow \textcolor{red}{टित आत्मनेपदानां टेरे} (पा॰सू॰~३.४.७९)~\arrow भोक्~ष्य~ते~\arrow भोक्ष्यते।} किन्तु भोजनं \textcolor{red}{भोजः}।\footnote{\textcolor{red}{भुज्‌}\-धातोः \textcolor{red}{नन्दि\-ग्रहि\-पचादिभ्यो ल्युणिन्यचः} (पा॰सू॰~३.१.१३४) इत्यनेन \textcolor{red}{अच्‌}\-प्रत्यये विभक्ति\-कार्ये। बाहुलकाद्भावेऽच्। तेन \textcolor{red}{भोजः} \textcolor{red}{भोगः} इति द्वावपि समानार्थकौ। यथा भागवते \textcolor{red}{याचिष्णवे भूर्यपि भूरिभोजः} (भा॰पु॰~१०.८१.३४) इत्यत्र। अत्र टीकाकाराः~– \textcolor{red}{भूरिभोजो बहुभोगवान्} (भा॰पु॰ गू॰दी॰~१०.८१.३४, भा॰पु॰ अ॰प्र॰~१०.८१.३४) \textcolor{red}{भूरिभोजो बहुभोगः} (भा॰पु॰ नि॰प्र॰~१०.८१.३४)। \textcolor{red}{भोगः} इति तु \textcolor{red}{भुज्‌}\-धातोः \textcolor{red}{भावे} (पा॰सू॰~३.३.१८) इत्यनेन घञि \textcolor{red}{पुगन्त\-लघूपधस्य च} (पा॰सू॰~७.३.८६) इत्यनेन गुणे \textcolor{red}{चजोः कु घिण्ण्यतोः} (पा॰सू॰~७.३.५२) इत्यनेन कुत्वे विभक्तिकार्ये सिद्धम्।} \textcolor{red}{भोजमाचरतीति भोजति}।\footnote{भोज~\arrow \textcolor{red}{सर्वप्रातिपतिकेभ्य आचारे क्विब्वा वक्तव्यः} (वा॰~३.१.११)~\arrow भोज~क्विँप्~\arrow भोज~व्~\arrow \textcolor{red}{वेरपृक्तस्य} (पा॰सू॰~६.१.६७)~\arrow भोज~\arrow \textcolor{red}{सनाद्यन्ता धातवः} (पा॰सू॰~३.१.३२)~\arrow धातु\-सञ्ज्ञा~\arrow \textcolor{red}{शेषात्कर्तरि परस्मैपदम्} (पा॰सू॰~१.३.७८)~\arrow \textcolor{red}{वर्तमाने लट्} (पा॰सू॰~३.२.१२३)~\arrow भोज~लट्~\arrow भोज~तिप्~\arrow भोज~ति~\arrow \textcolor{red}{कर्तरि शप्‌} (पा॰सू॰~३.१.६८)~\arrow भोज~शप्~ति~\arrow भोज~अ~ति~\arrow\textcolor{red}{अतो गुणे} (पा॰सू॰~६.१.९७)~\arrow भोज~ति~\arrow भोजति।} तस्य लृड्लकारे \textcolor{red}{भोक्ष्यति}।\footnote{\textcolor{red}{आगम\-शास्त्रमनित्यम्} (प॰शे॰~९३.२) इति परिभाषयाऽऽगम\-कार्यस्यानित्यत्वाद्बाहुलकादिडभावः। यद्वा भोजनं \textcolor{red}{भोक्}। णिजन्तात् \textcolor{red}{भुज्‌}\-धातोः (भुज्~णिच्~\arrow भोजि) \textcolor{red}{सम्पदादिभ्‍यः क्विप्} (वा॰~३.३.१०८) इत्यनेन भावे क्विपि \textcolor{red}{णेरनिटि} (पा॰सू॰~६.४.५१) इत्यनेन णिलोपे \textcolor{red}{भोज्} इति प्रातिपदिकं निष्पन्नम्। तस्य प्रथमाविभक्तावेकवचने \textcolor{red}{भोक्} \textcolor{red}{भोग्} इति रूपद्वयम्। भोज्~सुँ~\arrow \textcolor{red}{हल्ङ्याब्भ्यो दीर्घात्सुतिस्यपृक्तं हल्} (पा॰सू॰~६.१.६८)~\arrow भोज्~\arrow \textcolor{red}{चोः कुः} (पा॰सू॰~८.२.३०)~\arrow भोग्~\arrow \textcolor{red}{वाऽवसाने} (पा॰सू॰~८.४.५६)~\arrow भोक्, भोग्। यथर्ग्वेद\-संहितायाम् \textcolor{red}{युवा॑नो रु॒द्रा अ॒जरा॑ अभो॒ग्घनो॑} (ऋ॰वे॰सं॰~१.६४.३) इति मन्त्रे पदपाठे \textcolor{red}{अ॒भो॒क्ऽहन॑} सायण\-भाष्ये \textcolor{red}{अभोग्घनः। भोजयन्तीति भोजः। न भोजः अभोजः। तेषां हन्तारः। ‘बहुलं छन्दसि’ (पा॰सू॰~३.२.८८) इति हन्तेः क्विप्। ‘झयो होऽन्यतरस्याम्’ (पा॰सू॰~८.४.६२) इति हकारस्य घत्वम्। ‘इन्हन्पूषार्यम्णां शौ’ (पा॰सू॰~६.४.१२) इति नियमाद्दीर्घाभावः}। तदाचरिष्यति भोक्ष्यति। नामधातोरनु\-दात्तैकाच्त्वात् \textcolor{red}{एकाच उपदेशेऽनुदात्तात्‌} (पा॰सू॰~७.२.१०) इति सूत्रेणेडागम\-निषेधः।} यद्वाऽत्रावनार्थ\-\textcolor{red}{भुज्‌}\-धातुः।\footnote{\textcolor{red}{भुजँ पालनाभ्यवहारयोः} (धा॰पा॰~१४५४)। भुज्~\arrow \textcolor{red}{शेषात्कर्तरि परस्मैपदम्} (पा॰सू॰~१.३.७८)~\arrow \textcolor{red}{लृट् शेषे च} (पा॰सू॰~३.३.१३)~\arrow भुज्~लृँट्~\arrow भुज्~तिप्~\arrow भुज्~ति~\arrow \textcolor{red}{स्यतासी लृलुटोः} (पा॰सू॰~३.१.३३)~\arrow भुज्~स्य~ति~\arrow \textcolor{red}{एकाच उपदेशेऽनुदात्तात्‌} (पा॰सू॰~७.२.१०)~\arrow इडागम\-निषेधः~\arrow \textcolor{red}{पुगन्त\-लघूपधस्य च} (पा॰सू॰~७.३.८६)~\arrow भोज्~स्य~ति~\arrow \textcolor{red}{चोः कुः} (पा॰सू॰~८.२.३०) भोग्~स्य~ति~\arrow \textcolor{red}{आदेश\-प्रत्यययोः} (पा॰सू॰~८.३.५९)~\arrow भोग्~ष्य~ति~\arrow \textcolor{red}{खरि च} (पा॰सू॰~८.४.५५)~\arrow भोक्~ष्य~ति~\arrow भोक्ष्यति।} रावणस्याभिप्रायोऽयम् \textcolor{red}{मद्वशगा सती सीता राज्यं भोक्ष्यत्यधिष्ठात्री भूत्वा तत्प्रतिपालयिष्यति}।\end{sloppypar}
\section[अगाहत्]{अगाहत्}
\centering\textcolor{blue}{अगाहत्पुत्रपौत्रैश्च कृत्वा वदनमालिकाम्।\nopagebreak\\
विभीषणस्तु रामस्य सन्निधौ हृष्टमानसः॥}\nopagebreak\\
\raggedleft{–~अ॰रा॰~५.२.५२}\\
\fontsize{14}{21}\selectfont\begin{sloppypar}\hyphenrules{nohyphenation}\justifying\noindent\hspace{10mm} आत्मनेपदस्यानित्यत्वात्प्रयोगोऽयं परस्मैपदी।\footnote{\textcolor{red}{अनुदात्तेत्त्व\-लक्षणमात्मने\-पदमनित्यम्} (प॰शे॰~९३.४) इत्यनेन। \textcolor{red}{गाह्‌}\-धातुः (\textcolor{red}{गाहूँ विलोडने} धा॰पा॰~६४९) आत्मने\-पदी। तस्य लङ्लकारे प्रथमपुरुष एकवचने \textcolor{red}{अगाहत} इति रूपम्। \textcolor{red}{गाहूँ विलोडने} (धा॰पा॰~६४९)~\arrow गाह्~\arrow \textcolor{red}{अनुदात्तङित आत्मने\-पदम्} (पा॰सू॰~१.३.१२)~\arrow \textcolor{red}{अनद्यतने लङ्} (पा॰सू॰~३.२.१११)~\arrow गाह्~लङ्~\arrow गाह्~त~\arrow \textcolor{red}{लुङ्लङ्लृङ्क्ष्वडुदात्तः} (पा॰सू॰~६.४.७१)~\arrow अट्~गाह्~त~\arrow अ~गाह्~त~\arrow \textcolor{red}{कर्तरि शप्‌} (पा॰सू॰~३.१.६८)~\arrow अ~गाह्~शप्~त~\arrow अ~गाह्~अ~त~\arrow अगाहत। स्वीकृत आत्मने\-पदस्यानित्यत्वे परस्मैपदे \textcolor{red}{अगाहत्} इति रूपम्। \textcolor{red}{गाहूँ विलोडने} (धा॰पा॰~६४९)~\arrow गाह्~\arrow \textcolor{red}{अनुदात्तेत्त्व\-लक्षणमात्मने\-पदमनित्यम्} (प॰शे॰~९३.४)~\arrow \textcolor{red}{शेषात्कर्तरि परस्मैपदम्} (पा॰सू॰~१.३.७८)~\arrow \textcolor{red}{अनद्यतने लङ्} (पा॰सू॰~३.२.१११)~\arrow गाह्~लङ्~\arrow गाह्~तिप्~\arrow गाह्~ति~\arrow \textcolor{red}{लुङ्लङ्लृङ्क्ष्वडुदात्तः} (पा॰सू॰~६.४.७१)~\arrow अट्~गाह्~ति~\arrow अ~गाह्~ति~\arrow \textcolor{red}{कर्तरि शप्‌} (पा॰सू॰~३.१.६८)~\arrow अ~गाह्~शप्~ति~\arrow अ~गाह्~अ~ति~\arrow \textcolor{red}{इतश्च} (पा॰सू॰~३.४.१००)~\arrow अगाहत्। एतेन \textcolor{red}{यूयं विवस्त्रा यदपो धृतव्रता व्यगाहतैतत्तदु देवहेलनम्} (भा॰पु॰~१०.२२.१८) इति भागवते रासपञ्चाध्याय्यां गोपीवस्त्रापहारे मध्यमपुरुष\-बहुवचन\-विवक्षायां कृतः \textcolor{red}{व्यगाहत} इति परस्मैपदप्रयोगोऽपि व्याख्यातः।}\end{sloppypar}
\section[निर्दहिष्यति]{निर्दहिष्यति}
\centering\textcolor{blue}{निर्दहिष्यति रक्षौघांस्त्वत्कृते नात्र संशयः।\nopagebreak\\
अनुज्ञां देहि मे देवि गच्छामि त्वरयाऽन्वितः॥}\nopagebreak\\
\raggedleft{–~अ॰रा॰~५.३.४९}\\
\fontsize{14}{21}\selectfont\begin{sloppypar}\hyphenrules{nohyphenation}\justifying\noindent\hspace{10mm} \textcolor{red}{निर्धक्ष्यति}\footnote{निस्~\textcolor{red}{दहँ भस्मीकरणे} (धा॰पा॰~९९१)~\arrow निस्~दह्~\arrow \textcolor{red}{शेषात्कर्तरि परस्मैपदम्} (पा॰सू॰~१.३.७८)~\arrow \textcolor{red}{लृट् शेषे च} (पा॰सू॰~३.३.१३)~\arrow निस्~दह्~लृट्~\arrow निस्~दह्~तिप्~\arrow निस्~दह्~ति~\arrow \textcolor{red}{स्यतासी लृलुटोः} (पा॰सू॰~३.१.३३)~\arrow निस्~दह्~स्य~ति~\arrow \textcolor{red}{दादेर्धातोर्घः} (पा॰सू॰~८.२.३२)~\arrow निस्~दघ्~स्य~ति~\arrow \textcolor{red}{एकाचो बशो भष् झषन्तस्य स्ध्वोः} (पा॰सू॰~८.२.३७)~\arrow निस्~धघ्~स्य~ति~\arrow \textcolor{red}{खरि च} (पा॰सू॰~८.४.५५)~\arrow निस्~धक्~स्य~ति~\arrow \textcolor{red}{आदेश\-प्रत्यययोः} (पा॰सू॰~८.३.५९)~\arrow निस्~धक्~ष्य~ति~\arrow ससजुषो रुः~\arrow (पा॰सू॰~८.२.६६)~\arrow निरुँ~धक्~ष्य~ति~\arrow निर्~धक्~ष्य~ति~\arrow निर्धक्ष्यति।} इति प्रयोक्तव्ये \textcolor{red}{निर्दहिष्यति} इति प्रयुक्तम्। \textcolor{red}{निर्दहतीति निर्दहः}।\footnote{\textcolor{red}{नन्दि\-ग्रहि\-पचादिभ्यो ल्युणिन्यचः} (पा॰सू॰~३.१.१३४) इत्यनेन।} \textcolor{red}{निर्दह इवाऽचरिष्यतीति निर्दहिष्यति}\footnote{प्रयोगस्यास्य सिद्धिः \textcolor{red}{कृष्णिष्यति} (बा॰म॰~२६६५) इतिवत्। निर्दह~\arrow \textcolor{red}{सर्वप्राति\-पदिकेभ्य आचारे क्विब्वा वक्तव्यः} (वा॰~३.१.११)~\arrow निर्दह~क्विँप्~\arrow निर्दह~व्~\arrow \textcolor{red}{वेरपृक्तस्य} (पा॰सू॰~६.१.६७)~\arrow निर्दह~\arrow \textcolor{red}{सनाद्यन्ता धातवः} (पा॰सू॰~३.१.३२)~\arrow \textcolor{red}{शेषात्कर्तरि परस्मैपदम्} (पा॰सू॰~१.३.७८)~\arrow \textcolor{red}{लृट् शेषे च} (पा॰सू॰~३.३.१३)~\arrow निर्दह~लृट्~\arrow निर्दह~तिप्~\arrow निर्दह~ति~\arrow \textcolor{red}{स्यतासी लृलुटोः} (पा॰सू॰~३.१.३३)~\arrow निर्दह~स्य~ति~\arrow \textcolor{red}{आर्धधातुकस्येड्वलादेः} (पा॰सू॰~७.२.३५)~\arrow निर्दह~इट्~स्य~ति~\arrow निर्दह~इ~स्य~ति~\arrow \textcolor{red}{अतो लोपः} (पा॰सू॰~६.४.४८)~\arrow निर्दह्~इ~स्य~ति~\arrow \textcolor{red}{आदेश\-प्रत्ययोः} (पा॰सू॰~८.३.५९)~\arrow निर्दह्~इ~ष्य~ति~\arrow निर्दहिष्यति।} इति प्रयोगे परिहारः।\end{sloppypar}
\section[निद्राति]{निद्राति}
\centering\textcolor{blue}{अभिज्ञानार्थमन्यच्च वदामि तव सुव्रत।\nopagebreak\\
चित्रकूटगिरौ पूर्वमेकदा रहसि स्थितः।\nopagebreak\\
मदङ्के शिर आधाय निद्राति रघुनन्दनः॥}\nopagebreak\\
\raggedleft{–~अ॰रा॰~५.३.५३}\\
\fontsize{14}{21}\selectfont\begin{sloppypar}\hyphenrules{nohyphenation}\justifying\noindent\hspace{10mm} इह \textcolor{red}{स्म} इति योजनीयम्। एवं \textcolor{red}{लट् स्मे} (पा॰सू॰~३.२.११८) इत्यनेन भूत\-काले लड्लकारः।\footnote{निपूर्वकात् \textcolor{red}{द्रा}\-धातोः (\textcolor{red}{द्रा कुत्सायां गतौ} धा॰पा॰~१०५४) लड्लकारे प्रथमपुरुष एकवचने \textcolor{red}{निद्राति}। नि~द्रा~\arrow \textcolor{red}{शेषात्कर्तरि परस्मैपदम्} (पा॰सू॰~१.३.७८)~\arrow \textcolor{red}{वर्तमाने लट्} (पा॰सू॰~३.२.१२३)~\arrow नि~द्रा~लट्~\arrow नि~द्रा~तिप्~\arrow नि~द्रा~ति~\arrow \textcolor{red}{कर्तरि शप्} (पा॰सू॰~३.१.६८)~\arrow नि~द्रा~शप्~ति~\arrow \textcolor{red}{अदिप्रभृतिभ्यः शपः} (पा॰सू॰~२.४.७२)~\arrow नि~द्रा~ति~\arrow निद्राति।}\end{sloppypar}
\section[वदस्व]{वदस्व}
\centering\textcolor{blue}{ततः प्रहस्तो हनुमन्तमादरात्पप्रच्छ केन प्रहितोऽसि वानर।\nopagebreak\\
भयं च ते माऽस्तु विमोक्ष्यसे मया सत्यं वदस्वाखिलराजसन्निधौ॥}\nopagebreak\\
\raggedleft{–~अ॰रा॰~५.४.६}\\
\fontsize{14}{21}\selectfont\begin{sloppypar}\hyphenrules{nohyphenation}\justifying\noindent\hspace{10mm} अत्र \textcolor{red}{भासनोपसम्भाषा\-ज्ञान\-यत्न\-विमत्युपमन्त्रणेषु वदः} (पा॰सू॰~१.३.४७) इत्यनेनोपसम्भाषायां ज्ञाने वाऽऽत्मनेपदम्।\footnote{\textcolor{red}{वदँ व्यक्तायां वाचि} (धा॰पा॰~१००९)~\arrow वद्~\arrow \textcolor{red}{भासनोपसम्भाषा\-ज्ञान\-यत्न\-विमत्युपमन्त्रणेषु वदः} (पा॰सू॰~१.३.४७)~\arrow \textcolor{red}{लोट् च} (पा॰सू॰~३.३.१६२)~\arrow वद्~लोट्~\arrow वद्~थास्~\arrow \textcolor{red}{कर्तरि शप्} (पा॰सू॰~३.१.६८)~\arrow वद्~शप्~थास्~\arrow वद्~अ~थास्~\arrow \textcolor{red}{थासस्से} (पा॰सू॰~३.४.८०)~\arrow वद्~अ~से~\arrow \textcolor{red}{सवाभ्यां वामौ} (पा॰सू॰~३.४.९१)~\arrow वद्~अ~स्~व~\arrow वदस्व।}\end{sloppypar}
\section[पश्यध्वम्]{पश्यध्वम्}
\centering\textcolor{blue}{शब्देनैव विजानीमः कृतकार्यः समागतः।\nopagebreak\\
हनूमानेव पश्यध्वं वानरा वानरर्षभम्॥}\nopagebreak\\
\raggedleft{–~अ॰रा॰~५.५.१३}\\
\fontsize{14}{21}\selectfont\begin{sloppypar}\hyphenrules{nohyphenation}\justifying\noindent\hspace{10mm} अत्र \textcolor{red}{कर्तरि कर्म\-व्यतिहारे} (पा॰सू॰~१.३.१४) इत्यनेन क्रिया\-विनिमय आत्मनेपदम्।\footnote{परस्मै\-पदिनः \textcolor{red}{दृश्‌}\-धातोः (\textcolor{red}{दृशिँर् प्रेक्षणे} धा॰पा॰~९८८) लोड्लकारे मध्यम\-पुरुषे बहुवचने \textcolor{red}{पश्यत} इति रूपम्। दृश्~\arrow \textcolor{red}{शेषात्कर्तरि परस्मैपदम्} (पा॰सू॰~१.३.७८)~\arrow \textcolor{red}{लोट् च} (पा॰सू॰~३.३.१६२)~\arrow दृश्~लोट्~\arrow दृश्~थ~\arrow \textcolor{red}{कर्तरि शप्} (पा॰सू॰~३.१.६८)~\arrow दृश्~शप्~थ~\arrow दृश्~अ~थ~\arrow \textcolor{red}{पाघ्रा\-ध्मास्थाम्ना\-दाण्दृश्यर्त्ति\-सर्त्तिशदसदां पिब\-जिघ्र\-धम\-तिष्ठ\-मन\-यच्छ\-पश्यर्च्छ\-धौ\-शीय\-सीदाः} (पा॰सू॰~७.३.७८)~\arrow पश्य्~अ~थ~\arrow \textcolor{red}{लोटो लङ्वत्‌} (पा॰सू॰~३.४.८५)~\arrow ङिद्वत्त्वम्~\arrow \textcolor{red}{तस्थस्थमिपां तान्तन्तामः} (पा॰सू॰~३.४.१०१)~\arrow पश्य्~अ~त~\arrow पश्यत। आत्मनेपदे च \textcolor{red}{पश्यध्वम्} इति। दृश्~\arrow \textcolor{red}{कर्तरि कर्मव्यतिहारे} (पा॰सू॰~१.३.१४)~\arrow \textcolor{red}{लोट् च} (पा॰सू॰~३.३.१६२)~\arrow दृश्~लोट्~\arrow दृश्~ध्वम्~\arrow \textcolor{red}{कर्तरि शप्} (पा॰सू॰~३.१.६८)~\arrow दृश्~शप्~ध्वम्~\arrow दृश्~अ~ध्वम्~\arrow \textcolor{red}{पाघ्रा\-ध्मास्थाम्ना\-दाण्दृश्यर्त्ति\-सर्त्तिशदसदां पिब\-जिघ्र\-धम\-तिष्ठ\-मन\-यच्छ\-पश्यर्च्छ\-धौ\-शीय\-सीदाः} (पा॰सू॰~७.३.७८)~\arrow पश्य्~अ~ध्वम्~\arrow पश्यध्वम्।}\end{sloppypar}
\vspace{2mm}
\centering ॥ इति सुन्दरकाण्डीयप्रयोगाणां विमर्शः ॥\nopagebreak\\
\vspace{4mm}
\pdfbookmark[2]{युद्धकाण्डम्}{Chap3Part2Kanda6}
\phantomsection
\addtocontents{toc}{\protect\setcounter{tocdepth}{2}}
\addcontentsline{toc}{subsection}{युद्धकाण्डीयप्रयोगाणां विमर्शः}
\addtocontents{toc}{\protect\setcounter{tocdepth}{0}}
\centering ॥ अथ युद्धकाण्डीयप्रयोगाणां विमर्शः ॥\nopagebreak\\
\section[निवसस्व]{निवसस्व}
\centering\textcolor{blue}{भुङ्क्ष्व चेमानि पक्वानि फलानि तदनन्तरम्।\nopagebreak\\
निवसस्व सुखेनात्र निद्रामेहि त्वराऽस्तु मा॥}\nopagebreak\\
\raggedleft{–~अ॰रा॰~६.७.१६}\\
\fontsize{14}{21}\selectfont\begin{sloppypar}\hyphenrules{nohyphenation}\justifying\noindent\hspace{10mm} अत्र \textcolor{red}{कर्तरि कर्म\-व्यतिहारे} (पा॰सू॰~१.३.१४) इत्यनेनाऽत्मनेपदम्।\footnote{\textcolor{red}{नि}पूर्वकात्परस्मै\-पदिनो \textcolor{red}{वस्‌}\-धातोः (\textcolor{red}{वसँ निवासे} धा॰पा॰~१००५) लोड्लकारे मध्यम\-पुरुष एकवचने \textcolor{red}{निवस} इति रूपम्। नि~वस्~\arrow \textcolor{red}{शेषात्कर्तरि परस्मैपदम्} (पा॰सू॰~१.३.७८)~\arrow \textcolor{red}{लोट् च} (पा॰सू॰~३.३.१६२)~\arrow नि~वस्~लोट्~\arrow नि~वस्~सिप्~\arrow नि~वस्~सि~\arrow \textcolor{red}{कर्तरि शप्} (पा॰सू॰~३.१.६८)~\arrow नि~वस्~शप्~सि~\arrow नि~वस्~अ~सि~\arrow \textcolor{red}{सेर्ह्यपिच्च} (पा॰सू॰~३.४.८७)~\arrow नि~वस्~अ~हि~\arrow \textcolor{red}{अतो हेः} (पा॰सू॰~६.४.१०५)~\arrow नि~वस्~अ~\arrow निवस। आत्मनेपदे च \textcolor{red}{निवसस्व} इति। नि~वस्~\arrow \textcolor{red}{कर्तरि कर्मव्यतिहारे} (पा॰सू॰~१.३.१४)~\arrow \textcolor{red}{लोट् च} (पा॰सू॰~३.३.१६२)~\arrow नि~वस्~लोट्~\arrow नि~वस्~थास्~\arrow \textcolor{red}{कर्तरि शप्} (पा॰सू॰~३.१.६८)~\arrow नि~वस्~शप्~थास्~\arrow नि~वस्~अ~थास्~\arrow \textcolor{red}{थासस्से} (पा॰सू॰~३.४.८०)~\arrow नि~वस्~अ~से~\arrow \textcolor{red}{सवाभ्यां वामौ} (पा॰सू॰~३.४.९१)~\arrow नि~वस्~अ~स्~व~\arrow निवसस्व। अपि च~– \textcolor{red}{त्वराऽस्तु मा} इति कथम् \textcolor{red}{माङि लुङ्} (पा॰सू॰~३.३.१७५) इत्यनेन सर्व\-लकारापवादत्वेन लुङो विधानात्। अस्य समाधानं दीक्षितैर्बाल\-मनोरमायामुक्तम्~– \textcolor{red}{‘माऽस्तु’ इत्यादौ तु ‘मा’ इत्यव्ययान्तरं प्रतिषेधार्थकमित्याहुः। ‘आङ्माङोश्च’ (पा॰सू॰~६.१.७४) इति सूत्रभाष्ये तु ङितो माशब्दस्य निर्देशात्प्रमाच्छन्द इत्यत्र तु न भवतीत्युक्तम्। ‘मा’\-शब्दस्याव्ययान्तरस्य सत्त्वे तु तदेवोदाह्रियेत। ‘माऽस्तु’ इत्यत्र तु ‘अस्तु’ इति विभक्ति\-प्रतिरूपकमव्ययमित्यन्ये} (बा॰म॰~२२१९)।}\end{sloppypar}
\section[निबोध]{निबोध}
\centering\textcolor{blue}{तमाह रावणो राजा भ्रातरं दीनया गिरा।\nopagebreak\\
कुम्भकर्ण निबोध त्वं महत्कष्टमुपस्थितम्॥}\nopagebreak\\
\raggedleft{–~अ॰रा॰~६.७.५१}\\
\fontsize{14}{21}\selectfont\begin{sloppypar}\hyphenrules{nohyphenation}\justifying\noindent\hspace{10mm} अत्र \textcolor{red}{नि}\-पूर्वकोऽवगमार्थको \textcolor{red}{बुध्‌}\-धातुः (\textcolor{red}{बुधँ अवगमने} धा॰पा॰~११७२)। तत्राऽत्मनेपदत्वाल्लोड्\-लकारे मध्यम\-पुरुषैक\-वचने \textcolor{red}{निबुध्यस्व} इति पाणिनीयम्।\footnote{नि~बुध्~\arrow \textcolor{red}{अनुदात्तङित आत्मनेपदम्} (पा॰सू॰~१.३.१२)~\arrow \textcolor{red}{लोट् च} (पा॰सू॰~३.३.१६२)~\arrow नि~बुध्~लोट्~\arrow नि~बुध्~थास्~\arrow \textcolor{red}{दिवादिभ्यः श्यन्} (पा॰सू॰~३.१.६९)~\arrow नि~बुध्~श्यन्~थास्~\arrow नि~बुध्~य~थास्~\arrow \textcolor{red}{थासस्से} (पा॰सू॰~३.४.८०)~\arrow नि~बुध्~य~से~\arrow \textcolor{red}{सवाभ्यां वामौ} (पा॰सू॰~३.४.९१)~\arrow नि~बुध्~य~स्~व~\arrow निबुध्यस्व।} किन्तु \textcolor{red}{बोधनं बोधः} भावे घञि\footnote{\textcolor{red}{भावे} (पा॰सू॰~३.३.१८) इत्यनेन।} ततश्च गुणे।\footnote{\textcolor{red}{पुगन्त\-लघूपधस्य च} (पा॰सू॰~७.३.८६) इत्यनेन।} \textcolor{red}{नितरां बोधो निबोधः}। \textcolor{red}{निबोधमाचर} इति \textcolor{red}{निबोध} इत्थमाचार\-क्विबन्ताद्धातोर्लोट्। मध्यम\-पुरुष एकवचने \textcolor{red}{सेर्ह्यपिच्च} (पा॰सू॰~३.४.८७) इत्थं \textcolor{red}{हि} आदेशे \textcolor{red}{अतो हेः} (पा॰सू॰~६.४.१०५) इत्यनेन हेर्लुकि \textcolor{red}{निबोध} इत्यपि पाणिनीयम्।\footnote{निबोध~\arrow \textcolor{red}{सर्वप्राति\-पदिकेभ्य आचारे क्विब्वा वक्तव्यः} (वा॰~३.१.११)~\arrow निबोध~क्विँप्~\arrow निबोध~व्~\arrow \textcolor{red}{वेरपृक्तस्य} (पा॰सू॰~६.१.६७)~\arrow निबोध~\arrow \textcolor{red}{सनाद्यन्ता धातवः} (पा॰सू॰~३.१.३२)~\arrow धातुसञ्ज्ञा~\arrow \textcolor{red}{लोट् च} (पा॰सू॰~३.३.१६२)~\arrow निबोध~लोट्~\arrow निबोध~सिप्~\arrow निबोध~सि~\arrow \textcolor{red}{कर्तरि शप्} (पा॰सू॰~३.१.६८)~\arrow निबोध~शप्~सि~\arrow निबोध~अ~सि~\arrow \textcolor{red}{अतो गुणे} (पा॰सू॰~६.१.९७)~\arrow निबोध~सि~\arrow \textcolor{red}{सेर्ह्यपिच्च} (पा॰सू॰~३.४.८७)~\arrow निबोध~हि~\arrow \textcolor{red}{अतो हेः}~\arrow निबोध।} यद्वाऽत्र \textcolor{red}{बुध्‌}\-धातुर्भ्वादि\-परस्मैपदी (\textcolor{red}{बुधँ अवगमने} धा॰पा॰~८५८)। तस्य लोड्लकारे मध्यमपुरुष एकवचन\-रूपमिदम् \textcolor{red}{निबोध}।\footnote{नि~बुध्~\arrow \textcolor{red}{शेषात्कर्तरि परस्मैपदम्} (पा॰सू॰~१.३.७८)~\arrow \textcolor{red}{लोट् च} (पा॰सू॰~३.३.१६२)~\arrow नि~बुध्~लोट्~\arrow नि~बुध्~सिप्~\arrow नि~बुध्~सि~\arrow \textcolor{red}{कर्तरि शप्} (पा॰सू॰~३.१.६८)~\arrow नि~बुध्~शप्~सि~\arrow नि~बुध्~अ~सि~\arrow \textcolor{red}{पुगन्त\-लघूपधस्य च} (पा॰सू॰~७.३.८६)~\arrow नि~बोध्~अ~सि~\arrow \textcolor{red}{सेर्ह्यपिच्च} (पा॰सू॰~३.४.८७)~\arrow नि~बोध्~अ~हि~\arrow \textcolor{red}{अतो हेः} (पा॰सू॰~६.४.१०५)~\arrow नि~बोध्~अ~\arrow निबोध।} एतेन \textcolor{red}{तान्निबोध द्विजोत्तम} (भ॰गी॰~१.७) इति गीता\-वचनमपि व्याख्यातम्।\end{sloppypar}
\section[न्यहनन्]{न्यहनन्}
\centering\textcolor{blue}{दशकोट्यः प्लवङ्गानां गत्वा मन्दिररक्षकान्।\nopagebreak\\
चूर्णयामासुरश्वांश्च गजांश्च न्यहनन् क्षणात्॥}\nopagebreak\\
\raggedleft{–~अ॰रा॰~६.१०.१७}\\
\fontsize{14}{21}\selectfont\begin{sloppypar}\hyphenrules{nohyphenation}\justifying\noindent\hspace{10mm} \textcolor{red}{न्यघ्नन्}\footnote{नि~\textcolor{red}{हनँ हिंसागत्योः} (धा॰पा॰~१०१२)~\arrow नि~हन्~\arrow \textcolor{red}{शेषात्कर्तरि परस्मैपदम्} (पा॰सू॰~१.३.७८)~\arrow \textcolor{red}{अनद्यतने लङ्} (पा॰सू॰~३.२.१११)~\arrow नि~हन्~लङ्~\arrow \textcolor{red}{लुङ्लङ्लृङ्क्ष्वडुदात्तः} (पा॰सू॰~६.४.७१)~\arrow \textcolor{red}{आद्यन्तौ टकितौ} (पा॰सू॰~१.१.४६)~\arrow नि~अट्~हन्~लङ्~\arrow नि~अ~हन्~लङ्~\arrow नि~अ~हन्~झि~\arrow \textcolor{red}{तिङ्शित्सार्व\-धातुकम्} (पा॰सू॰~३.४.११३)~\arrow \textcolor{red}{कर्तरि शप्} (पा॰सू॰~३.१.६८)~\arrow \textcolor{red}{अदि\-प्रभृतिभ्यः शपः} (पा॰सू॰~२.४.७२)~\arrow नि~अ~हन्~झि~\arrow \textcolor{red}{सार्वधातुकमपित्} (पा॰सू॰~१.२.४)~\arrow ङित्त्वम्~\arrow \textcolor{red}{गम\-हन\-जन\-खन\-घसां लोपः क्ङित्यनङि} (पा॰सू॰~६.४.९८)~\arrow नि~अ~ह्~न्~झि~\arrow \textcolor{red}{हो हन्तेर्ञ्णिन्नेषु} (पा॰सू॰~७.३.५४)~\arrow नि~अ~घ्~न्~झि~\arrow \textcolor{red}{झोऽन्तः} (पा॰सू॰~७.१.३)~\arrow नि~अ~घ्~न्~अन्ति~\arrow \textcolor{red}{इतश्च} (पा॰सू॰~३.४.१००)~\arrow नि~अ~घ्~न्~अन्त्~\arrow \textcolor{red}{संयोगान्तस्य लोपः} (पा॰सू॰~८.२.२३)~\arrow नि~अ~घ्~न्~अन्~\arrow नि~अघ्नन्~\arrow \textcolor{red}{इको यणचि} (पा॰सू॰~६.१.७७)~\arrow न्यघ्नन्।} इति प्रयोक्तव्ये \textcolor{red}{न्यहनन्} इति प्रयोगस्तु \textcolor{red}{गण\-कार्यमनित्यम्} (प॰शे॰~९३.३) इति नियमात् \textcolor{red}{नि}\-पूर्वकात् \textcolor{red}{हन्‌}\-धातोः (\textcolor{red}{हनँ हिंसागत्योः} धा॰पा॰~१०१२) लङि \textcolor{red}{झि}\-प्रत्ययेऽडागमे शपि \textcolor{red}{झोऽन्तः} (पा॰सू॰~७.१.३) इत्यनेनान्तादेशे \textcolor{red}{अतो गुणे} (पा॰सू॰~६.१.९७) इत्यनेन पर\-रूपे \textcolor{red}{इतश्च} (पा॰सू॰~३.४.१००) इत्यनेनेकार\-लोपे \textcolor{red}{संयोगान्तस्य लोपः} (पा॰सू॰~८.२.२३) इत्यनेन तकार\-लोपे \textcolor{red}{न्यहनन्}।\footnote{\textcolor{red}{गण\-कार्यमनित्यम्} (प॰शे॰~९३.३) इति नियमादत्र \textcolor{red}{अदि\-प्रभृतिभ्यः शपः} (पा॰सू॰~२.४.७२) इति सूत्रं न प्रवर्तते। शपः पित्वात्त् \textcolor{red}{सार्वधातुकमपित्} (पा॰सू॰~१.२.४) इत्यस्याप्रवृत्तौ शपो ङित्त्वं न। अङिति शपि परे \textcolor{red}{गम\-हन\-जन\-खन\-घसां लोपः क्ङित्यनङि} (पा॰सू॰~६.४.९८) इति सूत्रं न प्रवर्तते यतो हनोऽकार\-लोपाभावः। अकार\-लोपाभावे \textcolor{red}{हो हन्तेर्ञ्णिन्नेषु} (पा॰सू॰~७.३.५४) इत्यस्यापि प्रवृत्तिर्न। अतः कुत्वाभावः। नि~\textcolor{red}{हनँ हिंसागत्योः} (धा॰पा॰~१०१२)~\arrow नि~हन्~\arrow \textcolor{red}{शेषात्कर्तरि परस्मैपदम्} (पा॰सू॰~१.३.७८)~\arrow \textcolor{red}{अनद्यतने लङ्} (पा॰सू॰~३.२.१११)~\arrow नि~हन्~लङ्~\arrow \textcolor{red}{लुङ्लङ्लृङ्क्ष्वडुदात्तः} (पा॰सू॰~६.४.७१)~\arrow \textcolor{red}{आद्यन्तौ टकितौ} (पा॰सू॰~१.१.४६)~\arrow नि~अट्~हन्~लङ्~\arrow नि~अ~हन्~लङ्~\arrow नि~अ~हन्~झि~\arrow \textcolor{red}{तिङ्शित्सार्व\-धातुकम्} (पा॰सू॰~३.४.११३)~\arrow \textcolor{red}{कर्तरि शप्} (पा॰सू॰~३.१.६८)~\arrow \textcolor{red}{गण\-कार्यमनित्यम्} (प॰शे॰~९३.३)~\arrow शब्लुगभावः~\arrow नि~अ~हन्~अ~झि~\arrow \textcolor{red}{झोऽन्तः} (पा॰सू॰~७.१.३)~\arrow नि~अ~हन्~अ~अन्ति~\arrow \textcolor{red}{इतश्च} (पा॰सू॰~३.४.१००)~\arrow नि~अ~हन्~अ~अन्त्~\arrow \textcolor{red}{अतो गुणे} (पा॰सू॰~६.१.९७)~\arrow नि~अ~हन्~अन्त्~\arrow \textcolor{red}{संयोगान्तस्य लोपः} (पा॰सू॰~८.२.२३)~\arrow नि~अ~हन्~अन्~\arrow नि~अहनन्~\arrow \textcolor{red}{इको यणचि} (पा॰सू॰~६.१.७७)~\arrow न्यहनन्।}\end{sloppypar}
\section[योत्स्यामि]{योत्स्यामि}
\centering\textcolor{blue}{घातयित्वा राघवेण जीवामि वनगोचरः।\nopagebreak\\
रामेण सह योत्स्यामि रामबाणैः सुशीघ्रगैः॥}\nopagebreak\\
\raggedleft{–~अ॰रा॰~६.१०.५६}\\
\fontsize{14}{21}\selectfont\begin{sloppypar}\hyphenrules{nohyphenation}\justifying\noindent\hspace{10mm} आत्मने\-पदस्यानित्यत्वात्प्रयोगोऽयम्।\footnote{\textcolor{red}{अनुदात्तेत्त्व\-लक्षणमात्मने\-पदमनित्यम्} (प॰शे॰~९३.४)। आत्मनेपदे तु \textcolor{red}{युधँ सम्प्रहारे} (धा॰पा॰~११७३) इति धातोर्लृट्युत्तम\-पुरुष एकवचने \textcolor{red}{योत्स्ये} इति रूपम्। यथा गीतायाम्~– \textcolor{red}{न योत्स्य इति गोविन्दमुक्त्वा तूष्णीं बभूव ह} (भ॰गी॰~२.९)। युध्~\arrow \textcolor{red}{अनुदात्तङित आत्मनेपदम्} (पा॰सू॰~१.३.१२)~\arrow \textcolor{red}{लृट् शेषे च} (पा॰सू॰~३.३.१३)~\arrow युध्~लृँट्~\arrow युध्~इट्~\arrow युध्~इ~\arrow \textcolor{red}{स्यतासी लृलुटोः} (पा॰सू॰~३.१.३३)~\arrow युध्~स्य~इ~\arrow \textcolor{red}{एकाच उपदेशेऽनुदात्तात्‌} (पा॰सू॰~७.२.१०)~\arrow इडागम\-निषेधः~\arrow \textcolor{red}{पुगन्त\-लघूपधस्य च} (पा॰सू॰~७.३.८६)~\arrow योध्~स्य~इ~\arrow \textcolor{red}{खरि च} (पा॰सू॰~८.४.५५)~\arrow चर्त्वम्~\arrow योत्~स्य~इ~\arrow \textcolor{red}{टित आत्मनेपदानां टेरे} (पा॰सू॰~३.४.७९)~\arrow योत्~स्य~ए~\arrow \textcolor{red}{अतो गुणे} (पा॰सू॰~६.१.९७)~\arrow योत्~स्ये~\arrow योत्स्ये। युध्~\arrow \textcolor{red}{अनुदात्तेत्त्व\-लक्षणमात्मने\-पदमनित्यम्} (प॰शे॰~९३.४)~\arrow \textcolor{red}{शेषात्कर्तरि परस्मैपदम्} (पा॰सू॰~१.३.७८)~\arrow \textcolor{red}{लृट् शेषे च} (पा॰सू॰~३.३.१३)~\arrow युध्~लृँट्~\arrow युध्~मिप्~\arrow युध्~मि~\arrow \textcolor{red}{स्यतासी लृलुटोः} (पा॰सू॰~३.१.३३)~\arrow युध्~स्य~मि~\arrow \textcolor{red}{अतो दीर्घो यञि} (पा॰सू॰~७.३.१०१)~\arrow युध्~स्या~मि~\arrow \textcolor{red}{एकाच उपदेशेऽनुदात्तात्‌} (पा॰सू॰~७.२.१०)~\arrow इडागम\-निषेधः~\arrow \textcolor{red}{पुगन्त\-लघूपधस्य च} (पा॰सू॰~७.३.८६)~\arrow लघूपध\-गुणः~\arrow योध्~स्या~मि~\arrow \textcolor{red}{खरि च} (पा॰सू॰~८.४.५५)~\arrow योत्~स्या~मि~\arrow योत्स्यामि।}\end{sloppypar}
\section[ससृजे]{ससृजे}
\centering\textcolor{blue}{अस्त्रं राक्षसराजस्य जघान परमास्त्रवित्।\nopagebreak\\
ततस्तु ससृजे घोरं राक्षसं चास्त्रमस्त्रवित्।\nopagebreak\\
क्रोधेन महताऽऽविष्टो रामस्योपरि रावणः॥}\nopagebreak\\
\raggedleft{–~अ॰रा॰~६.११.२८}\\
\fontsize{14}{21}\selectfont\begin{sloppypar}\hyphenrules{nohyphenation}\justifying\noindent\hspace{10mm} \textcolor{red}{सृजँ विसर्गे} (धा॰पा॰~१४१४) परस्मैपदी धातुः। ततः \textcolor{red}{ससर्ज} इति रूपम्।\footnote{\textcolor{red}{सृजँ विसर्गे} (धा॰पा॰~१४१४)~\arrow सृज्~\arrow \textcolor{red}{शेषात्कर्तरि परस्मैपदम्} (पा॰सू॰~१.३.७८)~\arrow \textcolor{red}{परोक्षे लिट्} (पा॰सू॰~३.२.११५)~\arrow सृज्~लिँट्~\arrow सृज्~तिप्~\arrow \textcolor{red}{परस्मैपदानां णलतुसुस्थलथुस\-णल्वमाः} (पा॰सू॰~३.४.८२)~\arrow सृज्~णल्~\arrow सृज्~अ~\arrow \textcolor{red}{लिटि धातोरनभ्यासस्य} (पा॰सू॰~६.१.८)~\arrow सृज्~सृज्~अ~\arrow \textcolor{red}{उरत्} (पा॰सू॰~७.४.६६)~\arrow \textcolor{red}{उरण् रपरः} (पा॰सू॰~१.१.५१)~\arrow सर्ज्~सृज्~अ~\arrow \textcolor{red}{हलादिः शेषः} (पा॰सू॰~७.४.६०)~\arrow स~सृज्~अ~\arrow \textcolor{red}{पुगन्त\-लघूपधस्य च} (पा॰सू॰~७.३.८६)~\arrow \textcolor{red}{उरण् रपरः} (पा॰सू॰~१.१.५१)~\arrow स~सर्ज्~अ~\arrow ससर्ज।} अत्र कर्म\-व्यतिहारादात्मनेपदम्।\footnote{\textcolor{red}{कर्तरि कर्मव्यतिहारे} (पा॰सू॰~१.३.१४) इत्यनेन। प्राप्त आत्मने\-पदे लिटि प्रथम\-पुरुष एक\-वचने \textcolor{red}{ससृजे} इति रूपम्। सृज्~\arrow \textcolor{red}{कर्तरि कर्मव्यतिहारे} (पा॰सू॰~१.३.१४)~\arrow \textcolor{red}{परोक्षे लिट्} (पा॰सू॰~३.२.११५)~\arrow सृज्~लिँट्~\arrow सृज्~तिप्~\arrow \textcolor{red}{लिटस्तझयोरेशिरेच्} (पा॰सू॰~३.४.८१)~\arrow सृज्~एश्~\arrow सृज्~ए~\arrow \textcolor{red}{लिटि धातोरनभ्यासस्य} (पा॰सू॰~६.१.८)~\arrow सृज्~सृज्~ए~\arrow \textcolor{red}{हलादिः शेषः} (पा॰सू॰~७.४.६०)~\arrow सृ~सृज्~ए~\arrow \textcolor{red}{उरत्} (पा॰सू॰~७.४.६६)~\arrow \textcolor{red}{उरण् रपरः} (पा॰सू॰~१.१.५१)~\arrow सर्~सृज्~ए~\arrow \textcolor{red}{हलादिः शेषः} (पा॰सू॰~७.४.६०)~\arrow स~सृज्~ए~\arrow \textcolor{red}{सार्वधातुकमपित्} (पा॰सू॰~१.२.४)~\arrow ङित्त्वम्~\arrow \textcolor{red}{ग्क्ङिति च} (पा॰सू॰~१.१.५)~\arrow लघूपध\-गुण\-निषेधः~\arrow स~सृज्~ए~\arrow ससृजे।}\end{sloppypar}
\section[लिप्यसे]{लिप्यसे}
\centering\textcolor{blue}{भूतं भविष्यदभजन्वर्तमानमथाचरन्।\nopagebreak\\
विहरस्व यथान्यायन् भवदोषैर्न लिप्यसे॥}\nopagebreak\\
\raggedleft{–~अ॰रा॰~६.१२.२७}\\
\fontsize{14}{21}\selectfont\begin{sloppypar}\hyphenrules{nohyphenation}\justifying\noindent\hspace{10mm} अत्र \textcolor{red}{लिप्‌}\-धातोः (\textcolor{red}{लिपँ उपदेहे} धा॰पा॰~१४३३) कर्मवाच्ये लड्लकारे मध्यमपुरुष एकवचनरूपम्।\footnote{लिप्~\arrow \textcolor{red}{भावकर्मणोः} (पा॰सू॰~१.३.१३)~\arrow \textcolor{red}{वर्तमान\-सामीप्ये वर्तमानवद्वा} (पा॰सू॰~३.३.१३१)~\arrow \textcolor{red}{वर्तमाने लट्} (पा॰सू॰~३.२.१२३)~\arrow लिप्~लँट्~\arrow लिप्~थास्~\arrow \textcolor{red}{सार्वधातुके यक्} (पा॰सू॰~३.१.६७)~\arrow लिप्~यक्~थास्~\arrow लिप्~य~थास्~\arrow \textcolor{red}{थासस्से} (पा॰सू॰~३.४.८०)~\arrow लिप्~य~से~\arrow लिप्यसे।} वर्तमानसामीप्याद्भविष्यति वर्तमानता।\footnote{\textcolor{red}{वर्तमान\-सामीप्ये वर्तमानवद्वा} (पा॰सू॰~३.३.१३१) इत्यनेन।}\end{sloppypar}
\section[गमिष्यामहे]{गमिष्यामहे}
\centering\textcolor{blue}{अलङ्कृत्य सह भ्राता श्वो गमिष्यामहे वयम्।\nopagebreak\\
विभीषणवचः श्रुत्वा प्रत्युवाच रघूत्तमः॥}\nopagebreak\\
\raggedleft{–~अ॰रा॰~६.१३.४२}\\
\fontsize{14}{21}\selectfont\begin{sloppypar}\hyphenrules{nohyphenation}\justifying\noindent\hspace{10mm} \textcolor{red}{गमेरिट् परस्मैपदेषु} (पा॰सू॰~७.२.५८) इत्यनेनेड्विधानात् \textcolor{red}{गम्‌}\-धातोश्च (\textcolor{red}{गमॢँ गतौ} धा॰पा॰~९८२) परस्मैपदत्वात् \textcolor{red}{गमिष्यामः}।\footnote{गम्~\arrow \textcolor{red}{शेषात्कर्तरि परस्मैपदम्} (पा॰सू॰~१.३.७८)~\arrow \textcolor{red}{लृट् शेषे च} (पा॰सू॰~३.३.१३)~\arrow गम्~लृँट्~\arrow गम्~मस्~\arrow \textcolor{red}{स्यतासी लृलुटोः} (पा॰सू॰~३.१.३३)~\arrow गम्~स्य~मस्~\arrow \textcolor{red}{गमेरिट् परस्मैपदेषु} (पा॰सू॰~७.२.५८)~\arrow गम्~इट्~स्य~मस्~\arrow गम्~इ~स्य~मस्~\arrow \textcolor{red}{अतो दीर्घो यञि} (पा॰सू॰~७.३.१०१)~\arrow गम्~इ~स्या~मस्~\arrow \textcolor{red}{आदेश\-प्रत्यययोः} (पा॰सू॰~८.३.५९)~\arrow गम्~इ~ष्या~मस्~\arrow \textcolor{red}{ससजुषो रुः} (पा॰सू॰~८.२.६६)~\arrow गम्~इ~ष्या~मरुँ~\arrow \textcolor{red}{खरवसानयोर्विसर्जनीयः} (पा॰सू॰~८.३.१५)~\arrow गम्~इ~ष्या~मः~\arrow गमिष्यामः।} समुपसर्ग\-संयोजने तु \textcolor{red}{समो गम्यृच्छिभ्याम्} (पा॰सू॰~१.३.२९) इत्यनेनाऽत्मनेपदे \textcolor{red}{सङ्गंस्यामहे}\footnote{सम्~गम्~\arrow \textcolor{red}{समो गम्यृच्छिभ्याम्} (पा॰सू॰~१.३.२९)~\arrow \textcolor{red}{लृट् शेषे च} (पा॰सू॰~३.३.१३)~\arrow सम्~गम्~लृँट्~\arrow सम्~गम्~महिङ्~\arrow सम्~गम्~महि~\arrow \textcolor{red}{स्यतासी लृलुटोः} (पा॰सू॰~३.१.३३)~\arrow सम्~गम्~स्य~महि~\arrow \textcolor{red}{गमेरिट् परस्मैपदेषु} (पा॰सू॰~७.२.५८)~\arrow इडभावः~\arrow \textcolor{red}{अतो दीर्घो यञि} (पा॰सू॰~७.३.१०१)~\arrow सम्~गम्~स्या~महि~\arrow \textcolor{red}{टित आत्मनेपदानां टेरे} (पा॰सू॰~३.४.७९)~\arrow सम्~गम्~स्या~महे~\arrow \textcolor{red}{मोऽनुस्वारः} (पा॰सू॰~८.३.२३)~\arrow सं~गं~स्या~महे~\arrow \textcolor{red}{अनुस्वारस्य ययि परसवर्णः} (पा॰सू॰~८.४.५८)~\arrow सङ्गंस्यामहे।} उपसर्ग\-लोपेऽपि \textcolor{red}{गंस्यामहे} किन्तु \textcolor{red}{गमिष्यामहे} इति त्वत्यन्तमसामञ्जस्यावहमिति चेत्। \textcolor{red}{गमनं गमि} इत्यत्र \textcolor{red}{घिनुण्} प्रत्ययः। \textcolor{red}{शमित्यष्टाभ्यो घिनुण्} (पा॰सू॰~३.२.१४१) इत्यत्र \textcolor{red}{इति}\-शब्द\-प्रयोगाद्गमेरपि घिनुण्। घिनुण्प्रत्ययेऽनुबन्ध\-कार्ये नपुंसक\-लिङ्गे क्रिया\-विशेषणत्वाद्द्वितीया। \textcolor{red}{हे} इति सम्बोधनम्। \textcolor{red}{स्याम} इति \textcolor{red}{अस्‌}\-धातोः (\textcolor{red}{असँ भुवि} धा॰पा॰~१०६५) विध्यर्थे लिङ्लकार उत्तम\-पुरुष\-बहु\-वचनम्।\footnote{\textcolor{red}{असँ भुवि} (धा॰पा॰~१०६५)~\arrow अस्~\arrow \textcolor{red}{शेषात्कर्तरि परस्मैपदम्} (पा॰सू॰~१.३.७८)~\arrow \textcolor{red}{आशिषि लिङ्लोटौ} (पा॰सू॰~३.३.१७३)~\arrow अस्~लिङ्~\arrow अस्~मस्~\arrow \textcolor{red}{यासुट् परस्मै\-पदेषूदात्तो ङिच्च} (पा॰सू॰~३.४.१०३)~\arrow अस्~यासुँट्~मस्~\arrow अस्~यास्~मस्~\arrow \textcolor{red}{लिङः सलोपोऽनन्त्यस्य} (पा॰सू॰~७.२.७९)~\arrow अस्~या~मस्~\arrow \textcolor{red}{श्नसोरल्लोपः} (पा॰सू॰~६.४.१११)~\arrow स्~या~मस्~\arrow \textcolor{red}{नित्यं ङितः} (पा॰सू॰~३.४.९९)~\arrow स्~या~म~\arrow स्याम।} अर्थात् \textcolor{red}{हे जना अयोध्यां प्रति वयं गमि गमनं प्रति स्यामोद्यता भवेम}। यद्वा \textcolor{red}{गम्} इति \textcolor{red}{ड}\-प्रत्ययान्तम्।\footnote{\textcolor{red}{गमॢँ}\-धातोः \textcolor{red}{अन्येष्वपि दृश्यते} (पा॰सू॰~३.२.१०१) इत्यनेन भावे \textcolor{red}{ड}\-प्रत्यये \textcolor{red}{डित्यभस्याप्यनु\-बन्धकरण\-सामर्थ्यात्} (वा॰~६.४.१४३) इत्यनेन टिलोपे विभक्ति\-कार्ये \textcolor{red}{गम्}। \textcolor{red}{अन्येष्वपि} इत्यनेन बाहुलकाद्भावेऽपि। \textcolor{red}{अपिशब्दः सर्वोपाधि\-व्यभिचारार्थः} (का॰वृ॰~३.२.१०१)। गं गीतमितिवत्। यथा शब्दकल्पद्रुमे – \textcolor{red}{गं, क्ली॰, गीयते इति (गै गाने + भावे बाहुलकात् डः) गीतम्, इत्येकाक्षरकोषः}। गम्~\arrow \textcolor{red}{अन्येष्वपि दृश्यते} (पा॰सू॰~३.२.१०१)~\arrow गम्~ड~\arrow गम्~अ~\arrow \textcolor{red}{टेः} (पा॰सू॰~६.४.१४३)~\arrow \textcolor{red}{डित्त्व\-सामर्थ्यादभस्यापि टेर्लोपः} (ल॰सि॰कौ॰~३४३)~\arrow ग्~अ~\arrow ग~\arrow विभक्ति\-कार्यम्~\arrow ग~अम्~\arrow \textcolor{red}{अमि पूर्वः} (पा॰सू॰~६.१.१०७)~\arrow गम्।} \textcolor{red}{इषु}\-धातुः (\textcolor{red}{इषँ गतौ} धा॰पा॰~११२७) दिवादिः। तस्य वर्तमान\-काले कर्म\-व्यतिहार आत्मनेपदम्।\footnote{\textcolor{red}{कर्तरि कर्मव्यतिहारे} (पा॰सू॰~१.३.१४) इत्यनेन।} \textcolor{red}{श्वः} इति सन्निधानेनापि वर्तमान\-समीपे लट्।\footnote{\textcolor{red}{वर्तमान\-सामीप्ये वर्तमानवद्वा} (पा॰सू॰~३.३.१३१) इत्यनेन।} \textcolor{red}{श्वो गं गमनमिष्यामहेऽभिलषामः}\footnote{इष्~\arrow \textcolor{red}{कर्तरि कर्मव्यतिहारे} (पा॰सू॰~१.३.१४)~\arrow \textcolor{red}{वर्तमान\-सामीप्ये वर्तमानवद्वा} (पा॰सू॰~३.३.१३१)~\arrow \textcolor{red}{वर्तमाने लट्} (पा॰सू॰~३.२.१२३)~\arrow इष्~लट्~\arrow इष्~महिङ्~\arrow इष्~महि~\arrow \textcolor{red}{दिवादिभ्यः श्यन्} (पा॰सू॰~३.१.६९)~\arrow इष्~श्यन्~महि~\arrow इष्~य~महि~\arrow \textcolor{red}{अतो दीर्घो यञि} (पा॰सू॰~७.३.१०१)~\arrow इष्~या~महि~\arrow \textcolor{red}{टित आत्मनेपदानां टेरे} (पा॰सू॰~३.४.७९)~\arrow इष्~या~महे~\arrow इष्यामहे।} प्रयोगेऽस्मिन्नियमेव मे मनीषा।\end{sloppypar}
\section[प्रार्थयामि]{प्रार्थयामि}
\label{sec:prarthayami}
\centering\textcolor{blue}{प्रार्थयामि जगन्नाथ पवित्रं कुरु मे गृहम्।\nopagebreak\\
स्थित्वाद्य भुक्त्वा सबलः श्वो गमिष्यसि पत्तनम्॥}\nopagebreak\\
\raggedleft{–~अ॰रा॰~६.१४.३६}\\
\fontsize{14}{21}\selectfont\begin{sloppypar}\hyphenrules{nohyphenation}\justifying\noindent\hspace{10mm} \textcolor{red}{अनुदात्तेत्त्व\-लक्षणमात्मने\-पदमनित्यम्} (प॰शे॰~९३.४) इति \textcolor{red}{चक्षिङ्} (धा॰पा॰~१०१७) इत्यत्र ङित्करणेन ज्ञाप्यते। अतः \textcolor{red}{प्रार्थयामि}।\footnote{\textcolor{red}{अनुदात्तेत्त्व\-लक्षणमात्मने\-पदमनित्यम्} (प॰शे॰~९३.४) इत्यनेन। प्र~\textcolor{red}{अर्थँ उपयाच्ञायाम्} (धा॰पा॰~१९०५)~\arrow प्र~अर्थ्~\arrow \textcolor{red}{सत्याप\-पाश\-रूप\-वीणा\-तूल\-श्लोक\-सेना\-लोम\-त्वच\-वर्म\-वर्ण\-चूर्ण\-चुरादिभ्यो णिच्} (पा॰सू॰~३.१.२५)~\arrow प्र~अर्थ्~णिच्~\arrow प्र~अर्थ्~इ~\arrow प्र~अर्थि~\arrow \textcolor{red}{सनाद्यन्ता धातवः} (पा॰सू॰~३.१.३२)~\arrow धातु\-सञ्ज्ञा~\arrow \textcolor{red}{अनुदात्तेत्त्व\-लक्षणमात्मने\-पदमनित्यम्} (प॰शे॰~९३.४)~\arrow \textcolor{red}{शेषात्कर्तरि परस्मैपदम्} (पा॰सू॰~१.३.७८)~\arrow \textcolor{red}{वर्तमाने लट्} (पा॰सू॰~३.२.१२३)~\arrow प्र~अर्थि~लट्~\arrow प्र~अर्थि~मिप्~\arrow प्र~अर्थि~मि~\arrow \textcolor{red}{कर्तरि शप्‌} (पा॰सू॰~३.१.६८)~\arrow प्र~अर्थि~शप्~मि~\arrow प्र~अर्थि~अ~मि~\arrow \textcolor{red}{सार्वधातुकार्ध\-धातुकयोः} (पा॰सू॰~७.३.८४)~\arrow प्र~अर्थे~अ~मि~\arrow \textcolor{red}{एचोऽयवायावः} (पा॰सू॰~६.१.७८)~\arrow प्र~अर्थय्~अ~मि~\arrow \textcolor{red}{अतो दीर्घो यञि} (पा॰सू॰~७.३.१०१)~\arrow प्र~अर्थय्~आ~मि~\arrow प्र~अर्थयामि~\arrow \textcolor{red}{अकः सवर्णे दीर्घः} (पा॰सू॰~६.१.१०१)~\arrow प्रार्थयामि। \pageref{sec:prarthaya}तमे पृष्ठे \ref{sec:prarthaya} \nameref{sec:prarthaya} इति प्रयोगस्य विमर्शमपि पश्यन्तु।}\end{sloppypar}
\vspace{2mm}
\centering ॥ इति युद्धकाण्डीयप्रयोगाणां विमर्शः ॥\nopagebreak\\
\vspace{4mm}
\pdfbookmark[2]{उत्तरकाण्डम्}{Chap3Part2Kanda7}
\phantomsection
\addtocontents{toc}{\protect\setcounter{tocdepth}{2}}
\addcontentsline{toc}{subsection}{उत्तरकाण्डीयप्रयोगाणां विमर्शः}
\addtocontents{toc}{\protect\setcounter{tocdepth}{0}}
\centering ॥ अथोत्तरकाण्डीयप्रयोगाणां विमर्शः ॥\nopagebreak\\
\section[निजघ्ने]{निजघ्ने}
\centering\textcolor{blue}{ब्राह्मणानृषिमुख्यांश्च देवदानवकिन्नरान्।\nopagebreak\\
देवश्रियो मनुष्यांश्च निजघ्ने समहोरगान्॥}\nopagebreak\\
\raggedleft{–~अ॰रा॰~७.२.४७}\\
\fontsize{14}{21}\selectfont\begin{sloppypar}\hyphenrules{nohyphenation}\justifying\noindent\hspace{10mm} \textcolor{red}{आङो यमहनः} (पा॰सू॰~१.३.२८) इत्यनेनात्मनेपदे \textcolor{red}{न्याजघ्ने}।\footnote{न्या~हन्~\arrow \textcolor{red}{आङो यमहनः} (पा॰सू॰~१.३.२८)~\arrow \textcolor{red}{परोक्षे लिट्} (पा॰सू॰~३.२.११५)~\arrow न्या~हन्~लिँट्~\arrow न्या~हन्~त~\arrow \textcolor{red}{लिटस्तझयोरेशिरेच्} (पा॰सू॰~३.४.८१)~\arrow न्या~हन्~एश्~\arrow न्या~हन्~ए~\arrow \textcolor{red}{लिटि धातोरनभ्यासस्य} (पा॰सू॰~६.१.८)~\arrow न्या~हन्~हन्~ए~\arrow \textcolor{red}{हलादिः शेषः} (पा॰सू॰~७.४.६०)~\arrow न्या~ह~हन्~ए~\arrow \textcolor{red}{कुहोश्चुः} (पा॰सू॰~७.४.६२)~\arrow न्या~झ~हन्~ए~\arrow \textcolor{red}{अभ्यासे चर्च} (पा॰सू॰~८.४.५४)~\arrow न्या~ज~हन्~ए~\arrow \textcolor{red}{असंयोगाल्लिट्कित्} (पा॰सू॰~६.४.९८)~\arrow कित्त्वम्~\arrow \textcolor{red}{गम\-हन\-जन\-खन\-घसां लोपः क्ङित्यनङि} (पा॰सू॰~६.४.९८)~\arrow न्या~ज~ह्~न्~ए~\arrow \textcolor{red}{हो हन्तेर्ञ्णिन्नेषु} (पा॰सू॰~७.३.५४)~\arrow न्या~ज~घ्~न्~ए~\arrow न्याजघ्ने।} \textcolor{red}{विनाऽपि प्रत्ययं पूर्वोत्तर\-पद\-लोपो वक्तव्यः} (वा॰~५.३.८३) इत्यनेन \textcolor{red}{आङ्} इत्यस्य लोपे \textcolor{red}{निजघ्ने}।\footnote{परस्मैपदे तु \textcolor{red}{निजघान} इति रूपम्। नि~हन्~\arrow \textcolor{red}{शेषात्कर्तरि परस्मैपदम्} (पा॰सू॰~१.३.७८)~\arrow \textcolor{red}{परोक्षे लिट्} (पा॰सू॰~३.२.११५)~\arrow नि~हन्~लिँट्~\arrow नि~हन्~तिप्~\arrow \textcolor{red}{परस्मैपदानां णलतुसुस्थलथुस\-णल्वमाः} (पा॰सू॰~३.४.८२)~\arrow नि~हन्~णल्~\arrow नि~हन्~अ~\arrow \textcolor{red}{लिटि धातोरनभ्यासस्य} (पा॰सू॰~६.१.८)~\arrow नि~हन्~हन्~अ~\arrow \textcolor{red}{हलादिः शेषः} (पा॰सू॰~७.४.६०)~\arrow नि~ह~हन्~अ~\arrow \textcolor{red}{कुहोश्चुः} (पा॰सू॰~७.४.६२)~\arrow नि~झ~हन्~अ~\arrow \textcolor{red}{अभ्यासे चर्च} (पा॰सू॰~८.४.५४)~\arrow नि~ज~हन्~अ~\arrow \textcolor{red}{हो हन्तेर्ञ्णिन्नेषु} (पा॰सू॰~७.३.५४)~\arrow नि~ज~घन्~अ~\arrow \textcolor{red}{अत उपधायाः} (पा॰सू॰~७.२.११६)~\arrow नि~ज~घान्~अ~\arrow निजघान।}\end{sloppypar}
\section[काङ्क्षे]{काङ्क्षे}
\centering\textcolor{blue}{सत्येन च शपे नाहं त्वां विना दिवि वा भुवि।\nopagebreak\\
काङ्क्षे राज्यं रघुश्रेष्ठ शपे त्वत्पादयोः प्रभो॥}\nopagebreak\\
\raggedleft{–~अ॰रा॰~७.९.५}\\
\fontsize{14}{21}\selectfont\begin{sloppypar}\hyphenrules{nohyphenation}\justifying\noindent\hspace{10mm} \textcolor{red}{काङ्क्षामि}\footnote{\textcolor{red}{काक्षिँ काङ्क्षायाम्} (धा॰पा॰~६६७)~\arrow काक्ष्~\arrow \textcolor{red}{इदितो नुम् धातोः} (पा॰सू॰~७.१.५८)~\arrow \textcolor{red}{मिदचोऽन्त्यात्परः} (पा॰सू॰~१.१.४७)~\arrow का~नुँम्~क्ष्~\arrow कान्~क्ष्~\arrow \textcolor{red}{नश्चापदान्तस्य झलि} (पा॰सू॰~८.३.२४)~\arrow कांक्ष्~\arrow \textcolor{red}{अनुस्वारस्य ययि परसवर्णः} (पा॰सू॰~८.४.५८)~\arrow काङ्क्ष्~\arrow \textcolor{red}{शेषात्कर्तरि परस्मैपदम्} (पा॰सू॰~१.३.७८)~\arrow \textcolor{red}{वर्तमाने लट्} (पा॰सू॰~३.२.१२३)~\arrow काङ्क्ष्~लट्~\arrow काङ्क्ष्~मिप्~\arrow काङ्क्ष्~मि~\arrow \textcolor{red}{कर्तरि शप्} (पा॰सू॰~३.१.६८)~\arrow काङ्क्ष्~शप्~मि~\arrow काङ्क्ष्~अ~मि~\arrow \textcolor{red}{अतो दीर्घो यञि} (पा॰सू॰~७.३.१०१)~\arrow काङ्क्ष्~आ~मि~\arrow काङ्क्षामि।} इति प्रयोक्तव्ये \textcolor{red}{काङ्क्षे}\footnote{\textcolor{red}{काक्षिँ काङ्क्षायाम्} (धा॰पा॰~६६७)~\arrow काङ्क्ष् (पूर्ववत्)~\arrow \textcolor{red}{कर्तरि कर्म\-व्यतिहारे} (पा॰सू॰~१.३.१४)~\arrow \textcolor{red}{वर्तमाने लट्} (पा॰सू॰~३.२.१२३)~\arrow काङ्क्ष्~लट्~\arrow काङ्क्ष्~इट्~\arrow काङ्क्ष्~इ~\arrow \textcolor{red}{कर्तरि शप्} (पा॰सू॰~३.१.६८)~\arrow काङ्क्ष्~शप्~इ~\arrow काङ्क्ष्~अ~इ~\arrow \textcolor{red}{टित आत्मनेपदानां टेरे} (पा॰सू॰~३.४.७९)~\arrow काङ्क्ष्~अ~ए~\arrow \textcolor{red}{अतो गुणे} (पा॰सू॰~६.१.९७)~\arrow काङ्क्ष्~ए~\arrow काङ्क्षे।} इति प्रयोगस्तु \textcolor{red}{कर्तरि कर्म\-व्यतिहारे} (पा॰सू॰~१.३.१४) इत्यनेनात्मनेपदे सति।\footnote{एवमेव \textcolor{red}{न काङ्क्षे विजयं कृष्ण न च राज्यं सुखानि च} (भ॰गी॰~१.३२) इत्यत्र। विजय\-राज्य\-सुखानामकाङ्क्षा न राजोचिता।}\end{sloppypar}
\vspace{2mm}
\centering ॥ इत्युत्तरकाण्डीयप्रयोगाणां विमर्शः ॥\nopagebreak\\
\vspace{4mm}
\centering इत्यध्यात्म\-रामायणेऽपाणिनीय\-प्रयोगाणां\-विमर्श\-नामके शोध\-प्रबन्धे तृतीयाध्याये द्वितीय\-परिच्छेदः।\nopagebreak\\
\vspace{4mm}
\centering इत्यध्यात्म\-रामायणेऽपाणिनीय\-प्रयोगाणां\-विमर्श\-नामके शोध\-प्रबन्धे तृतीयोऽध्यायः।\\
\vspace{4mm}
\centering\textcolor{blue}{\fontsize{16}{24}\selectfont बुद्ध्या श्रीगुरुपादपद्मरजसा संशुद्धया सादरं\nopagebreak\\
कृत्वा लेखकमाप्तशीलयशसं शिष्यं शिशुं राघवम्।\nopagebreak\\
बालो नष्टविलोचनो गिरिधरः शब्दान् विभाव्याऽत्मना\nopagebreak\\
बध्नाति स्म निबन्धमेतममलं तोषाय सीतापतेः॥}\nopagebreak\\
\vspace{4mm}
\centering इत्यध्यात्म\-रामायणेऽपाणिनीय\-प्रयोगाणां\-विमर्शः।


\backmatter
% Nityanand Misra: LaTeX code to typeset a book in Sanskrit
% Copyright (C) 2016 Nityanand Misra
%
% This program is free software: you can redistribute it and/or modify it under
% the terms of the GNU General Public License as published by the Free Software
% Foundation, either version 3 of the License, or (at your option) any later
% version.
%
% This program is distributed in the hope that it will be useful, but WITHOUT
% ANY WARRANTY; without even the implied warranty of  MERCHANTABILITY or FITNESS
% FOR A PARTICULAR PURPOSE. See the GNU General Public License for more details.
%
% You should have received a copy of the GNU General Public License along with
% this program.  If not, see <http://www.gnu.org/licenses/>.

\renewcommand\chaptername{}
\chapter[सङ्केताक्षरसूची]{सङ्केताक्षरसूची}
\markboth{सङ्केताक्षरसूची}{}
\fontsize{14}{21}\selectfont
\begin{longtable}{ll}
अ॰को॰ & अमरकोषः\\
अ॰को॰ व्या॰सु॰ & अमरकोषे व्याख्यासुधा\\
अ॰पु॰ & अग्निपुराणम्\\
अ॰रा॰ & अध्यात्मरामायणम्\\
अ॰शा॰ & अभिज्ञानशाकुन्तलम्\\
अग॰सं॰ & अगस्त्यसंहिता\\
अहि॰सं॰ & अहिर्बुध्न्यसंहिता\\
आ॰रा॰ & आनन्दरामायणम्\\
आ॰श्रौ॰सू॰ & आपस्तम्बश्रौतसूत्रम्\\
आ॰स॰श॰ & आर्यासप्तशती\\
आ॰सं॰ & आनन्दसंहिता\\
ई॰उ॰ & ईशावास्योपनिषद्\\
उ॰को॰ & उणादिकोषः (दयानन्द\-सरस्वती\-व्याख्या\-सहितः)\\
उ॰रा॰च॰ & उत्तररामचरितम्\\
ऋ॰वे॰सं॰ & ऋग्वेदसंहिता\\
ऋ॰वे॰सं॰ सा॰भा॰ & ऋग्वेदसंहितायां सायणभाष्यम्\\
ऋ॰वे॰सं॰ सा॰भा॰ उ॰प्र॰ & ऋग्वेदसंहितायां सायणभाष्य उपोद्घातप्रकरणम्\\
ए॰को॰ & एकाक्षरकोषः\\
क॰उ॰ & कठोपनिषद्\\
क॰उ॰ रा॰कृ॰भा॰ & कठोपनिषदि श्रीराघवकृपाभाष्यम्\\
क॰पु॰ & कल्किपुराणम्\\
क॰स॰सा॰ & कथासरित्सागरः\\
का॰ & कादम्बरी\\
का॰वि॰प॰ & काशिकाविवरणपञ्जिका (न्यासः)\\
का॰वृ॰ & काशिकावृत्तिः\\
का॰वृ॰वा॰ & काशिकावृत्तिवार्त्तिकम्\\
का॰सू॰वृ॰ & काव्यालङ्कारसूत्रवृत्तिः\\
कि॰ & किरातार्जुनीयम्\\
कि॰ घ॰व्या॰म॰ & किरातार्जुनीये घण्टापथव्याख्यायां मङ्गलाचरणम्\\
कु॰स॰ & कुमारसम्भवम्\\
कु॰स॰ स॰व्या॰ & कुमारसम्भवे सञ्जीविनीव्याख्या\\
कू॰पु॰ & कूर्मपुराणम्\\
कृ॰य॰ तै॰आ॰ & कृष्णयजुर्वेदतैत्तिरीयारण्यकम्\\
कृ॰य॰ तै॰ब्रा & कृष्णयजुर्वेदतैत्तिरीयब्राह्मणम्\\
कृ॰य॰ तै॰सं॰ & कृष्णयजुर्वेदतैत्तिरीयसंहिता\\
ग॰पु॰ & गरुडपुराणम्\\
ग॰सं॰ & गर्गसंहिता\\
गी॰गो॰ & गीतगोविन्दम्\\
गो॰गृ॰सू॰ & गोभिलगृह्यसूत्रम्\\
गो॰पू॰ता॰उ॰ & गोपालपूर्वतापिन्युपनिषद्\\
च॰सं॰ सू॰स्था॰ & चरकसंहितायां सूत्रस्थानम्\\
चा॰नी॰ & चाणक्यनीतिः\\
त॰बो॰ & तत्त्वबोधिनी (ज्ञानेन्द्रसरस्वतीकृता)\\
त॰वा॰ & तन्त्रवार्तिकम्\\
त॰स॰ & तर्कसङ्ग्रहः\\
त॰स॰ न्या॰बो॰व्या॰ & तर्कसङ्ग्रहे न्यायबोधिनीव्याख्या\\
त॰स॰ प॰व्या॰ & तर्कसङ्ग्रहे पदकृत्यव्याख्या\\
तै॰उ॰ & तैत्तिरीयोपनिषद्\\
द॰उ॰ & दशपाद्युणादिपाठः\\
द॰उ॰वृ॰ & दशपाद्युणादिवृत्तिः (युधिष्ठिरमीमांसकसम्पादिता)\\
द॰रू॰ & दशरूपकम् (धनञ्जयकृतम्)\\
दु॰स॰श॰ & दुर्गासप्तसती (मार्कण्डेयपुराणान्तर्गता)\\
दे॰भा॰पु॰ & देवीभागवतपुराणम्\\
धा॰पा॰ & धातुपाठः\\
धा॰पा॰ ग॰सू॰ & धातुपाठे गणसूत्रम्\\
ध्व॰ & ध्वन्यालोकः\\
न॰उ॰ & नलोपाख्यानम्\\
न॰का॰ & नन्दिकेश्वरकाशिका\\
नर॰पु॰ & नरसिंहपुराणम्\\
ना॰पु॰ & नारदपुराणम्\\
नान्दी॰पु॰ & नान्दीपुराणम्\\
नै॰च॰ & नैषधीयचरितम्\\
न्या॰सू॰ & न्यायसूत्रम्\\
प॰उ॰ & पञ्चपाद्युणादिपाठः\\
प॰उ॰ श्वे॰वृ॰ & पञ्चपाद्युणादिपाठे श्वेतवनवासिवृत्तिः\\
प॰त॰ & पञ्चतन्त्रम्\\
प॰त॰ अ॰टी॰ & पञ्चतन्त्रेऽभिनवराजलक्ष्मीटीका\\
प॰म॰ & पदमञ्जरी\\
प॰ल॰म॰ & परमलघुमञ्जूषा\\
प॰ल॰म॰ ज्यो॰टी॰ & परमलघुमञ्जूषायां ज्योत्स्नाटीका (कालिकाप्रसादशुक्लकृता)\\
प॰शे॰ & परिभाषेन्दुशेखरः\\
प॰स्मृ॰ & पराशरस्मृतिः\\
परा॰उ॰ & पराशरोपपुराणम्\\
पा॰सू॰ & पाणिनिसूत्रम् (अष्टाध्यायी)\\
प्र॰ना॰ & प्रतिमानाटकम्\\
प्रौ॰म॰ & प्रौढमनोरमा\\
बा॰म॰ & बालमनोरमा\\
बृ॰उ॰ & बृहदारण्यकोपनिषद्\\
बृ॰ब्र॰सं॰ & बृहद्ब्रह्मसंहिता\\
बृ॰सं॰ & बृहत्संहिता\\
ब्र॰उ॰ & ब्रह्मबिन्दूपनिषद्\\
ब्र॰पु॰ & ब्रह्मपुराणम्\\
ब्र॰सू॰ & ब्रह्मसूत्रम्\\
ब्रह्मा॰पु॰ & ब्रह्माण्डपुराणम्\\
भ॰का॰ & भट्टिकाव्यम्\\
भ॰गी॰ & भगवद्गीता\\
भ॰गी॰ रा॰भा॰ & भगवद्गीतायां रामानुजभाष्यम्\\
भ॰नी॰ & भर्तृहरिनीतिशतकम्\\
भा॰उ॰ पा॰सू॰ & भाष्य उद्द्योते पाणिनीयसूत्रम्\\
भा॰प॰ & भाष्ये पस्पशाह्निकम्\\
भा॰पा॰सू॰ & भाष्ये पाणिनीयसूत्रम्\\
भा॰पु॰ & श्रीमद्भागवतपुराणम्\\
भा॰पु॰ अ॰प्र॰ & श्रीमद्भागवतेऽन्वितार्थप्रकाशिका\\
भा॰पु॰ गू॰दी॰ & श्रीमद्भागवते गूढार्थदीपिका\\
भा॰पु॰ नि॰प्र॰ & श्रीमद्भागवते निगूढार्थप्रकाशः\\
भा॰पु॰ बा॰प्र॰ & श्रीमद्भागवते बालप्रबोधिनी\\
भा॰पु॰ वं॰टी॰ & श्रीमद्भागवते वंशीधरटीका\\
भा॰पु॰ वी॰रा॰व्या॰ & श्रीमद्भागवते वीरराघवव्याख्या\\
भा॰पु॰ श्री॰टी॰ & श्रीमद्भागवते श्रीधरटीका\\
भा॰पु॰ सि॰प्र॰ & श्रीमद्भागवते सिद्धान्तप्रदीपः\\
भा॰प्र॰ & भाष्ये प्रदीपः (कैयटकृतः)\\
भा॰प्र॰ पा॰सू॰ & भाष्ये प्रदीपे पाणिनीयसूत्रम्\\
भा॰रा॰ & श्रीभार्गवराघवीयम्\\
भा॰शि॰ & भाष्ये शिवसूत्रम्\\
भा॰शि॰सू॰ & भाष्ये शिवसूत्रम्\\
भृ॰दू॰ & भृङ्गदूतम्\\
म॰अ॰ & मधुराष्टकम्\\
म॰पु॰ & मत्स्यपुराणम्\\
म॰भा॰ & महाभारतम्\\
म॰भा॰ भा॰दी॰ & महाभारते भारतदीपिका (नीलकण्ठकृता)\\
म॰सु॰स॰ & महासुभाषितसङ्ग्रहः\\
म॰स्मृ॰ & मनुस्मृतिः\\
म॰स्मृ॰ कु॰टी॰ & मनुस्मृतौ कुल्लूकभट्टटीका\\
म॰स्मृ॰ मे॰टी॰ & मनुस्मृतौ मेधातिथिटीका\\
म॰स्मृ॰ राघ॰टी॰ & मनुस्मृतौ राघवानन्दटीका\\
मा॰भा॰ & मानसभारती (रामचरितमानसस्य संस्कृतपद्यानुवादः)\\
मा॰धा॰वृ॰ & माधवीया धातुवृत्तिः\\
मी॰सू॰ & मीमांसासूत्रम्\\
मे॰को॰ & मेदिनीकोषः\\
मे॰दू॰ & मेघदूतम्\\
या॰स्मृ॰ & याज्ञवल्क्यस्मृतिः\\
यो॰सू॰ & योगसूत्रम्\\
यो॰सू॰ भो॰वृ॰ & योगसूत्रे भोजवृत्तिः\\
यो॰हृ॰ दी॰टी॰ & योगिनीहृदये दीपिकाटीका\\
र॰वं॰ & रघुवंशम्\\
र॰वं॰ द॰टी॰ & रघुवंशे दर्पणटीका (हेमाद्रिकृता)\\
र॰वं॰ स॰व्या॰ & रघुवंशे सञ्जीविनीव्याख्या\\
र॰वं॰ स॰व्या॰म॰ & रघुवंशे सञ्जीविनीव्याख्यायां मङ्गलाचरणम्\\
रा॰उ॰ता॰उ॰ & रामोत्तरतापिन्युपनिषद्\\
रा॰च॰मा॰ & रामचरितमानसम्\\
रा॰मी॰ & रामायणमीमांसा (करपात्रस्वामिकृता)\\
रा॰र॰स्तो॰ & रामरक्षास्तोत्रम्\\
ल॰म॰ & लघुमञ्जूषा\\
ल॰वि॰स्मृ॰ & लघुविष्णुस्मृतिः\\
ल॰शे॰ & लघुशब्देन्दुशेखरः\\
ल॰शे॰ म॰ & लघुशब्देन्दुशेखरे मङ्गलाचरणम्\\
ल॰सि॰कौ॰ & लघुसिद्धान्तकौमुदी\\
ल॰सि॰कौ॰ भै॰टी॰ & लघुसिद्धान्तकौमुद्यां भैमीटीका (भीमसेनशास्त्रिकृता)\\
लि॰ & लिङ्गानुशासनम्\\
वा॰ & वार्त्तिकम्\\
वा॰प॰ & वाक्यपदीयम्\\
वा॰प॰ हे॰टी॰ & वाक्यपदीये हेलाराजटीका\\
वा॰रा॰ & वाल्मीकीयरामायणम्\\
वा॰रा॰ क॰टी॰ & वाल्मीकीयरामायणे कतकटीका\\
वा॰रा॰ ति॰टी॰ & वाल्मीकीयरामायणे तिलकटीका\\
वा॰रा॰ भू॰टी॰ & वाल्मीकीयरामायणे भूषणटीका\\
वा॰रा॰ शि॰टी॰ & वाल्मीकीयरामायणे शिरोमणिटीका\\
वा॰सं॰ & वाराहसंहिता\\
वाम॰पु॰ & वामनपुराणम्\\
वाम॰पु॰ स॰मा॰ & वामनपुराणे सरोमाहात्म्यम्\\
वायु॰पु॰ & वायुपुराणम्\\
वि॰पु॰ & विष्णुपुराणम्\\
वि॰स॰ना॰ & विष्णुसहस्रनामस्तोत्रम्\\
वि॰स॰ना॰ स॰भा॰ & विष्णुसहस्रनामस्तोत्रे सत्यभाष्यम्\\
वे॰सा॰ & वेदान्तसारः\\
वै॰भू॰सा॰ & वैयाकरणभूषणसारः\\
वै॰सि॰का॰ & वैयाकरणसिद्धान्तकारिका\\
वै॰सि॰कौ॰ & वैयाकरणसिद्धान्तकौमुदी\\
व्यु॰वा॰ का॰प्र॰ & व्युत्पत्तिवादे कारके प्रथमा\\
श॰ब्रा॰ & शतपथब्राह्मणम्\\
श॰र॰ & शब्दरत्नः\\
शि॰पु॰ & शिवपुराणम्\\
शि॰म॰ & शिवमहिम्नःस्तोत्रम्\\
शि॰व॰ & शिशुपालवधम्\\
शि॰व॰ स॰व्या॰म॰ & शिशुपालवधे सर्वङ्कषाव्याख्यायां मङ्गलाचरणम्\\
शि॰सू॰ & शिवसूत्रम् (माहेश्वरसूत्रम्)\\
शु॰य॰वा॰मा॰ & शुक्लयजुर्वेदवाजसनेयिमाध्यन्दिनसंहिता\\
श्लो॰वा॰ & श्लोकवार्त्तिकम्\\
श्वे॰उ॰ & श्वेताश्वतरोपनिषद्\\
स॰र॰ & सङ्गीतरत्नाकरः\\
स॰र॰ सु॰टी॰ & सङ्गीतरत्नाकरे सुधाकरटीका\\
सा॰का॰ & साङ्ख्यकारिका\\
सा॰का॰गौ॰भा॰ & साङ्ख्यकारिकायां गौडपादभाष्यम्\\
सा॰द॰ & साहित्यदर्पणः\\
सी॰सु॰नि॰ & श्रीसीतासुधानिधिः\\
सी॰सु॰नि॰ भ॰टी॰ & श्रीसीतासुधानिधौ भक्तिटीका\\
स्क॰पु॰ & स्कन्दपुराणम्\\
स्व॰शि॰ & स्वराष्टकशिक्षा\\
ह॰च॰चि॰ & हरचरितचिन्तामणिः\\
ह॰ना॰ & हनुमन्नाटकम्\\
ह॰ना॰व्या॰ & हरिनामामृतव्याकरणम्\\
ह॰व॰ & हरिवंशः\\
हि॰ & हितोपदेशः\\
\end{longtable}


\strictpagecheck
\checkoddpage
\ifoddpage
\mbox{}
\thispagestyle{empty}
\newpage
\else
\mbox{}
\thispagestyle{empty}
\newpage
\mbox{}
\thispagestyle{empty}
\newpage
\fi
\includepdf[width=\paperwidth,trim=0mm 0mm 204mm 0mm]{arapvcover.pdf}

\end{document}
